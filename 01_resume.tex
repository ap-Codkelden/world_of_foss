\chapter{Резюме}

\section{Предисловие} 

Внимаю читателей предлагается нечто вроде моих избранных трудов~--- сборник заметок, посвящённых различным гуманитарным аспектам существования мира FOSS (Free and Open Source Software). Заметки эти сочинялись на протяжении последних пятнадцати лет. Часть из них представляет интерес как исторические свидетельства~--- памятники эпохи. Однако многие сохраняют актуальность: сколько раз я ловил себя на том, что в разговоре на одну из затронутых здесь тем, пересказывал написанное собой же, любимым, лет несколько назад. Ибо поколения пользователей UNIX, Linux и прочего Open Source меняются, и перед вновь пришедшими в этот мир их применителями встают одни и те же вопросы. Надеюсь, что мои заметки помогут найти ответы на некоторые из них.

В этом сборнике я попытался как-то структурировать свои заметки~--- иногда по темам, иногда~--- хронологически. Вне зависимости от времени сочинения заметок, они публикуются более или менее <<как было>>~--- лишь в несколько причёсанном виде. Так, по возможности исправлялись <<ачипятки>> и прямые фактические ошибки. Конечно, не обошлось и без некоторой стилистической правки. Местами вносились актуализирующие комментарии, обычно выделенные курсивом. Ну и для связки тематически связанных материалов везде по умолчанию подразумевается наличие неопределённого артикля\dots думаю, читатель и сам знает, какой неопределённый артикль используется на Руси для связки несвязуемого и при  впихании невпихуемого.

Первоисточники и прототипы всех собранных тут заметок легко найти в Сети, на моих многочисленных сайтах, на ресурсах, с которыми я сотрудничал, и на тех наборах страниц, которые специализируются на переразмещении чужих материалов. Ссылок я не привожу, потому как никто их искать всё равно не будет~--- как не искали их поколения начинающих линуксоидов. Да и сделан этот сборник именно для того, чтобы избавить читателей от столь тяжких трудов. Здесь вообще нет того, что модно называть <<пруфлинками>>, ибо сборник предназначен для оффлайного чтения.

Впрочем, некоторые материалы в том виде, в каком они здесь представлены, в сети не найти~--- например, заметки о деле Ханса Рейзера я впервые собрал воедино только здесь.

Некоторые заметки написаны в соавторстве с моими коллегами и товарищами~--- Владимиром Поповым и Алисой Деевой. Впрочем, Алиса~--- косвенный соавтор и многих других материалов, ибо являлась их вдохновителем.

Есть здесь и страницы, написанные по моей просьбе Алексеем Жбановым aka allez и Олегом Свидерским aka Uncel\_Theodore. 

Многие материалы сборника родились в ходе обсуждения взаимноинтересных тем с Сергеем Голубевым. А также при на форумах~--- как ныне действующих, так и уже не существующих. И, разумеется, из разговоров в \href{Джуйке}{http://www.juick.com}. Участников которых я не смог бы перечислить поимённо. Однако я искренне признателен всем моим товарищам и коллегам~--- без них не было бы ни этого сборника в целом, ни составляющих его заметок.

А в заключение хотелось бы вспомнить коллег и товарищей, к верхним людям ушедших~--- Олега Свидерского и Евгения Яворских, известного своим читателям как Акустик. Не помянуть, а вспомнить: про них~--- последняя притча из этого сборника. Не только про них~--- но это уже из совсем другой книжки\dots