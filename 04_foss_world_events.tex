\chapter{События FOSS-мира}

\section{Девять дней, которые потрясли Open Source} 
\begin{timeline}
14 ноября 2006
\end{timeline}

\hfill \begin{minipage}[h]{0.45\textwidth}
— У меня есть все основания думать, что я и один справлюсь со своим делом.

— \dotsВ таком случае\dots у меня есть не меньшие основания предполагать, что и я один смогу справиться с вашим делом.
\begin{flushright}
\textit{Ильф и Петров, Двенадцать стульев}
\end{flushright}
\bigskip\end{minipage}

Как известно, большевикам, чтобы потрясти устои старого мира, потребовалось десять дней. По крайней мере, таковы были результаты подсчетов американского писателя Джона Рида, очевидца, а возможно, и участника тех событий, сочинившего по сему поводу соответствующую книгу, которая так и называется: <<Десять дней, которые потрясли мир>>.

Для того, чтобы потрясти мир Open Source, достаточно оказалось 9 дней. Которые уложились в интервал с 25 октября по 2 ноября. Впрочем, точными вычислениями с учетом времени суток и часовых поясов я не занимался, вполне возможно, что хватило всего восьми астрономических дней\dots

Итак, первый из этих дней выпал на 25 октября, когда компания Oracle объявила о выпуске собственного дистрибутива, предназначенного для работы с её собственной же СУБД. Впрочем, если принадлежность СУБД Oracle одноименной фирме никем не оспаривается, то относительно дистрибутива возникают вполне обоснованные сомнения. Ибо являет он собой ни что иное, как пересобранный и несколько разгруженный Red Hat.

Как известно, после расщепления Red Hat на ветвь чистого Open Source~--- Fedora Core, и корпоративный RHEL, полностью свободное распространение последнего прекратилось. Разумеется, дистрибутив этот, в соответствии с лицензией GPL, по прежнему можно безвозмездно (то есть даром) скачать с официальных серверов~--- в том числе и в виде исходников. Из которых, впрочем, можно собрать точный функциональный аналог <<настоящего>> RHEL~--- только без фирменной техподдержки. Процедура сборки таких клонов и их распространение, в соответствие с буквой и духом GPL, абсолютно законна, чем уже давно пользуются майнтайнеры, выпуская своего рода <<нишевые>> продукты~--- Scientific Linux, ориентированный на использование в сфере науки и образования, и CentOS~--- своего рода <<RHEL для бедных>>, то есть всякого рода малого бизнеса и пользователей-индивидуалов.

Так что, казалось бы, и Oracle не свершила ничего крамольного, просто использовав возможность разжиться готовой системой на законных основаниях. Почему же её действия получили столь громкий резонанс в кругах, тем или иным образом связанных с движением Open Source?

<<Что позволено Юпитеру, не позволено быку>>~--- говаривали древнеримские греки. К нашему случаю эта фраза применима в инвертированном виде: что вполне естественно для <<бюджетных>> майнтайнеров, выглядит несколько странно в исполнении гиганта софтверной индустрии. От которого можно было бы ожидать если не создания дистрибутива <<с нуля>> (в наши дни это просто нецелесообразно), то по крайней мере существенной доработки существующего. Ну и вообще обратной отдачи на благо сообщества.

Нельзя сказать, что Oracle совсем не приложили руку к своему дистрибутиву, получившему имя 
\textit{Unbreakable Linux}
 (от перевода воздержусь~--- смысл интуитивно понятен, а точный эквивалент подобрать затрудняюсь). Но приложение это было, так сказать, со знаком <<минус>>: из их фирменной сборки был изъят ряд компонентов, в частности, <<конкурирующие>> программы~--- Postgres и MySQL. Впрочем, ознакомиться с этим дистрибутивом <<вживе>> мне пока не удалось. Хотя он и доступен для свободного скачивания, однако ему предшествует очень нудная процедура регистрации.

Предлагает Oracle и техническую поддержку для корпоративных пользователей~--- причем по ценам, чуть ли не вдвое более низким, чем Red Hat.

Собственно, последнее и было вторым фактом, вызвавшим возмущение общественности, каковая усмотрела желание <<подсидеть>> фирму-производителя родительского дистрибутива. Не случайно ответным шагом компании Red Hat стал лозунг~--- 
\textit{Unfakeable Linux}
 (в переводе, думаю, также не нуждается). Хотя мне первый фактор~--- то есть отказ от собственного вклада в развитие Open Source,~--- представлялся более важным.

Однако, когда отгорели костры первых эмоций, настало время задуматься: а так ли все суицидально, как показалось сначала? По здравом размышлении, можно представить себе два сюжета развития дальнейших событий. Первый вытекает из факта предоставления Oracle технической поддержки своего дистрибутива в целом. А это неизбежно вызовет создание соответствующей инфраструктуры и её развитие, в том числе и финансирование разработчиков открытого софта примерно в тех же формах, как это ныне делают Red Hat и Novell. То есть, в конечном счете, того самого обратного вклада в движение Open Source, которым не пренебрег еще ни один коммерческий пользователь его продукции. И в итоге на поле открытого софта появится просто еще один корпоративный игрок~--- и игрок далеко не последней категории.

Второй сюжет~--- это превращение <<Несгибаемого Linux'а>> от Oracle просто в довесок к их же СУБД, своего рода <<стартер>> для её запуска и среду функционирования. И в этом случае мы получаем просто еще один <<нишевый>> продукт~--- правда, ниша его оказывается ох какой глубокой (в финансовом выражении). Но на развитии Open Source это скажется, по моему, весьма мало~--- как мало заметно влияние на него Linux'ов для встроенных устройств и тому подобных узко специализированных систем.

Другой вопрос, как тот или иной сценарий скажутся на положении старейшего коммерческого Linux-дистрибьютора~--- компании Red Hat. В первом случае она оказывается в положении Кисы Воробьянинова из процитированного в эпиграфе диалога. И, вероятно, ей не останется иного выхода, кроме как, подобно Ипполиту Матвеевичу, в той или иной форме вступить в число пайщиков-концессионеров.

При <<нишевом>> развитии событий Red Hat тоже ожидает не так уж много хорошего. По данным от Линуксцентра~--- крупнейшего онлайнового магазина России, распространяющего дистрибутивы, до 80\% покупок RHEL (вкупе с фирменной технической поддержкой) выполняется с целью обеспечения работы Oracle. Ведь именно RHEL долгое время был одной из двух официально сертифицированных Linux-платформ для этой СУБД. Но тут у Red Hat возникает возможность <<обратиться к истокам>>~--- и вплотную заняться внедрением Linux в десктопную сферу. Не возьмусь судить, насколько это может быть коммерчески выгодно~--- но вот польза сообществу Open Source от этого была бы несомненно.

Так что первое потрясение Open Source при ближайшем рассмотрении оказывается не столь уж фатальным~--- хотя на мир свободного софта оно влияние окажет безусловно~--- и чуть позже я еще вернусь к вопросу, как именно это влияние может проявиться. А вот масштабы второго пока оценить трудно.

Второе потрясение оказалось приуроченным ко второму ноября, когда компания Novell~--- уже более двух лет являющаяся владельцем второго из <<корпоративных>> дистрибутивов Linux, Suse,~--- и корпорация Microsoft объявили о начале сотрудничества в технической, маркетинговой и патентной сферах.

Ну, с техническим сотрудничеством все более или менее понятно: оно направлено на обеспечение совместимости Windows и Linux (точнее, конкретно SLES) в гетерогенных средах, совместимости форматов документов (впрочем, далеко не все из них фигурируют в опубликованных материалах), увязку фирменных служб каталогов (ActiveDirectory и eDirectory).

Маркетинговое сотрудничество также вопросов не вызывает: отныне Microsoft <<дает добро>> тем своим пользователям, которые нуждаются еще и в Linux'е, на применение SLES от Novell. Конкретно этот <<одобрямс>> пока выражается в том, что Microsoft будет распространять купоны на техническую поддержку SLES~--- как это будет выглядеть в реале, я, честно говоря, представляю с трудом.

Так что наибольшее внимание широких народных масс привлекла патентная сторона соглашения. Согласно которой Microsoft предоставляет своего рода индульгенцию разработчикам Novell на использование в Linux'е своих патентованных технологий. А также дает иммунитет пользователям Linux-дистрибутивов Novell от судебных преследований по патентным искам. Более того, индульгенция распространяется, как будто бы, и на независимых разработчиков открытого софта, не используемого в коммерческих целях.

Все это, конечно, очень благородно и должно только приветствоваться. Вот только о том, что патентованные технологии Microsoft тем самым станут открытыми в плане доступности кода и свободными с точки зрения условий распространения, не говорится ни слова. И потому сразу же возникает тот же самый вопрос, который встал после предположения Эрика Реймонда о возможности включения в ядро Linux фрагментов проприетарного кода, в частности, драйверов устройств.

С этим вопросом тесно связан и другой момент, который можно назвать идеологическим или, если угодно, психологическим: не есть ли допуск разработчиков Open Source к патентованным технологиям попыткой вовлечения их в дебри проприетаризма? И не будет ли в дальнейшем найдена юридическая лазейка для требования оплаты патентных отчислений? Или, того паче, для <<прихватизации>> свободного кода, созданного с использование кода патентованного (или тесно с ним интегрированного). Существующие законы о патентном и авторском праве, а также свободные, GPL-совместимые, лицензии, как будто бы не могут такого допустить. Однако, зная изощренность Microsft в юридическом крючкотворстве, помноженную на финансовую мощь корпорации, кто рискнет утверждать невозможность подобного исхода?

И если такой исход будет иметь место~--- последствия его предсказуемы с трудом. Опыт давнишней истории с System V и BSD учит, что избавиться от самой малости проприетарного кода гораздо сложнее, чем его включить. И это в свое время сыграло роковую роль в судьбе всех BSD-систем вообще и FreeBSD в частности.

Кстати: в ответ на соглашение Microsoft и Novell компания Red Hat в буквальном смысле грудью бросается на амбразуру дзота. Предлагая своим клиентам, в случае патентных претензий третьей стороны (предлагается угадать с трех раз, какой именно), <<встать на их место>> и принять удар на себя. Юридически или технологически~--- это другой вопрос. Судя по всему, предполагается переписывать код, вызвавший патентные трения. Вот только насколько легко это будет сделать? И не окажется ли это той самой соломинкой, которая ломает хребет верблюду?

Есть и еще один осложняющий фактор. Не секрет, что многие разработчики Open Source давно уже не являются энтузиастами-любителями, сочиняющими код в качестве хобби (вспоминается фраза из давнишней статьи в ныне не существующем компьютерном еженедельнике <<Софтмаркет>>: <<В свободное от службы время поручик был программером и сочинял разные программки>>). Нет, многие из них состоят в штате IT-компаний и получают зарплату именно за разработку открытого софта. В частности, кое-кто из ключевых разработчиков ядра Linux расписывается в платежных ведомостях компаний Novell и Red Hat. И как они поведут себя в том случае, если встраивание закрытого кода в ядро будет их служебным заданием? Если положительно~--- то корпорации, выплачивающие им зарплату, вполне могут полагать, что подобно сыну турецкоподданного, и сами прекрасно справятся с делом дальнейшего развития Linux.

С другой стороны, те же самые ключевые разработчики ядра не только непосредственно программируют сами~--- кроме того, они еще и аккумулируют код, написанный множеством независимых разработчиков, которые получают зарплату совсем в других местах и совсем за другое (тут на память приходит Кон Коливас, врач-анестезиолог~--- но имя таким разработчикам легион). И возможность такой аккумуляции зиждилась и зиждется исключительно на доверии: на убежденности широких программирующих масс в том, что ключевые разработчики действуют в интересах сообщества, а не той или иной корпорации. Если это доверие будет поколеблено (обоснованно или необоснованно~--- это другой вопрос), распадется вся <<вертикаль власти>>, связывающая сообщество. И последствия этого могут быть весьма печальными. В частности, многочисленные форки ядра Linux, что, при наложении на уже имеющееся изобилие его дистрибутивов, даст картину воистину апокалиптическую. В лучшем случае, это будет откат <<чистого>> Open Source на позиции до 1999 года, когда в прессе и народе впервые заговорили о Linux-буме. 

Да и <<коммерческому>> Linux'у это счастья не принесет~--- база его разработчиков резко сузится. Причем за счет наиболее квалифицированных энтузиастов~--- и наибольших энтузиастов среди квалифицированных.

И, наконец, нельзя забывать о <<несгибаемых>>~--- Ричарде Столлмене, GNU/FSF и разработчиках Debian. Которые заведомо не пойдут на компромисс ни с какими проявлениями проприетаризма. И они могут оказаться третьим центром кристаллизации среды Open Source.

В общем, не буду выступать в роли пророка. Одно ясно~--- после этих <<Девяти дней>> мир Open Source изменится~--- и весьма сильно. К добру это будет, или к худу, мы узнаем уже через несколько месяцев\dots

\section{Урожай сенсаций} 
\begin{timeline}18 Сентябрь 2007 г\end{timeline}

Осень~--- пора сбора урожая, в том числе и урожая сенсаций. Вспомним осень минувшую, когда бурно обсуждались сначала альянс Novell и Microsoft, а потом~--- антитеза Oracle vs Red Hat.

Но похоже, что нынешняя осень будет особенно урожайной на сенсации. Судите сами: на дворе еще только сентябрь, а уже два неожиданных известия дают повод для обсуждения.

Во-первых, это объявление об открытии исходных текстов ОС реального времени QNX. Во-вторых~--- сообщение о банкротстве SCO Group, фирмы, пусть и одиозной в глазах всех сторонников Open Source, но являвшейся не последним игроком в ряду Unix-ориентированных компаний.

До конца осени еще далеко. Что ждет нас впереди?

\section{QNX, открой личико!} 
\begin{timeline}18 Сентябрь 2007 г\end{timeline}
QNX\dots Таинственная, несмотря на прекрасную документированность (в том числе и на русском) система, о которой многие слышали, но мало кто видел. Микроядерная ОС реального времени, с собственным, весьма своеобразным, графическим интерфейсом, именуемым Photon (и действительно работающим с субсветовой скоростью). Система промышленного назначения, вовсе не рассчитанная на десктопы (хотя при определенных условиях могла бы использоваться и в этом качестве). Которой приписывают управление ядерными реакторами и ракетными комплексами. Быстрая и компактная. Бесплатная при индивидуальном применении и более чем дорогая при использовании коммерческом. И~--- до недавнего времени~--- система сугубо закрытая.

И вот~--- сенсация: производители~--- а в этом качестве традиционно рассматривается фирма QNX Software Systems, которая с некоторых пор (точнее, вот уже почти три года) фактически принадлежит автомобильному концерну Harman\dots Так вот, производители начали поэтапное открытие исходных текстов QNX.

Первым этапом было предоставление доступа к исходным текстам знаменитого микроядра QNX Neutrino, главной библиотеки C, некоторых драйверов для взаимодействия оборудования. Получить это богачество можно уже сейчас.

В последующем планируется обречь этой же участи исходники графического интерфейса пользователя~--- не менее знаменитого Photon'а, файловой системы, базовых утилит. После чего эту систему можно будет считать почти столь же открытой, как Linux или любую BSD.

Почти~--- потому что лицензия, под которой открываются исходники, весьма своеобразна. Это не BSD-лицензия, и тем более не GPL любого рода. Имя ей~--- QNX hybrid software model. На деталях её пока задерживаться не буду. Замечу только, что она напомнила мне лицензию, под которой Trolltech распространяет Qt: возможность сторонним разработчикам вносить изменения в код, бесплатность для некоммерческого использования и платность~--- для использования коммерческого. Причем разработчики не обязаны делать достоянием общественности свои достижения, а вполне могут сохранять их в составе закрытых систем (и здесь мы видим влияние скорее лицензии BSD, нежели GPL).

О причинах, толкнувших на такой шаг разработчиков, не берусь даже гадать. А вот о его возможных последствиях порассуждать интересно.

В первую очередь, каких последствий точно не воспоследует.

А именно, не следует ожидать, что разработчики Open Source всё бросят и кинуться писать драйверы для QNX, призванные поддерживать все изобилие PC'шного оборудования. То есть <<десктопизации>> этой ОС не произойдет. Хотя приток независимых разработчиков в областях традиционного использования QNXС, конечно будет~--- вероятно, это и есть один из резонов производителя открыть исходники.

Во-вторых, наивно было бы ожидать и того, что конечные пользователи Linux или BSD будут сносить свои годами проверенные и привычные системы, устанавливая вместо них (или даже вместе с ними) QNX.

В-третьих, не следует думать, что своим актом разработчики превратили QNX в систему открытую и свободную (в понимании ли FSF или движения Open Source). Нет, она остается собственностью соответствующей компании (а в конечно счете, вероятно, концерна Harman). И последней вольно как открыть исходники, так и закрыть их~--- если нынешнее мероприятие почему-либо не оправдает ожиданий собственника. В 90-х годах это проделывали и Sun со своим Solaris'ом, и DEC с True64 Unix (хотя исходники последней закрывал, кажется, уже Compaq).

Нет, значение открытия исходников QNX видится мне в другом. А именно~--- во взаимном обмене идеями. Как известно, все хорошее, что появляется в Linux'а, очень быстро перекочевывает в BSD-системы, и наоборот. Так что теперь и многое хорошее из QNX теоретически может перекочевать в открытые Unix-подобные системы. И не обязательно на уровне кода~--- возможно, на уровне идей.

А что хороших идей в QNX заложено немало~--- думаю, спорить не будет никто из тех, кто хоть раз видел эту систему. Одна идея сверхлегкого и сверхбыстрого Photon'а дорого стоит. Кто знает, а вдруг она найдет свое предназначение, например, в Minix3? Или~--- в DragonFlyBSD? Ведь из всех монолитно-ядерных систем она идеологически наиболее близка <<микроядерщикам>>.

\section{История одного банкротства. Конец SCO?} 
\begin{timeline}18 сентября 2007 г\end{timeline}

\hfill \begin{minipage}[h]{0.45\textwidth}
This is the end of Solomon Grundy
\begin{flushright}
\textit{Древнеанглийский стишок}
\end{flushright}
\bigskip\end{minipage}

Очередная сенсация~--- объявление о банкротстве SCO согласно главе 11 Кодекса о банкротстве США. Это не значит, что деятельность компании прекращается, а имущество её идет с молотка. Глава 11 предусматривает для фирмы, объявившей себя банкротом, нечто вроде финансовой <<передышки>>, в частности, запрещая кредиторам на опредленный срок взыскивать с нее долги (с деталями можно ознакомиться здесь).

Эта новость на форумах тематики Open Source обсуждается повсеместно, активно и злорадно. Действительно, компания SCO <<прославилась>> своей тяжбой с IBM относительно незаконного раскрытия той использования исходного кода UNIX, а в дальнейшем~--- откровенным шантажом майнтайнеров Linux-дистрибутивов и их пользователей. Чем и снискала законную <<любовь>> всех сторонников свободного софта. Однако в этих обсуждения проскальзывает немало неточностей, во многом обусловленных запутанностью наименований, которые я и хотел бы прояснить в этой заметке. 

Для начала вкратце напомню историю вопроса. Потом посмотрим, может ли это событие быть основанием для радости (и тем более злорадства). А под конец попробуем извлечь мораль из всей этой истории.

Собственно, компания Santa Cruz Operations (известная также под аббревиатурой SCO~--- к современной SCO Group она имеет косвенное отношение)~--- одна из первых фирм, которая начала выпускать UNIX для PC, практически сразу после появления i386 (другой такой фирмой была, как ни странно, Microsoft). Название её происходит от местечка Санта Крус (Калифорния), где располагалась её штаб-квартира.

Разрабатывавшаяся в SCO система изначально называлась SCO UNIX (ныне она известна под именем SCO OpenServer). Она никогда не блистала технологическими изысками и новинками, но считалась очень устойчивой и надежной. И потому получила довольно широкое распространение в банковской сфере. В частности, начиная с прошлого века, версии этой ОС на протяжении многих лет используются в Российском Сбербанке, где под нее было написано немало специализированного софта (например, система коммунальных платежей).

Это одна линия истории. Другая же связана с именем Novell. В первой половине 90- годов проглого века эта компания была одержима хватательным рвением~--- ею были приобретены DR DOS, WordPerfect, QuattroPro, и в числе прочего~--- также права на исходный код и торговую марку UNIX (оставшиеся как бы бесхозными после разделения их прародителя~--- AT\&T). На базе чего была создана система UNIXWare.

>>Офисное>> направление деятельности Novell особого успеха не имело, как и её UNIX-бизнес. В результате один из основателей фирмы, Рэй Нурда (Ray Noorda), ушел в оставку, а все новоприобретения~--- распроданы в розницу: офисные пакеты~--- фирме Corel, где они составили пакет Corel Office, агония которого продолжается чуть не по сей день, UNIXWare~--- SCO, дополнив её собственную систему, получившую отныне имя SCO OpenServer.

Рэй же Нурда основал новую компанию~--- Caldera Systems, которая, в частности, приобрела у Novell DR DOS. Однако основным направлением её деятельности стала Linux-дистрибуция. Первоначально Caldera Linux представляла собой цельнотянутый Red Hat, дополненный некоторыми не вполне свободными компонентами, типа рабочего стола и средств интеграции с сетями Novell Netware. Однако вскоре этот дистрибутив обзавелся собственным инсталлятором (ИМХО, одним из лучших графических инсталляторов своего времени), средой Caldera Network Desktop. И вообще приобрел своеобразие, отмеченное именем Caldera OpenLinux.

Свободная версия Caldera OpenLinux~--- представляла собой очень компактный, но аккуратно укомплектованный дистрибутив, широкому использованию которого препятствовали только слабые средства интернационализации. А коммерческая версия включала в себя немало проприетарных продуктов, вплоть до Wabi (средства эмуляции Windows от фирмы Sun), WordPerfect и CorelDRAW~--- квази-портов (квази~--- потому что на самом деле они работали в режиме эмуляции), StarOffice, и так далее.

Наконец, Caldera становится одним из соучредителей альянса United Linux~--- наряду с Suse (тогда это была Европа), Turbolinux (Япония) и Conectiva (Бразилия); таким образом, альянс этот охватил чуть не все континенты (африканской Ubuntu тогда еще и в проекте не было).

Целью альянса была стандартизация дистрибутивов Linux. Тем не менее, никакими славными делами он не отметился, тихо и незаметно прекратив свое существование года через два. Что и знаменовало собой начало конца.

Однако это еще впереди. А пока, на рубеже тысячелетий, Caldera выглядит процветающей и преуспевающей фирмой. Тогда как у SCO, несмотря на наличие такого могучего клиента, как Сбербанк, дела, судя по всему, идут не блестяще: видимо, две UNIX-системы оказались ей не под силу. И она продает оба своих флагманских продукта~--- SCO OpenServer и SCO UNIXWare~--- фирме Caldera.

Та же, в свою очередь, по-видимому, испытывает разочарование в своем Linux-бизнесе: первый Linux-бум рубежа тысячелетий схлынул. И Caldera начинает переориентироваться на UNIX-бизнес~--- во главу угла ставятся новоприобретенные SCO OpenServer и SCO UNIXWare. В соответствие с чем меняется и название~--- отныне фирма называется SCO Group, и SCO здесь уже условное буквосочетание, а не аббревиатура (базируясь в штате Юта, она ни малейшего отношения к местечку Санта Крус не имеет).

Прежняя же SCO, оставшись без своих титульных систем, переименовывается по названию последнего своего продукта~--- Tarantella. Впрочем, жизнь её была недолгой: остатки древней Santa Crus Operations были приобретены Sun'ом, и ныне это просто подразделение последнего.

Однако вернемся к героине нынешней сенсации~--- SCO Group (далее вторую часть названия я буду опускать, но прошу помнить, что к почившей в Бозе Santa Cruz Operations она имеет очень опосредованное отношение).

Непосредственно после приобретения двух новых UNIX'ов развитие Caldera OpenLinux прекратилось: последняя её версия датируется концом 2002 года (и выступает уже под именем SCO Linux). Однако, судя по всему, и UNIX-бизнес у свеже-переименованной компании тоже не особо заладился. И она предпринимает отчаянный шаг~--- начинает тяжбу с IBM о нарушении той условий лицензирования исходного кода UNIX.

Нормальному советскому человеку понять суть претензий SCO довольно сложно; да и антисоветскому человеку~--- тоже, если он в должной мере не владеет юридическим английским. В меру своего понимания попробую их изложить:

IBM, будучи обладателем лицензии на исходники первозданного UNIX (полученной в лохматые годы еще от AT\&T), выложила часть их в открытый доступ, откуда они были заимствованы разработчиками ядра Linux. Тем самым потенциально подвергнуться судебному преследования могли и разработчики этой ОС, и майнтайнеры дистрибутивов, и даже конечные пользователи (в первую очередь, конечно, корпоративные).

Правда, аргументы, выдвинутые SCO в подтверждение своих притязаний, смотрелись смехотворно: это был исходный текст функции \texttt{malloc}, опубликованный в открытой (<<бумажной>>) печати отцами-основателями UNIX еще во времена былинные. Да и вообще весь иск в глазах любого здравомыслящего человека выглядел смешным. Однако сутяжничество~--- американский национальный вид спорта, и потому процесс продолжался около двух лет.

Правда, SCO тут же нарвалась на встречный план (пардон, иск) со стороны Novell. Которая, со своей стороны, оспорила права SCO и на исходный текст первозданного UNIX, и на саму торговую марку.

Перипетии обоих процессов пересказывать не буду. Во-первых, о них не писали только те СМИ, обозреватели новостей в которых страдали патологической ленью. Во-вторых, оба процесса SCO в первом приближении проиграла. В первом приближении~--- потому что теоретически, по правилам этого спортивного состязания, может оспорить оба судебных судебных решения. Насколько в нынешних условиях она имеет возможность сделать это практически~--- вопрос.

И вот~--- закономерный итог всех предшествующих событий, сообщение о банкротстве. Как я уже говорил, в соответствие с главой 11 американского кодекса, это вовсе не значит, что SCO немедленно прекращает своё существование. Так что в строке из эпиграфа к заметке стоило бы поставить вопросительный знак, однако в оригинале его нет, а мы ведь свято чтим настоящее авторское право~--- даже если оно принадлежит народу.

Финансовая <<передышка>> часто позволяет обанкротившимся компаниям выпутаться из трудной ситуации~--- примером чему может служить SGI. Сможет ли это сделать SCO~--- поживем, увидим. Но мне представляется, что в любом случае с юридическими претензиями её на UNIX и Linux можно считать поконченным~--- <<не до грибов, Петька>>, как сказал Василий Иванович в ответ на сообщение~--- <<Белые в лесу>>.

Так что же, следует радоваться и кидать в воздух шапки? Ведь бабло (пардон, добро) в очередной раз победило зло, и восторжествовала справедливость. Не будем спешить.

Во-первых, радоваться несчастью ближнего, даже если он сам был его кузнецом,~--- не в нашем, товарищи, духе. А во-вторых, окончательное банкротство SCO может иметь негативные последствия, которые не в последнюю очередь затронут Россию.

Вспомним, что весь Российский Сбербанк работает под управление SCO UNIX. Более того, в конце июля текущего года Сбербанк объявил о результатах тендера по тотальному обновлению системного программного обеспечения до версии OpenServer 6. Он был выигран Премьер-Партнером SCO в России~--- компанией Бизнес-Консоль.

Впрочем, широкие массы IT-общественности, как и положено, узнали об этом из сообщения буржуазной прессы, датированного 30 августа. И успел ли состояться этот самый тотальный апгрейд~--- остается не вполне ясным. И если нет~--- то состоится ли он теперь?

Да впрочем, это и не важно. Конечно, такого рода системы способны долго существовать и без всяких апгрейдов. Однако возникает вопрос с текущей технической поддержкой. Разумеется, для Сбербанка она осуществляется не напрямую SCO, а какими-нибудь её российскими авторизованными партнерами (скорее всего, фигурирующей выше Бизнес-Консолью). Однако очевидно, что в сложившейся обстановке эти партнеры (или этот партнер) будут постепенно переориентироваться на другие сферы деятельности или просто прекратят свое существование. Да и физический износ парка компьютеров и, соответственно, необходимость в новых инсталляциях, никто еще не отменял.

Предвижу возражение: вместо апгрейда продукции SCO Сбербанк может просто сменить ОС. Да, может. Но, во-первых, это потребует времени~--- представьте себе количество его отделений по городам и весям Руси. А во-вторых, как уже упоминалось, именно под SCO UNIX написаны многие специализированные банковские системы. Конечно, их, скорее всего, можно перекомпилировать под какой-либо иной UNIX~--- но и это, с учетом отладки, не сделать в одночасье. В любом случае, процесс смены ОС не пройдет бессбойно и безболезненно.

В общем, мне уже видятся бабушки и дедушки, отстаивающие очереди в тщетной надежде получить свои пенсии, ошибки в счетах за коммунальные услуги (которые по какой-то роковой случайности никогда не бывают в пользу потребителей оных), и прочие апокалиптические картины. И все это, между прочим, в преддверии грядущих выборов. Так что от злорадства по поводу судьбы SCO я бы воздержался\dots

Остается извлечь из произошедшего события мораль, свежую и оригинальную: вот к чему может привести ориентация государственных структур на закрытые проприетарные продукты. А Сбербанк, несмотря на свое формально акционированное положение, в значительной степени остается структурой государственной, в том числе и в глазах широких масс электората.

Действительно, представим себе, что на месте SCO UNIX в банковских системах используется какой-либо коммерческий, но основанный на открытых исходниках продукт, например Red Hat или Suse. И соответствующую компанию в одночасье постигает банкротство (от чего Господь борони и ту, и другую). Что же~--- свято место пусто не бывает: немедленно будет создан форк, неотличимый от исходной системы. Теми же останутся и люди, оказывающие техподдержку. Конечно, сбои неизбежны и в этом случае~--- но масштабы их не сопоставимы с теми, которые можно ожидать при умирании продукта проприетарного.

Так может быть, произошедшее заставит наконец задуматься наши власти предержащие о роли свободного софта в современных условиях?

\section{Третья сенсация осени} 
\begin{timeline}23 октября 2007 г\end{timeline}
Начинает осуществляться предсказание, что нынешняя осень будет богата сенсациями: Microsoft предложила миру две свободные лицензии. Нет ли у нее намерения на корню скупить мир открытого софта?

Позор на мою седую репортерскую голову~--- на эту сенсацию я реагирую с изрядным запозданием. Дело в том, что новости, касающиеся Microsoft, я обычно пропускаю~--- даже если они попадаются на FOSS-ресурсах: мы с этой компанией существуем в параллельных мирах, и все, что касается Microsoft, Windows и тому подобных материй, интересует меня не больше, чем геолога, работающего в корякской тундре~--- погода где-нибудь на Багамах. Но тут, проглядывая как-то новости на нескольких сайтах, случайно наткнулся на известие о свободных лицензиях от Microsoft~--- свидетельстве того, что жители солнечной Багамии неожиданно заинтересовались жизнью тружеников тундры. А пролистав несколько сопряженных новостей, понял, что это и есть следующая сенсация текущей осени из числа предсказанных в одной из предшествующих заметок.

Давным давно, лет 10 назад, Сергей Леонов в Компьютерре написал нечто вроде следующего: я поверю в серьезность Linux'а, когда не линуксоиды будут ругать Microsoft, а Microsoft начнет ругать Linux. Момент это наступил пару-тройку лет назад~--- с публикаций о совокупной стоимости владения Linux и Windows. Разумеется, с выводами в пользу второй~--- кто же будет говорить, что у него все плохо, а рядом все хорошо.

С этого момента и можно говорить об интересе багамцев к тундровикам (пардон, Microsoft к Linux'у). Правда, поначалу он носил сугубо негативный характер, и сводился к утверждениям того, что, мол, у нас на Багамах тепло и дешево, а у вас в тундре холодно и дорого. Цифры и графики сравнения совокупной стоимости владения с web-страниц специализированных сайтов перекочевали на рекламные страницы компьютерных журналов~--- кажется, не было такого, кто бы не отметился таким образом.

Интересно, а как подобная реклама согласуется с соответствующими российскими законами? Не случайно ведь производители стиральных порошков или моющих средств, говоря о их несравненных достоинств в сравнении с конкурентами, никогда не называют конкретных имен, а используют термин <<обычный стиральный порошок>>, <<обычный подгузник>>, и так далее. Представляете себе рекламу Microsoft в таком контексте:


\begin{shadequote}{}
Совокупная стоимость владения для системы Windows на столько-то процентов ниже стоимости владения обычной операционной системой.
\end{shadequote}

Улыбает, правда?

Ну да бог с ней, стоимостью владения. Тем паче, что на стадии развития негативной ветви своего интереса к Linux'у Microsoft выдала перл не хуже. Я имею ввиду баннер с предложением дешево продать Linux-сервер.

Кстати, Microsoft, как с ней нередко бывает, в своей рекламе оказалась\dots ну скажем мягко, не совсем точна. И когда лица, заинтересованные в покупке дешевого Linux-сервера, стали обращаться по этому поводу в Российское представительство компании, выяснилось, что Linux-серверов на продажу у них и нет. Впрочем, обещание рассмотреть запросы было дано~--- правда, насколько мне известно, пока безрезультатно. Или опять Linux-серверов на всех не хватило, как сладких пряников в песне Булата Окуджавы?

Однако постепенно интерес Microsoft к Linux'у приобрел более позитивный характер. В итоге это выразилось в прошлогоденм соглашении с Novell о начале сотрудничества в технической, маркетинговой и патентной сферах. Оно вызвало бурную реакцию сообщества FOSS~--- по преимуществу отрицательную. Однако пока, по прошествии года, ничего сверхъестественно фатального в результате него не произошло. Попытки шантажа Linux-компаний угрозами патентных исков никем всерьез восприняты не были. А некоторые положительные последствия~--- например, улучшение взаимодействия между службами каталогов Active Directory и eDirectory~--- имеют место быть. Но давайте посмотрим дальше\dots

А дальше происходит вот что~--- неожиданное объявление о том, что Microsoft предложила сразу две лицензии распространения открытого софта, Microsoft Public License (Ms-PL) и Microsoft Reciprocal License (Ms-RL). И более того, обе они были признаны Open Source Initiative (OSI)~--- организацией, определяющей степень свободы лицензий. Так вот, рассмотрев предложения от Microsoft, OSI признала, что они соответствуют всем десяти критериям лицензий открытых лицензий~--- Open Source Definition. И следовательно, код, распространяемый под Ms-PL и Ms-RL, может использоваться в проектах, лицензируемых, например, под GPL, и наоборот.

Убедиться в том, что Microsoft отныне приобщилась к миру открытого софта, можно, ознакомившись с полным списком <<совместимых>> лицензий в алфавитном порядке. Хотя любопытно, что в списке лицензий по категориям ни Ms-PL, Ms-RL нет. Почему? Не успели внести? Или OSI затрудняется в определении того, к какой категории их приписать?

Интересно также, что Microsoft подала свою заявку в обычном режиме, как это делают любые другие организации и частные лица. И в первой редакции~--- как Microsoft Permissive License,~--- она была отвергнута из-за несовместимости с другими Open Source-лицензиями. И лишь вторая попытка редмонтцев внедриться в ряды последователей FOSS увенчалась успехом.

Вникать в тексты лицензий от Microsoft полагаю излишним. Уж если пуристы из OSI признали их свободными и совместимыми (в они в этом праве отказывают т.~н.~<<старой>> лицензии BSD или, например, последней редакции лицензии проекта XFree86), то простые трудящиеся, не имеющие юридического образования, вполне могут положиться на их мнение.

Интересней другой вопрос~--- а какова тайная цель, преследуемая компанией при создании своих открытых лицензий? Ведь сообщество FOSS всегда ожидает подвоха с её стороны. И, возможно, не без оснований. Ибо буквально через несколько дней Стив Балмер в интервью, данном им на конференции по Web-2, заявил о намерении его компании начать приобретение фирм Open Source направленности.

И тут поневоле приходит на память старая истина советских начальников:


\begin{shadequote}{}
Если пьянку нельзя пресечь~--- её надо организовать и возглавить.
\end{shadequote}

Не убедилась ли Microsoft в результате своих предыдущих действий, что движение FOSS ей пресечь не удастся? И значит, пора принимать меры к тому, чтобы его возглавить. Ну а потом уже организовать~--- в меру своего понимания, <<как надо>>~--- ведь тайное знание того, что именно надо конечному пользователю, во все времена отличало эту компанию. Даже тогда, когда сам пользователь еще и не подозревал о своих потребностях\dots

И в этом аспекте события последних лет укладываются в стройную и логичную цепочку. Первым звеном в которой можно считать приглашение на работу Дэниэля Роббинса~--- создателя дистрибутива Gentoo,~--- в качестве эксперта по свободному программному обеспечению (июнь 2005 года).

Не склалось~--- разрулив свои финансовые проблемы, DRobbins покинул Microsoft. Но на протяжении более чем полугода (вплоть до февраля 2006-го) ему ведь за что-то платили зарплату? Ту самую, что позволила ему избавиться от финансовых трудностей (интересно, а почему ни одна компания, бизнес которой был связан с Open Source, во времена оных не предложила ему достойной работы?). То есть~--- чего-то рассматривал, давал экспертные заключения, и так далее~--- то есть выполнял действия, вполне понятные любому, кому приходилось быть или бывать экспертом в любой сфере.

Следующее звено~--- осень 2006 года, заключение уже упоминавшегося соглашения с Novell. Которое, как я уже говорил, вопреки всем ожиданиям, ничего плохого, кроме хорошего, второй брачующейся стороне пока не принесло (за исключением, разве что, осуждения коммьюнити, на что Novell ответила прекращением контактов с ним). Что вполне укладывается в модель <<возглавить и организовать>>: заманивают. Особенно с учетом просочившихся сведений о попытках заключить аналогичное соглашение с Red Hat, Mandriva и Canonical (спонсором разработки Ubuntu)~--- попытках, успехом не увенчавшихся. Ответом на что были угрозы патентными исками к Linux-компаниям. То есть мы видим оборотную сторону любого заманивания в виде политики кнута и пряника.

И, наконец, последние события: открытые лицензии и намерения по покупке компаний FOSS-ориентации,~--- лежат все в том же русле. Сначала втереться в ряды FOSS-сообщества, с целью использования его потенциала как разработчиков, а потом~--- скупка на корню всего и вся и ликвидация движения FOSS как такового. После чего никаких угроз монополии Microsoft уже не предвидится даже в отдаленной перспективе.

Вероятен ли такой сценарий? Если обратиться к истории и проследить творческий путь компании~--- более чем. Но вот реализуется ли он? Конечно, там, у них (а теперь и у нас)~--- бездушный и бездуховный мир чистогана, где всё покупается и все продается. А поскольку все мы люди, все мы человеки, Microsoft может предложить такие условия, от которых не откажутся многие приверженцы FOSS~--- тем более, что имеет к тому все возможности.

И все же\dots Пример Дэна Роббинса показывает, что разработчику свободного софта ужиться в мире софта проприетарного не так просто. Ну а о том, что Брут-RMS никогда не продастся большевикам (пардон, проприетарщикам), думаю, не сомневается никто.

Однако старая шутка линуксоидов о скором выходе Microsoft Linux вполне может стать реальностью. Осуществится ли такое, и какие будет иметь последствия для сообщества FOSS~--- покажет время.

\section{Мир без солнца} 
\begin{timeline}Июнь 2009\end{timeline}

Разговоря о продажи фирмы Sun циркулируют в Сети давно. А ныне факт покупки её компанией Oracle можно считать почти свершившимся: юридические вопросы с иском акционеров, недополучивших, как им кажется, своего бабла, по мнению знающих людей, будут улажены легко (и очевидным для нас способом). Какие следствия для мира FOSS будет иметь исчезновение старейшей UNIX-компании? Напомню, что на её иждивении находится ряд крупных свободных проектов~--- Openoffice.org, MySQL, VitrualBox, не говоря уже о собственно ОС~--- OpenSolaris, и ряде средств разработки. Не загнутся ли они под чутким руководством Ларри Эллисона? 

Наибольшие опасения вызывает судьба OpenSolaris: а нужна ли будет Oracle ещё одна ОС, в добавление к собственному клону RHEL? ОС, за время своего <<свободного плавания>> не достигшая ни полностью работоспособного состояния, ни критической массы комьюнити? Мне кажется, что ответ будет отрицательным. Но так ли это страшно? Все здоровые инновации OpenSolaris (а их немало) могут быть легко инкорпорированы в Linux. И, чем чёрт не шутит, вдруг новые хозяева изменят лицензию на ZFS? После чего она легко впишется в Linux-ядро.

А за остальные свободные проекты Sun'а волноваться нечего: MySQL выступит <<легковеным>> дополнением к собственно Oracle, Openoffice.org не бросят, как востребованный конечным пользователем, VirtualBox, Sun Studio \textit{etc}.~--- как интересные для всех разработчиков.

И как знать, не увидим ли мы вскоре нового монополиста~--- теперь уже в сфере свободного софта? Да ещё в сцепке с собственной аппаратной платформой: не зря ведь Ларри обмолвился, что <<Sparc не бросим, потому что он хороший>>.