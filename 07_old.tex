\chapter{Старое}

\textsl{Под этой рубрикой я собрал свои старые заметки, написанные на рубеже тысячелетий и даже раньше. В основном они представляют интерес как исторические свидетельства. Хотя некоторые местами сохранили актуальность}.

\section{ОС в свободном исполнении} 
\begin{timeline}1998, весна\end{timeline}

\textsl{Это мой первый опыт сочинительства на Linux-темы. Опубликовано в виде письма в редакцию журнала Мир ПК, 1998, №4, с. 8. Размещается в виде, распознанном со скана (за что спасибо @Taciturn'у с Джуйки), с минимальной орфографический и стилистической правкой.}\medskip

Поводом для этого письма послужила <<Колонка редактора>> из №11/97. Стиль мне понравился: сдержанная ирония без наскучивших выпадов в адрес злобного монстра. Поэтому хочу поделиться некоторыми соображениями. Я не принадлежу ни к поклонникам, ни к ненавистникам Microsoft. Каковы бы ни были недостатки её продуктов, их доминирование~--- принудительная сила реальности (О. Куваев), особенно в России. И доминирование это нарастает со временем.

Ещё пару-тройку лет назад в каждом компьютерном журнале (и в Вашем в том числе) обсуждались сравнительные характеристики разных ОС для ПК. И список их (ОС) состоял не из одного и не из двух названий. Где же ныне NextStep для Intel? А OS/2? Как система для настольных персоналок она мертва, a Solaris вряд ли кто всерьёз рассматривал в качестве таковой. Да и Mac~--- если не умирает, то уж по крайней мере чувствует себя плохо.

Примерно то же происходит в области приложений. Несколько лет назад мой любимый текстовый процессор AmiPro на равных конкурировал с Word, a WordPerfect был уже недосягаемой вершиной. Где они теперь? У WordPerfect шансов на нашем рынке нет: последняя русская версия, если не ошибаюсь, 6.1 (ещё под Win 3.1). А AmiPro, этот <<Друг профессионала>> (или <<Подруга>>?) утерял(а) всё, даже собственное имя, сменив его на безликий WordPro. Мортиролог можно продолжить. Но безальтернативность убивает свободу. Ведь свобода~--- это свобода выбора, пусть и неправильного.

И тем не менее альтернатива существует, хотя компьютерная пресса и обходит её дружным молчанием. Ваш журнал~--- почти единственный, который регулярно, пусть и редко, упоминает о ней. Это~--- Linux, FreeBSD и другие некоммерческие Unix-подобные системы.

Однако все прочитанные мной книги и статьи о Linux имеют одну характерную особенность. Подробно описывается процесс установки системы. А что дальше? Перекомпилировать ядро? А потом ещё раз, добиваясь высшего совершенства~--- полного соответствия своим устройствам. А ещё можно написать программу для какого-нибудь своего уникального устройства, включить её в ядро и быть единственным в мире обладателем операционной системы, специально поддерживающей\dots ну, не знаю что. А что же все-таки можно под Linux делать?

Обычно пишут, что <<софта>> под Linux <<немерено>>, но упоминают, как правило, опять-таки компиляторы. Но не компилятором же единым жив человек! Утверждают, что всякого freeware и shareware очень много в Internet. И это действительно так. Но скачивать все подряд, наугад, руководствуясь лишь невнятным указанием, что это <<программа для модема>>, или <<хорошая программа для модема>>, или <<очень популярная (интересно, среди кого?) программа для модема>>, по меньшей мере, неразумно: это потребует времени, сопоставимого с возрастом Метагалактики. А рассчитывать на такое долголетие трудно, да ещё и поработать с этими программами хочется\dots

Пройдемся еще по Internet. Сколько Web-серверов в России построено на Linux или FreeBSD? А сколько, для сравнения, на WinNT или коммерческих Unix? А еще на подходе HURD\dots Вот и получается:

\begin{itemize}
	\item Linux \textit{etc}.~--- единственная в настоящее время реальная альтернатива Windows для настольных персоналок. 
	\item Неплохо бы иметь представление о параллельном мире (по Евклиду), это не помешает каждому грамотному пользователю ПК. Ведь главное все же~--- знать, что возможность выбора есть, даже если никогда ею не воспользуешься. 
	\item Освещение темы Linux в печати явно недостаточно. Особенно это касается приложений, не относящихся к средствам разработки. А ведь для Linux есть и StarOffice, и ApplixWare, и аналоги Photoshop и многое другое (сам убедился). 
\end{itemize}

Поэтому предлагаю рассмотреть возможность создания в Вашем журнале раздела, посвящённого Linux, FreeBSD, HURD, X~Window в свободном исполнении, а также приложениям для них. Есть же у вас раздел Macworld\dots А значение Linux сейчас если и ниже Mac'а, то не намного, особенно в Сети.


\textit{Post Scriptum от сего дня: не смотря на то, что отдельные статьи по Linux'у в журнале 
Мир ПК
 (а затем и некоторых других) стали периодически появляться, состояние с русскоязычными печатными материалами продолжало оставаться неудовлетворительным, а с Интернетом тогда были напряги не только в городах и весях нашей необъятной Родины, но даже местами и в её столице. Да и там сайтов Linux- и UNIX-тематики было не густо. Вот я и решил восполнить пробелы, сначала на бумаге, в журнале 
Byte/Россия
 (первом компьютерном журнале на Руси, в котором появился специализированный раздел 
Byte/UNIX
), а потом и в Сети. С чего и началась моя карьера линуксописателя, но об этом~--- в другой раз и в другом месте.}

\section{Куда падают камни?} 
\begin{timeline}1998, январь\end{timeline}

\textsl{Написано немало лет назад по поводу статьи Михаила Ваннаха <<Камень, падающий вверх\dots>>, опубликованной в журнале <<Компьютерра>>, \textnumero1, 1998}\medskip

Традиционно статьи преподобного Михаила Ваннаха~--- в Компьютерре одни из самых интересных среди всех, прямо не связанных с комьютерной тематикой. Они отличаются нестандартостью подхода и парадоксальностью выводов, тем не менее, как правило, достаточно обоснованных. Не является исключением и его последняя статья. Тем не менее она вызывает вызывает определенные\dots не столько замечания, сколько соображения, вызванные дискуссионностью ряда её положений.

Начну по порядку, с первой гипотезы Михаила Ваннаха и её обоснования (автор называет это иллюстрацией). 

Сначала~--- цитата:

\begin{shadequote}{}
Гражданские технологии почти всегда обладают более высоким качеством и более высоким отношением цена/качество, чем технологии военные.
\end{shadequote}

Сразу скажу, что с известными оговорками, о которых я скажу ниже, с преимуществом гражданских технологий перед военными нельзя не согласиться. Однако обоснование этого представляется мне совершенно иным. 

Утверждение, что война не есть нормальное состояние общества, может быть распространено только на настоящий исторический момент (\textit{да и то~--- со множеством оговорок~--- вряд ли, начиная с августа 1914 года, был хоть один момент, когда в какой-либо точке земного шара не происходило локальных конфликтов}). Поскольку только для него можно (и то достаточно условно) использовать слово <<общество>> в единственно числе. А до этого обществ было достаточно много. К стати, постоянно подчеркиваемое Георгием Кузнецовым (
\textit{тогдашний главред Компьютерры}
.) различие менталитета русского (или российского? или советского?) человека от человека американского~--- не более чем иллюстрация положения Л.Гумилева, что этнические различия~--- это различия стереотипа поведения. А ведь как ни относись к концепции Гумилева в целом, в точности эмпирических наблюдений ему не откажешь. 

Во всяком случае, в истории очень многих обществ неоднократно бывали эпохи, когда война всех против всех, война как средство производства была именно нормальным их состоянием. В качестве примеров достаточно вспомнить кельтско-германское общество Европы или кочевые общества Евразийских степей. И они, как и всякие традиционные общества, способны были существовать и существовали на протяжении длительного времени. По крайней мере, гораздо дольше, чем современная цивилизация евро-американского типа. Общий стиль жизни евразийских кочевников, не смотря на смену языков и религий, почти не менялся по крайней мере с VII-VIII вв. до н.э. до XVI-XVII вв.~--- эры нашей, варварских обществ Европы~--- с VI в. до н.э. и до конца эпохи Великого переселения. А отголоски его сохранялись в т.~н.~феодальной Европе~--- и до XVIII в. 

Кстати, т.~н.~буржуазная демократия, на которой основана современная евро-американская цивилизация и, следовательно, компьютерные технологии, в основе своей имеет не демократию Древней Греции (менее демократического, с современной точки зрения, общества трудно себе представить) или общехристианскую мораль, а именно стремление полудиких готских или франкских баронов отстоять свою личную независимость от столь нелюбимого преподобным Михаилом (да и многими другими) государства. 

Таким образом, нельзя согласиться с тем, что слабость военных технологий определяется узостью рынка. И не в отсутствии конкуренции~--- ведь в условиях перманентной войны, распадающейся на множество столь же постоянных локальных конфликтов, техническое превосходство должно было бы сказаться исторически мгновенно. Однако этого почти никогда не происходило. На мой взгляд, объяснение отставания военных технологий почти во все эпохи (подчеркиваю, почти~--- но об этом позднее) объясняется чисто практической причиной: эффективность ведения боевых действий связана не с технологическим превосходством, не с индивидуальным боевым мастерством (каковое можно рассматривать как частный случай мастерства технологического~--- в Японии, например, мастер рукопашного боя или фехтования, почитался наравне с прославленными поэтами, художниками, кузнецами или мастерами чайной церемонии). А, как это ни прискорбно для людей полагающих себя интеллигенцией, почти исключительно (и почти всегда) с превосходством в живой силе и её организации. 

Известный Н. Буонопарте однажды произнес фразу, варьирующую в разных переводах, о том, что счастье (или правда) всегда на стороне 
б\'{о}льших батальонов. А он кое-что в этом понимал. Военная машина революционной Франции была создана до Бонапарта (в том числе и теми, кого потом революция полечила от головной боли самым эффективным средством~--- гильотиной), и, впервые в истории, основывалась на всеобщей воинской повинности. Каковую не следует путать со всеобщей вооруженностью варварского общества~--- там владение оружием и участие в боевых действиях были просто атрибутом полноправного свободного человека: не хочешь~--- переходи в личную зависимость от другого, кто, в том числе, обязуется и защищать тебя. 

Прочие же европейские армии комплектовались на основе рекрутского набора или (как в Англии) на квазидобровольной основе. Что и дало решающее преимущество генералам революционной Франции. И легло в основу их тактики (создание которой тоже приписывается будущему императору)~--- атакам глубоких пехотных колонн, поддержанных столь же глубокими колоннами тяжелой кавалерии. Атакам под ядра и картечь, не считаясь с потерями (кое на что похоже, не правда ли?). Но того, что доходило, хватало, чтобы проломить линейные порядки противника. 

Так зачем же, спрашивается, совершенствовать технологии? Не надо, чтобы штыки были совершенней или порох~--- лучше. Лишь бы штык колол, и пули летели. А главное~--- что бы и того, и другого хватило на всех загнанных под ружье\dots 

Так что причина отставания военных технологий не узости рынка, а в его массовости. Зачем делать штык из высокосортной стали, если вооруженный им солдат с вероятностью, близкой к 50\%, будет убит, пустив его в ход один раз. И приклад его ружья, будь он хоть чудом деревообработки, будет сломан о вражескую голову. Напротив, все, что предназначено для ведения боевых действий, должно быть максимально простым и дешевым при массовом производстве. Именно поэтому всю Вторую Мировую войну в немецкой армии продержался технически несовершенный, мало пригодный для полевой войны, ориентированный первоначально только на вооружение мотоциклетных команд и танковых экипажей, пистолет-пулемет ERMA (в простонаречии именуемый <<шмайссером>>), хотя в опытных образцах существовало более совершенное оружие~--- штурмовая винтовка Шмайссера (настоящего)~--- примерный аналог и предтеча автомата Калашникова и винтовок серии M. 

Что же касается эффективности массовой военной продукции в гражданской жизни\dots Будучи геологом, неоднократно наблюдал попытки использования таковой в мирных целях. Не знаю, как комплект полевой одежды армии США, о котором говорит преподобный Михаил, но нет ничего менее пригодного для горно-таежных или горно-тундровых условий, чем любое обмундирование армии советской. 

То же и для стрелкового оружия. Оно (и наше, основанное на т.~н.~патроне Мосина калибра 7,62~мм, и американское серии M, с очень сходным по баллистике патроном, и немецкое, с патроном имени братавьев Маузер калибра 7,92) приспособлено для вывода из строя (не убийства, а именно потери способности выполнять боевую задачу) одного-единственного зверя~--- человека. В результате такое оружие слишком сильно для, скажем мелких копытных, и безнадёжно слабо для крупного и опасного зверя. А уж подранков\dots Ведь раненого солдата проигравшей армии можно дострелить и потом, в спокойной обстановке. А зверь ждать не будет. Лично видел лося, раненого четырьмя карабинными пулями (из кавалерийского карабина калибра 7,62), причем одна пробила желудок, а другая~--- превратила легкие в нежный мясной фарш), который сохранил способность не только убегать, но и драться. И, в конце концов, был убит пулей Майера 12 калибра из гражданской двустволки не высшего разбора. 

Если же говорить об оружии штучном~--- никакое армейское оружие с ним и рядом не лежало. Можно возразить, что оно несопоставимо дороже, и вспомнить пресловутое соотношение качество/цена. Первое~--- безусловно верно, а вот второе\dots Если штуцер фирмы Голланд и Голланд, стоимостью в некую астрономическую сумму, спас вам жизнь при столкновении с медведем, какую цифру вы поставите в числитель этой дроби? 

Но это лирика, напрямую не связанная с технологиями. А потому вернусь к вопросу, могут ли военные технологии быть лучше гражданских. Несколько примеров такого рода мне известно. 

Выше я уже упоминал о двух крупнейших <<варварских>> сообществах Евразии~---  кельто-германском (правильнее сказать, древнеевропейском, поскольку в него, по крайней мере временами, или балты, славяне, иллирийцы и т.~н.~<<народы между кельтами и германцами>>) и степном кочевническом. Так вот, вопреки расхожему мнению, и те, и другие достигли весьма высоких для своего времени технологических высот, во многом превосходя своих так называемых цивилизованных современников. И проявилось это технологическое совершенство в первую очередь в военной области. Скажем, кельтское оружие далеко превосходило по качеству обработки металла и, так сказать, тактико-техническим данным, оружие римское. Чем подтверждается ранее сформулированное положение, что для массовой армии, каковой была римская, нужно не хорошее оружие, а много оружия. Правда, и в гражданских областях производства, скажем землеобработке, древнеевропейцы опережали греко-римский мир: достаточно сказать, что конская плужная запряжка~--- их изобретение. И позднее, вплоть до высокого средневековья, франкские и сделанные по их технологии скандинавские и славянские мечи широко экспортировались в <<высокотехнологичные>> страны Ближнего и Среднего Востока. 

Аналогично и в степном мире. Гунны первыми на Дальнем Востоке начали производство высокосортной стали. Собственно, с этого и начался здесь железный век. Позднее победы тюркютов Первого каганата над окружающими народами, такими, как жужани, были во многом определены их технологическим превосходством~--- слава тюркютов как железоплавильщиков и кузнецов докатилась и до Византии. Но, повторю, в этом случае технологическое превосходство оказалось определяющим только благодаря равенству прочих условий~--- численности, организации, способу ведения войны. 

В чем же причина этого исключения из общего правила? В высокой роли войны в жизни этих обществ? Но для победы в войне, как я пытался показать, важно не хорошее оружие, а много оружия. А ведь даже стрелы кочевников, т.е., в сущности, расходный материал, отличает чрезвычайная тщательность выделки, обусловленная не только повышением баллистических качеств. В том, что война была средством производства и оружие, в сущности, было продуктом не военных, а гражданских технологий, пользующихся массовым спросом? В какой-то мере это так. Однако численность варварских сообществ была ничтожной по сравнению с так называемым цивилизованным миром, и о действительно массовом спросе говорить трудно. 

Думается, объяснение в том, что в варварском обществе оружие было необходимым атрибутом свободного, полноправного человека; не имевшие же оружия находились в разных формах личной зависимости. И поэтому оружие было необходимо для достижения максимально возможной в данных условиях личной независимости. А тут уже качество выходило на первые роли. Какой толк в многочисленных армиях, если ты был убит в поединке с равным из-за того, что у тебя сломался меч? Ну и, конечно, вопрос личного престижа, тесно связанного с личной независимостью. Как бы независим ни был человек, его самооценка должна подтверждаться оценкой равных:

\begin{shadequote}[r]{О.~Куваев, <<Территория>>}
Если тебе суждено быть лидером~--- равные выдвинут тебя лидером.
\end{shadequote}

Напрашивается такая аналогия. При советской власти, как мне кажется, максимально возможной личной независимости (или, если угодно, минимальной степени личной зависимости) и от людей, и от обстоятельств можно было достигнуть, работая научным сотрудником академического института, и достигнув строго определенного положения (не ниже, но и ни в коем случае не выше)~--- кандидата наук, старшего научного сотрудника: ниже ты обычно был мальчиком на подхвате, выше~--- на тебя возлагались обязанности, имеющие мало отношения к науке; бывали и исключения, были вариации в разных институтах, но тенденция именно такова. А что было атрибутом этого положения? Два диплома~--- об ученой степени и ученом звании. Так вот, дипломы эти в ВАКе заполнял один из лучших каллиграфов Советского Союза. А помните, с чем была связана задержка с их выдачей? Правильно, каллиграф впадал в запой. Равного ему не было, но на снижение качества не шли. И возмущения широкой научной общественности это не вызывало. 

Как вывод из всего изложенного: наиболее передовые технологии развиваются не в связи с областью приложения (военной, гражданской или какой еще), а там, где они способствуют достижению максимально возможной в данных условиях личной независимости. И современное высокотехнологическое евро-американское общество~--- лучший тому пример. Особенно компьютерные технологии. Это, по-моему, уже трюизм, но, честное слово, я его изобрел сам, независимо. Появление убогого, нелюбимого и ругаемого писюка аналогично патентованию полковником Кольтом механизма проворота барабана при спуске курка~--- и то, и другое уравняло людей в правах. 

Вернее, предоставило им более или менее равные возможности. Чем и дало им возможность добиваться личной независимости. Виртуоз револьвера Дикого Запада~--- такая же реальная фигура второй половины прошлого века, как и кустарь-одиночка с персональным компьютером~--- нынешнего (
\textit{теперь уже~--- прошедшего, в новом тысячелетии фигура эта становится почти анахронизмом}
). и определяющим в поведении для того и другого было стремление к личной независимости. 

И ведь первые добились её. Когда прошло время, выжившие ганфайтеры свою независимость сохранили: их приглашали шерифами и маршалами с высокими окладами и широкими полномочиями. Время вторых тоже проходит. Но тех, кто достиг определенного положения, приглашают на хорошие должности в крупные компании. И они выбирают, какое предложение принять. А что такое свобода, как не возможность выбора, и что такое независимость, как не возможность уйти, не боясь уморить голодом детей? 

Вот и все, что я хотел сказать по поводу первой гипотезы Михаила Ваннаха. 

В заключении~--- несколько разрозненных замечаний. 

Первое~--- относительно широко известного прецедента в Нюрнберге. Ни в коем случае не могу согласиться с его оценкой преподобным Михаилом. Это такой же акт государственного насилия, как и любой другой. Причем насилия одного государства, замаскированного под суд мировой общественности. Вспомните: Дёница, топившего англо-американские транспорты и пассажирские суда, оставили в живых. Советскому Союзу судьба его была в лучшем случае безразлична. Если не сказать больше~--- будущий потенциальный противник был уже определен. А Кейтеля и Йодля повесили. Это были последние, кто точно знал, кто, когда и как начал Вторую Мировую войну. 

Да и судьи кто? Самый страшный преступник из судимых Нюрнбергским трибуналом в палаческом ремесле был не более чем подмастерьем по сравнению с заурядным следователем ГПУ-НКВД\dots 

А на счет приказов\dots Армия, солдаты которой обсуждают приказ, проигрывает сражение до его начала. И это правило всех времен и народов. Другое дело, что погибнуть, сохранив независимость, или сохранить жизнь, приняв государство~--- вопрос личного выбора. Следует только помнить, что ничего нельзя приобрести, ничего не теряя. И отдавать себе отчет: без номинальной защиты государства каждый изрядную часть времени должен посвящать не работе за компьютером, а отработке приемов боя без оружия, стрельбе или фехтованию. Но это тоже каждый выбирает сам. 

И наконец, собственно по основной теме статьи. Не знаю, чью точку зрения это подтверждает, но является историческим фактом. Самый эффективный способ криптографической защиты был применен американцами во время военных действий на Тихом океане во Второй Мировой. Во флот было призвано примерно сорок тысяч индейцев-навахов (за точность цифры не ручаюсь~--- но много). Обучены на радистов. Направлены на каждый из боевых кораблей. Откуда и вели передачи. Открытым текстом. На родном языке. А в этноцентричной Японии не нашлось не только ни одного наваха, но и ни одного профессора навахского языка\dots 

\section{О Linux'е, католиках и гугенотах} 
\begin{timeline}1998.06.22\end{timeline}

\textsl{Эта заметка была сочинена мной в 1998 году и является одним из первых продуктов моего линуксописательства. Размещается здесь ввиду утраты доступа к оригиналу. Статьи, послужившие поводом для этой заметки, увы, также недоступны.}\medskip

Ответ на: \textbf{Алексей Костарев. <<ОС Linux, религия, движение Реформации XV века~--- аналогии>>. Невод, декабрь 1998 г.}

С большим интересом прочел эту заметку (как, впрочем, и большинство материалов НЕВОДа~--- одного из лучших в русскоязычном Интернете Linux-сайта). Поскольку она представляет, в свою очередь, реплику на статью Андрея Шипилова~--- пришлось прочитать и её. 

Относительно исходного материала (А.Шипилов. <<Религиозные войны. 20 век. Линукс на Comdeх.>> Компьютерра, N 48 от 11.12.98) многого не скажешь~--- ничего нового там не обнаружил, кроме обычного для данного журнала иронизирования над всем (чувствуется, что автор в анамнезе, как сказано,~--- писатель-сатирик). Аналогии с мусульманством (<<нет Бога, кроме\dots>>) притянуты за уши и свидетельствуют о незнании предмета. Каковы бы ни были разногласия между мусульманскими течениями, ислам всегда ощущал себя единым целым (Исключение, конечно, исмаилиты, карматы и тому подобные секты, но в своем крайнем выражении они к мусульманству не относятся). Не случайно, скажем, один из известных людоедов XX века Иди Амин, будучи номинальным мусульманином, нашел политическое убежище в Саудовской Аравии~--- гостей-единоверцев мусульмане не выдают (сравните судьбу другого людоеда, Бокассы).

И в теологическом плане мусульмане всегда ощущали свое генетическое родство с иудеями и христианами, называя тех и других, как и себя, людьми Книги, то есть имеющими Священное Писание и чтущие его, в отличие от зороастристов, маздеистов, митраитов, богословная традиция которых (не менее развитаятая, кстати, и оказавшая определяющее влияние на иудаизм в отношении эсхатологии и космогонии) была преимущественно устной (письменная фиксация Авесты ортодоксальных зороастристов~--- явление очень позднее, а, скажем, митраизм стал мировой религией вообще без таковой). 

Другое дело~--- реплика Алексея Костарева. Построенная на методе аналогии, базирующаяся на знании фактического материала (не смотря на скромное заявление автора, что оно~--- <<на уровне общеобразовательного>>), логично написанная, она вполне может послужить <<руководством к действию>>. Поэтому и счел своим долгом написать несколько строк. 

Во первых, будучи не профессионалом-историком, но историком-любителем (профессионал~--- тот, кто, владея соответствующими методиками, например, источниковедческими, археологическими и т.д., извлекает факты и создает концепции, любитель~--- тот, кто эти факты и концепции знает, ибо интересуется), должен отметить некоторые фактические неточности.

Во-первых, дело происходило не в XV, а в XVI веке~--- Виттенбергские тезисы Мартина Лютера были прибиты на двери Замковой церкви этого города в 1517 году.

Во вторых, учение Жана Кальвина~--- ни в коем случае не развитие взглядов Лютера, а совершенно самостоятельная концепция, восходящая в первооснове к Блаженному Августину (тогда как у Лютера можно уловить влияние Пелагия). И вообще, протестантсво, лютеранство, кальвинизм и т.д., как названия конфессий, в значительной мере порождение советской историографии. Последователи Лютера, насколько я знаю, называют свою веру Евангелической, а последователи Кальвина и концептуально связанные с ними течения (гугеноты Франции, пресвитериане Шотландии, утраквисты Чехии и другие) именуют свою церковь Реформированной. И друг с другом себя не смешивают. Известно, что избрание Фридриха Пфальцского (реформата) королем Богемии (т.е. Чехии) вызвало негодование евангелических князей Германии. В последовавшей за этим Тридцатилетней Войне (1618-1648 гг.) ни один из них не поддержал Фридриха. Так как не относился к нему как к единоверцу. 

Англиканство же, <<созданное>> Генрихом VIII Тюдором, имеет к Лютеру и Калькину весьма отдаленное отношение. Догматически и, особенно, ритуально т.~н.~Высокая (государственная) англиканская церковь очень близка к католицизму. И происхождение ее~--- сугубо политическое, аналогично Галликанской церкви Франции, которая, обособившись политически, в религиозном отношении осталась в системе католицизма. Так что, не будь Генрих VIII столь любвеобильным, а тогдашний папа~--- столь бескомпромиссным, ни о каком англиканстве мы бы и не слышали. Потому что Низкая Церковь и разнообразные реформатские течения типа пуританства, пресвитерианства, всякого рода индепендентов и т.д. развивались под влиянием взглядов Кальвина. 

Теперь по существу вопроса. Проводимые автором аналогии между противостоянием католицизма и протестанства в начале Нового Времени и софтверных монополий (в лице Microsoft \textit{etc}.) и сообщества разработчиков открытого софта (не знаю, как правильно можно назвать это очень разнородное множество) очень интересны и, действительно, напрашиваются. Я, во всяком случае, прочитал их с большим удовольствием. Очевидно, что автор статьи, поскольку он пишет на Linux-сайте, является не только пользователем, но и приверженцем этой операционной системы, как и всей концепции открытого программного обеспечения во всех его видах. И поэтому можно не спрашивать, на чьей стороне находятся его симпатии в этом противостоянии. Поэтому в своей исторической аналогии он эту свою симпатию переносит на оппонентов католицизма, рассматривая их, насколько я понял, в качестве носителей всякого рода добродетелей. Тогда как католиков~--- как аналог Империи Зла. А вот это уже~--- вопрос очень неоднозначный. 

Нельзя отрицать роль протестантизма (в широком смысле слова) в создании того, что принято называть современной буржуазной демократией, либерализмом и прочими словами, употребляемыми в хвалебном или ругательном смысле. В общем, современного общества европейского типа, во многом основанного на святости частной собственности. И определяемой этим независимости отдельного индивидуума. Однако заслуга ли это протестантизма? Скорее~--- того индоевропейского варварского субстрата, уцелевшего в Западной Европе, как в заповеднике (поскольку Западная Европа~--- не более, чем один из полуостровов Евразии). Именно в варварском понятии о Законе, как том, что повелось от века, о неотчуждаемой земельной собственности (одаль Норвегии, с некоторыми оговорками~--- аллод Франкского государства) лежит исток чувства независимости и личностное начало и средневекового рыцаря, и городского ремесленника, и современного частного предпринимателя. А протестанская идеология просто наложилась на этот субстрат. Там, где его не было или он не сохранился, как на Востоке Европы, реформатство прекрасно обходилось без демократических институтов. Ведь сейчас трудно представить, но в конце XVI- начале XVII веков дворянство Литвы, Чехии, Венгрии, Австрии было почти поголовно реформатским. А назвать эти страны столпами демократии того времени~--- несколько затруднительно. Достаточно вспомнить хотя бы Януша Радзивилла\dots 

И вообще преставление об общинах протестантов как о демократических институтах основано, по моему, только на выборности пресвитеров, существовавших в некоторых его течениях. Но ведь мы тоже выбирали народных депутатов (и не так давно). Правда, называлось это не просто демократией, а советской демократией, основанной на демократическом централизме (кто застал~--- знает, что это такое, а кто не застал~--- тому и знать не надо). К тому же в условиях замкнутости любой общины (а обособленность ранних протестантов от иноверцев~--- первооснова их поведения), да еще внешней угрозы, любое всенародное (в масштабах этой общины) собрание приобретает характер воровской сходки~--- тому в истории мы тьму примеров сыщем. Достаточно вспомнить тех же большевиков. 

К стати, о большевиках. Подсознательно они прекрасно понимали свою психологическую близость к ранним реформатам. Не зря в трудах советских историков постоянно проглядывает симпатия к последним. И при том тем большая, чем радикальнее и, следовательно, нетерпимее, те были~--- пуритане~--- лучше пресвитериан, а индепенденты Кромвеля~--- еще лучше. 

Относительно терпимости протестантов~--- это даже не миф, а прямая ошибка автора. Так, испанцы-католики, не смотря ни на что, пытались искать компромиссов с язычниками: конкиста Кортеса или Писсаро~--- не более чем высутпление горстки авантюристов (кстати, плохо вооруженных, почти без пороха, в проржавевших латах и с иззубренными мечами) на одной из враждующих сторон аборигенов. А то, что в результате они оказались хозяевами огромных стран~--- всего лишь превратность военной судьбы (не таково ли происхождение варяжской династии на Руси?). И после этого индейская аристократия~--- та, на стороне которой оказались испанцы,~--- заняла высокое положение в новых вице-королевствах. Немало родовитых мексиканских (и даже испанских) семей возводили свой род к табасским или тлашкаланским вождям. 

Реформаты~--- квакеры Новой Англии, буры Южной Африки~--- дали обратный пример полного отсутсвия не только возможности, но и желания сосуществования с аборигенами. Следствие~--- судьба восточных алгонкинов и готтентотов, коих ныне можно пересчитать по пальцам. 

По поводу особенных зверств католиков~--- также не вполне корректное замечание. История Религиозных войн во Франции пестрит свидетельствами о сожженных гугенотами монастырях (разумеется, со всеми обитателями), расправах над пленными и т.д. И Варфоломеевская ночь~--- отнюдь не коварный заговор католических фанатиков Екатерины Медичи и герцога Гиза, а стихийное возмущение парижан засильем чужаков-гугенотов (большая часть которых происходила с юга и в то время французами не являлась) и северо-германских наемников. Именно потому резня и приняла массовый характер~--- для решения политической цели достаточно было перебить несколько десятков, в крайнем случае сотен человек~--- Колиньи, Генриха Навварского и их приближенных. Достаточно прочитать Проспера Мериме~--- вступление к <<Хроникам времен Карла IX>>. 

Об испанской инквизиции~--- она была учреждена в конце как раз XV века, при Фердинанде и Изабелле, то есть до возникновения протестантизма как такового. И целью её была борьба с крещеными маврами и иудеями, оставшимися в Испании после завершения Реконкисты: тем и другим был преложен выбор~--- покинуть страну или креститься. И не потому она была создана, что большинство и мавров, и иудеев продолжало тайно исповедовать веру отцов, а потому, что финансовые группировки моранов финансировали предприятия магрибских пиратов~--- врагов испанской короны и союзников Франции во время Итальянских войн конца XV~--- начала XVI веков (тоже о нетерпимости католиков~--- французский король был католиком и носил титул Христианнейшего; другое дело, что политически он располагался в блоке, противостоявшем папе и Испании). Да и масштабы деятельности инквизиции сильно преувеличены вражеской пропагандой (в руках реформатов оказалось то, что послужило в дальнейшем основой свободной прессы, а также полиграфические мощности) и позднейшей (англо-американской и советской) историографией: документально известно, что за все время существования инквизиции (с девяностых годов XV века до взятия французами Мадрида в 1807, если не ошибаюсь, году, то есть за два с лишним века) репрессиям подверглось несколько тысяч (точного числа, к сожалению, не помню) человек: по сравнению с известными нам примерами масштабы просто смешные. 

К чему я все это написал? Не к тому, чтобы укорить автора в ошибках или неточностях. А потому, что проведенная им аналогия очень показательна. И вызывает другую, возможную, аналогию~--- а не повторит ли сообщество сторонников свободного софта судьбу реформатов в плане нетерпимости к инакомыслящим? Ведь, как реформаты взывали к авторитету Ветхого Завета (Израиль, поражающий филистимлян~--- излюбленный образ времен Религиозных войн), так и <<свободософтовцы>> (прошу прощения за неизящество выражения) аппелируют к первозданному UNIX'у. А обращение к корням всегда чревато тем, что каждый, как сказал патер Браун~--- естественно, католик,~--- <<\dotsчитает свою библию.>>\dots А <<\dotsбесполезно читать свою библию и не читать при этом библии других людей>>. И примеры этого можно наблюдать в нашей жизни. Лично мне не доводилось видеть фанатиков Windows или Word'а, а вот фанатиков протестантстких операционок~--- встречать приходилось. 
\textit{Напомню, что это написано 15 лет назад, с тех пор ситуация сильно изменилась.}


Так пусть же столь тонко подмеченная Алексеем Костаревым аналогия послужит предупрежедением протестантам-компьютерщикам~--- дабы не уподобились они генералу Сент-Клэру~--- старому англо-индийскому солдату протестантского толка из цитировавшейся выше <<Сломанной шпаги>>. Читавшие Честертона (тоже, кстати, католика, хотя и неофита) помнят, чем он кончил\dots 

\section{Криптосвобода} 
\begin{timeline}2001, осень\end{timeline}

В этой заметке не будет почти ни слова ни о Linux, ни об Open Source. События 11 сентября (2001 года) заставляют по иному взглянуть на ряд вещей, окраска которых до сего времени казалась вполне определенной. Одна из них~--- проблема криптографии в частной переписке: является ли злом возможность доступа к ней государственных спецслужб, или таковое может быть оправдано ситуацией.

До недавнего времени для всех людей, относящихся к внутренне цивилизованным, ответ был более чем однозначным. Ныне же раздаются голоса, что, может быть, это не так уж и плохо, если поможет изловить террористов? И, может быть, даже мочить их (в том числе и в сортирах).

На технической стороне криптографического дела останавливаться не буду (тем более что не очень-то в ней и разбираюсь). Но предлагаю задержать внимание на двух аспектах проблемы:

\begin{enumerate}
	\item Способствует ли криптография как таковая свободе? 
	\item Какие последствия могут иметь ограничения, накладываемые на криптографические технологии? 
\end{enumerate}

Должен сказать, что рассуждения о криптосредствах как одном из инструментов защиты свободы личности вызывают у меня, к сожалению, только ироничную усмешку. Наивно ожидать, что власти предержащие (кстати, наиболее точный перевод этого выражения на современный русский язык~--- держащиеся за власть) для подавления свободы в Сети (или любой иной свободы) будут ломать коды пользователей.

Нет, при необходимости у них в распоряжении имеются более действенные средства. Например, роты (полки, дивизии~--- сколько потребуется) автоматчиков, ходящих по домам и штык-ножами обрезающих витые пары, оптоволокно, телефонные кабели. А заодно~--- конфисковывающих модемы, сетевые карты, да и сами компьютеры~--- тоже.

За примерами далеко ходить не нужно~--- можно даже не говорить о талибах и 
\textit{офлайнистане}
. Но многие помнят, по крайней мере по рассказам, что одним из первых распоряжений после 22 июня 1941 года было~--- сдать радиоприемники. Дабы вражеской пропаганде не подвергаться. И ведь шли, и сдавали. А на особо несознательных (или дюже продвинутых) известно какое средство воздействия имелось.

Это было давно и неправда, скажете вы мне. Не согласен. История, когда её забывают, мстит тем, что имеет обыкновение повторяться. Да и вообще все понятия свободы и независимости (применительно к личности) могут рухнуть в одночасье. Хотите пример? Ну что же, пожалуйста~--- Германия, 1933-й.

У нас годами создавали представление о Германии как вотчине затянувшегося средневековья. А ведь это страна с вековыми традициями личной и общественной свободы. Германская сельская община~--- марка,~--- и магдебургское (поищите этот город на карте) городское право были \textbf{самыми} демократичными институтами средневековой Европы.

Да и вообще, именно германские (уже в широком смысле) представления~--- о неотчуждаемой собственности, обычном праве и т.д.,~--- легли в основу развития всей западной буржуазной демократии. Вся англо-американская юридическая система базируется на германском праве. Чтобы осознать это, достаточно прочитать подряд Сагу о Ньяле, а потом (в любом порядке) пару романов Гарднера о Перри Мейсоне.

Так что надеяться, что ничего подобного не может произойти в стране, где общеевропейские демократические традиции отсутствовали вовсе, где даже князья нехилых родов (от Гедиминовичей до Рюриковичей) величали себя не иначе как государевыми холопами (и отнюдь не фигурально говоря), было бы вдвойне наивно. Тем более что мы-то знаем~--- было, и не раз.

Это не к тому, что я против защиты приватности, в том числе и посредством криптографических технологий. Более того, я очень даже 
\textbf{за}. Просто нужно отдавать себе отчет, что в действительно критической ситуации не помогут ни PGP, ни файрволлы.

В свое время (на закате советской власти) мне довелось провести несколько полевых сезонов в Северной Корее (она же~--- КНДР, хотя сами они называют свою страну просто: Корея). Так вот, там в каждом гостиничном номере имело место быть подслушивающее устройство. Положение его было строго фиксировано: между портретами Великого Вождя и Дорогого Руководителя (для ясности замечу, что речь идет о 80-х годах прошлого века). Да оно, собственно, и не скрывается.

Глубоко убежден, что 99 и 9 десятых в периоде процентов этих устройств не работало за практической ненадобностью: ведь нужно не только подслушать, но и прослушать. А это, учитывая долю ненормативной лексики в речи советских специалистов любого ранга, задача нетривиальная и вызывающая в памяти историю о загадочности не только русской души, но и анатомии: <<Вася, надень шапку на~\dots, а то уши отморозишь>>.

Так что действие этих устройств~--- сугубо психологическое, призванное выработать у прослушиваемого (чуть было не написал~--- у пользователя) 
\textit{синдром Большого Брата}
. И бороться с такой практикой, на мой взгляд, можно только одним способом: не позволять этому самому синдрому развиться.

Таким образом, плавно переходим ко второму из заявленных к рассмотрению аспектов~--- что же даст законодательное ограничение криптотехнологий? В плане отлова террористов (видимо, на предмет дальнейшего замачивания в сортирах)~--- сказать трудно. Я тоже испытываю по этому поводу изрядные сомнения. В меру своей некомпетентности рискну предположить, что нецеленаправленная (=тотальная) перлюстрация может быть действенной только в том случае, если фильтруются \textbf{абсолютно все} почтовые сообщения. Что а) практически неосуществимо и б) потребует такого роста компетентных, как говорили у нас встарь, органов, количественного и качественного, что система ГПУ-НКВД покажется районным детсадом супротив Артека.

А вот то, что наличие <<черных ходов>> и прочих back doors в системах криптозащиты, вне зависимости, используются они на самом деле или нет (и даже существуют ли вообще~--- достаточно слуха об их наличии) будет способствовать <<синдрому Большого Брата>>~--- ясно любому, заставшему времена <<развит\'{о}го социализма>>. Вспомните историю про майора, которому <<ваша шутка понравилась>>.

Цель любого террора~--- не устранение врагов: как учит, опять-таки, история, их стараются убрать тихо и незаметно. Нет, она~--- в запугивании всех остальных, особенно~--- абсолютно посторонних, людей. И если законодательные ограничения криптотехнологий будут приняты, террористов с Боингов можно только поздравить~--- цели своей они добились на двести процентов.

И, кстати, стоит только начать. А далее ничто не помешает запретить в законодательном порядке и все проявления открытого и свободного софта. На том, например, основании, что его адепты в состоянии разработать криптосистемы, недоступные для т.~н.~<<профи>> из компетентных органов, и распространяемые безо всяких ограничений.

Заодно не нужно надеяться, что это затронет только Америку: как я уже говорил, в любой стране может случиться все, что угодно. А уж в нашей~--- и подавно. Ведь в ней уже происходило все, что угодно, плюс еще кое-что\dots

\section{О судьбе ресурсов Open Source в Сети} 
\begin{timeline}2001\end{timeline}

\textsl{От автора: эта заметка была написана много лет назад для онлайновой Софтерры. Наткнувшись на нее при разборке архивов, я подумал, что и нынче она не совсем потеряла актуальность.}\medskip

За прошедшие несколько лет мне довелось написать не один десяток заметок, связанных с Linux и открытым софтом, большая часть из которых размещалась на моих сайтах, кочевавших, в силу ряда причин, с хоста на хост. И, начиная с некоторого времени, я стал довольно часто получать просьбы о разрешить переразмещение своих материалов. Именно это и заставило меня задуматься о судьбе сетевых ресурсов вообще и ресурсов Open Source в частности. Наконец, собравшись с силами, я и решился сочинить данную заметку, в которой постараюсь максимально абстрагироваться от сугубо личных аспектов. 

Однако сначала нужно остановиться на общей характеристике сетевых ресурсов по обозначенной тематике и их особенностях в условиях постсоветской действительности. То есть ресурсах в пространстве, именуемом Рунетом, Префикс <<Ру>> в данном контексте символизирует лишь языковую принадлежность, но не физическое размещение сайтов или, паче того, не подданство или пятый пункт анкеты их авторов. 

Все Linux-ресурсы (буду для краткости пользоваться этим термином, но не следует забывать, что он подразумевает все, связанное с Open Source во всех его проявлениях), и не только Рунтовские, я разделил бы на три группы. Первая~--- это сайты крупных и долгоживущих проектов. Этой группе свойственна стабильность во времени и (сетевом) пространстве: история многих из них исчисляется уже десятилетиями.

Вторая группа~--- сайты более или менее крупных разработчиков дистрибутивов и иных, иногда даже коммерческих, программных комплексов. Их стабильность тесно коррелирует с положением создателей, в качестве коих могут выступать фирмы, отдельные лица или их объединения. Если одни сайты (Red Hat, скажем, или Suse) существуют и развиваются уже скоро десять лет, то другие~--- исчезают в одночасье или годами влачат вялотекущее существование. 

Наконец, третья группа, самая многочисленная~--- сайты Linux-энтузиастов, каковыми являются не только разработчики софта, разного рода группы пользователей, но и все более крепнущая группировка Linux-описателей и обозревателей новостей. Именно о них-то в основном и пойдет речь дальше. 

Сразу оговорюсь, что я не преследую дать какую-либо оценку рассматриваемым ресурсам с точки зрения объема контента, его полезности и т.д. В качестве примеров мной привлекаются сайты, которые в силу тех или иных причин я посещаю регулярно и за развитием которых слежу. Это отнюдь не значит, что прочие сайты~--- плохи или не заслуживаю упоминания: просто нельзя ведь объять необъятное\dots 

Так вот, отличительной особенностью таких ресурсов являются условия их хостинга. Насколько я могу судить по имеющимся (обычно косвенным) данным, они размещаются а) на серверах фирм или госучреждений (любого профиля, но, в общем случае, с Linux'ом никак не связанных), где работают их создатели, и б) на всякого рода серверах, предоставляющих услуги бесплатного хостинга. 

В первом случае размещение сайта может осуществляться с официального (или молчаливого) разрешения начальства. Или~--- просто явочным порядком, пользуясь его, начальства, некомпетентностью (или обычным пофигизмом). Случаи активной поддержки таких сайтов руководством, хотя и имеют место быть, вряд ли типичны и многочисленны. 

Как следствие, стабильность таких <<квази-фирменных>> сайтов может быть далекой от идеала. Что особенно характерно для госбюджетных контор: выгорание хаба или развал винчестера могут вызвать их паралич на неопределенный срок (это я по собственному опыту говорю). А всякого рода внешние обстоятельства могут привести к исчезновению ресурса. Примером чему~--- судьба замечательного (некогда лучшего в своей группе) html-редактора WebMaker (автор~--- Алексей Дець), сайт которого молчит уже который год (что отнимает шансы на его реанимацию). 

С сайтами, пользующимися бесплатным хостингом, другая сложность. В свое время я посвятил немало времени изучению проблемы бесплатных хостов, в результате чего вывел нехитрую закономерность их жизненного цикла. Сначала, сразу после создания, такой сервер функционирует просто замечательно, налагая лишь минимальные обязанности рекламного характера типа показа баннеров (а то и вовсе обходясь без таковых). 

Такая лафа имеет естественным следствием наплыв пользователей. По достижении некоей критической массы которых качество сервиса столь же естественным образом падает. Далее возможно несколько сценариев развития: 

\begin{enumerate}
	\item хроническая агония бесплатного хоста, вплоть до летального исхода; 
	\item ужесточение рекламных обязательств пользователей (к баннерам добавляются принудительные pop-up'ы и прочие прелести) на предмет отсечения халявщиков (или максимального затруднения жизни для тех, кому на халяву и уксус сладок); 
	\item беззастенчивое урезание пользователей бесплатного хостинга под любыми предлогами (типа~--- необновления в течении такого-то срока), в том числе и просто выдуманными; 
	\item наращивание его ресурсов (пропускной способности каналов, дисковых объемов и прочего) с одновременными мерами по компенсации понесенных затрат. 
\end{enumerate}

Компенсация эта может выражаться как теми же рекламными акциями, так и просто переходом к платному хостингу. Стоимость которого может быть вполне символической (10-50~у.~е.) в масштабах вечности, но более чем реальной для отечественных Linux-энтузиастов (большинство из которых излишними доходами не отягощены). Да и, по моим наблюдениям, <<малоплатный>> хостинг, в большинстве случаев, отличается от чисто бесплатного тем, что предоставляет услуги того же качества и количества, что и последний, но~--- уже за некоторые деньги. 

Я не хочу сказать худого про действия хостеров. Во-первых, <<халява, сэр>>~--- она халява и есть, грех на нее обижаться. Во-вторых, даже не лучший по качеству бесплатный хостинг все же лучше, чем никакого: вспомним, сколько интересных проектов выросло из бесплатной странички на \url{geosites.com} и подобных серверах. Перефразируя известное высказывание, можно сказать, что бесплатный хостинг~--- как секс: если он хорош, это здорово, если же плох, то это все же лучше, чем\dots эээ\dots~--- <<сам с собою одною рукою>>. И наконец, в третьих: существует некоторая категория бесплатных хостеров (хотя их скорее следует отнести к системам поддержки проектов), на удивление ответственно относящихся к добровольно принятым на себя обязательствам (и тут я не могу не вспомнить добрым словом Чертовы Кулички~--- недавно я там с удивлением обнаружил свою страницу давностью в несколько лет; воистину~--- <<хранить вечно>>). 

Однако факт остается фактом: в рассматриваемом контексте следствием является частая миграция Linux-ресурсов, а то и просто их, казалось бы, бесследное исчезновение: подсчитайте количество <<мертвых>> ссылок в любой их коллекции. Например, на \url{rus-linux.net}, почти исчерпывающей подборке русскоязычных Linux-ресурсов. И сколько интересных материалов скрылось с горизонта за последние три-четыре года. 

Все это было бы очень грустно, если бы не феномен эпохи Интернета~--- фактор переразмещения. На котором я хотел бы остановиться подробнее. 

Как было сказано в преамбуле, я довольно часто получаю письма с просьбой о таковом. И еще чаще~--- письма вроде: <<прочитал твою статью на \texttt{http://www.khren-znaet-gde.ru} и хочу сказать\dots>>. С одной стороны, доставляет мне известные неудобства: после многоточия обычно следует сообщение о допущенной мной там или здесь ошибке. 

Поймите меня правильно: я ни в коем случае не против таких писем . более того, очень даже <<за>>. Ведь именно таким образом я узнаю массу нового и для меня интересного. Однако ошибки, на которые в них указывается, с большой долей вероятности, исправлены в оригинальном материале. А иногда таковой и просто снят за утратой актуальности\dots 

С другой стороны, мне весьма лестно, что мои писания переразмещаются на многих сайтах. Это вселяет надежду~--- что бы ни случилось со мной или моим сайтом, они будут существовать, пока к ним сохраняется интерес. Хотя, должен заметить, что последнее время я по умолчанию разрешения на переразмещение не даю. По причинам, упомянутым в предыдущем абзаце, и еще некоторым, важнейшую из которых я назвал бы <<политкорректностью>>.

На вопросе <<политкорректности>> хотелось бы остановиться отдельно. Он включает несколько аспектов. Первое~--- ссылки на первоисточник, от которых не освобождает никакие представления о свободе и свободные лицензии. Почему-то во многих случаях считается достаточным привести e-mail автора~--- хотя, повторяю, автору это ничего не дает, кроме лишней головной боли (в виде писем <<как бы по делу>>, не говоря уже об откровенном спаме. 

И даже ссылка на конкретный материал с авторского сайта~--- не вполне корректный способ цитирования. Ибо настоящий авторский сайт~--- это цельное произведение, подобное книге. А ведь никому не приходит в голову ссылаться на главу из книги, а не на книгу как на целостность\dots 

И главное: переразместитель, по моему глубокому убеждению, должен не только пользоваться плодами чужого труда, но и принимать на себя некоторые обязанности. Из которых важнейшая~--- отслеживание изменений в оригинале переразмещенной статьи. Ведь сетевые материалы по определению не есть нечто неизменное\dots 

В общем, если культура цитирования бумажных изданий, складываясь веками, к настоящему времени устоялась (по крайней мере, среди образованной части населения, то культура онлайнового цитирования находится в зародыше. Хотя роль её в настоящее время трудно переоценить. 

Ведь именно практика свободного переразмещения в Сети спасла от забвения множество вещей, забвения отнюдь не заслуживающего. Для примера вспомним авторскую песню в своем классическом выражении, казалось бы, прочно забытую к середине 80-х. И вдруг волшебным, как из небытия, образом возникшую снова во второй половине 90-х. 

И в лице не только своих всенародно известных представителей. Но и тех, кто и в старые-то времена пользовались очень широкой известностью лишь в очень узких кругах. Ну кто, скажите на милость, лет десять-пятнадцать назад помнил, что Михаил Анчаров был автором всенародно известной <<Баллады о воздушном десанте>>, Владимир Ланцберг~--- культовых, как говорят нынче, для моего круга и поколения <<Пора в дорогу, старина>> и <<Мы условимся, трупов не будет>>. А уж <<Юнкера>>, <<Кони вороные>>, <<И падал я в душные травы\dots>> Бориса Алмазова, казалось, вообще канули в Реку Забвения\dots 

Впрочем, это~--- совсем отдельная тема. Как-нибудь, выкроив время и собравшись с мужеством, я сочиню статью о внутреннем родстве между явлениями советской авторской песни и движением Open Source\dots 

Можно сказать, что <<попавшее в Сеть . не вырубишь топором>>. Однако не нужно иллюзий на тему <<рукописи не горят>>. Мы-то знаем, что как раз они и горят, да еще как. Правда, и попавшее в Сеть вырубить можно~--- не топором, но рубильником в мировом масштабе\dots Тем не менее, Сетевые материалы я сравнил бы скорее не с печатной, но с устной традицией. 

Каковая, при всей своей кажущейся хрупкости, гораздо устойчивее того самого, <<написанного пером>>. Не зря же в седой и бронзовой индоевропейской древности только устная традиция считалась священной. Так было и у кельтов, и у иранцев, и у индоариев. Хотите примеры? Их есть у меня\dots 

О скольких авторах античной древности, начиная с Гекатея Милетского, зафиксированных письменно, знаем мы только то, что они были авторами чего-то (в лучшем случае . авторами чего именно они были)? А ирландские эпические сюжеты, именуемые в русских переводах ирландскими~--- не путать с исландскими~--- сагами (по ирландски они назывались просто scela, то бишь повесть), зафиксированные письменной традиции не раньше X в. н. э. (от Р. Х., как стало модно говорить в последнее время), восходят часто к рубежу эр (а то и к III-II вв. до н.э.. И все это время сохранялись исключительно в традиции устной. То есть устная традиция существует вплоть до поголовного физического истребления её носителей. Или пока они не потеряют к своей традиции интереса окончательно~--- такое тоже подчас случается. 

Итак, какова же судьба сетевых Linux-ресурсов? Да будут они существовать вечно, не на сайте автора . так на ином, где к ним был проявлен интерес. И появляться, после своего кажущегося исчезновения, снова и снова, в самых неожиданных местах. И так будет, пока тема материала сохраняет интерес для сообщества Open Source (хотя бы и исторический), пока существует само явление Open Source и пока существует Сеть как наследница устной традиции наших предков.

\section{О причинах, по коим геокомпьютинг должен стать основным полем приложения сил сообщества Open Source} 
\begin{timeline}2000 г, апрель\end{timeline}

\textsl{Первая версия этой заметки была сочинена в ночь с 1 апреля на день геолога 2000~г. и размещена на моем старом сайте на Чертовых Куличках. Давеча, разбирая свои архивы, я перечитал это сочинение~--- и вдруг понял, что оно отнюдь не потеряло актуальности. Поскольку намеченная в нем тенденция не только не ослабла, но и продолжает прогрессировать. Впрочем, судите сами. Ссылки, по причине лени, не актуализированы.}

\subsection{Что же такое геокомпьютинг}

Геокомпьютинг, с моей точки зрения,~--- это все, имеющее отношение в раной мере к геологии и компьютером. Утвердившийся термин <<геоинформатика>> мне не нравится, так как плотно ассоциирован с геоинформационными системами. Коими отнюдь не исчерпывается применение компьютеров в науках о Земле. Более того, именно здесь геоинформационные системы наименее применимы. По крайней мере, примеров успешной реализации ГИС-проектов в геологии пока немного. Почему? Причины, думается, в следующем.

Во первых, геология - наука кривая в прямом и переносном смысле (у верблюда спросили: почему у тебя шея кривая? - а что у меня прямое? - ответил тот). Так вот, все геометрическое в геологии - криво. То есть: стратирафические границы, осевые поверхности складок, линии надвигов; и даже трансформные разломы только аппроксимируются более или менее прямыми линиями. ГИС же эффективно работают при условии прямизны границ. Не зря же самые удачные примеры их использования относятся к лесопосадкам и избирательным участкам\dots

Вторая причина - геология по сути своей наука индуктивная. А индукция - это то, что великий Шерлок Холмс называл дедукцией: не иначе как его создатель Артур Конан Дойл, будучи студентом-медиком, не отличался прилежанием в области классической логики. И всякое геологическое действо (а геокартирование - особенно) происходит от частного к общему. Любой съемщик понимает, что для того, чтобы нарисовать карту в двухсоттысячном масштабе, площадь надо отходить как минимум со стотысячной детальностью. В основе же ГИС - принцип прямо противоположный: детализация генерализованного изображения.

Ну и в третьих, ГИС - это инструмент не столько исследования, сколько представления уже, так сказать, на-исследованного. Не умаляя важности этого, скажу: если я знаю, что нужно, представить, то уж как - соображу по возможности. Но проблема-то обычно упирается именно в незнание того, что\dots

Так что, не умаляя роли ГИС, скажу, что как инструмент исследования в науках о Земле более подходящим представляется то, что именуют задумчиво image processor. Применительно к геологии, это, в первую очередь, цифровая картография во всех её проявлениях. Поскольку геологическая картография~--- это основа любых построений в области наук о Земле. А потому возникает вопрос, существует ли для открытых и свободных платформ

\subsection{Приложения для геокомпьютинга}

Как оказывается, существует. В первую здесь следует упомянуть GMT - The Generic Mapping Tools. Это пакет программ, созданный профессорами-геофизиками Гавайского университета Паулем Весселем (Paul Wessel) и Уолтером Смитом (Walter H. F. Smith) еще в 80-х годах.

Эта программа распространяется бесплатно в исходных текстах (в полном виде около 50 Мбайт), компилируемых для любой Unix- или Unix-подобной системы. В настоящее время её можно обнаружить в системе портов FreeBSD и в портежах Gentoo. Она включает более 50 отдельных модулей для обработки двухмерных и трёхмерных картографических данных, построения на этой основе контурных карт, shadow map и истиннно трёхмерных блок-диаграмм. Которые могут быть записаны в формате EPS.

Помимо собственно пакета, на сайте авторов доступны многочисленные примеры его применения (главным образом для геологических объектов Гавайских островов), а также очень подробная документация в форматах PDF и PS.

Все модули GMT работают исключительно в режиме командной строки. Однако для этого пакета разработана и интерактивная оболочка iGMT, написанная на TclTk и работающая в графическом режиме. Её можно обнаружить на сайте Сейсмологического факультета Гарвардского университета.

Следующий пакет, заслуживающий упоминания~--- это GRASS, представляющий собой нечто среднее между ГИС и имидж-процессором. Он предназначается как для построения векторных карт, так и для обработки растровых изображений~--- космо- и аэрофотоматериалов, результатов спектрозональных съемок и тому подобного. Распространяется бесплатно, как в исходных текстах, так и в виде бинарных пакетов. Имеются также дополнительные базы картографических данных и примеры применения.

Кроме того, на ряде сайтов американских университетов и правительственных служб можно обнаружить упоминания о таких системах для геокомпьютинга, как SPRING, Xmap8 (нынешняя версия которого носит название Geotoush) и еще нескольких, теоретически заявленные как свободные и доступные для бесплатного скачивания. Практически, однако, скачиванию предшествует длительная и сложная процедура онлайновой регистрации. Которая подчас заканчивается предложением подождать письма с идентификатором и паролем для доступа на ftp-сервер. Возможно, это мое личное везение, но ожидание это оказывается, как правило, тщетным:-)

Таким образом, список работоспособного инструментария для геокомпьютинга под Linux сводится к двум позициям. К тому же ни GMT, ни GRASS не удовлетворяют в полной мере требованиям к таковому: первая~--- как чрезвычайно сложная в использовании, вторая~--- как имеющая ограниченные возможности работы с форматом DEM (Digital Elevation Modelling), основным для анализа геологического строения в региональном масштабе. И потому~--- оправдана ли постановка вопроса, вынесенного в заголовок статьи?

Правда, а есть ли к тому причины? Думаю, есть, и даже две. Первая - и, скажу честно, главная, - мне бы очень этого хотелось, потому что это то, чем я пытаюсь заниматься и что мне интересно.

Вторая же - более общего характера, и именно к ней я хотел бы привлечь внимание сообщества Open Sources, в первую очередь, конечно, нашего, российского.

\subsection{Технологическая перспектива}

Не секрет, что ныне - один из редких периодов в истории информационных технологий, когда программная индустрия не в состоянии выполнить своего сакрального предназначения. Каковым является (помимо, конечно, обеспечения работой нас, писателей в жанре технологической новеллы) выколачивание у пользователей денег на развитие аппаратных средств. Путем создания все более ресурсопожирающих программных комплексов~--- без этого остановится развитие индустрии хардверной.

Действительно, ныне практически с любым программным обеспечением общего назначения (и даже со многими узкопрофессиональными пакетами) с успехом справилась бы система на базе Celeron-400 с 64 мегабайтами памяти, любой (!) реально доступной видеокартой и самым маленьким, какой только удастся достать по блату, жестким диском. И даже (страшно сказать) большинство игр (за исключением уж самых навороченных) можно запустить на любой машине, каковую можно обнаружить в прайсах двух-трехлетней давности.

Но ведь развитие компьютерного железа не стоит на месте: частоты массовых (то есть недорогих) процессоров (и от Intel, и от AMD) давно перевалили за 2 гигагерца, память своеобычным ныне объемом обойдется в сущие копейки, а о жестких дисках и говорить не приходится: за разумные деньги можно приобрести их только такого объема, который не по силам заполнить даже при тотальном скачивании mpeg-музыки (и потому, дабы свято место пусто не осталось, приходится качать фильмы, как правило, посредственные:-)).

В перспективе же~--- тактовые частоты <<камней>> от Intel переваливают за 4 Ghz (и его эквивалент от AMD), 64-битные вычисления, память DDR2 с фантастической (и совершенно незадействованной) пропускной способностью\dots Как же убедить пользователя в необходимости приобретения этого богачества?

Появление все новых и новых версий Windows с их нарастающими хардверными аппетитами положения уже не спасает: даже адепты её признают, что в корпоративной среде <<нет никакой необходимости устанавливать Windows~2000 (\textit{не говоря уже о XP и тем более Longhorn}~--- А.~Ф.) на каждую рабочую станцию и каждый сервер сети>> (Шен Дэйли. 10 шагов для перехода к Windows~2000. Windows~2000 Magazine/RE, \textnumero1(4), 2000, с.~48). А значит~--- нет и необходимости в upgrade каждой машины. А это - бандитизм и бесчинство со стороны пользователя, не так ли?

Надежды, возлагаемые ранее на бурное развитие мультимедиа-технологий, похоже, себя тоже не оправдывают. И звук, и видео вполне успешно прокручиваются на современной машине средне-офисного ранга, а массового внедрения трёхмерной графики ожидать уже не приходится. В частности, по чисто психологическим причинам: на протяжении всей истории человек стремился аппроксимировать три измерения реального мира его двухмерными представлениями. Почему, в частности, я не думаю, что 3D-интерфейсы обречены на успех: первый такой интерфейс я увидел году в 92-м (3D-Room от Hewlett-Packard, если мне не изменяет память), и очереди пользователей за ними с тех пор не наблюдается.

То же касается и игровых приложений. Круг фанатичных геймеров, видимо, замкнулся, количественного роста его ожидать не приходится. А возможности аппаратуры ныне перекрывают с лихвой потребности самых крутых игр. Не говоря уж о том, что гейминг окучивает в первую очередь видеосистемы, но никак не комплекс аппаратных средств современной персоналки.

Так спрашивается, куда же девать лишние мегагерцы и мегабайты (а то и гигагерцы с гигабайтами)? Ответствую, аки отрок Феодосий: в технологии геокомпьютинга!

Обоснуй! - резонно скажете вы мне на это.

Что ж, попробую. Для начала~--- первая к тому посылка,

\subsection{Ресурсы}

Обработка картографической и аэро-космофотографической информации - в числе немногих действительно (а не искуственно, как текстовые процессоры) ресурсоемких задач, с которыми сталкивалась компьютерная индустрия за все время своего существования. Рядом (и даже впереди) я поставил бы (если не считать профессиональной индустрии развлечений) только метеорологию и всякого рода аэрокосмические приложения. Однако трудно представить себе, что рынок метеорологов-любителей или слесарей-надомников с персональным <<Бураном>> станет когда-либо массовым, и каждому потребуется суперкомпьютер с трехзначным числом процессоров. А вот рынок геосистем - может, о чем речь пойдет чуть ниже, в посылке второй.

Так вот, решение задач геокомпьютинга посредством ГИС и имидж-процессоров требует изрядной вычислительной мощности, оперативной памяти и дискового пространства (во времена оны, когда я активно этим занимался, у меня за неделю работы иногда набиралось по несколько гигабайт данных). А для визуализации результатов, особенно трёхмерной, весьма желательна могучая видеосистема, включающая и качественный (а значит - дорогой!) монитор. До недавнего времени эти задачи были вообще недоступны для настольных персоналок и решались (да и решаются) на рабочих станциях стоимостью в десятки тысяч условных единиц.

Ныне, начиная со времени появления процессоров класса \mbox{P-III/Athlon} (не говоря уже о Pentium-4 и AMD64), положение немного изменилось. И мощные ГИС и имидж-процессоры вполне могут функционировать на PC среднедомашнего (то есть игрового!) уровня. Однако и геософт ведь не стоит на месте: взамен аскетической командной строки GMT приходят развитые графические интерфейсы. Да и аппетиты растут: просто построения карты кажется недостаточным, хочется и 3D-моделей, и реалистичной генерации ландшафтов, и виртуальных облетов территории\dots

И ведь ГИС, хотя они и <<ГЕО>>, к Земле отнюдь не привязаны: есть материалы и по Марсу, и по Венере, и по старушке Луне, наконец. А там - свои проблемы обработки, требующие своих решений. И, соответственно, ресурсов.

Промежуточный вывод: геокомпьютерные технологии на сегодняшний день могут утилизировать вычислительные ресурсы почти любого масштаба. И почти во всем спектре производимых аппаратных средств. Вот только кому это нужно, кроме фанатиков от геологии? На сей предмет у меня заготовлена вторая посылка~---

\subsection{Перспективы массовости}

На все сказанное в предыдущем пункте можно возразить: это удел узких профессионалов, каких - единицы на всю нашу планету, народу это не нужно (поскольку то, что нужно народу - не это).

Сейчас это действительно так. Однако: дело идет к тому, что скоро GPS'ками высокой прецизионности будут оснащаться не только транспортные средства специального назначения, не только каждый катер и прогулочная яхта, но и автомобиль, мобильный телефон и прочие носимые устройства. Не говоря уже о доступности просто карманных систем позиционирования. Собственно говоря, процесс уже пошел: мой друг, выезжая на дачу на своем мотороллере, без GPS'ки не обходится.

И сейчас легко представить себе геолога (или представителя любой другой относительно массовой полевой профессии), привязывающего точку наблюдения не по трем засечкам компаса (или, паче того, по лаптям правее солнца), а по GPS тех же габаритов и веса; но - немыслимой при привязке в лаптях точности. У них это давно уже не мечты, да и у нас становится реальностью.

Точная йифровая привязка полевых наблюдений сразу вызовет астрономический рост объема данных. А цифровые данные, не подвергнутые количественной обработке, все равно что не существуют. И обработка эта требует соответствующих аппаратных ресурсов.

Но это - не все. Неизбежен рост прецизионности позиционирующей аппаратуры - ведь законы конкуренции действенны не только в компьютерной индустрии. Это приведет к тому, что морально устареют топографические карты, выполненные посредством мензулы и кипрегеля (и даже фотограмметрии аэроснимков).

Новые средства позиционирования потребуют разработки картографических основ принципиально нового вида и наполнения. Каких - даже не могу себе представить. Но что разработка их задействует достаточно вычислительных ресурсов - ясно. А учитывая потребность экспедиционщиков в количестве таких материалов - производство должно быть массовым. Я уж не говорю о таких мирных пользователях картографической продукции, как вооруженные силы: читавшие Виктора Суворова представляют себе, сколько экземпляров карт требуется для обеспечения боеспособности артиллерийского полка. Да и любого другого - тоже.

Дальше - больше. Ведь помимо профессиональных экспедиционщиков, существуют, так сказать, экспедиционщики-любители. Именуемые обычно туристами. Они - разные: от спортсменов-первопроходимцев маршрутов до отпускников-автомобилистов, прокладывающих маршрут из пункта Б1 в пункт Б2 (>>как известно, в Петушках нет ни пунктов А, ни пунктов Ц, а есть только пункты Б>>). И всем им потребуются карты нового поколения, программные средства для работы с ними и с данными прецизионного позиционирования, мощные компьютеры для запуска этих программ.

А все это вместе будет требовать вычислительных ресурсов, вычислительных ресурсов и вычислительных ресурсов (для подготовки карт, подготовки программного обеспечения и для его использования, соответственно).

Промежуточный вывод: развитие средств глобального позиционирования создает предпосылки для массового спроса на технологии геокомпьютинга самого разного уровня: если провести аналогию с системами обработки текстов (с чего и начался массовый спрос на персональные компьютеры), то это будут системы класса QuarkPress и Framemaker, во первых, класса Word - во вторых, и класса Lexicon - в третьих.

\subsection{Общий вывод}

Представляется, что колоссы компьютерной индустрии не могут не осознавать, что ноги у них имеют шанс из кремниевых неожиданно превратиться в глиняные. И потому не искать сферы приложения для своих mips'ов и гигагерцов. А таковой, как я пытался обосновать, в массовом масштабе ныне может быть только геокомпьютинг в широком смысле этого слова.

Потэому можно прогнозировать всплеск инвестиций в развитие ГИС, имидж-процессоров и ассоциирующих с ними продуктов (тех же генераторов ландшафтов, например). Что вызовет разработку новых и активизацию существующих коммерческих продуктов этого направления. Причем - не только узкоспециализированных, глубоко профессиональных систем, но и систем более или менее массового использования.

Поскольку именно предложение, вопреки Марксу, рождает (вернее, по-рождает, при соответствующих усилиях) спрос, таковой и будет порожден: должны же софверщики окупить свои производственные затраты. Что вызовет приток в эту нишу новых средств и новых участников. И - так далее, то есть система станет саморазвивающейся.

\subsection{Следствия для сообщества Open Sources}

Все это я написал не для того, чтобы дать совет хардверным и софтверным фирмам - они и без него обойдутся. Моя цель - обосновать тезис: впервые за всю историю всему миру Open Source дается шанс: не стоять в позиции для парада, отражая рипосты коммерческих производителей. Типа - на удар с кварты PhotoShop'ом мы ответим с терции GIMP'ом, на укол MSOffce - отводом Koffice, и т. д.

Нет, Linux -сообщество может оказаться на острие, что называется, прогресса. Ведь коммерческие ГИС сотоварищи развиваются уже давно. И неизбежно несут на себе (и долго еще будут нести) груз тяжкого наследия. В виде ориентации на профессиональные применения, архаических черт интерфейса, привязки к традиционной картографической базе и прочее, прочее, прочее. А кто будет спорить с фактом, что перестроить поточную линию сложнее, чем штучное производство?

Ныне работоспособных ГИС-подобных систем под Linux - раз, два - и обчелся (почти буквально). И это, товарищи, правильно - сложившиеся стереотипы, готовые наработки и тому подобные тормоза не будут мешать при создании новых продуктов.

А продукты эти, безусловно, будут конкурентоспособны. Даже не обсуждая вопросы качества и функциональности - просто по цене. Современные коммерческие ГИС - огромные программные комплексы с массой опциональных (но обычно - необходимых!) модулей; суммарная цена их (не по прайс-листам, а в реальности) достигает десятков тысяч долларов на рабочее место. Конечно, с выпуском массовых коммерческих продуктов цена падать будет, но - далеко не сразу. Помните, не так давно стандартный текстовый процессор стоил с полштуки американских рублей. Да и сейчас - почти столько же, потому как покупается в составе офисного комплекта.

Даже при худшей функциональности очевидна перспектива бесплатного и открытого софта для геокомпьютинга. А, как я пытался показать, функциональность его в данном случае может быть и выше: нет необходимости копировать чужие решения и обеспечивать совместимость с ними. Как нет и памяти прошлых решений - не всегда удачных и уж точно принятых в других условиях.

При этом я не призываю устроить всепланетную богадельню от Open Sources. Поскольку системы для геокомпьютинга идеально вписываются в модель его распространения. То есть сам софт распространяется бесплатно, деньги же берутся за установку, обучение, адаптацию под задачу; в общем, за то, что называется звучным заграничным словом support или не вполне адекватным нашим - поддержка.

Так вот ГИС, какими бы дружественными к пользователю они ни были, не тот софт, с которым сможет управляться любая кухарка. Просто по самой своей природе он требует некоторых специальных знаний и умений. И здесь поле для support'а - практически неограниченное.

Более того, Земля дана нам в единственном экземпляре, но объектов на ней - уж очень много. И каждый класс объектов потребует своих программных решений. И что сложнее - адаптировать готовую систему с открытыми исходниками под вашу задачу или ждать, пока кто-нибудь напишет коммерческую программу для её решения?

Ну а уж за это сам бог велел денег взять. Как сказал кто-то из великих инженеров прошлого: за то, что переключил контакты~--- с Вас~1 доллар, за то, что знаю, как это сделать~--- \$999\dots

\section{Потому и не любят} 
\begin{timeline}2002, весна\end{timeline}

\textsl{Эта заметка сочинялась, как своего рода открытое письмо госпоже SKV в ответ на её статью <<Почему они нас не любят или записки тетки-бухгалтера>>, опубликованную в онлайновой <<Компьютерре>>, если не изменяет память, весной 2002 года (ныне недоступна). Сочинялась по горячим следам, и так нигде не была опубликована. Давеча, копаясь в архивах, я на нее наткнулся - и подумал, что некоторую актуальность она сохраняет. Хотя многие реалии изменились~--- в частности, открытые и свободные программы разрабатывают нынче далеко не только из любви к искусству.}\medskip

Поскольку статья, послужившая предметом дальнейшего разговора, ныне недоступна, в двух словах перескажу её сюжет. Он сводится к жалобам на то, что компьютерщики~--- очень нехорошие люди, которые очень не любят бухгалтеров и не желают облегчить их нелёгкий труд разработкой специальных бухгалтерских программ под Linux. Поскольку соответствующие программы под Windows~--- платны и весьма дороги. Что и подвигла госпожу SKV ознакомиться с положением дел на альтернативной платформе.

Должен заметить, что автор этой заметки при своей попытке знакомства с Linux'ом действовала очень разумно и логично. Рассудив, что если первоисточник информации о Linux'е вообще~---  сайт \url{linux.org}, то первоисточником сведений о Linux'е русском должен быть его аналог в зоне ru. На форум которого она в поисках истины и отправилась. Продолжать, вероятно, не надо~--- там и у закаленного мужика уши подчас завянут. Остается только пожалеть, что она не пришла на один из более иных ресурсов, где джентльмены простили бы ей даже и <<социальный статус человека успешного и состоятельного>>.

А в целом обсуждаемая заметка показалась мне весьма интересной~--- как своего рода социологический показатель. Тем не менее, я не стал бы писать настоящие строки, если бы не пара-тройка толстых в ней намеков, показавшихся мне~--- ну, не то чтобы оскорбительными, но вполне дающими повод к ответу в том же духе.

Для начала хотелось бы рассмотреть вопрос из заголовка обсуждаемой статьи~--- <<Почему они нас не любят>>. Где под ОНИ подразумеваются, по контексту, трудящиеся информационной сферы вообще, адепты Open Source в частности и линуксоиды в особенности. А под словом НАС - работники финансовой сферы. Далее по тексту я для обозначения первых и вторах, соответственно, буду использовать идеограммы ОНИ и ВЫ, в требуемых грамматических формах. Ответить на этот вопрос не сложно, однако~--- не в одну фразу, а как минимум в две.

Первая~--- проста, это вопрос на вопрос: <<А кто ВАС любит?>> Упаси Боже, здесь нет ничего личного - местоимение, повторяю, относится не к Вам, а к ВАШЕЙ, сударыня, профессиональной касте.

Действительно, ВАС не любят все. Не любят руководители любых предприятий любого ранга~--- за требования того, что при соввласти называлось финансовой дисциплиной. То есть, по простому,~--- соблюдение внешних приличий при окучивании госбюджетных (или каких-либо иных) средств. Впрочем, как показывает практика, на этом уровне взаимопонимание легко достижимо~--- за исключением случаев клинической жадности.

Не любят ВАС и начальники подразделений (также вне зависимости от масштаба)~--- потому что, по искреннему убеждению бухгалтерии, вся их деятельность должна иметь целью только соблюдение финансовой отчетности.

И уж, разумеется, ВАС не любят рядовые сотрудники-исполнители~---  например, за маленькую и несвоевременно выплачиваемую зарплату. Объясняемую обычно финансовыми трудностями предприятия вследствие внешних условий. К слову сказать, сотрудников финслужб эти внешние условия почему-то обычно не затрагивают\dots

Честно говоря, я тоже не любил бы ВАС, ибо большую часть своей трудовой карьеры провел во втором качестве. Если бы не встреча на заре юности с неким бухгалтером, который умел внятно объяснить начинающему начальнику отряда суть требований финансовой отчетности. А главное~--- как и где их можно нарушать в интересах дела (а никакое дело при соблюдении всех финансовых требований делаться просто не могло, и это знали все), чтобы тебя не взяли за~\dots~сами знаете что. Чтобы понять меру его гражданского мужества, добавлю~--- было это более тридцати лет назад, когда нарушение финансовой дисциплины легкой словесной эквилибристикой могло быть приравнено к хищению социалистической собственности. Со всеми вытекающими последствиями\dots

Следствие сказанного выше~--- ИМ не за что любить ВАС. Поскольку все ОНИ с неизбежность попадают в одну из перечисленных категорий. И это, скорее всего, для Вас, сударыня, не секрет. Однако, судя по вынесенной в заголовок фразе, именно ОНИ, в отличие от иных категорий трудящихся, не любят ВАС каким-то особенным, нетрадиционным, способом.

Вероятно, это действительно так. И причину Вы интуитивно поняли сами. Она - в том, что ВЫ совершенно искренне убеждены в том, что ОНИ ВАМ что-то должны. Цитирую: 


\begin{shadequote}{}
Сделайте кто-нибудь консервативный офисный дистрибутив с тремя опциями\dots, напишите к нему хорошую документацию\dots
\end{shadequote}
и при этом
\begin{shadequote}{}
ознакомьте меня с ним.
\end{shadequote}

И все это, судя по контексту~--- на халяву, то есть по себестоимости носителя и доставки - в явном виде это не сказано, но подразумевается из жалоб на дороговизну легального проприетарного ПО для трудящихся бухгалтерий всех стран. А уж 


\begin{shadequote}{}
\dotsесли я посчитаю достойным
\end{shadequote}

… может быть и соблагоизволю купить (по той же цене).

Да неужели ВЫ всерьез думаете, что это не лениво? Ведь те, кто клепает открытые и свободные программы, чем бы они свою деятельность ни мотивировали на словах, на деле занимаются ею (в очень существенной мере) из любви к искусству. И, по большому счету, им глубоко безразлично, что ВЫ думаете по этому поводу. ВАМ они ничем не обязаны и не обязаны ничего. Им сугубо параллельны ВАШИ заботы о том, чтобы девушки-операционистки не имели в рабочее время возможности раскладывать Солитера или резаться в~\dots~ну, во что нынче там в офисах режутся? Как, впрочем, и ВАМ сугубо до лампочки, 


\begin{shadequote}{}
как конструктор рисует свои чертежи, а дизайнер - свои Гауссовые кривые.
\end{shadequote}

Это нужно ВАМ~--- чтобы ВАШИ девушки-операционистки при деле были. Так и флаг ВАМ в руки~--- найдите на просторах Рунета очередного молодого и в меру голодного гения, и он за умеренную мзду склепает ВАМ систему учета именно для ВАШЕГО предприятия. И даже две~--- одну вместо 1С, для фискальных органов, и другую~--- всамделишнюю, для гендиректора. А заодно уж и третью~--- лично для ВАС, на предмет дальнейшего ВАШЕГО успеха и состоятельности.

А теперь попробуем отделить зёрна от плевел~--- ибо если бы первых не было, не стоило и писать ответ. Я всегда скептически относился к идее офисного использования открытого софта. Однако при чтении рецензируемой заметки подумалось, что и здесь для него есть сфера применения~--- та, где человек выступает в качестве придатка к калькулятору (закаленного и отточенного орудия, как сказал герой известной новеллы Альфреда Бестера).

Конечно, сие есть профанация идеи Open Sources (основа которой - в том, что любой исполнитель \textbf{понимает}, что делает). Однако я не настолько идеалист, чтобы не знать~--- есть области деятельности, где шаг влево/вправо равнозначен побегу, а прыжок на месте~--- провокации. И именно там~--- место предельно \textit{закрытым}, то есть изолированным от пользователя, решениям, основанным на \textit{открытых} системах. Нечто подобное описал в свое время Владимир Попов в статье, которая увидела свет на страницах <<Системного администратора>>. 

Только ВЫ не должны думать, что кто-нибудь сделает это для ВАС, в расчете на ВАШЕ благоизволение~--- спасение утопающих, как известно, есть дело рук самих утопающих. Да еще и не рассчитывайте, что при этом обойдетесь без квалифицированного сисадмина (в дополнение к тем, которые <<честно говоря, не очень-то в курсе>>).

Разумеется, ВАС никто не неволит~--- сообществу Open Source как-то без разницы, выбросите ВЫ самортизированную 1С или купите свежую её версию (вкупе с Windows'ами, Microsoft Office'ами и прочими прибамбасами). Но иначе уж будьте добры мириться с играющими в Тетрис операционистками под ОС, с коей <<время, надежность и безопасность>> суть вещи несовместные.

Не извиняюсь, если <<получилось\dots не по доброму как-то, зло>>~--- материал к добру не располагает.

И ещё о зёрнах: в здравости самой высказанной автором идеи~--- о закрытой пользователю монофункциональной системе для выполнения нескольких предопределенных действий,~--- я за прошедшее время убеждаюсь все больше и больше. Время от времени с форумов доносятся жалобы~--- почему, например, на компьютере под Linux фильмы нельзя смотреть так же просто, как на бытовом плейере. Так давно уже~--- можно. Легким движением руки - всовыванием в CD-привод диска с Movix,~---  тысячедолларовый компьютер нечувствительно превращается в телевизор баксов за 200 (о чём говорится в заметке <<Почему компьютер не видка>>). Так неужели превратить его в кассовый терминал (или что там у НИХ в бухгалтериях)~--- сложнее? И, к слову сказать, FreeBSD как платформа такого терминала будет смотреться ничуть не хуже, нежели Linux\dots