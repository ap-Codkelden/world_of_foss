\chapter{FOSS и СМИ, лицензии, копирастёж и деньги}
\section{Журнал Linux Format: обзор первого номера} 

\begin{timeline}
20 сентября 2005
\end{timeline}
Отечественная пресса, систематически освещающая вопросы Linux, Unix и Open Source, не поражает изобилием наименований. До недавнего времени должное внимание этим темам уделяли внимание журналы Открытые системы и Системный администратор. Хотя на ниве издания тематических Linux-выпусков отметились и Домашний компьютер, и Chip, и Upgrade Special. Однако специализированного Linux-журнала на русском языке до недавнего времени не было.

Минувшей весной эта пустота была, наконец, заполнена ежеквартальным журналом Chip Linux. Однако необходимость в издании ежемесячного журнала, целиком посвященного тематике Open Source в целом от этого не уменьшилась.

И вот~--- свершилось: на только что закончившейся выставке LinuxWorld компания Линуксцентр представила первый в России ежемесячный журнал~--- Linux Format.

Правда, полностью отечественным его пока назвать можно с определенной условностью: это перевод одноименного английского (и, соответственно, англоязычного) журнала. И не просто перевод, а, можно сказать, точное его зеркало, причем выходящее, фактически, без запаздывания. То есть ныне вышедший сентябрьский номер перевода соответствует сентябрьскому же номеру оригинала. И что же мы увидим внутри? Пересказывать содержание не буду, остановлюсь только на статьях, привлекших мое внимание.

Центральный материал номера, безусловно,~--- специальный репортаж Debian на перепутье, посвященный выходу версии 3.1 этого дистрибутива. В нем, кроме особенностей последнего, рассмотрена также история Debian и его взаимоотношения с <<потомками>>, из которых наибольшую популярность завоевал Ubuntu. Не знаю, как для <<действующих>> пользователей Debian, но для тех, кто, подобно автору этих строк, практически не имел с ним дела, в этом материале можно найти много интересной и полезной информации.

К <<материалу номера>> тесно примыкает статья про Xandros~--- один из как бы коммерческих клонов Debian. Не могу сказать, что я~--- поклонник этого дистрибутива (в уже далеком прошлом он был известен как Corel Linux) и самого принципа создания <<окноподобных>> Linux'ов. Однако всегда полезно знать, что делается в <<смежной>> области.

Вторым по важности мне видится материал, посвященный сравнению текстовых редакторов. В статье рассмотрены особенности как классических Vim и Emacs, так и самых мощных редакторов для графического режима~--- NEdit и Kate, освещены также CoolEdit и gedit. Уделено внимание также легко- и средневесам~--- nano и Minimum Profit (о последнем, каюсь, я услышал впервые). Должен заметить, что выводы автора примерно совпали с моими представлениями. В частности, приятно было увидеть высокую оценку моего любимого NEdit\dots

Очень интересно интервью с Жилем Дювалем~--- первым разработчиком дистрибутива Mandrake и основателем компании MandrakeSoft (ныне~--- Mandriva). В нем <<из первых рук>> рассказывается о причинах, по которым компания оказалась на грани банкротства, и о том, как она выпуталась из такого положения.

Это, так сказать, материалы общего назначения. Однако в журнале уделено достаточно места и более специальным проблемам. Так, можно прочитать забавную статью про <<исправление улыбки>> с помощью стандартных инструментов GIMP. Начинающим программистам адресованы циклы статей про Perl и PHP. И, наконец, исчерпывающее руководство по системе коллективной работы над проектом~--- Subversion. Которое видится очень актуальным~--- ведь многие проекты Open Source ныне используют именно его (чему примером KDE).

Крупные материалы номера перемежаются с небольшими заметками, построенными по типу <<Вопрос~--- ответ>>. Они рассчитаны в основном на начинающих пользователей, но и более искушенным там есть что посмотреть. В частности, лично мне очень кстати оказались соображения о выборе CMS (системы управления контентом сайта).

Оценка переводного журнала читателем в значительной мере зависит от исполнения переводов. И тут приятно отметить, что качество таковых~--- вполне на уровне. Хотя в некоторых статьях некоторые пассажи и показались мне спорными, тем не менее случаев явных ошибок мне не встретилось.

На высоте также полиграфическое оформление журнала. Хотя дело и не обошлось без нескольких мелких ошибок верстки~--- их появление в первом номере вполне объяснимо (и, я даже сказал бы, неизбежно). Думаю, в последующих выпусках такого не повторится.

Неотъемлемой частью журнала является сопровождающий его диск~--- двусторонний DVD, все 9 гигабайт которого заполнены разнообразным софтом. В соответствие с центральным материалом номера, и на диске основной объем занимает дистрибутив Debian~3.1 Sarge. Он дополняется последней версией одного из самых популярных дистрибутивов~--- Fedora Core~4. Немало там и всяких приложений, вплоть до игр.

Резюмирую: первый номер Linux Format произвел на меня очень хорошее впечатление, давненько не доводилось мне читать <<бумажный>> журнал с удовольствием. А прилагаемый диск окажется очень полезным для всех граждан нашей страны, не избалованных чрезмерно хорошим коннектом (к сожалению, таковых пока большинство).

Несколько слов о перспективах. По полученным от редакции журнала данным, в планах их~--- не только <<зеркалирование>> англоязычного издания, но и публикация оригинальных материалов отечественных авторов. По своему уже более чем 15-летнему опыту знакомства с компьютерной прессой рискну предположить, что со времнем <<удельный вес>> таких публикаций будет возрастать.

\section{Linux-пресса на Руси: Вопросы истории} 
\begin{timeline}2006, апрель\end{timeline}

На этих страницах я хотел бы проследить, как менялось освещение Linux на страницах печатной компьютерной периодики.

Здесь не будет ни полной библиографии статей, ни даже упоминаний всех изданий, уделивших место для предмета нашего разговора. Моя цель, скорее, выявить тенденции развития Linux-прессы в прошлом и её перспективы~--- в грядущем. Термин <<Linux-пресса>> употребляется здесь для краткости~--- не следует забывать, что речь идет также о Unix и Open Source вообще.

Но сначала~--- несколько слов о том, с чего все началось. У истоков Linux-прессы лежал журнал <<Монитор>> (вскоре прекративший свое существование), который в 1994 году напечатал статью Владимира Водолазкого о том, как легко и без головной боли установить Linux. В этой статье, вероятно, слово Linux впервые прозвучало в русскоязычном окружении. Актуальность же она сохраняла еще годы спустя, будучи не только источником информации об этой ОС, но и руководством к действию: ксерокопии её ходили по рукам в сопровождении горы дискет с дистрибутивом Slackware, и, вероятно, не одно поколение <<линуксоидов первого призыва>> ставило свою первую систему с этого <<комплекта>>. Итогом той давней публикации стала книга Владимира <<Путь к Linux>>~--- первое (1999 год) отечественное издание на эту тему. Но это~--- уже другая история.

Первым журналом, в котором сложилась устойчивая традиция линуксописания, не прерывающаяся и по сей день, стал <<Мир ПК>>. В нем, начиная с 1995 года, регулярно начали публиковаться как переводные, так и оригинальные статьи о самых разных аспектах ОС Linux. Причем отечественные авторы преобладали, и их публикации проявили отчетливую тенденцию к циклизации: до сих пор памятны циклы статей Игоря Хименко (по совместительству~--- одного из соратников Сергея Кубушина, развивавшего тогда на Украине инновационный дистрибутив KSI; однако и это отдельная тема), посвященные файловой системе Linux, процессам, командной оболочке bash. Правда, со временем традиция циклических публикаций в <<Мир ПК>> угасла…

Начиная с 1997 года, многие общекомпьютерные журналы начинают публиковать статьи по тематике Unix, Linux и Open Source~--- не часто, но относительно регулярно такие материалы появляются в журналах <<PC Magazine/RE>>, <<LAN>> и <<Открытые системы>>. Характер этих публикаций различен. Если <<PC Magazine/RE>> печатает почти исключительно переводные статьи весьма случайного содержания, то <<LAN>> и <<Открытые системы>> (тот же издательский дом <<Открытые системы>>, что и <<Мир ПК>>) отдают предпочтение отечественным авторам, и в их материалах можно наметить ту же тенденцию к циклизации, впрочем, также не получившую развития. Упомяну тут и журнал <<СУБД>>, проживший короткую, но яркую жизнь (оборвавшуюся вскоре после приснопамятного августа 1998 года)~--- это был практически единственный компьютерный журнал <<академического>> стиля. И, хотя специальных материалов о Linux в нем не было, его тематика тесно пересекалась с Unix и Open Source.

В общем, разнородность содержания и отсутствие систематического подхода к Linux-тематике характерна для большинства общекомпьютерных изданий и по сей день. Но были и попытки целенаправленного освещения Unix, Linux и Open Source.

Первым в этом ряду должно упомянуть журнал Byte Россия. Начав свое существование в 1998 году под эгидой издательского дома <<Питер>>, он с первых же номеров начал регулярно печатать материалы по Unix и Open Source, причем переводные статьи сразу же составляли меньшинство по сравнению с работами отечественных авторов. Очень скоро эти материалы были выделены в специальный раздел~--- Byte/Unix, своего рода журнал в журнале, редактором которого стал Алексей Выскубов. Это была первая попытка создания специализированного издания по тематике Unix, Linux и Open Source вообще: помимо статей о Linux, здесь уделялось внимание и BSD-системам, и общим вопросам идеологии свободного программного обеспечения. В частности, именно на его страницах увидели свет переводы статей Николая Безрукова об особенностях разработки программ с открытыми исходными текстами, сохраняющие актуальность и по сей день. Благо, они доступны в сети.

Вообще трудно переоценить роль Byte Россия в развитии русской Linux-прессы того времени~--- тираж его в 2000 году достиг 17 тысяч, и не последнюю роль в этом сыграли материалы Unix-раздела. Однако, в конце 2000 года журнал был продан другому издательскому дому, политика его резко изменилась, раздел Byte/Unix был ликвидирован, полностью сменилась редакционная команда. И хотя в начале 2001 года в нем еще по инерции публиковались некоторые материалы по Linux, интерес к нему со стороны линуксописателей (и, подозреваю, линуксочитателей) был утрачен безвозвратно.

Однако развитие Linux-прессы продолжалось, правда, существенно сместившись в сторону онлайновых СМИ. Эстафету подхватила Компьютерра: в начале 2001 года в рамках этого издательского дома возник проект Софтерра (редактор Сергей Scout Кащавцев) со специальным разделом~--- FreeOS (Федор Сорекс), посвященным свободному ПО вообще и ОС Linux (а также Free-, Open- и прочим BSD), в частности. Все материалы проекта публиковалась в Сети, но некоторые статьи раздела попадали также и на страницы <<бумажной>> Компьютерры.

Наибольшим успехом проекта можно считать выход тематического номера журнала <<Домашний компьютер>>, тоже относящегося к издательскому дому Компьютерра (ДК, 2002, \textnumero12). Созданный под идейным руководством Максима Отставнова, он содержал материалы по всем аспектам устройства и использования Linux, в том числе и в домашних условиях. К сожалению, этот успех был последним: в течение первых месяцев следующего года проект FreeOS плавно сошел на нет (да и Софтерра как таковая~--- тоже) , материалы его исчезли из прямого доступа с сайта журнала и, похоже, утрачены безвозвратно.

Следующим номером в эстафетном забеге Linux-прессы оказался Upgrade~--- усилиями Алены Приказчиковой и Сергея Голубева, начиная с середины 2002 года, почти каждый номер в <<софтверном>> разделе содержал материалы про Linux и Open Source, чему способствовал и приток новых авторов. Тенденция к систематическому освещению темы нашла свое воплощение в тематическом выпуске Upgrade Special, посвященном Linux вообще и дистрибутивам Live CD в особенности (середина 2004 года). Однако, как и в случае с ДК, он одновременно стал символом упадка интереса к Linux и Open Source со стороны редакции Upgrade: в последующее время материалы по этой тематике появлялись там лишь эпизодически.

Сказанное не значит, что остальная компьютерная периодика перестала уделять внимание Linux-тематике. С самого дня своего создания (начало 2001 года) значительную активность на этом поприще проявлял журнал Chip~--- вплоть до выпусков спецномеров, целиком посвященных Linux, и сопровождавшихся компактом с тем или иным его дистрибутивом. В 2002 году возник журнал <<Системный администратор>>, коему по титулу положено освещать вопросы, связанные с сетевым администрированием~--- а в этой сфере Linux и свободные BSD-системы доминируют и по сей день (особенно на территории РФ). И материалы о них появляются на его страницах с завидной регулярностью. Время от времени к тематике Linux и BSD обращается <<Хакер>>~--- правда, подчас с материалами весьма спорными.

В ряду общекомпьютерных периодических изданий следует выделить упомянутый ранее журнал <<Открытые системы>>. Его специфика в том, что, наряду со статьями популярного направления он публикует и сугубо научные материалы, посвященные теоретическим вопросам Computer Science (что не удивительно, учитывая его академические корни). К сожалению, популярность его все время падает, а в последние годы он практически пропал из розничной продажи.

Журналы, публикующие материалы по нашей тематике лишь эпизодически, на общую картину Linux-прессы влияли слабо. Её путь, со дня зарождения и почти до сегодняшнего момента, можно описать как серию попыток создать специализированное издание, профилированное на Unix, Linux и Open Source. Ни одно из этих предприятий успехом не увенчалось. Причины неудач можно выискать разные, но в основе каждый раз лежало одно: абсолютное равнодушие руководства общекомпьютерных изданий (даже таких, как <<Открытые системы>>, которых, казалось бы, даже титул обязывал) к Linux-тематике. Дело помощи линуксоидам следовало брать в руки самих линуксоидов.

Таким образом, мы плавно подошли к 2005 году. Я не пророк, но думаю, что он войдет в историю как второй, после 1998 года, переломный рубеж в развитии Linux в России. Если первый ознаменовался выходом прототипа первого российского дистрибутива~--- Mandrake Linux от IPLabs Linux Team (в последующем~--- Altlinux) и попыткой создания первого специализированного издания (Byte/Unix), то в течении 2005 года, во-первых, в России резко активизировалась деятельность лидеров мирового дистростроения (Red Hat и Novell/Suse), и, во-вторых, к нему приурочены первые массовые выставки-конференции, специально посвященные тематике Open Source (Open Source Forum Russia, LinuxWorld Russia, LinuxLand на Софтуле).

А для Linux-прессы год этот памятентем, что в течении его начали выходить первые специализированные периодические издания, полностью посвященные Linux и Open Source~--- Chip Linux Special и Linux Format.

Chip Linux Special~--- ежеквартальное издание, которое берет свое начало с весны 2005 года. Он генетически связан со специальными выпусками журнала Chip, и издается тем же издательским домом <<Бурда>>, что и прародитель. Комплектуется исключительно оригинальными статьями отечественных авторов. Специфика журнала~--- сконцентрированность вокруг материалов по <<титульной>> ОС, расширение тематики в сторону других свободных систем (например, BSD-семейства), насколько я знаю, не планируется.

Эта статья в целом уже была написана, когда поступила печальная весть: журнал Chip Linux Special более издаваться не будет. О причинах этого мне ничего не известно, да и сами источники сложно назвать официальными: помимо личного общения с членами команды Chip Special Linux можно назвать только новость на Linux.org.ru, в которой (со ссылкой на телефонный разговор с издательским домом <<Бурда>>) сообщалось, что журнал было решено закрыть. Хочется надеяться, что <<слухи о его смерти сильно преувеличены>> и Chip Special Linux еще порадует нас своими выпусками~--- благо приобрести его можно было <<от Москвы и до Находки>>.

Linux Format~--- ежемесячный журнал, родившийся в сентябре 2005 года усилиями фирмы Линуксцентр, занимавшейся онлайновой торговлей дистрибутивами свободных ОС и профильной литературой. Он представляет собой перевод одноименного английского журнала, своего рода русское его зеркало: каждый номер по содержанию соответствует аналогичному номеру оригинала (хотя и выходит с некоторым запозданием, обусловленным затратами времени на перевод, верстку и печать). За одним исключением: начиная с 4-го номера, в журнал включаются и оригинальные статьи отечественных авторов. Рискну предположить из исторических аналогий, что с течением времени процент последних будет возрастать. Ведь все патриархи отечественной компьютерной прессы (Мир ПК, Компьютер-Пресс, PC Magzine/RE) начинались некогда как чисто переводные издания.

Отличительная черта Linux Format~---открытость. Все стороны его существования~--- от содержания очередного номера до причин недоставки конкретного экземпляра~--- можно обсудить на его форуме. Члены редакционной команды в обсуждении участвуют~--- и, должен вам заметить, к обоснованным мнениям читателей всегда прислушиваются.

Впрочем, много говорить об этом издании не буду: раз вы читаете материал этого номера, то знакомы с ним в достаточной степени. А потому завершу свое сочинение рассуждениями на тему, каким видится идеальный журнал по тематике Unix, Linux и Open Source.

\section{\dotsИдеальный журнал} 
\begin{timeline}2006, апрель\end{timeline}

Традиционно <<толстые>> компьютерные журналы разделяются на две части: блок новостей и, так сказать, <<тело>> журнала~--- собственно материалы номера. Оправдана ли такая организация в век тотальной интернетизации? В век, когда все, имеющие хоть какое-то подключение к Сети, получают интересующие их последние известия из онлайновых источников, новостные разделы даже компьютерных еженедельников выглядят сборником анекдотов с бородой Карла Маркса. Что же тогда говорить о <<новостях>> ежемесячников?

Так что же, ликвидировать новостные блоки? Отнюдь~--- это было бы политически неправильным. Увы~--- изрядная часть населения постсоветских пространств лишена прелестей Интернета, и Печатное Слово, пусть несколько устаревшее (да и доставленное, силами российской почты, с запозданием), для нее~--- единственный источник информации. А потому предлагается компромиссный вариант: заменить сборники анек… пардон, новостей~--- аналитическими их обзорами. Которые, давая достаточно сведений читателям, не имеющим подключения к Сети, в то же время не вызывали бы раздражения своей <<бородатостью>> у тех, кто таковое имеет. А в идеале~--- были бы просто интересны сами по себе. Конечно, составление таких обзоров~--- дело нелегкое, но оправдается повышением читательского внимания.

Теперь об основной части~--- статьях. Журнал, рассчитывающий на самоокупаемость (а в идеале~--- и на принесение прибыли) обязан ориентироваться на самые широкие пользовательские массы И потому должен содержать материалы нескольких градаций: для совсем начинающих, для <<действующих>> пользователей, и для тех, кто ставит своей задачей углубленное изучение каких-либо частных вопросов. Хорошо это или плохо~--- обсуждать не будем, такое <<смешение жанров>> на данном этапе развития Linux-прессы является необходимостью. Как показала трагическая кончина журнала <<СУБД>> (да и безрадостная судьба аккумулировавших его <<Открытых систем>>), специализированное издание <<для профи>> пока не имеет шансов выжить на постсоветском пространстве. С другой стороны, ориентация издания на <<чайников>> чревата потерей интереса к нему, как только <<чайники>> таковыми быть перестанут (а с помощью хорошего журнала это произойдет очень быстро).

Какой видится компоновка материалов? Возможны варианты: по степени <<продвинутости>> предполагаемого читателя, по тематике, в том числе и с выделением некоего центрального материала и его <<системного окружения>>. Впрочем, я~--- категорический противник <<темы номера>>, что приемлемо для еженедельника, но для ежемесячного издания смертельно: ведь если читателю не интересна именно эта тема номера, он на целый месяц лишается возможности что-либо почерпнуть из любимого журнала.

Формат большинства журналов общего назначения предполагает преимущественно двух- или, реже, четырехполосные статьи, что для глубокого изложения многих животрепещущих проблем явно недостаточно. Конечно, проблема эта для периодических изданий не решаема в принципе: увеличение объема статей повлечет за собой сужение тематики номера и риск утраты читательских симпатий. Однако некий компромисс тут возможен~--- в виде пролонгированных из номера в номер тематических циклов.

Для журнала тематики Unix и Linux очень существенен общий дизайн~--- и здесь положение, в большинстве случаев, не удовлетворительное. Многоколоночная верстка, пришедшая из мира рекламно-развлекательной периодики, и терпимая в периодике, так сказать, литературно-повествовательной, оказывается проклятием в изданиях технического профиля. В нашем случае это относится в первую голову к командам и листингам. Писать без них серьезно про Linux и Unix~--- это все равно, что писать про музыку без нот, или про живопись~--- без репродукций. Но~--- увы~--- длинные команды или содержимое конфигурационных файлов вписывается в облик страницы традиционного современного журнала ничуть не лучше, чем <<парень в джинсах и кожаной куртке>>~--- в интерьер ресторана для новорусского истеблишмента.

Я прекрасно понимаю, что общий дизайн издания определяется множеством привходящих факторов (в том числе и политических). Но и тут возможны варианты. Например, давать необходимые команды и листинги <<внеформатными>> врезками. Или~--- сделать онлайновое дополнение к журналу, которое содержало бы именно ту часть статей, которая подлежит использованию методом Cut\&Paste. Такое дополнение, не дублируя содержание основного материала, будет способствовать его практическому использованию.

\section{К вопросу о торговых марках} 
\begin{timeline}2001\end{timeline}
Интересный вопрос: если система выглядит как UNIX, делает то же, что UNIX, причем почти тем же образом, что UNIX, почему она~--- не UNIX? Ответ прост~--- угадайте с одного раза: потому, что UNIX~--- торговая марка, права собственности на которую принадлежат\dots да поди разберись, кому они нынче принадлежат. А ведь UNIX существует уже больше тридцати лет, и всем доподлинно известно, что это такое, более того~--- имя это будут помнить еще долго, даже если самой системе суждено будет уйти в историю. Но может ли кто-нибудь сходу назвать имен хотя бы пары-тройки правообладателей этой марки, сменившихся за все эти годы?

Поводом для обращения к этой теме послужила реклама очередного компьютера Intel Inside на процессоре Pentium то ли~III, то ли 4, в очередной раз попавшаяся на глаза в очередном журнале, с обычным в таких случаях пояснением (видимо, для особо продвинутых), что Intel Inside\textregistered и Pentium\textregistered~--- зарегистрированные торговые знаки компании Intel Corporation. Таких реклам я видел несчетно, и всегда они по непонятной причине вызывали у меня чувство незавершенности. Но на сей раз меня как озарило: в них не хватает знака Trade Mark или Registered после цифры! А ведь было бы круто, согласитесь:
\begin{shadequote}{}
Intel\textregistered Pentium\textregistered~4\textregistered (или \texttrademark, без разницы)
\end{shadequote}
А технологически (но не морально-юридически) устаревший Pentium~III открывает еще более богатые возможности: знак~\textregistered или \texttrademark можно поставить и после латинской единицы, и после двойки, и после тройки. Закрепив, таким образом, свои права на, по крайней мере, первые три члена натурального ряда чисел. И требовать лицензионные отчисления с тех, кто воспроизведет публично латинские цифры IV (за одну единицу), VII (и за две единицы, и за двойку), а паче всего~--- за~VIII: сразу и за три единицы, и за двойку как компонент последовательности, и за тройку по совокупности.

Это напоминает мне деятельность одного нашего достославного издательства, выпускавшего (и, вроде бы, выпускающего и поныне) книжки про Конана-варвара. Не могу отказать себе в удовольствии процитировать пару фраз с оборота титула:

\begin{shadequote}{}
Конан\texttrademark, Conan\texttrademark~--- зарегистрированные торговые марки\dots запрещается воспроизведение этой книги в любой форме\dots а также использование имени Конан\texttrademark.
\end{shadequote}
Не скрою, увидев это впервые, я был в немалом восхищении и, в меру собственной испорченности, начал размышлять о последствиях. Думается, в Ирландии Конаны ныне встречаются если и не на каждом шагу, то уж не реже, чем на Руси~--- Ярославы или Всеволоды. И вряд ли у каждого из них стоит в паспорте (или чем там они личность удостоверяют) ссылка на собственника торговой марки: дескать, не просто Конан я, 
а Конан\texttrademark, что есть зарегистрированная торговая марка компании Имя рек. И следовательно, каждому можно вчинить иск с требованием компенсации\dots

Но это еще не все. Ведь Артур КОНАН Дойл тоже не удосужился предвидеть, что второе его имя будет зарегистрированной торговой маркой. Сам сэр Артур, к прискорбию правообладателей, ответить по всей строгости закона уже не сможет, но есть ведь потомки~--- а сказано, что грехи отцов падут на детей до третьего и четвертого колена: пока время не ушло, и поколения не сменились, стоит подать на них в суд за присвоение торговой марки.

Да и в многочисленных изданиях книг писателя-правонарушителя нигде не отмечено (по крайней мере мне не попадалось), что написал их Arthur Conan\texttrademark Дойл или хотя бы Артур Конан\texttrademark Дойл. А уж своего брата-издателя засудить~--- милое дело.

Если же вспомнить минимум про двух ирландских королей-Конанов (без~\texttrademark) начала нашей эры,~--- перспективы возникают поистине необъятные. Сами-то они злонамеренно улизнули от ответственности, но ведь, как известно, все ирландцы происходят от королей. Значит, за нарушение прав на торговую марку и ответить должны все. А это уже пахнет иском на государственном уровне~--- ко всей Ирландской Республике. Да и к Северной Ирландии заодно~--- наверняка и там потомки Конанов-королей найдутся. А Северная Ирландия~--- часть Соединенного Королевства, со всеми вытекающими\dots

Однако это все мелочи по сравнению с гениальным достижением известной фирмы, выразившемся в знаменитом словосочетании Microsoft\textregistered Windows\texttrademark. Тут уж и впрямь дух захватывает от полета фантазии~--- окон-то в мире производится ого-го сколько! Получается, что в печати слово <<окно>> во избежание нарушения этого самого~\texttrademark рекомендуется употреблять только в единственном числе, поскольку на X~Window System пока, вроде, никто своей торговой марки не навесил. С первой же половиной символа софтверной индустрии вышла недоработка: можно ведь было и более предусмотрительно значки расставить~--- скажем, так: (Micro{\textregistered}Soft\textregistered)\texttrademark. Тем самым в фирменную собственность можно было бы оприходовать сразу и все мелкое, и все мягкое. А заодно и теплое, с мягким часто путаемое\dots

Post Scriptum: все упомянутые (и не упомянутые) в статье торговые марки являются вроде как собственностью их правообладателей.

\section{Еще раз к вопросу о так называемых пиратах} 
\begin{timeline}2004 г\end{timeline}
На заре своей карьеры чукчи-писателя я несколько раз порывался описать свое видение проблемы т.~н.~пиратов (в первую очередь в IT-сфере, разумеется,~--- но и в аспекте общечеловеческих представлений тоже). Сначала~--- все не мог собраться по разным причинам, потом~--- потому как лень было разбираться с написанным ранее другими (а написано, как все вы знаете, на эту тему немерянно).

А в конце концов пришел к точке зрения великого историка Ключевского. Который, как известно, в свое время (еще в позапрошлом веке) категорически отказывался обсуждать проблему призвания варягов на Русь, мотивируя это тем, что все точки зрения, хоть как-то основанные на фактическом материале, уже высказаны. И ни одну из высказанных точек зрения, базируясь на имеющейся фактуре, доказать невозможно (в скобках заметим~--- и опровергнуть тоже, по крайней мере~--- из числа разумных). Тем более, что личного интереса к проблеме пиратства у меня уже не было по понятным причинам.

Однако времена меняются, и даже в проблеме Рюрика сотоварищи наметились позитивные сдвиги. Так что, может быть, пора вернуться и к нашим IT-флибустьерам, <<братьям по кров\'{и} упругой и густой>>?

Ну и последним (по времени, не по значению) побуждающим мотивом послужила очередная дискуссия на очередном форуме, ныне в Бозе почивше. Которая, с одной стороны, не успела разрастись до полной необозримости, с другой~--- затронула почти весь круг related themes. Так что нижеизложенное будет в значительной мере состоять из цитат с этого самого трейда. Все цитаты даны методом cut and past, без правки.

Историю российского компьютерного пиратства можно исчислять примерно с момента времени минус 10 лет. Когда где-то на исходе 93-го~--- заре 94-го на Руси появились первые сборники под значимым именем All for Windows. Еще на <<золоте>>, без всякого полиграфического оформления (или с лабелами, напечатанными на принтере, подчас матричном). А под сакраментальным в дальнейшем термином Windows имелась в виду версия 3.1 этой системы. К слову~--- были такие диски и под прозванием All for OS/2, а чуть позже, уже в тиражную эпоху, появились и аналогичные подборки All for Linux\dots

Поначалу диски этого типа не то что не тиражировались~--- но даже и не наличествовали в открытой продаже. Писались они в основном под заказ, со стриммерной ленты (происхождение которой не было ведомо верховному божеству ни одной из мировых религий). Для вхождения в <<стриммерную>>, если так можно было выразиться, требовался не просто пропуск~--- а весьма надежная рекомендация общих знакомых знакомых знакомых, как минимум\dots

Если мне не изменяет память, цена да диски эти составляла около 30 уев (впрочем, и слова такого тогда еще не было; хотя ихние доллары уже тогда ничем не отличались от наших баксов). Точного эквивалента уже не помню, но сумма по тем временам~--- очень значительная\dots

Как-то мне довелось услышать, применительно к <<пиратскому>> софту, выражение: это не пиратский софт, а трофейный (копирайт, если не ошибаюсь, принадлежит Алексею Смирнову из Altlinux'а). Так вот, программы с тех дисков в полной мере соответствовали такому определению. А их распространители, следовательно, должны были бы называться вовсе не пиратами, а, скажем, призоньерами или каперами\dots

Хотя тут можно вспомнить и времена почти былинные~--- со второй половины 80-х по самое начало 90-х, эпоху массовой PC'кизации постсоветской России. Когда все (все!) попадавшие на родину мирового социализма программы имели именно трофейное происхождение. И не важно, что брались они не в абордажных боях, а выпрашивались в кулуарах международных совещаний.

Ну и совсем уж седая древность~--- это времена ЕС'ок и СМ'ок, софт к которым добывался в боях в прямом смысле этого слова~--- пусть и бойцами невидимомого фронта. Впрочем, в то героическое время я, в информационном смысле, был крайне мал, и свидетельствовать о том не могу\dots

А в промежутке времени, между классическими буканьерами советской эпохи и их золоторежущими эпигонами середины 90-х, тоже был интересный период времени. Боюсь, многим в это трудно поверить, но это~--- было. В один из первых в Москве отделов, торгующих <<правильным>> софтом (в Доме книги на Арбате, тогда еще имени всероссийского старосты Калинина), заходили и спрашивали: а скоро ли завезут Borland C, или там Borland Pascal, или еще какое FoxPro. И когда сие радостное событие свершалось~--- у отдела буквально выстраивалась очередь, почти как в винный отдел во времена Горбачева. И покупали~--- эти вот коробки, сопровождаемые килограммами документации, кто~--- по безналу на свои конторы (часто госбюджетные, излишними финансами не отягощенные), а кто~--- и за свои кровные, наличные, сопоставимые с месячной зарплатой госбюджетного трудящегося.

Не прошу прощения за столь длинную преамбулу: она призвана проиллюстрировать ошибочность тезиса о врожденной, чуть ли не генетической, тяге советского (и постсоветского) человека к контрафактной продукции.

И к тому же эта преамбула неожиданно подвела меня (и, надеюсь, моих читателей) к первой цитате из помянутого выше топика:
\begin{shadequote}{}
Если закон вынуждены нарушать 99,9\% граждан, то проблема явно не в самих гражданах\dots
\end{shadequote}

Впрочем, на тему эту кто только не высказывался~--- начиная с бессмертного Салтыкова-Щедрина\dots И смешно звучат высказывания, что это началось только сейчас. Ни один руководитель советского производства, начиная с уровня младшеначальственного, без нарушений законов, инструкций, подзаконных актов и прочего просто не в состоянии был бы выполнять свои должностные обязанности. О чем все знали снизу доверху, и что культивировалось совершенно сознательно, дабы в любой нужный момент любого руководителя можно было бы взять\dots ну сами знаете за что. Надеюсь, никто из моих читателей не испытывает иллюзий в отношении того, что с тех времен что-нибудь изменилось.

Будем считать первую цитату~--- \textit{pro} ворованный софт. Теперь~--- \textit{contra}:
\begin{shadequote}{}
Ну поймите же, наконец, что нельзя быть честными только тогда, когда Вас это устраивает.
\end{shadequote}
С одной стороны, все верно: <<Честь у честного не купишь, честный чести не уступит, честь нужна ему как свет>>. Но с другой: можно ли считать нечестным человека, который к тому вынужден? Школьника, которого учат (в классе~--- не в подпольной воровской школе) набивать буковки на ворованной копии Word'а? Студента, учащегося проектированию на ворованной копии AutoCAD, или программированию~--- на столь же контрафактном компиляторе? Да еще под управлением ОС, копия которой, с точки зрения буквы закона, столь же далека от лицензионной чистоты\dots

А можно ли считать нечестным того, кто создает им такие условия? То есть лиц, занимающихся материально-техническим снабжением (или как там это называется) школы/ВУЗа/детсада. Предположим, что такой ответственный дядя~--- для определенности назовем его условно директором,~--- решит быть честным по отношению к правообладателю соответствующего софта. И весь свой бюджет вбухает в легальное программное обеспечение. Как он поступит по отношению к своим воспитанникам? Да, ИМХО, самым бесчестным образом! Потому как он туда поставлен для того, чтобы создавать своим воспитанникам нормальные условия учебы. Включающие, помимо всего прочего, школьные завтраки, удобные парты и еще\dots да мало ли чего еще им требуется\dots

Я, конечно, понимаю, что детей дяди Билла, дяди Ларри и прочих дядей жалко. Но ведь своих-то детей~--- завсегда жалче. И возразить против этого может либо очень циничный, либо очень наивный человек.

Предвижу возражение, очень емко и лаконично сформулированное одним из участником дискуссии:
\begin{shadequote}{}
Тогда Вам к Кремль!
\end{shadequote}

Несмотря на блеск и афористичность, вынужден отмести. Мне (и, надеюсь, всем читателям, имеющим потомство) не очень понятно, почему человек, умеющий учить детей, должен идти в Кремль (или даже в президентский кабинет). Где он в лучшем (для него) случае из хорошего директора школы превратиться в обычного (если уж очень повезет~--- в не самого плохого) чиновника. Столь же неспособного изменить существующее положение вещей, как не смог это сделать ни один из отягощенных благими намерениями ранее\dots

Из чего~--- следующая цитата из серии \textit{contra}:
\begin{shadequote}{}
На мой взгляд главная причина по которой процветает пиратство в России это то что государство не борется с пиратством, точнее делает вид что борется.
\end{shadequote}

Возникает вопрос~--- а почему они только делают вид? Попробуем найти объяснение в следующей цитате:
\begin{shadequote}{}
\dotsтак как пришлось самому столкнуться и увидеть, что за <<пиратами>> стоят настоящие бритоголовые бандиты.
\end{shadequote}
Вполне верю. Однако за каким видом деятельности они не стоят? Разве что за абсолютно убыточным\dots И еще:
\begin{shadequote}{}
\dotsребята, которые рассуждают о высокой миссии пиратов по созданию в России цивилизованного рынка видео (музыки, информатики), просто не понимают, каким харям по концовке они несут бабки.
\end{shadequote}

Однако вспомним~--- на протяжении 70-ти лет советской власти весь советский народ, как один человек, нёс свои бабки, недоплаченные им в виде хлебной пайки, сначала комиссарам в кожанках, потом наркомам в скромных кителях, потом министрам в галстуках не в тон. А ныне отдает их в виде подоходного налога чиновникам в костюмах от Диора. Не они ли маячат за теми самыми бандюганами в цепурах?

И что мы имеем в итоге? А имеем мы то, что имеем. Предположим, наше доблестное правительство\dots Чуть было не написал: <<партия и правительство>>, но вовремя вспомнил, что партий у нас теперь много. А при советской власти все жаловались~--- система у нас однопартийная, потому что больше нам не прокормить; оказалось~--- могем, если захотим\dots

Так вот, предположим, что будет, если правительство наше начнет бескомпромиссную борьбу с пиратством. Как раньше боролись, например, с пьянством. Помните~--- был указ о борьбе с пьянством, потом~--- об усилении борьбы с пьянством, потом~--- об искоренении пьянства\dots Простой соввласть еще несколько лет~--- был бы и указ об усилении искоренения. Так вот на такое предположение отвечаем цитатой:
\begin{shadequote}{}
Например у нас в городе в декабре 2003 началась антипиратская компания, несколько заказных рейдов (заказывали конкурентов) и всё. Большинство фирм как работали на пиратском софте так и работают, ни один магазин пиратских дисков не закрыт. План выполнен, отчёт в Москву отправлен, можно пользоваться пираткой дальше.
\end{shadequote}
Это~--- в масштабах города. А в масштабе всесороссийском имеем еще более яркий пример: тендер на поставку в средние школы изобилия всякого разного Windows. помнится, я тогда сочинил заметку на эту тему, которая нынче показалась мне не лишенной актуальности.

В общем, история Государства Российского учит нас, что наведение порядка <<только кнутом>> всегда давало только одну его разновидность~--- кладбищенско-лагерную. Как сказал незабвенный же Алексей Константинович aka Толстой:
\begin{verse}
Приемами не сладок,\\
Но разумом не хром,\\
Такой завел порядок~---\\
Хоть покати шаром.
\end{verse}
В общем, если вспомнить одну из первых цитат этой заметки, что нельзя быть честным, когда это устраивает. Потому что~--- можно быть честным только тогда, когда это получается.

Все ли так безрадостно? Нет. Ибо на горизонте уже замаячил <<призрак батьки Махно>>. Со времен <<Красных дьяволят>> мы привыкли воспринимать его как персонаж сугубо карикатурный. Однако  неслабого был ума мужик, этот батька, хотя и не отягченный излишним образованием. И не призывал он грабить награбленное, хотя эксцессы и случались (но по сравнению с доблестными красными командирами был он аки ангел, только без крылышек). А хотел он одного~--- как сказали бы нынче, \textit{fuck off, people}, от меня: и красные, и белые, и прочие зеленые). И в своем Гуляй-поле учредил он народную республику~--- так, как это представлялось ему (и его согражданам в большинстве своем). Куда не хотел пускать ни красных, ни белых\dots

Это я веду к тому, что жить в пиратском обществе и быть свободным от него нельзя, как говорил один бородатый пират-теоретик. Но можно~--- по возможности абстрагироваться от его реалий. Красть (в данном случае я имею ввиду конкретно софт)~--- большинству нормальных людей\dots ну скажем так, не приятно (вне зависимости от тяжести и неотвратимости наказания). Не красть~--- казалось бы, невозможно. Да и западло кажется, честно говоря, покупать легальный проприетарный софт, явно своих денег не стоящий. И притом на кабальных условиях, прямо провоцирующих нормального человека к их нарушению.

Но ведь есть~--- Open Source во всех своих проявлениях. И именно он позволяет нам создать свое Гуляй-поле, пусть даже в масштабах отдельно взятого персонального компьютера. Мир в одночасье мы таким образом не переделаем~--- но, верю, суммарное количество добра на земле хоть чуть-чуть, да увеличится. И когда приходишь к такому выводу, споры о вреде или пользе пиратства становятся просто скучными\dots

\section{Еще раз про свободу и халяву} 
\begin{timeline}6 сентября 2005 г\end{timeline}
Эта заметка написана как отклик на статью Виктора Вислобокова <<Ещё раз о свободном ПО, лицензиях и т.д.>> Она поднимает тему свободного и проприетарного (по-русски~--- частнособственнического) программного обеспечения. Казалось бы, тема эта столько раз обсуждалась как в Сети, так и на бумаге, что в ней давно пора бы уже поставить все точки над \textit{i}. Тем более, что документы основоположников движения за свободный софт не оставляют оснований для двусмысленного их толкования.

И, тем не менее, такое толкование, когда теплое смешивается с мягким, встречается сплошь и рядом. И вследствие этого статья Виктора представляется чрезвычайно своевременной и актуальной.Однако начать я хотел бы с воспоминаний о совсем другой статье, достаточно древней в масштабах компьютерной эры. Статье Евгения Рыбникова <<Дюжина ножей в спину Open Source>>. Потому что она и являет собой квинтэссенцию мифов, легенд и предубеждений, сформировавшихся вокруг открытого и свободного программного обеспечения. И уж если такие предубеждения бытуют в среде IT-профессионалов~--- а смею вас уверить, тот, кто скрывается под псевдонимом Евгения Рыбникова, действительно высокий профессионал в этой области\dots Так вот, если такие предубеждения бытуют в среде IT-профессионалов, то что же говорить, о людях, по роду своей основной деятельности лишь вынужденных использовать компьютеры? И потому боюсь, что к теме открытого и свободного программного обеспечения придется обращаться еще не раз.

А теперь~--- к статье Виктора. Для начала, во избежание недоразумений, оговорюсь: я категорически и полностью согласен со всеми её основными положениями. И в настоящей заметке лишь хотел бы а) несколько дополнить её, и б) высказать свое мнение по некоторым частным её моментам, которые представляются мне спорными.

Статья начинается с классификации программного обеспечения по степени, если так можно выразиться, доступности. Выделяются такие группы:

\begin{enumerate}
\item Коммерческое ПО 
\item Условно-бесплатное ПО 
\item Бесплатное ПО 
\item ПО с открытыми исходными текстами 
\item Свободное ПО 
\end{enumerate}

О коммерческом софте говорить особенно нечего~--- на Руси любой ребенок знает, что такое Windows и Word. А также то, что покупать их по коммерческим каналам совсем не обязательно~--- на то есть базары и лотки. Так что~--- замнем для ясности.

В группу условно-бесплатного софта (т.~н.~shareware) объединено несколько различных на первый взгляд способов распространения программ. Однако все они объединены одним: распространяемые на этих условиях программы для полноценного использования требуется тем или иным способом оплатить. Еще мне хотелось бы подчеркнуть здесь, что собственно shareware-программ в настоящее время практически не встречается. Как правило, под этим псевдонимом выступают либо trial-версии, либо варианты коммерческих программ с обрезанной функциональностью.

Бесплатное программное обеспечение (freeware) в полном смысле отвечает своему русскому имени~--- но не английскому, тут возможны разночтения, связанные со знаменитым отличием бесплатного пива от свободного слова. Общее для этой группы программ~--- то, что за их использование платить не нужно или не обязательно. Последняя категория (т.~н.~donationware) по сути своей гораздо больше соответствует shareware в его исконном значении: понравилась программа (или просто захотелось отблагодарить разработчика)~--- заплатил, не понравилась~--- не заплатил\dots

К сказанному здесь Виктором можно добавить одно: по моим наблюдениям, длящимся вот уже почти 15 лет, статус freeware обычно~--- сугубо временный. В случае успеха такой бесплатный проект очень быстро становится shareware (в понимании пункта 2) или даже просто обычным коммерческим. А в случае неудачи~--- тихо и незаметно прекращает свое существование. Исключений из этого правила очень мало. Мне на память приходит только Arachnophilia Пауля Лютуса~--- типичный, кстати сказать, пример donationware.

А вот по поводу ПО с открытыми исходными текстами я бы поспорил. Как показало обсуждение, Виктор имел ввиду лишь одну, совершенно конкретную разновидность софта с открытыми исходниками~--- т.~н.~\textit{comercial with sources}. То есть~--- обычное проприетарное ПО, сопровождаемое (подчас за отдельную плату) исходными текстами. Однако это~--- относительно редкий способ распространения программ широкого назначения. Мне в этой связи вспоминается, пожалуй, только BSD/OS, известная также как BSDi~--- операционная система BSD-семейства, основанная на той же кодовой базе, что и FreeBSD, но, в отличие от последней, распространяемая за деньги. А вообще 
\textit{comercial with sources}
 более характерен для узкоспециализированных программ и заказных разработок.

Обычно же термину Open Source придается гораздо более широкое понятие, подразумевающее просто доступность (физическую и юридическую) исходных текстов некой программы. И таким образом, свободное программное обеспечение, составляющее следующую группу, оказывается лишь подмножеством ПО с открытыми исходниками. Ибо свободный софт подразумевает доступ к исходным текстам~--- очевидно, что без этого невозможно реализовать ни право на изучение программы, ни право на её модификацию.

При этом открытое ПО оказывается почти столь же неразрывно связанным с ПО свободным: ведь мало радости от свободы изучения или модификации, если нет права распространять модифицированную (например, исправленную) версию. И потому на практике обычно оказывается, что все открытые, но не совсем свободные проекты либо постепенно эволюционируют в сторону полного Free Software, либо утрачивают свой свободный статус.

Ну собственно по главной теме статьи~--- свободному ПО~--- также есть о чем подискутировать с Виктором. Правда, большая часть моих вопросов разрешилась в личной переписке. Тем не менее, согласиться с тем, что подлинно свободное ПО~--- это ПО под лицензией GPL, я не могу. Ибо софт, распространяемый под лицензией BSD и родственными (MIT, X-консорциума и другими, обычно объединяемыми понятием <<университетские лицензии>>) предоставляет пользователи все степени свободны~--- свободу использования, изучения и модификации, свободу распространения. Более того, в некотором отношении BSD-лицензия даже <<более свободна>>, так как не требует непременно свободного распространения продукта, созданного с использованием BSD-лицензированного кода.

Конечно, как правильно отметил Виктор, при этом возникает некоторый риск: человек, использовавший 95\% свободного кода, добавляет в него 5\% функциональности и получившийся продукт распространяет как закрытый коммерческий. Однако этим можно пренебречь: ибо по условиям BSD-лицензии он может закрыть только 5\% своего личного вклада, права <<наложить лапу>> на изначально свободные 95\% он не получает. И на практике такие коллизии решаются очень просто: на базе свободного кода создается fork (порождение) со столь же свободным статусом. Примером тому~--- недавние баталии вокруг изменившейся лицензии XFree86: сторонники полной свободы софта просто создали полностью свободный Xorg, который успешно развивается. Хотя, убей меня Бог, я так и не понял, в чем же новая лицензия XFree86 ограничивает свободу\dots

А вообще, именно BSD-лицензированные продукты замечательно демонстрируют мирное сосуществование свободного и проприетарного софта. Примером чему~--- MacOS X, основанная на свободных микроядре Mach и системном обрамлении FreeBSD, надстроенными проприетарным графическим интерфейсом.

Подчеркну~--- я вовсе не утверждаю, что BSD-лицензия однозначно лучше GPL: в некоторых случаях лучше сработает одна, в иных же~--- другая. Более того, на мой взгляд, в 99 случаях из ста вообще не имеет значения, под какой именно из свободных лицензий распространяется тот или иной программный продукт. Тем не менее, можно представить ситуации, когда BSD-лицензия оказывается единственно приемлемым выбором. Правда, практически такие ситуации пока, вроде бы, еще не возникали (и слава Богу), поэтому рискну прибегнуть к несколько утрированной аналогии.

Как известно, принцип расщепления ядра урана~--- общеизвестный научный факт, находящийся, если так можно выразиться, в свободном доступе. Технологическая цепочка создания на его основе ядерной бомбы, также базируется на массе научных принципов, столь же общеизвестных. И что же, руководствуясь логикой GPL, технология создания ядерной бомбы также должна быть открытой и общедоступной?

\section{И обратно о лицензиях: 1. По мотивам RMS} 
\begin{timeline}Март 25, 2009\end{timeline}
Первым поводом для настоящей заметки послужило предостережение Ричарда Столлмена, высказанное им в статье \href{http://www.gnu.org/philosophy/javascript-trap.html}{The Javascript Trap} (см., например, \href{http://citcity.ru/20806/}{соответствующую новость} на CitCity). Суть его в двух словах следующая: Некоторые web-приложения, написанные на свободных языках сценариев (например, на JavaScript), для ускорения загрузки сжимают код таким образом, что он становится не читаемым для пользователя. Таким образом, получается, что на основе свободного софта создаются в сущности проприетарные программы, так как предоставление читаемого исходного кода~--- непременное условие принадлежности к классу FOSS.

В качестве модельного примера приводится Google Docs~--- онлайновый сервис, выполняющий функции офисного пакета, то есть текстового процессора, электронной таблицы и презентационной программы, работающих без установки на локальную машину. Используя набор технологий AJAX, он в конечном счёте основывается на JavaScript (не только на нём, но в данном контексте это не существенно). То есть на свободном языке сценариев, исходники на котором по определению открыты и доступны. То есть сценарий Google Docs можно загрузить на локальную машину и работать с ним в своё удовольствие. Так в чём же проблема?

Оказывается, в том, что Google Docs~--- сценарий на JavaScript, исходный текст которого оптимизирован для максимального ускорения загрузки путём полного отсутствия комментариев, максимальной <<разгрузки>> от пробелов, сокращения имён методов до одного символа. В результате код оказывается не читаемым для пользователя и, следовательно, недоступным для изучения и модификации:


\begin{shadequote}{}Компактный код~--- не исходный код, а исходный код этой программы оказывается недоступным для пользователя.\end{shadequote}

То есть, по мнению Столлмена, пользователю под видом свободного программного обеспечения подсовывается проприетарный софт, о чём он может и не подозревать: хотя в любом браузере имеется возможность отключения исполнения сценариев JavaScript, но функций различения скриптов с <<нормальным>> и <<компактным>> кодом не предусмотрено ни в одном. Так что пурист FOSS может и не подозревать, что он использует классово чуждый ему закрытый софт.

Впрочем, по мнению Столлмена, Google Docs~--- не единственная угроза целомудрию адепта FOSS. Более существенными ему представляются такие технологии, как Flash и Silverlight. Особенно последняя, так как она не только основана на проприетаных (и, следовательно, закрытых) кодеках, но и не предоставляет альтернативных способов их воспроизведения. Тогда как Flash, используя формат, до недавнего времени закрытый (а ныне его можно квалифицировать только как полуоткрытый), всё же допускает использование свободных заменителей фирменному Flash Player.

В качестве методов сохранения девственности FOSS-пользователей предлагаются следующие:

\begin{itemize}
	\item выработка критерия <<отделения зёрен от плевел>>~--- то есть <<кошерного>> кода Javascript от <<трефного>>; 
	\item переделка свободных браузеров таким образом, чтобы, во-первых, сообщать пользователю о степени <<кошерности>> кода web-приложений, а во-вторых, предоставлять ему возможность выбора между кодом <<чистым>> и <<не чистым>>; 
	\item истинно свободный код JavaScript в случае его <<компактизации>> должен внутри себя содержать ссылку на сайт с исходниками и документацией, и сопровождаться специально разработанной на этот случай лицензией. 
\end{itemize}

Казалось бы, как говорил товарищ Саахов,


\begin{shadequote}{}\dotsвсё это конечно так, всё верно, бумага написана правильно, всё хорошо. Это с одной стороны. Но есть и другая сторона медали.\end{shadequote}

Так что с другой стороны, давайте посмотрим, так ли страшен чёрт вместе со всеми его малютками с точки зрения растления непорочных FOSS-душ?

Начнём с того, что с негодованием отметаем растлевающее влияние Flash и Silverlight: каждый пользователь свободного софта, скачивая соответствующие плагины с сайтов Adobe или Microsoft, не может не отдавать себе отчета в том, что он имеет дело с заведомо проприетарными программами. Мифического FOSS-пользователя, невинного до незнания того, что существуют проприетарные программы, можно уподобить Адаму и Еве до их грехопадения.

В случае с Silverlight ситуация более чем ясна: для сохранения своих сердец в чистоте пуристы FOSS должны просто отказаться от его использования. Для Flash же всегда существует альтернатива~--- пользователь, вместо проприетарного Flash Player, может выбрать любой из числа свободных плейеров swf-формата, благо таковые, в виде gnash, klash (возможно, и других) имеются. То есть он ничем не отличается от пользователя, читающего, редактирующего и даже создающего документы в формате Word'а посредством OpenOffice.org.

Более того, наличествуют даже свободные средства для изготовления swf-роликов~--- SWFTools, Spalah и ещё несколько. Не знаю, насколько свободные Flash-ориентированные программы воспроизводят функциональность проприетарных прототипов~--- но чем только ни пожертвуешь ради сохранения непорочности?

Однако вернёмся к главному герою <<Предупреждения Столлмана>>~--- Google Docs с его <<компактизированным>> кодом JavaScript. Не будучи программистом, не могу судить, насколько такая компактизация делает код не читаемым для разработчика. Замечу только, что в принципе разработчику вольно оптимизировать свой код так, как ему видится наилучшим. А уж писать или не писать комментарии~--- это вообще его сугубо личное дело~--- вроде, никакая лицензия (кроме внутренних побуждений) его к этому не обязывает. Что же до того самого пользователя, который постоянно фигурирует в статье Столлмана, то для него в большинстве случаев равно не читаем будет ни реальный, ни компактный код.

Далее, следует подчеркнуть, что в случае с Google Docs пользователь имеет дело не столько с программой самой по себе, сколько с сервисом: пользователю предоставляется не только (и даже не столько) средство для создания своих документов, но и обеспечивается их хранение на сервере, резервирование, доступ по сети, в том числе и защищённый.

Способен ли пользователь, как бы глубоко он ни разобрался в коде, обеспечить аналогичный сервис? Если он располагает такими возможностями, для него не составит труда нанять команду программистов, которая сочинит для него и аналогичные средства его реализации. После чего он может выступить прямым конкурентом Google в мировом масштабе. Кстати, насколько я знаю, такие конкуренты у него имеются и так.

Наконец, пользователь, прибегающий к такого рода сервисам, так или иначе уже отдаётся во власть проприетаризма: ведь никто ещё не спорил, что их обеспечение, такое, как серверы, линии связи и так далее, всегда являются чьей-то частной собственностью.

Так что никакой практической пользы в <<антирастлительных>> мерах, предложенных Столлменом, я не вижу. С задачей различения свободного и проприетарного кода в web-приложениях рассматриваемого типа прекрасно справляются сами пользователи первого; и если они допускают для себя лично использование не вполне свободных плагинов и онлайновых сервисов~--- значит, оно им зачем-нибудь да нужно. Пользователей же второго этот вопрос нимало не волнует: они и так живут в проприетарном мире.

А вот переделка браузеров~--- ИМХО, момент абсолютно отрицательный. Ибо поведёт к их усложнению, которое чревато ошибками. Особенно в таком деле, как распознавание кода, создавая, скажем, лазейки для троянов.

Ну и, наконец, про дополнительную лицензию и говорить нечего: их и так развелось столько, что в них запутались и пользователи, и разработчики.

Вообще, рассмотренная ситуация напоминает мне историю борьбы с предустановкой OEM Windows на ноутбуки : плюсы либо сомнительны, либо копеечны, а вот минусы вполне реальны и в некоторых случаях могут оказаться существенными. Так что в обоих ситуация хорошо бы задуматься~--- а стоит ли овчинка выделки?

\section{И обратно о лицензиях: 2. По мотивам ESR}
\begin{timeline}Март 25, 2009\end{timeline}
\begin{shadequote}[r]{Владимир Высоцкий}
Ведь жизнь таких, как мы, сама накажет строго \\
Тут мы согласны, скажи, Серёга. 
\end{shadequote}


Не успело утихнуть эхо предупреждения Столлмена, как оно сменилось выступлением Эрика Раймонда, сделанным им на собрании LI LUG (Группы пользователей острова Лонг-Айленд). Видеозапись его можно 
\href{http://www.archive.org/details/LILUG_20090310_ESR}{видеть}
 
\href{http://www.archive.org/details/LILUG_20090310_ESR}{здесь}
, а репортаж о заседании, содержащий краткое, с купюрами, содержание собственно выступления~--- 
\href{http://dotcommie.net/feed/index.php?id=160}{прочитать здесь}
  (разумеется, по английски). Суть же этой, без преувеличения, сенсационной, речи, можно передать четырьмя словами:


\begin{shadequote}{}GPL больше не нужна\end{shadequote}

Услышать такое из уст одного из основоположников движения FOSS, казалось бы, равносильно грому с ясного неба. Однако вспомним, что Раймонд никогда не относился к лицензионным вопросам с пуризмом, свойственным Столлмену. Собственно, с этого он и начал своё выступление:


\begin{shadequote}{}Одна из моих еретических идей заключается в том, что мы придаём лицензиям чрезмерное внимание. И в первую очередь, я думаю, что мы не нуждаемся в <<обязывающих>> лицензиях\dots\end{shadequote}

\dotsто есть, продолжу своими словами, в лицензиях обязывающих открывать исходный код любого производного продукта, содержащего GPL-лицензированный код. И на основе которых можно возбудить преследование за нарушение этих правил. Далее он объясняет свою точку зрения, почему такого рода лицензионные дела не нужны. Попробую передать его объяснение так, как его понимаю я.

Опять же кратко суть объяснений Раймонда можно передать словами эпиграфа: жизнь, то есть рынок, сама накажет тех, кто, используя открытый код в проприетарных программах, не желает делиться своими собственными наработками. То есть, попросту говоря, закрытая модель разработки программного обеспечения в конечном счёте менее эффективна, нежели открытая, и потому выполненные в её рамках продукты рано или поздно проиграют в рыночной конкуренции.

Несколько неожиданно, не правда ли? Ведь расхожее мнение гласит, что, например, лицензии BSD-стиля благоприятствуют возникновению закрытых проприетарных программ, в той или иной мере использующих код FOSS без <<взаимоотдачи>> сообществу. То есть, говоря по простому, даёт возможность кому-либо <<наложить лапу>> на изначальный свободный продукт. Правда, ИМХО, мнение это основано на недоразумении. Некогда я уже \href{http://alv.me/?p=279}{писал}, как в реальности может выглядеть это самое <<наложение лапы>>: открытые компоненты законченного программного решения всё равно останутся открытыми, и любому вольно получить к ним доступ и развивать далее.

Однако вернёмся к обоснованию Раймонда. Собственно, в разных формах эту мысль он высказывал неоднократно: закрывая разработки, основанные на открытом коде, производитель лишается поддержки сообщества, и может рассчитывать только на своих сотрудников~--- тех, кого он в состоянии нанять. Число их может быть больше или меньше, но оно всё равно ограничено. Тогда как число разработчиков открытого софта и его пользователей, обеспечивающих обратную связь, потенциально можно считать безграничным. И анализ динамики последних лет показывает, что чисто количественно чаша весов всё больше смещается в сторону последних. Разработчики открытых продуктов могут выпускать их версии более часто, за счёт обратной связи с пользователями эффективней исправлять ошибки, и так далее. В результате производители проприетарных продуктов, базирующихся на открытых исходниках, оказываются в ловушке собственной жадности.

Возникает резонный вопрос: если порок жадности будет наказан рынком, а добродетель поделиться с сообществом~--- им же вознаграждена, то к чему лицензии типа GPL, которые в сущности призваны делать то же самое? Только, скорее всего, менее эффективно.


\begin{shadequote}{}Именно поэтому я думаю, что мы больше не нуждаемся в GPL или подобных ей <<обязывающих>> лицензиях\end{shadequote}

-- примерно такими словами Раймонд подводит итог этой части своего выступления.

В ходе его продолжения Раймонд высказывает мысль, что в качестве альтернативы GPL можно рассматривать BSD-лицензию. И добавил, что стоимость разработки открытых и частично зарытых решений зависит от размера компании: маленькие компании просто не могут позволить себе штат разработчиков для поддержки значительных объемов кодовой базы, хотя это вполне по силам компаниям большим.

И уж совсем в заключение Раймонд отметил, что, хотя FOSS ныне процветает, это не значит, будто GPL пора ликвидировать:


\begin{shadequote}{}Я не думаю, что время, когда ликвидация GPL будет более полезна, чем вредна, настанет и в будущем.\end{shadequote}

Что можно добавить к его словам? Пожалуй, что ничего. Только попросить прощения у читателей за то, что сенсация, обещанная в начале заметки, в конце её так и не состоялась: Раймонд отнюдь не призвал к отмене <<Хованщины>> с оперных подмостков и поголовному переходу на сою. А лишь объяснил, почему, с его точки зрения, без первой можно обойтись, и почему вторая также вполне пригодна к употреблению.

\section{Кто оплачивает банкет?} 
\begin{timeline}15 июня 2005 г\end{timeline}

\hfill \begin{minipage}[h]{0.45\textwidth}
Мой снобизм~--- он как лучик путеводный,\\
Помогает воспринять судьбу как должно:\\
Мол, художник~--- он обязан быть голодным. \\
Он худой, но гордый, он~--- художник.
\begin{flushright}
\textit{Тимур Шаов}
\end{flushright}
\bigskip\end{minipage}

А кто вообще должен (и должен ли?) финансировать такие родственные области человеческой деятельности, как разработка Open Source и фундаментальную науку? И кто их финансирует в настоящее время?

Должно, однако, для начала заметить, что родственность Open Source и фундаментальной науки~--- не мое изобретение. Блестящее обоснование этого тезиса содержится в статье Николая Безрукова Разработка программ с открытыми исходниками как особый вид научных исследований.

У читателя большей части популярных публикаций на тему Open Source вполне может создастся впечатление, что разработка программ этого класса осуществляется на голом энтузиазме, напоминающем ударников-стахановцев и прочих строителей Магнитки. Что, конечно, очень бла-а-родно, но в наш коммерциализованный век вызывает законное недоверие (особенно с учетом аналогии строителей Магнитки и БАМа~--- забайкальских комсомольцев, сокращенно ЗК).

Проблема финансирования Open Source включает в себя три аспекта:


\begin{enumerate}
	\item кто может быть заинтересован (и может ли?) в финансировании разработки Open Source;
	\item в какой форме заинтересованные стороны могут осуществлять финансирование разработки Open Source;
	\item каким образом финансирование это может распределяться среди разработчиков.
\end{enumerate}


Должен сразу признаться~--- ответов на эти вопросы я не знаю. Однако они~--- совершенно те же, что возникают при финансировании т.~н.~фундаментальной науки. А поскольку в этой сфере мировым научным сообществом накоплен весьма большой, хотя и не всегда положительный, опыт, по аналогии рискну высказать некоторые соображения.

Но сначала зададимся иным вопросом: а нужно ли финансирование разработки программ с открытыми исходниками вообще? Ведь теоретически предполагается, что это~--- предприятие самоокупаемое, а то и просто прибыльное. Однако теоретически, как говаривал дед Щукарь, она лошадь, а практически она падает\dots

Напомню, что открытость исходников некоей программы и даже свобода её распространения отнюдь не подразумевает бесплатности: вспомним бессмертное столлменовское <<свободное слово, а не бесплатное пиво>>. Тем не менее практика такова, что подавляющее большинство открытых и свободных ОС вкупе со своими приложениями распространяется фактически бесплатно. То есть пресловутая интеллектуальная собственность источником дохода служить не может (как это имеет место быть в случае проприетарного софта).

Прибыльность разработки Open Source обычно обосновывается туманными фразами о сопровождении и поддержке программных продуктов, распространяемых бесплатно. В смысле~--- по себестоимости носителей и документации. Когда в отрицание бесплатности открытого софта приводят американские цены на дистрибутивы типа Red Hat или Suse, забывают, что 100- (или даже 200-) долларовые их коробки содержат штук пять-шесть изрядной толщины книжек. А в Америке любая специальная книжка меньше полтинника ихних денег, насколько я знаю, сама по себе не стоит. Если же речь идет о дистрибутивах ценой в несколько тысяч долларов~--- то тут уже в стоимость включено именно сопровождение продукта в явном виде.

Тем не менее, достоверных известий о фирмах или персонах, обогатившихся за счет техподдержки Open Source, у меня нет. Что понятно: люди, покупающие дистрибутив Linux (для примера), имеют целью разобраться в системе. И либо этой цели достигают, и тогда в техподдержке не нуждаются. Либо бросают это занятие, и тогда не нуждаются в поддержке тем более. Если же речь идет о корпоративных пользователях~--- не думаю, что разумный руководитель рискнет перевести весь свой документооборот на Linux или BSD, не имея в штате квалифицированного специалиста по одной из этих систем. Так что мечты о заработке на техподдержке~--- столь же наивны, как и мечты о самоокупаемости фундаментальной науки, блестяще-утопично сформулированные на заре перестройки Максимом Максимовым в статье <<Реанимация>> (Знание-сила, 1989, \textnumero11).


\textit{Маленькое отступление. Впрочем, Максимова прославила другая работа~--- <<На грани~--- и за ней>> в \textnumero3 того же журнала за 1988 г. Именно из нее большинство из нас узнало о Бруно Беттельгейме~--- чуть ли не единственном из великих неофрейдистов XX века, чьи труды остаются не переведенными на русский язык по сей день. Кто читал статью Максимова~--- легко догадаются, почему\dots}


Однако вернемся к теме раздела. Конечно, само по себе издание и распространение дистрибутивов прибыль приносить может. Так же, как её приносит, скажем, издание детективов Марининой и их продажа. Однако к аналогии с книжным бизнесом я обращусь несколько позднее.

А пока: кто заинтересован в развитии Open Source и, соответственно, мог бы финансировать его разработку? Кроме его разработчиков и пользователей, разумеется. Но ведь разработчики Open Source, хотя и заняты своим делом в большой степени из любви к искусству, также хотят есть-пить. А пользователи~--- они ведь пользуют Open Source в значительной мере в виду его бесплатности, практической или теоретической. Что удовлетворению означенной потребности разработчиков отнюдь не способствует\dots

Первая сторона, заинтересованная в развитии Open Source~--- пользователи любого коммерческого софта. Впрочем, не смотря на внешнюю парадоксальность, это очевидно: только конкуренция со стороны открытых и свободных программ может подвигнуть коммерческих разработчиков на совершенствование своей продукции.

Как ни странно, в качестве второй стороны, наиболее заинтересованной в развитии открытого софта, видятся производители коммерческих UNIX-систем (хотя, подозреваю, что сами они не всегда понимают свое счастье). Почему~--- обосновать не трудно.

Молодому человеку, сызмальства привыкшему к Linux (FreeBSD, OpenBSD \textit{etc}.) и пришедшему на службу со школьной (или университетской) скамьи, работать в Windows~--- что серпом по\dots сами знаете чему. Гораздо легче ему будет перейти на Solaris или AIX. А достигнув по службе должного положения, он, безусловно, приложит все усилия к тому, чтобы внедрить POSIX-системы в родном трудовом коллективе~--- ведь, по тем или иным причинам, далеко не всегда можно обойтись свободными их реализациями.

За иллюстрацией этого достаточно спуститься в не столь уж далекое прошлое~--- 1995 год. Когда (не заставшим того времени поверить в это трудно~--- но факт имел место быть) в качестве реальной альтернативы для массовых настольных систем рассматривались OS/2 и Windows~95. Причем ни у кого не вызывало сомнений технологическое превосходство первой. Но: Windows~95 пришла в дома, а OS/2~--- нет (о причинах распространяться здесь неуместно). И через несколько лет на работу вышло поколение, вскормленное и вспоенное в <<окнах>>. Результат не замедлил воспоследовать: кто и когда последний раз видел OS/2 на пользовательском десктопе?

Наконец, третья заинтересованная сторона~--- это государство, причем~--- любое (правда, применительно к нашему государству об этом и не подумаешь). Причины этого также достаточно тривиальны~--- здесь и баланс между монополизацией и свободной конкуренцией, и массовый независимый аудит программной продукции, и снижение себестоимости рабочего места госчиновника: список легко продолжить. И не на последнем месте в нем окажутся соображения государственной безопасности: как бы <<подоконная>> кольчужка коротка не оказалась\dots

Теперь посмотрим, горят ли заинтересованные стороны желанием оказать развитию Open Source посильную финансовую помощь?

Конечно, странно было бы ожидать от массы Windows-пользователей благотворительных пожертвований в FSF и аналогичные dot-org'и (а для последних это нередко~--- существенный источник финансирования, примером тому проект OpenBSD). Тем не менее, наиболее продвинутая часть их косвенно в финансировании Open Source участвует. Хотя бы тем, что проявляет интерес к публикациям на эту тему~--- и бумажным, и сетевым. В результате чего компьютерные издания печатают больше статей соответствующего профиля, регулярно выплачивая авторам гонорары, чем и способствует материальному благополучию писателей-POSIX'ивистов (навроде вашего покорного слуги), а также продолжению их скорьного пропагандистского труда.

Теперь о корпорациях~--- производителях UNIX-машин и разработчиках проприетарных версий UNIX и софта для них. Об инвестициях IBM в Linux-компании знают, наверное, все. Однако это лишь одна сторона дела. Меньшее внимание привлекает то, что, скажем, изрядное количество разработчиков свободного браузера Mozilla (из некоторых источников явствует, что~--- большинство) по совместительству являются штатными сотрудниками AOL (хотя скорее~--- наоборот) и работают над проприетарным (хотя и бесплатным, но не свободным) браузером Netscape. А команды разработчиков свободного офисного пакета OpenOffice и проприетарного~--- StarOffice, суть множества пересекающиеся (не знаю уж, насколько точно, но очень значительно). Ну и недавнее приобретение Suse Linux компанией Novell (до некоторого времени правообладателя торговой марки UNIX)~--- также факт показательный.

И, наконец, государство. О прямом госбюджетном финансировании разработки свободного софта в большинстве стран, как будто бы, слышно не очень много. Кроме, разве что, Китая, где курс на внедрение открытого и свободного софта~--- прямо-таки генеральная линия Партии и Правительства. Проскальзывает информация о господдержке Linux (в сфере наробраза, например) в паре-тройке стран Латинской Америки (Мексика, Бразилия). Ну и есть сведения, что французская компания Mandrakesoft (производитель одноименного дистрибутива Linux, ныне именуемого Mandriva) была некогда выведена из затянувшегося кризиса благодаря правительственным ассигнованиям.

Это~--- с одной стороны, Однако~--- вспомним, что BSD (предок свободных операционок Free-, Net- и OpenBSD) почти полтора десятка лет разрабатывалась в Университете Беркли. На денежки, между прочим, оборонного ведомства США~--- то есть прямого госбюджета.

Или~--- история финского студента Линуса Торвальдса. Который благополучно проучился (а потом и проработал) в университете лет семь, за которые свой Linux и разработал. А ведь высшее образование в Финляндии, насколько мне известно, бесплатное~--- сиречь финансируемое из госбюджета. Так что можно сказать, что Linux был создан за счет финской казны (и финских же налогоплательщиков; также как FreeBSD~--- за счет налогоплательщиков американских).

Все это я говорю к тому, что романтическое представление о создателях свободного софта как об энтузиастах-одиночках (со всеми вытекающими из этого следствиями, как положительными, так и отрицательными) подчас весьма далеки от истины. Многие из них~--- и сотрудники коммерческих фирм, и выходцы из академическо-университетской среды,~--- занимаются этим делом в рамках своих прямых должностных обязанностей, за что и получают зарплату.

Значит ли это, что движение Open Source имеет достаточную финансовую базу? Отнюдь. Государственные программы имеют обыкновение рано или поздно заканчиваться по самым разным причинам. Так, не с падением ли мировой системы социализма связано прекращение финансирования проекта BSD в 1991 году?. Вливания от коммерческих фирм зависят от конъюнктуры рынка. А интерес широких народных масс ко всему, связанному с Linux и Open Source, в значительной мере лишь дань моде.

Так что какие-либо дополнительные источники финансирования не помешали бы движению Open Source. Однако уже упоминавшаяся аналогия между ним и фундаментальной наукой наводит меня на мысль: а пойдут ли они ему на пользу? Ибо второй вопрос после получения финансирования~--- как это финансирование будет распределяться среди собственно разработчиков?

Опять же по аналогии с наукой можно предположить три модели распределения финансов. Первая, наиболее эффективная, модель описана в легенде про Папу Римского (не помню уже кого и которого) и знаменитого художника Титиана. Коему Папа просто подарил дворец вместе с ежегодной рентой, необходимой на его содержание. Дабы у мужика голова по пустякам не болела\dots

Модель эта не столь утопична, как может показаться: по доброй традиции, основанной товарищем Сталиным, примерно по такой схеме осуществлялось финансирование науки в Советском Союзе. Правда, у товарища Сталина была некоторая страховка на случай, если товарищ Титиан начнет манкировать своими обязанностями: всегда можно немножечко расстрелять товарища Титиана. Когда такой возможности не стало, советская наука очень быстро пришла к тому, к чему пришла\dots

К слову сказать~--- модель эта минимум однажды эффективно сработала и в мире открытых исходников. Не напоминает ли вам положение Линуса в компании Transmeta то, в котором оказался Титиан Папскою милостью, или товарищ Курчатов~--- волею гения всех времен и народов?

Вторая модель~--- финансирование проектов в форме грантов, как это принято в отношении научных исследований в т.~н.~цивилизованных странах. А снекоторых пор и Россия приобщилась к этой практике. Так что о достоинствах и недостатках такой схемы распространяться не буду~--- наши <<дети капитана Гранта>> (они же~--- <<Джорджа Сороса птенцы>>) испытали их на собственной шкуре.

Хотя и для этой модели есть примеры удачного использования в истории движения Free Software~--- разработка систем линии BSD 4.X в Университете Беркли. Правда, термина Free Software тогда еще не существовало.

Наконец, третья модель~--- гонорарная, аналогичная принятой в книгоиздательской практике. Я уже говорил, что собственно прибыль при распространении открытого софта могут извлекать издатели и продавцы дистрибутивов. Причем прибыль эта в значительной мере определяется сопровождающей печатной продукцией (документацией). Так почему бы им, в соответствии с Божьей заповедью, не поделиться её толикой с теми, кто обеспечивает им предмет издания и продажи?

История показала, что эта модель, пожалуй, наиболее эффективна. Она прекрасно сработала в случае FreeBSD и Linux Slackware. И тот, и другой проект долгое время развивались при поддержки компании Walnut Creek~--- известного продавца компакт-дисков (а потом и дистрибутивов). Доход от продажи дистрибутивов (плюс разнообразной атрибутики~--- маечек, кружечек \textit{etc}.), насколько мне известно,~--- один из основных источников финансирования проекта OpenBSD. Ну и для компаний, собирающих дистрибутивы Linux (типа Red Hat и Suse), это~--- также некоторое подспорье.

Но~--- не более: ибо мир информационных технологий породил и еще одну модель финансирования собственной деятельности~--- ту самую пресловутую техническую поддержку и поставку готовых решений, о которых я уже говорил ранее. И к которой по ряду причин отношусь несколько скептически. Однако недавний отказ Red Hat от официальной поддержки собственного пользовательского дистрибутива показывает, что заинтересованные стороны моего скептицизма не разделяют. Что ж, Бог им в помощь\dots

Впрочем, в нашей стране затронутые здесь вопросы носят чисто теоретический характер. И ныне разработчикам Open Source (как и научным работникам) следует находить утешение в строках Тимура Шаова, приведенных в качестве эпиграфа этого раздела.

\section{Откуда и куда пошел свободный софт} 
\begin{timeline}2004 г\end{timeline}

Эта заметка навеяна чтением статьи Владимира Попова <<Размышления на тему: заработать на Open Source>>. И потому начну её с цитаты из оной:

\begin{shadequote}{}Критерий рентабельности заведомо не приложим к научной, гуманитарной, медицинской и многим другим сферам деятельности человека (не говоря уж о сфере искусства).\end{shadequote}

Возникает вопрос~--- а приложим ли этот критерий собственно к открытому программному обеспечению? Что особенно актуально в свете все более частых попыток <<Заработать на Open Source>>. Конечно, само слово <<заработать>> в этом контексте имеет двойной смысл~--- но к этому я еще вернусь. А пока все же попробую ответит на свой вопрос. Для чего, как обычно, придется обратиться к истории.

Конечно, понятия Unix-way и Open Source~--- далеко не полностью пересекающиеся множества. Но, тем не менее, на протяжении всей своей истории они были тесно, можно сказать~--- неразрывно,~--- связанными. И потому рассмотрим их одним массивом.

С чего началась дорога, получившая потом название Unix-way? Как всем известно, началась она с проекта Multics. Что это было такое? Это была совместная разработка Bell Labs, General Electric и Массачуссетского технологического института (MIT). То есть~--- вполне академический проект. Где работали основоположники Unix? Работали они в той же Bell Labs, хотя и принадлежащей коммерческой компании, но имевшей целью все же научные разработки. А кто сказал, что научная работа не может финансироваться частным капиталом? Только не те, кому посчастливилось получить Нобелевскую премию.

Дальше~--- больше. Чей вклад в становление Unix в современном его виде был наиболее весом (после его основоположников, конечно же)? Университета Беркли, штат Калифорния. Кем был Ричард Столлмен до того, как он начал свой крестовый поход за освобождение гроба Господня (то есть, пардон, за свободу софта)? Был он, как известно, научным сотрудников в лаборатории искусственного интеллекта в том же MIT. Чем занимался Энди Танненбаум, создатель Minix~--- системы, вдохновившей Линуса Торвальдса на его бренный труд по написанию терминальной программки? А занимался Энди преподаванием а Амстердамском университете. Да и Minix свой написал он, собственно говоря, для того, чтобы обучать скубентов основам Unix.

Наконец, кем был сам Линус Торвальдс в то время, когда его терминальная программа медленно, но верно превращалась в операционную систему? Был он студентом, а потом научным сотрудником университета в Хельсинки. И число таких примеров можно умножить до бесконечности.

Из всего сказанного можно сделать вывод: создание Unix как открытой (в смысле~--- соответствующей открытым стандартам) системы и её производных, представленных открытым и свободным софтом во всех его проявлениях, происходило в значительной мере в сугубо академической среде. А если вспомнить о том, что вся софтверная индустрия базируется, в сущности, на математических алгоритмах, развивавшихся в рамках <<чистой>> математики со времен Евклида и Бируни, становится окончательно ясно: Unix-way и Open Source есть порождение той области человеческой деятельности, которую именуют <<чистой>> или фундаментальной наукой. Впрочем, задолго до меня к тому же выводу пришел Николай Безруков в статье <<Разработка программ с открытыми исходниками как особый вид научных исследований>>.

Таким образом, поставленный ранее вопрос сводится к более общему: а можно ли заработать на занятиях фундаментальной наукой? И тут впору вспомнить о двух смыслах русского слова заработать. Первый~--- это получать некоторую плату (в Кодексе законов о труде при соввласти она так и называлась~--- заработная, хотя ниже мы увидим, что это определение не вполне точно) за свою работу. И второй~--- это извлечение прибыли, то есть предпринимательская деятельность.

Если в наш вопрос подставить первый смысл слова заработать, то ответ на него будет сугубо положительным. В обоснование чего можно привести не только немецких профессоров XIX века (весьма обеспеченных по тем меркам людей), но и судьбу Линуса Торвальдса: ведь в сущности на протяжении нескольких лет он получал свою зарплату в университете именно за то, что разрабатывал Linux.

За чей счет выплачивается зарплата научного работника? Я уже затрагивал этот вопрос в своем сочинении <<Кто оплачивает банкет?>> и потому повторю лишь вкратце: научная работа вообще оплачивается обществом в целом. А уж посредством чего и кого эта оплата осуществляется~--- вопрос другой.

Посредником в оплате научной работы может быть государство в целом. Так, можно сказать, что создание Linux'а финансировалось финским налогоплательщиком посредством правительства своей страны, в которой существует бесплатное образование.

Это могут быть отдельные государственные ведомства. Так, агентство DARPA было не более чем передаточным механизмом между государственным бюджетом США (складывающимся из денег тех же налогоплательщиков~--- но теперь уже американских) и Университетом Беркли.

Это могут быть корпорации~--- ведь компании типа AT\&T или IBM суть не что иное, чем социальные объединения, превосходящие по масштабам несредние даже государства.

Наконец, общество (или отдельные, наиболее прогрессивные, его представители) могут выступать финансистами научных работ и непосредственно~--- в виде пожертвований. Не секрет ведь, что это существенный источник средст к существованию многих проектов Open Source.

За века существования науки как важной сферы человеческой деятельности (и за многие уже десятилетия~--- как массовой профессии) было выработано три формы распределения финансовых источников среди заинтересованных лиц. Это~--- прямая выплата так называемой зарплаты (так называемой~--- потому что суть явлеления лучше передается словами жалование или оклад содержания), финансирование в форме безвозмездных кредитов~--- грантов, и гонорарная оплата результата. Каждая из этих форм имеет свои достоинства и недостатки, обсуждать которые здесь не место (об этом опять же~--- в \href{http://alv.me/?p=294}{сочинении о банкете}). Однако никаких иных механизмов пока не придумано и в мире свободного софта: Линус получал оклад содержания в своем университете, создатели BSD жили за счет грантов мирного американского ведомства, а средства к существованию Столлмена вполне подпадают под понятие гонораров.

Так что зарабатывать на кусок хлеба своей работой на ниве Open Source вполне можно (правда, боюсь, не в нашей стране). А вот может ли быть открытый софт объектом предпринимательской деятельности? То есть~--- служить для извлечения прибыли. Здесь однозначный ответ дать трудно.

По аналогии с научной работой можно видеть, что сама по себе разработка открытого софта никакой прибыли принести не в состоянии~--- что-то я не слышал об акциях компаний по разработке специальной теории относительности или квантовой механики. Прибыль начинается на стадии так называемого внедрения научных разработок в жизнь, то есть когда наука перестает быть чистой наукой и становится скорее технологией. В софтверной индустрии с таким внердением можно сопоставить а) тиражирование, б) обучение и в) сопровождение программной продукции. Как тут обстоит дело с прибыльностью?

Очевидно, что норма прибыли при тиражировании свободных программ стремится к нулю. Ведь себестоимость носителей практически нулевая, сами программы~--- общедоступны, и наварить здесь практически невозможно. Вспомним пример из книги Линуса~--- когда стоимость подвоза воды превышает допустимый пользователями уровень, кто-нибудь обязательно протянет водопровод. Так что единственное, что тут остается~--- это пытаться компенсировать норму прибыли её массой. То есть~--- объемом продаж. А объем этот напрямую связан с количеством пользователей Open Source. Причем~--- в первую очередь пользователей-индивидуалов~--- в силу специфики свободных лицензий компания уже в 100 человек вполне может наладить тиражирование своими силами и в потребных масштабах. Но пользователей-индивидуалов за десктопами~--- отнюдь не большинство, и вовсе не все они~--- поклонники свободных разработок. Так что и места под солнцем для Linux-предпринимателей оказывается не так много.

И тут тиражирование тесно смыкается с обучением~--- поскольку существовать без него не может. А под обучением я понимаю не только (и даже не столько) всякого рода сертифицированные курсы, сколько~--- самую элементарную печатную документацию. Ведь каждому, кто видел в книжном магазине США или Европы научные монографии, становится ясным: цена коробочных дистрибутивов Red Hat или Suse на 90\% состоит из стоимости сопутствующей полиграфии. То есть компании-распространители дистрибутивов~--- это не столько разработчики программного обеспечения, сколько~--- издатели книг и сопутствующей продукции (вспомним Ходжу Насреддина~--- <<дом, сад и принадлежащий им водоем>>). Так что структура их доходов в этой части оказывается такой же, как у любого книжного издательства.

Книжные издательства в мире существуют уже веками, и даже те из них, что занимаются издательством исключительно научной литературы (или по преимуществу ее) особо не бедствуют~--- вспомним Эльсивьер. Однако и здесь для Linux-компаний (назовем их так) не так уж все зд\'{о}рово. Ибо обычные книгоиздатели занимаются тиражированием уникальных, то есть безальтернативных, авторских произведений (будь то научные монографии или детективные романы)~--- или по крайней мере тех, которые воспринимаются обществом в этом качестве:-).

Сопровождающая же софт книжная продукция представляет собой лишь один из многих вариантов получения информации об оном софте~--- тут и штатная документация системы (типа man-страниц), и разного рода онлайновые источники. И с развитием Интернета роль печатной документации все более снижается. Так что единственной побудительной причиной к приобретению коробки с руководством может быть только литературный талант написавшего её автора:-)~--- необходимую сумму заний пользователь легко может получить другими путями. Так что особенно раскатывать губы на тиражирование+обучение также не стоит.

Наконец, поддержка. Традиционные формы поддержки можно разбить на две формы: привычное многим эникейство~--- индивидуальное или в масштабе (рабочей) группы товарищей,~--- и поддержка корпоративная. Первая форма, блестяще зарекомендовавшая себя в мире Windows (и обеспечивающая кусок хлеба со стаканом пива не одному уже поколению эникейщиков), в мире Open Source явно не сработает. Ибо начинающий пользователь-индивидуал, скажем, Linux или а) очень быстро перестанет быть начинающим, и в услугах эникейщика нуждаться не будет, или б) бросит это занятие~--- и тогда ему эникейщик не потребуется тем более, или в) получит в свое распоряжение идеально настроенную под его задачи систему, в которой можно будет, ничего не меняя, работать веками~--- и больше обращаться к эникейщику повода у него не будет (разве что по старой дружбе пивком угостить).

Корпоративная поддержка\dots Да, это то, на чем, в основном, делают деньги и Red Hat, и Suse, пытаются~--- IBM и Hewlett-Packard, возможно~--- кто-то еще. Однако уже ограниченность списка~--- списка по настоящему удачных компаний в этой сфере,~--- свидетельствует о том, что ниша эта не столь уж обширна, как кажется. Предвижу возражение~--- с распространением Linux'а в широких массах простых американских (и прочих) миллиардеров ниша эта будет расширяться. Отнюдь~--- возражу я. Потому что с распространением Linux будет расти и количество специалистов в оном (хотя, как показывает практика.~--- не обязательно их качество), и так назваемую поддержку вполне можно будет осуществить в рамках локального предприятия. А в условиях российской дешивизны рабочей силы фактор этот будет особенно весом.

Это я не к тому, что поддержка Linux~--- дело бесперспективное. Хочу лишь подчеркнуть, что сфера эта столь же ограничена, как и область тиражирования/обучения. А с распространением Linux к тому же будет все менее рентальбельной.

Все ли так мрачно в области коммерческого использования Open Source? Не совсем, потому что есть еще и четвертый способ извлечения прибыли из оного, причем~--- находящийся на грани с зарабатыванием денег в первом смысле этого слова. Однако о нем я планирую поговорить в следующей заметке.

А пока приходится констатировать (с сожалением или нет~--- другой вопрос), что все прошлое, настоящее и, в значительной мере, будущее Open Source связано с академической средой (финансируемой из государственных или частных средств~--- в данном случае не важно) и исследовательскими отделами корпораций~--- тех из них, которые могут позволить себе содержание оных. Иными словами~--- за счет все той же общественной поддержки, непосредственной или опосредованной.

А будущее Linux-компаний в их современной форме видится весьма проблематичным~--- по крайней мере, для новых участников этого бизнеса широкой сферы приложения сил (и, соответственно, извлечения прибыли) не просматривается.

И большинство участников Linux-бизнеса хорошо это понимает. Если внимательно пройтись по Distrowatch 'у, легко увидеть, что в каждой стране существует один, максимум~--- два дистрибутива с <<национальным орнаментом>> и претензиями на как бы коммерческий статус. А многие страны и вообще обходятся дистрибутивами интерациональными.

Исключением, как обычно, оказывается Россия, где собственный дистрибутив не собирает только ленивый. Однако и тут~--- большинство из нас занимается этим главным образом в индивидуальном порядке, из любви к искусству, в целях общего образования, и по достижении достаточного уровня коего существенно охладевает к этому занятию. А сфера традиционной Linux-дистрибуции поделена между командами Altlinux и ASPLinux. Каждая из которых имеет собственную субнишу, целевую аудиторию и сложившийся круг пользователей.

Из всего сказанного, казалось бы, должна следовать исключительная жесткость конкуренции между Linux-компаниями~--- особенно на внутринациональных рынках. Однако это не так: в среде базирующегося на Open Source бизнеса не столкнешься (за буквально единичными исключениями:-)) с таким обливанием грязью конкурентов, какое, по словам видевших это людей, практикуют американские производители стиральных порошков. Почему?

Думаю, что руководители Linux-компаний прекрасно понимают, что само их существование обусловлено не столько <<выхватыванием куска изо рта конкурента>>, сколько~--- той самой общественной поддержкой, о которой я только что говорил. И всякого рода скандальные PR-акции приведут только к дискредитации идеи свободного софта. И, как следствие,~--- к потере этой общественной поддержки.

\section{Как же заработать на Open Sources?} 
\begin{timeline}15 июня 2005 г\end{timeline}

Выполняю свое обещание~--- несколько сгладить мрачно апокалиптическую картину, нарисованную мною в прошлой заметке, относительно перспектив коммерческого использования свободного софта. Но для этого нам придется вернуться к тому, кому и зачем нужен Linux (буду говорить так для краткости, но на самом деле все сказанное относится и к FreeBSD, а частично~--- и к другим BSD-системам).

Однажды на одном из форумов я затеял опрос~--- для чего пользователи переходят на Linux и прочие свободные ОС POSIX-семейства. И, как и ожидалось, смысл большей части ответов можно резюмировать так: чтобы получить надежную и устойчивую систему, идеально заточенную под конкретные задачи данного пользователя.

Другое дело~--- что задачи у всех бывают разные. Большинство обращается к Linux сотоварищи или для разработки софта, или для администрирования. Или~--- для того, чтобы в домашних условиях учиться тому или другому занятию. Некоторые авторы полагают, что только для этих целей свободные POSIX-системы и пригодны. Но при этом забывают еще об одной категории, самой многочисленной~--- так называемых простых пользователях. А у них задачи~--- еще более разнообразны. Для кого-то компьютер~--- это гейм-станция, для иного~--- музыкальный центр. А некоторые, как это ни странно, выполняют на компьютере свою непосредственную работу. Обычно никак с компьютерами не связанную. И если перспективы Linux в области игр или мультимедиа не вполне ясны, то как рабочий инструмент для очень многих и многих он оказывается оптимальным.

Будучи, наверное, одним из первых в истории Руси простых (то есть профессионально не связанных ни с разработкой, ни с администрированием) пользователей, целиком перешедших на Open Sources, не могу отказать себе в удовольствии поделиться собственными впечатлениями. Так вот, для меня Linux, а потом и FreeBSD, оказались идеальной средой для работы с текстами~--- не исходными, а обычными, написанными по преимуществу на русском языке. Средой для сочинения текстов новых, их оформления и распространения. А главное~--- для обработки и использования сочиненного ранее~--- а за свою жизнь насочинял я немало.

Не гордыни ради, а токмо представления масштабов для: около ста статей, полторы монографии, отчетов и проектов разного рода~--- несть числа, и это~--- только по геологии; а уж околокомпьютерных сочинений сколько~--- уже и не упомню. Так вот, лишь в средствах шелла и сопутствующих утилитах нашел я методы эффективной борьбы со всем этим изобилием~--- нахождения нужных текстов, извлечения из них фрагментов, компиляции или, напротив, расчленения, и так далее.

И такие задачи возникают перед многими. В годы, предшествовавшие приобщению к POSIX'ивизму (и превращению в чукчу-писателя), одним из источников средств к существованию для меня было эникейство в индивидуальном порядке. И практически все мои клиенты приобретали компьютеры с одной целью~--- выполнять дома ту работу, ради которой ранее им приходилось ходить на службу (а возможность слушать музыку или там в игрушки поиграть~--- рассматривалась в качестве необязательной опции).

А были среди них люди самых разных профессий~--- научные работники и переводчики, редакторы и бухгалтера, даже одна поэтесса и одна певица, сочинявшая собственные аранжировки (или как это называется~--- именно в этом случае я, увы, оказался совершенно некомпетентен). И, конечно, требования к компьютеру (вернее, наличному на нем софту) у них были разные. Но задача обработки текстов (а для финансовой сферы~--- еще и цифр) стояла перед всеми.

Так вот, практически для всех для них (кроме певицы, вероятно) Linux был бы ничуть не худшим решением, нежели DOS/Windows, а для некоторых~--- просто лучшим (особенно если бы за софт пришлось платить всамделишние деньги). При одном условии~--- если бы кто-нибудь его им поставил и заточил адекватно задачам~--- ведь задачи же, как мы уже выяснили, были разными.

И тут впору вспомнить, что любой дистрибутив Linux или BSD-системы представляет собой лишь полуфабрикат готовой индивидуальной системы для практического использования. И любой из них может быть превращен в закаленное и отточенное орудие для выполнения данной задачи, имеющее все необходимые функции~--- и ни одной лишней. Другое дело, что с FreeBSD на это затрачивается часик-другой, с Source Based Linux (типа Gentoo или Archlinux) придется повозиться сутки. А с пакетными дистрибутивами результата не будет, пока не оставишь от них камня на камне (или~--- rpm'а на rpm'е:-)).

Реально ли это? Мой личный опыт (уже не эникейщика, а POSIX'ивиста) показывает~--- более чем. Может ли это выполнить каждый отдельно взятый пользователь (не имеющий, напомню, специального образования и специфических навыков) на отдельно стоящем персональном компьютере? Теоретически~--- ну конечно же, может. Для этого только что и нужно, как прочитать несколько толстых книжек по Linux и, особенно, UNIX (скажем, <<Секреты UNIX>> Джеймса Армстронга, объемом в какие-нибудь тысячу с небольшим страниц), пару сотенок мануальников и HOW-TO'ёв (с побочной пользой~--- практикой в английском:-)), научиться сочинять простенькие шелл-скрипты и макросы для текстового редактора, ну и освоить еще несколько мелких дел.

Правда, возникает вопрос~--- а когда этот пользователь будет заниматься своей непосредственной работой? Ну, это~--- его личное дело (>>Меньше спите. Или больше работайте>>~--- как сказал персонаж из <<Территории>> Олега Куваева). Главная загвоздка тут в другом: в один далеко не прекрасный день такой пользователь с ужасом обнаруживает, что копаться в конфигах или разбираться с опциями компилятора для него стало интереснее, чем переводить контракты, подводить квартальные балансы или даже вычислять P/T-условия выплавления базальтовых магм.

И в этот миг на свете станет еще одним переводчиком, бухгалтером, геологом меньше~--- и одним POSIX'ивистом больше. Что, конечно, прекрасно~--- да вот только если так пойдет дело~--- кто же будет хотя бы начислять им зарплату? Да и за что~--- ведь вся их деятельность превратится в процесс обеспечения самих себя новыми задачами по настройке и совершенствованию собственной системы\dots

И вот тут наступает Час POSIX'ивиста. Именно его может позвать наш бухгалтер/переводчик/геолог для оборудования своего рабочего места. Которое будет включать в себя не просто установку системы, а полный комплекс по её обработке (рашпилем там, или алмазным надфилем~--- это уже зависит от задач и субстрата, сиречь исходного дистрибутива,~--- но к этому мы еще вернемся). Причем в той степени, какая потребуется, чтобы избавить пользователя от необходимости приобретать хоть какие-то знания о внутреннем устройстве системы.

Возникает вопрос~--- а можно ли это выполнить в рамках UNIX-подобной системы? Ведь традиционно считается, что пользователь Linux должен читать горы мануальников, разбираться в правах доступа и т.д. (см. перечисленное выше). Отвечаю: именно в POSIX-системе такое возможно. Потому что обычно индивидуальный пользователь ее~--- не просто пользователь, но и сам себе админ. То есть он вынужден устанавливать и настраивать систему, устанавливать и обновлять прикладной софт, и так далее.

Здесь же речь идет о создании некоего комплекса, того, что именовалось на заре советской компьютеризации Автоматизированным Рабочим Местом~--- АРМом бухгалтера, переводчика, геолога. То есть~--- монофункциональной системы с сознательно урезанными до необходимого уровня возможностями. Пользователь такой системы не должен в сущности даже иметь root-аккаунта: все, что от него требуется~--- это уметь включить питание, элементарным нажатием двух-трех клавиш запустить пару-тройку (или пару-тройку десятков~--- сколько требуется для выполнения произваодственной задачи) приложений или утилит с требуемыми опциями (а создание скриптов, обеспечивающих такую возможность~--- одна из задач нашего POSIX'ивиста) и выйти с сохранением данных и корректным завершением сеанса; например, по нажатию сакраментальной комбинации из трех пальцев, а уж о корректности всего остального должен позаботиться POSIX'ивист.

И устанавливать программы такому пользователю не придется. Весь комплекс необходимого для его задач софта будет установлен единовременно. Обновления? А нужны ли они, если комплекс этот будет тщательно продуман изначально, подогнан как под задачи, так и под железо? Ведь классические программы в стиле UNIX way меняются мало (в смысле качества~--- давно уже лучше некуда, а в смысле функциональности~--- в том-то и суть UNIX way, чтобы не прикручивать к утилите find функции заварки кофе). По настоящему (не удовлетворения любопытства для) необходимость в обновлениях связана а) с обеспечением безопасности и б) появлением нового оборудования, не поддерживаемого софтом. Но в данном случае ни то, ни другое не актуально: root'овых прав у пользователя нет вообще, а оборудование в таком АРМе не меняется до полной физической амортизации.

А, вообще говоря, все описанное в предыдущих абзацах,~--- не более чем изобретение велосипеда. Именно по такой схеме начиналась всеобщая PC'фикация все страны (тогда еще~--- Советов). То есть: IBM PC/XT с <<черным>> DOS'ом (необходимости в NC <<по делу>>~--- не возникает) и программой бухучета (или там учета кадров), запускаемой batch-файлом, вызываемым нажатием Any Key:-). Правда, реализация этого, как правило, оставляла желать лучшего, но речь сейчас~--- об идее. И представьте, как это может быть реализовано на базе современного <<железа>>~--- раз, и с полной возможностью лишить пользователя возможности (пардон за тавтологию) совершить потенциально опасное действие в принципе~--- два.

Конечно, возникает вопрос. Создание такой системы (при качестве реализации выше среднего уровня)~--- весьма кропотливая работа. Её, в сущности, можно сравнить с ружьем штучного разбора (да еще и с ручной высокохудожественной гравировкой). И много ли заработает наш POSIX'ивист-индивидуал на столь же индивидуальных пользователях?

Скорее всего, не очень. Потому что вопрос этот адресован не ко мне. И упирается в повышение материального благосостояния советского (пардон, российского) гражданина~--- а тут уже UNIX way бессилен. Однако\dots

Однако все сказанное относится не только к обеспечению трудящегося-индивидуала. А имеет силу и для любого трудового коллектива~--- будь то частная фирма или госпредприятие. И даже, я бы сказал, большую силу. Потому что функционально ограниченные АРМы (на которых, в частности, невозможно резаться в tetris или linе, смотреть порнографию по Сети и заниматься прочими увлекательными занятиями в рабочее время) востребованы скорее в служебной, нежели домашней, обстановке. А тут уже:


\begin{itemize}
	\item совершенно другие масштабы~--- это понятно; 
	\item соврешенно другие объемы кропотливой ручной работы~--- ведь АРМы для ста банковских операционисток, выполняющих одинаковую работу, будут практически идентичными, и достаточно отрихтовать руками один экземпляр~--- а дальше просто тиражировать его;
	\item совершенно иные условия работы~--- ведь все эти сто АРМов можно единовременно инсталлировать по локалке;
	\item и, как следствие совершенно иные соотношения трудозатрат/трудоооплат.
\end{itemize}


Впрочем, для последних более существенна проблема спроса~--- есть ли он? Ну, во-первых, отсутствие спроса в настоящее время прямо связано с отсутствием предложения (часто ли в советских магазинах спрашивали черную икру?~--- спросом не пользовалась\dots). А во-вторых, даже не смотря на отсутствие предложения, спрос есть (свидетельством чему~--- <<Заметки тетки-бухгалтера>>, опубликованные еще пару-тройку лет назад в онлайновой Компьютерре).

Предвижу возражение и с другого фланга: а не является ли идея таких АРМов профанацией идеи свободного софта? Одним из краеугольных камней которого, как известно, является свобода выбора. В какой-то мере~--- да, но по сути~--- нет. Потому что начинающий пользователь свободной POSIX-системы все равно свободы выбора не имеет: он просто в силу отсутствия знаний не в силах выбрать адекватный почтовый клиент или текстовый редактор из того легиона программ, который лежит на полудюжине сидюков любого т.~н.~user-ориентированного дистрибутива Linux. Свободу эту он получит только тогда, когда изучит их все, то есть превратится в POSIX'ивиста,~--- но этот вопрос мы уже обсудили, не так ли?

Тем не менее, у него есть свобода выбора другого~--- заниматься всем этим самому или предоставить это дело тому, кто может это осуществить по уровню знаний и должностной инструкции, то есть тому же нашему POSIX'ивисту. А вот у последнего эта самая свобода выбора сохраняется в полном объеме~--- и, более того, он имеет не только право, но и реальную возможность ею воспользоваться.

Более того, рискну высказать крамольную с точки зрения абстрактного свободолюбия мысль: существует немало сфер человеческой деятельности, где свобода выбора не только не полезна, но и просто противопоказана. И пример с банковскими операционистками тут далеко не единственный\dots

Итак, <<кратко резюмирую сегодняшний базар>>:

Будущее коммерческого использования свободных POSIX-систем~--- не в дистрибуции Linux на десктоп каждой секретарши на смену Windows-десктопу, дабы она могла им управлять, как кухарка~--- государством. А в создании специализированных монофункциональных систем на базе некоего дистрибутива общего назначения.


Я хотел еще тут поговорить о том, каким должен быть дистрибутив, выбираемый в качестве базы для наших гипотетических АРМов, и какими качествами должен обладать их создатель. Но понял, что эта тема столь широка, многогранна и необъятна, что являет собой совсем другую историю.

\section{Вот и вспомнил я старые истины\dots}
\begin{timeline}Август 15, 2012\end{timeline}

В заглавие заметки вынесены слова из песни Бориса Алмазова~--- очень мало и незаслуженно мало известного автора и исполнителя, одного из тех первых, кого когда-то безграмотно окрестили бардами. И текст, и исполнение можно найти в Сети.  А я продолжу цитату:

\begin{shadequote}{}Вот и вспомнил я старые истины Пожар, мол, не страшен нищему Одна голова не больна А если больна~--- всё одна.\end{shadequote}

А начал я всё это с того, что есть вечные истины. Что нельзя заработать денег на образовании. На медицинском обслуживании. На фундаментальной науке. В конечном счёте~--- нельзя заработать их и на открытом и свободном софте.  Всё это~--- общее достояние человечества. И дотироваться должно~--- человечеством же. Иначе только опять придётся Визбор Иосича вспомнить~--- про руины великих пожарищ и питекантропа с копьём.  Хотя, похоже, что питекантропы, <<умные и красивые>> (с) Алиса Деева, понимали эти истины куды лучше многих людей современного физиологического типа~--- иначе не выжили бы. А они выжили~--- и стали предками многих из нас.

Ну так вот, врач, поклявшийся именем товарища Гиппократа (это ведь типа воиской присяги), не может не оказать помощь тому, кто в ней нуждается. Учитель, взявшийся учить, не может не учить детей. Научный работник не может не заниматься наукой~--- просто потому, что мозги так у него устроены. И никто из них прибыли извлекать из своей работы не может: иначе мы имеем то, что имеем. Когда врачи не лечат, а калечат, педагоги не учат, а мучат\dots Ну а научные работники умотали в те страны, где наука давно стала разновидностью шоу-бизнеса.

А вот из результатов их работы вполне можно извлечь прибыль. Но извлекать её будут совсем другие люди. С другим устройством мозгов (в терминах <<хорошо>> и <<плохо>> не оцениваем).

И в нормально устроенном обществе они будут её извлекать. И будут платить налоги. Правительству~--- а в конечном счёте обществу. Тому самому, которое, в лице лучших своих представителей, придумает, как дотировать тех, кто никакой прибыли не извлекает. Но от которых в конечном счёте зависит прибыль всех других.

Казалось бы, очевидные вещи, понятные каждому, да?

А вот почему все делают вид, что их не понимают?

Почему RMS призывает разработчиков свободного софта зарабатывать на всякой фигне типа саппорта?  Почему Тео де Раадт зарабатывает деньги на свой проект продажей кружек и маечек? Почему, в конце концов, ребята, реально чего-то делающие, вешают у себя на сайтах номера своих вебманевых кошельков?

\section{Копирайт и научно-технический прогресс: три стороны одной медали}
\begin{timeline}Август 30, 2010\end{timeline}
Случайно наткнулся на такую вот статью: <<Отсутствие копирайта способствовало технологической революции в Германии 19 века>>, представляющую собой перевод статьи Франка Тадеуша (Frank Thadeusz) <<The Real Reason for Germany's Industrial Expansion?>> из онлайнового <<Шпигеля>>.

Статья не бесспорна, но интересна и вызывает разные мысли, частью из которых я и хотел бы здесь поделиться. Для начала остановлюсь как раз на спорных положениях.

Прямые сопоставления Германии и Англии середины позапрошлого века с точки зрения количества <<печатной продукции>>, в частности, научной, представляются не правомерными. Не будем забывать, что в Англии (точнее, в Великобритании вообще~--- Шотландии это касалось не в меньшей степени) профессорствование в университете было <<джентльменским>> занятием. А поскольку истинный джентльмен чужд всякого меркантилизма, оплачивалось это дело крайне плохо.

И потому профессора, не отягощенные собственным  состоянием, в свободное от преподавания время занимались совсем другими, нежели написание книг, делами. Например, в 19-м веке подавляющее большинство профессоров Оксфорда и Кембриджа принимало сан~--- это давало ощутимую прибавку к жалованию. Ну а о частных уроках и тому подобных занятиях и говорить не приходится~--- достаточно вспомнить биографии Льюиса Кэролла или Толкиена.

Кстати, и основной целью системы университетского образования Великобритании было в первую голову воспитание джентльменов, а отнюдь не воспроизведение научных работников или подготовка инженеров для промышленности.

В Германии (опять же, точнее, в германоязычных странах) преподавание и в университетах, и даже в сельских школах рассматривалось как государственная служба (возможно, даже служение). И профессора ведущих университетов обладали не меньшим общественным престижем, нежели высокие чины администрации или военные. Что сопровождалось получением соответствующего жалования. Так что свободное время они вполне могли посвящать сочинению не только чисто научных трудов, но и научно-популярных работ <<для народа>>.

Напрашивается аналогия с русской классической литературой: наличие поместья, достаточного количества крепостных и не очень вороватого управляющего создавало предпосылки для занятий изящной словесностью~--- с соответствующим результатом.

Далее, в Англии не только беллетристика, но и научные работы издавались в значительной мере силами профессиональных издателей, то есть самых обычных бизнесменов, озабоченных в первую очередь тем, чтобы <<отбить бабло>>. В Германии научные книги, те самые, премиум-издания, о которых говорится в обсуждаемой статье, издавались в основном университетами, в которых работали их авторы. Это обеспечивало престиж университету и автору, а иногда даже приличные гонорары последнему. Так что дальнейшее <<распирачивание>> ни тех, ни других уже особенно не волновало. Более того, было фактором положительным, способствуя известности автора (а вместе с ним и его университета) в широких массах.

В общем, можно констатировать, что жесткая копирайтная модель, о которой говорится применительно к Англии, отнюдь не способствовала научно-техническому прогрессу. Но и в Германии таковой осуществлялся не благодаря отсутствию копирайта, а в силу совершенно других факторов.

Однако, проецируя германскую ситуацию на современность, можно разглядеть и третью сторону копирастической медали. Которая явочным порядком ныне реализуется благодаря Интернету. То есть книга, изданная первично в бумажном исполнении на гонорарной основе (хотя нынче речь обычно идёт не о гонорарах, а об отчислениях с продаж), в случае проявленного к ней интереса широко тиражируется на сайтах и блогах, в самых разных форматах, от отсканированного DjVu до PDF. Причём что интересно: такое, вроде бы <<пиратское>>, тиражирование вовсе не вызывает снижения продаж <<бумажного>> издания. Скорее наоборот: читатель, хотя бы приблизительно ознакомившийся с содержанием книги по полуслепому скану, купит её с большей вероятностью, нежели если бы покупал кота в мешке.

Это не голословное утверждение, и проверено на собственном опыте. Так, моя книжка про FreeBSD, которая в виде DjVu- и прочих файлов годами валялась  на десятках, если не сотнях, ресурсв, была, тем не менее, продана фантастическим для такой тематики тиражом. И допечатки тиража, собственно, прекратились только полного устаревания её контента. Совпавшим, кстати, с пресловутым кризисом, вообще сильно ударившим по продажам толстых бумажных книг.

\section{Копирастёж vs антикопирастёж: монета обрастает гранями}
\begin{timeline}Сентябрь 16, 2010\end{timeline}

\hfill \begin{minipage}[h]{0.45\textwidth}
Вроде не пиратили \\
И могли бы жить \\
Им бы копирайты \\
Взять~--- и отменить
\bigskip\end{minipage}

В предыдущей заметке я говорил о трёх сторонах медали, выявленных в модели копирайта. Однако с тех пор она стала стремительно обрастать дополнительными гранями, превращаясь в какой-то ромбододекаэдр. Грани эти содержат вопросы:

\begin{itemize}
	\item должна ли информация распространяться свободно? 
	\item должны ли создатели контента и его распространители получать за это деньги? 
	\item и если должны, то как? 
	\item эффективна ли нынешняя копирайтная модель? 
	\item нужны ли в наш цифровой век какие-либо законы о копирайте вообще? 
	\item и если нужны, то какие?
\end{itemize}

Это основные вопросы, которые обрастают серией дополнительных. Например, что первично:

\begin{itemize}
	\item паршивость нынешних схем распространения контента или поганость большей части контента, распространяемого по этим схемам? 
	\item можно ли обеспечить сочетание легальности, эффективности и справедливости при онлайновом распространении контента (вопрос, что такое справедливость, пока замнём для ясности)? 
	\item и какими методами должно это осуществляться~--- законодательно-административными или технологическими? 
\end{itemize}

Сформулировав положения, которые показались мне первоочередными, я понял, что рассмотреть их все в рамках одной заметки мне слабо. Поэтому буду делать это постепенно, по мере сил и возможностей~--- уверен, что при этом возникнут и другие дополнительные вопросы. Но \href{http://alv.me/?p=1174}{сначала посмотрим}, что такое копирайт, вокруг которого и пойдёт весь сыр-бор. И что такое~--- его противоположность.

\section{Копирастёж vs антикопирастёж: что это?}
\begin{timeline}Сентябрь 16, 2010\end{timeline}

Копирайт в обыденной жизни в последнее время рассматривают как синоним понятия авторского права. Так ли это?

Для начала вспомним, что существуют неимущественные и имущественные авторские права. Первые (право считаться автором, право на имя~--- как ни странно, это не совсем одно и то же, право на публикацию или её отзыв, право на защиту контента, в том числе его названия, от искажений) возникают в силу создания автором этого контента и являются неотчуждаемыми, то есть принадлежат автору во веки веков. Нарушение их называют плагиатом: обычно он выражается в приписывании себе авторства контента или включении чужого контента в собственный без указания настоящего автора.

Имущественные авторские права~--- это в первую очередь право на распространение контента, в том числе исключительное. От рождения оно также принадлежит автору, однако может быть передано другому лицу или любому предприятию на условиях, определяемых авторским договором.

Совокупность неимущественных и имущественных авторских прав символизируется знаком охраны авторского права~--- пресловутой латинской c в кружочке. При этом, как ни парадоксально, на самом деле этот знак ни от чего не охраняет, а только указывает лицо, группу лиц или предприятие, которым принадлежат какие-либо из авторских прав на контент, размещённый на маркированном им носителем.

По английски знак охраны авторских прав называется copyright~--- и именно в данном смысле это слово первоначально вошло в русский типографский жаргон.

Однако основное значение данного английского слова~--- это право на копирование. То есть близкое к тому, что у нас называется имущественными авторскими правами. Каковые, принадлежа изначально автору, очень редко за ним остаются: лишь немногие авторы занимались самоизданием, и ещё меньше~--- в этом деле преуспело. Почему разумные люди и передают право на копирование и распространение людям, у которых это получается лучше. В случае книг это будут издатели, в случае всякого рода медиа-продукции~--- какие-нибудь антрепренёры или там продюсеры.

Впрочем, в медиа-бизнесе я не копенгаген, поэтому дальше речь пойдёт в основном о книгоиздании. При привлечении примеров из смежных сфер (в том числе и софтверной индустрии) это будет оговариваться особо.

В общем случае владельцы имущественных авторских прав так и называются~--- правообладатели. Вот к ним и был привязан термин копирайт уже в совсем обиходной трактовке. Разумеется, в народе их стали звать ласковым словом копирасты, а их деятельность~--- копирастёжем. В их число попадают медиа-магнаты, книгоиздатели и проприетарщики от софта. Впрочем, ввиду контаминации понятий авторского права и копирайта эти слова часто стали применять и к собственно авторам, то есть создателям контента, подпадающего под действие копирайта. Хотя некоторые, как мы увидим ниже, полагают их скорее жертвами копирастёжа.

Разумеется, достаточно быстро появилась и оппозиция. Первой ласточкой тут стало движение FreeSoftware, исповедовавшее принцип copyleft. Впрочем, его влияние сначала ограничивалось почти исключительно областью разработки программ, поначалу оказывая мало влияния на другие сферы человеческой деятельности, и потому пока о нём говориться не будет. Тем более, что на эту тему достаточно сказано в более иных материалах, в том числе и автора этих строк.

А вот более массовый класс-антагонист, охвативший области издательской и медиа-деятельности, возник в лице антикопирастов. Их основным занятием стала борьба за полную отмену копирастёжа, либо, хотя бы, за изменение модели копирайта. Правда, каким образом эту модель следует менять, как правило, не уточняется~--- но к этому вопросу нам придётся возвращаться ещё неоднократно в последующих заметках.

Как и всякое оппозиционное движение, антикопирастёж быстро разделился на два уклона, которые можно условно назвать правым и левым (к взаимоотношениям copyright'а и copyleft'а они имеют некоторое отношение, хотя и косвенное).

Представители первого уклона полагают, что от копирастёжа страдают в первую очередь авторы контента. И именно их интересы надо защищать от злобных копирастов. Для чего следует сильно модернизировать механизмы копирайта, но сам институт всё-таки сохранить (почему я условно и назвал их правыми). Как и все правые, конкретной программы для этого они, однако, не предлагают.

Левые уклонисты, напротив, причисляют авторов к кулакам-мироедам, стоящим на службе копирастов, и не желающих даром отдавать своё добро (то есть авторские права любого рода) народу: потребителям книжной, медиа- и софтверной продукции. Как и все левые, они имеют чёткую, простую и понятную программу действий: копирайт~--- зло, поэтому его надо отменить напрочь, а весь контент сделать свободным во всех отношениях (каких именно~--- рассмотрим в следующей заметке).

А недавно возникло и крайне-радикальное крыло, которое можно сопоставить с одним из течений анархистов, хотя его представители называют себя Пиратской партией. Их программа ещё проще и понятней


\begin{shadequote}{}Будьте счастливы, копируйте все!\end{shadequote}

Такова примерно расстановка сил на копирастическом фронте сегодняшенего дня.

\section{Сравнение мужей: журналист vs блоггер.}
\begin{timeline}Май, 2012\end{timeline}

Данный материал в наибольшей степени соответствует  тематике этнографического исследования, ибо в ней будут сравниваться не некие программы, а группы вполне конкретных людей. А уж что из этого сравнения получится~--- я и сам пока не знаю.

\subsection{Вступление}
Для начала~--- о жанре. Это~--- \textbf{блогометка}. Я решил реанимировать этот термин, некогда изобретённый мной в далёкие те годы для одного из проектов. На копирайт не претендую~--- термин настолько напрашивающийся, что странно было бы, если кто-то ещё не изобрёл его независимо и, возможно, раньше меня. В скобках замечу, что большинство написанных мной материалов принадлежат именно к данному жанру. 

Поводом для этой блогометки послужила блогометка другая~--- моего давнего товарища и коллеги Сергея Голубева <<Журналисты и блогеры>>. В которой, в свою очередь, рассматриваются соображения Андрея Колесова о различии журналистики и блоггерства. Именно на этом акцентируется авторский заголовок: \textbf{<<Публикации и блоги~--- в чем разница?>>}. Однако Сергей назвал эту


\begin{shadequote}{}<<подстраницу>> о различии журналиста и блогера\end{shadequote}

иначе. Не знаю, умышленно он это сделал, или так вышло случайно, но в этих немногих словах ему удалось уцепить суть вопроса.

Ибо речь идёт не различии журналистики и блоггерства как сфер деятельности, а о <<сравнении мужей>>, ими занимающихся. Да простят меня дамы, занимающиеся одним из этих занятий, но это просто этнографический термин. Отнюдь не мешающий прекрасной половине человечества участвовать в этом сравнении.

Так вот, я начал было писать комментарий к блогометке Сергея, но быстро понял, что он разрастается в отдельный материал. Который и предлагается вниманию читателей.

\subsection{Цитаты и комментарии}
Дав только что вводную установку, должен сразу определить свою позицию: в данном вопросе я не последовал завету Шарикова~--- не соглашаться с обоими. Хотя с мнением Андрея Колесова я действительно не согласен категорически. А с мнением Сергея~--- согласен более или менее. Хотя большинство его соображений я высказал бы в более категорической форме. Чем сейчас и займусь, чередуя цитаты из обоих <<исходников>> и свои соображения.

Итак, слово Андрею Колесову (далее~--- \textit{АК}):


\begin{shadequote}{}Для многих людей, блоги~--- это возможность просто высказаться. Журналист и без того, постоянно пишет.\end{shadequote}

Комментарий Сергея Голубева (далее~--- \textit{СГ}):


\begin{shadequote}{}\dotsблоги, в которых нет ничего, кроме самовыражения автора, долго не живут.
\end{shadequote}

Вообще-то, любой автор начинает писать исключительно ради самовыражения. Если он самовыражается так, что его выражения интересны не только ему~--- его читают. Если нет~--- не читают. Для некоторых авторов это превращается в профессию~--- то есть основной источник средств к существованию. При этом прямой корреляцией между <<профориентацией>> и интересностью с читаемостью тут нет. И это, как мы увидим дальше, не значит, что блоггер становится журналистом.


\begin{shadequote}[r]{АК}
Самореклама журналисту не очень нужна\dots
\end{shadequote}


\begin{shadequote}[r]{СГ}
Опять же, все не так линейно.
\end{shadequote}

Настолько нелинейно, что с точностью <<до наоборот>>. Журналиста читают постольку, поскольку он представляет некое издание. Это имело место ещё в бумажные времена, когда в ходу был оборот: <<А ты читал в газете \textit{``Правда''} про\dots?>> Правда, возможно, он возник от того, что никто в точности не знал, кто же на самом деле написал эту статью в газете \textit{``Правда''}\dots

В сетевых же СМИ это выражено ещё более ярко. Обычно говорят: <<на сайте таком-то было написано про\dots>>. И очень мало кто знает имя или хотя бы ник того, кем это было написано.

А поскольку журналисты~--- люди, более того~--- авторы, и ничто человеческое им не чуждо (в том числе и авторское честолюбие, в котором нет ничего худого), они вынуждены заниматься самопеарастёжем~--- просто чтобы не затеряться на фоне своего издания. Одни делают это активней, другие~--- пассивней, но делают почти все. И мера активности зависит не от личного честолюбия, а скорее от системы приоритетов и условий оплаты: что важнее~--- слава или деньги? Кстати, между славой и деньгами корреляция тоже далеко не прямая.

Блоггера же читают таким, каков он есть. Или не читают. Но если уж читают~--- это сам по себе факт признания его лично, а не представляемого им издания. И особых усилий к самопеарастёжу ему уже не требуется. К <<продвижению>> своего блога~--- да, но это совсем другая история.

То есть сказанное совсем не значит, что блоггеры самопеарастёжем не занимаются. Занимаются, и ещё как. Но для них это подчинено главной цели~--- продвижению ресурса. Тогда как известное СМИ и так достаточно <<продвинуто>>, и для журналиста оно~--- в том числе и способ <<продвинуть>> себя лично. Хотя бы для строчки в резюме: <<Работал для таких-то изданий>>.

\begin{shadequote}[r]{АК}
\dotsон и без того публикуется, причем в более популярных местах.
\end{shadequote}


\begin{shadequote}[r]{СГ}
Есть блоги, которые у целевой аудитории популярней некоторых <<профильных>> СМИ. Причем, без всякой рекламы, участия в мероприятиях и п.т.
\end{shadequote}

И тут я не могу не согласиться с мнением Сергея: среди людей, читающих материалы <<по делу>>, блоги и личные сайты (границу между ними провести невозможно) обычно куда более востребованы, чем <<бумажные>> СМИ и даже их электронные версии. Более того, большинство ныне популярных <<безбумажных>> порталов, которые уже можно отнести к категории СМИ (вне зависимости от того, имеют ли они этот статус официально), выросли из <<хомяков>> 90-х годов~--- предтечи нынешних блогов.

\begin{shadequote}[r]{АК}
Блоги~--- это <<потеря времени>>. Это~--- время, когда ты не пишешь статьи, т.е. не зарабатываешь деньги.
\end{shadequote}


\begin{shadequote}[r]{СГ}
Вообще-то заработки некоторых блогеров ничуть не уступают заработкам журналистов. Полагаю, такой блогер скажет, что писать статьи~--- это потеря времени.
\end{shadequote}

И опять-таки должен присоединиться к мнению Сергея. С той только банальной оговоркой, что тут вовсе <<не в деньгах счастье>>. Ибо:

\begin{itemize}
	\item статья для любого издания должна подчиняться его правилам, хотя бы чисто оформительским (врезки, картинки \textit{etc}.), тогда как у пишущего для себя руки полностью развязаны; 
	\item автор, публикующийся в любом издании, должен вписываться в некий объём, ограниченный как сверху, так и, как ни странно, снизу; и добро бы речь шла только о первой границе~--- поставить предел растеканию мыслию по древу можно, хотя и трудно (ибо плохо отражается на презренных килознаках); но обратная ситуация~--- когда тема исчерпана, а до конца полосы ещё тысяча символов~--- куда мучительней; и, к слову, мне известны издания, которые практикуют оплату не за килознаки, а за факт раскрытия темы~--- но они как раз и выросли из <<хомяков>> и блогов; 
	\item самое главное, что написанное для издания увидят только его читатели (а в случае издания бумажного~--- когда материал, возможно, потеряет актуальность); написанное для блога увидят все, кто интересуется темой; а это, даже для тематических изданий, множества весьма слабо пересекающиеся. 
\end{itemize}

В один прекрасный день я осознал это (на самом деле этот список можно было бы продолжать ещё долго)~--- и (почти) перестал писать для изданий.

А возвращаясь к вопросу о заработках, приведу цитату из другого раздела той же страницы Андрея Колесова:


\begin{shadequote}{}
время, которые ты не стучишь по клавишам,~--- это время, в которое ты не зарабатываешь деньги.
\end{shadequote}

Что, опять же безотносительно к деньгам, заставляет вспомнить слова Резерфорда, сказанные им сотруднику, которого постоянно заставал в лаборатории в любое время суток:


\begin{shadequote}{}
Если вы всё время работаете~--- то когда же вы думаете?
\end{shadequote}

Не в обиду моим коллегам, материалы которых я сейчас обсуждаю, но чтение ряда СМИ, вне зависимости от профильности или непрофильности, вызывает в памяти другие слова, теперь уже попугая Флобера:


\begin{shadequote}{}
Jamais, jamais, jamais\dots
\end{shadequote}

\noindentи заставляет если не заплакать по французски, то заржать <<испацтуло>>.

Но поехал дальше. Слова


\begin{shadequote}[r]{АК}
Статьи и блоги~--- это совершенно разные жанры.
\end{shadequote}

пока пропустим~--- к этому вопросу я вернусь на странице, завершающей блогометку. А вот на цитате


\begin{shadequote}[r]{АК}
Блоги в чем-то проще, но во многом~--- намного сложнее и более трудоемко
\end{shadequote}
остановлюсь сейчас. Трудоёмкость что статьи, что блогометки определяется исключительно количеством вложенного в неё труда. Работать, как известно, вообще довольно трудно. И потому <<статей>> в СМИ, написанных левой задней ногой журналиста, куда больше, чем аналогичных по исполнению блогометок в блогах читаемых авторов. Может быть, это следствие предыдущего высказывания? Сводящегося к народному


\begin{shadequote}{}
Фигли тут думать~--- трясти надо!
\end{shadequote}

И не отсюда ли идёт мнение, что вести блоги~--- более трудоёмко? Ибо блоггер, не связанный ни сроками, ни объёмами, имеет таки время подумать~--- а это не самая лёгкая работа.

Но АК причину видит в другом:


\begin{shadequote}{}
основная проблема~--- дискуссии, это большим затраты времени и сил
\end{shadequote}

На что СГ резонно отвечает, что


\begin{shadequote}{}
\dotsнет прямой зависимости от интенсивности дискуссии и успешностью блога.
\end{shadequote}

К его примеру с комильфошками можно добавить и пример нашего с Сергеем старого товарища~--- Владимира Попова: его материалы, размещённые на <<блогоподобных>> ресурсах, всегда имели высокий PR (подчас больший, чем на <<морде>> сайта), хотя в дискуссиях он почти никогда не участвовал.

И последнее~--- об ответственности:



\begin{shadequote}[r]{АК}
Принципиально иная модель \textit{отвественности} (это~--- очень важно!)\dots За \textit{граматические} ошибки в моих статья несет \textit{отвественность} выпускающий редактор, а не я!
\end{shadequote}

Курсив мой~--- пардон, но так в цитируемом фрагменте. Это я не крючкотворства ради (грешен, своих опечаток тоже не замечаю), а для иллюстрирования к высказыванию Сергея,


\begin{shadequote}{}
что в действительно полезном материале на грамматические ошибки и некоторую корявость стиля читатель уже не обращает внимания.
\end{shadequote}

А по существу вопроса об ответственности\dots Юридическую сторону оставим юристам. Но, Андрей, Вам ли с Вашим журналистским стажем не знать, что читателем любое лыко в материале будет поставлено в строку не главному редактору (которого читатель, как правило, не знает), и не выпускающему редактору (о его существовании читатель и не подозревает), а именно автору~--- будь он журналистом или блоггером. Правда, для этого автор должен быть достаточно известным, а лыко -– достаточно <<липовым>>. Иначе его просто никто не заметит.

Ну а на вопрос Сергея


\begin{shadequote}{}
Так в чем же главное отличие журналиста от блогера?
\end{shadequote}

я постараюсь ответить в следующем раздельчике.

\subsection{Так в чём же разница?}

В конце предыдущего раздельчика я обещал ответить на вопрос, поставленный Сергеем. И постараюсь нынче исполнить своё обещание.

Первое отличие банально, лежит на поверхности и бросается в глаза~--- может быть, именно поэтому на нём не было акцентировано внимание: журналист работает журналистом. Блоггер блоггером обычно не работает~--- он может работать кем угодно.

Предвижу возражения по первому тезису: журналист не обязательно должен быть штатным сотрудником некоего издания и получать в нём жалованье. Согласен. Но все профессиональные (то есть зарабатывающие себе этим на хлеб насущный) журналисты, по крайней мере в околокомпьютерной сфере, с каким-либо изданием (или кругом изданий) связаны обязательно~--- формальными или неформальными узами. Настоящих вольных стрелков~--- ландскнехтов от компьютерной журналистики, тех, кто сегодня предлагает свой меч компьютер в одно издание, завтра в другое, нынче нет. Да, пожалуй, что и не было на просторах постсоветской Руси. Ибо это занятие не могло служить гарантированным источником хлеба насущного.

Так что, даже если журналист-профессионал не расписывается в ведомости издания за фиксированный оклад содержания, а получает свои шекели покилознаково~--- считать его свободным художником не больше оснований, чем проходчика или буровика, сидящего на сдельщине: долговременные, пусть и неформальные, отношения с изданием предполагают, что он вписывается в его график.

Блоггер же, повторяю, может работать где угодно. И, как правило, где-то и работает. Даже в том случае, если доход его от блогерства многократно превышает формальную зарплату, некое место постоянной службы он обычно имеет. Но даже если и нет~--- это не значит, что он блоггер по профессии. Потому что удачливое блоггерство неразрывно связано с SEO. А вот это~--- уже профессия. И потому всякий удачливый блоггер~--- неизбежно немного SEO'шник и немного рекламный (или саморекламный) агент.

С сущности, блоггер как раз и выступает в роли того самого вольного стрелка-ландснехта. Только торгует он не непосредственно своим мечом, а военной добычей, которую ему этот меч обеспечил: рекламными площадками на своих ресурсах. И действительно имеет возможность, по крайней мере теоретическую, выбирать, кому их продать. То есть, кроме того, удачливый блоггер ещё немного и маркитант.

Тут впору затронуть вопрос о свободе слова, печати \textit{etc}.~--- одну из вечных тем, поднимаемых, когда речь заходит об информации в самом широком смысле слова, и путях информирования общества в частности: ведь и журналисты, и блоггеры выполняют эту функцию: первые~--- по долгу службы, вторые~--- по призванию (или по тому чувству, которое они призванием полагают).

Казалось бы, тут всё ясно. Прожжённый журналюга, связанный постоянными отношениями с каким-либо изданием (пусть, повторяю, совсем не формальными), даже теоретически не может считаться независимым. Ибо публиковать что-либо, несоответствующее профилю издания, ему не позволят. Говоря <<профилю>>, я вкладываю в это понятие самый глубокий смысл, например: в глянцевом журнале для кошатников не будут публиковать статьи про методы дрессировки служебных собак, в атеистическом издании~--- религиозные проповеди, в издании религиозной тематики, напротив,~--- сочинения записных безбожников\dots эту цепочку можно продолжать долго.

С другой стороны, блоггер~--- часть того самого общества, которое он призван информировать, и которое так в информации нуждается, плоть от плоти его. А потому сочинения его будут восприниматься этим обществом с большим доверием, нежели сочинения профессионального журналиста, которого обществе \textit{a priori} полагает продажным (вне зависимости от того, насколько это обвинение обосновано). И это доверие~--- одна из составных частей успеха блоггерства вообще и удачливости отдельных блоггеров в частности.

На самом деле, как мы скоро увидим, всё не так просто. Но к этому я вернусь после рассмотрения второй составляющей успеха блоггерства и удачливости блоггера.

И тут на память приходят слова Евстигнеева, произносимые им в качестве режиссёра самодеятельности в бессмертном советском фильме \textit{<<Берегись автомобиля>>}:


\begin{shadequote}{}
\dotsестественно, что актёр, не получающий зарплаты, будет играть с большим вдохновением. Ведь тогда актёр должен где-то работать\dots
\end{shadequote}

Мы с советских времён привыкли, и не без оснований, иронизировать над этими словами. Однако в свете темы \textit{Журналиста и Блоггера} они неожиданно обретают новый смысл. Чтобы прояснить его, позволю маленькое отступление.

Когда я уже почти четверть века назад впервые приобщился к той сфере, которую потом назовут IT, я, вследствие острого чувства информационного голода, читал всю компьютерную прессу: сначала совсем <<ненашу>>, потом~--- <<нашу>> переводную и смешанную, потом оригинальную. И по прессе <<ненашей>> и <<нашей>> переводной заметил: записной колумнист <<толстого>> компьютерного журнала занимается сочинением колонок без перерывов года два-три.

Потом года на два-три же его имя исчезает со страниц журнала. Из косвенных источников часто можно было узнать, то это время он посвятил бизнесу, консультационной деятельности (не обязательно по тематике его бывших колонок) или даже возвращался к началу своих начал~--- практической работе в IT-сфере.

А потом, как феникс из пепла, снова появлялся в журнале в амплуа колумниста~--- и, возможно, по совсем другой тематике.

Исключения из этого правила были редки, и однозначно показывали, что штатный колумнист за три-пять лет банально исписывается. Если, конечно, он не занимается ничем другим, кроме текущей редакционной работы журналиста-штатника. Но~--- продолжу цитирование:


\begin{shadequote}{}
Нехорошо, неправильно, если он целый день, понимаете ли, болтается в театре
\end{shadequote}

Ибо периодическая смена рода деятельности для получения нового опыта и новых впечатлений~--- насущная необходимость для творческого работника. А сочинительство в любом жанре, в том числе и в журналистике~--- работа творческая, что бы ни говорили по этому поводу.

Но ведь блоггера, напрямую не получающего гонораров за свои сочинения, сама жизнь ставит в условия, когда он должен временами отвлекаться от своего уютного бложика и окунаться в реальную жизнь ради куска хлеба с маслом. Окружающий же нас реал способен дать массу впечатлений даже самому не впечатлительному человеку.

Участь сия не минует и того блоггера, который зарабатывает своим блогом больше иного журналиста: потому что, повторяю, для этого ему приходится постоянно переходить в амплуа маркитанта-сеошника, а это хороший источник впечатлений и эмоций, столь необходимых для творчества.

Конечно, журналист тоже живёт не на облаке. И тоже получает массу впечатлений от своей профессиональной деятельности. Однако они и ограничены его профессиональной сферой, тогда как у блоггера~--- далеко выходят за её рамки.

И тут становится ясно, что в словах (продолжаю цитирование любимого фильма)


\begin{shadequote}{}
Насколько Ермолова лучше играла бы вечером, если бы днём она стояла у шлифовального станка
\end{shadequote}

\noindentкроется глубокое философское содержание. Результатом является то, что сочинения блоггера более разнообразны и несут на себе более яркий отпечаток действительности. И это, наряду с представляющейся объективностью, вторая причина, почему ярких неординарных блоггеров народ читает охотнее, чем журналистов, возможно, столь же ярких и неординарных.

Поэтому~--- прошу прощения, но ещё одна цитата из того же кладезя. На этот раз последняя:


\begin{shadequote}{}
Есть мнение, что народные театры вскоре вытеснят, наконец, театр профессиональный. И это правильно.
\end{shadequote}

Вариант, озвученный в первой фразе, я не исключаю: вполне может быть, что традиционные СМИ отомрут уже на нашей памяти. Или изменятся до неузнаваемости.

А вот со второй фразой согласиться не могу. Почему~--- ответ кроется во вскользь затронутом выше вопросе о видимой зависимости журналиста и независимости блоггера. Но это~--- уже в следующем раздельчике.

\subsection{Следствия}
А теперь посмотрим, какие следствия может повлечь за собой массовая <<блоггеризация>> источников информации для общества. Для чего необходимо сначала рассмотреть вопрос о <<продажности>> журналистов и <<независимости>> блоггеров.

Первый тезис в нашем народе сомнений не вызывает: все журналюги, а тем более журналасты и журнализды, давно продались на корню. Продались президенту и правительству, коммунистам и олигархам, Госдепу и Моссаду, мировому монополисту или отечественному производителю\dots моя фантазия исчерпана, предлагаю читателю дописать нужное.

Так что, не зная точно, кому именно продалась вся журналистская братия оптом и в розницу, и, тем более, не имея никакой инсайдерской информации о котировках на этом рынке, этот вопрос оставлю на обсуждение тем Неуловимым Джо от журналистики, которых никто не купил. А поговорить предлагаю чуть о другом.

Как уже было отмечено на одной из предыдущих страниц, профессиональный журналист теми или иными узами связан с каким-либо изданием. И должен писать в рамках, в этом издании очерченном. Точно так же, как вольный бич-проходчик, сидящий на сдельщине, копает канаву в направлении, указанном ему начальником партии (нет, не коммунистической, а геологической). Точно так же, как столь же вольный бичара, но промывальщик, сидящий на повремёнке, тоже моет шлихи не там, где ему почесалось под левой подмышкой.

Интересно, что никто не обвиняет в продажности проходчика или промывальщика, хотя они получают своего прямого или тарифную ставку за выполнение приказов начальника. А хорошие промывальщики и хорошие проходчики~--- за хорошее их выполнение. Как не обвиняют себя в этом и сами обвинители, будь они слесарями, чиновниками, офисными работниками и даже начальниками чином пониже. Ибо и они делают то же самое.

Как раз у журналиста степеней свободы в отношениях с начальством несколько больше, чем у офисной секретарши. По крайней мере, как мы скоро увидим, пока больше.

А вот блоггер на этом фоне выглядит как д'Артаньян весь в белом. Или как батька Махно в тачанке под чёрным знаменем. Короче, символом независимости от всех и вся. Ибо начальство над ним если и есть~--- то за пределами блогосферы, там, где он получает зарплату. Или, если зарабатывает на блоге, в силу специфики далёк от всяких политических, идеологических и прочих привходящих моментов.

Казалось бы, существование блоггерства, и блоггерства успешного, напрочь опровергает слова некоего 
В.\,И.\,Ульянова в скобках Ленина:


\begin{shadequote}{}
Жить в обществе и быть свободным от общества нельзя. Свобода буржуазного писателя, художника, актрисы есть лишь замаскированная (или лицемерно маскируемая) зависимость от денежного мешка, от подкупа, от содержания\dots
\end{shadequote}

Ибо блоггер действительно свободен от некоего конкретного денежного мешка, подкупать его никто не собирается, и брать на содержание~--- тем более. Но всё это не делает его независимым от общества, в котором он живёт, как вне блогосферы, так и внутри неё. И, будучи его членом, он также не свободен от манипулирования, целенаправленного или случайного, этим самым обществом.

Более того, он может служить инструментом для манипулирования. Потому что энергичный и удачливый блоггер действительно способен донести некую мысль до множества своих читателей~--- как я уже говорил, в народе веры блоггеру куда больше, чем любому иному пропагандисту и агитатору. Так что со стороны манипуляторов дело остаётся за малым: эту самую мысль блоггеру подкинуть в той или иной форме. Например, в форме борьбы за свободу выбора или технологический прогресс.

Если не подниматься до вершин обществоведения или, напротив, не погружаться в клоаку <<настоящей>> политики, а ограничиться рамками даже не IT-сферы, а той её части, которая имеет отношение к открытому и свободному софту, то совсем недавно мы видели ряд примеров тому. Здесь можно вспомнить

\begin{itemize}
	\item и кампанию против OEM'ной Windows, трактовавшуюся чуть ли не как борьба со вселенским злом; 
	\item и агрессивное продвижение <<ультра-прогрессивных>> интерфейсов вроде GNOME~3 с его GNOMEShell'ом; 
	\item и протекающая буквально сейчас эпопея с внедрением \texttt{systemd} сотоварищи. 
\end{itemize}


Всё это в значительной мере осуществлялось (и осуществляется) силами блоггеров.

Впрочем, не будем лезть в проблемы глобальные и всемирно-общественные, а ограничимся более узкими вопросами сообщества. Ибо блоггеры народ всё больше молодой и увлекающийся. В том числе и идеями улучшения мира. А потому легко ведутся на фразы, обёрнутые в идеи свободы, выбора и прогресса. А то, что за этими фразами стоит желание неудачливого политика чуть подправить своё реноме, или стремление некоей фирмы к доминированию в своём сегменте рынка~--- это ведь на лозунгах не написано\dots

Да, до сих пор организация таких кампаний носила мелко-дилетантский характер. Но это пока~--- от отсутствия опыта и наработок. И то, и другое~--- дело наживное, и манипуляторы быстро ликвидируют эту прореху. Собственно, процесс их обучения можно видеть при сопоставлении кампаний \textit{pro} GNOME~3 и 
\textit{pro} \texttt{systemd}. Первая~--- это грубый лобовой напор в расхваливании новой системы, увенчавшийся лишь относительным успехом.

А вот последняя кампания готовилась более тонко и загодя, начиная с превентивного клеймения потенциальных противников как ретроградов, обскурантов, консерваторов и вообще врагов прогресса; разве что в несчастьях Галилео Галилея их лично не обвиняли. Плюс были учтены уроки великих манипуляторов пошлого. В частности, такой: если постоянно повторять некий набор тривиальных истин, то в конце концов блоггеры, а за ними и их читатели поверят, что они изобретены повторяющим их. Впрочем, это тема совсем другого памфлета.

А пока подведу итог. Если технология манипулирования через <<независимых>> блоггеров получит развитие (а к тому дело и идёт), то действительно, народные театры вытеснят театр профессиональный. Ибо в организованных СМИ и профессиональной журналистике просто не будет необходимости: те же цели будут достигаться с куда меньшими издержками. Как говорится, добровольно и с песнями.

И тогда останется только с грустью вспомнить тех самых прожжённых журналюг прошлого, которых нынче обвиняют в продажности.

Так что не в интересах блоггеров вытеснять журналистов. Как и не в интересах последних~--- вытеснять блоггеров. Ибо тогда степеней свободы в выборе источников информации станет одной меньше. А ведь их всего две\dots

\section{Конец бумажного книгоиздания?} 
\begin{timeline}Ноябрь, 2011\end{timeline}
Заметка о судьбе книгоиздания и книготорговли, планировавшаяся первоначально на несколько абзацев, по логике повествования разрослась в цикл, который и предлагается вниманию читателей.

В цикле этом речь пойдёт исключительно о художественной, исторической, публицистической и тому подобной литературе. И ещё, всё сказанное в данном цикле, за исключением единичных особо оговоренных моментов, касается русскоязычной литературы и её распространения на просторах бывшего Союза~--- там, где она востребована. О проблемах книгоиздания и книготорговли в мировом масштабе пусть рассуждают мировые\dots нужное вписать в меру своей испорченности или политической грамотности.

\subsection{Часть 1}
Несколько лет назад я вдруг обратил внимание, что народ в московском метрополитене почти прекратил читать. Ну или количество читающей публики резко сократилось. Поначалу я не придал этому значения. А потом поймал себя на том, что не вижу в книжных магазинах ничего, что хотел бы приобрести. Да со временем и в книжные стал заходить существенно реже\dots

Одновременно, поскольку я продолжал поддерживать связи с книгоиздательским миром, до меня стали доходить слухи об остром кризисе жанра в этой отрасли. Но поскольку та часть книгоиздательского мира, с которой я общался, была достаточно специфической, масштабов этого кризиса я не представлял. Точнее, перестав писать бумажные книжки, о нём не задумывался до тех пор\dots

До тех пор, пока на Джуйке с подачи Евгения Чайкина aka StraNNicK не началось весьма активное обсуждение статьи Елены Прудниковой <<Радостный взлет цунами>>. Которая, в свою очередь, некоторым образом представляет собой обсуждение статьи Анастасии Якоревой <<Конец одной книги>>.

Пересказывать их содержание я не буду~--- интересующийся темой легко прочтёт сам. Скажу только, что в них констатируется падение оборота книготорговли~--- до 15-20\% в год. За точность цифры ничего не скажу, но то, что книг покупают меньше~--- это факт, подтверждаемый наблюдениями в книжных магазинах.

Во второй статье факт этот объясняется, с одной стороны, конкуренцией с более иными формами проведения досуга, типа компьютерных игр и Интернета, с другой --конкуренцией с электронными читалками.

По первому пункту Елена Прудникова резонно возражает:


\begin{shadequote}{}
Люди, которые проводят досуг перед телевизором или в развлекательных центрах, не читали книг раньше и не будут читать их впредь.
\end{shadequote}

И с этим нельзя не согласиться. Более того, как я уже говорил, впервые о кризисе книгоиздания я услышал от тех, кто издаёт вовсе не <<досугозаполнительную>> литературу, а естественно-научную, например. Которую в Интернете заменить нечем.

А вот со вторым её возражением,


\begin{shadequote}{}
что все больше людей пользуются интернет-версиями изданий только для того, чтобы определить, стоит ли книга того, чтобы её прочесть
\end{shadequote}

\noindentвсё не так просто. Особенно если не ограничивать проблему только онлайновым чтением, но не забывать и о тех самых электронных читалках. Но к этому вопросу мы вернёмся после того, как рассмотрим объяснения Елены Прудниковой, коих она предлагает четыре:
\begin{enumerate}
	\item неадекватные цены; 
	\item качество издаваемой продукции;
	\item неквалифицированный маркетинг; 
	\item дискомфортные условия продажи.
\end{enumerate}

Рассмотрим эти пункты последовательно, пропустив пока второй~--- до выяснения общей обстановки.
\subsubsection{Цена}
Да уж, что есть, то есть, цены на книги ни в~\dots~ухо, ни в Красную Армию. Причём обусловлены они вовсе не жадностью книгоиздателей и тем более авторов, как часто думают люди, далёкие от знания реалий. А многоступенчатостью распространения, с накруткой на каждом этапе, и накрутками конечных продавцов. Цитирую:


\begin{shadequote}{}
за время путешествия от типографии к прилавку её цена до последнего времени ухитрялась вырасти в три-четыре раза
\end{shadequote}

И подтверждаю, что это действительно так, и даже хуже. Одна из моих книжек, отнюдь не боевик-блокбастер (а вполне такая техническая), в некоторых регионах продавали в 10 (!) раз дороже отпускной цены издательства, которую я, по понятным причинам, знаю совершенно точно. А уж если речь идёт о продаже в Ближнем Зарубежье~--- тут вообще туши свет. А ведь, что бы ни вещали наши политики, информационное пространство России, Украины и Белоруссии одно. По крайней мере, если не выходить за пределы IT-сферы. Тем не менее, ей богу, в Израиле мои книжки продавались дешевле, чем в Украине.


\textit{Отступление специально для любителей считать деньги в чужих карманах: автору с этих накруток не обламывается ни копейки. Он получает фиксированный процент с отпускной цены издательства. Которому, в свою очередь, тоже ничего не перепадает от цены книжки на прилавке.}


Всё это верно,

\begin{shadequote}[r]{Александр Галич}
\dotsно, однако, Не подходит это дело к моменту. 
\end{shadequote}

Во-первых, читатель, как правило, понятия не имеет о том, из чего слагается конечная цена книжки на прилавке магазина. Так что все приведённые выше расчёты его, в отличие от нас, многогрешных, нимало не волнуют.

Во-вторых, книги всегда были относительно дороги:

\begin{itemize}
	\item и при социализме, потому что их часто приходилось приобретать отнюдь не в магазине, а, скажем, на толкучке в столешниковом переулке, и вовсе не по цене, напечатанной сзади, 
	\item и при <<перестройке>>, когда книгоиздатели и книготорговцы бросились снимать сливки, пенки и тому подобную сметану со своего высококультурного начинания, 
	\item и при <<пост-советском режиме>>, когда окончательно воцарился бездушный и бездуховный дух чистогана. 
\end{itemize}


В-третьих, не смотря на это, я не знаю ни одного настоящего книгочия, которого остановила бы цена приобретения желанной книги. Разве что чуть притормозила~--- до получки, скажем.

В-четвёртых и главных, на самом деле для настоящего книгочия реальные затраты на приобретение книг не так уж и велики~--- но это уже относится к пункту второму, который будет рассмотрен в следующей заметке.

Так что, хотя цена~--- конечно, весомый фактор, но он имел место быть всегда. Но почему же именно в последние пять лет он оказался таким критичным? Так что объяснение по первому пункту принимается лишь частично. Переходим, в соответствие с принятой ранее договорённостью, к пункту третьему --

\subsubsection{Неквалифицированный маркетинг}
И с этим не поспоришь~--- так себе маркетинг книжной продукции. В большинстве случаев он сводится к аннотациям на обороте титула или форзаце. Причём часто они настолько не соответствуют содержанию книги, что возникает сомнение~--- а открывал ли её автор, с позволения сказать, аннотации?

Однако и тут возникает очередное однако: а когда этот самый книжный маркетинг был лучше. Ссылка на то, что


\begin{shadequote}{}
при социализме книги издавались и раскупались миллионными тиражами
\end{shadequote}

не проходит: тиражи определялись в соответствие с директивными указаниями (вспомним бессмертную работу Брежнева Ленинским курсом, изданную на татарском, башкирском, каракалпакском, якутском и многих других языках массовым тиражом). А раскупались не благодаря <<маркетингу>> (функции которого выполняли тогда штатные рецензенты литературных журналов), а часто вопреки ему. Хрестоматийный пример~--- первое книжное издание \textit{Двенадцати стульев}, разошедшееся мгновенно, не смотря на молчание критиков.

На моей памяти лучшим книжным маркетологом Советского Союза был старик, продававший с рук книги в подземном переходе от станции метро <<Площадь Свердлова>> к улице Горького. Главный его маркетинговый тезис звучал так:


\begin{shadequote}{}
Книга, которую ищет вся читающая Москва!
\end{shadequote}

И ведь обычно это соответствовало действительности, так что покупали.

В отношении маркетинга ничего не изменилось и ныне. Если некая книга (или, скорее, автор) удостаивается специального продвижения, то делается это так, что лучше бы не делалось вовсе. Опять же цитирую автора статьи:


\begin{shadequote}{}
Десять лет такой работы, и читатель начал плеваться при виде рекламы, а издательства~--- бояться её.
\end{shadequote}

В этом отношении резонней поступают издатели научно-технической, в частности, околокомпьютерной, литературы, возлагая маркетинг книг на их авторов. Что же, лозунг


\begin{shadequote}{}
Дело помощи утопающим~--- дело рук самих утопающих
\end{shadequote}

\noindentкак заметил один из героев Бориса Акунина, давно уже должен быть внесён в Конституцию Российской Федерации отдельной статьёй. Возможно, самой первой\dots

Ну а если вернуться к нашим книгочиям~--- всякого рода официальный (ныне фирменный) маркетинг не колыхал их в социалистические времена точно так же, как не колышет сейчас. Прежде информация о книгах, заслуживающих прочтения, передавалась из уст в уста. Ныне~--- через форумы, блоги, ЖЖ, социальные сети. А поскольку возможности оных гораздо шире, чем изустной молвы, можно сказать, что <<народный>> маркетинг только окреп и закалился.

Так что и третий пункт объяснения Елены Прудниковой на самом деле не объясняет книгоиздательского кризиса последних лет.

\subsubsection{Дискомфортные условия продажи}
А вот по этому пункту позволю себе не согласиться с автором целиком и полностью. Да, есть магазины, где пробираешься промеж книжных стеллажей, как гротах сьяновских каменоломен, да ещё заполненных людьми, как в московском метро в часы <<пик>>. Есть магазины, где книги рассыпаны по полкам так, как будто их раскидывал сеятель облигаций в исполнении Остапа Бендера.

Но встречаются и магазины достаточно просторные, где книги тщательно рассортированы по тематике, авторам \textit{etc}. А бывают даже такие, в которых стоят компьютеры, выдающие квиточек с местоположением искомой книжки с точность до зала, ряда стеллажей, конкретного стеллажа, полки в нём и даже места на полке.

Опять же, книгочии, регулярно посещающие определённые магазины, ориентируются в них, даже самых бардачных, гораздо лучше сотрудников. И весь дискомфорт им не помеха~--- они целеустремлённо следуют сквозь толпу случайных\dots не могу назвать их покупателями, скорее посетителей. И без особых проблем находят то, что им нужно.

Но и это не самое главное. Потому что существует институт Интернет-торговли. И как раз в отношении книжек он нынче вполне отлажен: можно заказать книги, в том числе и те, что давно прошли в оффлайновых магазинах, с куроерской доставкой по данному городу или почтой по всей стране, практически с любой формой оплаты~--- от наличными курьеру до web money и даже, иногда, наложенного платежа. Цены на книги в Интернет-магазинах~--- как правило, не выше, чем в магазинах обычных, а подчас и ниже. Стоимость курьерской доставки~--- вполне пристойная, и при заказе нескольких наименований практически незаметна на общем фоне, сроки~--- в пределах двух-трёх рабочих дней.

Конечно, в случае доставки почтой в другой город в дело вступают почтовые тарифы и почтовые же сроки. Но, даже с учётом пересылки, книги оказываются дешевле, нежели в обычных магазинах того же региона, куда они попадают через тридцать три посредника. Или не попадают вообще~--- и тогда Интернет оказывается единственным источником книжной премудрости.

Что характерно~--- книжная Интернет-торговля существует у нас уже полтора десятка лет. Но именно за последние годы она приобрела свой нынешний отлаженный вид, это во-первых. А во-вторых, на эти же годы пришлась и интернетизация множества городов и весей нашей необъятной Родины, в том числе и тех, где до недавнего времени о Сети не приходилось и мечтать.

Так что, вместе с развитием <<народного>> маркетинга, как раз в последние пять лет создались благоприятные условия для расцвета книжной торговли. И почему же они оказались реализованы в виде блистательного провала? Ведь, как мы убедились перед этим, ценовой фактор существовал всегда, и не мог оказать какого-то особо тлетворного влияния именно в это время.

А вот тут самое время вернуться к пункту второму.  О чём~--- в следующем раздельчике.

\subsection{Часть 2}
Итак, в прошлой заметке мы выяснили, что причиной нынешнего падения оборота книготорговли не могут быть ни цены, потому что они всегда были завышенными, ни маркетинг, потому что он всегда был плох, ни условия продажи. Более того, последние два фактора, казалось бы, должны играть на руку продажам книг~--- за счёт сетевого <<народного>> маркетинга и торговли через Интернет, соответственно. Однако этого не наблюдается.

Почему? Чтобы ответить на этот вопрос, обратимся ко второму объяснению, предлагаемому Еленой Прудниковой~--- качеству выпускаемой продукции. Рассмотрев вопрос цен и придя к выводу, что они завышены (мы его тоже рассмотрели, и пришли к выводу, что такими они были всегда), она пишет:

\begin{shadequote}{}
\dotsза сумму, обозначенную на ценнике, читатель вправе потребовать если не шедевр, то добротную вещь, которую он будет читать не один раз. Однако добротной вещи он не получит.
\end{shadequote}

Вот тут-то и зарыта собака! Как я уже говорил, настоящий книгочий за стоящую книгу готов заплатить настоящую цену. Но нету их, стоящих книг, просто не стало. Куда же они делись? Ведь совсем-совсем недавно казалось, что их~--- море израилеванное. Чтобы ответить на этот вопрос, Елена Прудникова предлагает вспомнить, с чего всё начиналось. Вспомним это и мы~--- только спустившись немного глубже.

Сейчас уже как-то забывается, что в начале перестроечного книгоиздания был выпуск <<диссидентской>>~--- произведений, написанных в советское время, однако имеющих <<антисоветский>> характер, и потому в своё время не печатавшихся. Народ поначалу набросился на них, ожидая обнаружить там раскрытие тайн мадридского двора кремлёвских кабинетов и лубянских подвалов.

Однако прочтение одной-двух, много трёх таких книжек вызывало чувство искреннего почтения к литературному вкусу экспертов Главлита и кураторов КГБ, в сущности проделывавших работу литредакторов: вне зависимости от меры <<антисоветскости>>, это были, за редким исключением, просто плохие повести, рассказы, романы. Настолько плохие, что сейчас не припомнить ни их авторов, ни названий. Те же, что не были плохими, интересующиеся уже давно прочли в самиздате или тамиздате (это опять таки к вопросу о <<народном>> маркетинге). Так что интерес к такого рода литературе пропал очень быстро.

Почти одновременно опустели прилавки больших книжных магазинов~--- нечего стало продавать. И вы думаете, это был кризис книгоиздания? Ни фига подобного. Ибо поднялся мутный вал переводной литературы~--- сначала детективов и боевиков, потом \textit{science fiction}, а там и до \textit{fantasy} дело дошло.

Стоило всё это хозяйство несусветные по тем временам (сравнительно с зарплатой госслужащего) деньги, при том, что качество изданий варьировало от плохого до хуже некуда.  А продавалось с лотков в переходах метро, подземных переходах, просто на улице, часто на морозе и пронизывающем ветру~--- то есть в местах, для торговли, мягко говоря, не очень приспособленных.

И, разумеется, ни малейшего маркетинга, кроме <<народного>>: значительная часть авторов была знакома старым книгочеям по единичным публикациям в советских издательствах и самиздату, и знаниями своими они делились прямо у лотков. Часто в обсуждении активно участвовали продавцы~--- среди них встречались очень литературно подкованные люди.

Помню, как на станции Царицыно время в ожидании электрички можно было провести за очень приятной беседой с девушками, торговавшими книжками с лотка. Например, обсудить вклад, который Спрэгг де Камп как редактор внёс в рассказы Роберта Говарда. Наверняка, не все нынешние любители фэнтези это знают\dots


\textit{Отступление. Между прочим, с лотков же торговали и компьютерной литературой~--- тогда она практически сводилась к двум позициям, книгам Брябрина сотоварищи и Фигурнова. Которые, кстати, стоили дороже Чейза и Агаты Кристи, и найти их было гораздо труднее. Но об этом~--- в другой раз и в другом месте.}


Так вот, не смотря на цены, качество полиграфии (точнее, отсутствие такового), никакой маркетинг и и более чем никакие условия продажи, кризисом книготорговли и книгоиздания не пахло. Ибо, хотя далеко не всё, что продавалось с лотков, было шедеврами, но и откровенного дерьма среди переводной литературы встречалось не много.

Ну и плюс, конечно, фактор новизны~--- советская власть не особо баловала нас всеми перечисленными жанрами, а фэнтези тогда многие открывали для себя впервые, как та девушка, что открыла прокладки\dots не помню имени кого.

Однако постепенно чувство новизны притупилось. Да и поток качественной переводной литературы начал иссякать~--- ибо и на западе вовсе не каждый день клепают нетленку. Но и тут кризиса не случилось, потому что подоспел третий вал произведений~--- в тех же жанрах, но уже отечественных авторов.

Правда, как правило, сюжеты их были вторичными. Даже в жанре славянского (или, скорее, славяно-варяжского) фэнтези, претендующего на оригинальность, большая часть оригинального шла как бы от оппозиции ихним западным эльфам и прочим оркам. Тем не менее, произведения той волны в основном были вполне читабельны.

Однако на горизонте собирался уже четвёртый вал~--- вал сериалов. Количество книг одних и тех же авторов стало исчисляться уже не томами, а погонными метрами заполненных ими стеллажей или кубометрами развалов. Авторы были как те же самые, так и новые, причём среди последних в изобилии оказались представлены и переводные, фамилий которых ранее не было слышно.

Прогресс? Казалось бы, да. Однако, мысленно прикидывая погонный метраж, созданный за столько короткий срок, поневоле приходит мысль, некогда высказанная Резерфордом в адрес своего сотрудника: если они столько работают, то когда же они думают? И желание знакомиться с их творчеством, даже путём пролистывания у книжной полки, пропадало напрочь.

Да, я забыл сказать, что параллельно, и даже опережающими темпами, шло исчезновение книжных лотков с их подчас очень интеллигентными продавцами. Постепенно они свелись к развалам типа 
\textit{``Любая книга за \#\# рублей''}
, уцелевшим и поныне. Книжная торговля возвращалась в благоустроенные магазины.

Появился тот самый пресловутый маркетинг~--- хотя бы в виде плакатиков или объявлений по матюгальнику в больших магазинах. Да и цены\dotsну не то чтобы снизились. Но по крайней мере в Москве и Питере не стало столь резкой диспропорции между стоимостью книги и средними заработками активной части населения. Хотя в Замкадье эта диспропорция не только сохранилась, но местами даже и выросла~--- однако тут, как я уже говорил, на помощь начал приходить Интернет.

И ещё обозначилась тенденция к вымыванию из ассортимента изданий прошлых лет, вне зависимости от их удачности или не удачности с точки зрения литературной (про коммерческую не говорю за отсутствием информации).

Нет, отдельных авторов, раскрученных (или успевших раскрутиться) продолжали переиздавать с регулярностью, ненамного уступающей бессмертной трилогии нашего дорого и незабвенного товарища Леонида Ильича Брежнева.

Не оскудевали и полки с переводной классикой, как правило, в виде полупудовых глыб типа <<Весь Толкин в одном томе>>, читать которые могли только те интеллигенты, которые в молодости достаточно позанимались спортом или поработали пролетариями. Да и с классикой отечественной дело несколько наладилось.

Но пропущенную книгу автора средней раскрученности (повторяю, вне зависимости от её литературных достоинств) найти стало возможно только на упомянутых выше развалах \textit{``Любая книга за''\dots}
 Да и то не всегда, в основном там концентрировались как раз книги, литературных достоинств бесспорно лишённые.

Все эти процессы завершились примерно одновременно~--- где-то году к 2005-му. Вот тут-то и начался тот самый кризис в околокнижном бизнесе, промежуточный результат которого мы наблюдаем сейчас (промежуточный~--- потому что итоговый может быть ещё хуже, но об этом в следующей серии). Народ перестал покупать книги, народ перестал читать книги. По крайней мере бумажные новоизданные. Не потому что он перестал читать вообще~--- а потому что почти все новые книги, которые лежали на полках магазинов, стали плохими. Как опять же резонно замечает Елена Прудникова, люди


\begin{shadequote}{}
\dotsне желают читать ту литературу, которую предлагают им магазины
\end{shadequote}

Помнится такой случай. В одном из больших книжных магазинов почти в центре Москвы воздвигли, как у них было в обычае (возможно, в обычае и сейчас~--- давно там не был), пирамиду из только что поступившей в продажу книги. В тот день это были очередные похождения Гарри Поттера. И я был невольным свидетелем разговора двух пацанов лет 10-12~--- то есть, казалось бы, целевой аудитории. Так вот, один из них говорит другому:


\begin{shadequote}{}
Смотри, опять Потного Гарри навалили.
\end{shadequote}

Мне это очень понравилось, и с тех пор я означенного персонажа иначе, чем 
\textit{Потным Гарри}, не называю.

Казалось бы, Русь Великая утрачивает славу самой читающей страны мира? Нет, от этой позорной участи мы оказались избавлены. Чем, как и благодаря кому~--- об этом далее.

\subsection{Часть 3}
Предыдущую заметку я закончил на оптимистической ноте: одновременно с тем, как угроза книжного голода обозначилась более чем реально, появилось и средство борьбы с ней. Надеюсь обосновать, что это действительно так. Но перед этим должен остановиться на нескольких пессимистических моментах.

Для начала вернёмся немного назад~--- к низкому качеству литературы, отдыхающей на книжных полках, всех магазинов, и от любых издательство. То есть коллег-книгочиев, казалось бы, поставили в безвыходные условия сложившейся многоголовой монополии, причём естественно сложившейся (интересно, описано ли такое явление? памятный с марксистких времён картельный сговор тут явно не к месту). А в таких условиях невольно вспоминается про без-рыбьи и без-птичьи, а также про дворника в отсутствие горничной. К счастью, обращаться к столь крайним мерам нет необходимости, ибо (опять цитирую Елену Прудникову)


\begin{shadequote}{}
Каждая вновь выпущенная книга вылетает из типографской машины не на пустую полку, а конкурирует со всеми книгами, выпущенными ранее.
\end{shadequote}

И это не может не радовать, особенно с учётом того, что


\begin{shadequote}{}
За годы социализма в стране были напечатаны даже не миллионы, а миллиарды книг\dots
\end{shadequote}

Здесь прервём цитату и для начала вспомним, какой процент от этих миллиардов составляют бессмертные труды незабвенного товарища Леонида Ильича\dots А можете мне поверить, очень немаленький, особенно с учётом бывших союзных и автономных республик. Причём, что характерно, труды классиков марксизма, ленинизма и тем более сталинизма такими тиражами не издавались никогда.

Далее, не забудем и классиков великой многонациональной советской литературы~--- их тоже было ох как немало. С учётом аналогичных классиков из братских стран социализма~--- так ещё больше. Ну а если сюда приплюсовать просто откровенную халтуру и макулатуру~--- а таковой и при советской власти тоже было вдоволь, например, производственные романы из жизни не сталеваров, и не плотников, и даже не монтажников-высотников, а вообще хрен кого\dots Таким образом обозначенные выше миллиарды сокращаются если не на порядки, то во многие разы~--- точно.

Продолжаю цитату:


\begin{shadequote}{}
\dotsи большая часть этого книжного моря где-то существует.
\end{shadequote}

И высказываю категорическое с ней несогласие. Если исключить перечисленное в предыдущих абзацах, то и из оставшегося многое ныне уже не существует. На счёт рукописей, которые не горят, не знаю~--- хотя подозреваю, что горят, и очень даже неплохо. А что уже вышедшая из типографии продукция с песнями тонет, да простят меня читающие это дамы, в дерьме, вытекающем из прорванной в книгохранилище канализации. Автор этих строк сие наблюдал неоднократно, и даже участвовал в эвакуационных работах. На прочих стихийных бедствиях, а также том, сколько библиотечных фондов оказалось ныне за чертой нашего современного государства, останавливаться не буду.

А перейду к следующей цитате, ещё более оптимистичной


\begin{shadequote}{}
В любой читающей семье в шкафу стоит не одна сотня томов, в нечитающей~--- несколько десятков.
\end{shadequote}

И вызывающей ещё более пессимистичные возражения. Поскольку это верно только в отношении читающих семей\dots эээ\dots среднего, так скажем, возраста, большую часть жизни проживших на одном месте (и, скорее всего, если это место~--- один из крупных городов).

Молодёжь же, в силу ряда причин, на которых здесь неуместно останавливаться, ныне существенно более мобильна. И перевозить им, даже очень читающим, эти самые сотни томов за сотни, а то и тысячи километров не всегда с руки. Не потому, что означенные тома не нужны~--- просто, увы, часто приходится выбирать между нужным, очень нужным и жизненно необходимым. И в число последних книги не всегда попадают~--- в том числе и потому, что до не столь уж далёкого времени личные книжные запасы казались столь же легко восполнимыми, как одежда или домашний инвентарь.

Но, как мы уже видели, нынче это стало совсем не так. Если ещё недавно при необходимости перечитать, скажем, Геродота, можно было просто пойти в магазин и купить его, то нынче это не так однозначно (даже не принимая в расчёт ценовой фактор, который в отношении такого рода литературы стал очень весомым). И если с Геродотом вопрос ещё как-то решается, то уж в поиска Арриана точно придётся побегать до пропотения, а Диодора Сицилийского или Курция Руфа так просто не найти. О научных монографиях исторического направления, случайно затесавшихся в стройные шеренги фоменковцев или пристроившихся к колоннам гумилёвцев, и говорить не приходится.

Так вот, процессе перемещения по стране (и тем более за её пределами) утрачивается изрядная часть тех самых миллиардов книг. Утрачивается не вообще~--- но для данного конкретного индивидуума безвозвратно. Это я вам говорю со всей ответственностью, поскольку носило меня\dots ну не от Амура до Туркестана, как товарища Сухова, а всего только от Кригизии до Корякии и от Кореи до Корсики. И возобновить их оказывается негде.

Потому что те самые миллиарды томов, за вычетом перечисленного ранее, конечно, где-то продолжают существовать. Однако для нашего конкретного индивидуума они недоступны.

Почти недоступны. Потому что, переходя, наконец, к оптимистической части своего повествования, позволю себе ещё одну цитату\dots


\begin{shadequote}{}
\dotsв современном городе даже страстный библиофил вполне способен прожить, вообще не заходя в книжный магазин.
\end{shadequote}

И вот тут я подписываюсь обеими руками и ногами. Но не потому, что у него дома стоят сотни томов, а по совершенно другой причине, о которой~--- на следующей странице.

\subsection{Часть 4}
Ну вот, после очередной порции уныния в предыдущей заметке мы наконец вышли на финишную прямую к светлому оптимистическому будущему. Так что же спасёт русских книгочиев? Вы будете очень громко смеяться, но спасут их\dots китайцы.

Выше мы обозначили роковой для книжного бизнеса рубеж~--- 2005-2006 годы. Странным образом он совпал по времени с распространением так называемых электронных книг, в просторечии именуемых, да простят меня читающие это дамы, e-book'ами, reader'ами или попросту читалками (последний термин и будет использоваться в дальнейшем).

Нет, читалки существовали и раньше, ещё в прошлом тысячелетии, однако на протяжении долгого времени широкой популярностью не пользовались. О причинах судить трудно~--- можно предположить, что главное была высокая стоимость. Не последним являлось та, что вывод текста осуществлялся на LCD-дисплей, то есть нагрузка на зрение была точно такой же, как и при чтении с обычного компьютерного монитора. Хотя на одном из форумов меня поправили, что были и вполне комфортные модели, но это очень индивидуально: я знаю людей, которые спокойно читают с двухдюймового экрана дебильника.

Однако на рубеже 2005-2006 годов в продаже появляются читалки принципиально иного типа, основанные на технологии так называемых электронных чернил (E-ink). Собственно на технологии я здесь останавливаться не буду. В рамках настоящего повествования достаточно сказать, что чтение с такого рода устройств создавало те же ощущения, что и чтение с бумажного листа (почему эту технологию называют также электронной бумагой). И, соответственно, нагрузка на глаза сильно снижалась. Правда, в отличие от LCD-читалок, читалки <<электробумажные>> (в дальнейшем~--- ЭБ) требовали внешнего источника освещения~--- читать с них в темноте (например, под одеялом) невозможно.

Кроме того, энергия аккумулятора расходовалась только при <<пролистывании>> страниц, благодаря чему время работы без подзарядки составляло очень многие часы. Собственно, автономия этих устройств измеряется не временем, а количеством просмотренных страниц, число которых обычно начинается с 10 тысяч.

Читалки на e-ink, точнее, их программное обеспечение (а изрядная их часть работает под управлением ОС Linux) способны выводить текст во многих общеупотребимых форматах~--- PDF, HTML, TXT, RTF, иногда DjVu, а также в форматах специализированных~--- ePUB и FB2. Последний, разработанный специально для использования в читалках нашими соотечественниками во главе с Дмитрием Грибовым и Михаилом Мацневым, снискал наибольшую популярность для русскоязычных текстов. Именно в нём по умолчанию доступны произведения нашей литературы на крупнейших Интернет-порталах соответствующего направления (парадоксально, что один из них, до некоторого времени самый известный, располагается в Эквадоре).

На протяжении первых лет своего существования ЭБ-читалки не могли похвастаться широким кругом пользователей. И причиной тому было сочетание двух факторов. Если LCD-читалки, в сущности, представляли собой облегчённые КПК, обеспечивающие ряд дополнительных возможностей, то ЭБ-читалки~--- в сущности, монофункциональные устройства. Кроме вывода текста, они способны, разве что, отображать картинки и воспроизводить звук~--- даже сетевые средства, как правило, напрочь отсутствуют. То есть электронные книги должны скачиваться из Интернета на компьютер и уже с него тем или иным способом (обычно~--- по USB-кабелю или посредством SD-карточек) переноситься на читалку.

И при этом цена ЭБ-читалок была совершенно запредельной для столь ограниченных по возможностям устройств: она начиналась, если мне не изменяет память, с 400 примерно уёв, нарастая, в зависимости от диагонали экрана (а типичные её значения укладывались в диапазон 5-10 дюймов) и <<брендовости>> (по простому~--- жадности) производителя до\dots да собственно до пределов жадности и возрастая.

Причём, в отличие от всей прочей компьютерной техники, надежд на снижение цены с ростом массовсти этих изделий не было. Ибо изрядная часть стоимости ЭБ-читалок приходилась на отчисления владельцу патента на технологию E-ink (он так и зовётся~--- E Ink Corporation). И отчисления эти, по агентурным сплетням из осведомлённых кругов, составляли около ста бкасов с каждого устройства. Производителю которого, кроме погашения расходов, надо было и себя не обидеть.

Так что, казалось бы, ЭБ-читалки не могли составить конкуренции чтению с компьютерного монитора для оцифрованного контента, с одной стороны, и не могли восполнить дефицит качественной бумажной литературы~--- с другой. Именно к этим тяжким годам и относится наблюдение, с которого я начал этот цикл заметок: народ перестал читать в метро.

Неужели для книгочиев обозначился тупик? А вот фиг, простите за невольную рифму. Потому что в течении последнего года чётко обозначилась тенденция к подешевлению ЭБ-читалок. Сначала это было ползучее снижение, то есть младшие (с точки зрения диагонали) модели дошли до терпимых 200 с копейками ихних тугриков. А в последние\dots даже не месяцы, а буквально недели произошёл настоящий обвал цен. И ныне стоимость моделей с экраном 5>> стартует с 4 тысяч рублей~--- обычных, постсоветских. Есть и устройства с вообще демпинговыми ценами, менее трёх тысяч, но они функционально ограничены, в частности, не понимают наш любимый формат FB2.

Правда, говоря о снижении и тем более обвале цен, я выразился не совсем правильно (или даже совсем неправильно). Цены на <<брендовые>> ЭБ-читалки почти не изменились, как и уже давно продающиеся полу-бренды (в число последних можно отнести существенно украинский PocketBook). Но появилось много моделей от фирм, ранее на нашем рынке не представленных, таких, как WEXLER и Hanvon. И сразу по низким (первая) или бросовым (вторая) ценам. Правда, и ползучее  снижение цен на продаваемые модели также имело место быть~--- но оно затронуло не очень именитых (правильней сказать~--- не зазнавшихся) производителей, вроде Gmini~--- его читалки продаются по 5-6 тысяч.

Чем это обусловлено~--- трудно сказать. Самое простое объяснение, конечно~--- фирма E Ink Corporation резко снизила патентные отчисления, движимая альтруизмом и любовью к книгочиям всех стран и народов. Поскольку мы с вами верим в лучшие чувства людей, даже тех, которые представляют бездушный и бездуховный мир чистогана, примем это объяснение в качестве основного. Однако элементарная аккуратность бывшего научного работника заставляет меня рассмотреть и другие варианты.

Первый~--- конкуренция со стороны LCD-читалок. Как уже было сказано ранее, они появились задолго до читалок ЭБ и популярности не снискали. Однако жизнь не стоит на месте~--- и в последнее время читалки с жидкокристаллическими дисплеями очень сильно усовершенствовались: и воспроизведение текста на них стало более чем приличным, и время автономии существенно возросло. К тому же они обладают функциями, ЭБ-читалкам в принципе недоступными: возможностью чтения в абсолютно тёмной обстановке и способностью воспроизводить не только звук, но и видео. И всё это~--- при цене, в среднем на треть меньше, чем у ЭБ-читалок. А LCD-модели начального уровня, с экраном 4>>, вообще можно приобрести за 2 тысячи рублей. И, однако, текст они выводят так, что он воспринимается даже остатками моих глаз.

А второе объяснение, почти конспирологическое, было высказано в обсуждении на Джуйке участником, известным как @Zmeyko: китайским производителям, способным обеспечить массовое производство читающих устройств, удалось свести патентную составляющую их цены практически к нулю. Разумеется, исключительно с помощью доброго слова (типа~--- или берите \$5, или сосите и худейте).

Возможно, будущее прояснит, какое из предложенных объяснений правильно. Пока же можно констатировать массовое распространение ЭБ-читалок. По крайней мере, в Москве. В первой заметке цикла я уже говорил, что почти полное отсутствие читающей публики в столичном метро в текущем году сменилось большим количеством людей, скрашивающих переезды читалками. А в последнее время я вижу ребят и девчат с читалками даже в переполненных маршрутках.

Ну а какие следствия это может иметь для книгоиздательского дела~--- поговорим в чуть позже.

\subsection{Часть 5}
Таким образом, в прошлой заметке мы выяснили, что смерть от книжного голода нашим книгочиям не грозит, и спасут от неё китайцы. Вернее, обеспечат материальную базу для спасения~--- собственно, уже обеспечили. Однако кроме инструмента для чтения необходимо иметь и объект оного~--- а тут уж не помогут ни китайцы, ни даже марсиане.

И тут самое время вспомнить про те миллиарды томов, о которых Елена Прудникова говорила как о <<где-то существующих>>. Мы с вами тоже обсудили этот вопрос и пришли к выводу (>>я думаю, что мы обсудим, и я, понимаешь, решим>>~--- Тимур Шаов), что если они и существуют, то та часть из этих миллиардов, которая могла бы заинтересовать читателя, в существующем виде, то есть в виде томов, для него практически недоступна.

Чтобы обладатели китайских читалок получили доступ ко всему океану русской (точнее, русскоязычной, в том числе и переводной) литературы, океан этот надо разделить на цифровые ручейки. То есть отсканировать тексты, <<распознать>> их, претворить в какой-либо машинно-читаемый формат и разместить результаты на общедоступных сетевых ресурсах.

И такая работа ведётся очень давно, как минимум со времён знаменитой Библиотеки Мошкова. Правда, до недавнего времени такие онлайновые фонды были предназначены для чтения или в распечатках, или в онлайне~--- с компьютерных мониторов.

Однако, как я уже говорил, одновременно с широким распространением читалок появились и библиотеки, специально (или преимущественно) ориентированные на их обладателей. То есть с преобладанием текстов в форматах fb2 и epub, хотя, как правило, они включают и html-версии для предварительного ознакомления в онлайне, а часто также версии в PDF, иногда~--- в DjVu и более иных форматах.

Объемы наиболее крупных таких фондов ныне исчисляются сотнями гигабайт. И там можно найти\dots ну, конечно же, не всё, что издано на русском языке, но 
\textit{buono parte}
: от литературных памятников Древнего Мира до современных боевиков, детективов и прочей фантастики. По поводу последних время от времени возникают конфликты с правообладателями (или теми, кто себя таковыми считает), но на этой теме мы сейчас останавливаться не будем.

Замечу только, что подобно магазинам, торгующим через Интернет бумажнымии книгами, существуют и системы онлайновой продажи электронных книг. Правда, как у них самих дело обстоит с правообладанием и, главное, обламывается ли что с этого ныне живущим авторам~--- для меня не вполне ясно. Был бы признателен за комментарии от лиц, которые в теме.

Пока же наша цель~--- поглядеть, как появление читалок сказалось (и, главное, может сказаться) на книгоиздательском деле. И тут впору опять вспомнить те две статьи, которые, собственно, и послужили поводом для сочинения начала цикла. Так, Анастасия Якорева считает, что упадок книжного дела объясняется


\begin{shadequote}{}
конкуренцией книг с электронными носителями и интернетом
\end{shadequote}

Про <<другие формы проведения досуга>> из продолжения этой фразы умолчим: мы, вроде, выяснили, что те, кто проводит его за


\begin{shadequote}{}
интернетом, торгово-развлекательными центрами, телевидением
\end{shadequote}

\noindentпри отсутствии оных не бросились бы читать книга, а лузгали бы семки, пили портвейн по подворотням и били морды друг другу на танцплощадках.

Впрочем, и мнение Елены Прудниковой о неконкурентоспособности цифрового контента вследствие технической отсталости вынужден поставить под сомнение. Особенно когда она ссылается на такого большого авторитета в этой области, как Дарья Донцова. Ну и на слова


\begin{shadequote}{}
Кто не верит~--- пусть попробует отправить электронную почту из любого российского райцентра.
\end{shadequote}

\noindentмогу возразить: те, кому это действительно надо, умудряются отправлять электронную почту (и даже активно участвовать в форумах) не то что из райцентров, но даже из глубины корякских тундр.

А вот слова

\begin{shadequote}{}
\dotsчто все больше людей пользуются интернет-версиями изданий только для того, чтобы определить, стоит ли книга того, чтобы её прочесть? И если стоит~--- покупают бумажный вариант.
\end{shadequote}

\noindentзаслуживают более подробного рассмотрения. Потому что по результатам <<партитурного чтения>> (оказывается, я всю жизнь по первости читал книги именно <<партитурно>>, только, подобно герою Мольера, не подозревал об этом) html-файла действительно определяется, стоит ли книгу прочесть. И если стоит~--- она скачивается с одной из упомянутых выше библиотек и переносится на читалку. А вот если оказывается, что её стоит не только прочесть, но и перечитывать~--- тут-то дело и доходит до покупки бумажного варианта.

Или~--- не доходит. Потому что, как обычно, возникает два варианта:

\begin{enumerate}
	\item книгу можно прочесть (а подчас и не дочитать до конца), но уж перечитывать её точно не захочется~--- и тогда покупать её бумажную версию никто не будет; 
	\item книгу нужно прочесть, и потом неоднократно перечитывать, может быть, на протяжении всей оставшейся жизни~--- но тогда её бумажной версии, скорее всего, купить не удастся, в силу причин, озвученных в одной из прошлых заметок. 
\end{enumerate}

То есть, в сущности, никакой конкуренции между между бумажными и электронными книгами не существует. В первом случае, скорее всего, книга была бы отвергнута и в бумажном виде~--- на стадии её пролистывания у стеллажа. Возможно, необоснованно, но вероятность купить бумажную халтуру намного порядков превышает возможность случайно нарваться\dots даже не на шедевр, а просто на приличное произведение.

Во втором же случае, как неоднократно повторялось на протяжении всего цикла, читателю всё равно не удастся купить понравившуюся книгу. Просто потому, что издатели не переиздают хороших книжек, которые, что называется, <<прошли>>. То есть они Сами Себе Злобные Буратины (синдром ССЗБ).

Так что же, со временем народные театры электронные книги вытеснят театры профессиональные книги бумажные? Есть такое мнение, и оно не лишено оснований. Но это, товарищи, не правильно. Рискну предположить, что многие, если не большинство, из читающих эти строки любят книгу не только как источник знаний (вместилище информации), а саму по себе, как явление, не побоюсь громкого слова, культуры (иначе они бы это не читали). Любят её в единстве содержания и формы. И было бы очень грустно, если бы это явление исчезло: значит, одним явлением культуры будет меньше. А их нынче и так не густо.

По моему глубокому убеждению, онлайновое чтение, электронные читалки и бумажные книги~--- вещи не взаимоисключающие, а взаимодополняющие. По крайней мере, должны ими быть. При одном необходимом условии: книгоиздатели и книгопродавцы это осознают, исцелятся от синдрома ССЗБ и смогут перестроиться соответственно изменившимся реалиям жизни.

В противном случае им просто настанет кирдык. То есть кирдык, конечно, настанет не всем: многие книгоиздатели смогут переключиться на производство туалетной бумаги (а что, тоже целлюлоза), а иные книгопродавцы займутся торговлей презервативами и памперсами. Но мы-то бумажные книжки потеряем напрочь. И, кстати, потеряем вместе со всеми потенциальным Пушкиными и Шекспирами, но это~--- совершенно отдельная песня.

А пока, в следующем номере нашей программы, рассмотрим, чисто гипотетически, вопрос, можно ли этого избежать. И если можно~--- то как.

\subsection{Часть 6}
На протяжении предыдущих пяти заметок мы долго и упорно пытались разрешить первый из вековых русских вопросов~--- кто виноват? Так что теперь нам осталась сущая мелочь~--- ответить на вопрос: что делать, блин? И главное, кому?

Понятное дело, что книгочиям ничего делать не надо, кроме выполнения своей боевой задачи~--- читать книги, бумажные или, за отсутствием таковых, электронные. Книгопродавцам тоже беспокоиться особо не о чем~--- на крайняк они могут превратить книжный магазин в салон по продаже элитной мебели или гаванских сигар имени Че Гевары (кроме шуток, было такое в Нерезиновой).

А вот книгоиздателям явно надо что-то делать~--- если они, конечно, не желают переключиться на производство туалетной бумаги.

Первое деяние очевидно~--- развивать онлайновую торговлю. Практика продажи компьютеров и комплектующих показывает, что удачливы в онлайновой торговле те, кто ранее удачно занимался торговлей оффлайновой. Потому что за кадром и той, и другой~--- одна и та же логистика. И с этим у давно существующих издательств всё нормально: склады есть, складские запасы~--- тоже, остаётся только наладить и приём заказов и своевременную доставку конечному пользователю.

Второе деяние~--- очевидно не менее: перестать печатать плохие книжки, а печатать только хорошие. Остаётся решить, какие книжки считать хорошими, а какие~--- плохими.

С классикой мировой и отечественной литературы всё понятно. Скажем, наше фсио, ас наш Пушкин, может нравиться или не нравиться. Но никто, даже такой его нелюбитель, как ваш покорный слуга, не может сказать, что книжки, написанные им~--- плохие.

Однако одной классикой жив не будешь. Что-то не знаю я людей, коротающих досуг исключительно за чтением Гомера, чередуемого с Шекспиром и даже Львом Толстым. разве что в тюрьме или в восемнадцатимесячном рейсе по Индийскому океану~--- в обоих случаях ничего другого просто нету. В нормальных же условиях у людей есть привычка читать что-нибудь новое (если, конечно, у них вообще есть привычка читать).

Утверждая на одной из предыдущих страниц, что все ныне издаваемые книги плохие, я, разумеется, преувеличивал: среди написанных за двадцать лет постсоветской власти встречаются и неплохие произведения, и даже откровенно хорошие. Вот только распознать их среди многометровых серий очень трудно. И в итоге читатели не покупают ни плохие книги, ни хорошие. По меткому выражению Елены Прудниковой, серии


\begin{shadequote}{}
\dotsстали братской могилой для десятков перспективных книг.
\end{shadequote}

Как избежать этого? И вот тут на помощь могут прийти электронные книги как средство первичного ознакомления как с конкретными произведениями, так и, главным образом, с образцами творчества данного автора вообще. Вероятность того, что читатель, знакомый с романами Имя рек по электронным версиям, и составивший о нём благоприятное впечатление, приобретёт его новую книгу в бумажном виде, возрастает многократно.

При этом издательству вовсе не обязательно полагаться на <<народные>> библиотеки электронных книг~--- они вполне могут заниматься их распространением и сами. И даже не обязательно бесплатно~--- я знаю немало людей, которые готовы покупать их за деньги. И даже делают это~--- в случае, если цена представляется адекватной, а форма оплаты~--- удобной.

Что до цены~--- поскольку себестоимость электронной версии изданной книги стремится к нулю (ведь она уже существует в машинно-читаемом виде и не нуждается в редподготовке, верстке и так далее), то и продавать их можно очень дёшево. И, при больших объёмах продаж, это дело вполне может оказаться прибыльным для издательства и приносить некоторый доход авторам. Кроме того, изучение спроса на электронные книги позволит выявить те из них, которые заслуживают переиздания в бумажном виде.

Ну а удобных форм оплаты нынче существует сколько угодно~--- от банковских карточек до Web Money и Яндекс-денег. Разве что оплаты путём отправки СМС, как мне не раз попадалось в онлайновых книжных магазинах, не надо: эта форма в народе стала прочно ассоциироваться с мошенничеством\dots

Правда, для выполнения описанных выше действий требуется два необходимых условия: предварительно распространяемые книги должны быть написаны, во-первых, и изданы~--- во вторых. То есть они касаются уже действующих авторов. А вот где взять новых? И каким образом издатель может выявить авторов, потенциальные  книги которых заслуживают публикации и читательского внимания? Рискну предложить один из вариантов по аналогии с изданием научной литературы.

Традиционно, ещё с советских времён, считается, что научное книгоиздание~--- занятие сугубо убыточное. Хотя мой старый товарищ Валентин Зинкевич, директор издательства Научный мир, издаёт научную литературу уже более 15 лет~--- и, судя по нашей недавней поездке в Ростов Великий и Кострому, помирать с голоду пока не собирается. А на проклятом Западе существуют научные издательства, успешно функционирующие уже не первое столетие.

С четверть века назад в журнале \textit{<<Природа>>} была опубликована статья~--- к сожалению, не помню ни автора, ни названия, и поиски её в онлайне успехом не увенчались. Написана она была сотрудником издательства \textit{<<Наука>>}, посетившим США в перестроечные времена, на заре эпохи всякого рода телемостов и обменов, на предмет постановки научного книгоиздания в этой стране.

И начинается эта статья с утверждения, что в США издание научной литературы~--- дело прибыльное. Не приносящее баснословных доходов, окупаемое в течении длительного времени, но, тем не менее, прибыльное. И достигается это тем, что соответствующие специализированные издантельства публикую, во-первых, хорошие книги, и во-вторых, книги долговременного спроса.

Каким образом осуществляется фильтрация? Как пишет автор статьи, в те далёкие, ещё доинтернетовские, времена в крупных специализированных издательствах существовали своего рода научно-литературные агенты. Они ездили по научным конференциям и всякого рода мероприятиям, ошивались в кулуарах, вступали в контакты с участниками, слушали околонаучные разговоры. И, таким образом, составляли представление о том, книги какой тематики могут быть интересны читателям, и кто из действующих научных работников мог бы такие книги написать. После чего издательство заключало с <<отфильтрованным>> автором договор~--- и, по прошествии должного времени, книга шла в производство и появлялась на прилавках.

Насколько я понимаю, агентам этим вовсе не обязательно было иметь специальное образование~--- да и при обилии научных направлений никакое образование не помогло бы в них ориентироваться. Скорее от них требовалась журналистская хватка и интуиция~--- это примерно такая же работа, какой занимается главный герой романа Саймака 
\textit{Почти как люди}
.

В наши дни изрядную часть такого рода работы можно выполнять через Интернет~--- и так, собственно, и делается. Причём не только <<у них>>, но и <<у нас>>. Так, издательство, с которым я сотрудничал на протяжении ряда лет~--- 
\textit{БХВ-Петергбург}
, специализирующееся преимущественно на выпуске компьютерной литературы, активно занимается мониторингом тематических ресурсов~--- сайтов, блогов \textit{etc}.~--- и, выявив авторов, работы которых могут заинтересовать читателей, обращается к ним с предложением написать книгу по мотивам их онлайновых материалов. А поскольку потенциальные авторы, как правило, уже достаточно известны в сети, некий предварительный <<народный>> маркетинг будущей книге уже обеспечен.

Ни о чём подобном в области художественной и публицистической литературы я не слышал. А ведь литературных сайтов и блогов нынче ничуть не меньше, чем околокомпьютерных ресурсов\dots

Мы рассмотрели проблему книжного дела по одну сторону баррикады~--- книгоиздательскую. Однако есть и другая сторона медали~--- авторская, о которой до сих пор практически не было. А ведь, как уже было сказано, прежде чем мнигу можно будет издать и, тем более, распространять, она должна быть написана.

Правда, говоря о проблеме авторов книг, нам придётся вступить на зыбкую почву копирастии, чего я хотел всячески избежать~--- уж

\begin{shadequote}[r]{Александр Галич}
Больно тема какая-то склизкая, \\
Не марксистская, ох не марксисткая 
\end{shadequote}

Да и жёвана и пережёвана в последнее время до такой степени, что остаётся её только выплюнуть.

Однако логика сюжета заставляет сказать несколько слов и по копирастическому вопросу~--- но уже в следующей заметке.

\subsection{Часть 7}
Последняя заметка была посвящена одному из гипотетических вариантов, по которому книгоиздатели гипотетически смогли бы выбраться из вовсе не гипотетического, а более чем реального кризиса. Однако для этого требуется некоторое участи и второй стороны книжного дела, а именно авторов.

Конечно, за время существования мировой литературы написано столько, что даже русскоязычную (оригинальную и переводную) её часть~--- читать не перечитать. Однако без притока <<свежей сочинительской крови>> действительно есть риск (опять-таки цитирую Елену Прудникову из обсуждения к упомянутой статье), что


\begin{shadequote}{}
народ начнет переходить на читалки, пусть читает старые книги. Не думаю, что кто-то станет задаром писать новые.
\end{shadequote}

И с этим нельзя не согласиться. Разумеется, автор будет писать и задаром, для типа самоутверждения~--- некоторое время. Пока очень кушать не захочется. А когда захочется~--- займётся чем-то другим. Например, переквалифицируется в управдомы. И управдомом будет скорее всего плохим. И таким образом общество потеряет потенциально хорошего автора. Но приобретёт кинетически плохого управдома. Который сможет проесть плешь всем тем, кто не платил за его книги.

Так что подумайте, ребята: когда вам пудрят мозги в домуправлении~--- может быть, это тот автор, за книжки которого вы некогда пожлобились заплатить?

В общем, скучно мне стало писать на эту тему. Любителям острых шахматных ощущений рекомендую ознакомиться вот с обсуждениями её на Юниксфоруме. Откуда вырву с мясом цитату~--- высказывание моего старинного знакомого, хорошего товарища, но непримиримого идеологического противника sash-kan'а:


\begin{shadequote}{}
здесь как раз выплывает на повестку дня порочная связь с трупом копирайта и именно в процессе насилования этого трупа такая торговля (и онлайн- и оффлайн-) и хиреет не по дням, а по часам.
\end{shadequote}

Трудно с этим не согласиться. Но~--- позволю процитировать сам себя, в чуть отредактированном виде:


\begin{shadequote}{}
Прежде чем прекратить насиловать труп копирайта, надо найти где-нибудь симпатичную живую девушку, которая давала бы сама. 
\end{shadequote}

Да ещё и на взаимовыгодных условиях~--- в том числе взаимовыгодных финансово. А, как поётся песне


\begin{shadequote}{}
Где ж её нам найти, Где ж её взять? Ведь её не найдёшь, Девятью пять.
\end{shadequote}

Так что пока, увы, альтернативой трупу копирайта является --


\begin{shadequote}{}
тихо, сам с собою, правою рукою.
\end{shadequote}

Ну и разумеется, как


\begin{shadequote}{}
устал работать правой~--- работай левой.
\end{shadequote}

В общем, тема мне осточертела, и я с ней завязываю. По крайней мере, на ближайшее время.

Тем не менее, я очень доволен~--- косвенно тема эта послужила поводом для знакомства с сочинениями Елены Прудниковой. Каковые, при полном идеологическом несогласии, читаю с большим удовольствием.

\section{Будущее линуксописательства} 
\begin{timeline}Февраль 9, 2012\end{timeline}
Последнее время ловлю себя на том, что, встречая некий вопрос на одном из посещаемых мной форумов FOSS-тематики, всё чаще отвечаю: да прочтите же вы, наконец, какую-нибудь толстую книжку про UNIX или Linux. Согласен, ответ не вполне политкорректный. Но за без малого десять лет обретания на окололинуксовых форумах реально стали раздражать вопросы, которые в качестве ответа требуют пересказа нескольких десятков страниц из любой книжки указанной тематики.

Однако со временем меня начала грызть совесть. Потому как такой ответ подразумевает, что не худо бы добавить, какую конкретно книжку следует прочесть. Или, хотя бы, обозначить диапазон рекомендуемых книжек. Желательно, конечно, на русском языке. И тут я призадумался\dots

Разумеется, в запасе всегда был вариант, потибренный у Бернарда Шоу. Который на вопрос одной юной девушки, что бы мне почитать умного, ответил:


\begin{shadequote}{}
Читайте меня
\end{shadequote}

Но увы~--- во-первых, я не Бернард Шоу. Во-вторых, рекомендовать читать себя всем и каждому я бы не стал: всё-таки я пишу скорее не man'ы, а романы, и рассчитаны они на тех, кто может отличить беллетристику от документалистики. А в-третьих и главных, последняя из моих книжек была издана в 2006 году. А в динамично развивающемся IT-мире пять лет~--- срок изрядный. Впрочем, к этому вопросу мы ещё вернёмся.

Ну ладно, оставляю себя, любимого, и начинаю вспоминать, что же я ещё рекомендовал в аналогичных случаях? В произвольном порядке из подсознания всплывают:

\begin{itemize}
	\item Кай Петцке, \textit{Линукс: от понимания к применению};
	\item Мэтт Уэлш сотоварищи, \textit{Запускаем Линукс}; 
	\item Виктор Костромин, \textit{LINUX для пользователя}; 
	\item Джеймс Армстронг, \textit{Секреты UNIX}. 
\end{itemize}


Поскольку ложная скромность никогда не была в числе моих многочисленных недостатков, рискну в этот список добавить и свой \textit{Доступный UNIX} (с указанной выше оговоркой). А вот свой 
\textit{Гуманистический Linux}~--- не включил бы. И не из скромности, опять-таки, а потому что это книжка совсем иного жанра.

Специфика всех книжек моего <<золотого списка>> в том, что они в принципе рассчитаны на совсем неподготовленного пользователя, так называемого ньюба. И в то же время это не книжки из серии \textit{``\dotsдля чайников''} или \textit{``Освой~\dots~за 5 минут''}. Нет, они требуют вдумчивого чтения, и за пять минут их <<ниасилить>>. Они предназначены для тех начинающих пользователей, которые как можно скорее хотят избавиться от своего <<начинального>> статуса и перейти в категорию пользователей действующих. Говоря словами сэра Артура Конана Дойла, они предназначены


\begin{shadequote}{}
Мальчикам, которые уже наполовину мужчины, и мужчинам, которые ещё наполовину мальчики.
\end{shadequote}

Причем предназначены они именно для пользователей~--- грезящим о карьере сисадмина или системного UNIX-программера следует читать более иные книжки. Первым, например, любую из книжек Эви Немет, вторым~--- 
\textit{UNIX изнутри}
 Юреша Вахалии и (или) 
\textit{Современные операционные системы}
 Эндрю Танненбаума.

Маленькое отступление: меня всегда умиляет, когда на форумах книжки Эви Немет рекомендуют по всякому поводу и без всякого повода. Не потому что они плохи, напротив. Просто в них содержится масса материала, абсолютно ненужного конечному пользователю. И в то же время нет очень многого, что ему жизненно необходимо.

Однако вернёмся к нашему (пардон, моему) <<Золотому списку>>. Сверяю годы издания входящих в него книжек. Получается соответственно:

\begin{itemize}
	\item 2000
	\item 2000
	\item 2002
	\item 2001
\end{itemize}



Да, кажется, я что-то упустил в этой жизни. Ведь наверняка за минувшее десятилетие было издано что-то ещё, кроме меня. Так что отправляюсь на поиски по онлайновым книжным магазинам. И что же там вижу?

А в том-то и дело, что в рамках поставленной задачи~--- практически ничего нового не вижу. Есть переиздание книжки Стахнова 
\textit{Linux}
~--- первое издание я не читал, но просматривал: оно не столько о Linux вообще, сколько конкретно о тогдашнем Red Hat. Так что нынешнее, подозреваю, в основном о Fedora. И изобилие книжек, посвящённых популярным (или активно популяризируемым) дистрибутивам, в первую очередь, конечно же, Ubuntu, далее, с большим отрывом~--- Mandriva, ну а всё остальное~--- мелочи.

Есть книжки, как отечественные, так и переводные, преследующие ограниченный круг задач, типа: 99 советов по Linux или 100\% самоучитель Linux. А есть, наоборот, книги, претендующие на всеохватность, подобно 
\textit{Всему Linux}
 и 
\textit{Полному руководству}
 Михаэля Кофлера (подозреваю, что это одна и та же книга, изданная в разных издательствах под разными названиями; если так~--- то это всё дериваты, идущие от начала нулевых, как минимум).

Наконец, есть многие множества книжек Колисниченко на самые разные темы~--- от новичка к профессионалу в любой системе, включая FreeBSD. И в любой сфере приложения сил~--- от shell-программирования до администрирования серверов.

Нет в нашем печальном списке только одного: свежих книжек, подобных тем, что я перечислил в <<списке золотом>>.

Хотя нет, ура! Вижу два приятных сюрприза~--- переиздание 
\textit{Запускаем Linux}
, теперь за авторством Дальхаймера и Уэлша, и 
\textit{Самоучитель. Linux для пользователя}
 Виктора Костромина, 2008 и 2011 года издания, соответственно. Однако при внимательном рассмотрении всё оказывается не так уж и замечательно. Первая книга~--- перевод достаточно старого 5-го издания (в оригинале с тех пор вышло как минимум 6-е). А вторая~--- вообще просто допечатка тиража одноимённой книжки десятилетней давности, той самой, что была включена в указанный выше список. Утверждаю это со всей ответственностью, потому что из общения с Виктором точно знаю, что он второго, модернизированного, издания не готовил.

Таким образом можно констатировать, что в рамках интересующей нас темы за последние 5 лет не было издано

\begin{itemize}
	\item ни одной новой книжки отечественных авторов; 
	\item ни одной новой переводной книжки; 
	\item ни одной <<осовремененной>> редакции книжек из моего <<золотого списка>> (как я уже говорил книжки Уэлша с Дальхаймеров и Костромина не в счёт). 
\end{itemize}



В то же время было издано достаточно большое количество книжек категории <<для чайников>> и дистрибутив-специфических руководств. Не могу утверждать, что они плохие, ибо ни одной не читал. Но дело в том, что они, в отличие от аналогичных книжек Windows-тематики, практически не имеют своего читателя. Ибо книжки тематики UNIX/Linux читают те, кто нуждается в углублённых знаниях. А минимума, необходимого для элементарного практического использования, легко нахватать в Интернете.

Для очистки совести я зашёл в свой близлежащий оффлайновый магазин, который являет собой нечто вроде слепка московского книжного рынка~--- в нём не найти раритетов, но и лавкой по распродаже дамских детективов он также не является.

Раньше в нём была цельная полка, заставленная книжками про UNIX, Linux и современных материи самой разной ориентации~--- от меня до Танненбаума. Ныне же я не обнаружил там ни одной книжки <<нашего круга>> вообще. Правда, в более <<масштабных>> оффлайновых магазинах (а их на всю Москву, грубо говоря, четыре) таковые обнаружить можно~--- но опять-таки в реализации <<для чайников>>. И ажиотажа вокруг них не наблюдается.

То есть мы имеем ситуацию, описанную \href{http://expert.ru/2011/11/15/radostnyij-vzlet-tsunami/}{Еленой Прудниковой}  и \href{http://gistoria.info/?p=160}{мной}  для литературы художественной:

\begin{itemize}
	\item книжки <<для чайников>> не читают, потому что Linux-чайники вообще не читают книжек; 
	\item ни одной новой переводной книжки; 
	\item ни одной <<осовремененной>> редакции книжек из моего <<золотого списка>> (как я уже говорил книжки Уэлша с Дальхаймеров и Костромина не в счёт). 
\end{itemize}

Книжки, аналогичные таковым из <<золотого списка>>, не читают, потому что их нынче просто нет. 

Надо учесть ещё, что из пяти причин <<нечтения>>, рассмотренных в статьях по указанным ссылкам, первая~--- завышенность цены~--- для компьютерной литературы вообще много более весома, нежели для литературы художественной: компьютерная книжка стоимостью от тысячи и более рублей на полках оффлайновых магазинов и в прайсах магазинов сетевых~--- дело обычное.

И не надо думать, что дело тут в жадности книгоиздателей. Копаясь в сети при подготовке этой заметки, в одном онлайновом магазине, не из последних, я обнаружил свою книжку 
\textit{Гуманистический Linux или Ubuntu и сородичи}  по цене\dots не буду называть абсолютных цифр, скажу только, что она втрое превышала отпускную цену издательства. Которую, как вы понимаете, я знаю абсолютно точно~--- мне с неё ройялти платят.

Так что обвинения авторов и издателей в жадности ни на чём не основано: ни тем, ни другим с этой книготорговой накрутки не перепадает ни копейки. Хотя именно они несут основные затраты~--- моральные и материальные, соответственно. Затраты же онлайновых книготорговцев минимальны. Страшно даже представить, сколько эта книжка стоила в магазине оффлайновом, которому надо ещё отбивать аренду торговых площадей.

В общем, сочетание завышенной цены с проблематичностью оправданности ожиданий в отношении содержания приводит к тому, что Linux-книжки не читают. Тем более, что фактор конкуренции со стороны онлайновых материалов тут ещё больше, чем в литературе художественной. Причём ни о какой контрафактности их не может быть и речи: все они распространяются абсолютно свободно, под рядом свободных лицензий типа 
\textit{Creative Commons} и, главное, под лицензией \textit{Человеческой Порядочности и Здравого Смысла}
.

Так что вымирание компьютерной литературы вообще и книжек тематики UNIX/Linux будет идти и дальше по нарастающей. Причём, в отличие от художественной литературы, никаких путей к исправлению ситуации не видно. Если бумажное производство изящной словесности может сколь угодно долго существовать на произведениях, сочинявшихся веками, то в нашей сфере самая-рассамая классика жанра нуждается в постоянном обновлении.

А для этого авторы книжек должны быть заинтересованы в обновлении, иногда коренном, своих старых книжек и в сочинении новых. А такой заинтересованности нет~--- ни материальной, ни моральной. Материальной~--- потому что ввиду маленьких тиражей ройялти автора составляет сущие гроши. Моральной~--- потому что книжки становятся никому не интересными. И большинство линуксописателей либо бросают это занятие вообще, либо, если они уже стали сочинителями-наркоманами, развивают собственные онлайновые проекты.

А теперь вернёмся к началу заметки и попробуем ответить на вопрос: что же читать начинающему линуксоиду, если новых ярких книжек по этой тематике мы уже никогда не увидим? Ответ прост~--- читайте всё те же книжки из <<золотого списка>>. Да, многое в них потеряло актуальность. Многие новшества, появившиеся в последние годы, в них не описаны. Но специфика UNIX и Linux такова, что их непреходящие ценности не подвержены старению. И книжка Кернигана и Пайка 
\textit{UNIX: универсальная среда программирования}
, впервые изданная на русском языке в 1992 году, в этом отношении не менее актуальна, чем 20 лет назад. Не случайно она была издана в 2003 году в новом переводе с того же оригинального издания 1984 года (под названием \textit{Unix. Программное окружение}).