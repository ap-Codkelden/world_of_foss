\chapter{Притчи и легенды}

\textsl{Эти притчи и легенды были услышаны в разное время, от разных людей и в разных местах. Может быть, в этой книжке они и ни при чём~--- да в них намёк: линуксоидам урок.}

\section{Батыр и мамлюк} 
\begin{timeline}Июль 19, 2012\end{timeline}

\textsl{Эту древнюю красивую легенду рассказал некогда Шурик: после приключений с <<Кавказской пленницей>> на следующие каникулы он поехал не на Кавказ, а в Азиатские степи.}\medskip

Мерялись силами два джигита. Один~--- степной батыр, сам взрастивший своего коня, саблю которому отковал его кровный брат-кузнец, а чепрак под седло вышили руки любимой женщины. И второй~--- мамлюк от двора великого подышаха, при дворе его выросший и вскормленный, и амуницию свою получивший с казённых складов. Не из рук подышаха, конечно~--- тот такими мелочами не занимался. А перепоручал их своему сорок пятому визирю, потому как вороват тот был и ни на что больше не годился.

Результат предсказуем: победа осталась за степным батыром. Сразу и окончательно.

А вот если чуть изменить условия задачи. И против батыра нашего поставить\dots да, мамлюка с подышахова двора. Но двора того подышаха, который знал, что сила его~--- на мамлюках держится. И потому ни обучение, ни снаряжение их не перепоручал проворовавшимся визирям, а всем занимался лично.

И тут результат был бы уже не столь однозначен. Да и не допустил бы умный подышах такого поединка. Потому что и батыра, и мамлюка к делу приставил бы. К тому, для которого каждый лучше подходит. Первого~--- коней у соседей угонять, чтобы свежая кровь в табуны текла. А второго~--- в строю стоять, когда дело до серьёзной заварушки дойдёт.

Правда, в какой же стране такой подышах найдётся? Я такой страны не знаю. И в этом суть притчи, рассказанной нам Шуриком\dots

\section{Шашлык и люля} 
\begin{timeline}Август 31, 2012\end{timeline}

\textsl{Это тоже одна из рассказанных Шуриком легенд, которая объясняет, почему мало какая из открытых систем была доведена до законченного решения для конечных пользователей.}\medskip

В одной далёкой стране бегали по горам дикие бараны, архары они называются. И были они свободны, а охота на них~--- доступна для всех, без различия рода, звания, пола и даже сексуальной ориентации. Разумеется, обладавших должными умениями.

И жил в тех горах один вольный джигит, умениями такими обладавший. Ходил он по горам со своим карамультуком, и баранов тех стрелял. Потому что ему это интересно было. Ну и свежую печёнку архаров очень он уважал. А с остальным мясом не знал, что делать. Большую часть соседям по аилу раздавал, что-то продавал задёшево, чтобы на порох да свинец хватало. Не интересно ему было коммерцией заниматься~--- ему лишь бы по горам походить\dots

Так и было, пока не встретил он в кишлаке, в долине, одного мясника. Тот и предложил ему~--- а продавай мне все излишки мяса, я его на шашлык разделаю, отвезу в город и продам знакомому, владельцу ошхоны (столовки по нашему). И завернём мы на свободных архарах вот такой бизнес~--- тебе не только на порох и свинец хватит, но и на подарки для твоей пэри.

На том и порешили. Продолжает джигит охотиться, мясо, что не съедает, отвозит в кишлак и мяснику продаёт. Не дорого продаёт~--- не нужно ему было много денег: что с ними в горах делать? А мясник то мясо на шашлык разделывает и в город отвозит, да своему приятелю продаёт. А тот уже из него шашлык жарит~--- да такой, что к нему в ошхону весь город сбегается, да ещё и из других городов приезжают~--- специально шашлыка из архара отведать.

И всё было замечательно. Да вот однажды решил наш мясник, что неправильно это, туши на шашлык разделывать. Почему~--- только Аллаху ведомо. Может, подумал, что не нужно народу шашлык грызть, для зубов это вредно. А может, жадность его обуяла~--- ведь в шашлык требухи какой с жилами не подмешаем. Взял он, и порубил всё мясо-свежатинку в нежный мясной фарш. Привёз в город своему приятелю и говорит~--- ты с шашлычком-то завязывай, а давай торгуй теперь люля-кебабом.

Ошханщику делать нечего~--- к тому времени весь его бизнес на шашлыке из архаров строился. Ну и попробовал люля-кеба делать. А народ собрался и говорит: не нужен нам твой люля-кебаб, без вкуса и запаха, мы такой же в любой другой ошхоне поесть можем. И перестал он ходить в ту ошхону.

Ошханщик наш и подумал: как можно иметь дело с человеком, который сегодня мне шашлык из свежих архаров привозит, завтра~--- фарш на люля-кебаб, в который неизвестно что подмешано, а послезавтра вообще собачатину за мясо впаривать попробует. И в чайханщика переквалифицировался. А может быть, как злые языки говорили, в базарного сводника.

И ведь действительно, мясник сначала пробовал собачатиной да мертвечиной торговать. Пока однокишлачники про то не прознали и крепко его не побили. И тогда уехал он в очень далёкий город, в чужой стране. И стал там шаурмой приторговывать, да такой, что непонятно было, лаяла его шаурма раньше, или мяукала. А может, и вопросы глупые задавала\dots

Ну а джигиту уже на свинец с порохом не хватает, не то что на подарки~--- инфляция. А ни к чему другому, кроме как по горам ходить, у него душа не лежала. И пошёл он в соседние горы, к тамошнему беку в охотники наниматься. А беку охотники были без надобности~--- для него специально фазанов фисташками выкармливали. А надобные были ему аскеры, солдаты то есть. И нанялся наш джигит в войско бека, стал по его горам ходить, да врагов бека стрелять и резать.

Вот что бывает, когда свежее мясо в мясной фарш рубят\dots

\section{Легенда о советском рудознатце} 
\begin{timeline}Май 22, 2011\end{timeline}

\textsl{Эту старинную легенду я услышал много лет назад в одном рудничном посёлке. Я не скажу, в каком именно~--- чтобы не быть несправедливым к другим рудничным посёлкам, где вы могли бы услышать точно такую же легенду\dots}\medskip

Подобно историям Шурика и Юрия Деточкина, неизвестно, в каком именно из уголков нашей некогда необъятной Родины происходили описанные события, и происходили они ли вообще. Как я уже сказал, мне довелось услышать её на рубеже 60-х и 70-х годов прошлого столетия. И услышать от нескольких людей, вполне информированных, но, тем не менее, более чем заслуживающих доверия.

Все услышанные мной версии совпадали в главном, хотя и расходились в деталях, что позволяет предполагать, что в основе этой легенды лежит событие, действительно имевшее место быть. Расхождения же между версиями, как мы увидим ниже, не менее показательны, чем совпадения.

Судя по оценкам времени, началась эта история не позднее начала 60-х годов. Жил да был в одном из населённых пунктов бывшего Советского Союза (как я уже говорил, неизвестно какого) некий мужик. Населённый сей пункт, как уже говорилось, скорее всего, был рудничным посёлком.

А работал тот мужик, скажем условно, слесарем~--- хотя, возможно, столяром или фрезеровщиком. Важно то, что никакого специального образования он не имел, и профессия его не имела никакого отношения ни к геологии, ни к горно-рудному делу. Хотя из дальнейшего можно предположить, что работал где-то в этой сфере или около. Вероятно также, что профессия его была скорее по металлу, нежели по дереву.

И в один прекрасный день он чисто из любопытсва занинтересовался~--- а где же добывают те полезные металлы, с которыми он имеет дело каждый день. И начал он собирать сведения (инфу, как сказали бы мы сейчас) по этой теме. Гугла тогда не было, как и Интернета, и сведения свои он черпал из газет. Сначала~--- из поселковой многотиражки (что подкрепляет гипотезу, что дело происходило таки не просто в посёлке, а в посёлке рудничном). Не пренебрегал он и газетами центральными, типа \textit{<<Известий>>} и \textit{<<Труда>>}
.

Получив, таким образом, общее представление о горно-рудной базе во всесоюзном, так сказать, масштабе, он перешёл к газетам республиканского, областного и районного уровней. А там добрался до поселковых газет и многотиражек более иных, нежели собственное, предприятий.

Почему я и думаю, что какое-то отношение к горно-рудной промышленности он всё-таки имел: много ли простых советских трудящихся знали о самом факте существования Хайдарканской Геолого-разведочной партии или Навоинского Горно-обогатительного комбината? Не говоря уже о знании того, что практически каждая ГРП, рудник или ГОК выпускают свои газеты. В которых бодро рапортуют о своих трудовых успехах~--- приросте запасов, рекордам по добыче, и так далее. И на которые, тем не менее, может подписаться любой советский трудящийся.

Разумеется, никто, кроме сотрудников соответствующих предприятий, на них не подписывался~--- хватало добровольно-принудительного подписного минимума для коммунистов и беспартийных. А читало эти многотиражки~--- и того меньше народу, преимущественно те, кому это было положено по должности. Даже от отсутствия чтива они популярностью не пользовались: вопреки мифам нашего времени, как раз на заброшенных в тайгу или пустыню рудниках проблемы с печатным словом не было, можно было подписаться и на толстые журналы, и на собрания сочинений классиков, и даже на детективы из библиотеки \textit{<<Подвиг>>}.

Ну а наш герой изучал все эти газеты доскональнейшим образом. И извлечённую информацию сводил воедино. Сначала в терадках, потом, для пущей наглядности, начал выносить на карту Советского Союза. Масштаба и размера имевшихся в продаже карт ему скоро стало не хватать, и он собственноручно изготовил нечто вроде макета такой карты, на которую нанёс множество горно-рудных объектов (легенда гласит, что чуть ли не все). И не только местоположение самих рудников и ГОКов, но и каким-то образом умудрился привязать к ним всю текстовую и численную информацию, щедро черпаемую им из газет. Создав, таким образом, нечто вроде рукотворной ГИС~--- во времена, когда таких систем в компьютерном исполнении не было ещё и в проекте.

Вот так герой наш много лет жил-поживал, газеты читал, да ГИС апгрейдил с регулярностью, до которой современным софтостроителям~--- что до Пекина раком. Хотя и делал это в свободное от основной работы время. И был, по-видимому, счастлив. Но в один прекрасный момент заинтересовались им в высших сферах.

Почему? Тут начинаются первые расхождения. По одной из версий, кому-то из соседей показалось странным, что простой советский слесарь (повторяю, профессиональная принадлежность его определена условно) кипами и пачками получает газеты со всего Союза. И он сигнализировал о столь необычном поведении в компетентные органы~--- мол, не шпион ли?

Мне это кажется маловероятным: ведь это были времена далеко не сталинские, а как минимум позднехрущовские, а то и раннебрежневские, когда энтузиазм миллионов угасал на глазах. Да и у компетентных органов другие заботы появлялись во множестве.

Согласно второй версии, герой наш случайно познакомился в пивной с заезжим геологом и показал ему результаты своей многолетней работы. Геолог восхитился (субверсия~--- перепугался) и доложил об этом руководству. Это уже больше похоже на правду, но также не очень вероятно. Мощь геологов по части выпивки общеизвестна, но сколько же надо принять на грудь, чтобы забросить свои непосредственные дела и бежать докладывать руководству о каком-то слесаре. Мне, например, столько не выпить\dots

Третья версия~--- наиболее вероятна: герой наш не захотел быть собакой на сене, а решил поделиться своими результатами с народом. Каким образом мог поступить человек в его положении? Только написав куда следует. И вот в вопросе~--- куда именно следует,~--- существует, опять-таки, две субверсии.

Субверсия первая~--- что написал он в Министрество геологии СССР. Резонно предположив, что Минчермет занимается чёрными металлами, Минцветмет~--- цветными, Средмаш~--- радиоактивными, а в его ГИС охвачено всё горно-рудное богатство страны (горючими полезными ископаемыми он почему-то брезговал~--- вероятно, инстинктивно), а это должно проходить по ведомству Мингео. Логично и правдоподобно, но в сущности сводится к предыдущей версии, ибо ожидать реакции от министерских чиновников~--- ещё более наивно, нежели от заезжего полевика-геолога. Да и выпить они могут существенно меньше

И потому самое вероятное, что наш герой написал в ЦК КПСС. Тем более, что столь дотошный и аккуратный человек наверняка и на своей основной работе был передовиком производства, ударником коммунистического труда, а следовательно, вполне мог быть принят в Партию по разнарядке.

Как это ни покажется странным нынешнему поколению, в ЦК КПСС на письма трудящихся, особенно коммунистов, реагировали: сам был свидетелем одного такого случая и практически участником~--- другого, но об этом не сейчас. Реагировали, спуская вниз по партийной и ведомственной линиям предписания~--- разобраться и об исполнении доложить. После чего бюрократическая машина начинала крутиться~--- не очень быстро, но уже необратимо.

В нашем случае эта машина докрутилась до того, что приехала полномочная комиссия для ознакомления с результатами работы нашего героя (суб-субверсия~--- его самого вызвали в Москву вместе с его ГИС на заседание такой комиссии). И вот тут комиссия действительно восхитилась и испугалась: повторяю, на его карте были нанесены практически все действующие горнорудные объекты страны~--- открытые, не совсем открытые и вовсе секретные, то есть золото-урановые. Да ещё с пространственно привязанными сведениями о запасах, бортовых содержаниях, объемах добычи и так далее.

И, естественно, рукотворная ГИС была реквизирована~--- или, говоря политкорректней, взята для проверки в соответствующей организации.

Относительно финала истории также мнения расходятся. Согласно одним свидетельствам, мужика этого определили куда следует~--- не по договору, а по приговору. Я в это не верю~--- времена, повторяю, были уже не те.

Вторая версия гласит, что нашему герою без защиты диссертации (и, повторяю, без высшего образования) присвоили учёную степень кандидата геолого-минералогических наук, забрали в Москву и определили старшим научным сотрудником в один из закрытых институтов, дабы он продолжал там свой скорбный труд на благо Отчизны.

В самом по себе факте такого карьерного взлёта нет ничего невероятного, и примеры тому известны. Однако геологический мир тесен, и в этом случае, не смотря на всю секретность предполагаемого НИИ, о его дальнейшей судьбе имелись бы вполне достоверные сведения (как и обо всей этой истории в целом).

Наконец, третья версия, наиболее вероятная: нашего героя сердечно поблагодарили за проделанную работу и выписали ему премию в размере годового (условно) оклада. После чего он, лишённый дела своей жизни, благополучно спился и умер.

Слабым местом любой из этих версий (как и всей истории) является отсутствие сведений о дальнейшей судьбе первой в истории горно-рудной ГИС. Наиболее вероятно, что эту рукотворную ГИС задвинули дальний угол какого-нибудь ведомства или института. И со временем выкинули~--- при первой же инвентаризации, как не числящуюся на балансе.

Как говаривали древние, если эта легенда и не правда~--- то хорошо придумана. Хотя и немного грустно. Но ведь все древние красивые легенды немного грустны, не так ли?

\textsl{Post Scriptum. Читатель вправе спросить: а какое отношение эта легенда имеет к миру FOSS? Самое непосредственное~--- отвечу я. Ибо показывает, что потребность в информации, буде она у человека возникнет, удовлетворяется всегда~--- вне зависимости от того, открыты или закрыты её <<исходники>>. Нужно только желание информацию получить и умение с ней работать.}


\section{Притча о верхних людях} 

Как-то подумалось: всё больше и больше старых моих товарищей к верхним людям уходит. И попробовал я представить~--- какие они, верхние люди? Да так представить, чтобы и принципами своими атеистическими не поступиться, и про естественно-научное образование не забыть. Ну и чтоб братву от агностизма и деизма не обидеть.

И увиделась мне такая картина.

Ночь. Тайга глухая, непролазная. Прогалинка. Посреди неё костерок дымокурный. Вкруг него верхние люди сидят. Те, что когда-то в нашем мире жили, и кого ты уважал. И чьему суду подчинишься~--- без прокурора и судебного исполнителя, сам.

Водку они там у костерка пьют, и о делах своих верхних разговаривают. И ничего им больше не надо\dots

Но случается так, что кто из них устанет, от костра отойдёт, на отдых. И тогда они из нашего мира зовут. Не абы кого~--- зовут тех, кто для их компании подойдёт. Верхние люди -- они ведь мудрые, и много чего им ведомо\dots

А когда позванный приходит, они его очень спрашивают -- куды там гестапам всяким и прочим\dots По твоим понятиям спрашивают. И всё-всё тебе припомнят -- где и когда слабину дал, что был сделать должен, а не сделал, кому помочь надо было -- а не помог.

Много чего они вспомнят, и оправданий, типа что ну никак иначе не получалось, не примут. Потому что они верхние люди, и много чего им ведомо\dots

А потом за всё отчехвостят тебя в хвост и в гриву, и делом, и словом -- тем словом, что сильнее дела бьёт. Так что мало не покажется\dots

А потом скажут~--- да, сукин ты кот, много за тобой хрени всякой, и про всю мы знаем, и отмазки все твои знаем, что иначе, мол, не мог. Но ведь подлянок ты не делал, а за всю хрень сполна сам расплатился\dots

И подвинутся, и место у костерка дадут, и стопаря накатят, и оставят средь своих~--- тех, что уже там, и кто из нашего мира ещё придёт.

А если суда их не пройдёшь~--- оправят в рай, или там ад какой, пускай, мол, там с тобой разбираются, по ихним законам.

Если бы верхние люди хотели, назначили бы в мире нашем пророка, и религью такую учредили. Мол, в верхних людей веришь -- к ним попадёшь. А не веришь -- невесть где сгинешь. Но они не хотят\dots

И потому зовут они тех только, кто и так к ним придёт, нету кому пути ни в ады, ни в раи, ни в прочие кущеря.

Тех, кто судит себя сам, без бога и чёрта, и без прочего глюкавого. А они ведь только приговор формулируют\dots