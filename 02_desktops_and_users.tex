\chapter{Десктопы и пользователи}

\textsl{В этой рубрике объединены заметки, посвящённые в основном этологии и/или этнографии сообщества FOSS (уж кому какой термин больше нравится применительно случаю). В истории предшествовавшей литературы наиболее близким к ним жанром является анималистика. Та самая, блестящие образцы которой оставили нам Эрнест Сэтон-Томпсон, Чарлз Робертс и Джералд Даррелл. Не хочу равнять себя с великими именами. Но я старался учиться у них. А также у Конрада Лоренца и Нико Тинбергена~--- великих этологов.}
\section{Рецепты против принципов, или Почему компьютер~--- не видак} 
\begin{timeline}
2003, осень
\end{timeline}

Обсуждение темы <<Зачем Linux дома>> продолжает давать мне сюжеты для сочинений. В частности, я наконец-то смог сформулировать, почему сравнение в удобстве использования компьютера и видеомагнитофона всегда казалось мне некорректным. Да все из-за той же подмены понятий.

Действительно, что такое видеомагнитофон (магнитофон просто, телевизор, радиоприемник~--- нужное подчеркнуть)? Это~--- исключительно инструмент для потребления. Потребления продукции, созданной кем-то другим (случай продукции собственной рассмотрим чуть ниже). Причем продукции, в отличие от продуктов питания, не жизненно важной. То есть потребляемой исключительно с целью развлечения (случай профессиональных теле- или радио-обзирателей также не рассматривается).

А потому развлекаемый, если так можно выразиться, пользователь-потребитель вправе требовать от такого инструмента минимума сложностей в использовании. Не желает он понимать принципы работы привода видака или передачи сигнала на телевизор. Потому как иначе это превратится в работу, а его задача~--- как раз отдохнуть от своих (в том числе и производственных) проблем за любимой <<Великолепной семеркой>> или <<Тремя мушкетерами>> (нужное вписать).

Более того, пользователь-потребитель видеомагнитофона имеет полную возможность обойтись без всяких знаний о его устройстве. Потому как его потребность~--- не забивать голову техническими подробностями,~--- сполна удовлетворяется производителями такого рода техники. Иначе её просто не стали бы массово покупать~--- вспомним тех же радиолюбителей с паяльниками, много ли их было в процентом отношении? Чай, не кусок хлеба, обходились Маяком на фабричной Спидоле (случай профессиональных шпионов или убежденных диссидентов не рассматриваем также).

Так что развлекаемый пользователь вполне может обойтись минимумом самых простых рецептов, как то: вставить кассету, нажать кнопку 
\keystroke{Вперед}, после просмотра нажать кнопку \keystroke{Eject}. Хотя и тут требуется некий минимум подготовки~--- например, какой стороной кассету засовывать. Иначе возникнет нештатная ситуация, требующая уже дополнительных рецептов (типа~--- использовать деревянную линейку) и дополнительных знаний (на что этой линейкой давить).

Однако все это~--- лишь до тех пор, пока происходит пассивное потребление кем-то созданной продукции. Если же пользователь видика/телевизора, насмотревшись передач типа <<Сам себе режиссер>> или там крутой порнографии, решит создать собственный шедевр в том же духе, ситуация тут же меняется.

Во-первых, ему потребуется инструмент для созидания, а не потребления. Сиречь~--- кинокамера. Во-вторых~--- умение ею пользоваться. Под которым нужно понимать не только владение её интерфейсом (грубо говоря, знания тех же кнопок запуска-останова), но и умение снимать. То есть~--- понимание перспективы, освещенности, навыки создания какого-никакого сюжета. Думаю, все согласятся со мной, что мало что может быть страшнее видеоролика, снятого по методу <<что увидел, то и снимаю>>. Помнится, меня всегда доставал просмотр экспедиционных слайдов~--- обычно именно таким образом он и осуществлялся.

А уж если такой пользователь в итоге изберет видеосъемку своей профессией~--- тут ему потребуется и многое другое. В том числе не лишними окажутся знания о физических принципах фото- и видеосъемки. Не случайно лучший из лично известных мне фотографов, Владимир Родионов, по образованию~--- физик-оптик с неслабым опытом инженерной работы в очень нестандартных условиях (см. \href{http://www.rwpbb.ru}{www.rwpbb.ru}).

Да и на видеомагнитофон такой пользователь (а он ведь по прежнему~--- пользователь видеомагнитофона по определению, не так ли?~--- ибо производством оных не занимается) будет смотреть совершенно иначе. Для него это будет уже не инструмент потребления, а аппарат, способный подчеркнуть или затушевать собственное мастерство (или~--- отсутствие такового). То есть превратится в деталь производственного цикла, в орудие производства.

И тут требования к удобству интерфейса даже видеомагнитофона отступают на второй план. Мало горя будет в том, что на конкретной модели кнопка запуска расположена не совсем там, где хотелось бы, если это~--- единственный (или единственный доступный финансово) инструмент, способный обеспечить, скажем, требуемую контрастность или яркость.

Вернемся, однако, к рецептам и принципам. Пока пользователь видеотехники остается чистым потребителем, он вполне может обходиться минимумом рецептов. Однако первые же попытки креативного характера приводят к резкому возрастанию потребности в HOW TO: куда встать, под каким углом держать, откуда направить свет, и так далее. И здесь перед ним два пути: экстенсивный или интенсивный. Первый~--- копить все те же практические рецепты, наработанные эмпирически. Однако скоро а) их становится очень много и б) все рецепты по определению охватывают только стандартные ситуации, ничего нетленно-непреходящего с их помощью не создашь. И приходится нашему пользователю волей-неволей обращаться к истокам~--- то есть базовым принципам. Бия себя по голове за то, что в школе не читал внимательно <<Физику>> Пёрышкина\dots

Теперь обратимся к компьютерам. В отличие от видеомагнитофонов, они изначально создавались не для потребления, а для креатива (чего бы то ни было). И многими по сей день используются главным образом для работы. В том числе, а то и в первую очередь~--- для работы дома. Помнится, для меня первая персональная персоналка обеспечила именно возможностью не ходить на службу (благо, был безработным, и ходить было некуда).

Конечно, компьютер имеет и потребительски-развлекательную функцию. Каждый пользователь, профессионально с компьютером связанный (не в смысле~--- профессиональный компьютерщик, а~--- выполняющий свою профессиональную работу главным образом на компьютере), отнюдь не прочь послушать музыку или посмотреть киношку без отрыва <<от станка>>. Однако функция эта не просто вторична по времени, она не принципиально важна и по значению. Думаю, если просмотр видеоролика будет мешать работе профессионально критичной программы, любой профессионал предпочтет смотреть его по видику (тому самому, развлекательно-потребительскому).

Очевидно, что для профессиональной работы важнее функциональность, а не удобство использования (мне безразлично, насколько удобно я не могу выполнить свою задачу). А развлекательная составляющая~--- вроде бесплатного приложения.

Говорят, что есть и чисто развлекаемые пользователи-потребители компьютеров, покупающие навороченные P-4 вместо музыкальных центров или домашних видеотеатров. Хотя мне таковых видеть и не доводилось. Однако рискну предположить, что это в основном~--- именно люди, профессионально с компьютерами связанные (или те самые энтузиасты цифрового контента), иначе не вижу тут ни практической, ни финансовой целесообразности. Может, я и отстал от жизни, но мне кажется, что нормального качества телевизор стоит дешевле, чем высококлассный монитор (а только на таком просмотр фильма и доставит удовольствие истинному ценителю).

Это я всё к тому, что даже и в развлекательном аспекте компьютер остается где-то инструментом креативным, со всеми вытекающими последствиями. Кроме того, он при этом практически не теряет своего универсализма. Если на видеомагнитофоне слушать Баха, скажем так, несколько затруднительно, а на CD-плейере, напротив, фильмы Бессона обычно не смотрят, то тот же мультимедиа-компьютер призван выступать в обоих качествах (а зачастую, повторюсь, на нем даже еще и работают). И потому ожидать, что он будет так же прост в обращении, как монофункциональный развлекатель~--- по меньшей мере излишне оптимистично.

Вернее, чисто развлекательную функцию компьютера тоже можно упростить до состояния видеомагнитофона. Замечательным примером чему служит Linux-дистрибутив под названием MoviX. Это~--- один из т.~н.~LiveCD, то есть система на компакте, способная с оного не только запускаться, но и полноценно функционировать. Функции MoviX'а, правда, весьма ограничены. А именно, он умеет только крутить мультимедийные файлы (видео и аудио). Но зато умеет это очень хорошо. И, главное, ничуть не сложнее, чем бытовой агрегат соответствующего назначения.

Так что достаточно легкого движения рук~--- вставки диска MoviX в привод и комбинации из трех пальцев,~--- чтобы волшебным образом превратить тысячебаксовый компьютер в элегантный видеомагнитофон или CD-плейер красной ценой в пару сотен. Благо, и обратное превращение ничуть не сложнее\dots

Но упростить производственную-то функцию компьютера~--- все равно не удастся (работать вообще довольно трудно, как говорил, если не ошибаюсь, Антон Палыч Чехов). И потому лозунги типа <<С выходом Windows~3.0 (3.1, 95, 98, ME, XP), обращаться с компьютером \textit{наконец-то} стало также просто, как с бытовой техникой>>, которые я лично слышу уже более 10 лет,~--- лукавы, как минимум, вдвойне. Во-первых, это та самая подмена понятий, с которой я начал заметку~--- если под обращением понимать не только развлекательную сторону, но и производственную (см. вышеуказанный тезис А.П.~--- создавать видеофильмы никогда не будет также просто, как их просматривать). А во-вторых, сама регулярность появления таких лозунгов (я не случайно выделил слова наконец-то) вызывает подозрения, что и с развлекательной стороной все еще сохраняется некоторая напряженка.

В частности, за что еще не люблю Windows,~--- за то, что она, обещая избавление от всех и всяческих проблем хотя бы в потребительском аспекте, своих обещаний не выполняет (впрочем, это~--- к вопросу о злокозненности). Давеча встала задача просмотреть (в Виндах) CD'шки с записью <<Прогулок с динозаврами>>. И вы думаете, это было также просто, как кассету на видике? Хрен в ухо, как сказали бы в Одессе (в варианте для дам, разумеется): штатный MediaPlayer (из ME) смотреть их отказался, собственный плейер с первого диска сначала установил сам себя, и только потом начал показывать, да еще в окошке. Разворот в полноэкранный режим был интуитивно не ясен, выход из оного~--- только тремя пальцами, на произвольном масштабировании машина (не из слабеньких) просто висла. А перед просмотром второго диска тот же плейер установил сам себя второй раз. Ну как тут не вспомнить про

\medskip\texttt{\$ xine имя\_файла}\medskip

или аналогичное действо с Mplayer'ом\dots Конечно, возможно, что это руки у меня кривые, и нужно было внимательно прочитать инструкцию. Но не от RTFM'а ли нас обещал избавить отец Выньдоуз? Впрочем, это~--- опять вдогонку предыдущей статье.

А мы тем временем вернемся к рецептам и принципам. Худо-бедно, но с развлечениями на компьютерах можно справиться посредством первых. А вот с производством любого рода? Рассмотрим это на примере более-менее близкой мне области~--- работы с текстами, претендующими на оригинальность (то есть выдумываемыми из головы).

Подобно нашему видеолюбителю, профессиональный текстовик, пересев с пишущей машинки (или с письменного стола со стопой бумаги и паркеровской ручкой) за компьютер, быстро узнает множество простых рецептов, как то: нажав клавишу \keystroke{Insert}, он может забить неправильно введенный текст (как забивочные листки на машинке, только проще), клавишами \keystroke{Delete} или \keystroke{Backspace} можно уничтожить лишнюю букву (в отличие от замазки, без следов), и так далее.

Жить нашему текстовику становится лучше, становится веселей. Но еще не до конца. Потому как он узнает об управляющих последовательностях, с помощью которых может мгновенно переместиться в требуемое место текста и продолжить набор-редактирование, о глобальном поиске и замене, об автоматической проверке правильнописания, и о многом, многом другом.

Однако суть работы текстовика при этом не меняется, как не меняется и стиль мышления. И то, и другое по прежнему линейно, оригинальный текст создается от начала и до конца, как и за пишущей машинкой, расширяются только возможности возврата к написанному и внесения в него корректив. И потому следующая мысль~--- а не структурировать ли текст изначально, внеся соответствующую разметку рубрик, подрубрик, параграфов? И вот это~--- уже скачок качественный, ведь для структуризации (ненаписанного еще) текста последний весь, целиком, уже должен быть вот здесь, в~\dots~(ну, сами знаете где).

К слову сказать, современные ворд-процессоры WISIWIG-типа пользователя к такой структуризации отнюдь не стимулируют. За ненужностью народу, вероятно. В одной книжке про Word мне как-то встретилась фраза, что стилевая разметка~--- это штука очень сложная, которая по силам только высоким профессионалам (в Word'е же, вероятно). Простым людям, видимо, проще вручную придавать заголовку каждой главы кегль и начертание (и при этом помнить, какое оформление было придано Главе 1, какое~--- Главе 2, и так далее). Или центрировать заголовки клавишей \keystroke{BackSpace}.

Впрочем, я опять отвлекся, перестройка мышления текстовика~--- тема совершенно отдельная. Вернемся к принципам. С созданными и отредактированными текстами подчас приходится продолжать работать~--- делить на фрагменты, соединять, извлекать части одного документа и вставлять в другой. И вот тут-то и обнаруживается неэффективность рецептов.

Возьмем простую задачу~--- создание единого документа из нескольких существующих, причем~--- включенных в определенной последовательности. Мне этот пример кажется очень показательной, и я не устаю его повторять. Можно: открыть документ 1, перейти в его конец, щелкая мышью по контекстному меню, (или даже воспользовавшись существующим рецептом~--- макрокомандами с привязанными к ним горячими клавишами) вставить туда документ 2, и так далее. Быстрее, конечно, чем подклеивать бумажные листы силикатным клеем, но все ли это, что может компьютер?

Нет, если узнать (или вспомнить) несколько принципов более общего порядка. Любой документ~--- суть файл. Любой файл может быть выведен на устройство вывода (экран или, скажем, принтер). А любой вывод может быть перенаправлен с одного устройства вывода на другое. Но ведь любое устройство вывода~--- тоже файл. Значит, вывод любого файла может быть перенаправлен не в файл устройства, а в другой, например, текстовый, файл. Остается только отыскать команду, которая это сделает. И такая команда легко находится~--- это 
\texttt{cat}. В результате конструкция

\medskip\texttt{\$ cat file1 file2 file3 > file-all}\medskip

создаст нам результирующий документ в один присест. Причем составные части его расположатся в той последовательности, в какой нам нужно. И, если мы заранее озаботились тем, чтобы структура наших рабочих файлов хоть как-то коррелировала (не обязательно плоско-линейно, но по какому-либо принципу) со структурой итоговой работы (а это та самая структуризация мозгов, о которых я говорил чуть раньше)~--- результирующий документ будет структурирован должным образом, причем без всяких дополнительных усилий.

Задача обратная~--- поделить наш правильно (!) структурированный документ на отдельные части в соответствие с его внутренней структурой (например, разбить книгу на главы, на предмет раздельного использования). Для автоматизации процесса нам достаточно знать о другом относительно общем понятии~--- регулярных выражений, причем лишь в той его части, которая описывается термином шаблон (pattern). И задача сводится к тому, чтобы отыскать в нашем файле строки, начинающиеся последовательностью символов <<Глава>> и каждую последовательность символов в промежутке записать (то есть вывести) в самостоятельный файл. Что можно сделать многими способами, но один из них~--- штатен и элементарен, это команда \texttt{split} (во FreeBSD) или \texttt{csplit} (в Linux).

Последний пример показывает, что, хотя понимание принципов и не избавляет уж совсем от обращения к рецептам, но зато позволяет вычислить последние при их незнаниии. Дело в том, что во FreeBSD команда 
\texttt{split}
 универсальна, и служит для разделения файла по любому параметру~--- размеру, номеру линии или шаблону. Одноименная же команда в Linux выполняет только первые две функции, опции 
\texttt{-p}
 (
\texttt{--pattern}
) в ней не предусмотрено. Что поначалу может расстроить. Однако если понимать, что в принципе разделение файлов по шаблону почти ничем не отличается от такового по номеру линии или размеру (и то, и другое суть действия, основанные на анализе последовательности символов), остается только изыскать соответствующий рецепт. Что можно сделать просто в лоб~--- поискать слово 
\texttt{split}
 в каталоге с 
\texttt{man}
-страницами командой 
\texttt{grep}
 (строго говоря, не слово, а последовательность символов, и командой 
\texttt{zgrep}
, так как страницы эти обычно в gzip-сжатом виде). Чем и обнаруживается man-страница с описанием команды 
\texttt{csplit}
, прочитать которую~--- уже вопрос элементарной грамотности.

Но особенно явно превосходство принципов над рецептами выступает при конфигурировании всего и вся~--- от общесистемных опций до шрифта меню конкретного приложения. Рецептурный подход~--- использование специализированных средств настройки, оформленных в виде самостоятельных утилит или встроенных в прикладные программы. Самостоятельные утилиты такие многочисленны и разнообразны, не зря же Владимир Попов как-то заметил, что число утилит конфигурирования давно превзошло количество конфигурируемых параметров. Интерфейс у них разный в разных дистрибутивах, да к тому же еще и может меняться от версии к версии. Так что доскональное знание какого-либо DrakX из Mandrake ничем не поможет при работе с 
\texttt{sysinstall}
 из FreeBSD, и наоборот.

Если же не <<поступаться принципами>>~--- достаточно раз и навсегда понять, что все параметры настройки системы описываются в соответствующих конфигурационных файлах, которые суть обычные тексты, могущие быть открытыми в любом текстовом редакторе и там модифицированы надлежащим образом. Что и проделывают, только в завуалированном виде, все настроечные утилиты.

Что касается объектов конфигурирования, то есть соответствующих файлов, и субъектов оного~--- параметров настройки,~--- то они определяются стандартными утилитами работы с текстами, например, командой 
\texttt{find}
~--- для поиска файлов по маске, и командой 
\texttt{grep}
~--- для изыскания в них подходящих по смыслу фрагментов. Ну и, разумеется, осмысленного анализа результатов того и другого\dots

Вопреки сложившемуся убеждению, для этого не обязательно быть UNIX-гуру или к таковому обращаться. Во-первых, тот же гуру даст, скорее всего, именно конкретный рецепт на злобу дня. Во-вторых, многие вещи в системе настраиваются один раз в жизни, и вполне возможно, что наш гуру благополучно забыл о том, как именно он это делал. Так что своей просьбой вы просто вынуждаете его вторично проделать ту самую цепочку логических рассуждений, отталкивающихся от общих принципов, которую вы легко (и с пользой для духовного самосовершенствования) могли бы проделать сами.

И еще к слову~--- с успехом (надеюсь) применяя принципиальный подход к жизненно важным для работы настройкам, можно не поступаться принципами и при настройке вещей развлекательного свойства. Например: звуковая карта определена и в ядре настроена правильно, соответствующий софт установлен и работает, но mpeg-файлы, скажем, воспроизводиться не желают.

Вероятно, в user-ориентированных дистрибутивах существуют какие-нибудь специальные утилиты для настройки этого хозяйства, и можно обратиться к ним. А можно просто вспомнить, что звук воспроизводится устройством, устройство есть файл, а файл имеет определенные атрибуты принадлежности (хозяину, то есть пользователю имя\_рек, группе и всем прочим) и атрибуты доступа (в просторечии именуемые правом чтения, исполнения и изменения). И остается только проверить, а имеет ли данный пользователь должные права доступа к этому файлу? И если выясняется, что файл 
\texttt{/dev/audio}
 открыт для всеобщего использования во всех отношениях~--- посмотреть, а не есть ли его имя лишь символическая ссылка на файл реального устройства, отвечающего за воспроизведение звука, и проверить права доступа к тому.

И опять к слову: понятие атрибутов принадлежности и доступа, одно из краеугольных в UNIX-системах, существует и в тех Windows, которые можно назвать всамделишними (то есть NT/2000/XP, даже в ME вроде бы есть зачатки~--- семейный доступ в систему и прочее). Да вот только пользователи их об этом часто не подозревают. Не потому, что чайники, а потому, что их от этого знания тщательно оберегают.

В результате к нештатным ситуациям (потеря пароля~--- кто от этого застрахован, все мы люди, все мы человеки) пользователи Windows оказываются просто морально не готовы: нужно дергаться, звать админа, даже (страшно подумать) лезть в книги (не для того ли мы отказывались от \texttt{man}- и \texttt{info}-страниц?) для поиска рецептов, соответствующих ситуации.

А в UNIX (вернее, Linux/*BSD, за прочие не скажу по незнанию)~--- все просто, если помнить о файле (файлах) паролей, однопользовательском режиме (или возможности загрузки с внешнего носителя) и о том, что дисковые устройства нужно монтировать (насколько я знаю, в NT сотоварищи диски тоже как бы монтируются, только <<дружелюбно>> и <<прозрачно>> для пользователя; в итоге ему остается только удивляться сообщениям об ошибках, выдаваемых при неправильном извлечении USB-драйва).

В общем, подведу итог. Рецептурный подход вполне приемлем (возможно, даже идеален) при потребительско-развлекательных задачах. И оказывается, мягко говоря, не самым эффективным при задачах производственно-креативных. А отработав принципиальный подход на них, становится уже в лом искать, какая кнопочка отвечает за масштабирование окна данной программы воспроизводства видео\dots Проще задать сиюминутную геометрию в командной строке или (раз навсегда) в соответствующем файле ресурсов.

\section{О свободе выбора в чтении документации} 
\begin{timeline}
2005, лето
\end{timeline}

\hfill \begin{minipage}[h]{0.45\textwidth}
Расплата за ошибки~---\\
Она ведь тоже труд.\\
Хватило бы улыбки,\\
Когда под ребра бьют\dots
\begin{flushright}
\textit{Булат Окуджава}
\end{flushright}\bigskip\end{minipage}

В обсуждении на одном из форумов темы о пригодности Gentoo для начинающего пользователя прозвучала мысль, что Gentoo ограничивает свободу пользователя, вынуждая его к чтению документации. Так ли это? Давайте посмотрим.

Начать с того, что жить в Линуксе и быть свободным от него нельзя (почти по Карлу Марксу). А документация~--- неотъемлемый компонент Линукса (а также всех иных UNIX'ов сотоварищи). И от нее также нельзя быть свободным, как в обществе нельзя быть свободным от его законов.

Однако у пользователя все равно остается выбор~--- читать документацию, или не читать её. Так же как в обществе у любого его члена есть выбор~--- знать или не знать его законы. Приходится лишь помнить, что незнание законов общества не освобождает от ответственности за их нарушение. Так и не-чтение документации не освобождает от расплаты за ошибки, совершенные по этой причине.

И в этом отношении между Gentoo и любыми другими дистрибутивами и операционками разница чисто количественная. Пользователю Gentoo желательно начать чтение еще до установки своей системы~--- иначе он, пожалуй, и установить-то её сможет разве что случайно. А пользователь, скажем, Mandrake может заняться этим через некоторое время после иинсталляции.

То есть различие примерно то же, что и в обществах <<цивилизованном>> и <<варварском>> (кавычки уместны, потому что ни тот, ни другой термин не отражают существа явления). В <<цивилизованном>> обществе за первое нарушение закона, скорее всего, мягко пожурят (или там по попе нашлепают). А в обществе <<варварском>> первое же нарушение закона вполне может стать последним: прирежут\dots

Тем не менее, тысячелетия своей истории человечество существовало в <<варварских>> условиях. И ничего, выжило\dots А отдельные индивидуумы и по сию пору неуютно чувствуют себя в условиях цивилизованных.

Аналогичный случай и с дистрибутивами. Конечно, большинство пользователей начинает знакомство с Linux'ом с чего-либо юзерофильного~--- результаты опросов и личные наблюдения позволяют предположить, что обычно в этой роли выступает Red Hat (ныне Fedora) или Mandrake (\textit{ныне Mandriva}). И, опять-таки, в большинстве случаев это оправданно: Windows-подобие таких систем позволяет отодвинуть постижение законов POSIX-мира (а чтение документации, как уже было сказано, один из них) на неопределенный срок.

Однако всегда находились и находятся индивидуумы с ярко выраженными наклонностями к экстриму. И вот для них-то вполне приемлемым вариантом первого выбора может оказаться Linux-дистрибутив типа Gentoo или, скажем, FreeBSD. Только нужно помнить: в этом случае никакого снисхождения к их неопытности от окружающего мира им ждать не приходится. И законы его придется постигать сразу. Может быть, это не так уж и плохо?

\section{Нужны ли Linux'у пользователи?} 
\begin{timeline}
Февраль 2006
\end{timeline}

\textsl{Эта заметка написана мной совместно с Владимиром Поповым, и публикуется с его разрешения.}

\hfill \begin{minipage}[h]{0.45\textwidth}
Есть люди, умеющие пить водку, и есть люди, не умеющие пить водку, но все же пьющие её.

И вот первые получают удовольствие от горя и от радости, а вторые страдают за всех тех, кто пьет водку, не умея пить её.
\begin{flushright}
\textit{Исаак Бабель}
\end{flushright}
\bigskip\end{minipage} 

Время от времени на многих форумах поднимается вопрос~--- как увеличить число пользователей Linux? Какие меры нужно принять, чтобы претворить в жизнь лозунг <<Linux~--- на каждый десктоп!>>? Правда, в результате всех этих обсуждений возникает вопрос встречный: а нужно ли поголовное внедрение Linux? Кого нужно на Linux перетягивать? И, главное: кому нужно это перетягивание?

Чего ради тех, кто давно и осознанно работает в Linux (и прочих BSD) <<по жизни>>, должно волновать число пользователей этих ОС? Число разработчиков~--- другое дело. Чем их больше, тем больше шансов на появление чего-то интересного и принципиально нового. То Интернет придумают, то цифровое сжатие видео. Или, между делом, сломают защиту DVD: просто так, для прикола. Этим доплачивать нужно~--- правда, способов эффективного дотирования обществом разработчиков Open Source до сих пор не придумано. Как, впрочем, и способов поддержки любой общественно-полезной (и даже необходимой) деятельности, не влекущей за собой коммерческой выгоды.

Число интересующихся~--- тоже небезразлично: а вдруг среди них будущий Торвальдс? Этим надо помогать, даже ценой потери собственного времени. Да и веселее с ними\dots

А число пользователей, у которых нет большего интереса, как мышкой поелозить\dots Их увеличение ведет только к росту сообщений на форумах: <<так не правильно! Я так не хочу/не понимаю/не буду>>. И с чего они взяли, что кто-то обязан повышать комфортность их <<пользования>>?

Чтобы осознать абсурдность приведенного выше лозунга, достаточно очертить круг тех, кто не может стать пользователем Linux.

Это, во-первых, пользователи, категорически не способные к освоению компьютера (но, согласно известной максиме Бени Крика, все-таки его использующие). И это~--- отнюдь не признак их глупости, а, как и способность к питию водки, просто индивидуальная особенность: есть же люди, не отличающие ямба от хорея и \textit{до} от \textit{фа}. Так и здесь: нам известно немало пользователей с полуторадесятилетним стажем, так и не освоивших запись на дискету или отключение показа непечатаемых символов в Word. Вынужденные работать на компьютере, они, продолжая словами Бени, страдают: и за себя, и за всех тех, кто на компьютере работать не умеет. И Linux только усугубит их страдания.

Далее, из числа пользователей Linux следует исключить тех, кто испытывает идиосинкразию к чтению~--- а таких, увы, становится все больше даже в нашей стране, некогда бывшей самой читающей в мире. Потому что если в Windows (и тем более на Маке) кое-какие полезные навыки можно получить методом научного тыка, то в Linux без чтения документации и, возможно, даже толстых книг, обойтись практически невозможно.

На Linux никогда не перейдут запойные игроманы и те, кто использует компьютер исключительно в качестве развлекательного центра. И причины понятны: игр под Linux катастрофически мало, и нет в нем ничего, что оправдывало бы смену ОС для домашней аудио- и видеостанции.

Из пользователей-креативщиков вербовать сторонников Linux также в большинстве случаев бессмысленно: работа в нем профессионалов по созданию мультимедийного контента или спецов высокой полиграфии будет попросту неэффективной. Нет в нем и инструментов для работы профессионального художника.

Так кто же остается в сухом остатке, кроме разработчиков софта и профессиональных администраторов компьютерных сетей? Да всего-навсего одна категория пользователей: те, для кого по долгу службы (или велению души) важна эффективность работы с текстовым контентом, дополняемая коммуникационными возможностями. Обработка текстов и коммуникации~--- это то, для чего создавался UNIX, и это~--- сферы приложения его наследника как пользовательской среды. И именно креативщики-текстовики (в любой области~--- от технических писателей и научных работников до поэтов и писателей просто) имеют возможность использовать инструменты UNIX и Linux эффективно. Остается только продемонстрировать им эту эффективность~--- в чем авторы и видят главную задачу линуксописательства.

Интересно, что среди <<действующих>> пользователей Linux весьма высок процент профессиональных юристов и переводчиков. И тому можно видеть две причины. Во-первых, и те, и другие, безусловно, входят в сословие текстовиков-креативщиков. А во-вторых, и для юристов, и для переводчиков более, чем для остальных представителей этого сословия, важны аспекты легальности используемого ими софта.

Так давайте же не будем пропагандировать Linux среди тех, кто по любым причинам не может использовать его эффективно. Сэкономив время и силы для популяризации его тем, кому он действительно может быть полезен\dots

\section{Еще раз о Linux'е и его пользователях} 
\begin{timeline}20 марта 2007\end{timeline}

События последних месяцев~--- дело Поносова, круглый стол в Росбалте, выступление Медведева, наконец, круглый стол, организованный ЛДПР,~--- привели к тому, что слово Linux достигло слуха многих и многих людей, прежде и не подозревавших о существовании какой-либо операционной системы помимо Windows, а то и операционной системы вообще. Еще большее внимание привлекло к себе явление Open Source~--- как практически значимая в наших условиях возможность не разделить участи Поносова.

Рост информированности общества в отношении Linux и Open Source (хотя бы на ознакомительном уровне) нашел наглядное отражение в посещаемости ресурсов этой тематики. На протяжении текущего года она неуклонно возрастает, что особенно показательно для <<сайтов средней руки>> с точки зрения популярности. Ибо на сайтах наиболее популярных относительная картина не столь отчетливо просматривается вследствие высоких абсолютных цифр, а сайты-аутсайдеры как никем не посещались, так и не посещаются. С этой точки зрения типичным примером можно считать динамику посещений сайта \url{http://posix.ru}~--- про него я точно знаю, что его администрация не прикладывала никаких усилий к росту популярности.

При этом на всех сайтах рассматриваемой тематики, статистика которых доступна, существенно возросла доля заходов с поисковых машин относительно таковых с <<соплеменных>> ресурсов. И, пожалуй, впервые запросы от отечественных поисковиков типа Яндекса, Рамблера и Апорта приблизились по количеству к запросам от Google. А это уже прямой показатель роста именно <<непрофильных>> посетителей, так как Google безраздельно первенствует среди <<действующих>> линуксоидов. Глагол <<гуглить>> (\href{http://en.wikipedia.org/w/index.php?title=Google_(verb)\&oldid=582812741}{to google}) прижился у них задолго до того, как был утвержден в качестве официальной нормы английского языка.

Возникает вопрос~--- а не воспользоваться ли благоприятным моментом и не развернуть ли революционную агитацию в пользу Linux и Open Source с целью резкого наращивания пользовательской массы открытого софта? И если да~--- то среди кого и в какой форме?

Более года назад мы с Владимиром Поповым написали статью~--- <<Нужны ли Linux'у пользователи?>>. В которой выразили свое единодушное отношение к этой проблеме: Linux'у нужны не пользователи вообще, а пользователи совершенно определенного склада. В этой заметке я несколько конкретизирую данный вывод с учетом изменившихся за год реалий.

Для начала рассмотрим, кто мог бы потенциально составить пользовательскую базу Linux'а. Я разделил бы их на три группы, в порядке убывания стимулов к изучению системы и её использованию.

Первая группа~--- это те, кто хочет изучать Linux вообще, <<от сих до сих>>. В том числе и для использования в своей практической работе~--- но не обязательно: исходным стимулом может быть просто любопытство и стремление дать тренировку мозгам (своего рода <<интеллектуальный преферанс>>, по выражению Владимира Попова). Тем не менее, в основном пользователи первой группы профессионально связаны с IT-сферой, будучи разработчиками программного обеспечения или системными администраторами. И основным стимулом для изучения Linux'а у них будет стремление к повышению своей профессиональной квалификации и расширению кругозора~--- даже в том случае, если на практике они Linux будут использовать ограничено или не использовать вообще. Эту группу можно назвать энтузиастами.

Отдельную прослойку в составе энтузиастов составляют те, кто еще только готовится к деятельности в IT-сфере или не вполне определился в своих устремлениях. Именно среди них наиболее высок процент <<интеллектуальных преферансистов>>.

Вторая группа~--- это пользователи, которые готовы изучать Linux в тех пределах, в каких необходимо для использования в своей практической деятельности, с IT-сферой никак не связанной. Рост эффективности последней для них и является основным стимулом для изучения. А стимулы для использования могут быть весьма разными: от разумного стремления сэкономить на софте до необходимости применять только легальное программное обеспечение. А многие из них на собственном опыте убеждаются, что использование Linux'а действительно более эффективно для выполнения их задач.

Соответственно, и пределы необходимого минимума знаний для представителей этой группы оказываются разными. Кому-то будет достаточно освоить текстовый редактор, браузер и программу для работы с почтой, иному же потребуется владение комплексом UNIX-утилит и навыки шелл-скриптинга, а возможно, даже и всамделишнего программирования. Тем не менее, они не ставят себе целью превратиться в профессиональных админов или разработчиков, и в познании системы готовы ограничиться только тем, что нужно для их практической работы. И потому таких пользователей резонно назвать прагматиками.

Профессиональный состав второй группы очень пестр, и объединяют её два момента: работа в сферах, непосредственно не связанных с IT, и преобладание задач, связанных с обработкой текстов (в самом широком смысле слова) и коммуникациями любого рода. В основном это представители так называемых творческих профессий. Среди моих личных знакомых есть юристы, переводчики, журналисты, научные работники, некомпьютерные инженеры. А потенциально в числе этой группы я вижу и поэтов с писателями\dots

Третья группа~--- это те, кто не хочет ничего использовать и ничего изучать, и сам факт существования компьютеров и операционных систем им глубоко параллелен. А будучи по жизни вынужденными на компьютерах работать (что поделать, такова принудительная сила реальности), они будут использовать все, что угодно: то, что поставил приятель дома или сисадмин на работе, то, что приказало начальство, или, наконец, просто то, что использует большинство людей вокруг. Эту группу пользователей можно назвать попутчиками.

Эту категорию называют обычно секретаршами-блондинками, хотя с не меньшим основанием в нее можно включить брюнетов-чиновников и шатенов-менеджеров, а также вполне рыжих и даже лысых деятелей науки, культуры и искусства, а также руководящих работников разного ранга. Короче говоря, за небольшими исключениями, почти все остальное человечество, вне зависимости от пигментации волосяного покрова и радужной оболочки глаза, расовой, национальной и конфессиональной принадлежности.

Что же, почти все человечество~--- это потенциальные пользователи Linux'а, спросите вы меня. Я на этот вопрос я отвечу положительно. С одной лишь оговоркой~--- почти все практически работающее человечество, выполняющее производственные задачи при помощи компьютера. Подобно тому, как любая кухарка могла бы управлять государством (под наблюдением комиссара, разумеется), так и любая секретарша способна вести свое делопроизводство в Linux'е. Только вместо комиссара ей потребуется сисадмин, который ей этот Linux установит и должным образом настроит. К ней же никаких претензий по уровню компьютерной подготовки предъявляться не может в принципе~--- за исключением практических навыков работы с теми программами, которые нужны для выполнения своих служебных обязанностей.

В том, что большинство представителей вида Homo Sapiens ничего не знают о компьютерах и операционных системах, нет ничего ни плохого, ни страшного: деятельность человеческая многогранна, и интересы и амбиции большинства людей лежат вне IT-сферы. И это вовсе не значит, что операционная система (в данном случае Linux) для массового использования должна быть адаптирована до уровня понимания <<некомпьютерным>> человечеством: достаточно, если её будут понимать те, кто ему её устанавливает. А кто они?~--- да те самые представители первой группы пользователей, о которых речь шла выше. Благо, если в управленческой модели товарищей Ленина и Бухарина комиссара требовалось приставить к каждой кухарке, да еще и снабдить его надлежащим инвентарем в виде именного революционного маузера, то одного пользователя-первокатегорника вполне хватит на многие десятки, а то и сотни представителей категории третьей. В скобках замечу, что представители второй категории, как правило, вполне в состоянии комиссарить (пардон, админить) сами себя\dots

Конечно, говоря о всем трудящемся человечестве как потенциальных пользователях Linux, я несколько преувеличил. Для весьма значительной его части эта платформа неприемлема хотя бы потому, что под нее отсутствуют требуемые им приложения. Или просто на других платформах их задачи решаются эффективнее. Впрочем, эта тема подробно рассмотрена в указанной выше статье. 

Но есть и пользователи, которым использование Linux просто противопоказано. Хотя именно они часто рассматриваются как потенциальные пользователи этой ОС. И, главное, сами себя они позиционируют в этом качестве, почему их и придется выделить в отдельну, четвертую, группу.

Четвертая группа потенциальных пользователей весьма незначительна численно, но чрезвычайно активна. Это те, кто хочет использовать Linux, не желая его изучать вообще. Даже в тех ограниченных пределах, в каких это считают необходимым пользователи второй группы. Мотивация их для применения Linux бывает разной, но наиболее частыми причинами выступают две:
\begin{enumerate}
	\item <<Linux~--- это круто>>;
	\item <<Linux~--- это бесплатно>>.
\end{enumerate}
Представители четвертой категории, не смотря на свою малочисленность и эфемерность статуса Linux-пользователей (<<детская болезнь крутизны>> проходит с возрастом, не обязательно биологическим, а стремление к халяве быстро разбивается о скалы реальности), представляют собой весьма активную и заметную со стороны часть сообщества (по крайней мере на просторах Рунета). Именно они являются главными ревнителями идеи <<Linux'а для каждой кухарки>>~--- то есть адаптации системы до уровня, доступного пониманию того, кто о ней ничего не знает и знать не хочет. Их лозунг~--- <<Все, как в венде, но бесплатно>>. Идеал настройки при этом~--- единственная кнопка с надписью на албанском языке \textbf{Зделать пес\dotsто}
 (да простят меня читающие это дамы, если по русски звучит не вполне нормативно~--- но суть дела передает адекватно).

Ну и разумеется, не обходятся они и без выражения претензий ко всему остальному сообществу: почему это оно, такое и разэтакое, не торопится, что бы для лично для меня, любимого, все было\dots ну тем самым, албанским, образом. Разумеется, забесплатно. И очень обижаются, когда им в самой деликатной форме, без обращения к известному сайту символического направления и генеалогических исследований по женской линии, пытаются объяснить всю беспочвенность их притязаний. Если же в ответах на свои вопросы наш <<четвертичник>> углядит хоть малейшие следы неполиткорректности~--- возмущение его можно будет сравнить только с протестами советского народа против угнетения шахтеров Уэльса.

За пользователями четвертой группы в народе закрепилось название <<красноглазых>>. Однако это не вполне адекватное обозначение: фанатично настроенные граждане не редки и среди энтузиастов, особенно в младшей возрастной категории; правда, среди них фанатизм проходит со временем. Пользователей четвертой категории я бы назвал вопрошателями, ибо наиболее характерные их черты~--- нежелание читать что бы то ни было и любовь к задаванию вопросов на форумах, ответы на которые они читают лишь затем, чтобы найти повод задать следующий вопрос.

Разобравшись с тем, что представляют собой потенциальные пользователи Linux'а, можно и определиться с вопросом~--- как этот самый Linux пропагандировать и среди кого за него агитировать.

Очевидно, что энтузиасты в агитации не нуждаются: слово Linux они, как правило, знают, как и представляют сферу его применения. И либо уже применяют его в своей деятельности, либо готовятся к применению (в частности, посредством изучения и экспериментов в области <<интеллектуального преферанса>>). Либо~--- никогда применять не будут, по тем или иным причинам (отсутствию подходящих задач, служебной привязанности к более иной платформе, личной антипатии, наконец).В любом случае, энтузиасты нуждаются не в пропагандистских, а в чисто информационных материалах~--- благо, как правило, искать информацию они умеют или очень быстро учатся.

А вот потенциальные прагматики могут просто не слышать о существовании Linux'а. И, более того, не знать о том, что текст можно набирать не только с MS Word, а в Сеть выходить чем-то, отличным от Internet Explorer. Что поделать, 12 лет пропаганды под <<Веселым Роджером>> сыграли свою роль в деформации пользовательских представлений. Но и тут речь идет не столько о пропаганде и агитации как таковой, сколько о популяризации знаний о Linux, Open Source, сфере и методах их применения.

Попутчикам, как правило, достаточно донесения самого факта существования Linux~--- да и то, лишь при наличии их доброй на то воли, потому как они спокойно обойдутся и без этого. Более того, их по возможности следует оберегать от внедрения излишней информации. Потому что это либо вызовет раздражение (кому понравится навязывание знаний о вещах совершенно неинтересных), либо превращение части попутчиков в вопрошателей.

Ну и, наконец, вопрошателей не нужно ни агитировать, ни пропагандировать. Увы, как раз наоборот, сплошь и рядом приходится противодействовать их агитации за <<Всенародный Linux>> и пропаганде представлений о том, как он должен выглядеть. И в который раз повторять нехитрую истину, высказанную в свое время Страшилой Мудрым:
\begin{shadequote}{}
Река это не дорога, а дорога~--- это не река.
\end{shadequote}
А то ведь еще не достаточно информированные потенциальные прагматики могут и впрямь поверить, что по реке можно идти, аки посуху\dots

\section{И снова о массовом Linux'е} 

\begin{timeline}18 октября 2007 г\end{timeline}

Эта заметка родилась, с одной стороны, как своего рода полудополнение к ряду статей Владимира Попова. С другой~--- как следствие чтения многочисленных комментариев к иным статьям (в том числе и не имеющим никакого отношения к <<Десктопному Линуксу>>. Немало материалов к этой теме дают и обсуждения на форумах. С цитаты из одного из таких обсуждений я и начну. 

\hfill \begin{minipage}[h]{0.45\textwidth}
\dotsкомпьютер, который призван облегчать решение различных проблем, пока их не меньше добавляет
\begin{flushright}
\textit{Vityaz}, \texttt{unixforum.org}
\end{flushright}
\bigskip\end{minipage}

Поскольку цитата вырвана из контекста, который занял бы слишком много места, дополню, что в ней подразумеваются проблемы, которые компьютер создает пресловутому конечному пользователю~--- эпическим персонажам современного фольклора: секретарше-блондинке, брюнету-ченовнегу и шатену-манагеру. Ах да, еще и тётке-бухгалтеру. 

Так ли это? Давайте не будем лукавить и признаем, что~--- да, это так. Причем, что интересно, он создает им проблемы вне зависимости от аппаратной платформы и операционной системы на ней: и в Windows, и Linux, и в любой BSD. Даже в DOS'е черном, помнится, создавал. Почему? 

Ответ прост. Дело в том, что компьютеры не создавались для облегчения жизни кого бы то ни было~--- ни секретарши, ни манагера, ни чиновнега. Ни даже бухгалтера. Они создавались для обеспечения систем противовоздушной обороны, расчета траекторий баллистических ракет и космических кораблей, прогноза движения теплых и холодных воздушных масс с Атлантики, обеспечения отказоустойчивых коммуникаций\dots Иными словами, для решения многих и многих задач, которые без компьютеров решены бы не были. 

Это относится к компьютерам вообще. И сфера применения компьютеров, как неоднократно подчеркивал Владимир, неизмеримо шире того её сегмента, который ассоциируется с этим словом у большинства трудящихся~--- то есть сегмента настольных персоналок, каковые в настоящее время свелись к архитектуре i386 (PC или, в просторечии, <<писюки>>~--- ибо ныне и гордые Mac'и архитектурно являются ничем иным, как <<писюками>>). 

В отличие от Владимира, я не застал героических времен PDP-7 сотоварищи. То есть застать-то застал, но от компьютеров был очень далек. И мое знакомство с продукции IT-индустрии началось с персоналок~--- IBM PC/XT-совместимых, с процессором Intel 8086/8088, работавших под управлением <<черного>> DOS'п. Которые я полюбил всей душой. Потому что они позволяли мне сделать (в моей тогдашней профессиональной сфере) то, чего я не мог рассчитать посредством логарифмической линейки, арифмометра~--- <<железного феликса>>, и даже программируемого калькулятора со встроенным Basic'ом. Причем сделать, не обращаясь во всякого рода ВЦ, каковые обращения требовали немалых бюрократических усилий. Для меня это был чрезвычайно эффективный инструмент работы. А отнюдь не бытовой прибор, создающий удобства. 

Так что даже и персоналка изначально не предназначалась для упрощения жизни бухгалтерши Надежды Александровны, секретарши Люси, манагера Пети или ченовнега Васи. А предназначалась она для того, чтобы инженер Ваня, биолог Костя или геолог Лёха могли с его помощью сделать то, чего без него сделать не могли. Например, рассчитать статистику соотношения легких редкоземельных элементов и высокозарядных катионов для всех проб глубоководного бурения в Южной Пацифики и сравнить её со статистикой их же в экзотических террейнах Северо-Востока СССР. 

Справедливости ради надо заметить, что вышеуказанные Ваня, Костя или Лёха не забывали и о нуждах Надежды Ивановны или Люси. А также кадровички Нины Егоровны. Они устанавливали на их машины все тот же самый <<черный>> DOS и программу, необходимую для соответствующей работы~--- бухучета, учета кадров (тогда их не сочинял только ленивый). Ну а для Люси~--- разумеется, Lexicon. И писали простенький batch-файл, позволяющий запустить эту программу нажатием одной клавиши. 

Все были довольны: наши герои спокойно занимаются своей работой, Люся печатала на принтере из Lexicon'а приказы и распоряжения (причем все больше и больше, не смотря на призывы к внедрению безбумажного документооборота), Нина Егоровна учитывала кадры (которых, напротив, становилось все меньше и меньше), а Надежда Александровна начисляла им всем зарплату~--- правда, проверяя результаты работы компьютера на счетах, как она привыкла это делать еще при появлении настольных калькуляторов. 

А потом произошло классовое расслоение. Ваня, Костя и Лёха постепенно мигрировали на Linux~--- потому, что там они свою работу могли выполнять намного эффективней, чем в <<черном>> DOS'е или на PDP-7. Подчеркну при этом, что они были и оставались самыми обычными пользователями. А вот Надежда Александровна, Нина Егоровна и Люся почему-то стали работать в Windows~--- причем отнюдь не по собственному желанию. И вi думаете, как говорят в Одессе, эффективность их работы повысилась? Фиг с маслом~--- ответит вам любой одессит (конечно, немножко другими словами). Ибо у них тут же появилось множество новых забот: как им слушать музыку, смотреть кино, ставить и запускать новые игрушки. И выяснять, почему после установки очередной игрушки их система развались напрочь. Разумеется, работа при этом не делалась ни лучше, ни быстрее. 

И столь же очевидно, что всеми этими <<настроечными>> мероприятиями героини занимались не сами. Ибо не было у них ни должной подготовки, ни желания её приобрести (это отнюдь не в упрек, ни один человек не может знать и уметь все). И для начала они решили припахать к этому делу наших героев~--- не учтя того обстоятельства, что они были героями отнюдь не их романа. И что им по большому счету более или менее безразлично, будет ли нашим дамам за компьютером столь же комфортно, как в прокладках Тампакс. Причем опять-таки вне зависимости от целевой платформы~--- Windows ли это любого рода, или Linux, или даже FreeBSD. Потому как сфера их интересов лежала в области эффективности работы, а не в области комфорта. 

Вот, собственно, и вся коллизия. Суть которой~--- в несовпадении интересов тех пользователей, для кого компьютер являет собой рабочий инструмент, и тех, кто рассматривает его как бытовой прибор. Причем это несовпадение не связано ни с пигментацией волос, ни даже с профессиональной принадлежностью. Среди пользователей первой группы мне встречались не только научные работники и инженеры, но также переводчики, юристы, и даже~--- страшно сказать~--- ченовнеги и секретарши. А уж группа вторая~--- охватывает весь спектр социальных слоев и профессий нашего общества. 

Так что пользователи первой группы просто скажут спасибо разработчикам открытого софта~--- тем, которые обеспечили им среду для решения своих задач относительно малой (финансовой) кровью. И не так уж им интересны усилия маркетологов, цель которых~--- навязывание ненужного товара путем искусственного создания несуществующей потребности в нем. Ибо они прекрасно понимают, что ни к чему хорошему это не приведет. 

Я заранее предвижу все возможные комментарии на это сочинение: 

\begin{itemize}
	\item и про то, что компьютер действительно стал бытовым прибором; 
	\item и про драйверы, изобилия которых не добиться без того, что Linux станет массовой системой;
	\item и про вселенскую угрозу, исходящую от монополии Microsoft; 
	\item и всё остальное~--- поверьте, все возражения против изложенной выше точки зрения уже были высказаны много лет назад. 
\end{itemize}

И ответы на них были даны столь же давно~--- как в публикациях моих коллег и единомышленников, так даже и в моих. Ответы, для нас вполне весомые, хотя и не универсальные. 

И потому не следует думать, что каждый пользователь Linux спит и во сне видит, как бы ему соблазнить, охмурить или сагитировать очередного вендузяднега. Нет, он, как правило, в основном занимается своим делом, а в свободные минуты предпочитает общаться в своем кругу~--- среди тех, кто разделяет его интересы. 

Так что оставим соблазнение\dots ну это интимный вопрос, замнем для ясности, охмурёж~--- ксендзам, а агитацию~--- партийным активистам. 

\section{Заметки сноба, или Беру свои слова обратно} 

\begin{timeline}21 июня 2007 г\end{timeline}

Вот уже скоро восемь лет, как я пишу о Linux'е и Open Source~--- первые заметки сочинялись для онлайна и бумаги в 1999 году. Юбилей, однако,~--- двоичное тысячеление, в масштабах времени индустрии вполне можно приравнять к веку. И посему решил я, древний старец-линуксописатель, подвести некоторые итоги былого.

Время моего первичного приобщения к Linux'у выпало на 1997-1999 годы. Тогда это было не так просто, как в последствии,~--- требовалось прикручивать руками русский язык в консоли, подбирать X-сервер, способный запуститься на данной видеокарте (а видеочипов в те времена было не в пример больше, чем нынче, и каждый имел свой собственный X-сервер, и не фактом было, что они идеально подходили друг другу, как герои-любовники из сентиментальных романов). А уж настройка видеорежимов или русификация тех же Иксов казались высшим пилотажем.

И с источниками информации была напряженка. Не то что нынешнего изобилия русскоязычных сайтов и книг не предвиделось и в помине~--- англоязычные сайты тоже не попадались на каждом шагу, а книги, при желании, заказать можно было~--- хоть у самого товарища О'Рейли, но, пардон, за длинный и зеленый, как крокодил Гена, додефолтный бакс, во-первых, и при практической невозможности этот бакс ему оттранспортировать,~--- во-вторых. Ибо красных дипкурьеров тогда уже не было, а современных платежных систем~--- не было еще.

Разумеется, уже тогда были \mbox{\texttt{man}'ы}~--- на аглицком, конечно, но это казалось преодолимым даже теми, кто, подобно автору этих строк, всю начальную часть своей жизни учил немецкий (да и из него к тому времени забыл даже то, чего никогда не знал). Ибо не Бернардом Шоу были писаны, и не Оскаром Уайльдом, а теми, кто английский, возможно, знал не намного лучше русскоязычных читателей. И потому были просты для понимания.

Но много ли мог помочь стандартный man в ситуации, когда Иксы категорически не хотят русифицироваться? И не хотят потому, что в одном из файлов исходников была мелкая банальная ачипятка, ни на что более не влияющая. Её выискал в свое время Иван Паскаль, за что честь ему и хвала от сообщества. Правда, ну кто нонче помнит ту историю? А подобных историй в те времена были не одна и не две.

Google, кстати, в те времена только зарождался, и до нынешней славы ему было далеко. А найти что-то в популярных тогда поисковых машинах типа Altavista и других, ныне забытых, было не самой легкой задачей, да и релевантность оставляла желать лучшего.

Не блестяще было и с Интернетом~--- не только с домашним, где царил сплошной dial-up, часто через АТС не то что не первой, но даже и не второй свежести, но и на выделенных служебных или фирменных линиях.

Не было, кстати, и онлайновой торговли дистрибутивами в ассортименте \url{http://distrowatch.com}, который мы нынче можем видеть, например, на Линуксцентре. Методов приобретения таковых было, по существу, три:

\begin{enumerate}
	\item у пиратов~--- исключительно на заказ и на <<золоте>>, а потому очень дорого;
	\item в больших книжных магазинах Москвы (и возможно, Питера)~--- в фирменном исполнении Walnut Creek или InfoMagic, то есть ничуть не дешевле;
	\item привоз знакомыми из-за Бугорщины~--- это уж как договоришься.
\end{enumerate}

О скачивании дистрибутивов бочками (пардон, пачками) трудно было помыслить не только в домашних, но и в служебных условиях.

В общем, как говорили встарь, было трудно, но интересно. Может быть, именно интерес и позволял пользователям моего поколения успешно прорываться сквозь все препоны и рогатки. Причем не только профессиональным программистм, админам и вообще айтишникам, но и лицам с вполне гуманитарным или полугуманитарным образованием.

Спрашивается~--- а на хрена им это все было надо? Ну уж не в игрушки резаться, или там домашние кинотеатры организовывать. А в первую очередь ради повышения эффективности работы. Для кого-то это была изощренность в методах обработки текстов, для других~--- развитые коммуникационные функции, для третьих~--- возможность решения нестандартных счетных задач. Да мало ли областей, в которых UNIX и Linux обеспечивали максимальную эффективность с минимальными затратами сил и средств\dots

Именно в то время я и занялся линуксописательством. Во-первых, для себя, любимого: описать предмет~--- один из самых эффективных способов в нем разобраться (особенно если описание предмета выполняется его же средствами).

Во-вторых, для других~--- смею надеяться, что некоторые из моих сочинений помогли кому-нибудь не наступать на те же грабли, что и я.

В-третьих, и главных~--- опять же для себя: мне это было просто интересно.

И в своих писаниях я всегда исходил из следующих посылок:
\begin{enumerate}
	\item всякий, кто берётся осваивать Linux, нуждается в его функциях,
	\item готов затратить время и силы на их освоение, и
	\item испытывает интерес к предмету освоения.
\end{enumerate}
Эти три условия казались мне необходимыми и достаточными, чтобы самый обычный человек с произвольным складом ума и любой начальной подготовкой, сводящейся к функциональной грамотности, смог бы Linux освоить в пределах, необходимых для решения его задач. И, подобно герою Александра Галича,

\begin{shadequote}{}
\dotsповсюду, и устно, и письменно,\\
Утверждал я, что все это истина. 
\end{shadequote}
Ныне я беру свои слова обратно: нет, Linux~--- это не система для обычных (конечных, они же, по выражению Владимира Попова, <<чисто вымытые>>) пользователей.

Что заставило меня изменить свое мнение? Да все те же многочисленные обсуждения на форумах. Которые показывают, что обычному пользователю не по силам читать документацию даже на русском языке, не говоря уже об английском. Что обычному пользователю во сто крат легче сутками ждать ответа на свой вопрос, нежели набрать два-три слова в поисковой строке Google. А уж о том, что для получения некоторых базовых знаний достаточно прочитать какую-нибудь книжку по теме, а не задавать вопрос, ответ на который потребовал бы пересказа таковой~--- это нашему обычному пользователю и в голову не придет.

И лейтмотивом всех вопросов такого, ныне обычного, пользователя, является~--- <<сделайте мне пес\dotsто>>. Чего бы этот вопрос не касался~--- установки ли Мандривы или настройки VPN, налаживания домашней локалки или развертывания корпоративного сервера.

И это при том, что в современных дистрибутивах при установке на типовые конфигурации практически не возникает проблем с <<железом>>.

При том, что многие дистрибутивы (и не обязательно отечественные) корректно русифицированы <<из коробки>>, а о русификации остальных написано столько, что приходит на память <<Война и мир>>.

При том, что ныне в Сети описано решение 99\% всех вопросов, которые только могут возникнуть перед начинающим пользователем. И не менее 90\\% ответов на них дается по русски.

При том, что каналы и тарифы, по крайней мере в нескольких больших городах, позволяют качать образы не только дистрибутивных CD, но и DVD, да еще и не единожды.

При том, что оставшимся обладателям плохих или дорогих каналов доступен заказ дисков в онлайновых магазинах с доставкой в любую точку России (и некоторых сопредельных стран) за более чем разумные деньги.

При том, наконец, в больших книжных магазинах полки с литературой по Linux и UNIX вплотную приблизились по насыщенности к стеллажам детективов, да и системы онлайновой книготорговли ими не оскудели.

Так что не нужен обычному пользователю Linux. Это~--- система исключительно для мрачных бородатых снобов, как выразился один из посетителей одного из форумов. Поскольку сказано это было в мой адрес~--- принимаю себя за эталон такового: да, мрачен изначально; да, благоприобретенно бородат; ну и безусловно сноб, ибо снобизм свой лелею, подобно герою песенки Тимура Шаова (точнее, начал лелеять, когда Тимурчик еще пешком под стол ходил).

Утешает одно~--- что среди моих личных и виртуальных друзей и знакомых таких снобов~--- подавляющее большинство, если не все вообще. Остается только выявить их отличительные признаки.

Так, например, мрачность~--- отнюдь не непременный атрибут сноба: многие из тех же знакомых являют собой очень жизнерадостных людей.

Тем более~--- борода также не является снобоопределяющим признаком: в круге моего общения многие аккуратно бреются, другие~--- не бородаты ввиду младости лет, иные же~--- не могут ими быть в силу принадлежности к прекрасному полу. Что, тем не менее, не мешает им всем быть снобами.

Давайте выясним, почему.

Первая отличительная особенность снобов~--- умение читать. Причем все подряд: выводы команд, сообщения об ошибках, документацию во всех её проявлениях, материалы онлайновых ресурсов, книги\dots А некоторые из снобов, особо возомнившие о себе, читают даже журнал <<Линуксформат>>\dots Короче, как сказали бы классики советской фантастики:

\begin{shadequote}{}
Грамотен~--- сноб.\\
Книжки читаешь~--- сноб.\\
Команду man знаешь~--- сноб, слишком много знаешь. 
\end{shadequote}
Далее, снобы не просто умеют читать~--- они обладают способностью отыскивать предмет чтения, и, как правило, не ленятся это делать.

Наконец, последнее~--- снобы и чтением, и поиском объектов чтения занимаются с интересом. И, как правило, знают, зачем они это делают.

Таким образом, мы очертили круг снобов~--- то есть тех, для кого предназначен Linux. И потому каждый, пожелавший приобщиться к этой системе, должен для начала точно позиционировать себя~--- сноб ли он или из тех, кто желает, чтобы ему <<сделали пес\dotsто>>. Во втором случае ему лучше за освоение Linux'а не браться, ибо, как поется в известной песне,

\begin{shadequote}{}
Никто не сделает пес\dotsто~---\\
Ни Бог, ни царь и ни герой\dots 
\end{shadequote}
А вот прирожденным снобам судьбой предначертано~--- \texttt{man} в руки и за дело. И да пребудет с ними удача\dots

В связи с этим возникает резонный вопрос~--- а имеет ли в настоящих условиях какой-нибудь смысл линуксописательство? Ведь, как я уже говорил, 99\% всех проблем, могущих возникнуть перед начинающим пользователем, описаны, а те, что не описаны, связаны либо со специфическими задачами, либо с очень новым оборудованием.

Так что сноб легко найдет решение в ворохе наличествующей информации. А обычный пользователь все равно ничего ни искать, ни читать не будет~--- даже если решение дано в соседнем трейде посещаемого им форума\dots

\section{Начал ли звонить колокол?} 

\begin{timeline}1 августа 2007 г\end{timeline}

\hfill \begin{minipage}[h]{0.45\textwidth}
Вам, Lena и Виктория, авторам комментов на Цитките, посвящается
\bigskip\end{minipage}

Призывы внедрить, расширить, приобщить, подсадить, вовлечь и даже распропагандировать раздаются в отношении Linux'а уже много лет~--- грешен, и сам подчас говорил нечто подобное. Однако в последнее время это призывы, во-первых, участились. А во-вторых, приобрели весьма экстремистский характер~--- вплоть до лозунга полного вытеснения Windows с рабочих столов всех, в том числе и так называемых конечных, пользователей. Что сочетается с требованием <<десктопизации Linux'а>>~--- то есть доведения его до состояния, понятного среднестатистическому <<подоконнику>>, не желающему вникать в устройство системы ни в малейшей степени.

Добро бы, когда такие призывы раздавались бы со стороны фирм, связанных с разработкой, внедрением и распространением дистрибутивов Linux и вообще решений на базе Open Source. Это, во-первых, их бизнес, во-вторых, они действительно занимаются <<расширением, приобщением, углублением>> и тому подобными делами. Но как раз они-то и проявляют сдержанность в высказываниях на эту тему. Оно и понятно~--- им, как никому другому, ясна вся сложность процесса <<повсеместного перевода делопроизводства на латинский алфавит>>\dots то есть, пардон, пользователей на Linux.

Вполне ожидаемо, что лозунг <<Linux forever>> раздается из уст <<юношей бледных со взором горящим>>, подчас даже с рубиновым блеском в глазах. Это~--- обычный максимализм неофитов, которому в свое время отдал дань каждый из нас. И который проходит со временем~--- как только краткий этап первичного (и потому восторженного) приобщения к системе перейдет в хроническую стадию просто работы в ней.

Не вызывают удивления такие призывы и от тех, кто услышал о Linux'е только вчера и, скорее всего, в связи с пресловутым <<делом Поносова>> и сопряженными событиями: ими, при полном незнании того, что представляет собой Linux и Open Source (и часто, похоже, при столь же полном нежелании вникнуть в тему) движет элементарная боязнь того, что завтра придет ОБЭП и возьмет за~\dots ну за что там обычно берет ОБЭП? За финансовую документацию, вероятно\dots

Более странно, что призывы к поголовному переходу на сою (опять же, пардон, на Linux) нынче можно услышать от действующих линуксоидов, вполне успешно использующих эту систему в практических целях. И еще более удивительно, что призывы эти подчас исходят от представительниц прекрасного пола. Коим я, как джентльмен, не могу не ответить. Что и попробую сделать в настоящей заметке.

Правда, свою позицию по данному вопросу излагал неоднократно, и устно, и, в основном, письменно (в том числе и совсем недавно): Linux на каждом десктопе не нужен, и десктопизировать его не нужно также. Тем не менее, в последнее время отчетливо проявилось несколько тенденций, подкрепляющих мою точку зрения. Поэтому, рискуя повторить кое-что из сказанного ранее, изложу свое нынешнее представление вопроса.

Маленький дисклаймер: далее я буду ссылаться на некоторые дистрибутивы, как примеры особенно (или в меру) дружественных к пользователю. Это не значит, что среди всех остальных дистрибутивов не найдется столь же (а может быть, и более) дружественных. Просто те, которые я упоминаю, <<щупались>> моими руками относительно недавно.

Начнем с утверждения, что для повсеместного перехода на Linux требуется его десктопизировать еще больше, чем это есть в ныне существующих дистрибутивах юзерофильской категории.

Дистрибутив типа Ubuntu/Kubuntu разворачивается до полностью работоспособного состояния за полчаса, и, при наличии подключения к Сети, не требует от пользователя никаких дополнительных действий, даже русификации. Кроме разве что сугубо косметических~--- типа переопределения обоев рабочего стола. Ставшая притчей во языцех доустановка мультимедийных кодеков (страшно сложная операция, скажу я вам по секрету)~--- ныне в прошлом: при щелчке на мультимедийном файле система заметит, что кодек для его воспроизведения отсутствует. Но тут же любезно предложит скачать его и установить, после чего проигрывание выбранного ранее начнется тут же. Не это ли воплощение принципа~--- <<сделайте мне пес\dotsто>>? Того самого, о котором мечтали поколения пользователей, бросивших Linux из-за невозможности найти в консоли кнопку \textbf{Пуск}\dots

Ничуть не сложнее обстоит дело нынче и с Debian'ом: сейчас он, точно также как и вариации на тему Ubuntu, может быть установлен в графическом режиме на полном автомате, с одной из предопределенных рабочих сред (GNOME, KDE, Xfce) и сопутствующими им наборами приложений. И вообще, сравнение Ubuntu и Debian вызывает последнее время столько вопросов на форумах, что я решил посвятить этой теме отдельную заметку.

Не намного больше времени уйдет на установку Zenwalk, который не считается эталоном дружественности. И тем не менее и тут по окончании инсталляции пользователь также получает полностью работоспособную систему со скромным, но достаточным на первое время набором приложений (включая пресловутые кодеки <<из коробки>>). Чуть-чуть прихрамывает русификация~--- но её излечение требует только многократно документированной правки пары-тройки конфигов. 

Archlinux имеет славу дистрибутива, относительно сложного для начинающего, установка которого требует определенных познаний. Однако познания эти на практике сводятся к некоторому минимальному представлению о дисковых разделах, файловых системах и пользовательских аккаунтах.

Что, кстати, не лишне и при установке любого другого дистрибутива, даже самого юзерофильного. А понятие пользовательского аккаунта вообще одно из краеугольных в идеологии UNIX, и без него не обойтись и в дальнейшей работе (во избежание последующих криков <<не могу войти рутом>>, оглашающих просторы форумов).

Ну так вот, возвращаясь к нашим Archlinux'ам: при наличии указанных выше познаний (согласитесь, что требования отнюдь не чрезмерные), дистрибутив этот также может быть развернут исторически мгновенно~--- и в почти пригодном к употреблению виде. <<Почти>>~--- потому что русификация его потребует все же некоторых мануальных пассов. Но достаточно простых и многократно описанных.

Правда, все описанные дистрибутивы столько хорошо ведут себя при одном непременном условии~--- наличии хорошего (то есть быстрого и, главное, дешевого) доступа к Сети. В отсутствие оного их доведение до ума может доставить некоторые трудности. Правда, в большинстве случаев они решаемы~--- приобретением полных снапшотов пакетных репозиториев, например, или скачиванием в другом месте. За исключением, пожалуй, Zenwalk~--- его эффективное использование в чисто оффлайновом режиме представляется мне пока проблематичным. Но главное, при отсутствии сети ни один из перечисленных выше дистрибутивов не заиграет всеми своими красками. Кроме, разве что, Debian'а, одна из форм распространения которого лежит на фантастическом числе дисков.

Тем не менее, и проблемы с коннектом не смертельны~--- нужно только подобрать более подходящий для такой ситуации дистрибутив. Таковым, кроме упомянутого выше Debian'а, может стать Mandriva, особенно в своих коммерческих ипостасях~--- Discovery, Powerpack или Powerpack+, в зависимости от потребностей. Любой из этих вариантов может быть полностью укомплектован без доступа в Интернет вообще. Причем будет включать в себя не только полный набор приснопамятных мультимедийных кодеков, но и такие коммерческие программы, как CrossOver и Cedega. И драйвера для видеокарт окажутся не забытыми.

Есть подозрение, что большая часть того, что сказано о Mandriva, приложимо и к таким дистрибутивам, как Suse, или к клонам Red Hat (вроде Scientific Linux, например). Однако лично в последнее время я с ними не общался~--- и потому от категорических утверждений воздержусь.

Короче говоря, ни в какой дополнительной десктопизации Linux уже не нуждается~--- эту тенденцию можно считать полностью реализованной.

Вторая тенденция развивается полным ходом. Это~--- создание дистрибутивов, способных работать без инсталляции на компьютер. То есть~--- мобильных систем, которые пользователь мог повсюду носить с собой и запускать на любой машине. В форме LiveCD они существуют давно, однако использование их было ограничено самой природой носителя: для переноса пользовательских настроек и данных требовался еще и какой-либо накопитель, способный к записи, что не способствовало комфорту работы <<на стороне>>. Был, правда, Puppy, который, будучи размещенным на CD-RW, обладал способностью сохранять настройки, но для переноса на нем же данных и он не был приспособлен.

Ныне же началось развитие дистрибутивов на USB-флэшках, так называемых брелках (некоторые полагают, что надо говорить~--- на <<брелоках>>, но это позорно и преступно). Пока <<фабричным>> способом так выпускается только один дистрибутив~--- Mandriva Flash (варианты на 2 и 4 Гбайт). Но очевидно, что братья во юзерофилии от нее не отстанут, и в ближайшее время следует ожидать аналогов и от остальных любителей <<чисто вымытых>> пользователей. А пока~--- любому квалифицированному линуксоиду не составит труда изготовить таких флэшек со своим любимым дистрибутивом для друзей и знакомых. Не говоря уже о себе, любимом.

А дальше~--- больше: можно ожидать создания флэш-дистров (назовем это явление так) на разные случаи жизни, с разным десктоп-окружением и разными наборами прикладного софта, от рабочего места секретарши-блондинки до рабочей станции наиглавнейшего босса. Ведь на это не потребуется отдельных машин~--- того, что раньше называли программно-аппаратными комплексами, нет, все это будет функционировать на любой подручной PC'шке.

Конечно, пока использование флэш-дистров также ограничено~--- не сошли еще с рабочих мест машины, BIOS которых не предусматривает загрузки с USB-носителей. Однако не далек день, когда, имея такой брелок в кармане, можно будет гарантированно загрузить свою систему со своими же самыми необходимыми данным в любой точке земного шара, охваченной компьютеризацией.

Есть и второе ограничение применимости флэш-дистров: ни двух-, ни четырех-, ни сколько-угодно-разумно-гигабайтная флэшка не в состоянии удовлетворить потребности всех пользователей как устройство массовой памяти~--- хранилище потребных данных (хотя очень многим из пользователей и такие объемы по делу избыточны). И тут вспоминаем о третьей тенденции развития современного IT-мира~--- онлайновых сервисах.

И тут, конечно, первым делом на память приходит Google. Думаю, что подавляющее большинство читателей этой заметки активно используют его поисковую машину (хотя, говорят, некоторых на \url{google.com} забанили, а у иных же Google, бл\dotsин, сломался). Многие пользуются почтовой системой gmail (оно же гымыло в народе) как дублирующей, а кое-кто переводит её в ранг основной. Служба Blogger'а позволяет не только вести личные дневники, но и создавать практически полноценные контент-сайты (по крайней мере, не хуже большинства доморощенных <<хомяков>>). Ну а всякие там Picasa, Блокнот, Документы и таблицы~--- это уже в сущности средства (почти) полноценной работы в (почти) любых условиях. По крайней мере, доступ к данным и средства их обработки они обеспечивают.

При этом нужно учесть, что в области онлайновой работы Google~--- отнюдь не впереди планеты всей. В Сети можно отыскать несколько проектов развития практически полноценных офисных пакетов, предназначенных для работы в онлайне. Например~--- \url{http://zoho.com/}. Не то чтобы он выделяется в ряду прочих какими-то несравненными достоинствами~--- просто эта ссылка попалась мне под руку при сочинении настоящей заметки (за что спасибо Навигатору).

Сказанное выше перекликается с тем, что изложено в статье Пола Грэма <<Microsoft мертва>>. Однако значит ли это, что не сегодня, так завтра Linux полностью вытеснит Windows с рабочих столов, а Билл Гейтс отправится просить милостыню? Отнюдь. Логическим завершением описанных тенденций будет то, что понятие операционной системы для большинства конечных пользователей просто потеряет физический смысл. Действительно, какая разница, что за ОС грузится на данной машине, если все, что от нее, машины, требуется~--- это войти в Сеть и запустить привычный текстовый процессор или электронную таблицу на удаленном сервере. Каковые будут выглядеть и функционировать совершенно одинаково, вне зависимости от того, были ли они запущены через <<подоконный>> Internet Explorer, Linux'овый Konqueror или кросс-платформенный FireFox.

Будет ли внедрение онлайнового стиля работы способствовать вытеснению Windows с пользовательских столов? В какой-то мере~--- да. Разумеется, некоторое количество пользователей покинут эту платформу просто по финансовым соображениям: зачем платить за среду, единственной функцией которой является запуск нескольких онлайновых служб, с чем прекрасно справится и практически бесплатный Linux. Но масштабы этого явления преувеличивать не следует. Ведь и Windows более или менее справится с этой задачей, и уходить с нее помешает просто обычная инерция мышления~--- людей, которые предпочтут заплатить и забыть, не так уж и мало. Тем более, что при таком раскладе Microsoft будет просто вынуждена снизить цены на свою продукцию~--- причем не только на операционки, но и на офисные приложения. Вплоть до того, что официальные диски будут продаваться по вполне пиратским ценам\dots

Не стал бы я, в отличие от Пола, преувеличивать и роль Macintosh'ей в вытеснении Windows~--- по крайней мере, на Руси и в сопредельных странах. И тут причина банальна~--- стоимость. Если в Американщине цена на Маки практически сравнялась с чисто PC'шными брендами, то у нас почему-то до этого далеко. В чем причина, в таможенной ли политике или особенной жадности дистрибьюторов Apple,~--- не знаю и гадать не хочу, но факт остается фактом. И потому круг пользователей этой платформы особенно не расширится~--- тем более за счет интересующихся компьютерами. Да, я лично знаком с юниксоидом старого закала, перешедшим на Mac. Однако знаете, что больше всего его восхитило в MacOS X? Наличие терминала, в котором запускается настоящий bash со всеми его прибамбасами\dots

В общем, пора подвести итоги. Да, смена парадигмы общения пользователей с компьютерами неизбежна, как крах мировой системы социализма. Однако, как и последний, ничего принципиально в мировой расстановке сил это не изменит.

Те, кто понимает, зачем им нужен Linux (и другие свободные UNIX-системы), и кто нуждается в их функциях, будут продолжать их использовать. 

Те, кому по жизни и по службе требуется функциональность Mac'а, будут их покупать и использовать.

Основная же масса пользователей, которым глубоко безразлично не только имя стоящей у них на машине операционной системы, но даже сам факт наличия таковой, будут по прежнему использовать Windows.

И это не хорошо и не плохо: это, как сказал бы Остап Бендер, медицинский факт, с которым надо считаться.

\section{Кое-что о русском Linux'е} 

\begin{timeline}26 декабря 2007 г\end{timeline}

\hfill \begin{minipage}[h]{0.45\textwidth}
Забыл я автора,\\
И Google, бл\dotsин, сломался\\
Распалась связь времен.
\begin{flushright}
\textit{Народное POSIX'ивистское}  
\end{flushright}
\bigskip\end{minipage}

Собственно, проблемы русского Linux'а будут рассматриваться в другой книжке, которую я надеюсь подготовить в обозримом будущем. А для настоящей заметки появился конкретный повод: присланная в редакцию сайта Citkit статья Алексея Дмитриева <<Русский Линукс: первоочередные задачи>>. Которая являет собой пример концентрированного собрания предрассудков, заблуждений и прямых ошибок, связанных с заявленной темой. 

Я, конечно, отдаю себе отчет в том, что написана она, судя по всему, автором, инсталлировавшим свой первый дистрибутив Linux'а пару-тройку месяцев назад. И потому прошу его принять все сказанное мной далее не в обиду, а как информацию к размышлению. Тем не менее, совсем без ответа я её оставить не могу. Ибо практически каждой своей фразой автор дает повод для претензий по существу. А уж поводом для мелких придирок может быть любое слово. 

В связи с этим всё дальнейшее изложение будет представлять собой своего рода виртуальный диалог по схеме <<цитата~$\rightarrow$  комментарий>>. Цитаты даются \textit{as is}. 

Начать можно прямо с первой фразы: 
\begin{shadequote}{}
Итак, Линукс уверенно укоренился на Российской почве.
\end{shadequote}
О, этот сакраментальный оборот \textit{итак}! Создающий впечатление, что только вчера в российскую почву бросили семя Linux'а~--- а сегодня он уже пустил в ней корни. Если это не просто литературный штамп, должен несколько разочаровать автора: Linux укоренился на российской почве\dots ну как минимум года с 94-го. Прошлого тысячелетия, разумеется. А уж времени жизни его предыдущим инкарнациям~--- вариантам UNIX'а,~--- счет идет на десятилетия. 


\begin{shadequote}{}
Последние версии таких дистрибутивов как MopsLinux и AspLinux, похоже, ни у кого не оставили сомнения~--- да, можем!
\end{shadequote}
Хочется спросить автора: а что, предпоследние версии этих дистрибутивов оставляли сомнения? Или повод для сомнений дает дистрибутив Altlinux от одноименной российской компании? Или русская редакция Scientific Linux, выпускаемая питерской фирмой Linux Ink, выглядит в чём-то сомнительной? 

Если уж спуститься совсем в седую древность, то можно вспомнить также KSI и BlackCat с братской Украины. А о пионерах отечественного дистростроения, УрбанСофт и IPLabs Linux Team (ныне они носят упомянутые выше имена Linux Ink и Altlinux), как-то даже и говорить становится неловко. 

В общем, куваевский бич применительно к дистрибутивам Linux свою знаменитую фразу~--- <<Могём, начальник! Кто сказал, что не могём?>>~--- мог бы смело произнести году в 2000, как минимум, а то и раньше. 

Если же вспомнить, что среди разработчиков международных проектов, например, FreeBSD, также немал процент русскоговорящих (хотя и не обязательно российскоподданных или российских резидентов), то отпадут последние сомнения в том, что <<наши люди~--- фигли говорить>>. 
\begin{shadequote}{}
Можем делать дистрибутивы не хуже, а для нас, учитывая все сложности русификации, и лучше западных.
\end{shadequote}
Оставим в стороне вопрос, что такое <<лучше>> и <<хуже>> применительно к дистрибутивам Linux, а также проблему Востока и Запада. Но вот утверждение о сложности русификации последних лежит в области легенд и мифов Древней Греции. По крайней мере наиболее распространенные <<западные>> дистрибутивы, такие, как Red Hat, Suse, Debian, Ubuntu, не говоря уже об изначально интернациональной Mandriva, давно способны работать с языком родных осин ничуть не хуже, нежели самые-рассамые отечественные разработки. 
\begin{shadequote}{}
Настало время определить пути развития русского Линукса\dots
\end{shadequote}
Что же такое случилось, что именно сейчас настало это время? Неужели выход последних версий MopsLinux и AspLinux? Ведь до сих пор русский Linux как-то спокойно развивался естественным образом, без всяких определений его пути посредством чьих бы то ни было директивных указаний\dots 
\begin{shadequote}{}
\dotsчтобы не копировать слепо западное направление развития.
\end{shadequote}
\dotsпричем не копируя слепо западное направление, а в рамках всеобщих тенденций дистростроения. В частности, и в направлении интернационализации~--- например, развития тюркскоязычных локалей. Ведь это вполне естественно~--- в дистрибутивах южноафриканского происхождения развивается и расширяется поддержка, скажем, языка зулусов, а в дистрибутивах России~--- татарская, казахская и другие локализации. 
\begin{shadequote}{}
Линус Торвальдс, с упорством маньяка, печет как блины новые релизы ядра~--- каждые 2-3 месяца новый релиз.
\end{shadequote}
Да уж, сильно сказано. Если Линус~--- маньяк, то я как минимум китайский император. 

И вообще-то Линус не печет релизы, а с упорством\dots не маньяка, а энтузиаста, занимается разработкой тех вещей, которые лично ему интересны. При этом рассматривая и координируя патчи сторонних разработчиков, касающихся тех подсистем ядра, которыми он сам лично не занимается. Причем делает это абсолютно добровольно. 

И релизы ядра выпускаются не с какой-то периодичностью, и не из маниакального стремления к переменам, а тогда и потому, когда таких патчей становится достаточно много и они проходят достаточную проверку. Если бы автор удосужился прочитать, как организована разработка ядра, например, в статье Романа Химова он бы, надеюсь, воздержался от таких заявлений. 
\begin{shadequote}{}
\dotsно может быть лучше довести до ума какой-нибудь один?
\end{shadequote}
Ввиду отмеченной выше добровольности работы Линуса (и других ключевых разработчиков) по координации разработки ядра, предъявлять им какие-либо претензии~--- как минимум, не тактично. 

Вообще, меня просто умиляют заявления типа~--- <<лучше бы разработчики делали то, а не это>>. Разработчики делают то, что считают нужным, что они умеют и что им интересно. Если вы полагаете, что они делают что-то не то или что-то не так~--- вспомните, что это же Open Source. То есть никто не запрещает вам сделать что-то другое, по вашему мнению, более нужное, или что-то, на ваш взгляд, лучшее. Так что флаг в руки, барабан на брюхо~--- и вперед, с песнями, к своей идеальной разработке. 

Маленькое отступление, имеющее, тем не менее, прямое отношение к теме. Наиболее часто заявления указанного выше характера можно слышать в отношении дистрибутивов: зачем разрабатывать их столько, когда лучше довести до ума какой-нибудь один? 

Ну во-первых, как было сказано в предыдущем абзаце, если вы считаете, что это лучше~--- берите какой-нибудь один и доводите его до того, что полагаете <<умом>>. 

А во-вторых, если дистрибутив разрабатывается~--- значит, он кому-то и зачем-то нужен. Как минимум~--- своему разработчику. Если дистрибутив не нужен никому~--- проект просто тихо умирает. Правда, мне известен дистрибутив, который жил, хотя не нужен был даже своим разработчикам~--- но это один единственный пример на без малого пятьсот зарегистрированных на \url{http://distrowatch.com}. Да и он в конце концов умер своей смертью\dots
\begin{shadequote}{}
Ведь что движет группой разработчиков ядра~--- они стремятся поспеть за стремительно развивающимся <<железом>>.
\end{shadequote}
В основном разработчики занимаются совершенствованием подсистемы управления памятью, планировщика задач, файловых систем, виртуализации, а также специализированными задачами типа систем безопасности, реального времени или кластерных вычислений. 

Разработка драйверов устройств для поддержки стремительно развивающегося <<железа>>~--- лишь одна, и далеко не главная, сторона их деятельности. Как раз наоборот~--- желающих заниматься этой рутинной работой из голого энтузиазма не так уж много. И это одна из причин того, что поддержка <<железа>> в Linux'е не идеальна. 
\begin{shadequote}{}
А что побуждает железо стремительно развиваться? Софт. И уж конечно не свободный софт, а проприетарный, и в первую очередь Микрософтовский.
\end{shadequote}
Тут автор просто перевернул все с ног на голову. Как раз наоборот: новый софт, и проприетарный, и свободный, развивается, чтобы использовать возможности стремительно развивающегося <<железа>>. И только тогда, когда развитие этого <<железа>> позволяет. Вы можете представить себе Windows Vista или KDE с трёхмерными эффектами, работающие на процессорах P-166 и видеокартах от S3 с мегабайтом бортовой памяти? То-то же\dots 
\begin{shadequote}{}
Свободные исходники и влившийся в них мощным потоком Линукс с самого начала противопоставляли себя Микрософт и иже с ними.
\end{shadequote}
Для начала, будучи старым бюрократом, не могу не отметить терминологической неточности. Если бы автор, опять же, не поленился ознакомиться с соответствующей литературой, он бы узнал, что никаких свободных исходников не существует. А существуют движение за свободные программы (Free Software~--- на всякий случай подчеркну, что его не следует путать с понятием Freeware, то есть программ бесплатных) и движение за открытые исходники (Open Source). И это не идентичные, хотя и родственные, явления. 

Далее, движение Free Software не могло с самого начала противопоставлять себя Микрософту хотя бы потому, что возникло тогда, когда Гейтсу и Аллену можно было только мечтать не то что о всемирной монополии, но и даже об общем признании. Ибо разрабатываемые компанией продукты в то время предназначались для сугубо маргинальной сферы IT-индустрии. 

Что же до Linux'а~--- так Линусу в 1991 году был глубоко параллелен сам факт существования компании Микрософт. Подозреваю, что в глубине души он ему параллелен и по сей день. 
\begin{shadequote}{}
А в итоге оказались в качестве догоняющего в навязанной им гонке <<железа>>.
\end{shadequote}
Вообще-то, за почти 10 лет работы в Linux'е (а также и BSD), я никакой такой <<гонки железа>> не заметил. Просто на каждом этапе развития оного разрабатываются программы, использующие новые его возможности. Например, в 1998 году KDE~1-й версии прекрасно работал у меня на машине с P-II/266 и 64-мя мегабайтами оперативной памяти. Ныне он столь же прекрасно работает при двухъядерном процессоре AMD с реальной тактовой частотой 3~Ггц и 2-мя гигабайтами <<мозгов>>~--- в своей третьей ипостаси, имеющей, по сравнению с 1-й, массу полезных и удобных, а то и просто красивых, <<фич>>. Появление которых и стало возможным вследствие развития <<железа>>. 
\begin{shadequote}{}
Несомненно, что эта гонка должна вестись, но она не должна быть единственным направлением разработок.
\end{shadequote}
Как уже было сказано выше, это и так далеко не единственное направление разработок. Тем более, что и гонка-то на самом деле не не имеет места быть\dots 
\begin{shadequote}{}
А другим направлением, к сожалению второстепенным, является <<доведение до ума>> уже существующего в мире свободных исходников.
\end{shadequote}
Тут автор высказался весьма лихо. Во-первых, именно <<доведением до ума>>, то есть исправлением ошибок, добавлением мелких полезных <<фич>> и тому подобной рутинной, но необходимой работой занимаются тысячи и тысячи людей во всем мире. И цитированная фраза являет собой вопиющее неуважение к их труду. 
\begin{shadequote}{}
И тут, как мне представляется, особую роль должны сыграть отечественные разработки.
\end{shadequote}
Так что отдавать какой-либо приоритет в этом именно отечественным разработчикам было бы странно. Тем более, что они и так, наряду с прочими народами, выполняют свой интернациональный долг, внося должный клад в процесс <<доведения до ума>>. 

Тем не менее, цитированная фраза знаменует возвращение автора к теме русского Linux'а, от которой он отвлекся рассуждениями о <<гонке железа>>. 
\begin{shadequote}{}
Делать дистрибутивы мы научились.
\end{shadequote}

Как было установлено выше, делать дистрибутивы <<мы>> (вероятно, в смысле~--- <<весь великий советский народ>>? Или в понимании Замятина?) научились давно. И продолжаем учиться~--- то есть совершенствовать уже сделанное. Впрочем, <<не-мы>> поступали и поступают точно также. Более того, нередки случаи, когда <<мы>> и <<не мы>> трудятся плечом к плечу над одними и теми же проектами. 
\begin{shadequote}{}
А что мешает распространению этих дистрибутивов в народные российские массы?
\end{shadequote}

Задается далее автор риторическим вопросом. Должен его огорчить~--- ничего не мешает. Более того, процесс распространения идет очень давно. Только на моей памяти оценочное число только активных пользователей Linux'а возросло на порядок. А если учесть пользователей пассивных~--- тех, кто даже и не знает, какая ОС установлена на его автоматизированном рабочем месте, домашнем десктопе, наладоннике, смартфоне (нужное дописать)~--- не знает, потому что его этот вопрос ни в малейшей степени не колышет,~--- то число таких пользователей возрастет еще как минимум на порядок. 

Тем не менее автор полагает, что распространению Linux'а в массах кто-то или что-то мешает (не происки ли мирового жидо-масонства или проклятых имперьялистов?). И предлагает свои варианты ответа, что мешает именно. 
\begin{shadequote}{}
Во-первых, плохое знание английского языка широкими российскими массами.
\end{shadequote}
Тут автор, подобно герою известного анекдота, как всегда оказался не прав. Широкие российские массы соответствующего возраста владеют английским, как правило, лучше, чем их ровесники по моему поколению (автор этих строк тоже не читает Шекспира в подлиннике, мягко говоря; Гёте и Шиллера, язык которых учил в школе и ВУЗе, в оригинале не читает тоже). Иногда при просмотре форумов создается впечатление, что многие из участники английским владеют лучше, нежели родным. Это раз. 

Два~--- для чтения документации отнюдь не обязательно владеть языком Чосера, как родным. Ибо написаны часто отнюдь не нативными носителями оного, и потому очень просты. 

Три~--- очень большое количество документации переведено и переводится на русский язык, о чем автор, вероятно, не осведомлен. 

И четыре~--- никакой словарь, будь он хоть трижды ABBYY Lingvo, не поможет в понимании специфических терминов, если нет понимания сути терминов этих. 

Так что увы~--- но 
\begin{shadequote}{}
\dotsсоздание современного, удобного, максимально полного, и современного англо-русского электронного словаря
\end{shadequote}

\dotsи на фиг никому не нужно. Всё равно без минимально-технического английского не обойдется ни один активный линуксоид. Ибо даже при стопроцентно русскоязычной документации ему придется читать и понимать системные сообщения, комментарии к конфигурационным и Make-файлам, всякого рода \texttt{README}, переводить которые на русский, хвала Аллаху, еще никто не додумался (хотя в Сети можно найти упоминания о проектах создания исконно русской операционки, в которой даже информация о ходе загрузки выводилась бы на Великом и Могучем). 

Для линуксоида же пассивного чтение документации~--- не его вахта, ибо все, что касается системной части, за него и для него должен сделать линуксоид активный. И ему останется только осваивать необходимые приложения~--- а уж с русификацией большинства их, как и рабочих сред типа KDE и GNOME, дело давно уже налажено. 
\begin{shadequote}{}
Второй сдерживающий фактор освоения Линукса~--- руководства, пресловутые <<маны>>.
\end{shadequote}

Да, это уже переходит всякие границы\dots Каждый, все-таки освоивший Linux (а таких людей, как было отмечено ранее, не так уж и мало), скажет, что смог сделать это благодаря чтению 
\texttt{man}'ов. И невдомек ему, бедняге, что это был сдерживающий его развитие фактор. Тем же, для кого понять man'ы <<задача почти непосильная>>~--- так может быть, оно им и не надо, этого Linux'а? Так что, действительно 
\begin{shadequote}{}
Простым переводом на русский проблему манов не решить
\end{shadequote}
И не потому, что
\begin{shadequote}{}
в переводе непОнятости накладываются на непон\`{я}тности
\end{shadequote}
Тут скорее нужно чего-то поправить в консерватории. 

Да и вообще, чтение \texttt{man}'ов <<в подлиннике>> предпочтительней по одной простой причине: перевод по определению всегда будет отставать от оригинала в отношении версии. И потому новые опции и <<фичи>> могут не найти в нем отражения. 

Кстати, уж что-что, а язык man'ов предельно прост. И не потому, что написаны они не нативными <<англичанами>>, а потому, что вполне сознательно сочинялись <<для всех>>~--- в том числе и не очень твердых в аглицкой мове. Я лично знаю людей, владеющих английским еще хуже меня~--- и тем не менее вполне успешно разбиравших \texttt{man pages} по интересующим их вопросам. 

Для тех же, кто не в ладах с вражьей мовой до степени полного непонимания (и заодно для автора), открою страшную тайну: работа по переводу \texttt{man}'ов ведется уже много лет, историю чего можно прочитать на сайте Виктора Вислобокова. Да и майнтайнеры дистрибутивов не остаются в стороне от этого процесса~--- причем не только наши, родные, но и такие западные акулы, как Novell и Red Hat. 

Ну а уж следующий тезис автора: 
\begin{shadequote}{}
Необходимо организовать народный проект по написанию внятных мануалов, а по-нашему~--- руководств.
\end{shadequote}

\dotsспособен вызвать только взрыв здорового смеха. К коему предлагаю присоединиться и автору. Ибо, как Мальчиш-Плохиш, готов выдать и вторую военную тайну: такой народный проект стихийно, без всяких <<указаний сверх>>, самоорганизовался много лет назад, начиная с первых сетевых публикаций и <<бумажных>> статей Владимира Водолазкого, онлайновых заметок Виктора Вагнера и Владимира Игнатова, проекта под названием <<Пособие для начинающих ковбоев>>, руководства по UNIX Сергея Кузнецова, и многих, многих других. Приношу свои извинения тем основоположникам линуксописания, кого не упомянул поименно~--- ибо имя вам легион, но мы о вас помним, даже если вы давно уже занимаетесь совсем другими делами. 

И этот <<народный проект>> активно развивается и по настоящий момент. Ибо буквально каждый день на многочисленных Linux-сайтах и в еще более обильных блогах линуксоидов появляются материалы по теме~--- и частные, посвященные решению некоей конкретной проблемы, и общие, вплоть до подробных руководств. 

Чтобы убедиться в этом, автору достаточно было бы зайти на сайт Виктора Костромина, на котором собраны ссылки на все русскоязычные материалы по UNIX, Linux и Open Source. Впрочем, все содержание статьи оставляет впечатление, что у автора не только Google сломался, но и вообще весь Интернет. 

Комментировать предлагаемое автором 
\begin{shadequote}{}
третье направление развития отечественного Линукса~--- программы проверки правописания и пунктуации
\end{shadequote}

\dotsне возникает никакого желания. Ибо это ни что иное, как предложение изобретать очередной велосипед. Возможно, даже с квадратными колесами. Вместо того, чтобы, говоря словами автора и его собратьев по классу, развивать и пополнять русскоязычные словари к существующим программам проверки орфографии и приспосабливать их к работе с <<не устаревшими>> кодировками (если под таковыми имеется ввиду UTF-8, то \texttt{aspell} успешно работает с ней уже много лет). 

В заключение хотелось бы призвать автора последовать своим собственным советам. Во-первых, 

Каждый линуксоид, освоивший какую-либо команду, и способный связно изложить свои знания~--- обязан поделиться с народом!

и действительно чего-нибудь освоить (хотя бы \texttt{man}-страницы). А во-вторых, 
\begin{shadequote}{}
главное, начать действовать в этом направлении.
\end{shadequote}

А не заниматься постановкой задач тем, кто давно в этом направлении именно действует. 

Ну и совсем под занавес. 

Linux и Open Source были и остаются явлениями интернациональными, не признающими ни государственных, ни конфессиональных, ни каких-либо иных границ. Свой посильный вклад в них вносят все, могущие это сделать, не зависимо от языка, вероисповедания, подданства и цвета кожи. Также как всем вольно черпать из этого источника то, что нужно для решения его задач. 

Надеюсь, что существующее положение будет сохраняться и впредь. Для меня апогеем развития <<русского Линукса>> было бы не создание еще одного исконно-посконного дистрибутива или приложения. А увидеть нашего соотечественника на посту лидера проекта Debian или FreeBSD, в составе совета директоров компании Mandriva, одним из ключевых разработчиков Linux Kernel\dots 

Занавес опускается, оставляя читателя в недоумении: зачем было разводить столько турусов на колесах вокруг, в общем-то, тривиальных и общеизвестных вещей. Отвечаю.

Увы, они тривиальны и общеизвестны для тех, кто использует Linux (или прочие BSD'и) в своей работе не первый год. Но в каждом новом поколении линуксоидов обязательно появляются первооткрыватели америк, искренне полагающие, что до них никто ничего не делал и ни до чего не додумался. И начинающие формулировать <<очередные задачи Советской власти на данном этапе>>. И комментируемая статья являет собой нечто вроде квинтэссенции формулировки таких задач. Заставляя вспомнить известный анекдот про Вовочку, когда он, придя 1 сентября первый раз из первого класса, радостно закричал родителям: <<Вот вы, козлы, тут сидите и не знаете, а пиписька-то х\dotsм называется!>>. Почему я и не смог оставить её без ответа. 

Дабы связь времен все-таки не разрывалась, и Google не терял бы устойчивости к поломкам, и не банил бы нас таинственный разум, поселившийся в глобальной сети Интернет~--- тот самый, который был описан Роджером Желязны и Фредом Саберхагеном в замечательной повести <<Витки>>.

\section{Три поколения в русском FOSS-движении}

\begin{timeline}Август 11, 2010\end{timeline}

Как ни относись к образу товарища Владимира Ильича Ульянова в скобках Ленина, нельзя отказать в даре предвидения. В частности, в работе <<Памяти Герцена>> (1912 год) он предсказал весь путь грядущего русского FOSS-движения. Что мы, опираясь на память детства (в школе эти строки заучивались наизусть в обязательном порядке), и проиллюстрируем ниже цитатами из указанной работы классика, а также комментариями к ним.

Итак, подобно тому, как
\begin{shadequote}{}
\dotsмы видим ясно три поколения, три класса, действовавшие в русской революции\dots
\end{shadequote}
\dotsтри поколения можно увидеть и в русском FOSS-движении:

\begin{shadequote}{}
Сначала~--- дворяне и помещики, декабристы и Герцен. Узок круг этих революционеров. Страшно далеки они от народа.
\end{shadequote}

С дворянами и помещиками всё ясно. Это~--- юниксоиды первого призыва, обретавшиеся в своих родовых поместьях~--- закрытых КБ, ящичных НИИ и тому подобных режимных предприятиях. Круг их был действительно узок, а с народом они сосуществовали в параллельных мирах.

Но из недр указанных учреждений вышли первые юниксоиды-декабристы, воззвавшие к народу через каналы Курчатника, Демоса, Релкома. Которые, сложись дело в августе 1991-го чуть иначе, могли бы и составить компанию декабристам в соответствующих местах.

Своим призывом они разбудили нашего однозначного Герцена~--- Владимира Водолазкого. Который, правда, революционной агитации не разворачивал. Но зато написал статью в журнале \textit{``Монитор''} (1994 год) под примерным (по памяти) названием: \textit{``Как без головной боли установить Linux''}. Ксерокопии этой статьи, в сопровождении стопки дискет с дистрибутивом древней версии Slackware, годами ходили потом по рукам. Так что можно сказать, что именно\dots

\begin{shadequote}{}
Ее подхватили, расширили, укрепили, закалили революционеры-разночинцы, начиная с Чернышевского и кончая героями ``Народной воли''.
\end{shadequote}
Безусловно, юниксоиды второго призыва (которых уже можно называть FOSSоидами), были в самых разных чинах, званиях, образованиях~--- от учителей и врачей до географов, почвоведов и даже, страшно сказать, писателей. И за кем из них следует закрепить амплуа Чернышевского~--- вопрос спорный. А вот на звание героев Народной воли однозначном может претендовать команда форума Linuxshop'а, составившая позднее ядро Linuxforum'а (нынешнего \texttt{Unixforum.org}). Именно благодаря им\dots
\begin{shadequote}{}
Шире стал круг борцов, ближе их связь с народом.
\end{shadequote}

И действительно, число разночинцев наших росло. И сохранили они связь с народом~--- то есть со своими производственными коллективами, школами, ВУЗами.
\begin{shadequote}{}
Молодые штурманы будущей бури
\end{shadequote}
звал их Герцен. Но это не была ещё сама буря.

Ну как тут не поразиться прозорливости Вождя? Ведь действительно, не линуксоидам-разночинцам суждено было стать штурманами бури. Ибо\dots

\begin{shadequote}{}
Буря~--- это движение самих масс. Пролетариат, единственный до конца революционный класс, поднялся во главе их\dots
\end{shadequote}

Пророчество, достойное Нострадамуса, Глобы и девицы Ленорман. О сугубо пролетарском происхождении вождей Русской революции~--- начиная с самого Ульянова (Ленина) и корифана его Троцкого~--- знают все. Но ведь не меньшими пролетариями оказались и штурманы нынешней FOSS-бури: пролетариями от политики, от бизнеса, от госаппарата. Именно этот пролетариат\dots
\begin{shadequote}{}
\dotsвпервые поднял к открытой революционной борьбе миллионы крестьян.
\end{shadequote}

То есть тех самых новообращённых линуксоидов <<от бороны>>, у которых ожидание доброго барина сменилось вожделением Большой Красной Кнопки.

\begin{shadequote}{}
Первый натиск бури был в 1905 году.
\end{shadequote}
И опять-таки~--- через столетие Вождь угадал с точностью до года: именно в 2005 году по тихой FOSS-заводи пошли барашки, предвещавшие будущие баллы по Бофорту. Правда, мы тогда этого ещё не поняли, приняв первые порывы ветра за освежающий бриз.

\begin{shadequote}{}
Следующий начинает расти на наших глазах.
\end{shadequote}

И продолжает это делать. посмотрим, до чего он вырастет к столетнему юбилею ленинской работы\dots

\section{Применители vs потребители}
\hypertarget{customers}{}

\begin{timeline}Октябрь 19, 2012\end{timeline}

<<Когда машины были большими>> и работали в пакетном режиме, люди, тем или иным образом связанные с компьютерами, делились на две неравные категории: те, кто имели дело непосредственно с вычислительным комплексом, назывались операторами, прочие же, те, кто обеспечивал их работой\dots да вроде никак они специально не назывались. Были, конечно, ещё и программисты~--- но они или каким-то боком примыкали к сословию операторов, либо попадали в категорию <<прочих>>.

Появление систем разделения времени и терминального доступа отменило сословие операторов. Но вместо этого вызвало к жизни кастовое деление~--- на системных администраторов, имеющих доступ к святая святых~--- управляющему терминалу, иначе консоли, и простых людей, вынужденных довольствоваться терминалами обычными. За этими самыми простыми людьми со временем закрепилось имя пользователей (users, они же юзер\'{а}, \'{ю}звери и прочие усер\textit{а}), хотя по жизни они могли быть и программистами.

Казалось бы, персоналки должны были уравнять всех в правах, подобно изобретению полковника Кольта. И поначалу так оно и было, поскольку каждый пользователь ПК поневоле был и сам себе админ, и где-то даже сам себе программер. Особенно начиная с того времени, как для самых демократичных из персональных платформ, x86, появились столь же демократичные UNIX-подобные операционные системы, открытые и свободные~--- сначала BSD-семейства, а затем и Linux. Их пользователи, решая свои профессиональные задачи, занимались и администрированием (хотя бы в масштабах одной отдельно взятой квартиры или даже персоналки).

Впрочем, и Mac'и, и Windows-машины тоже некогда использовались почти исключительно в производственных целях~--- именно в те годы и появилось выражение <<кустарь-одиночка с персональным компьютером>>. Который, в отличие от Виктор Михалыча Полесова, занимался задачами отнюдь не кустарными~--- а теми, что раньше требовали мощи корпоративных или государственных издательств, вычислительных центров и так далее.

А потом, с экспансией сначала мультимедии, а затем и массового Интернета, понятие <<пользователь>> стало размываться: этим словом (иногда с ругательным оттенком) стали называть и конструктора в CAD-системе, и писателя или переводчика, вбивающего в <<цифре>> свою или чужую нетленку, и меломана, и записного игрока, и просто завсегдатая <<одноглазников ффконтакте>>. И ныне, особенно с широким распространением сугубо развлекательных гаджетов, сравнимых по вычислительной мощности с суперкомпьютерами недавнего прошлого, вообще утратило всякую определённость.

Эта тенденция была понятна всем~--- требовалось её только как-то терминологически оформить. И такие попытки были. В частности, распространилось деление всех так называемых пользователей на потребителей контента и его создателей. С первым термином трудно не согласиться: назвать человека, околокомпьютерные интересы которого сводятся к социальным сетям и просмотру Youtube, пользователем персонального компьютера~--- как-то язык не поворачивается.

А вот второй термин представляется мне позорным и преступным. Ибо открытым текстом подразумевает, что все, кто не потребители, только и заняты тем, что клепают контент им на потребу.

Поэтому давеча я пришёл к выводу, что слово <<пользователь>> должно быть изъято из нашего лексикона~--- по крайней мере из своего я постараюсь его изъять (хотя по первости наверняка буду забывать). А взамен собираюсь оперировать терминами <<применитель>> и\dots ну куда же от него денешься, конечно же, потребитель.

С последним всё ясно~--- это тот, кто потребляет всё, что ему предлагают, от игр и фильмов до социальных сетей. И который, в сущности, действительно не нуждается в компьютере, каким бы тот <<потребительским>> не был: его потребности за глаза перекрываются гаджетами типа смартфонов и планшетов.

А вот применитель~--- этот тот, кто применяет компьютер для решения своих задач. И задачи эти могут быть самыми разными~--- от сочинения стихов и романов до проектирования самолётов и даже, страшно сказать, написания кода операционных систем. И который в своём деле обойтись без персоналки не может~--- будь она личной или служебной, настольной или мобильной. Но должна она быть именно полноценной рабочей машиной, а не нишевым гаджетом.

Это не значит, что применителю запрещается применение в своих целях тех же гаджетов. Однако цели его таковы, что гаджет для него~--- инструмент сугубо вспомогательный.

Так что страхи производителей ПК и комплектующих к ним, вызванные ростом продаж потребительских гаджетов на фоне снижения спроса на десктопы и даже ноуты, выглядят несколько наигранными. Да, возможно, что настоящие персоналки будут приобретаться только применителями~--- почти все потребители полностью перейдут на гаджеты. Да, вследствие этого производство ПК снизится, а цены на них, как товар не массовый, возрастут. Но, пока на свете существуют применители, спрос на них будет~--- и спрос платёжеспособный.

Потому что~--- рискну высказать крамольную мысль~--- для профессионала цена его профессионального инструмента большого значения не имеет. Или имеет только на стадии вхождения в профессию. А уж дальше~--- если профессионал не смог заработать на адекватный его задачам инструмент, значит, он ошибся в выборе профессии.

И кто знает, не возродится ли благородное ремесло <<кустаря-одиночки с персональным компьютером>> на новом витке диалектической спирали?

\section{Применители-понахьюсты} 

\begin{timeline}Август 5, 2013\end{timeline}

Те из моих читателей, что застали зарю перестройки, возможно, помнят первый телемост СССР--США, который вели Владимир Познер из Ленинграда и Фил Донахью из Сиэтла. Помнят хотя бы потому, что уже на следующий день половина страны задавалась вопросом:

-- Что общего между Познером и Донахью?

На что вторая половина отвечала:

-- То, что им всё понахью\dots

Давеча мне это вспомнилось в связи с этапами онтогенеза типичного линуксоида-применителя. На первом этапе он с увлечением предаётся настройке системы, в том числе и её интерфейса, благо большинство рабочих сред дают для этого массу возможностей. И изрядную часть времени, свободного от собственно применительства, тратит на подбор обоев своего десктопа (а то и создание собственных), опробование тем и стилей окон и пиктограмм, эксперименты с интерфейсными шрифтами.

Переход ко второму этапу знаменуется тем, что у линуксоида-применителя складываются  устойчивые предпочтения в отношении внешнего вида его рабочего стола. И потому свежеустановленную систему он настраивает несколькими отработанными до рефлекторного уровня действиями. А то и вообще обходится без них, выявив оптимальные для себя методы переноса настроек из старой системы в новую.

И, наконец, в один прекрасный момент линуксоид-применитель обнаруживает, что, по большому счёту, все эти интерфейсные красоты ему становятся по\dotsнахью. Запускается привычный терминал, текстовый редактор, ворд-процессор\dots (нужное вписать, ненужное вычеркнуть)~--- и хорошо, можно начинать заниматься применительством. Лишь бы буковки были читаемы, да цвета глаз не резали. Наступает третий этап онтогенеза~--- понахьический.

Для меня этот этап наступил в момент недавнего возвращения на Ubuntu. Неожиданно  я обнаружил, что меня не раздражает ни её цветовая гамма, ни кнопки-иконки на панели запуска, ни умолчальный вид окон. Тем более что кнопку закрытия окна во всех рабочих средах переносил на левый край строки заголовка~--- но не из эстетических соображений, а для избежания случайного нажатия. Единственно, что шрифты для моих глаз маловаты~--- но это тоже относится не к эстетике, а к медицине.

Тут я и понял окончательно одну из причин популярности Ubuntu не только среди начинающих пользователей, но и среди многоопытных применителей~--- не смотря на всю убогость возможностей её настройки после стандартной инсталляции: она вполне компенсируется их ненужностью. И при этом не создаёт отвлекающих факторов в виде желания в любую свободную минуту подправить в интерфейсе ещё чего-нибудь.

\section{Ещё раз о потребителях и применителях, а также иных} 

\begin{timeline}Август 30, 2013\end{timeline}

В заметке \hyperlink{customers}{<<Применители vs потребители>>} я не затронул несколько важных вопросов, касающихся этой темы. Постараюсь восполнить эти проблеы, черпая вдохновение в трудах классиков.

Во-первых, подозреваю, что у читателя упомянутой заметки могло создастся впечатление, что слово потребители я употребляю в уничижительном смысле, некую серую массу, ни на что не способную.

Прошу поверить, это не так. И никакого оттенка осуждения в слово потребители в общем случае я не вкладываю. Ибо прекрасно понимаю стремление человека, восемь часов в сутки занятого скучной конторской работой, провести свой досуг в просмотре фильмов о приключениях, в которых он никогда не примет участия, или за играми, которые позволят ему представить себя в роли героя~--- спасителя человечества.

Нет, в самом по себе потребительстве как способе проведения досуга ничего худого я не вижу. Страшен не потребитель сам по себе, а потребитель воинствующий, проникнувшийся идеей потребительства и полагающий, что все вокруг должны быть такими же потребителями, каким стал он сам. В литературе эта позиция блестяще иллюстрируется фигурой Глухова из книжки Стругацких <<За миллиард лет до конца света>>:

\begin{shadequote}{}
Меня злит не то, как он сделал свой выбор,~--- проговорил Вечеровский медленно, словно размышляя вслух.~--- Но зачем все время оправдываться? И он не просто оправдывается, он еще пытается завербовать других. Ему стыдно быть слабым среди сильных, ему хочется, чтобы и другие стали слабыми. Он думает, что тогда ему станет легче.
\end{shadequote}

Но это далеко не самое страшное, и воинствующий потребитель в реальной жизни вовсе не являет собой столь апокалиптической личности, какой его показывают братья~--- они ему очень сильно льстят. Ибо самое апокалиптическое, что я могу себе представить~--- это потребитель обыденный. Для которого потребительство стало настолько привычным, что он даже не подозревает: все используемые им для потребительства технологии некогда были придуманы для применительства, а вовсе не для его удовольствия. И до сих пор ещё сохранились люди, которые иногда занимаются применением этих технологий. Что опять же нашло прекрасное отражение в литературе~--- в романе Владимира Савченко <<Открытие себя>>:
\begin{shadequote}{}
Многое было в этих мечтаниях~--- одно сбылось, другое отброшено техникой. Но вот мечты о том, что транзисторы украсят туалеты прыщеватых пижончиков с проспекта, не было.
\end{shadequote}

Далее, опять-таки могло создаться впечатление, что, говоря о применителях, я молчаливо подразумеваю их высокую квалификацию в области информационных технологий. И это в общем случае не так. Вспоминая эпоху доминирования рабочих UNIX-станций, мы видим, что их пользовтаели~--- чистые применители без намёка на потребительство,~--- часто не имели ни малейшего представления об устройстве того UNIX'а, в котором они запускали свои применительские приложения. И ничуть от этого не страдали~--- критичным для них было не знание системы, а владение своим профессиональным инструментарием.

Скорее как раз наоборот~--- некоторым потребителям для удовлетворения своих потребительских потребностей необходимы довольно обширные познания в области компьютерных технологий. Хотя от необходимости иметь некоторые представления о своих потребительских приложениях не избавлен ни один <<чистый>> до стерильности потребитель, обычно они сводятся к набору простых рецептов, в идеале~--- к нажатию Большой Красной Кнопки с надписью \textbf{Сделай мне пе\dotsто}.

Однако тем потребителям контента, которые, подобно персонажу из бригады Венечки Ерофеева <<с претензиями>>, например, записным геймерам или страстным меломанам, приходится прибегать к изощрённым методам потребления, подчас требующих весьма обширных познаний в таких общесистемных материях, о которых квалифицированный применитель даже и не слыхал. Правда, такой потребитель в пределе перестаёт быть чистым потребителем, и тесно примыкает к той категории пользователей, о которых я раньше не говорил. И которую можно назвать гиками.

Количественно группа настоящих гиков немногочисленна. По роду своей основной деятельности они часто никак не связаны с UNIX'ами, ни с Linux'ами, ни вообще с компьютерными технологиями. По жизни они могут заниматься самыми разными делами. Копание во внутренностях операционной системы, сборка, а подчас и сочинение, пакетов для них~--- не более чем хобби. Но на том самом очень серьёзном уровне, когда хобби напрямую смыкается с профессией.

Причины, по которым люди избирают в качестве хобби копание во внутренностях Linux'а и его приложениях, могут быть самыми разными: и неудовлетворённость основной работой с точки зрения творческой самореализации, и стремление к периодической смене деятельности, и самое обычное любопытство, которое затягивает сильнее любого наркотика.

Многие из гиков со временем уходят в области профессионального применения Linux'а и FOSS. Однако это происходит далеко не всегда~--- как по объективным причинам, обусловленным принудительной силой реальности, так и по субъективным, потому что основная сфера деятельности необходима для самореализации не меньше, чем серьёзное хобби.

Роль гиков в развитии FOSS трудно преувеличить~--- достаточно сказать, что Linux изначально возник как побочный результат действий студента-энтузиаста. Правда, для Торвальдса разработка его операционки стала профессией. Однако не менее показателен и пример Кона Коливаса, совмещающего сочинение патчей к ядру, реализующий его собственный планировщик задач, с работой врача-анестезиолога.

И примеров как из первого, так и второго можно привести очень много. Так, для Билла Рейнолдса, также известного как Texstar, разработка и сопровождение дистрибутива PCLinuxOS стали профессией, а Жан-Филипп Гийомен, создатель дистрибутива Zenwalk, продолжает трудиться инженером по безопасности в корпорации Телиндус~--- в области, хоть и связанной с IT, но с разработкой Linux'а пересекающейся слабо.

Однако в последнее время гики, как и применители, становятся всё менее заметны на общем фоне пользователей-потребителей. Не потому, что любопытствующих энтузиастов становится меньше~--- изменились времена и сместились системы приоритетов. Ещё лет 10 назад роль независимого разработчика или применителя могла стать венцом профессиональной карьеры в любой сфере деятельности, то ныне для потенциальных гиков пределом мечтаний является служба в большой и престижной корпорации. Где они вынуждены делать не то, что считают нужным, а то, что прикажут. Прикажут же в большинстве случаев~--- создавать нечто на потребу потребителям. Последние же, согласно словам персонажа романа Куваева, завоюют мир, как раньше
\begin{shadequote}{}
\dotsмир захватывали чингисханы, тамерланы, македонские или орды, допустим, гуннов.
\end{shadequote}

После чего и наступит предсказанный Стругацкими конец света, и до него остался вовсе не миллиард лет\dots

\section{Наступит ли Эра Linux'а?} 

\begin{timeline}2001\end{timeline}

\textsl{Первая версия это заметки была написана (как обычно, по случаю) осенью 2001 г. и опубликована на онлайновой Софтерре (ныне кусок сайта Компьютерры). Поводом для нее послужило как бы интервью, взятое Максимом Отставновым у Евгения Козловского, касаемое опыта общения последнего с Linux'ом (Компьютерра, 2001, \textnumero33 (410), с. 40-41. Прошу понять меня правильно~--- я не ставлю себе задачей доказывать, какой Linux хороший. И убеждать, что все упомянутые в статье проблемы решаются не просто, а очень просто.}\medskip

И вообще, тему проблем я поднял только потому, что затронута честь одного из продуктов издавна (с января 98-го) любимой мной команды Altlinux (ранее~--- IPLabs Linux Team). Каковая всегда отличалась именно беспроблемностью (по Linux'овым стандартом) своих дистрибутивов. Непосредственно с вариантом MSI Edition мне довелось пообщаться только в бета-версии, но и этого было достаточно для заключения: редакция эта в плане беспроблемности ничуть не уступает всем прочим представителям своего поколения (в числе которых~--- Linux Mandrake Spring 2001 и Junior).

Так что проблемы, с которыми столкнулся Евгений, можно отнести только к фатальному невезению. Так, трудновато найти сетевую карту, которая не определялась бы почти любым из развитых (т.~н.~user-oriented) дистрибутивом (вернее, его инсталляционной программой). Программа kudzu, входящая в состав дистрибутивов от Altlinux (как, впрочем, и многих других), обычно вполне успешно выполняет роль P'n'P из Windows: вплоть до того, что успешно определяет телетюнеры на распространенных чипах (типа BTxxxx).

Ситуацию с CD иначе чем недоразумением не назовешь. К сожалению, Евгений не написал, каким образом он пытался получить к нему доступ. Не щелчком ли по иконке на рабочем столе KDE?~--- ни за что не поверю, что ответ типа Permission denied мог быть получен на \texttt{mount /mnt/cdrom} от лица root'а.

Так что, возможно, достаточно было проверить (стандартным и для Windows способом, через пункт \textbf{Свойства} контекстного меню по щелчку правой клавишей мыши), на правильное ли устройство указывает иконка CD-ROM. А на худой конец~--- посмотреть, ссылкой на какое устройство является \texttt{/dev/cdrom}: в системе с пишущим CD-R/RW их может быть два, одно~--- для стандартного IDE-привода, другое~--- для него же в режиме эмуляции SCSI.

А самое главное~--- именно для дистрибутивов Altlinux (не думаю, что MSI Edition в этом отношении отличается) нет необходимости лазать по Сети в поисках недостающих компонентов. Поскольку на их сервере имеет место прибывать метадистрибутив Sysiphus, содержащий немерянно всякого софта (в том числе и мультимедийной направленности), скомпилированного для беспроблемной установки в любых дистрибутивах Altlinux.

Повторяю, все это я написал не в упрек~--- а исключительно к вящей славе Linux вообще и дистрибутивов Altlinux в частности. Тем более, что, по собственному признанию Евгения, он и не пытался решать возникшие проблемы. сославшись на отсутствии мотивации. Вот вопрос мотивации перехода на Linux и является, собственно говоря, предметом настоящей заметки. Но сначала~--- цитата:

\begin{shadequote}{}
98\% известных мне <<офисных>> задач решается под Linux так же, как и под Windows. 
\end{shadequote}

Если знаки препинания в этой фразе расставлены в соответствие с авторским замыслом~--- а я склонен думать, что Евгений, при его опыте, не проглядел бы ошибки набора,~--- это на корню убивает всякое желание переходить на Linux. Но дело в том, что это в корне неверно: задачи решаются под Linux, также как и под Windows. Но~--- совершенно иными методами. 

Действительно, под Linux можно сочинять в StarWriter'е заметки, подобные этой, как и в Word'е под Windows. Однако делать это~--- примерно то же, что ехать по хайвею на вездеходе: <<можно бы, да на фига>>, на сей предмет <<мерседесы>> придуманы. А вот попытка проехать на <<мерседесе>> (или даже джипе) по корякской тундре вряд ли доставит удовольствие, здесь уж ГТС'ка потребна (а лучше~--- ГТТ'уй подавай). 

Так каковы же могут быть мотивы к Linux-миграции у человека творческой, как говорили при соввласти, профессии? Пример первый, предельно элементарный, но наглядно демонстрирующий, тем не менее, различие подходов. 

Дано: десять файлов, составляющих нетленный роман из жизни компьютеров (или любимых кошек, без разницы, рассеянных по разным каталогам (или, как принято говорить в мире Windows, директориям), а то и накопителям. 

Требуется: объединить их в некоем порядке в единый текст, записав последний в новый каталог. Будем считать, сколько погонных метров потребуется накрутить шарику мыши по её подмышнику, и сколько~--- бить указательным пальцем по её кнопке? Или, может, просто выполним команду:\medskip\texttt{\$ cat /path1/file1~\dots~/path10/file10 > /newpath/newfile}\medskip

Напомню, что при этом нет необходимости тупо фигачить по клавишам, вбивая многочленные пути и замысловатые имена: к вашим услугам клавиша автодополнения. А добавив сюда конвейеризацию команд, можно и сформатировать текст в соответствии с его логической структурой, и вывести на печать. Обращение же к скриптам все эти (и многие, многие другие) задачи автоматизирует до любого предела, устанавливаемого только собственной фантазией. 

Я уж не говорю о возможностях, предоставляемых внешне непритязательными текстовыми редакторами: и по созданию той же логической структуры, и по её разметке для дальнейшего форматирования, и по поиску фрагментов, смысл которых помнится лишь приблизительно, и их автоматической замене. 

Разумеется, работа в Windows может показаться комфортнее: щелкать мышью, развалясь в кресле~--- это почти как лететь в МИ-8 над той же корякской тундрой. Разумеется, при наличии погоды, керосина, хороших отношений с отделом перевозок (иначе ведь могут забыть снять~--- бывало и такое), не говоря уже о деньгах. Однако, если вспомнить об авральной загрузке в машину двух-трех тонн груза, столь же авральной разгрузке по посадке, днях и неделях ожидания борта, многокилометровых подходах при неудачном выборе точки заброски~--- чувство комфорта куда-то исчезает. Начинаешь с тоской вспоминать любимую ГТС'ку, на которой, при всех скинутых гусянках, вылетевших пальцах, лопнувших торсионах, проедешь, где угодно. 

Именно стремление к тому, чтобы выполнить ту же работу иначе (быть может~--- более эффективно)~--- и есть, наряду с элементарным любопытством, мотивом к переходу на Linux. Что же касается эффекта <<повелителя мух>>~--- не думаю, что это может послужить стимулом для человека, выросшего при советской власти и не ставшего партийным секретарем. Системой не нужно повелевать, достаточно её любить~--- и она, как женщина, ответит взаимностью\dots 

Так что на вопрос из заголовка заметки я отвечу однозначно отрицательно. И не буду призывать к поголовному переходу на Linux. Тем более, что ждать от него удвоения производительности не следует~--- последний раз такое случилось при переходе с XT'шки на AT'шку. Все дальнейшие upgrade-потуги были направлены (благодаря, к слову сказать, отцу Уындовсу иже с ним) только на сохранение \textit{status quo}. Однако задуматься, а есть ли у него мотивы для такого перехода~--- не повредит любому представителю творческой профессии. Ведь ныне~--- именно он самый медленный компонент компьютерной системы. И разве плохо повысить собственную производительность хотя бы на 5-10 процентов? А дальше уже вопрос скорее upgrade собственной личности, но никак не системы\dots

\section{Феномен Linux~--- в истории и перспективе} 
\begin{timeline}2002\end{timeline}

\textsl{Актуализирующее вступление: эта статья сочинялась для бумажной Компьютерры, где и была опубликована в начале 2002 г. (вероятно, и по сию пору доступна в онлайновой версии). А актуальной она показалась мне в связи с дискуссией об идеальном дистрибутиве на одном из форумов. Текст отредактирован лишь в стилистическом отношении~--- содержательная его часть осталась практически без изменений (исключая пару актуализирующих комментариев). А потому прошу учесть~--- писалось это скоро как три года назад, и кое в чем мои взгляды изменились.}\medskip

В этой заметке рассказ пойдет в особенности об операционной системе Linux~--- именно она символизирует явление, которое вошло в историю как феномен Open Source. Хотя Linux и не одинока на этой стезе, но только она породила множество прямых потомков в лице сотни дистрибутивов и даже послужила объектом для подражания (с точки зрения если не буквы, то~--- духа). В качестве примера можно привести ОС AtheOS. Да и реанимация проекта Hurd после многих лет вялотекущего существования, думается, во многом вызвана популярностью Linux.

В общем, то, что происходит в последние два-три года (\textsl{напоминаю~--- отсчет идет от конца 2001-го}), получило название Linux-бума. Однако ни один бум не может длиться вечно. Чувство восторженной новизны со временем ослабевает, события входят в более спокойное русло, и возникает вопрос: каким окажется будущее Linux? Попробуем представить его, обратившись к истории.

Начну с момента, когда, как мне кажется, ОС Linux впервые обратилась лицом к конечному пользователю. Волею случая, именно тогда началось и мое приобщение к Linux, поэтому я могу говорить о том, что в значительной мере происходило на моих глазах. За точку отсчета приму выход Linux Mandrake 5.1~--- первого произведения Гаэля Дюваля сотоварищи (номер версии тут смущать не должен~--- он соответствует таковому Red Hat, от которого Mandrake отпочковался). Утверждение это может показаться спорным. Ведь к тому времени большинство популярных дистрибутивов далеко отошло от сборки <<с помощью паяльника и осциллографа>>~--- А. Суханов), предоставив достаточно удобные средства установки и конфигурирования. Однако средства эти предназначались все же скорее сисадмину, чем рядовому пользователю.

Mandrake же изначально принял во внимание интересы именно конечного, в том числе и домашнего, пользователя. В него в качестве стандартной была включена первая и тогда единственная работоспособная интегрированная визуальная среда~--- KDE (поклонники Gnome меня, конечно же, не простят, но назвать его работоспособным тогда~--- как, впрочем, и сейчас,~--- можно было с большой натяжкой; а XFce~--- и до сих пор не назовешь по-настоящему интегрированной средой. Именно KDE позволила преодолеть психологический барьер множеству пользователей, не впитавших с молоком матери пристрастия к \texttt{sh} и \texttt{vi}.

Кроме того, Mandrake исходно был интернациональным по сути. Его разработчики пытались, хотя бы теоретически, охватить все изобилие языков нашей маленькой планеты. Последний фактор сыграл решающую роль в судьбе Mandrake на постсоветском пространстве: с локализации пакета (IPLabs Linux Team, ныне Altlinux) по-настоящему началась разработка отечественных дистрибутивов. Предтечами чего выступили и диски УрбанСофта, и Красная Шапочка, и украинские KSI и Black Cat. Однако в начале непрерывной традиции русскоязычия Linux~--- именно Linux Mandrake Russian Edition.

Следуя веяниям времени, по тропе, намеченной Mandrake, двинулись и прочие разработчики. Один за другим дистрибутивы приобретают красивые графические программы инсталляции, удобные средства настройки, расширяют круг поддерживаемых языков, обрастают мультимедийными приложениями. Однако~--- не все: ряд систем не просто сохраняет верность стилю True Linux, но и развивает его.

В итоге к настоящему моменту привычная классификация дистрибутивов на Red Hat-, Slackware-, Debian-клоны оказалась размытой. С одной стороны, появились кросс-дистрибутивные средства управления пакетами, формат которых~--- одна из основ традиции (такие, как исходно Debian'овский 
\texttt{apt}
, ныне успешно прикрученный ко многим rpm-based системам). А благодаря усилиям по стандартизации Linux будет, надеюсь, нивелирован и другой её столп~--- различие логической структуры файловых систем. С другой стороны, клоны основных систем (тот же Mandrake, или, паче того, Corel Linux) настолько отошли от своих прототипов, что об их происхождении подчас не напоминает даже формат пакетов (примером чему~--- Suse, прототипом которой некогда послужила Slackware).

И потому единственный критерий классификации дистрибутивов ныне~--- это их ориентация~--- на конечного ли пользователя или на системного администратора (речь идет о универсальных дистрибутивах. Специализированные системы с нестандартной, так сказать, ориентацией, выступающие под девизом <<Linux на одной дискете>>, здесь не рассматриваются). Под чем отнюдь не подразумевается их назначение~--- настольное (в первом случае) или серверное (во втором): и те и другие могут с успехом использоваться в обоих качествах. Просто в user-ориентированных системах не забыто, что администратор Web- или LAN-сервера~--- тоже пользователь, то есть человек, и ничто человеческое (в частности~--- и здоровая лень в отношении ручных настроек) ему не чуждо. Вторые же напоминают о том, что на настольной машине любой пользователь, как правило,~--- сам себе root и может испытывать желание сделать руками все.

Отсюда вытекают основные различия выделяемых групп. Типичные представители первой~--- современные версии Red Hat, Suse, Caldera OpenLinux, Mandrake, почти все вариации на тему Altlinux, ASPLinux и многие, многие другие. По умолчанию им свойственны: установка в графическом режиме (конечно, порядочные люди, как правило, предусматривают и возможность текстовой инсталляции в ручном режиме, однако это лишь дополнительная фича для нештатных ситуаций), автоматизация создания дисковых разделов и настройки основных параметров, выбор предопределенных групп пакетов~--- скажем, для разработчика, для сервера, для офисного применения. Апофигей этой группы~--- Corel Linux, в котором текстовой установки нет как класса, а многие настройки только автоматические.

Кроме того, user-ориентированные дистрибутивы постепенно обзаводятся едиными системами конфигурирования, функционально подобными 
\textbf{Панели управления}
 из Windows,~--- здесь <<впереди планеты всей>> Mandrake: его 
\textbf{Control Center}
 способен решить почти все пользовательские задачи в автоматическом режиме (включая выдвижение лотка CD-привода для вставки диска, содержащего требуемую библиотеку).

Вторая группа представлена в перую очередь Slackware и Debian, их клонами и, наконец, RockLinux, который можно поднять над ними в качестве штандарта (\textsl{со временем к ним добавились Gentoo, CRUX, Archlinux}). Здесь~--- все иначе: аскетичная установка в текстовом или, в лучшем случае, псевдографическом режиме, ручное создание разделов утилитами \texttt{fdisk} или \texttt{cfdisk} и ручной же выбор пакетов, локализация посредством текстового редактора (он же~--- и главный инструмент конфигурирования вообще). Короче говоря, перечисление отсутствующих возможностей способно вызвать дрожь у Windows-пользователя.

Тем не менее, популярность sysadmin-ориентированных дистрибутивов отнюдь не снижается. Почему? Ответить нетрудно. Именно они дают полную свободу в настройке системы под свои потребности. И пользователь Windows, начавший знакомство с Linux с продукции Corel (в надежде без напряга обрести мощь UNIX), очень быстро понимает, подобно цитируемому товарищем Бендером персонажу, что все в жизни имеет оборотную сторону.

Конечно, приятно, когда сразу после инсталляции можно общаться с системой на родном языке, печатать (в том числе и фотореалистичные изображения), писать CD (не вникая в параметры загрузки ядра) и слушать музыку (не собирая требуемых для того модулей). Однако <<раскошеливание на похороны>> в данном случае заключается в огромном количестве автоматически установленного софта (о назначении большинства представителей которого пользователь рискуете узнать никогда). Или, напротив, в отсутствии чего-то жизненно необходимого~--- так, в Mandrake при выборе офисной установки невозможно скомпилировать простейшую программу~--- никакие инструменты для этого стандартным набором не предусмотрены.

Конечно, все огрехи автоматизации в user-ориентированных дистрибутивах устранимы. Однако подчас это требует не меньших (если не больших) усилий, чем, скажем, установка русской локали, шрифтов и клавиатурных раскладок с нуля. А главное~--- пользователь, понадеявшийся на легкость установки и освоения современных дистрибутивов, оказывается морально не готов к такому повороту событий.

Итак, можно сказать, что если традиционные дистрибутивы пытаются возвысить юзера до сисадмина, то user-ориентированные~--- адаптируют систему до уровня понимания юзера. И расслоение Linux на две эти группы~--- реальность сегодняшнего дня.

Что же ждет нас завтра? Я, честно говоря, не верю в разлив основного Linux-русла по десктопам широких пользовательских масс~--- эта ОС займет место на винчестерах достаточно ограниченного круга юзеров. И~--- специфического, тех, кто в итоге приходит к выводу о необходимости разобраться в той или иной мере в устройстве системы. И потому крайние user-ориентированные возможности современных дистрибутивов останутся невостребованными~--- не в этом ли причина блистательного провала Corel Linux?

Тем не менее, возврата к эпохе паяльника и осциллографа уже не будет~--- к легкой жизни привыкаешь быстро. Так что следующий этап развития Linux видится мне в достижении разумного компромисса между автоматизацией и ручной работой. А вот какова будет форма этого компромисса~--- об этом <<давай хотя бы помечтаем>> (Тимур Шаов).

Рискну предположить, что большинство пользователей, независимо от квалификации, с удовольствием избавятся от необходимости собственноручно прикручивать принтер, сканер и прочее железо, передоверив эти функции программе установки. Не ожидаю активного протеста и против корректной автоматической локализации системы и приложений (включая спеллинг, шрифты и прочее).

А вот что следует оставить сугубо на усмотрение пользователя~--- это комплектацию системы. Тем более что штатные наборы пакетов отражают не только (а подчас и не столько) их взаимозависимости, сколько представления авторов дистрибутива о потребностях типичного разработчика или офисного труженика.

Возможно, это покажется беспринципным, но контроль зависимостей пакетов я, напротив, отдал бы на откуп системы~--- разбираться с этим большинству пользователей лень. Вернее, специальной интегрированной программы, позволяющей доустанавливать или удалять любые программные компоненты. К тому же мы знаем примеры более чем успешной реализации такого подхода~--- что в текстовом режиме (\texttt{pkgtool} в Slackware), что в графическом (тот же \textbf{Control Center} в Mandrake).

А еще лучше, если программа установки софта будет одновременно и общим конфигуратором системы. Желательно~--- идентичным по интерфейсу с программой установки самой ОС.

Если суммировать все сказанное и довести до логического завершения, мы получим. нечто вроде FreeBSD с её \texttt{sysinstall}, не так ли? Внешне, разумеется, а не с точки зрения внутреннего устройства ядра, файловой системы \textit{etc}.

Это я не к тому, что следует сменить Linux на *BSD (в этих системах у пользователя возникнут свои проблемы). Однако почему бы не присмотреться к тому, что делается в стане братьев во Open Source? Братьев по классу, по духу~--- ну и пусть, что не по лицензии.

\textsl{Post Scriptum сего дня: с радостью должен отметить, что, подобно большинству пророков, я также был посрамлен в своем пророчестве~--- о невозможности возврата к эпохе паяльника и осциллографа. Иначе чем объяснить столь широкую популярность Gentoo, не имеющего ни инсталлятора, ни конфигуратора? Или~--- течение Linux from Scratch, время от времени приобретающее характер эпидемии. И если последнее~--- явление сугубо возрастное (не в биологическом, понятно, смысле~--- на определенной стадии своего развития большинство пользователей Linux испытывает непреодолимую потребность собрать собственную систему), то Gentoo прочно занял свое место среди популярных дистрибутивов, потеснив и Debian, и Slackware. А соплеменные ему по духу CRUX и Archlinux обрели пусть и не столь широкую, но вполне сложившуюся пользовательскую базу.}

\section{Еще немного о посетителях и комментариях} 
\begin{timeline}20 марта 2007 г\end{timeline}

\hfill \begin{minipage}[h]{0.45\textwidth}
Скажите, если у нас все так хорошо, а у них все так плохо, то почему у нас все так плохо, а у них все так хорошо?
\begin{flushright}
\textit{Старый советский анекдот}
\end{flushright}
\bigskip\end{minipage}

Прочитал я соображения Владимира Попова в заметке <<Поговорим?>>~--- об особенностях общения on-line, и захотелось мне дополнить их рассмотрением одного частного случая Сетевых комментаторов. 

Время от времени на ресурсах, ориентированных на Unix, Linux и Open Source, появляются пользователи очень интересной категории~--- восхвалители Windows. Которые в обществе линуксоидов начинают доказывать, какой Linux плохой и какая Windows хорошая. Когда же им, сначала вежливо, пытаются объяснить, что они нашли для своих выступлений неподходящее место, очень возмущаются и начинают говорить чуть ли не о нарушении прав человека. И уж во всяком случае не преминут обвинить сообщество в необъективности. Дескать, хвалить <<линух>> и ругать <<венду>> тут можно, а вот критиковать первый и рассказывать о преимуществах второго~--- нельзя.

Что на это ответишь? Да, так оно и есть. И рассказывать на Linux-форумах о достоинствах Windows и недостатках Linux\dots ну не то чтобы нельзя, а просто бессмысленно. Большинство посетителей таких ресурсов давно знают все преимущества Linux и все его недостатки. И столь же давно научились эффективно использовать первые и бороться со вторыми. Или эти самые недостатки их просто не волнуют. Как не расстраиваюсь я, например, из-за того, что моя кофемолка не имеет встроенного аудио-плейера. А несравненные достоинства Windows интересуют действующих линуксоидов ничуть не больше, чем жителя средней полосы России~--- погода в Новой Зеландии.

Ситуация складывается примерно такая же, как если бы в клуб собаководов пришел заядлый кошатник и начал бы читать лекцию об особенностях разведения персидких кошек. Хотя нет, сравнение не точное: большинство собаководов просто любит животных вообще, и потому, вполне вероятно, с интересом послушают и рассказ о кошках. Как, впрочем, и наоборот\dots

Вот более точная аналогия. Представьте себе компанию любителей мяса, разговаривающих о том, как лучше есть сырую лосятину~--- слегка присолив её, или еще и диким лучком присыпав? И вот в ней появляется убежденный вегетарианец, и начинает рассказывать о несравненном вкусе морковных котлет, их целительной пользе для здоровья, делиться тонкостями рецептов приготовления оных. А заодно и объяснять своим собеседникам, что мясо вредно, да и есть его вообще невозможно.

Ясно, что ничего хорошего из этого не выйдет: вегетарианца нашего в лучшем случае вежливо проигнорируют (в худшем~--- могут и послать на известный ресурс, некогда основанный Леонидом Кагановым). И самому ему придется слушать о таких вещах, которые, возможно, вызывают у него физическое отвращение.

Именно в таком положении оказывается любитель Windows на любом Linux-форуме. Собеседники ведут разговор о непонятных и неприятных для него вещах~--- каких-то консолях, шеллах, командах. А он начинает парить им мозги, какие замечательные эффекты есть в Windows Vista и как быстро запускается MS Word. А заодно и доказывать, что консоль и вообще Linux~--- это все плохо, работать там неозможно, и вообще~--- <<туфта этот ваш линух>>.

Причем что интересно: среди моих реальных и виртуальных знакомых~--- действующих линуксоидов, нет ни одного, кто ходил бы по Windows-форумам даже только с целью восхваления своей операционки, не говоря уже о том, чтобы тратить время на критику Windows. А вот обратная картина наблюдается сплошь и рядом. Поневоле приходит на память вопрос, заданный Рабиновичем после лекции о преимуществах социализма над капитализмом. Тот самый, который вынесен в качестве эпиграфа. Не есть ли это свидетельство комплекса неполноценности определенной части Windows-пользователей, развившегося вследствие употребления этой замечательной операционной системы? И подобного тоске вегетарианца по настоящей пище, вкармливавшей Homo Sapiens на протяжении нескольких десятков тысячелетий. Ведь самый простой способ эту тоску заглушить~--- убедить других (да в первую очередь самого себя) в том, что мясо вредно\dots

\section{Какой Linux~--- худший?} 
\begin{timeline}Сентябрь 5, 2009\end{timeline}

\textsl{В соавторстве с Владимиром Поповым, публикуется с его разрешения.}

\hfill \begin{minipage}[h]{0.45\textwidth}
Мещанин понимает~--- пустота не полезна \\
Еда не в прок, и свербит тоска \\
Тогда мещанин подползает к поэзии\\ 
Из чужого огня каштаны таскать
\begin{flushright}
\textit{Михаил Анчаров}
\end{flushright}
\bigskip\end{minipage}

В дискуссиях по вопросу, какой же из дистрибутивов Linux является лучшим, в печати и на форумах сломано больше копий, чем на всех рыцарских турнирах за всю историю Средневековья. А если подойти к проблеме с другой стороны и определить, какой же из Linux'ов~--- худший, и сколько их, таких? Тогда те дистрибутивы, которые не попадут в число худших, автоматически окажутся лучшими, не так ли? Давайте посмотрим.

Поводом для сочинения настоящей заметки послужил материал под таким вот названием: <<2009's 10 Worst Linux Distributions>>. На содержании её задерживаться не будем, как и на рассмотрении прямых фактических ошибок, которых там вдоволь. Важен общий тон, который нас и спровоцировал на данное сочинение.

Автор абсолютно искренне верит, что дистрибутивы Linux создаются с единственной целью: порадовать пользователей IBM PC. Можно сказать больше: основная их масса сегодня уверена, что средства вычислительной техники и развиваются-то с единственной целью: максимального удовлетворения индивидуального\dots нет, даже не пользователя~--- это слово кажется в данном контексте неуместным~--- а потребителя. Ибо пользователь чем-то пользуется, что-то использует или, более того, что-то или кого-то пользует. Как говаривали в старое время, 
\textit{нашу семью пользует доктор имярек}
.

Потребитель же только потребляет. И венцом развития информационных технологий, по его мнению, является электронно-вычислительно-мультимедийно-кофе-в-постель-подавательный комбайн с одной кнопкой, большой и красной: зделайте мну пес\dots хорошо, говоря иначе.

Казалось бы, логично, ага? Всё для человека, всё для блага человека, всё для счастья человека. Не за то ли боролись поколения наших предков, начиная с безымённого кроманьонца плейстоценовой тундростепи и кончая героями информационной революции?

Однако если довести это рассуждение до логического предела, то получится, что основное назначение:

\begin{itemize}
	\item пороха~--- пиротехника,
	\item высокомолекулярного синтеза~--- полиэтиленовые кульки, 
	\item электротехники~--- лампочка в сортире, а 
	\item радиохимии~--- <<светящиеся>> стрелки наручных часов. 
\end{itemize}

Можно согласиться, что всё, что ни делается в мире, делается для <<всё более полного удовлетворения>>. Но, для удовлетворения~--- кого? Не стоит ли при этом отметить, что для некоторых представителей вида (как минимум от Пифагора до Эйнштейна) удовлетворение было накрепко связано с познанием и творчеством?\dots

Впрочем, где-то на уровне подсознания, потребитель догадывается, что кое-что делается и не для него вовсе. И не так, как ему представляется естественным. Он не пытается советовать машиностроителям делать корпуса из бумаги только на том основании, что он попробовал: удобно резать ножницами. Не требует понижения напряжения в ЛЭП до 220 вольт только потому, что ему так привычнее. И не решает: какой тип двигателя должна использовать баллистическая ракета только потому, что один из этих типов (внутреннего сгорания) даже ему известен.

Интересно, что при этом тот же потребитель охотно судит о том, какими должны быть ОС. Наверное, наивно полагая, что разговор затеян исключительно для того, чтобы, в конечном счёте, сделать его эпизодическое общение с компьютером максимально приятным и, по возможности, бездумным. По аналогии с автомобилями это выглядит как требование перехода к автоматической коробке передач абсолютно для всего, что движется на колёсах (один из сочинителей попытался представить себе ГТТ с автоматической коробкой~--- и аж заколдобился).

Однако, помилуйте! Мы вовсе не против комфортности современных седанов, но оставьте в покое карт. И локомотив. И седельный тягач, автобус, болид F-1 и всех прочих представителей семейства <<самоедущих>>. Поверьте: даже ваше личное удовлетворение зависит отнюдь не только от мягкости сидений в вашем авто. Но и от эффективности грузоперевозок, общего технического прогресса и количества энтузиастов, желающих в нём участвовать.

Так и с вычислительной техникой. Не всё, что обсуждается, имеет конечной целью удовлетворение индивидуального пользователя IBM PC. Более того: как раз те, кто делает товар, ориентированный на конечного покупателя, не особенно-то к <<базарному>> общению и склонны: приоритеты другие.

Так и с дистрибутивами Linux. Одни из них~--- просто полигоны или <<тренировочные залы>> для обкатки технологий, которые потом будут использованы в промышленных системах (Fedora, OpenSuse, если выйти за пределы собственно Linux'а~--- OpenSolaris). Другие~--- оригинальные разработки, ориентированные на решение определённых задач, пусть даже и очень широкого круга (RHEL, Suse, Debian и бессчётное количество более иных). Третьи~--- узкоспециализированные системы или конструкторские наборы для их построения. Четвёртые~--- <<начальная школа>> для потенциальных разработчиков или платформы для разработчиков <<кинетических>>.

Разумеется, все они могут применяться (и успешно применяются) в качестве обычных настольных станций. В том числе и пользователями, профессионально с вычислительной техникой не связанными. Но ни одна из этих систем не рассчитана на бездумное потребление: не предусмотрели в них разработчики той самой сакраментальной красной кнопки. Так что с точки зрения потребителя все эти дистрибутивы <<плохи>>. Разница между ними лишь в том, что одни свою <<плохость>> маскируют красивыми инсталляторами, другие же (подобно Gentoo или Slackware) сразу лишают иллюзий относительно своих потребительских качеств.

Лишь единичные дистрибутивы, подобно Ubuntu или Mandriva, ориентированы, казалось бы, приблизительно на тот же круг пользователей, с опорой на который начиналось победное шествие MS Windows. То есть: на использование непрофессиональным потребителем на персональном (не в архитектурном, а в <<правовом>> смысле) компьютере.

Да и в отношении пресловутой Ubuntu надо подчеркнуть, что она создавалась для применения лишь \textbf{в том числе и} на персональном компьютере. Потому что, конечно же, Шаттлворт, начиная этот проект, рассчитывал (и рассчитывает, как можно прочитать между строк его многочисленных заявлений) на выход в сферы всамделишних применений~--- промышленных серверов, мобильных систем \textit{etc}.

Только действует он методами, противоположными, скажем, политике комании Red Hat. Которая отказалась от развития пользовательской линии своего дистрибутива вообще, перевалив это дело на плечи энтузиастов из сообщества. И сосредоточила усилия на внедрение в корпоративный сектор.

Нет, экспансия Ubuntu идёт тем путём, каким шла Microsoft, продвигая Windows~95 как игровую и медиа-платформу бытового назначения. Ведь мало кто помнит нынче, что до 1995 года Windows никем всеръёз не воспринималась~--- ну ещё одна графическая псевдомногозадачная оболочка над DOS, не более.

Но после 1995 года эта оболочка\dots А Windows~95, что бы ни говорили, оставалась оболочкой, что сразу по её выходе блестяще показал Эндрю Шульман в своей книге Неофициальная Windows'95\dots Так вот, эта оболочка пришла в дома. И, как пророчески заметил тогда же Георгий Кузнецов (в то время главный редактор Компьютерры) её неизбежно должны были принести на работу. Что, собственно, и произошло~--- тем более, что промышленная платформа в виде Windows~NT была заготовлена загодя. Только о том, что эта Windows~--- внутри вовсе не та Windows, предпочитали не говорить вслух.

\textsl{Отступление: можно припомнить, что в 1997-1998 года, скажем, хостинг на Windows-серверах предлагал чуть ли не один-единственный провайдер на всю Москву. Это была отдельная услуга, которая стоила вдвое дороже, чем стандартный хостинг на UNIX-машинах. Видимо, с намёком на то, что администрировать сможет любая блондинка. А сколько нынче Web-серверов крутится под разными версиями Windows?}\medskip

Примерно таков же метод действия Canonical: сначала~--- бесплатная рассылка максимально <<юзерофильного>> дистрибутива с целью создания пользовательской базы. И параллельно~--- фоновая разработка специализированных решений: для промышленных серверов, мобильных систем \textit{etc}. Правда, в отличие от случая с Windows, когда в потребительскую и промышленную сферы продвигались две принципиально разные ОС, тут в любом случае речь идёт о Linux, внутренне одной и той же системой, различающейся, в зависимости от назначения, лишь <<обвязкой>>.

Так что и потребителям Ubuntu не следует обольщаться~--- этот дистрибутив для них столь же <<плох>>, как и все остальные.

В итоге сказанного мы приходим к выводу, что с потребительской точки зрения все дистрибутивы Linux одинаково <<плохи>>, будь они столь же разукрашеными, как раскрашенный Штирлиц из <<цветной>> версии <<Семнадцати мгновений весны>>.

Рассужадая же обратным порядком, как кадий Абдуррахман из <<Повести о Ходже Насреддине>>, мы приходим к выводу, что все дистрибутивы Linux одинаково хороши~--- для пользователя профессионального, знающего, что ему нужно от системы. И пользователю этому совсем не обязательно быть профессионалом именно в IT-сфере: достаточно просто не рассчитывать на большую красную кнопку, а создавать её своими руками, применительно к собственным целям и задачам.

\section{Трибализм в мире FOSS} 
\begin{timeline}Август 8, 2010\end{timeline}

\hfill \begin{minipage}[h]{0.45\textwidth}
Был митинг на заводе Лихачёва \\
И на заводе Красный пролетарий:\\
Убил Лумумбу враг народа Чомбе,\\
Ему помог Мобуту с гнусной харей.
\begin{flushright}
\textit{Народное}
\end{flushright}
\bigskip\end{minipage}

Недавно опубликованная заметка Марка Шаттлворта о трибализме в сообществе Open Source~--- <<Tribalism is the enemy within>>~---  довольно активно обсуждается как в мировом масштабе, так и в Рунете. Последнее, в конце концов, вызвало к жизни заметку Владимира Попова <<Трибализм ``от Марка''>>, которая и послужила поводом для заметки настоящей.

Но начнём по порядку~--- с эпиграфа.
 Именно во времена сочинения этой песенки в русском языке и появилось слово трибализм~--- как обозначение идеологии африканских вождей, по мере освобождения Африки от гнёта проклятых колонизаторов постепенно превращавшихся в президентов, а то и императоров.

В более широком смысле трибализм свойственен племенам и народам, долгое время существовавшим в условиях полной изоляции, например, эскимосам и индейцам атапасской группы. И подчас не подозревавших о том, что существуют ещё и другие люди. во всяком случае, у эскимосов Туле (самый север Гренландии) первая встреча с европейцами в начале 19-го века вызвала шок: до тех пор они полагали, что кроме них на свете существуют только белые медведи, моржи, тюлени и прочая живность.

Не случайно общее самоназвание и эскимосов, и атапасков было очень простое~--- люди. Когда же они столкнулись с представителями других племён и народов, они несколько конкретизировали своё имя, назвавшись людьми настоящими.

Ничего не напоминает? Мне~--- так даже очень. А именно~--- сообщество Ubuntu. Для изрядного числа новообращённых убунтоидов
\large\begin{shadequote}{}
Ubuntu == Linux
\end{shadequote}\normalsize

И, подобно эскимосам и атапаскам, они не подозревают о существовании других дистрибутивов Linux (о прочих UNIX-подобных системах FOSS-мира и говорить излишне). Чем и демонстрируют пример племенного мышления (именно так в русскоязычных новостях перевели термин трибализм). Но что понятно и простительно для племён, заброшенных на край ледяной пустыни, в места, где человек выживает на пределе своих возможностей (а подчас~--- и за пределами их), выглядит довольно странно в наш благополучный век тотальной информатизации. Так что Марку следовало бы начинать борьбу с трибализмом с сообщества, сформировавшегося вокруг его собственного дистрибутива.

Причём именно в Ubuntu и его ближайших сородичах трибализм наиболее активно воплощается в жизнь практически~--- путём создания <<не такого, как у всех>>, проявляющегося даже в мелочах. Достаточно вспомнить локализации KDE в Kubuntu, в которых на протяжении долгого времени хладнокровно игнорировались работы команды интернационализации собственно проекта KDE. И таких примеров можно привести ещё много.

Нельзя сказать, что сообщества пользователей более иных дистрибутивов совсем свободны от трибализма. Но обычно он менее бросается в глаза по ряду причин. в частности, и потому, что большинство этих самых более иных пользователей перепробовали более одной системы, имеют представление о мире FOSS в целом, и их нынешний выбор~--- вполне осознан.

Среди пользователей Ubuntu таких тоже немало. Но количественно они просто теряются на фоне новообращённых трибалистов\dots

\section{Унификация интерфейсов: во зло или во благо?} 
\begin{timeline}Октябрь 17, 2011\end{timeline}

Разговоры об унификации пользовательских интерфейсов ведутся очень давно~--- с тех стародавних времён, когда уже забылось, что Венечка Ерофеев был бригадиром на кабельных работах. Но вот вопрос о том, зло это или благо, унификация?~--- уже и не поднимается. Казалось бы, это останется покрыто таким же мраком неизвестности, как то, во зло или во благо пил Венечка между улицей Чехова и неведомым подъездом, в котором он проснулся поутру:

\begin{shadequote}{}
Никто этого не знает, и никогда теперь не узнает. Не знаем же мы вот до сих пор: царь Борис убил царевича Димитрия или же наоборот?
\end{shadequote}
Однако буквально сегодня коллега Hymnazix aka Сергей Голубев неожиданно поднял эту тему в своём блоге в статье: <<Пользовательские интерфейсы: постановка задачи.>> Прочитав её, я вдруг обнаружил, что давненько не брал в руки шашечку\dots то есть не писал пасвилей и памфлетов. И ощутил непреодолимое желание возобновить это занятие. Что и делаю.

Прошу понять правильно: это памфлет не на статью Сергея, с которым я связан давними товарищескими отношениями~--- она послужила только поводом. А на\dots впрочем, проницательный и терпеливый читатель, асиливший мою заметку до конца, сам поймёт, на кого и на что.

Начну я, однако, с цитатирования заключительного абзаца статьи Сергея, так сказать, её квинтэссенции:
\begin{shadequote}{}
С точки зрения пользователя интерфейс должен быть унифицирован\dots если я научился выполнять какую-то операцию в одной системе, то с другими у меня никаких проблем не возникнет.
\end{shadequote}

Точнее, автор подразумевает, что не должно возникнуть. Из чего следует, что унификация интерфейсов понимается им как абсолютное благо. Давайте же посмотрим, чем это обосновывается.

А обосновывается это аналогией с автомобилестроением~--- приём не новый, но от этого не перестающий быть менее жизненным. Так что следующая цитата:
\begin{shadequote}{}
\dotsводитель, один раз освоив управление машиной, запросто меняет модели, не задумываясь о том, где на новой руль, а где педаль газа. Он твердо знает~--- точно там, где им надлежит быть.
\end{shadequote}

Отлично.  А теперь представим, что ему предстоит ездить по горно-тундровой местности~--- ну жизнь так вот сложилась. И потому новая его модель~--- не Лексус или Порш следующего в натуральном ряду чисел номера, а Гусеничный Транспортёр Санитарно-Медицинский ГАЗ-71 (это его паспортное название, в просторечии~--- ГТСка).

Для начала наш водитель  с удивлением обнаружит вместо руля рычаги, а тормоза не найдёт вовсе. А если он сядет не на ГТСку, а на ГГТ, удивление его возрастёт ещё больше -– он не найдёт там рычаг переключения передач, и будет довольно долго смекать, как ему таки включить пониженную передачу.

Столь же верно и обратное: при некотором опыте вождения гусеничного транспорта, усевшись на колёса, остаёшься в недоумении~----куда девались  рычаги, которыми интуитивно понятно, как поворачивать, а толкнув которые от себя одновременно оба два, можно столь же прозрачно для пользователя переключить передачу. И почему эта падла не останавливается на месте, когда я снял ногу с педали газа?

Можно ли унифицировать <<интерфейс>> гусеничных вездеходов и городских легковушек? Вероятно, если поднапрячься, то можно (и такие попытки даже были на практике~--- замена рычагов на штурвал в некоторых гусеничных машинах). Только вот будет ли кому от этого лучше? Ведь гусеничные и колёсные машины приспособлены для решения совершенно разных задач, и ожидать от них одинакового удобства унифицированного <<интерфейса>> было бы нелепо.

В своё время, без малого 20 назад, Джим Сеймур, колумнист журнала PC Magazine, высказался относительно унифицированных интерфейсов примерно так (по памяти, бумажый журнал давно потерян, а в сети номеров тех лет нет и уже, наверное, никогда не будет):
\begin{shadequote}{}
\dotsу моей магнитолы и моего телевизора разные кнопки управления, и это ничуть не портит мне жизнь.
\end{shadequote}

Сказано это было в годы появления первых интегрированных офисных пакетов, каковые начались с унификации интерфейсов электронных таблиц и текстовых процессоров, в них включаемых. Некоторые читатели, возможно, помнят, что до того времени в качестве стандартного интерфейса электронных таблиц рассматривался стиль Lotus 1-2-3~--- Lotus и Borland даже долго судились по этому поводу (пока обе фирмы не накрылись медным тазом~--- в том числе и вследствие собственного сутяжничества). Самый же распространённый текстовый процессор тех лет, WordPerfect, имел интерфейс совершенно отличный. И тем не менее, оба они прекрасно справлялись со своими задачами в руках одних и тех же зачастую пользователей. Которым различие структуры меню и <<горячих>> клавиш мешало ничуть не больше, нежели Джиму Сеймуру~--- разные кнопки/ручки на магнитоле и на телевизоре.

Что же получилось после того, как в рамках MS Office  интерфейс электронных таблиц и текстовых процессоров привели к общему знаменателю? А получилось, как всегда в таких случаях: стало очень неудобно работать и с теми, и с другими, а вслед за ними и с прочими <<унифицированными>> программами. Нынешнее поколение этого уже не осознаёт, потому что, как известно, <<стерпится~--- слюбится>>. Но оценить величину потери могут только те, кто видел бывших секретарей-машинисток, делопроизводительниц и референтш, виртуозно порхающим пальчиками по горячим клавишам Лексикона или WordPerfect'а. Или тот, кому самому довелось обрабатывать огромные электронные таблицы полностью на рефлекторном уровне, нимало не пудря себе мозги теориями интерфейсов.

И это касается интерфейса программ если не родственных, то и не очень далёких~--- условно, офисного назначения (хотя изначально электронные таблицы были скорее орудием инженеров, нежели клерков). Что же доброго может получиться от унификации интерфейса систем твердотельного моделирования и медиапроигрывателей? Не ГТТ ли, управляемый ручками от магнитолы?

Так что единственное добро можно сформулировать опять таки словами Сергея:
\begin{shadequote}{}
Цель унификации~--- минимизация нагрузки на память.
\end{shadequote}
Вот только действительно ли это добро? Как известно, любой орган человеческого, не упражняемый должным образом, атрофируется. А мозг человеческий, управляющий, в том числе, и памятью, этому подвержен чуть ли не в наибольшей степени. И напротив, банальное школьное упражнение по заучиванию стихов наизусть весьма тренировке памяти способствует. Так что всеобщая унификация интерфейсов, помимо чисто технических недостатков, приведёт к дальнейшей деградации человечества.

В этой связи стоит опять обратиться к статье Сергея~--- теперь к самому её началу, в связи с чем~--- очередная цитата:
\begin{shadequote}{}
На пресс-конференции РАСПО <<Создание Национальной программной платформы~--- важный шаг на пути построения информационного общества>> был задан вопрос о пользовательских интерфейсах\dots

По реакции участников пресс-конференции я понял, что представители большого СПО-бизнеса не склонны считать это детским садом, не заслуживающего их внимания.
\end{shadequote}
Как можно понять из всего опубликованного по этому вопросу, НПП предназначена в существенной мере (если не главным образом) для служащих государственных структур. И если при её разработке немалое внимание уделяется <<унификации интерфейсов>>~--- не есть ли это молчаливое признание того, что большинство наших госчиновников не в состоянии запомнить разницу между ручками магнитолы и телевизора? Тех же, кто это способен асилить, не надо ли срочно унифицировать до общего уровня? Путем принудительного внедрения унифицированных интерфейсов, минимизирующих нагрузку на память и тем самым способствующих её атрофии\dots

В своём памфлете я затронул лишь одну из граней проблемы интерфейсов~--- на самом деле их много. Тут и действительно универсальные интерфейсы, хотя и в своей сфере применения~--- это CLI, интерфейс командной строки. И преемственность интерфейсов в рамках одной линии программ~--- точнее, как мы видим в последние годы, ничем не обоснованный отказ от такой преемственности, и многое другое. Но нельзя объять необъятное\dots

\section{О гномах, окнах и МММ, или кредит доверия} 
\begin{timeline}Июнь 7, 2012\end{timeline}

В этой заметке не будет никаких сравнений, и поэтому отнести её к сериалу про сравнение мужей я не могу. Тем не менее она носит чисто этнографический характер. А поскольку дальнейшее развитие событий даст для заметок этой направленности ещё много поводов, видимо, придётся учреждать соответствующую рубрику.

Поводом для этой заметки послужило очередное высказывание Линуса в адрес GNOME~3.  Которое заставило обратиться к истории\dots нет, сначала не GNOME, а совсем другой среды. Той, что когда-то звали просто оболочкой DOS (злые языки добавляли~--- дешёвой). И которая потом стала Операционной Системой (причём весьма дорогой).

Как вы, конечно, догадались, речь идёт о Windows. Появившись в 1-й своей ипостаси в 1985 году как (одна из многих) графических оболочек для DOS, она не умела ничего, кроме как запускать себя и свои приложения, которых тогда не было. Версия 2-я научилась, хотя и не сразу, запускать DOS-программы в квази-многозадачном режиме. Ну а начиная с версии 3.1, она называлась гордым именем Операционной Среды, в которой, несколько приноровившись, можно было работать.

Windows~95 получила уже титул Операционной Системы, хотя по сути своей оставалась всё той же надстройкой над DOS. Как и сменившая её Windows~98. В последнем легко убедиться покупателям ноутбуков, в прайсах стыдливо помеченных <<без ОС (DOS)>>. Далеко не всегда за аббревиатурой в скобках скрывается всем известная FreeDOS: часто это та самая MS DOS 7, которой в отдельном виде никто не наблюдал.

Однако я отвлёкся. Опять же, Windows~95 имела массу недостатков, которые честно признавались производителем (по крайней мере, некоторые из них). Но столь же честно обязалось, что уж в следующей версии они обязательно будут исправлены.

И некоторые действительно исправлялись в Windows~98 и, особенно, её сервис-паках, и по выходе SP2 эта Windows (почему именно эта~--- скоро скажу) стала совсем уже настоящей. Что было неправильно~--- и для коррекции тут же появилась Windows~ME, применение которой наталкивалось на ряд трудностей. Но зато она оказалась последней в своей линейке.

Потому что параллельно развивалась другая Windows, с самого начала представлявшая собой настоящую операционную систему: Windows~NT. Только вот для простого народа она не предназначалась, позиционируясь как серверная, на крайняк~--- как система для рабочих станций.

Но в конце концов обе линии Windows слились воедино~--- в лице сначала Windows~2000, а потом и Windows~XP. Это были последние винды, которые я видел хотя бы краем глаза, и обе были пригодны к использованию. Впрочем, это мнение не только моё, но и многих лично мне знакомых квалифицированных пользователей Windows.

Но потом колесо Фортуны свершило очередной оборот~--- и появилась Windows Vista, о которой те же мои знакомые отзывались не иначе как с использованием обсценной лексики. Видимо, такими слова о ней говорили не только мои личные знакомые, и звучала эта лексика не только на русском языке. Потому что не успело отгреметь горное эхо, поминающее Vista'ему мать, как был выпущен <<психокорректирующий>> релиз в виде Windows~7. Каковой, по словам всё тех же моих знакомых, был лучшей виндой из всех винд в истории виндовства.

Но, как говорится, <<недолго музыка играла, недолго фраер танцевал>>. Потому что грядёт Windows~8, в которой всё опять будет перевёрнуто с ног на голову. Но об этом пусть болит то, на что переворачивают, у тех, кто этим будет пользоваться.

В рамках же этой истории сконцентрирую внимание на двух моментах, которые имеют отношение к этнографии. Первый~--- кредит доверия пользователей. Десять лет, десять дней, десять часов \textit{etc}\dots (с 1985 по 1995). Чёрный Владыка Саурон в Чёрных Горах Мордовии ковал свою супер-мега-ОСь, в которую обещал засадить все свои фичи и все свои баги (ну бэкдоры там и прочие излишества всякие нехорошие). И ему верили~--- потому что каждая фича была обещана ну непременно в следующей версии. Возможно, верили по тому, что по части багов и прочих излишеств он свои обещания выполнял сполна. А подчас и с лихвой.

А второй момент~--- это то, что производители Windows рискнули пускаться на эксперименты типа Vista только тогда, когда окончательно поверили в неограниченность кредита доверия. И теперь, вероятно, с интересом наблюдают~--- схавает ли пипл их <<восьмятое>> порождение? Ничем, кстати, особо не рискуя: схавает~--- хорошо, не схавает~--- <<ах прости, дескать, вышла ошибка>>, ибо старый бронепоезд на запасном пути пары ещё не сбросил.

А теперь вернёмся к истории GNOME. Вряд ли многие помнят, но до-первый GNOME был изделием жутко красивым~--- с Enlightment'ом в качестве оконного менеджера, раздвигающимся при старте занавесом и прочей фанаберией. Недостаток у него был единственный: он достойно умел делать только одно дело. А именно, свопировать. На машинах, где KDE до-первое и первое просто летало, при запуске GNOME индикатор активности винчестера не гас ни на минуту.

Конечно, это компенсировалось всё тем же доверием пользователей~--- ведь GNOME, в отличие от KDE, был истинно свободной рабочей средой (о том, что Qt стало свободной задолго до обретения GNOME хоть какой-нибудь юзабельности, все быстро забыли).

На кредите доверия к свободе GNOME продержался до своей второй версии, в которой в его идеологии произошёл первый крутой перелом: если ранее эта среда позиционировалась как самая совершенная и ни на что не похожая, то теперь её стали выдавать за <<лучшую Windows чем сама Windows>>.

Этого лозунга хватило ненадолго~--- видимо, вследствие одиозности не к ночи помянутого имени в некоторых кругах. И где-то к середине жизненного цикла 2-й ветки (хронологически это примерно 2005 год) происходит метаморфоза: GNOME объявляется средой, самой простой в настройке и использовании. Что, в целом, соответствовало действительности. Правда, простота настройки достигалась тем, что многие из конфигурируемых параметров просто не конфигурировались собственными средствами, а лишь через реестр GNOME Editor.

Интересно, что смена философической парадигмы GNOME совпала по времени и с его широким распространением. Вызванным, правда, не несравненными достоинствами среды, а усилиями фирмы Canonical, распространявшей Ubuntu (с GNOME по умолчанию) бесплатно и <<на каждом километре, здесь и по всему свету>>. А также тем, что Fedora перестала полагать себя <<песочницей Red Hat'а>> и обратилась лицом к пользователю. Возможно, что не последнюю рль сыграл и переход Suse под крыло Novell, незадолго до того прикупившей Mono. Ну и неудачный эксперимент с KDE~4.0, когда очень ранняя тестовая версия получила имя релиза, также сыграл свою роль.

Так или иначе, стечение обстоятельств, не без доброй воли разработчиков, привело к тому, что в конце нулевых годов среда GNOME, во-первых, стала вполне комфортной для работы, и, во-вторых, приобрела сравнимое с KDE обеих веток распространение. Апофеозом чему стала Fedora~14 с GNOME по умолчанию~--- вылизанным и блестяще интегрированным со всеми дистрибутив-специфичными компонентами.

Однако тут наступает время очередного финта ушами~--- или, если угодно, эксперимента над пользователями: GNOME~2 объявляется ретроградным отстоем обскурантов, и на смену ему приходит ультра-революционная, гипер-модернистская и мега-прогрессивная среда GNOME~3, унаследовавшая от предшественницы только название. Ах да, ещё и тенденцию: она была практически лишена собственных средств самого элементарного конфигурирования. Каковые были даже не обещаны в будущем~--- вполне откровенно предлагалось делать их самим.

На такой эксперимент в отношении пользовательского доверия не решалась даже Microsoft\dots Вровень я могу поставить только эпопею с МММ. Впрочем, гражданин Мавроди подошёл к делу более осмотрительно. И выпустил в свет новую свою <<систему>> только тогда, когда значительная часть пользователей первой версии вымерла физически (в том числе и вследствие её <<пользования>>), а потенциальные пользователи версии второй (в те далёкие уже времена ходившие пешком под стол) достигли дееспособного возраста.

Разработчики GNOME~3 уроков Мавроди не учли. Интересно, как это отразится на будущем этой среды? Если никак~--- можно считать, что кредит пользовательского доверия воистину безграничен, и лепить горбатого, почём зря. И не это ли сокровенный смысл проведённого эксперимента~--- как репетиции перед более масштабными экспериментами? Но это~--- тема совсем другого памфлета.

\section{Размышлизм о десктопном Linux’е} 
\begin{timeline}Сентябрь 1, 2013\end{timeline}

Вся история развития десктопного Linux'а, по крайней мере в лице его наиболее известных и распространённых дистрибутивов, несколько напоминает онанизм в библейском (а не вульгарном) понимании этого термина.

Как известно, библейский Онан, вынужденный, согласно обычаю левирата, жениться на вдове своего старшего брата, при общении с ней практиковал один из простейших способов контрацепции~--- прерванный половой акт. Но ведь чем-то подобным занимаются и майнтайнеры ряда дистрибутивов Linux'а, претендундующих на десктопность.

Несколько раз за время своей истории Linux вплотную подходил к той грани оргазма, после которой следует завершить процесс по крайней мере зачатием законченных решений~--- пользовательских и специализированных. И каждый раз процесс прерывался, что обосновывалось неготовностью. И необходимостью перекроить всё~--- на этот раз правильно и окончательно.

Последний по времени пример тому мы видели при всеобщем внедрении \texttt{systemd}'а. Но впереди и пример грядущий~--- Wayland'изация, которая тоже угрожает охватить все дистрибутивы. Или почти все. Потому что есть пример и завершения акта~--- дистрибутивы семейства Ubuntu и возникшие на их базе многочисленные клоны, как общего назначения, так и специализированные для разных сфер применения.

Правда, разработчики тех дистрибутивов, которые гордо именуют себя майнстримом, относят Ubuntu'иодов, вместе со Slackware'щиками и прочими Gentoo'шниками, к категории маргиналов. Однако это именно те маргиналы, суммарное количество которых давно уже превзошло число майнстримщиков~--- но это немного другая тема.

Побуждения библейского Онана понятны: родившийся от его брака с Фамарью сын считался бы сыном его старшего брата Ира~--- со всеми вытекающими из этого правами, в том числе и наследования имущества в обход детей брата младшего, то есть самого Онана. И потому манкирование Онаном своими супружескими обязанностями вполне оправдано в глазах современного здравомыслящего человека. Но не в глазах иудейского бога\dots

Сложнее понять побуждения сторонников гипермодернизма~--- ведь конспирологические версии, что это давно проплачено ребятами из Редмонта, которые, в свою очередь, продались маленьким зелёным человечкам с Марса или сгинувшего Фаэтона, мы отвергаем с негодованием, не так ли? Так что остаётся только допустить панический страх создателей десктопного Linux'а перед его реальной десктопизацией~--- но на эту тему я вроде бы уже писал\dots

Из Библии мы знаем, иудейский бог покарал Онана за его кощунство смертью. Мера наказания гипермоденистов от Linux'а будет куда банальней, и не потребует никакого Бога из Машины. Просто ни один из разрабатываемых ими десктопных Linux'ов никогда не станет по настоящему десктопным\dots

Апологеты \texttt{systemd} и Wayland, пытаясь низвести до маргинального уровня всех, кто не принял генеральную линию их партии, обрекают сами себя на положение маргиналов в глобальном масштабе. И если затея их удалась бы~--- то в число маргиналов попал бы весь десктопный Linux, причём уже навсегда. Хвала Ахурамазде, пока ещё число Linux-маргиналов всякого рода превосходит количество <<твёрдых искровцев>>, и у начинающих пользователей выбор остаётся. Ну а старики его и так сделают.

\section{Всё для блага человека} 
\begin{timeline}Сентябрь 8, 2013\end{timeline}

Время от времени на форумах, в Джуйке и во всяких социальных, с позволения сказать, сетях возникают дискуссии о величии прогресса в FOSS-мире. В них обычно чётко обозначаются две антагонистические стороны:


\begin{enumerate}
	\item прогрессисты~--- как правило, разработчики или сочувствующие, ратующие\dots нет, не за вертикальный прогресс, а за прогресс просто, прогресс всего и вся, и во всех направлениях; 
	\item обычные применители, которые и сейчас неплохо живут, и хотели бы в будущем жить не хуже, а потому, согласно Писареву, немножко консерваторы; однако при этом они не против в  будущем жить и получше~--- а потому немножко прогрессисты, иначе в этих дискуссиях бы не участвовали: их можно назвать консервистами. 
\end{enumerate}


Так вот, основной аргумент корсервистов против  прогрессистов-беспредельщиков таков:
\begin{shadequote}{}
Если старое~--- это в большинстве случаев доброе, хотя бы потому, что проверено временем, то новое может оказаться как добрым, так и не очень. И потому обычному применителю нет никакого смысла в этом новом разбираться, пока не будет доказано, что оно доброе на самом деле.
\end{shadequote}
Очевидно, что при этом  бремя доказательства лежит, в соответствии с юридическими нормами всех стран, претендующих на цивилизованность, на утверждающей стороне, то есть на прогрессистах.

В своих возражениях прогрессисты, как обычно, валят всё с больной головы на здоровую и выдвигают два противоречащих друг другу аргументах.  Первый в простых словах сводится к тому, что
\begin{shadequote}{}
Ты сначала разберись в \texttt{systemd} (GNOME~3, Wayland ~--- нужное дописать), а потом говори, что это плохо.
\end{shadequote}
Нимало при этом не смущаясь тем обстоятельством, что консервист потому и не хочет разбираться в озвученном \texttt{systemd} \textit{etc}., что ни один прогрессист ещё не доказал ему, что это лучше старого, работающего и уже хотя бы потому доброго.

А уж когда консервист напоминает прогрессисту про чайник Рассела, последний вообще встаёт на дыбы и, не забыв про аргумент промежуточный (а ты мне докажи, что \texttt{systemd}~--- это плохо), пускает в ход свой главный  аргумент:
\begin{shadequote}{}
А вообще, все эти \texttt{systemd}'ы с ихними Wayland'ами придуманы не для вас, козлов пользователей, а для нас, разработчиков, чтобы нам было хорошо. А если нам будет хорошо, то и вам, козлам пользователям будет сухо и комфортно.
\end{shadequote}
На логике этого ответа останавливаться не буду~--- она такова, что  от неё  не только переворачивается в урне прах Аристотеля, но и основатели троичной логики всколыхнутся в своих вечных пристанищах (уж не знаю, каковы они у них~--- курганы ямной культуры или воды Ганга).

Не буду вспоминать и том, что суше и комфортней применителям было в те времена, когда разработчики сочиняли свои программы без всяких IDE и прочих веяний прогресса, а напрямую в кодах. Это давно сказано известным лектором про литераторов:
\begin{shadequote}{}
Раньше писатели писали гусиным пером вечные произведения, а ныне~--- вечным пером произведения гусиные.
\end{shadequote}
А остановлюсь только на тезисе: <<Если нам хорошо~--- то хорошо будет и вам>>.

Он напомнил мне одну историю из жизни. В начале августа 1998 года жил я в Ближнем Замкадье, в хорошей такой деревеньке, называлась посёлок Новомосковский, что у города Щербинка, и ныне в его административном подчинении. Рассказывать о нём можно долго, но сейчас разговор не о том.

А вот время действия важно, потому что как раз незадолго до того прошёл ураган, памятный многим москвичам, жителям Подмосковья и гостям тех и других. Прошёл, кстати, в вечер памяти Визбор Иосича, ознаменованный концертом по телевизору. Ради которого отложил я работу, подлежащую сдаче наутро: типа~--- фиг с ней, после концерта докую. В середине концерта ураган и случился, оборвало все провода и посёлок на пару недель (как потом выяснилось) остался без света. Так что наутро мне пришлось брать в рузлак системник и тащить его в город-герой Москву, на службу, к розетке, дабы работу доделать.

Но я опять отвлёкся. К моему сюжету ураган имеет только то отношение, что после него из района приехала комиссия из района, на мурсидесах, в направлении нашего посёлка смогла проехать по, с позволения сказать, дороге, метров пятьсот, и официально признала её находящейся в аварийном состоянии.

Справедливости ради надо сказать, что ураган был тут ни при чём~--- разве что как повод для комиссии. Потому что дорога находилась в аварийном состоянии уже тогда, когда я там поселился, и за прошедшие пять лет её аварийное состояние стало только лучше. То есть аварийней. Вплоть до того, что официальный рейсовый автобус к нам уже года три как не ходил, и старенькие бабуси (а таких на посёлке тогда было немало) ковыляли домой от станции на своих двоих, а то и на троих.

Разумеется, комиссия тут же постановила это безобразие прекратить, и начались ремонтные работы. Их хватило на то, чтобы привести в порядок первую сотню метров. Вот в этот момент и происходит  событие, о котором я веду речь.

Случилось так, что аккурат в это время сгорел у меня монитор (ураган тут опять же был ни при чём, просто годы его вышли) и, после некоторых приключений, купил я новый, трубочный, семнадцатиинчёвый, имени товарища Rolsen'а. И надо было мне его домой доставить.

Как обычно, в нужный момент никого из конных товарищей в пределах досягаемости не было, потому сблатовал я тачку на улице. Честно обрисовал водителю ситуёвину с дорогой ~--- на что он мне со свойственной старым московским бомбилам (они безошибочно по виду и повадке узнавались) лихостью сказал: не дрейф, пацан, прорвёмся. Ну загрузили монитор и поехали.

Долго ли, коротко ли, а под базар за жизнь доехали мы до железнодорожного переезда в городе Щербинка. Откуда вели две дороги: прямо~--- прекрасная (по тогдашним масштабам) шоссейка на Астафьевский аэродром и музей Астафьево, и налево~--- к нам прямо по кривой, на посёлок. Первые сто метров которой, как я уже сказал, блистали свежим асфальтом, как у кота\dots уши.

Лихо свернули мы туда\dots и через сто метров водила мне с тоской в глазах сказал: а давай лучше я тебя по обычному московскому тарифу отвезу в Астафьево\dots

И теперь, когда я слышу разговоры о комфорте разработчиков на благо пользователей, всегда вспоминаю тот эпизод. Давайте вообще будем ехать не туда, куда нам надо, а туда, куда легче. Искать кошельки не там, где потеряли, а там, где светлее. И работать не с тем софтом, который для нас лучше по делу, а с тем, который проще разрабатывать.

Но ведь это именно это нам и предлагают прогресситы~--- создатели \texttt{systemd}'ов и прочих Wayland'ов\dots


\textit{P.S. А история кончилась тем, что довёз меня тот мужик до моей калитки и от предложенной компенсации за качество дороги отказался. Со словами: ты мне всё честно рассказал, а я сам дурак, что не поверил. Старые московские бомбилы~--- это были люди. Как и старые программеры~--- и советские, и анти-советские.}

\section{Размышлизм о графических интерфайсах} 
\begin{timeline}Сентябрь 2, 2013\end{timeline}

Нынешние тенденции развития графических сред блестяще опровергают марксовый тезис о спросе, рождающем предложение. На самом деле всё обстоит с точностью до наоборот: сначала рождается предложение некоторой фичи, после чего нас пытаются убедить, что эта фича нам жизненно необходима.

На самом деле спрос на графические интерфейсы исчерпываются тремя позициями. Первая, наиболее распространённая,~--- ожидание Большой Красной Кнопки с надписью 
\textbf{Сделайте мне пе\dotsто}.

Вторая позиция~--- утверждение: 
\textbf{а мне и так пе\dotsто}.

Третья же, наиболее оригинальная, гласит: 
\textbf{отстаньте от меня со своими предложениями, а пе\dotsто я сделаю себе и сам}.

Однако признать это публично~--- означало бы крах всей индустрии по созданию новых графических интерфейсов и совершенствованию старых. Ибо вожделения искателей Большой Красной Кнопки удовлетворить невозможно по определению~--- разве что в штучном исполнении, для лучших из лучших, за отдельную неприличную мзду. Прочие же вполне удовлетворяются интерфейсами существующими.

Вот и приходится разработчикам убеждать их в том, что, например, GNOMEShell~--- это не только ново и прогрессивно, но также удобно, не только эффектно, но и эффективно. А кто того не понимает~--- старый козёл, ретроград, обскурант и враг прогресса.

\section{Время решений, которое никогда не наступит. Часть 1: зачем нужен десктопный Linux} 
\begin{timeline}Июнь 11, 2012\end{timeline}

Это было\dots нет, не в степях Херсонщины, а в джунглях FOSS-мира. На протяжении многих лет изо всех его закоулков слышались радостные вопли о Linux'е с человеческим лицом, перемещаемые стонами о неготовности его к десктопу.

Зачем надо готовить Linux к десктопу? Процитирую одну из своих заметок исторического цикла (который вскоре составят отдельную онлайновую книжку):
\begin{shadequote}{}
Система, пришедшая в дома на пользовательские десктопы, неизбежно рано или поздно окажется в промышленном секторе.
\end{shadequote}
Это хорошо иллюстрируется случаем с Windows: её 95-я инкарнация, появившись на пользовательских десктопах сначала как платформа для запуска игрушек, быстро проникла на десктопы и офисных работников. А с выходом сестры во интерфейсе~--- Windows~NT~4.0~--- оказалась и достаточно массовой средой для серверов рабочих станций. И с тех пор только укрепляла свои позиции по всем фронтам.

Напротив, enterprise-систему, на пользовательские десктопы не попавшую, столь же неизбежно ждёт увядание и в её основной, промышленной, нише. И тому примером является судьба таких проприетарных UNIX'ов, как True64, HP AUX и Sun Solaris. На протяжении 90-х годов все они пытались вторгнуться в десктопные сферы (причём вместе со своими же аппаратными платформами, по мощности превосходившими настольные писюки многократно)~--- и не смогли это сделать. В том числе и потому, что не уделяли тому должного внимания. А нынче остаётся посмотреть, что с ними делается сейчас в сфере промышленной: первая мертва, две остальные существуют по инерции, на старых корпоративных контактах.

Приводить в пример IBM с её AIX, не обнаруживающей, вроде, признаков угасания, не надо~--- это то самое исключение, которое подтверждает правило. Ибо случай этот исключительный во всех отношениях: IBM~--- это близкий аналог нашего советского ВПК со всеми характерными особенностями, и десктопная нажива для них~--- что немецкий doppel горящей душе русского человека.

Кстати, как показывает история, системам, не преуспевшим на десктопах, мало чего светит и в противоположном секторе пользовательского круга, на всякого рода гаджетах. И тут достаточно вспомнить блистательный взлёт PalmOS и Symbian, не имевших никаких десктопных традиций~--- и их медленное, но неуклонное отступление под натиском Windows~Mobile/CE, перешедшее затем в паническое бегство.

И не говорите мне, что гаджетные версии Windows~--- это совсем не то же самое что Windows для десктопов: мне это известно ничуть не хуже, чем тот факт, что Windows~3.X/9X/ME имеют <<внутре>> мало общего с Windows~NT/2000/XP. Магия имени действует даже на тех, кто об этих отличиях знает~--- как показывает приведённое здесь воспоминание о возникновении web-хостинга на Windows-машинах. Что же говорить о пользователях гаджетов, изрядная часть которых полагает смартфон просто большим мобильником.

В меньшем масштабе история повторилось и при появлении нетбуков: модели с предустановленным Linux'ом быстро исчезли из прайс-листов. И не вследствие козней <<Империи зла>>, а из-за отсутствия спроса. Отдельные же сохранившиеся реликты~--- не более чем платформы для установки пиратской Windows.

Подозреваю, что в ближайшее время следует ожидать и ещё одного витка истории. С появлением Windows~8, в том числе и в мобильной модификации, быстро сдадут свои позиции гаджеты на Android'е. Ибо тут длинные руки Microsoft'а дотянулись до святая святых без-win'ного мира: до ARM-процессоров. И тут магия имени сработает в очередной раз: потребители гаджетной продукции (те, которых пользователями в привычном понимании назвать достаточно затруднительно~--- а их абсолютное большинство) снова предпочтут устройства с системой, хотя бы именем похожей на ту, что стоит на их настольных персоналках. А поскольку Android за всё время своего доминирования на гаджетах так и не порадовал, в отличие от былых PalmOS и Symbian, своими технологическими достоинствами, к ним присоединятся и многие из тех, кого действительно можно назвать пользователями.

Так что десктопизация Linux'а~--- непременное условие сохранения позиций этой ОС в обоих секторах, как в вышележащем, промышленном, так и в нижнем, <<гаджетном>>. И не только её, но и открытого и свободного софта вообще. Ибо из всего изобилия операционок FOSS-мира только она имеет хоть сколько-нибудь заметные позиции в настольном секторе.

\section{Время решений, которое никогда не наступит. Часть 2: время <<Ч>>} 
\begin{timeline}Июнь 14, 2012\end{timeline}

В прошлой заметке я попытался, умышленно утрируя ситуацию, обосновать, почему десктопизация Linux'а необходима. Разумеется, я не полагаю себя самым умным~--- эту необходимость понимали очень многие. Что вызывало, как уже было сказано, с одной стороны, призывы десктопизации Linux'а, с другой~--- победные реляции об успехах в космической области десктопизации, а с третьей~--- горестные стоны:
\begin{shadequote}{}
Linux к десктопу не готов!
\end{shadequote}
Однако не все из понимающих ограничивались словами~--- некоторые претворяли их в дела. И в итоге этих дел многие за призывами, реляциями и стонами не заметили, что тихо и незаметно, как любят писать на одном из известных ресурсов, но наступило время, когда Linux оказался к десктопу готов.

Это случилось, как можно видеть сейчас, спустя достаточное долгое время, примерно в 2005 году. Это не значит, что в одно прекрасное утро пользователи проснулись и увидели, что полки магазинов ломятся от коробок с дистрибутивами, из которых каждому, посредством Большой Красной Кнопки, было обещано сделать п\dots по его хотению. Почему, видимо, никто этой даты и не заметил.

Но, начиная с этого самого условного утра, один за другим начали появляться дистрибутивы (и, главное, их варианты, задумчиво именуемые ремиксами и респинами), которые


\begin{itemize}
	\item во-первых, легко и быстро устанавливались,
	\item во-вторых, при должном подборе варианта, в свежеустановленном виде были пригодны к немедленному использованию рядом категорий пользователей, и 
	\item в-третьих, такой свежеустановленный дистрибутив легко доводился до состояния, адекватного задачам пользователя.
\end{itemize}

 

Такие дистрибутивы, новые, в массовом количестве появлявшиеся на протяжении 2005-2010 годов, или старые, в тот же период времени стремительно эволюционировавшие в этом направлении, и знаменовали пресловутую готовность Linux'а к декстопу. Но именно~--- готовность: ни один из них нельзя было назвать готовым десктопным решением, каждый требовал некоторого допиливания и подтачивания. Пусть не мотопилой и топором, как в прежние времена, а более тонкими инструментами, вплоть до алмазного надфиля, но~--- требовал.

Так что наступило время заняться готовыми решениями: создавать специализированные пользовательские системы с соответствующей косметической отделкой. Однако именно этого сделано и не было. Ибо русло майнстрима переместилось в другую сторону: начали создаваться принципиально <<новые>> (хотя реально~--- просто переделанные старые, некогда забракованные или заброшенные) пользовательские интерфейсы, затем комплексы управления системными службами, а потом на горизонте замаячила и новая графическая система в целом.

Говоря аллегорически, можно описать происходившие процессы так: у пользователей Linux'а появился добротный дом. Он не был воздвигнут единым порывом творческого гения по одному законченному плану, а разрастался, достраивался и перестраивался постепенно. И потому в архитектуре его были пережитки и недостатки. Да, местами неискоренимые без полной перепланировки, но проживанию отнюдь не препятствующие.

В этой ситуации есть три варианта принятия решения~--- если исключить одномоментный снос старого дома и строительство на его месте дома нового:

\begin{itemize}
	\item заняться косметическими улучшениями дома, призванными минимизировать неудобства от его архитектурных недостатков; 
	\item начать строить рядом с существующим домом новый~--- с учётом ошибок в планировке первого, и имея его в качестве надёжного тыла; 
	\item проводить капитальный ремонт частично заселённого дома, ремонт необратимый и делающий невозможным возврат к пренему состоянию. 
\end{itemize}

Именно последний путь регулярно оказывается магистральным последние 10 лет развития Linux'а.

Нельзя сказать, что совсем никто не занимался и не занимается <<косметикой>>~--- то есть готовыми решениями на базе существующего Linux'а. Изрядную часть этой ноши взяла на себя фирма Canonical, развивая вокруг Ubuntu пользовательскую инфраструктуру. В частности, потому на неё и катят бочки с обеих сторон: и консерваторы (которые консервы любят), и ультра-радикалы (которые извлекают квадратный корень из единицы).

Однако желающих заняться этим делом оказалось не очень много. Для энтузиастов-разработчиков это скучно и не сулит славы. Для создателей коммерческих систем~--- не интересно, потому что, во-первых, не обещает немедленной финансовой выгоды, что было прекрасно продемонстрировано судьбой первых коммерческих дистрибутивов, типа Corel Linux. А во-вторых, и, возможно, главных\dots к этому <<во-вторых>> я вернусь через пару абзацев.

Не остался нехоженным и второй путь, отмеченный такими вехами, как AtheOS, практически умершая в своём потомке~--- Syllable, как DragonFlyBSD, тихо и незаметно развиваемая в тени старших сестёр своего семейства, как <<академическая>> MINIX 3. Наконец, как Barrelfish, из которой пока неизвестно что получится (и получится ли вообще).

Однако, чтобы пойти по второму пути, надо обладать либо безбашенностью Курта Скавена, либо, напротив, мудростью и равнодушием к внешней атрибутике Эндрю Таненбаума и Мэтта Дилона. Либо, наконец, просто иметь толику лишних денег, как Microsoft. Тех денег, которые можно бросить на пустую рыбозасолочную бочку в расчёте на то, что рано или поздно в ней заведётся осетрина.

Кроме того, при выборе второго пути нужно помнить, что, каким бы прогрессивным и новаторским он не был, и он не сулит мгновенного успеха. Потому что создания на базе новых систем всё тех же законченных пользовательских решений никто не отменял. А на них, как показывает пример DragonFlyBSD или MINIX 3, у разработчиков просто банально не хватает сил.

Так что, как я уже говорил, магистральным оказался третий путь~--- коренной переделки живого и работоспособного организма. Правда, пока он выражается в выбрасывании старых, добрых, но мещанских фарфоровых слоников и уютных, но вышедших из моды тюлевых занавесок. И заменой из обоями из картин модернистов, кубистов и прочих абстрационистов. Которые, как уже становится ясно, сделают этот дом непригодным для жизни ряда его давних обитателей.

Самое же главное вот в чём: пока одни обитатели дома мирно занимаются своими делами, другие долбят стены перфоратором и сносят несущие конструкции, заменяя их временными подпорками. О комфорте обитателей, то есть пользовательских решениях, все в очередной раз забыли. <<Строителям>> не до них~--- они обещают комфорт по завершении своей титанической работы. Мирные обыватели в них не видят смысла, не очень понимая, что останется от их дома по завершении капитального ремонта.

А сторонние зрители, то есть собственно разработчики решений, уже успели убедиться в том, что пути <<строителей светлого будущего>> неисповедимы: сегодня на стенах они заменяют репродукции Васнецова картинами Пабло Пикассо, а завтра, глядишь, извлекут на свет квадрат Малевича. И ещё в запасе остаётся <<Третья мировая война>> Анастасии Стрелецкой и, наконец, картина, символизирующая голод~--- задница, затянутая паутиной.

Что уже наблюдалось на примерах с devfs, HAL'м и рядом других. И никто не может исключить, что всё в очередной раз не закончится анекдотом про чукчу-хирурга, полосующего внутренности оперируемого большим хирургическим скальпелем с криками <<Опять ничего не получилось>>. Легко ли в такой ситуации найти желающих заниматься развитием пользовательской инфраструктуры? Вот их и не густо в нашем мире.

С год назад Андрей Боровский написал статью под умышленно провокационным заглавием~--- <<Почему Линукса нет и не будет на десктопах>>. Главный её тезис можно сформулировать примерно так:

\begin{shadequote}{}
Linux проиграл битву за десктопы.
\end{shadequote}
Из чего делается вывод, что заниматься декстопным Linux'ом бессмысленно, следует развить те сферы, где позиции его ныне сильны как никогда~--- а) сферу потребительских гаджетов и б) сферу встраиваемых устройств промышленного назначения.

Ни в коем случае не оспариваю фактографическую сторону этой статьи~--- более того, готов подписаться под рядом её утверждений. Например, под тем, что современный Linux имеет всё необходимое для десктопного успеха: драйверное обеспечение, пользовательский интерфейс и информационную поддержку. И по совокупности этих компонентов не уступает ни одной коммерческой системе. Собственно, именно это я и назвал в своей \href{http://alv.me/?p=1532}{предыдущей заметке} потенциальной <<готовностью к десктопу>>. Беда только в том, что эта потенция не реализуется: почти никто не предлагает перечисленные компоненты в совокупности, в виде законченного решения.

Однако с основным тезисом и его следствием я категорически не согласен. И оспаривать их начну как раз со следствия. Конечно, <<промышленный эмбеддинг>> и <<потребительский гаджетизм>>~--- штуки очень важные. Но, как я пытался обосновать в прошлой заметке, без прочного десктопного тыла ориентированная на них система рано или поздно захиреет. Скажем, бодрых рапортов о развитии NetBSD, некогда царя горы встроенных устройств, давненько мне не попадалось\dots

Что же до основного тезиса\dots Нет, Linux не проиграл битву за десктопы. Он от этой битвы уклонился именно в тот момент, когда в эту драку следовало ввязаться. И представится ли ему второй такой шанс~--- ведомо только Ахурамазде.

Впрочем, иногда в драку приходится ввязываться и без надежды на победу. Потому что альтернатива~--- это вечное ожидание того дня, когда всё будет перестроено и достроено до посинения. Дня, который при таком раскладе не наступит никогда.

\section{Камо грядеши?} 
\begin{timeline}Апрель 5, 2013\end{timeline}

Более года минуло с той поры, как я перестал быть пользователем Fedora и даже RFRemix~--- некоторое время назад  именно он обоснованно претендовал на звание умнейшего из медведей лучшего из дистров. Однако за развитием обоих продолжаю следить~--- поскольку в их рамках часто предлагаются решения, применимые и в более иных дистрах. Что вполне логично. Ибо кто более всех подготовлен к борьбе с трудностями, как не тот, кто эти трудности для себя (и для нас для всех) придумал? Смотреть соответствующий пассаж Ильфа и Петрова.

Однако нынче я хочу поговорить немного о другом. Просматривая ленту новостей проекта Russian Fedora, натолкнулся я на сообщение Петра Леменкова о повышении степени интеграции \texttt{systemd} и KDE. На содержании самой заметки останавливаться не буду. А только процитирую вопрос, которым она завершается:
\begin{shadequote}{}
А что вы и ваш дистрибутив будет делать, если когда и GNOME, и KDE перейдут на \texttt{systemd}?
\end{shadequote}
Здесь можно, с одной стороны, в очередной раз удивиться наивности автора (да простит меня Пётр, но иного слова не подберу). Ибо ответ на него очевиден, и давался неоднократно за последние лет 10. Произойдёт следующее:

\begin{itemize}
	\item <<лучшие из лучших>> пользователей Linux (то есть самые богатые) обзаведутся Mac'ами; собственно, уже обзаводятся~--- и первыми с корабля побежали те крысы, которые более всего приложили руку к организации течи; 
	\item <<лучшие из худших>> (то есть достаточно обеспеченные и законопослушные граждане, которых, надеюсь, среди нас большинство) приобретут лицензионную Windows;
	\item а вот <<худшие из худших>>~--- те, что, подобно попам, студентам и офицерам, вистуют на девятерной (соответственно, из жадности, бедности или по пьяни), как юзали Выньду Пераццкую, так и будут её юзать.
\end{itemize}

И это не прогноз онолитегов, и не прорицание Нострадамуса. Это описание сценария, происходившего неоднократно, начиная с 1999 года. Когда Linux в лице Mandrake (а на Руси~--- в её ипостаси Russian Edition) впервые, в реалиях той исторической ситуации, стал готов к выходу на десктоп.

Второй раз сценарий повторился в 2005 году, ознаменованном достижением должного уровня развития Системами Быстрого Развёртывания, начиная с Чернышевского Zenwalk'а и заканчивая героями Народной Воли Ubuntu и его клонами.

Ну а третий раз мы наблюдали этот сценарий сразу после выхода RFREmix 14, когда казалось, что пролетариат, единственный до конца революционный класс майнтайнеры Fedora готовы призвать к открытой, революционной борьбе миллионы крестьян широкие массы применителей IT.

Возникает глубокое подозрение, что большинство разработчиков Linux и майнтайнеров его дистрибутивов панически боятся того, что Linux станет по настоящему десктопным. То есть придёт на машины применителей. Ведь тогда им пришлось бы отвечать за тех, кого приручили. А не с азартом чукчи хирурга восклицать: Опять ничего не получилось! И начинать перекраивать всё заново по-живому. Обрекая тем самым Linux на участь испорченной и урезанной копии потребительских систем.

\section{Куда придеши?} 
\begin{timeline}Апрель 24, 2013\end{timeline}

Прошу прощения за смесь церковнославянского с нижегородским в заголовке, но он хорошо отражает суть дела. Потому как эта заметка представляет собой интегрированный ответ на комментарии к предыдущей~--- дабы не повторяться, ибо большая их часть однотипна. Но для начала попробую ответит на кардинальный вопрос:
\begin{shadequote}{}
чем, собственно, \texttt{systemd} так плох\dots
\end{shadequote}
Сам по себе \texttt{systemd} не плох, и не хорош. Просто это замена существующих схем инициализации, которые работают. Резко от них отличная, существенно более сложная и пока не продемонстрировавшая явных преимуществ с точки зрения пользователя. Но время от времени создающая неожиданные проблемы, давным-давно решённые в классических схемах и их <<мягких>> модернизациях. Ну и про агрессивную её пропаганду забывать не следует.

Большая часть остальных комментариев сводится к следующему положению:
\begin{shadequote}{}
Я работаю в Slackware (Debian, Crux\dots нужное дописать), и эти ваши \texttt{systemd}'ы меня не колышат.
\end{shadequote}
Увы~--- не колышат сейчас, так заколышат со временем. Если основные рабочие среды не будут работать вне \texttt{systemd}, то всем дистрибутивам Linux воленс-неволенс придётся её внедрять~--- иначе они потеряют большую часть своих пользователей.

Вариация на ту же тему:

\begin{shadequote}{}
Я использую WindowMaker (Openbox, IceWM\dots как и в предыдущем случае, нужное дописать), и мне эти ваши гномокеды до лампочки.
\end{shadequote}
Да, но подавляющее большинство пользователей таки используют KDE/GNOME, и майнтайнеры большинства дистрибутивов будут ориентироваться на них. Это во-первых. А во-вторых, следующим шагом той самой железной поступи будет интеграция \texttt{systemd} и Xorg, а для Wayland, скорее всего, эта интеграция будет изначальной. Хотя, конечно, остаётся ещё голая консоль\dots Но и её скоро можно будет русифицировать только средствами \texttt{systemd}.

Следующее положение:
\begin{shadequote}{}
\dotsлюди, которые будут пользовать старые версии дистров, пока это им позволит железо и собственные навыки в обновлении софта.
\end{shadequote}
Да, их есть. И среди действующих \textit{применителей} Linux (см.~\hyperlink{customers}{<<Применители vs потребители>>}) их много. Возможно, даже большинство. Они не нуждаются в регулярном обновлении прикладного софта, потому что нынче не то что давеча~--- принципиально новые (и ими востребованные) фичи появляются крайне редко. Они не нуждаются в обновлении системы~--- потому что установленная обеспечивает функционирование нужных им приложений. Они не нуждаются даже в обновлениях безопасности~--- потому что прекрасно знают цену страшилкам о злобных хацкерах, караулящих доверчивых линуксоидов в подворотнях Интернета.

Не нужен им и перманентный апгрейд <<железа>>~--- потому что они знают: любое <<железо>>, ныне существующее за пределами антикварной лавки или музея электронного мастерства, перекрывает большинство реальных применительских задач сполна. И даже с лихвой. А потому в их руках система проходит естественный жизненный цикл UNIX-машины~--- от покупки и установки ОСи и приложений до полной физической амортизации. Но\dots

\dotsно рано или поздно эта физическая амортизация наступает. И тогда оказывается, что заменить вышедшую из строя железяку или невозможно вообще, или~--- за сумму, в галактических кредо превосходящую цену новой машины. А, вопреки присказке, старый конь-дистр пашет глубоко, но на новую борозду-платформу не встаёт\dots

По поводу того, что
\begin{shadequote}{}
\dotsвы забыли про убунту
\end{shadequote}
Нет. Я о ней никогда не забываю. Но инкорпорация \texttt{systemd} в Xorg/KDE/GNOME/\textit{etc}. отсечёт от неё как минимум Kubuntu и Xubuntu. И останутся они с сиротливым Unity. Да и то, когда и если (если и когда) свой Mir доделают. Насколько я понимаю, там всего и делов~--- начать да кончить.

Так что под занавес:
\begin{shadequote}{}
Расскажите ваши действия.
\end{shadequote}
Мне, подобно многим из ранее высказавшихся, по барабану: я могу работать в любой UNIX-подобной системе. Даже в упомянутой в комментариях по ходу дела DragonFlyBSD, а также FreeBSD и OpenBSD. Возможно, и в NetBSD смог бы, хотя не пробовал. То есть пробовал, но не работал. Но ни одна из BSD-систем (не говоря уже о Solaris и её проблематичных форках) для массового употребления применителями в существующем виде не годится. А я в ответе за тех, кого приручил. И поэтому для них придерживаюсь стратегии, сформулированной \textsl{Viktor~W}.:
\begin{shadequote}{}
Покамест еникеи все на привычных местах расположены и нажимаются очевидным способом\dots
\end{shadequote}

\dotsу них будет openSUSE 12.1. А я составляю шпаргалки по \texttt{systemd} в openSUSE 12.3. А также смотрю новости по FreeBSD и читаю списки рассылки DragonFly\dots

\section{Мастерство работы в Unix} 
\begin{timeline}2006 г\end{timeline}

Это~--- наброски к книге, которая никогда не будет написана до конца. Идея её, как легко догадаться, навеяна замечательной книгой Эрика Реймонда <<The Art of Unix Programming>> (в русском переводе~--- <<Искусство программирования для Unix>>. М.: Издательский дом <<Вильямс>>, 2005, 544~с.). Каковую стоит прочитать каждому~--- даже тем, кто и в мыслях не держал сочинять программы ни для Unix, ни для какой-либо другой операционной системы. В ней идеолог движения Open Source наглядно демонстрирует, как, руководствуясь полутора десятками основополагающих принципов, следует сочинять программы. И, более того, приводит примеры программ, написанных на основе этих принципов.

\subsection{Вступление}
Как известно, программы сочиняются не сами для себя (хотя для разработчиков Open Source сочинение программ~--- в определенной мере самоцель). А в том числе и для того, чтобы их (хоть некоторые) использовали в реальной работе. И тут мы приходим к тому, что искусно (мастерски, артистично~--- к этому я еще вернусь) сочиненная программа предъявляет к своему пользователю определенные требования. Главное из которых~--- умение столь же мастерски (а возможно, и артистично) использовать заложенный в ней создателем потенциал (подобно тому, как кровная и выезженая лошадь требует от своего наездника соответствующего умения верховой езды~--- но и к этой аналогии еще придется обратиться). Можно сказать, что чем больше мастерства (артистизма) вложено в программу разработчиком, тем более высокие требования она предъявляет к тем же качествам пользователя. И вот именно этому я и хотел бы посвятить свои заметки.

Для начала~--- о заглавии. Как назвал свою книгу Эрик~--- я уже говорил. Но следует помнить, что английское \textit{Art} передается русским словом <<искусство>> не вполне адекватно. Также, как и производное от него слово Artist означает вовсе не обязательно актёра на сцене, а скорее художника~--- творца в самом широком смысле этого слова. То есть русским эквивалентом к нему выступил бы Мастер. Точнее, конечно, псевдо-русским~--- но в языках-источниках, опять-таки, этому слову придается совсем другой смысл. Потому я и позволил дать своим заметкам такое заглавие~--- <<Мастерство работы в Unix>>.

Отказался я и от напрашивающегося в заглавие слова~--- <<использование>>. Во-первых, потому, что оно вызывает вполне определенные ассоциации (из старого анекдота~--- <<Встретились я и моя компания. Использовали\dots>>). А во-вторых, потому, что, как, говоря словами Эрика, <<проектирование и реализация программного обеспечения должны быть радостным искусством>>, так не менее радостным должно быть и его применение для решения своих задач~--- и слово <<использование>>, подразумевающее потребительское отношение к деятельности творцов, тут невместно\dots

Другое дело~--- работа в Unix. Работа, как еще встарь отметил великий поэт, есть всегда. И даже тогда, когда предметом её оказывается то, что со стороны показалось бы в лучшем случае развлечением, а то и блажью~--- а со временем мы увидим, что деятельность создателей Unix и Linux могла бы предстать именно в таком качестве. Но: <<Настоящая работа всегда груба и яростна>>~--- эти слова героя <<Территории>> Олега Куваева вовсе не перечеркивают то, что она при этом остается <<радостным искусством>>\dots

Ну вот, вводные слова сказаны, пора переходить к предмету обсуждения. И тут следовало бы начать придется с седой <<бронзовой>> древности~--- для того, чтобы понять откуда пошло быть самое понятие Мастерства. Однако об этом речь пойдёт в совсем другой книжке\dots

А пока вопрос: что требуется от конечного пользователя для достижения мастерства работы в Unix? Попытка ответа на этот вопрос и составит предмет дальнейшего изложения. Однако сначала надо рассмотреть вопрос о том, что это такое~--- Unix, и определиться с тем, кто является его пользователями~--- как <<действующими>>, так и потенциальными.

\subsection{Что такое Unix}
Термин Unix и не вполне эквивалентный ему UNIX используется в разных значениях. Начнем со второго из терминов, как более простого. В двух словах, UNIX (именно в такой форме)~--- зарегистрированная торговая марка, первоначально принадлежавшая корпорации AT\&T, сменившая за свою долгую жизнь много хозяев и ныне являющаяся собственностью организации под названием Open Group. Право на использование имени UNIX достигается путем своего рода <<проверки на вшивость>>~--- прохождения тестов соответствия спецификациям некоей эталонной ОС (Single Unix Standard~--- что в данном случае можно перевести как Единственный Стандарт на Unix). Процедура эта не только сложна, но и очень недёшева, и потому ей подверглись лишь несколько оперционок из ныне здравствующих, и все они являются проприетарными, то есть представляют собой собственность неких корпораций.

В числе корпораций, заслуживших право на имя UNIX п\'{о}том разработчиков/тестировщиков и кровью (точнее, долл\'{а}ром) владельцев, можно назвать следующие:

\begin{itemize}
	\item Sun с её SunOS (более известной в миру под именем Solaris); 
	\item IBM, разработавшая систему AIX; 
	\item Hewlett-Packard~--- владелец системы HP-UX; 
	\item IRIX~--- операционка компании SGI. 
\end{itemize}

Кроме этого, собственно имя UNIX применяется к системам:

\begin{itemize}
	\item True64 Unix, разработанная фирмой DEC, с ликвидацией коей перешедшая к Compaq, а ныне, вместе с последней, ставшая собственностью той же Hewlett-Packard; 
	\item UnixWare~--- собственность компании SCO (продукту слияния фирм Caldera и Santa Cruz Operation). 
\end{itemize}

Будучи проприетарными, все эти системы продаются за немалые американские деньги. Однако это~--- не главное препятствие к распространению собственно UNIX'ов. Ибо общей их особенностью является привязка к определенным аппаратным платформам: AIX работает на серверах и рабочих станциях IBM с процессорами Power, HP-UX~--- на собственных машинах HP-PA (Precission Architecture), IRIX~--- на графических станциях от SGI, несущих процессоры MIPS, True64 Unix~--- предназначена для процессоров Alpha (к сожалению, в бозе почивших). Лишь UnixWare ориентирована на <<демократическую>> платформу PC, а Solaris существует в вариантах для двух архитектур~--- собственной, Sparc, и все той же PC. Что, однако, не сильно поспособствовало их распространенности~--- вследствие относительно слабой поддержки новой PC-периферии.

Таким образом, можно видеть, что UNIX~--- это понятие в первую очередь юридическое. А вот за термином Unix закрепилась технологическая трактовка. Так в обиходе IT-индустрии называют все семейство операционных систем, либо происходящих от <<первозданной>> UNIX компании AT\&T, либо воспроизводящих её функции <<с чистого листа>>, в том числе свободные ОС, такие, как Linux, FreeBSD и другие BSD, никакой проверке на соответствие Single Unix Standard никогда не подвергавшиеся. И потому их часто называют Unix-подобными.

Широко распространен также близкий по смыслу термин <<POSIX-совместимые системы>>, которым объединяется семейство ОС, соответствующих одноименному набору стандартов. Сами по себе стандарты POSIX (Portable Operation System Interface based on uniX) разрабатывались на основе практики, принятой в Unix-системах, и потому последние все являются по определению POSIX-совместимыми. Однако это~--- не вполне синонимы: на совместимость со стандартами POSIX, претендуют операционки, связанные с Unix лишь косвенно (QNX, Syllable), или несвязанные вообще (вплоть до Windows~NT/2000/XP).

Чтобы прояснить вопрос взаимоотношений UNIX, Unix и POSIX, пришлось бы опять немного углубиться в историю. А это, как уже говорилось, будет темой совсем другой книжки. Поэтому далее речь пойдет о работе в Unix-системах в самом широком смысле этого слова, без учета всякого рода торговых марок и прочих юридических заморочек. Хотя основные примеры, относящиеся к приемам работы, будут взяты из области свободных их реализаций~--- Linux, в меньшей степени FreeBSD, и еще в меньшей~--- из прочих BSD-систем.

\subsection{Пользователь Unix~--- кто он?}

\hfill \begin{minipage}[h]{0.45\textwidth}
Есть люди, умеющие пить водку, и есть люди, не умеющие пить водку, но все же пьющие её. 

И вот первые получают удовольствие от горя и от радости, а вторые страдают за всех тех, кто пьет водку, не умея пить её. 
\begin{flushright}
\textit{Исаак Бабель}
\end{flushright}
\bigskip\end{minipage}

Чтобы очертить круг пользователей Unix, вернемся к исторически сложившимся сферам её применения, очерченным в предыдущем разделе. А исторически первыми её пользователями были сами разработчики Unix. И не возьмусь судить, что для них было первичным~--- стремление поиграть в любимую игру, или сама по себе разработка, в качестве игры воспринимавшаяся. Так что одна из групп пользователей Unix очевидна~--- разработчики программного обеспечения, как системного, так и прикладного.

Далее~--- коммуникации и Интернет. Естественно, что в круг пользователей Unix попадают администраторы локальных сетей, web- и ftp-серверов, файловых серверов организаций и серверов баз данных. То есть~--- потенциально все, имеющие отношение к этим областям IT-индустрии.

Столь же очевидна роль Unix-пользователей в образовании и исследованиях в области Computer Science. Однако этим его образовательная роль не исчерпывается. Именно на Unix-машинах обучаются так называемой <<компьютерной грамотности>> студенты естественно-научных и инженерных специальностей, вплоть до геологов, геофизиков, океанологов, во многих множествах университетов мира (к сожалению, наша страна тут часто оказывается несчастливым исключением). Возможно, что в дальнейшем этим <<некомпьютерным>> студентам дела с Unix иметь и не придется~--- что ж, изучение его можно рассматривать как элемент базового инженерного или естественно-научного образования, подобно общей физике или матанализу.

А потом~--- кто знает? Человек предполагает, а судьба располагает. Могу сослаться на свой опыт: университетские курсы по термодинамике и физхимии (каюсь, весьма скверно усвоенные в промежутках между полевыми работами, пьянками с аморалками и изучением спецпредметов~--- примерно таков был ряд приоритетов нормального советского скубента-геолога) для меня оказались востребованными уже через год-два работы. А лет 20 спустя я с удовольствием вспоминал об университетском курсе программирования на Мир-2 с её Алголом~--- который также оказался совсем не лишним.

К Unix-пользователям из сферы образования (точнее, самообразования) можно отнести и легионы просто заинтересовавшихся этой системой~--- школьников, студентов произвольных специальностей, специалистов некомпьютерных профессий~--- вплоть до домохозяек (да-да, среди лично знакомых мне линуксоидов есть и домохозяйки). Стимулом для них является исключительно удовлетворение собственного любопытства и тренировка мозгов. Эрик Реймонд пишет, что программировать для Unix~--- занятие не только радостное, но и очень интересное. От лица простых пользователей (тех самых, о которых скоро пойдет речь) могу утверждать, что просто изучать Unix и приемы работы в нем~--- не менее интересно и захватывающе, даже если поначалу не имеет никакого практического применения. Однако его, этого применения, нет лишь до поры до времени: кто знает, сколько из юных посетителей многочисленных форумов выберут своей профессией программирование или администрирование компьютерных сетей. А если и нет, и их дальнейшая работа никак не будет связана с IT-индустрией, освоение Unix даст им совершенно незабываемый жизненный опыт~--- тот, которого они, в большинстве случаев, недополучают на уроках информатики в средней школе или на формальных курсах этой дисциплины в неспециализированных вузах (впрочем, говорят, что и в специализированных вузах дело подчас обстоит не лучше).

Итак, определилось две категории пользователей Unix, как <<действующих>>, так и потенциальных: люди, профессионально связанные с IT-индустрией, будь то разработчики или системные администраторы, и те, кто использует эту систему для образования (в том числе и самообразования). А как же те самые конечные пользователи настольных машин общего назначения?

Тут впору вспомнить лозунг, активно продвигаемый энтузиастами и их сообществами: 
\textit{Linux~--- на каждый десктоп!}
 То есть поговорить и превращении Linux (в данный момент выступающего как ипостась Unix вообще~--- просто как самая массовая ОС этого семейства) в столь же массовую систему, как MS Windows любого рода. И задаться парой вопросов: возможно ли это? и нужно ли?

Чтобы ответить на них, нужно действовать <<от противного>>: то есть достаточно очертить круг тех, кто 
\textbf{не может}
 стать пользователем Linux.

Это, во-первых, пользователи, категорически не способные к освоению компьютера (но, согласно приведенной в эпиграфе максиме Бени Крика, все-таки его использующие). И это~--- отнюдь не признак их глупости, а, как и способность к питию водки, просто индивидуальная особенность: есть же люди, не отличающие ямба от хорея и \textit{до} от \textit{фа}. И здесь здесь то же самое: мне известно немало пользователей с полуторадесятилетним стажем, так и не освоивших запись на дискету или отключение показа непечатаемых символов в Word. Вынужденные работать на компьютере, они, продолжая словами Бени, страдают: и за себя, и за всех тех, кто на компьютере работать не умеет. И Unix только усугубит их страдания.

Далее, из числа пользователей Unix следует исключить тех, кто испытывает идиосинкразию к чтению~--- а таких, увы, становится все больше даже в нашей стране, некогда бывшей самой читающей в мире. Потому что если в Windows (и тем более в MacOS) кое-какие полезные навыки можно получить методом научного тыка, то в Linux без чтения документации и, возможно, даже толстых книг, обойтись практически невозможно (впрочем, к этой теме мы еще вернемся в ближайших разделах).

На Unix никогда не перейдут запойные игроманы и те, кто использует компьютер исключительно в качестве развлекательного центра. И причины понятны: игр под Linux катастрофически мало, и нет в нем ничего, что оправдывало бы смену ОС для домашней аудио- и видеостанции.

Из пользователей-креативщиков вербовать сторонников Linux также в большинстве случаев бессмысленно: работа в нем профессионалов по созданию мультимедийного контента или спецов высокой полиграфии будет попросту неэффективной. Ну нет в нем инструментов для работы профессионального художника\dots

Аналогично~--- с теми, для кого необходимо работать с векторной, пусть даже технического плана, графикой. А также~--- активными пользователями CAD-систем. Разумеется, в мире Unix и Open Source есть кое-какие векторные рисовалки и CAD-системы, но ни те, ни другие для профессионального использования практически не пригодны.

Так кто же остается в сухом остатке, кроме разработчиков софта и профессиональных администраторов компьютерных сетей, а также занятых в первую очередь образованием и самообразованием энтузиастов? И вот тут пора вспомнить о двух первых, в историческом плане, сферах практического применения Unix: обработке текстов и коммуникациях (включая общение в Глобальной Сетью). Разве не к этому сводятся потребности (в первую очередь сугубо профессиональные) многих и многих пользователей компьютеров? Вынужденных удовлетворять их посредством заведомо избыточного Word'а и однозначно убогих Outloock'ов с Internet Explorer'ом.

И в итоге перед нами всего-навсего одна категория пользователей, потенциально ориентированных на работу в Unix: те, для кого по долгу службы (или велению души) важна эффективность работы с текстовым контентом, дополняемая коммуникационными возможностями. И именно креативщики-текстовики (в любой области~--- от технических писателей и научных работников до поэтов и писателей просто) имеют возможность использовать инструменты Unix максимально эффективно. Остается только продемонстрировать им эту эффективность~--- в чем автор и видит одну из основных задач своего сочинения.

Интересно, что среди <<действующих>> пользователей Linux весьма высок процент профессиональных юристов и переводчиков. И тому можно видеть две причины. Во-первых, и те, и другие, безусловно, входят в сословие креативщиков-текстовиков. А во-вторых, и для юристов, и для переводчиков более, чем для остальных представителей этого сословия, важны аспекты легальности используемого ими софта.

\subsection{Три метода решения пользовательских проблем в Unix}

\hfill \begin{minipage}[h]{0.45\textwidth}
Спасение утопающих~--- дело рук самих утопающих!
\begin{flushright}
\textit{Ильф и Петров}
\end{flushright}
\bigskip\end{minipage}

Нет, наверное, прикладной программы~--- будь она под DOS или Windows, Unix или MacOS, свободной или проприетарной, бесплатной или коммерческой,~--- использование которой не создавало бы время от времени каких либо проблем, требующих разрешения. Однако бытует устойчивое мнение, что наибольшее количество проблем возникает при пользовании именно свободных и бесплатных программ под Unix, конкретно~--- под Linux и BSD. Как, впрочем, чревато проблемами и использование самых этих ОСей. По крайней мере, сообщениями о такого рода проблемах полны форумы соответствующей тематики.

Я не буду вдаваться в дискуссию, насколько это мнение обоснованно, а насколько~--- обязано вековым предрассудкам и заблуждениям. Как бы в скобках рискну высказать свое скромное мнение~--- проблем при использовании Linux или FreeBSD ничуть не больше, чем при работе в Windows любого рода~--- а если уж совсем о личном, так и гораздо меньше. Однако то, что проблемы все же возникают, особенно у начинающих пользователей,~--- это медицинский факт, с которым приходится считаться. И, соответственно, требуется наметить методы их решения.

Собственно, проблемы с ОС или прикладной программой ничем не отличаются от любых других~--- например, с автомобилем, телевизором или стиральной машиной. И за все время существования человечества было придумано лишь три метода их разрешения.

Первый~--- это досконально разобраться в проблеме и все сделать самому. Второе~--- заплатить деньги тому, кто эту проблему может решить, будь то соответствующая сервисная служба или частное лицо, сделавшее решение пользовательских проблем своей профессией (в околокомпьютерном мире за ней закрепилось название~--- эникейщик). Третий~--- прибегнуть к помощи друга-понимальщика, который, соответственно, денег за помощь не возьмет.

Достоинства первого метода очевидны. Разобравшись в проблеме и найдя её решение самостоятельно, пользователь приобретает:

\begin{itemize}
	\item вполне конкретные знания по данному вопросу; 
	\item понимание общих методов решения определенного круга проблем; 
	\item и, самое главное, ничем не заменимую уверенность, что нет таких крепостей, которые не смогли бы взять большевики (пардон, POSIX'ивисты), что любая проблема имеет свое решение, и найти его~--- вполне в его, пользователя, силах. 
\end{itemize}


Недостатки, казалось бы, тоже лежат на поверхности: самостоятельное решение любой проблемы требует а) времени, б) чтения книг, сетевых материалов, документации, а подчас даже~--- о ужас~--- и некоторых размышлений. Однако не это главное~--- эти недостатки я отнес бы скорее к особенностям использования Open Source. Тем самым особенностям, которые могут переходит в достоинства. Согласитесь, вовсе не вредно освежить навыки чтения русских текстов, полученные в начальной школе на уроках родной речи (или как там нынче этот предмет называется?). Чтение же иноязычных (почти равно~--- англоязычных) источников, кроме восстановления навыка разбирать буковки, обеспечивает еще и дополнительную языковую практику, от избытка которой, вроде, еще никто и никогда не страдал. Что же до затраченного времени~--- это сторицей окупится тремя вышеперечисленными положительными факторами.

Главными недостатками первого метода решения проблем (назовем его, вслед за Олегом Куваевым, методом большого болота) мне видятся два. Первый имеет силу в основном для начинающего пользователя. Который просто подчас не знает, с какого конца к своей проблеме подступиться: то ли начинать читать man-страницы, то ли~--- беллетризованные новеллы, подобные тем, что сочиняет автор сих строк, то ли~--- страшно подумать~--- хвататься за <<толстые>> книги, описывающие (или делающие вид, что описывающие) <<основы основ>>.

Безусловно, официальная документация проектов Open Source~--- могучий инструмент познания. Как сказал собеседник бы Бени Крика, оставшийся неизвестным: <<Вы знаете тетю Маню?>>~--- Я знаю тетю Маню>>,~--- ответил ему Беня.~--- <<Вы верите тете Мане?>>~--- <<Я верю тете Мане>>\dots

Так вот, если знать верить тете Мане, она даст ответ почти на любой вопрос. Одна беда~--- спрашивать её нужно правильно, то есть не только верить, но и кое-чего знать~--- в том числе и об этой особе как таковой. Да и иметь общие представления о предмете вопроса~--- также не лишне: man- и info-документация сочиняется в первую очередь как справочник для тех, кто, зная в общих чертах суть дела, не может (или не хочет) обременять свою память подробностями.

А вот за общими представлениями пользователю неизбежно придется обращаться либо к около-линуксовой (условно говоря) беллетристике, либо к <<толстым>> руководствам. К чему именно~--- вопрос, однозначного ответа не имеющий, ибо последний зависит от множества факторов. Первый источник, при доле везения, может поспособствовать решению конкретной проблемы~--- хотя бы на <<рецептурном>> уровне. Тем не менее, и без второго обойтись вряд ли удастся\dots

Однако предположим, что наш начинающий пользователь тем или иным образом прорвался сквозь тернии начальных проблем, неизбежных при освоении <<чужой>> системы. Превратившись, тем самым, в пользователя <<действующего>>. Однако почивать на лаврах ему не придется: ведь проблемы возникают и у <<действующих>> пользователей, в том числе и достаточно опытных. И тут перед ними в полный рост встает второй недостаток самостоятельного их решения. Который легко сформулировать, процитировав бессмертный афоризм Козьмы Пруткова: <<Нельзя объять необъятное>>. Что применительно к нашим условиям можно трактовать как невозможность одинаково глубокого изучения всех аспектов устройства и использования свободных ОС и (особенно) их приложений.

Кроме принципиальной возможности, существенен еще и фактор целесообразности: далеко не всегда есть желание возиться с настройками, скажем, некоей мультимедийной программы, разбираясь попутно в кодеках и движках, если это а) не являет собой предмет профессиональной деятельности и б) по определению будет разовой операцией. И тут время вспомнить о двух оставшихся методах решения проблем.

О втором методе~--- платной поддержке, фирменной или индивидуальной~--- я, к сожалению (или~--- к счастью?) сказать ничего не могу, кроме самых общих соображений. К услугам фирменной техподдержки, обещаемой многими майнтайнерами дистрибутивов, в том числе отечественных, ни разу не довелось обращаться. Что же до индивидуального эникейства~--- похоже, в области софта Open Source это занятие популярности не приобрело. В отличие от Windows-сферы, где установка и настройка системы и программ для нее обеспечивало хлеб насущный (или, по крайней мере, пиво насущное) не одному поколению пользователей с некоторым опытом\dots

Остается третий метод~--- обратиться к другу-понимальщику. Метод идеальный~--- ведь <<друг всегда уступить готов место в лодке и круг>>. Правда, для этого надо, чтобы друг такой был~--- тот самый, с которым пройдены тысячи километров по горам и долам, с кем съедены пуды соли и выпиты цистерны водки. И который сделает для вас все. Правда, и самому нужно быть готовым сделать для него все. Но это~--- лишь один момент. Второй же~--- чтобы друг этот был еще и понимальщиком в той проблеме, которая возникла~--- а вот с этим уже сложнее: ведь возможных проблем немало, а друзей, по определению, много не бывает\dots

Благо, современная действительность в виде Интернета предлагает некий эквивалент дружеской помощи~--- в виде специализированных сайтов и форумов, поддерживаемых энтузиастами и их неформальными объединениями. На сайтах можно поискать ту самую около-линуксовую беллетристику, о которой я, как об источнике сведений, упоминал выше. А на форумах~--- задать вопрос по своей проблеме и, возможно, даже получить ответ. И, вполне вероятно, что это окажется самым простым и быстрым способом разрешения проблемы. Впрочем, только при выполнении некоторых условий.

Сначала~--- о сайтах. Их~--- много, перечислять все было бы затруднительно, а уж охарактеризовать их содержание~--- вообще непосильно. Подчеркну только их общие особенности, о которых следует помнить начинающему пользователю.

Содержание основных сайтов Рунета, посвященных тематике Unix, Linux и Open Source, весьма разнообразно. Там можно найти и переводы официальной документации проектов, и переводные же статьи с зарубежных (точнее, иноязычных) онлайновых источников, и более или менее подробные оригинальные статьи, и краткие советы по частным вопросам. Объединяет их одно: все эти ресурсы создаются и поддерживаются на голом энтузиазме. И потому содержат лишь то, что интересно (или в данный момент нужно) авторам материалов и переводчикам. Так что ожидать, что на сайтах найдутся решения на все случаи жизни~--- было бы несколько опрометчиво.

Конечно, можно написать автору сайта или отдельных его материалов письмо с просьбой осветить интересующую пользователя тему. Однако рассчитывать, что просьба будет удовлетворена~--- также не стоит. Ибо обычно все, что автор сайта знает~--- он уже и так написал. В ненаписанном же он либо не считает себя компетентным, либо этот вопрос ему не интересен. Хотя, конечно, бывают и исключения\dots

Так что вопросы, не освещенные в сетевых материалах, лучше задавать в форумах. Однако и тут хорошо бы придерживаться определенных правил. Правила эти многократно освещены в сетевых источниках, например, в классическом сочинении Эрика Реймонда Как следует задавать вопросы, так что заострю внимание только на некоторых моментах, которые кажутся мне наиболее важными.

Во-первых, следует по возможности избегать так называемых <<дурацких>> вопросов (я никого не хочу обидеть этим определением: <<дурацкие>> вопросы задают и весьма умные люди~--- в тех сферах, в которых они не вполне компетентны). Таковыми традиционно считаются вопросы, лишенные всякой конкретики. Прекрасный образец <<дурацкого>> вопроса был дан классиками советской фантастики: <<Дорогие товарищи ученые! Третий год у меня в подполе раздается какой-то стук. Объясните, пожалуйста, откуда он берётся>> (А. и Б. Стругацкие, <<Понедельник начинается в субботу>>).

Впрочем, не меньшее раздражение вызывают и вопросы, содержащие заведомо избыточную информацию. Например, детальное описание частотных характеристик монитора в вопросе о настройке звуковой карты. Или~--- полное содержание файла xorg.conf, сопровождающее вопрос о включении русской раскладки клавиатуры.

Как добиться баланса между недостаточной или избыточной информацией пользователю, который еще не очень отчетливо понимает, что существенно для решения данного вопроса, и что~--- нет? К сожалению, единственный рецепт, который можно тут дать, будет тривиальным~--- это размышление. Каковое вообще весьма способствует искоренению <<дурацких>> вопросов: в ряде случаев после обдумывания того, как сделать вопрос <<не дурацким>>, ответ на него находится сам, и необходимость в самом вопросе отпадает.

Впрочем, рискну предложить и другой рецепт (возможно, завсегдатаи форумов со мной и не согласятся~--- это чисто личное). Так вот, при сомнении лучше дать несколько меньше информации, чем заведомый её переизбыток. Лично меня многостраничные листинги конфигов, сопровождающие вопросы на форумах, просто вводят в ступор. При недостатке же информации, вполне возможно, последуют уточняющие вопросы, позволяющие сформулировать проблему адекватным образом. Впрочем, только при соблюдении второго непременного условия форумного общения.

Ибо второе условие, которое на самом деле должно быть первым,~--- это вежливость, вежливость, и еще раз вежливость. Самый <<наидурацкий>> из <<дурацких>> вопросов имеет шанс на ответ по делу, если он задан в политкорректной форме, а не как~--- <<все бросайте на фиг, и идите чинить мой велосипед>>. Тут, уверяю вас, реакция будет более чем адекватной, даже на тех форумах, где ненормативная лексика не приветствуется.

Однако не следует вдаваться и в другую крайность~--- начинать вопрос с самоуничижающего описания собственного <<ламерства>>: подчас это уничижение оказывается именно тем, которое паче гордыни. На мой взгляд, достаточно простыми словами обрисовать меру своей компетентности (или некомпетентности) в данном вопросе. Хотя и без этого можно обойтись: ведь по умолчанию понятно, что тот, кто задает вопрос, как минимум, не считает себя экспертом в затрагиваемой теме.

Третье из правил <<хорошего тона>>~--- избегать повторяющихся вопросов: на постоянных посетителей форума пятнадцатый вопрос из серии <<Как подключить win-модем имя рек>>, действует хуже красной тряпки на быка; даже в том случае, если вопрос сопровождается точным указанием на производителя модемного чипа.

Как избежать повторяющихся вопросов? Большинство форумов имеет функции поиска (те же, что таковой не имеют~--- вряд ли заслуживают наименования форума, их практическая польза сведется к нулю после недолгого времени функционирования). Далеко не всегда эти поисковые средства идеальны (чай, не Google), однако с некоторых попыток близкие темы отыскать удается. И лучше задать вопрос в такой, уже существующей, близкой по смыслу, теме, нежели плодить очередной топик: в крайнем случае, модератор перенесет ваш вопрос куда надо, или просто выделит в <<отдельное производство>>. Разумеется, при соблюдении все того же второго условия.

И, пожалуй, последнее. Ваш вопрос имеет тем больше шансов на ответ, чем явственней из него будет следовать~--- предприняли ли вы перед этим хоть какие-то усилия по его разрешению собственными силами, или нет. И обращение к средствам поиска форума~--- самое малое из таких усилий.

Но предположим, что все завершилось благополучно~--- помощь на форуме была получена, и пользовательская проблема была разрешена в порядке дружеской услуги. Что же дальше? А вот дальше-то и начинается самое главное\dots

<<Дар красен отдарком>>~--- эта древняя мудрость не потеряла своего значения и по сей день. И в ответ на дружескую услугу, как я уже говорил, всегда следует быть готовым оказать услугу ответную. О каком же ответе может идти речь в данном случае? Да все проще пареной репы: с вами поделились своими знаниями~--- и вы должны быть готовым ими же поделиться в ответ. Как? Элементарно: описать решение своей проблемы, на форуме ли, в виде ли статьи на сайте, заметки в блоге, и так далее~--- форм представления информации может быть немерянно. В расчете на то, что ваши материалы помогут кому-либо решить свои проблемы.

И тут мы возвращаемся к тому, с чего начали: к первому методу решения пользовательских проблем. Ведь чтобы поделиться способом решения какой-либо проблемы, в ней следует разобраться более или менее досконально. Благо, уже сказано, как: путем поиска в Google и аналогичных службах, чтения сетевых и литературных источников, сообщений на форумах. И, конечно же, изучения документации. Таким образом, способствуя увеличению общего количества информации на тему Open Source. И, рискну добавить словами одного из персонажей куваевской <<Территории>>, увеличивая суммарное количество добра на земле. Ибо поделиться с другими в благодарность за полученное~--- это единственный к тому способ. По крайней мере, я другого не знаю\dots

\subsection{Читать или не читать?}
Итак, мы пришли к выводу, что чтение документации~--- неизбежность для пользователя Unix. Или, иными словами,~--- один из обязательных компонентов мастерства работы в этой системе. Можно сказать, что работать в Unix и быть свободным от него нельзя. А внутренняя система документации~--- неотъемлемый компонент всех представителей этого семейства операционок. И от нее нельзя быть свободным точно так же, как в обществе нельзя быть свободным от его законов.

Правда, у пользователя все равно остается выбор~--- читать документацию, или не читать её. Так же как в обществе у любого его члена есть выбор~--- знать или не знать его законы. Приходится лишь помнить, что незнание законов общества не освобождает от ответственности за их нарушение. Так и не-чтение документации не освобождает от расплаты за ошибки, совершенные по причине её незнания.

Правда, существуют системы~--- так называемые user-ориенти\-ро\-ван\-ные дистрибутивы Linux (их еще называют дружественными к пользователю), которые, казалось бы, позволяют на первых порах не обременять себя изучением документации. Тогда как другие системы (все BSD и дистрибутивы Linux, в свое время метко названные <<дружественными к админу>>) заставляют вникнуть в документацию с первых шагов их освоения. А наиболее яркие их представители (например, один из дистрибутивов Linux~--- Gentoo) в обязательном порядке требуют этого еще до установки системы~--- иначе пользователь, пожалуй, и установить-то её сможет разве что случайно.

Какие из систем лучше для начального освоения системы~--- вопрос очень спорный, многократно обсуждавшийся на всех мыслимых форумах по нашей тематике, и здесь я на нем останавливаться не буду. Тем более, что на самом деле разница~--- чисто количественная: рано или поздно пользователь самого-рассамого юзерофильного дистрибутива Linux читать документацию все равно будет~--- другое дело, что он может заняться этим уже в процессе практической работы, по мере возможности и необходимости.

Можно провести историческую аналогию: различие между <<дружественными>> и <<недружественными>> Unix-системами в отношении документации~--- примерно такое же, как в отношении освоения законов обществ <<цивилизованных>> и <<варварских>> (кавычки уместны, потому что ни тот, ни другой термин не отражают существа явления). В <<цивилизованном>> обществе за первое нарушение закона, скорее всего, мягко пожурят (или там по попе нашлепают). А в обществе <<варварском>> первое же нарушение закона вполне может стать последним: зарежут нафиг\dots

Тем не менее, тысячелетия своей истории человечество существовало в <<варварских>> условиях. И ничего, выжило\dots А отдельные индивидуумы и по сию пору неуютно чувствуют себя в условиях цивилизованных.

Аналогичный случай и с освоением работы в Unix. Конечно, большинство пользователей его с чего-либо юзерофильного~--- результаты опросов и личные наблюдения позволяют предположить, что обычно в этой роли выступает дистрибутив Linux под названием Mandrake (ныне Mandriva), на протяжении многих (в масштабах времени индустрии) лет удерживавший пальму первенства в отношении <<любви к пользователю>>. И, опять-таки, в большинстве случаев это оправданно: Windows-подобие таких систем позволяет отодвинуть постижение законов POSIX-мира (а чтение документации, как уже было сказано, один из них) на неопределенный срок. Однако в конце концов ситуация может обернуться вполне по анекдоту о поручике Ржевском. Помните, как он ответил на вопрос дамы, любят ли гусары своих лошадей?

И потому всегда находились и находятся индивидуумы, которые не хотели бы, даже случайно, оказаться в положении лошади, любимой гусарами Ахтырского полка. И вот для них-то вполне приемлемым вариантом первого выбора может оказаться Linux-дистрибутив типа Gentoo или, скажем, любая из BSD-систем. Только нужно помнить: в этом случае никакого снисхождения к их неопытности от окружающего мира ждать не приходится. И законы его придется постигать сразу. Хорошо это или плохо~--- обсуждать, опять-таки, не будем: главное, чтобы сделанный выбор был осознанным, сопровождаясь пониманием всех возможных последствий.