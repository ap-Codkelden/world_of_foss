% версия 1.02
\documentclass[12pt,a4paper,twoside,openany,svgnames]{memoir}
\usepackage[utf8]{inputenc}
\usepackage{cmap}
\usepackage[T2A]{fontenc}
\usepackage[russian]{babel}
% переносы 
\tolerance 1414
\hbadness 1414
\emergencystretch 1.5em
\hfuzz 0.3pt
\widowpenalty=10000
\vfuzz \hfuzz
\raggedbottom

% для клавиш, решение с http://tex.stackexchange.com/a/16981/28759
\usepackage{tikz}
\usetikzlibrary{shadows}
% Дополнительные текстовые символы
\usepackage{textcomp} 
\usepackage[protrusion=true]{microtype}
\frenchspacing
\hyphenation{user-ори-ен-ти-ро-ван-ные за-ко-но-да-тель-но-ад-ми-ни-стра-тив-ны-ми}
\russianpar
\usepackage{enumitem}
\setlist[enumerate,itemize]{itemsep=0pt,parsep=0pt}
%\usepackage{lipsum}
% Раздел настройки параметров memoir
\indentafterchapter % Красная строка первого абзаца после заголовка имеет отступ
% Конец преамбулы документа
%\renewcommand{\refname}{Ссылки} % Раскомментировать, если есть раздел ссылок
%\renewcommand{\bookname}{Книга}
\addto\captionsrussian{\renewcommand{\bookname}{Книга}}
\addto\captionsrussian{\renewcommand{\refname}{Ссылки}}

% клавиши
\newcommand*\keystroke[1]{%
  \tikz[baseline=(key.base)]
    \node[%
      draw,
      fill=white,
      drop shadow={shadow xshift=0.25ex,shadow yshift=-0.25ex,fill=black,opacity=0.75},
      rectangle,
      rounded corners=2pt,
      inner sep=1pt,
      line width=0.5pt,
      font=\scriptsize\sffamily
    ](key) {#1\strut}
  ;
}

% для даты
\newenvironment{timeline}{\medskip\slshape\begin{flushright}}{\end{flushright}\medskip}
\renewcommand{\labelitemi}{\textbf{--}}

% Красивые цитаты
% http://tex.stackexchange.com/a/99058/28759

\usepackage{etoolbox}
\usepackage{xcolor}
% пакет tikz уже заюзан выше
\usepackage{framed}
\newcommand*\quotefont{\fontfamily{\rmdefault}} % selects Libertine as the quote font
\newcommand*\quotesize{60} % if quote size changes, need a way to make shifts relative
% Make commands for the quotes
\newcommand*{\openquote}
   {\tikz[remember picture,overlay,xshift=-4ex,yshift=-2.5ex]
   \node (OQ) {\quotefont\fontsize{\quotesize}{\quotesize}\selectfont``};\kern0pt}

\newcommand*{\closequote}[1]
  {\tikz[remember picture,overlay,xshift=4ex,yshift={#1}]
   \node (CQ) {\quotefont\fontsize{\quotesize}{\quotesize}\selectfont''};}

% select a colour for the shading
\definecolor{shadecolor}{gray}{0.95}

\newcommand*\shadedauthorformat{\emph} % define format for the author argument
% Now a command to allow left, right and centre alignment of the author
\newcommand*\authoralign[1]{%
  \if#1l
    \def\authorfill{}\def\quotefill{\hfill}
  \else
    \if#1r
      \def\authorfill{\hfill}\def\quotefill{}
    \else
      \if#1c
        \gdef\authorfill{\hfill}\def\quotefill{\hfill}
      \else\typeout{Invalid option}
      \fi
    \fi
  \fi}
% wrap everything in its own environment which takes one argument (author) and one optional argument
% specifying the alignment [l, r or c]
%
\newenvironment{shadequote}[2][l]%
{\authoralign{#1}
\ifblank{#2}
   {\def\shadequoteauthor{}\def\yshift{-2ex}\def\quotefill{\hfill}}
   {\def\shadequoteauthor{\par\authorfill\shadedauthorformat{#2}}\def\yshift{2ex}}
\begin{snugshade}\begin{quote}\openquote}
{\shadequoteauthor\quotefill\closequote{\yshift}\end{quote}\end{snugshade}}
% Конец красивых цитат
\usepackage{hyperref}

\author{Алексей Викторович Федорчук} %\thanks{\href{mailto:my@alv.me}{my@alv.me}}}
\title{Мир FOSS\\Заметки гуманитария Алексей Федорчук aka alv}

\date{2013}

\begin{document}

\maketitle

%\chapter*{\centering \begin{normalsize}Abstract\end{normalsize}}
\begin{quotation}
\noindent
Сборник заметок, посвященных гуманитарным аспектам мира FOSS, написанных с 1998 по 2012 годы. Публиковались на сайтах автора и ресурсах, с которыми он сотрудничал, а также в некоторых <<бумажных>> журналах. В настоящем издании исправлены, дополнены, взаимоувязаны, циклизованы и структурированы. Не рекомендуется к прочтению информационно-неполовозрелым и литературно невосприимчивым гражданам.
\end{quotation}
\clearpage
\tableofcontents
\clearpage

\chapter{Резюме}

\section{Предисловие} 

Внимаю читателей предлагается нечто вроде моих избранных трудов~--- сборник заметок, посвящённых различным гуманитарным аспектам существования мира FOSS (Free and Open Source Software). Заметки эти сочинялись на протяжении последних пятнадцати лет. Часть из них представляет интерес как исторические свидетельства~--- памятники эпохи. Однако многие сохраняют актуальность: сколько раз я ловил себя на том, что в разговоре на одну из затронутых здесь тем, пересказывал написанное собой же, любимым, лет несколько назад. Ибо поколения пользователей UNIX, Linux и прочего Open Source меняются, и перед вновь пришедшими в этот мир их применителями встают одни и те же вопросы. Надеюсь, что мои заметки помогут найти ответы на некоторые из них.

В этом сборнике я попытался как-то структурировать свои заметки~--- иногда по темам, иногда~--- хронологически. Вне зависимости от времени сочинения заметок, они публикуются более или менее <<как было>>~--- лишь в несколько причёсанном виде. Так, по возможности исправлялись <<ачипятки>> и прямые фактические ошибки. Конечно, не обошлось и без некоторой стилистической правки. Местами вносились актуализирующие комментарии, обычно выделенные курсивом. Ну и для связки тематически связанных материалов везде по умолчанию подразумевается наличие неопределённого артикля\dots думаю, читатель и сам знает, какой неопределённый артикль используется на Руси для связки несвязуемого и при  впихании невпихуемого.

Первоисточники и прототипы всех собранных тут заметок легко найти в Сети, на моих многочисленных сайтах, на ресурсах, с которыми я сотрудничал, и на тех наборах страниц, которые специализируются на переразмещении чужих материалов. Ссылок я не привожу, потому как никто их искать всё равно не будет~--- как не искали их поколения начинающих линуксоидов. Да и сделан этот сборник именно для того, чтобы избавить читателей от столь тяжких трудов. Здесь вообще нет того, что модно называть <<пруфлинками>>, ибо сборник предназначен для оффлайного чтения.

Впрочем, некоторые материалы в том виде, в каком они здесь представлены, в сети не найти~--- например, заметки о деле Ханса Рейзера я впервые собрал воедино только здесь.

Некоторые заметки написаны в соавторстве с моими коллегами и товарищами~--- Владимиром Поповым и Алисой Деевой. Впрочем, Алиса~--- косвенный соавтор и многих других материалов, ибо являлась их вдохновителем.

Есть здесь и страницы, написанные по моей просьбе Алексеем Жбановым aka allez и Олегом Свидерским aka Uncel\_Theodore. 

Многие материалы сборника родились в ходе обсуждения взаимноинтересных тем с Сергеем Голубевым. А также при на форумах~--- как ныне действующих, так и уже не существующих. И, разумеется, из разговоров в \href{Джуйке}{http://www.juick.com}. Участников которых я не смог бы перечислить поимённо. Однако я искренне признателен всем моим товарищам и коллегам~--- без них не было бы ни этого сборника в целом, ни составляющих его заметок.

А в заключение хотелось бы вспомнить коллег и товарищей, к верхним людям ушедших~--- Олега Свидерского и Евгения Яворских, известного своим читателям как Акустик. Не помянуть, а вспомнить: про них~--- последняя притча из этого сборника. Не только про них~--- но это уже из совсем другой книжки\dots
\chapter{Десктопы и пользователи}

\textsl{В этой рубрике объединены заметки, посвящённые в основном этологии и/или этнографии сообщества FOSS (уж кому какой термин больше нравится применительно случаю). В истории предшествовавшей литературы наиболее близким к ним жанром является анималистика. Та самая, блестящие образцы которой оставили нам Эрнест Сэтон-Томпсон, Чарлз Робертс и Джералд Даррелл. Не хочу равнять себя с великими именами. Но я старался учиться у них. А также у Конрада Лоренца и Нико Тинбергена~--- великих этологов.}
\section{Рецепты против принципов, или Почему компьютер~--- не видак} 
\begin{timeline}
2003, осень
\end{timeline}

Обсуждение темы <<Зачем Linux дома>> продолжает давать мне сюжеты для сочинений. В частности, я наконец-то смог сформулировать, почему сравнение в удобстве использования компьютера и видеомагнитофона всегда казалось мне некорректным. Да все из-за той же подмены понятий.

Действительно, что такое видеомагнитофон (магнитофон просто, телевизор, радиоприемник~--- нужное подчеркнуть)? Это~--- исключительно инструмент для потребления. Потребления продукции, созданной кем-то другим (случай продукции собственной рассмотрим чуть ниже). Причем продукции, в отличие от продуктов питания, не жизненно важной. То есть потребляемой исключительно с целью развлечения (случай профессиональных теле- или радио-обзирателей также не рассматривается).

А потому развлекаемый, если так можно выразиться, пользователь-потребитель вправе требовать от такого инструмента минимума сложностей в использовании. Не желает он понимать принципы работы привода видака или передачи сигнала на телевизор. Потому как иначе это превратится в работу, а его задача~--- как раз отдохнуть от своих (в том числе и производственных) проблем за любимой <<Великолепной семеркой>> или <<Тремя мушкетерами>> (нужное вписать).

Более того, пользователь-потребитель видеомагнитофона имеет полную возможность обойтись без всяких знаний о его устройстве. Потому как его потребность~--- не забивать голову техническими подробностями,~--- сполна удовлетворяется производителями такого рода техники. Иначе её просто не стали бы массово покупать~--- вспомним тех же радиолюбителей с паяльниками, много ли их было в процентом отношении? Чай, не кусок хлеба, обходились Маяком на фабричной Спидоле (случай профессиональных шпионов или убежденных диссидентов не рассматриваем также).

Так что развлекаемый пользователь вполне может обойтись минимумом самых простых рецептов, как то: вставить кассету, нажать кнопку 
\keystroke{Вперед}, после просмотра нажать кнопку \keystroke{Eject}. Хотя и тут требуется некий минимум подготовки~--- например, какой стороной кассету засовывать. Иначе возникнет нештатная ситуация, требующая уже дополнительных рецептов (типа~--- использовать деревянную линейку) и дополнительных знаний (на что этой линейкой давить).

Однако все это~--- лишь до тех пор, пока происходит пассивное потребление кем-то созданной продукции. Если же пользователь видика/телевизора, насмотревшись передач типа <<Сам себе режиссер>> или там крутой порнографии, решит создать собственный шедевр в том же духе, ситуация тут же меняется.

Во-первых, ему потребуется инструмент для созидания, а не потребления. Сиречь~--- кинокамера. Во-вторых~--- умение ею пользоваться. Под которым нужно понимать не только владение её интерфейсом (грубо говоря, знания тех же кнопок запуска-останова), но и умение снимать. То есть~--- понимание перспективы, освещенности, навыки создания какого-никакого сюжета. Думаю, все согласятся со мной, что мало что может быть страшнее видеоролика, снятого по методу <<что увидел, то и снимаю>>. Помнится, меня всегда доставал просмотр экспедиционных слайдов~--- обычно именно таким образом он и осуществлялся.

А уж если такой пользователь в итоге изберет видеосъемку своей профессией~--- тут ему потребуется и многое другое. В том числе не лишними окажутся знания о физических принципах фото- и видеосъемки. Не случайно лучший из лично известных мне фотографов, Владимир Родионов, по образованию~--- физик-оптик с неслабым опытом инженерной работы в очень нестандартных условиях (см. \href{http://www.rwpbb.ru}{www.rwpbb.ru}).

Да и на видеомагнитофон такой пользователь (а он ведь по прежнему~--- пользователь видеомагнитофона по определению, не так ли?~--- ибо производством оных не занимается) будет смотреть совершенно иначе. Для него это будет уже не инструмент потребления, а аппарат, способный подчеркнуть или затушевать собственное мастерство (или~--- отсутствие такового). То есть превратится в деталь производственного цикла, в орудие производства.

И тут требования к удобству интерфейса даже видеомагнитофона отступают на второй план. Мало горя будет в том, что на конкретной модели кнопка запуска расположена не совсем там, где хотелось бы, если это~--- единственный (или единственный доступный финансово) инструмент, способный обеспечить, скажем, требуемую контрастность или яркость.

Вернемся, однако, к рецептам и принципам. Пока пользователь видеотехники остается чистым потребителем, он вполне может обходиться минимумом рецептов. Однако первые же попытки креативного характера приводят к резкому возрастанию потребности в HOW TO: куда встать, под каким углом держать, откуда направить свет, и так далее. И здесь перед ним два пути: экстенсивный или интенсивный. Первый~--- копить все те же практические рецепты, наработанные эмпирически. Однако скоро а) их становится очень много и б) все рецепты по определению охватывают только стандартные ситуации, ничего нетленно-непреходящего с их помощью не создашь. И приходится нашему пользователю волей-неволей обращаться к истокам~--- то есть базовым принципам. Бия себя по голове за то, что в школе не читал внимательно <<Физику>> Пёрышкина\dots

Теперь обратимся к компьютерам. В отличие от видеомагнитофонов, они изначально создавались не для потребления, а для креатива (чего бы то ни было). И многими по сей день используются главным образом для работы. В том числе, а то и в первую очередь~--- для работы дома. Помнится, для меня первая персональная персоналка обеспечила именно возможностью не ходить на службу (благо, был безработным, и ходить было некуда).

Конечно, компьютер имеет и потребительски-развлекательную функцию. Каждый пользователь, профессионально с компьютером связанный (не в смысле~--- профессиональный компьютерщик, а~--- выполняющий свою профессиональную работу главным образом на компьютере), отнюдь не прочь послушать музыку или посмотреть киношку без отрыва <<от станка>>. Однако функция эта не просто вторична по времени, она не принципиально важна и по значению. Думаю, если просмотр видеоролика будет мешать работе профессионально критичной программы, любой профессионал предпочтет смотреть его по видику (тому самому, развлекательно-потребительскому).

Очевидно, что для профессиональной работы важнее функциональность, а не удобство использования (мне безразлично, насколько удобно я не могу выполнить свою задачу). А развлекательная составляющая~--- вроде бесплатного приложения.

Говорят, что есть и чисто развлекаемые пользователи-потребители компьютеров, покупающие навороченные P-4 вместо музыкальных центров или домашних видеотеатров. Хотя мне таковых видеть и не доводилось. Однако рискну предположить, что это в основном~--- именно люди, профессионально с компьютерами связанные (или те самые энтузиасты цифрового контента), иначе не вижу тут ни практической, ни финансовой целесообразности. Может, я и отстал от жизни, но мне кажется, что нормального качества телевизор стоит дешевле, чем высококлассный монитор (а только на таком просмотр фильма и доставит удовольствие истинному ценителю).

Это я всё к тому, что даже и в развлекательном аспекте компьютер остается где-то инструментом креативным, со всеми вытекающими последствиями. Кроме того, он при этом практически не теряет своего универсализма. Если на видеомагнитофоне слушать Баха, скажем так, несколько затруднительно, а на CD-плейере, напротив, фильмы Бессона обычно не смотрят, то тот же мультимедиа-компьютер призван выступать в обоих качествах (а зачастую, повторюсь, на нем даже еще и работают). И потому ожидать, что он будет так же прост в обращении, как монофункциональный развлекатель~--- по меньшей мере излишне оптимистично.

Вернее, чисто развлекательную функцию компьютера тоже можно упростить до состояния видеомагнитофона. Замечательным примером чему служит Linux-дистрибутив под названием MoviX. Это~--- один из т.~н.~LiveCD, то есть система на компакте, способная с оного не только запускаться, но и полноценно функционировать. Функции MoviX'а, правда, весьма ограничены. А именно, он умеет только крутить мультимедийные файлы (видео и аудио). Но зато умеет это очень хорошо. И, главное, ничуть не сложнее, чем бытовой агрегат соответствующего назначения.

Так что достаточно легкого движения рук~--- вставки диска MoviX в привод и комбинации из трех пальцев,~--- чтобы волшебным образом превратить тысячебаксовый компьютер в элегантный видеомагнитофон или CD-плейер красной ценой в пару сотен. Благо, и обратное превращение ничуть не сложнее\dots

Но упростить производственную-то функцию компьютера~--- все равно не удастся (работать вообще довольно трудно, как говорил, если не ошибаюсь, Антон Палыч Чехов). И потому лозунги типа <<С выходом Windows~3.0 (3.1, 95, 98, ME, XP), обращаться с компьютером \textit{наконец-то} стало также просто, как с бытовой техникой>>, которые я лично слышу уже более 10 лет,~--- лукавы, как минимум, вдвойне. Во-первых, это та самая подмена понятий, с которой я начал заметку~--- если под обращением понимать не только развлекательную сторону, но и производственную (см. вышеуказанный тезис А.П.~--- создавать видеофильмы никогда не будет также просто, как их просматривать). А во-вторых, сама регулярность появления таких лозунгов (я не случайно выделил слова наконец-то) вызывает подозрения, что и с развлекательной стороной все еще сохраняется некоторая напряженка.

В частности, за что еще не люблю Windows,~--- за то, что она, обещая избавление от всех и всяческих проблем хотя бы в потребительском аспекте, своих обещаний не выполняет (впрочем, это~--- к вопросу о злокозненности). Давеча встала задача просмотреть (в Виндах) CD'шки с записью <<Прогулок с динозаврами>>. И вы думаете, это было также просто, как кассету на видике? Хрен в ухо, как сказали бы в Одессе (в варианте для дам, разумеется): штатный MediaPlayer (из ME) смотреть их отказался, собственный плейер с первого диска сначала установил сам себя, и только потом начал показывать, да еще в окошке. Разворот в полноэкранный режим был интуитивно не ясен, выход из оного~--- только тремя пальцами, на произвольном масштабировании машина (не из слабеньких) просто висла. А перед просмотром второго диска тот же плейер установил сам себя второй раз. Ну как тут не вспомнить про

\medskip\texttt{\$ xine имя\_файла}\medskip

или аналогичное действо с Mplayer'ом\dots Конечно, возможно, что это руки у меня кривые, и нужно было внимательно прочитать инструкцию. Но не от RTFM'а ли нас обещал избавить отец Выньдоуз? Впрочем, это~--- опять вдогонку предыдущей статье.

А мы тем временем вернемся к рецептам и принципам. Худо-бедно, но с развлечениями на компьютерах можно справиться посредством первых. А вот с производством любого рода? Рассмотрим это на примере более-менее близкой мне области~--- работы с текстами, претендующими на оригинальность (то есть выдумываемыми из головы).

Подобно нашему видеолюбителю, профессиональный текстовик, пересев с пишущей машинки (или с письменного стола со стопой бумаги и паркеровской ручкой) за компьютер, быстро узнает множество простых рецептов, как то: нажав клавишу \keystroke{Insert}, он может забить неправильно введенный текст (как забивочные листки на машинке, только проще), клавишами \keystroke{Delete} или \keystroke{Backspace} можно уничтожить лишнюю букву (в отличие от замазки, без следов), и так далее.

Жить нашему текстовику становится лучше, становится веселей. Но еще не до конца. Потому как он узнает об управляющих последовательностях, с помощью которых может мгновенно переместиться в требуемое место текста и продолжить набор-редактирование, о глобальном поиске и замене, об автоматической проверке правильнописания, и о многом, многом другом.

Однако суть работы текстовика при этом не меняется, как не меняется и стиль мышления. И то, и другое по прежнему линейно, оригинальный текст создается от начала и до конца, как и за пишущей машинкой, расширяются только возможности возврата к написанному и внесения в него корректив. И потому следующая мысль~--- а не структурировать ли текст изначально, внеся соответствующую разметку рубрик, подрубрик, параграфов? И вот это~--- уже скачок качественный, ведь для структуризации (ненаписанного еще) текста последний весь, целиком, уже должен быть вот здесь, в~\dots~(ну, сами знаете где).

К слову сказать, современные ворд-процессоры WISIWIG-типа пользователя к такой структуризации отнюдь не стимулируют. За ненужностью народу, вероятно. В одной книжке про Word мне как-то встретилась фраза, что стилевая разметка~--- это штука очень сложная, которая по силам только высоким профессионалам (в Word'е же, вероятно). Простым людям, видимо, проще вручную придавать заголовку каждой главы кегль и начертание (и при этом помнить, какое оформление было придано Главе 1, какое~--- Главе 2, и так далее). Или центрировать заголовки клавишей \keystroke{BackSpace}.

Впрочем, я опять отвлекся, перестройка мышления текстовика~--- тема совершенно отдельная. Вернемся к принципам. С созданными и отредактированными текстами подчас приходится продолжать работать~--- делить на фрагменты, соединять, извлекать части одного документа и вставлять в другой. И вот тут-то и обнаруживается неэффективность рецептов.

Возьмем простую задачу~--- создание единого документа из нескольких существующих, причем~--- включенных в определенной последовательности. Мне этот пример кажется очень показательной, и я не устаю его повторять. Можно: открыть документ 1, перейти в его конец, щелкая мышью по контекстному меню, (или даже воспользовавшись существующим рецептом~--- макрокомандами с привязанными к ним горячими клавишами) вставить туда документ 2, и так далее. Быстрее, конечно, чем подклеивать бумажные листы силикатным клеем, но все ли это, что может компьютер?

Нет, если узнать (или вспомнить) несколько принципов более общего порядка. Любой документ~--- суть файл. Любой файл может быть выведен на устройство вывода (экран или, скажем, принтер). А любой вывод может быть перенаправлен с одного устройства вывода на другое. Но ведь любое устройство вывода~--- тоже файл. Значит, вывод любого файла может быть перенаправлен не в файл устройства, а в другой, например, текстовый, файл. Остается только отыскать команду, которая это сделает. И такая команда легко находится~--- это 
\texttt{cat}. В результате конструкция

\medskip\texttt{\$ cat file1 file2 file3 > file-all}\medskip

создаст нам результирующий документ в один присест. Причем составные части его расположатся в той последовательности, в какой нам нужно. И, если мы заранее озаботились тем, чтобы структура наших рабочих файлов хоть как-то коррелировала (не обязательно плоско-линейно, но по какому-либо принципу) со структурой итоговой работы (а это та самая структуризация мозгов, о которых я говорил чуть раньше)~--- результирующий документ будет структурирован должным образом, причем без всяких дополнительных усилий.

Задача обратная~--- поделить наш правильно (!) структурированный документ на отдельные части в соответствие с его внутренней структурой (например, разбить книгу на главы, на предмет раздельного использования). Для автоматизации процесса нам достаточно знать о другом относительно общем понятии~--- регулярных выражений, причем лишь в той его части, которая описывается термином шаблон (pattern). И задача сводится к тому, чтобы отыскать в нашем файле строки, начинающиеся последовательностью символов <<Глава>> и каждую последовательность символов в промежутке записать (то есть вывести) в самостоятельный файл. Что можно сделать многими способами, но один из них~--- штатен и элементарен, это команда \texttt{split} (во FreeBSD) или \texttt{csplit} (в Linux).

Последний пример показывает, что, хотя понимание принципов и не избавляет уж совсем от обращения к рецептам, но зато позволяет вычислить последние при их незнаниии. Дело в том, что во FreeBSD команда 
\texttt{split}
 универсальна, и служит для разделения файла по любому параметру~--- размеру, номеру линии или шаблону. Одноименная же команда в Linux выполняет только первые две функции, опции 
\texttt{-p}
 (
\texttt{--pattern}
) в ней не предусмотрено. Что поначалу может расстроить. Однако если понимать, что в принципе разделение файлов по шаблону почти ничем не отличается от такового по номеру линии или размеру (и то, и другое суть действия, основанные на анализе последовательности символов), остается только изыскать соответствующий рецепт. Что можно сделать просто в лоб~--- поискать слово 
\texttt{split}
 в каталоге с 
\texttt{man}
-страницами командой 
\texttt{grep}
 (строго говоря, не слово, а последовательность символов, и командой 
\texttt{zgrep}
, так как страницы эти обычно в gzip-сжатом виде). Чем и обнаруживается man-страница с описанием команды 
\texttt{csplit}
, прочитать которую~--- уже вопрос элементарной грамотности.

Но особенно явно превосходство принципов над рецептами выступает при конфигурировании всего и вся~--- от общесистемных опций до шрифта меню конкретного приложения. Рецептурный подход~--- использование специализированных средств настройки, оформленных в виде самостоятельных утилит или встроенных в прикладные программы. Самостоятельные утилиты такие многочисленны и разнообразны, не зря же Владимир Попов как-то заметил, что число утилит конфигурирования давно превзошло количество конфигурируемых параметров. Интерфейс у них разный в разных дистрибутивах, да к тому же еще и может меняться от версии к версии. Так что доскональное знание какого-либо DrakX из Mandrake ничем не поможет при работе с 
\texttt{sysinstall}
 из FreeBSD, и наоборот.

Если же не <<поступаться принципами>>~--- достаточно раз и навсегда понять, что все параметры настройки системы описываются в соответствующих конфигурационных файлах, которые суть обычные тексты, могущие быть открытыми в любом текстовом редакторе и там модифицированы надлежащим образом. Что и проделывают, только в завуалированном виде, все настроечные утилиты.

Что касается объектов конфигурирования, то есть соответствующих файлов, и субъектов оного~--- параметров настройки,~--- то они определяются стандартными утилитами работы с текстами, например, командой 
\texttt{find}
~--- для поиска файлов по маске, и командой 
\texttt{grep}
~--- для изыскания в них подходящих по смыслу фрагментов. Ну и, разумеется, осмысленного анализа результатов того и другого\dots

Вопреки сложившемуся убеждению, для этого не обязательно быть UNIX-гуру или к таковому обращаться. Во-первых, тот же гуру даст, скорее всего, именно конкретный рецепт на злобу дня. Во-вторых, многие вещи в системе настраиваются один раз в жизни, и вполне возможно, что наш гуру благополучно забыл о том, как именно он это делал. Так что своей просьбой вы просто вынуждаете его вторично проделать ту самую цепочку логических рассуждений, отталкивающихся от общих принципов, которую вы легко (и с пользой для духовного самосовершенствования) могли бы проделать сами.

И еще к слову~--- с успехом (надеюсь) применяя принципиальный подход к жизненно важным для работы настройкам, можно не поступаться принципами и при настройке вещей развлекательного свойства. Например: звуковая карта определена и в ядре настроена правильно, соответствующий софт установлен и работает, но mpeg-файлы, скажем, воспроизводиться не желают.

Вероятно, в user-ориентированных дистрибутивах существуют какие-нибудь специальные утилиты для настройки этого хозяйства, и можно обратиться к ним. А можно просто вспомнить, что звук воспроизводится устройством, устройство есть файл, а файл имеет определенные атрибуты принадлежности (хозяину, то есть пользователю имя\_рек, группе и всем прочим) и атрибуты доступа (в просторечии именуемые правом чтения, исполнения и изменения). И остается только проверить, а имеет ли данный пользователь должные права доступа к этому файлу? И если выясняется, что файл 
\texttt{/dev/audio}
 открыт для всеобщего использования во всех отношениях~--- посмотреть, а не есть ли его имя лишь символическая ссылка на файл реального устройства, отвечающего за воспроизведение звука, и проверить права доступа к тому.

И опять к слову: понятие атрибутов принадлежности и доступа, одно из краеугольных в UNIX-системах, существует и в тех Windows, которые можно назвать всамделишними (то есть NT/2000/XP, даже в ME вроде бы есть зачатки~--- семейный доступ в систему и прочее). Да вот только пользователи их об этом часто не подозревают. Не потому, что чайники, а потому, что их от этого знания тщательно оберегают.

В результате к нештатным ситуациям (потеря пароля~--- кто от этого застрахован, все мы люди, все мы человеки) пользователи Windows оказываются просто морально не готовы: нужно дергаться, звать админа, даже (страшно подумать) лезть в книги (не для того ли мы отказывались от \texttt{man}- и \texttt{info}-страниц?) для поиска рецептов, соответствующих ситуации.

А в UNIX (вернее, Linux/*BSD, за прочие не скажу по незнанию)~--- все просто, если помнить о файле (файлах) паролей, однопользовательском режиме (или возможности загрузки с внешнего носителя) и о том, что дисковые устройства нужно монтировать (насколько я знаю, в NT сотоварищи диски тоже как бы монтируются, только <<дружелюбно>> и <<прозрачно>> для пользователя; в итоге ему остается только удивляться сообщениям об ошибках, выдаваемых при неправильном извлечении USB-драйва).

В общем, подведу итог. Рецептурный подход вполне приемлем (возможно, даже идеален) при потребительско-развлекательных задачах. И оказывается, мягко говоря, не самым эффективным при задачах производственно-креативных. А отработав принципиальный подход на них, становится уже в лом искать, какая кнопочка отвечает за масштабирование окна данной программы воспроизводства видео\dots Проще задать сиюминутную геометрию в командной строке или (раз навсегда) в соответствующем файле ресурсов.

\section{О свободе выбора в чтении документации} 
\begin{timeline}
2005, лето
\end{timeline}

\hfill \begin{minipage}[h]{0.45\textwidth}
Расплата за ошибки~---\\
Она ведь тоже труд.\\
Хватило бы улыбки,\\
Когда под ребра бьют\dots
\begin{flushright}
\textit{Булат Окуджава}
\end{flushright}\bigskip\end{minipage}

В обсуждении на одном из форумов темы о пригодности Gentoo для начинающего пользователя прозвучала мысль, что Gentoo ограничивает свободу пользователя, вынуждая его к чтению документации. Так ли это? Давайте посмотрим.

Начать с того, что жить в Линуксе и быть свободным от него нельзя (почти по Карлу Марксу). А документация~--- неотъемлемый компонент Линукса (а также всех иных UNIX'ов сотоварищи). И от нее также нельзя быть свободным, как в обществе нельзя быть свободным от его законов.

Однако у пользователя все равно остается выбор~--- читать документацию, или не читать её. Так же как в обществе у любого его члена есть выбор~--- знать или не знать его законы. Приходится лишь помнить, что незнание законов общества не освобождает от ответственности за их нарушение. Так и не-чтение документации не освобождает от расплаты за ошибки, совершенные по этой причине.

И в этом отношении между Gentoo и любыми другими дистрибутивами и операционками разница чисто количественная. Пользователю Gentoo желательно начать чтение еще до установки своей системы~--- иначе он, пожалуй, и установить-то её сможет разве что случайно. А пользователь, скажем, Mandrake может заняться этим через некоторое время после иинсталляции.

То есть различие примерно то же, что и в обществах <<цивилизованном>> и <<варварском>> (кавычки уместны, потому что ни тот, ни другой термин не отражают существа явления). В <<цивилизованном>> обществе за первое нарушение закона, скорее всего, мягко пожурят (или там по попе нашлепают). А в обществе <<варварском>> первое же нарушение закона вполне может стать последним: прирежут\dots

Тем не менее, тысячелетия своей истории человечество существовало в <<варварских>> условиях. И ничего, выжило\dots А отдельные индивидуумы и по сию пору неуютно чувствуют себя в условиях цивилизованных.

Аналогичный случай и с дистрибутивами. Конечно, большинство пользователей начинает знакомство с Linux'ом с чего-либо юзерофильного~--- результаты опросов и личные наблюдения позволяют предположить, что обычно в этой роли выступает Red Hat (ныне Fedora) или Mandrake (\textit{ныне Mandriva}). И, опять-таки, в большинстве случаев это оправданно: Windows-подобие таких систем позволяет отодвинуть постижение законов POSIX-мира (а чтение документации, как уже было сказано, один из них) на неопределенный срок.

Однако всегда находились и находятся индивидуумы с ярко выраженными наклонностями к экстриму. И вот для них-то вполне приемлемым вариантом первого выбора может оказаться Linux-дистрибутив типа Gentoo или, скажем, FreeBSD. Только нужно помнить: в этом случае никакого снисхождения к их неопытности от окружающего мира им ждать не приходится. И законы его придется постигать сразу. Может быть, это не так уж и плохо?

\section{Нужны ли Linux'у пользователи?} 
\begin{timeline}
Февраль 2006
\end{timeline}

\textsl{Эта заметка написана мной совместно с Владимиром Поповым, и публикуется с его разрешения.}

\hfill \begin{minipage}[h]{0.45\textwidth}
Есть люди, умеющие пить водку, и есть люди, не умеющие пить водку, но все же пьющие её.

И вот первые получают удовольствие от горя и от радости, а вторые страдают за всех тех, кто пьет водку, не умея пить её.
\begin{flushright}
\textit{Исаак Бабель}
\end{flushright}
\bigskip\end{minipage} 

Время от времени на многих форумах поднимается вопрос~--- как увеличить число пользователей Linux? Какие меры нужно принять, чтобы претворить в жизнь лозунг <<Linux~--- на каждый десктоп!>>? Правда, в результате всех этих обсуждений возникает вопрос встречный: а нужно ли поголовное внедрение Linux? Кого нужно на Linux перетягивать? И, главное: кому нужно это перетягивание?

Чего ради тех, кто давно и осознанно работает в Linux (и прочих BSD) <<по жизни>>, должно волновать число пользователей этих ОС? Число разработчиков~--- другое дело. Чем их больше, тем больше шансов на появление чего-то интересного и принципиально нового. То Интернет придумают, то цифровое сжатие видео. Или, между делом, сломают защиту DVD: просто так, для прикола. Этим доплачивать нужно~--- правда, способов эффективного дотирования обществом разработчиков Open Source до сих пор не придумано. Как, впрочем, и способов поддержки любой общественно-полезной (и даже необходимой) деятельности, не влекущей за собой коммерческой выгоды.

Число интересующихся~--- тоже небезразлично: а вдруг среди них будущий Торвальдс? Этим надо помогать, даже ценой потери собственного времени. Да и веселее с ними\dots

А число пользователей, у которых нет большего интереса, как мышкой поелозить\dots Их увеличение ведет только к росту сообщений на форумах: <<так не правильно! Я так не хочу/не понимаю/не буду>>. И с чего они взяли, что кто-то обязан повышать комфортность их <<пользования>>?

Чтобы осознать абсурдность приведенного выше лозунга, достаточно очертить круг тех, кто не может стать пользователем Linux.

Это, во-первых, пользователи, категорически не способные к освоению компьютера (но, согласно известной максиме Бени Крика, все-таки его использующие). И это~--- отнюдь не признак их глупости, а, как и способность к питию водки, просто индивидуальная особенность: есть же люди, не отличающие ямба от хорея и \textit{до} от \textit{фа}. Так и здесь: нам известно немало пользователей с полуторадесятилетним стажем, так и не освоивших запись на дискету или отключение показа непечатаемых символов в Word. Вынужденные работать на компьютере, они, продолжая словами Бени, страдают: и за себя, и за всех тех, кто на компьютере работать не умеет. И Linux только усугубит их страдания.

Далее, из числа пользователей Linux следует исключить тех, кто испытывает идиосинкразию к чтению~--- а таких, увы, становится все больше даже в нашей стране, некогда бывшей самой читающей в мире. Потому что если в Windows (и тем более на Маке) кое-какие полезные навыки можно получить методом научного тыка, то в Linux без чтения документации и, возможно, даже толстых книг, обойтись практически невозможно.

На Linux никогда не перейдут запойные игроманы и те, кто использует компьютер исключительно в качестве развлекательного центра. И причины понятны: игр под Linux катастрофически мало, и нет в нем ничего, что оправдывало бы смену ОС для домашней аудио- и видеостанции.

Из пользователей-креативщиков вербовать сторонников Linux также в большинстве случаев бессмысленно: работа в нем профессионалов по созданию мультимедийного контента или спецов высокой полиграфии будет попросту неэффективной. Нет в нем и инструментов для работы профессионального художника.

Так кто же остается в сухом остатке, кроме разработчиков софта и профессиональных администраторов компьютерных сетей? Да всего-навсего одна категория пользователей: те, для кого по долгу службы (или велению души) важна эффективность работы с текстовым контентом, дополняемая коммуникационными возможностями. Обработка текстов и коммуникации~--- это то, для чего создавался UNIX, и это~--- сферы приложения его наследника как пользовательской среды. И именно креативщики-текстовики (в любой области~--- от технических писателей и научных работников до поэтов и писателей просто) имеют возможность использовать инструменты UNIX и Linux эффективно. Остается только продемонстрировать им эту эффективность~--- в чем авторы и видят главную задачу линуксописательства.

Интересно, что среди <<действующих>> пользователей Linux весьма высок процент профессиональных юристов и переводчиков. И тому можно видеть две причины. Во-первых, и те, и другие, безусловно, входят в сословие текстовиков-креативщиков. А во-вторых, и для юристов, и для переводчиков более, чем для остальных представителей этого сословия, важны аспекты легальности используемого ими софта.

Так давайте же не будем пропагандировать Linux среди тех, кто по любым причинам не может использовать его эффективно. Сэкономив время и силы для популяризации его тем, кому он действительно может быть полезен\dots

\section{Еще раз о Linux'е и его пользователях} 
\begin{timeline}20 марта 2007\end{timeline}

События последних месяцев~--- дело Поносова, круглый стол в Росбалте, выступление Медведева, наконец, круглый стол, организованный ЛДПР,~--- привели к тому, что слово Linux достигло слуха многих и многих людей, прежде и не подозревавших о существовании какой-либо операционной системы помимо Windows, а то и операционной системы вообще. Еще большее внимание привлекло к себе явление Open Source~--- как практически значимая в наших условиях возможность не разделить участи Поносова.

Рост информированности общества в отношении Linux и Open Source (хотя бы на ознакомительном уровне) нашел наглядное отражение в посещаемости ресурсов этой тематики. На протяжении текущего года она неуклонно возрастает, что особенно показательно для <<сайтов средней руки>> с точки зрения популярности. Ибо на сайтах наиболее популярных относительная картина не столь отчетливо просматривается вследствие высоких абсолютных цифр, а сайты-аутсайдеры как никем не посещались, так и не посещаются. С этой точки зрения типичным примером можно считать динамику посещений сайта \url{http://posix.ru}~--- про него я точно знаю, что его администрация не прикладывала никаких усилий к росту популярности.

При этом на всех сайтах рассматриваемой тематики, статистика которых доступна, существенно возросла доля заходов с поисковых машин относительно таковых с <<соплеменных>> ресурсов. И, пожалуй, впервые запросы от отечественных поисковиков типа Яндекса, Рамблера и Апорта приблизились по количеству к запросам от Google. А это уже прямой показатель роста именно <<непрофильных>> посетителей, так как Google безраздельно первенствует среди <<действующих>> линуксоидов. Глагол <<гуглить>> (\href{http://en.wikipedia.org/w/index.php?title=Google_(verb)\&oldid=582812741}{to google}) прижился у них задолго до того, как был утвержден в качестве официальной нормы английского языка.

Возникает вопрос~--- а не воспользоваться ли благоприятным моментом и не развернуть ли революционную агитацию в пользу Linux и Open Source с целью резкого наращивания пользовательской массы открытого софта? И если да~--- то среди кого и в какой форме?

Более года назад мы с Владимиром Поповым написали статью~--- <<Нужны ли Linux'у пользователи?>>. В которой выразили свое единодушное отношение к этой проблеме: Linux'у нужны не пользователи вообще, а пользователи совершенно определенного склада. В этой заметке я несколько конкретизирую данный вывод с учетом изменившихся за год реалий.

Для начала рассмотрим, кто мог бы потенциально составить пользовательскую базу Linux'а. Я разделил бы их на три группы, в порядке убывания стимулов к изучению системы и её использованию.

Первая группа~--- это те, кто хочет изучать Linux вообще, <<от сих до сих>>. В том числе и для использования в своей практической работе~--- но не обязательно: исходным стимулом может быть просто любопытство и стремление дать тренировку мозгам (своего рода <<интеллектуальный преферанс>>, по выражению Владимира Попова). Тем не менее, в основном пользователи первой группы профессионально связаны с IT-сферой, будучи разработчиками программного обеспечения или системными администраторами. И основным стимулом для изучения Linux'а у них будет стремление к повышению своей профессиональной квалификации и расширению кругозора~--- даже в том случае, если на практике они Linux будут использовать ограничено или не использовать вообще. Эту группу можно назвать энтузиастами.

Отдельную прослойку в составе энтузиастов составляют те, кто еще только готовится к деятельности в IT-сфере или не вполне определился в своих устремлениях. Именно среди них наиболее высок процент <<интеллектуальных преферансистов>>.

Вторая группа~--- это пользователи, которые готовы изучать Linux в тех пределах, в каких необходимо для использования в своей практической деятельности, с IT-сферой никак не связанной. Рост эффективности последней для них и является основным стимулом для изучения. А стимулы для использования могут быть весьма разными: от разумного стремления сэкономить на софте до необходимости применять только легальное программное обеспечение. А многие из них на собственном опыте убеждаются, что использование Linux'а действительно более эффективно для выполнения их задач.

Соответственно, и пределы необходимого минимума знаний для представителей этой группы оказываются разными. Кому-то будет достаточно освоить текстовый редактор, браузер и программу для работы с почтой, иному же потребуется владение комплексом UNIX-утилит и навыки шелл-скриптинга, а возможно, даже и всамделишнего программирования. Тем не менее, они не ставят себе целью превратиться в профессиональных админов или разработчиков, и в познании системы готовы ограничиться только тем, что нужно для их практической работы. И потому таких пользователей резонно назвать прагматиками.

Профессиональный состав второй группы очень пестр, и объединяют её два момента: работа в сферах, непосредственно не связанных с IT, и преобладание задач, связанных с обработкой текстов (в самом широком смысле слова) и коммуникациями любого рода. В основном это представители так называемых творческих профессий. Среди моих личных знакомых есть юристы, переводчики, журналисты, научные работники, некомпьютерные инженеры. А потенциально в числе этой группы я вижу и поэтов с писателями\dots

Третья группа~--- это те, кто не хочет ничего использовать и ничего изучать, и сам факт существования компьютеров и операционных систем им глубоко параллелен. А будучи по жизни вынужденными на компьютерах работать (что поделать, такова принудительная сила реальности), они будут использовать все, что угодно: то, что поставил приятель дома или сисадмин на работе, то, что приказало начальство, или, наконец, просто то, что использует большинство людей вокруг. Эту группу пользователей можно назвать попутчиками.

Эту категорию называют обычно секретаршами-блондинками, хотя с не меньшим основанием в нее можно включить брюнетов-чиновников и шатенов-менеджеров, а также вполне рыжих и даже лысых деятелей науки, культуры и искусства, а также руководящих работников разного ранга. Короче говоря, за небольшими исключениями, почти все остальное человечество, вне зависимости от пигментации волосяного покрова и радужной оболочки глаза, расовой, национальной и конфессиональной принадлежности.

Что же, почти все человечество~--- это потенциальные пользователи Linux'а, спросите вы меня. Я на этот вопрос я отвечу положительно. С одной лишь оговоркой~--- почти все практически работающее человечество, выполняющее производственные задачи при помощи компьютера. Подобно тому, как любая кухарка могла бы управлять государством (под наблюдением комиссара, разумеется), так и любая секретарша способна вести свое делопроизводство в Linux'е. Только вместо комиссара ей потребуется сисадмин, который ей этот Linux установит и должным образом настроит. К ней же никаких претензий по уровню компьютерной подготовки предъявляться не может в принципе~--- за исключением практических навыков работы с теми программами, которые нужны для выполнения своих служебных обязанностей.

В том, что большинство представителей вида Homo Sapiens ничего не знают о компьютерах и операционных системах, нет ничего ни плохого, ни страшного: деятельность человеческая многогранна, и интересы и амбиции большинства людей лежат вне IT-сферы. И это вовсе не значит, что операционная система (в данном случае Linux) для массового использования должна быть адаптирована до уровня понимания <<некомпьютерным>> человечеством: достаточно, если её будут понимать те, кто ему её устанавливает. А кто они?~--- да те самые представители первой группы пользователей, о которых речь шла выше. Благо, если в управленческой модели товарищей Ленина и Бухарина комиссара требовалось приставить к каждой кухарке, да еще и снабдить его надлежащим инвентарем в виде именного революционного маузера, то одного пользователя-первокатегорника вполне хватит на многие десятки, а то и сотни представителей категории третьей. В скобках замечу, что представители второй категории, как правило, вполне в состоянии комиссарить (пардон, админить) сами себя\dots

Конечно, говоря о всем трудящемся человечестве как потенциальных пользователях Linux, я несколько преувеличил. Для весьма значительной его части эта платформа неприемлема хотя бы потому, что под нее отсутствуют требуемые им приложения. Или просто на других платформах их задачи решаются эффективнее. Впрочем, эта тема подробно рассмотрена в указанной выше статье. 

Но есть и пользователи, которым использование Linux просто противопоказано. Хотя именно они часто рассматриваются как потенциальные пользователи этой ОС. И, главное, сами себя они позиционируют в этом качестве, почему их и придется выделить в отдельну, четвертую, группу.

Четвертая группа потенциальных пользователей весьма незначительна численно, но чрезвычайно активна. Это те, кто хочет использовать Linux, не желая его изучать вообще. Даже в тех ограниченных пределах, в каких это считают необходимым пользователи второй группы. Мотивация их для применения Linux бывает разной, но наиболее частыми причинами выступают две:
\begin{enumerate}
	\item <<Linux~--- это круто>>;
	\item <<Linux~--- это бесплатно>>.
\end{enumerate}
Представители четвертой категории, не смотря на свою малочисленность и эфемерность статуса Linux-пользователей (<<детская болезнь крутизны>> проходит с возрастом, не обязательно биологическим, а стремление к халяве быстро разбивается о скалы реальности), представляют собой весьма активную и заметную со стороны часть сообщества (по крайней мере на просторах Рунета). Именно они являются главными ревнителями идеи <<Linux'а для каждой кухарки>>~--- то есть адаптации системы до уровня, доступного пониманию того, кто о ней ничего не знает и знать не хочет. Их лозунг~--- <<Все, как в венде, но бесплатно>>. Идеал настройки при этом~--- единственная кнопка с надписью на албанском языке \textbf{Зделать пес\dotsто}
 (да простят меня читающие это дамы, если по русски звучит не вполне нормативно~--- но суть дела передает адекватно).

Ну и разумеется, не обходятся они и без выражения претензий ко всему остальному сообществу: почему это оно, такое и разэтакое, не торопится, что бы для лично для меня, любимого, все было\dots ну тем самым, албанским, образом. Разумеется, забесплатно. И очень обижаются, когда им в самой деликатной форме, без обращения к известному сайту символического направления и генеалогических исследований по женской линии, пытаются объяснить всю беспочвенность их притязаний. Если же в ответах на свои вопросы наш <<четвертичник>> углядит хоть малейшие следы неполиткорректности~--- возмущение его можно будет сравнить только с протестами советского народа против угнетения шахтеров Уэльса.

За пользователями четвертой группы в народе закрепилось название <<красноглазых>>. Однако это не вполне адекватное обозначение: фанатично настроенные граждане не редки и среди энтузиастов, особенно в младшей возрастной категории; правда, среди них фанатизм проходит со временем. Пользователей четвертой категории я бы назвал вопрошателями, ибо наиболее характерные их черты~--- нежелание читать что бы то ни было и любовь к задаванию вопросов на форумах, ответы на которые они читают лишь затем, чтобы найти повод задать следующий вопрос.

Разобравшись с тем, что представляют собой потенциальные пользователи Linux'а, можно и определиться с вопросом~--- как этот самый Linux пропагандировать и среди кого за него агитировать.

Очевидно, что энтузиасты в агитации не нуждаются: слово Linux они, как правило, знают, как и представляют сферу его применения. И либо уже применяют его в своей деятельности, либо готовятся к применению (в частности, посредством изучения и экспериментов в области <<интеллектуального преферанса>>). Либо~--- никогда применять не будут, по тем или иным причинам (отсутствию подходящих задач, служебной привязанности к более иной платформе, личной антипатии, наконец).В любом случае, энтузиасты нуждаются не в пропагандистских, а в чисто информационных материалах~--- благо, как правило, искать информацию они умеют или очень быстро учатся.

А вот потенциальные прагматики могут просто не слышать о существовании Linux'а. И, более того, не знать о том, что текст можно набирать не только с MS Word, а в Сеть выходить чем-то, отличным от Internet Explorer. Что поделать, 12 лет пропаганды под <<Веселым Роджером>> сыграли свою роль в деформации пользовательских представлений. Но и тут речь идет не столько о пропаганде и агитации как таковой, сколько о популяризации знаний о Linux, Open Source, сфере и методах их применения.

Попутчикам, как правило, достаточно донесения самого факта существования Linux~--- да и то, лишь при наличии их доброй на то воли, потому как они спокойно обойдутся и без этого. Более того, их по возможности следует оберегать от внедрения излишней информации. Потому что это либо вызовет раздражение (кому понравится навязывание знаний о вещах совершенно неинтересных), либо превращение части попутчиков в вопрошателей.

Ну и, наконец, вопрошателей не нужно ни агитировать, ни пропагандировать. Увы, как раз наоборот, сплошь и рядом приходится противодействовать их агитации за <<Всенародный Linux>> и пропаганде представлений о том, как он должен выглядеть. И в который раз повторять нехитрую истину, высказанную в свое время Страшилой Мудрым:
\begin{shadequote}{}
Река это не дорога, а дорога~--- это не река.
\end{shadequote}
А то ведь еще не достаточно информированные потенциальные прагматики могут и впрямь поверить, что по реке можно идти, аки посуху\dots

\section{И снова о массовом Linux'е} 

\begin{timeline}18 октября 2007 г\end{timeline}

Эта заметка родилась, с одной стороны, как своего рода полудополнение к ряду статей Владимира Попова. С другой~--- как следствие чтения многочисленных комментариев к иным статьям (в том числе и не имеющим никакого отношения к <<Десктопному Линуксу>>. Немало материалов к этой теме дают и обсуждения на форумах. С цитаты из одного из таких обсуждений я и начну. 

\hfill \begin{minipage}[h]{0.45\textwidth}
\dotsкомпьютер, который призван облегчать решение различных проблем, пока их не меньше добавляет
\begin{flushright}
\textit{Vityaz}, \texttt{unixforum.org}
\end{flushright}
\bigskip\end{minipage}

Поскольку цитата вырвана из контекста, который занял бы слишком много места, дополню, что в ней подразумеваются проблемы, которые компьютер создает пресловутому конечному пользователю~--- эпическим персонажам современного фольклора: секретарше-блондинке, брюнету-ченовнегу и шатену-манагеру. Ах да, еще и тётке-бухгалтеру. 

Так ли это? Давайте не будем лукавить и признаем, что~--- да, это так. Причем, что интересно, он создает им проблемы вне зависимости от аппаратной платформы и операционной системы на ней: и в Windows, и Linux, и в любой BSD. Даже в DOS'е черном, помнится, создавал. Почему? 

Ответ прост. Дело в том, что компьютеры не создавались для облегчения жизни кого бы то ни было~--- ни секретарши, ни манагера, ни чиновнега. Ни даже бухгалтера. Они создавались для обеспечения систем противовоздушной обороны, расчета траекторий баллистических ракет и космических кораблей, прогноза движения теплых и холодных воздушных масс с Атлантики, обеспечения отказоустойчивых коммуникаций\dots Иными словами, для решения многих и многих задач, которые без компьютеров решены бы не были. 

Это относится к компьютерам вообще. И сфера применения компьютеров, как неоднократно подчеркивал Владимир, неизмеримо шире того её сегмента, который ассоциируется с этим словом у большинства трудящихся~--- то есть сегмента настольных персоналок, каковые в настоящее время свелись к архитектуре i386 (PC или, в просторечии, <<писюки>>~--- ибо ныне и гордые Mac'и архитектурно являются ничем иным, как <<писюками>>). 

В отличие от Владимира, я не застал героических времен PDP-7 сотоварищи. То есть застать-то застал, но от компьютеров был очень далек. И мое знакомство с продукции IT-индустрии началось с персоналок~--- IBM PC/XT-совместимых, с процессором Intel 8086/8088, работавших под управлением <<черного>> DOS'п. Которые я полюбил всей душой. Потому что они позволяли мне сделать (в моей тогдашней профессиональной сфере) то, чего я не мог рассчитать посредством логарифмической линейки, арифмометра~--- <<железного феликса>>, и даже программируемого калькулятора со встроенным Basic'ом. Причем сделать, не обращаясь во всякого рода ВЦ, каковые обращения требовали немалых бюрократических усилий. Для меня это был чрезвычайно эффективный инструмент работы. А отнюдь не бытовой прибор, создающий удобства. 

Так что даже и персоналка изначально не предназначалась для упрощения жизни бухгалтерши Надежды Александровны, секретарши Люси, манагера Пети или ченовнега Васи. А предназначалась она для того, чтобы инженер Ваня, биолог Костя или геолог Лёха могли с его помощью сделать то, чего без него сделать не могли. Например, рассчитать статистику соотношения легких редкоземельных элементов и высокозарядных катионов для всех проб глубоководного бурения в Южной Пацифики и сравнить её со статистикой их же в экзотических террейнах Северо-Востока СССР. 

Справедливости ради надо заметить, что вышеуказанные Ваня, Костя или Лёха не забывали и о нуждах Надежды Ивановны или Люси. А также кадровички Нины Егоровны. Они устанавливали на их машины все тот же самый <<черный>> DOS и программу, необходимую для соответствующей работы~--- бухучета, учета кадров (тогда их не сочинял только ленивый). Ну а для Люси~--- разумеется, Lexicon. И писали простенький batch-файл, позволяющий запустить эту программу нажатием одной клавиши. 

Все были довольны: наши герои спокойно занимаются своей работой, Люся печатала на принтере из Lexicon'а приказы и распоряжения (причем все больше и больше, не смотря на призывы к внедрению безбумажного документооборота), Нина Егоровна учитывала кадры (которых, напротив, становилось все меньше и меньше), а Надежда Александровна начисляла им всем зарплату~--- правда, проверяя результаты работы компьютера на счетах, как она привыкла это делать еще при появлении настольных калькуляторов. 

А потом произошло классовое расслоение. Ваня, Костя и Лёха постепенно мигрировали на Linux~--- потому, что там они свою работу могли выполнять намного эффективней, чем в <<черном>> DOS'е или на PDP-7. Подчеркну при этом, что они были и оставались самыми обычными пользователями. А вот Надежда Александровна, Нина Егоровна и Люся почему-то стали работать в Windows~--- причем отнюдь не по собственному желанию. И вi думаете, как говорят в Одессе, эффективность их работы повысилась? Фиг с маслом~--- ответит вам любой одессит (конечно, немножко другими словами). Ибо у них тут же появилось множество новых забот: как им слушать музыку, смотреть кино, ставить и запускать новые игрушки. И выяснять, почему после установки очередной игрушки их система развались напрочь. Разумеется, работа при этом не делалась ни лучше, ни быстрее. 

И столь же очевидно, что всеми этими <<настроечными>> мероприятиями героини занимались не сами. Ибо не было у них ни должной подготовки, ни желания её приобрести (это отнюдь не в упрек, ни один человек не может знать и уметь все). И для начала они решили припахать к этому делу наших героев~--- не учтя того обстоятельства, что они были героями отнюдь не их романа. И что им по большому счету более или менее безразлично, будет ли нашим дамам за компьютером столь же комфортно, как в прокладках Тампакс. Причем опять-таки вне зависимости от целевой платформы~--- Windows ли это любого рода, или Linux, или даже FreeBSD. Потому как сфера их интересов лежала в области эффективности работы, а не в области комфорта. 

Вот, собственно, и вся коллизия. Суть которой~--- в несовпадении интересов тех пользователей, для кого компьютер являет собой рабочий инструмент, и тех, кто рассматривает его как бытовой прибор. Причем это несовпадение не связано ни с пигментацией волос, ни даже с профессиональной принадлежностью. Среди пользователей первой группы мне встречались не только научные работники и инженеры, но также переводчики, юристы, и даже~--- страшно сказать~--- ченовнеги и секретарши. А уж группа вторая~--- охватывает весь спектр социальных слоев и профессий нашего общества. 

Так что пользователи первой группы просто скажут спасибо разработчикам открытого софта~--- тем, которые обеспечили им среду для решения своих задач относительно малой (финансовой) кровью. И не так уж им интересны усилия маркетологов, цель которых~--- навязывание ненужного товара путем искусственного создания несуществующей потребности в нем. Ибо они прекрасно понимают, что ни к чему хорошему это не приведет. 

Я заранее предвижу все возможные комментарии на это сочинение: 

\begin{itemize}
	\item и про то, что компьютер действительно стал бытовым прибором; 
	\item и про драйверы, изобилия которых не добиться без того, что Linux станет массовой системой;
	\item и про вселенскую угрозу, исходящую от монополии Microsoft; 
	\item и всё остальное~--- поверьте, все возражения против изложенной выше точки зрения уже были высказаны много лет назад. 
\end{itemize}

И ответы на них были даны столь же давно~--- как в публикациях моих коллег и единомышленников, так даже и в моих. Ответы, для нас вполне весомые, хотя и не универсальные. 

И потому не следует думать, что каждый пользователь Linux спит и во сне видит, как бы ему соблазнить, охмурить или сагитировать очередного вендузяднега. Нет, он, как правило, в основном занимается своим делом, а в свободные минуты предпочитает общаться в своем кругу~--- среди тех, кто разделяет его интересы. 

Так что оставим соблазнение\dots ну это интимный вопрос, замнем для ясности, охмурёж~--- ксендзам, а агитацию~--- партийным активистам. 

\section{Заметки сноба, или Беру свои слова обратно} 

\begin{timeline}21 июня 2007 г\end{timeline}

Вот уже скоро восемь лет, как я пишу о Linux'е и Open Source~--- первые заметки сочинялись для онлайна и бумаги в 1999 году. Юбилей, однако,~--- двоичное тысячеление, в масштабах времени индустрии вполне можно приравнять к веку. И посему решил я, древний старец-линуксописатель, подвести некоторые итоги былого.

Время моего первичного приобщения к Linux'у выпало на 1997-1999 годы. Тогда это было не так просто, как в последствии,~--- требовалось прикручивать руками русский язык в консоли, подбирать X-сервер, способный запуститься на данной видеокарте (а видеочипов в те времена было не в пример больше, чем нынче, и каждый имел свой собственный X-сервер, и не фактом было, что они идеально подходили друг другу, как герои-любовники из сентиментальных романов). А уж настройка видеорежимов или русификация тех же Иксов казались высшим пилотажем.

И с источниками информации была напряженка. Не то что нынешнего изобилия русскоязычных сайтов и книг не предвиделось и в помине~--- англоязычные сайты тоже не попадались на каждом шагу, а книги, при желании, заказать можно было~--- хоть у самого товарища О'Рейли, но, пардон, за длинный и зеленый, как крокодил Гена, додефолтный бакс, во-первых, и при практической невозможности этот бакс ему оттранспортировать,~--- во-вторых. Ибо красных дипкурьеров тогда уже не было, а современных платежных систем~--- не было еще.

Разумеется, уже тогда были \mbox{\texttt{man}'ы}~--- на аглицком, конечно, но это казалось преодолимым даже теми, кто, подобно автору этих строк, всю начальную часть своей жизни учил немецкий (да и из него к тому времени забыл даже то, чего никогда не знал). Ибо не Бернардом Шоу были писаны, и не Оскаром Уайльдом, а теми, кто английский, возможно, знал не намного лучше русскоязычных читателей. И потому были просты для понимания.

Но много ли мог помочь стандартный man в ситуации, когда Иксы категорически не хотят русифицироваться? И не хотят потому, что в одном из файлов исходников была мелкая банальная ачипятка, ни на что более не влияющая. Её выискал в свое время Иван Паскаль, за что честь ему и хвала от сообщества. Правда, ну кто нонче помнит ту историю? А подобных историй в те времена были не одна и не две.

Google, кстати, в те времена только зарождался, и до нынешней славы ему было далеко. А найти что-то в популярных тогда поисковых машинах типа Altavista и других, ныне забытых, было не самой легкой задачей, да и релевантность оставляла желать лучшего.

Не блестяще было и с Интернетом~--- не только с домашним, где царил сплошной dial-up, часто через АТС не то что не первой, но даже и не второй свежести, но и на выделенных служебных или фирменных линиях.

Не было, кстати, и онлайновой торговли дистрибутивами в ассортименте \url{http://distrowatch.com}, который мы нынче можем видеть, например, на Линуксцентре. Методов приобретения таковых было, по существу, три:

\begin{enumerate}
	\item у пиратов~--- исключительно на заказ и на <<золоте>>, а потому очень дорого;
	\item в больших книжных магазинах Москвы (и возможно, Питера)~--- в фирменном исполнении Walnut Creek или InfoMagic, то есть ничуть не дешевле;
	\item привоз знакомыми из-за Бугорщины~--- это уж как договоришься.
\end{enumerate}

О скачивании дистрибутивов бочками (пардон, пачками) трудно было помыслить не только в домашних, но и в служебных условиях.

В общем, как говорили встарь, было трудно, но интересно. Может быть, именно интерес и позволял пользователям моего поколения успешно прорываться сквозь все препоны и рогатки. Причем не только профессиональным программистм, админам и вообще айтишникам, но и лицам с вполне гуманитарным или полугуманитарным образованием.

Спрашивается~--- а на хрена им это все было надо? Ну уж не в игрушки резаться, или там домашние кинотеатры организовывать. А в первую очередь ради повышения эффективности работы. Для кого-то это была изощренность в методах обработки текстов, для других~--- развитые коммуникационные функции, для третьих~--- возможность решения нестандартных счетных задач. Да мало ли областей, в которых UNIX и Linux обеспечивали максимальную эффективность с минимальными затратами сил и средств\dots

Именно в то время я и занялся линуксописательством. Во-первых, для себя, любимого: описать предмет~--- один из самых эффективных способов в нем разобраться (особенно если описание предмета выполняется его же средствами).

Во-вторых, для других~--- смею надеяться, что некоторые из моих сочинений помогли кому-нибудь не наступать на те же грабли, что и я.

В-третьих, и главных~--- опять же для себя: мне это было просто интересно.

И в своих писаниях я всегда исходил из следующих посылок:
\begin{enumerate}
	\item всякий, кто берётся осваивать Linux, нуждается в его функциях,
	\item готов затратить время и силы на их освоение, и
	\item испытывает интерес к предмету освоения.
\end{enumerate}
Эти три условия казались мне необходимыми и достаточными, чтобы самый обычный человек с произвольным складом ума и любой начальной подготовкой, сводящейся к функциональной грамотности, смог бы Linux освоить в пределах, необходимых для решения его задач. И, подобно герою Александра Галича,

\begin{shadequote}{}
\dotsповсюду, и устно, и письменно,\\
Утверждал я, что все это истина. 
\end{shadequote}
Ныне я беру свои слова обратно: нет, Linux~--- это не система для обычных (конечных, они же, по выражению Владимира Попова, <<чисто вымытые>>) пользователей.

Что заставило меня изменить свое мнение? Да все те же многочисленные обсуждения на форумах. Которые показывают, что обычному пользователю не по силам читать документацию даже на русском языке, не говоря уже об английском. Что обычному пользователю во сто крат легче сутками ждать ответа на свой вопрос, нежели набрать два-три слова в поисковой строке Google. А уж о том, что для получения некоторых базовых знаний достаточно прочитать какую-нибудь книжку по теме, а не задавать вопрос, ответ на который потребовал бы пересказа таковой~--- это нашему обычному пользователю и в голову не придет.

И лейтмотивом всех вопросов такого, ныне обычного, пользователя, является~--- <<сделайте мне пес\dotsто>>. Чего бы этот вопрос не касался~--- установки ли Мандривы или настройки VPN, налаживания домашней локалки или развертывания корпоративного сервера.

И это при том, что в современных дистрибутивах при установке на типовые конфигурации практически не возникает проблем с <<железом>>.

При том, что многие дистрибутивы (и не обязательно отечественные) корректно русифицированы <<из коробки>>, а о русификации остальных написано столько, что приходит на память <<Война и мир>>.

При том, что ныне в Сети описано решение 99\% всех вопросов, которые только могут возникнуть перед начинающим пользователем. И не менее 90\\% ответов на них дается по русски.

При том, что каналы и тарифы, по крайней мере в нескольких больших городах, позволяют качать образы не только дистрибутивных CD, но и DVD, да еще и не единожды.

При том, что оставшимся обладателям плохих или дорогих каналов доступен заказ дисков в онлайновых магазинах с доставкой в любую точку России (и некоторых сопредельных стран) за более чем разумные деньги.

При том, наконец, в больших книжных магазинах полки с литературой по Linux и UNIX вплотную приблизились по насыщенности к стеллажам детективов, да и системы онлайновой книготорговли ими не оскудели.

Так что не нужен обычному пользователю Linux. Это~--- система исключительно для мрачных бородатых снобов, как выразился один из посетителей одного из форумов. Поскольку сказано это было в мой адрес~--- принимаю себя за эталон такового: да, мрачен изначально; да, благоприобретенно бородат; ну и безусловно сноб, ибо снобизм свой лелею, подобно герою песенки Тимура Шаова (точнее, начал лелеять, когда Тимурчик еще пешком под стол ходил).

Утешает одно~--- что среди моих личных и виртуальных друзей и знакомых таких снобов~--- подавляющее большинство, если не все вообще. Остается только выявить их отличительные признаки.

Так, например, мрачность~--- отнюдь не непременный атрибут сноба: многие из тех же знакомых являют собой очень жизнерадостных людей.

Тем более~--- борода также не является снобоопределяющим признаком: в круге моего общения многие аккуратно бреются, другие~--- не бородаты ввиду младости лет, иные же~--- не могут ими быть в силу принадлежности к прекрасному полу. Что, тем не менее, не мешает им всем быть снобами.

Давайте выясним, почему.

Первая отличительная особенность снобов~--- умение читать. Причем все подряд: выводы команд, сообщения об ошибках, документацию во всех её проявлениях, материалы онлайновых ресурсов, книги\dots А некоторые из снобов, особо возомнившие о себе, читают даже журнал <<Линуксформат>>\dots Короче, как сказали бы классики советской фантастики:

\begin{shadequote}{}
Грамотен~--- сноб.\\
Книжки читаешь~--- сноб.\\
Команду man знаешь~--- сноб, слишком много знаешь. 
\end{shadequote}
Далее, снобы не просто умеют читать~--- они обладают способностью отыскивать предмет чтения, и, как правило, не ленятся это делать.

Наконец, последнее~--- снобы и чтением, и поиском объектов чтения занимаются с интересом. И, как правило, знают, зачем они это делают.

Таким образом, мы очертили круг снобов~--- то есть тех, для кого предназначен Linux. И потому каждый, пожелавший приобщиться к этой системе, должен для начала точно позиционировать себя~--- сноб ли он или из тех, кто желает, чтобы ему <<сделали пес\dotsто>>. Во втором случае ему лучше за освоение Linux'а не браться, ибо, как поется в известной песне,

\begin{shadequote}{}
Никто не сделает пес\dotsто~---\\
Ни Бог, ни царь и ни герой\dots 
\end{shadequote}
А вот прирожденным снобам судьбой предначертано~--- \texttt{man} в руки и за дело. И да пребудет с ними удача\dots

В связи с этим возникает резонный вопрос~--- а имеет ли в настоящих условиях какой-нибудь смысл линуксописательство? Ведь, как я уже говорил, 99\% всех проблем, могущих возникнуть перед начинающим пользователем, описаны, а те, что не описаны, связаны либо со специфическими задачами, либо с очень новым оборудованием.

Так что сноб легко найдет решение в ворохе наличествующей информации. А обычный пользователь все равно ничего ни искать, ни читать не будет~--- даже если решение дано в соседнем трейде посещаемого им форума\dots

\section{Начал ли звонить колокол?} 

\begin{timeline}1 августа 2007 г\end{timeline}

\hfill \begin{minipage}[h]{0.45\textwidth}
Вам, Lena и Виктория, авторам комментов на Цитките, посвящается
\bigskip\end{minipage}

Призывы внедрить, расширить, приобщить, подсадить, вовлечь и даже распропагандировать раздаются в отношении Linux'а уже много лет~--- грешен, и сам подчас говорил нечто подобное. Однако в последнее время это призывы, во-первых, участились. А во-вторых, приобрели весьма экстремистский характер~--- вплоть до лозунга полного вытеснения Windows с рабочих столов всех, в том числе и так называемых конечных, пользователей. Что сочетается с требованием <<десктопизации Linux'а>>~--- то есть доведения его до состояния, понятного среднестатистическому <<подоконнику>>, не желающему вникать в устройство системы ни в малейшей степени.

Добро бы, когда такие призывы раздавались бы со стороны фирм, связанных с разработкой, внедрением и распространением дистрибутивов Linux и вообще решений на базе Open Source. Это, во-первых, их бизнес, во-вторых, они действительно занимаются <<расширением, приобщением, углублением>> и тому подобными делами. Но как раз они-то и проявляют сдержанность в высказываниях на эту тему. Оно и понятно~--- им, как никому другому, ясна вся сложность процесса <<повсеместного перевода делопроизводства на латинский алфавит>>\dots то есть, пардон, пользователей на Linux.

Вполне ожидаемо, что лозунг <<Linux forever>> раздается из уст <<юношей бледных со взором горящим>>, подчас даже с рубиновым блеском в глазах. Это~--- обычный максимализм неофитов, которому в свое время отдал дань каждый из нас. И который проходит со временем~--- как только краткий этап первичного (и потому восторженного) приобщения к системе перейдет в хроническую стадию просто работы в ней.

Не вызывают удивления такие призывы и от тех, кто услышал о Linux'е только вчера и, скорее всего, в связи с пресловутым <<делом Поносова>> и сопряженными событиями: ими, при полном незнании того, что представляет собой Linux и Open Source (и часто, похоже, при столь же полном нежелании вникнуть в тему) движет элементарная боязнь того, что завтра придет ОБЭП и возьмет за~\dots ну за что там обычно берет ОБЭП? За финансовую документацию, вероятно\dots

Более странно, что призывы к поголовному переходу на сою (опять же, пардон, на Linux) нынче можно услышать от действующих линуксоидов, вполне успешно использующих эту систему в практических целях. И еще более удивительно, что призывы эти подчас исходят от представительниц прекрасного пола. Коим я, как джентльмен, не могу не ответить. Что и попробую сделать в настоящей заметке.

Правда, свою позицию по данному вопросу излагал неоднократно, и устно, и, в основном, письменно (в том числе и совсем недавно): Linux на каждом десктопе не нужен, и десктопизировать его не нужно также. Тем не менее, в последнее время отчетливо проявилось несколько тенденций, подкрепляющих мою точку зрения. Поэтому, рискуя повторить кое-что из сказанного ранее, изложу свое нынешнее представление вопроса.

Маленький дисклаймер: далее я буду ссылаться на некоторые дистрибутивы, как примеры особенно (или в меру) дружественных к пользователю. Это не значит, что среди всех остальных дистрибутивов не найдется столь же (а может быть, и более) дружественных. Просто те, которые я упоминаю, <<щупались>> моими руками относительно недавно.

Начнем с утверждения, что для повсеместного перехода на Linux требуется его десктопизировать еще больше, чем это есть в ныне существующих дистрибутивах юзерофильской категории.

Дистрибутив типа Ubuntu/Kubuntu разворачивается до полностью работоспособного состояния за полчаса, и, при наличии подключения к Сети, не требует от пользователя никаких дополнительных действий, даже русификации. Кроме разве что сугубо косметических~--- типа переопределения обоев рабочего стола. Ставшая притчей во языцех доустановка мультимедийных кодеков (страшно сложная операция, скажу я вам по секрету)~--- ныне в прошлом: при щелчке на мультимедийном файле система заметит, что кодек для его воспроизведения отсутствует. Но тут же любезно предложит скачать его и установить, после чего проигрывание выбранного ранее начнется тут же. Не это ли воплощение принципа~--- <<сделайте мне пес\dotsто>>? Того самого, о котором мечтали поколения пользователей, бросивших Linux из-за невозможности найти в консоли кнопку \textbf{Пуск}\dots

Ничуть не сложнее обстоит дело нынче и с Debian'ом: сейчас он, точно также как и вариации на тему Ubuntu, может быть установлен в графическом режиме на полном автомате, с одной из предопределенных рабочих сред (GNOME, KDE, Xfce) и сопутствующими им наборами приложений. И вообще, сравнение Ubuntu и Debian вызывает последнее время столько вопросов на форумах, что я решил посвятить этой теме отдельную заметку.

Не намного больше времени уйдет на установку Zenwalk, который не считается эталоном дружественности. И тем не менее и тут по окончании инсталляции пользователь также получает полностью работоспособную систему со скромным, но достаточным на первое время набором приложений (включая пресловутые кодеки <<из коробки>>). Чуть-чуть прихрамывает русификация~--- но её излечение требует только многократно документированной правки пары-тройки конфигов. 

Archlinux имеет славу дистрибутива, относительно сложного для начинающего, установка которого требует определенных познаний. Однако познания эти на практике сводятся к некоторому минимальному представлению о дисковых разделах, файловых системах и пользовательских аккаунтах.

Что, кстати, не лишне и при установке любого другого дистрибутива, даже самого юзерофильного. А понятие пользовательского аккаунта вообще одно из краеугольных в идеологии UNIX, и без него не обойтись и в дальнейшей работе (во избежание последующих криков <<не могу войти рутом>>, оглашающих просторы форумов).

Ну так вот, возвращаясь к нашим Archlinux'ам: при наличии указанных выше познаний (согласитесь, что требования отнюдь не чрезмерные), дистрибутив этот также может быть развернут исторически мгновенно~--- и в почти пригодном к употреблению виде. <<Почти>>~--- потому что русификация его потребует все же некоторых мануальных пассов. Но достаточно простых и многократно описанных.

Правда, все описанные дистрибутивы столько хорошо ведут себя при одном непременном условии~--- наличии хорошего (то есть быстрого и, главное, дешевого) доступа к Сети. В отсутствие оного их доведение до ума может доставить некоторые трудности. Правда, в большинстве случаев они решаемы~--- приобретением полных снапшотов пакетных репозиториев, например, или скачиванием в другом месте. За исключением, пожалуй, Zenwalk~--- его эффективное использование в чисто оффлайновом режиме представляется мне пока проблематичным. Но главное, при отсутствии сети ни один из перечисленных выше дистрибутивов не заиграет всеми своими красками. Кроме, разве что, Debian'а, одна из форм распространения которого лежит на фантастическом числе дисков.

Тем не менее, и проблемы с коннектом не смертельны~--- нужно только подобрать более подходящий для такой ситуации дистрибутив. Таковым, кроме упомянутого выше Debian'а, может стать Mandriva, особенно в своих коммерческих ипостасях~--- Discovery, Powerpack или Powerpack+, в зависимости от потребностей. Любой из этих вариантов может быть полностью укомплектован без доступа в Интернет вообще. Причем будет включать в себя не только полный набор приснопамятных мультимедийных кодеков, но и такие коммерческие программы, как CrossOver и Cedega. И драйвера для видеокарт окажутся не забытыми.

Есть подозрение, что большая часть того, что сказано о Mandriva, приложимо и к таким дистрибутивам, как Suse, или к клонам Red Hat (вроде Scientific Linux, например). Однако лично в последнее время я с ними не общался~--- и потому от категорических утверждений воздержусь.

Короче говоря, ни в какой дополнительной десктопизации Linux уже не нуждается~--- эту тенденцию можно считать полностью реализованной.

Вторая тенденция развивается полным ходом. Это~--- создание дистрибутивов, способных работать без инсталляции на компьютер. То есть~--- мобильных систем, которые пользователь мог повсюду носить с собой и запускать на любой машине. В форме LiveCD они существуют давно, однако использование их было ограничено самой природой носителя: для переноса пользовательских настроек и данных требовался еще и какой-либо накопитель, способный к записи, что не способствовало комфорту работы <<на стороне>>. Был, правда, Puppy, который, будучи размещенным на CD-RW, обладал способностью сохранять настройки, но для переноса на нем же данных и он не был приспособлен.

Ныне же началось развитие дистрибутивов на USB-флэшках, так называемых брелках (некоторые полагают, что надо говорить~--- на <<брелоках>>, но это позорно и преступно). Пока <<фабричным>> способом так выпускается только один дистрибутив~--- Mandriva Flash (варианты на 2 и 4 Гбайт). Но очевидно, что братья во юзерофилии от нее не отстанут, и в ближайшее время следует ожидать аналогов и от остальных любителей <<чисто вымытых>> пользователей. А пока~--- любому квалифицированному линуксоиду не составит труда изготовить таких флэшек со своим любимым дистрибутивом для друзей и знакомых. Не говоря уже о себе, любимом.

А дальше~--- больше: можно ожидать создания флэш-дистров (назовем это явление так) на разные случаи жизни, с разным десктоп-окружением и разными наборами прикладного софта, от рабочего места секретарши-блондинки до рабочей станции наиглавнейшего босса. Ведь на это не потребуется отдельных машин~--- того, что раньше называли программно-аппаратными комплексами, нет, все это будет функционировать на любой подручной PC'шке.

Конечно, пока использование флэш-дистров также ограничено~--- не сошли еще с рабочих мест машины, BIOS которых не предусматривает загрузки с USB-носителей. Однако не далек день, когда, имея такой брелок в кармане, можно будет гарантированно загрузить свою систему со своими же самыми необходимыми данным в любой точке земного шара, охваченной компьютеризацией.

Есть и второе ограничение применимости флэш-дистров: ни двух-, ни четырех-, ни сколько-угодно-разумно-гигабайтная флэшка не в состоянии удовлетворить потребности всех пользователей как устройство массовой памяти~--- хранилище потребных данных (хотя очень многим из пользователей и такие объемы по делу избыточны). И тут вспоминаем о третьей тенденции развития современного IT-мира~--- онлайновых сервисах.

И тут, конечно, первым делом на память приходит Google. Думаю, что подавляющее большинство читателей этой заметки активно используют его поисковую машину (хотя, говорят, некоторых на \url{google.com} забанили, а у иных же Google, бл\dotsин, сломался). Многие пользуются почтовой системой gmail (оно же гымыло в народе) как дублирующей, а кое-кто переводит её в ранг основной. Служба Blogger'а позволяет не только вести личные дневники, но и создавать практически полноценные контент-сайты (по крайней мере, не хуже большинства доморощенных <<хомяков>>). Ну а всякие там Picasa, Блокнот, Документы и таблицы~--- это уже в сущности средства (почти) полноценной работы в (почти) любых условиях. По крайней мере, доступ к данным и средства их обработки они обеспечивают.

При этом нужно учесть, что в области онлайновой работы Google~--- отнюдь не впереди планеты всей. В Сети можно отыскать несколько проектов развития практически полноценных офисных пакетов, предназначенных для работы в онлайне. Например~--- \url{http://zoho.com/}. Не то чтобы он выделяется в ряду прочих какими-то несравненными достоинствами~--- просто эта ссылка попалась мне под руку при сочинении настоящей заметки (за что спасибо Навигатору).

Сказанное выше перекликается с тем, что изложено в статье Пола Грэма <<Microsoft мертва>>. Однако значит ли это, что не сегодня, так завтра Linux полностью вытеснит Windows с рабочих столов, а Билл Гейтс отправится просить милостыню? Отнюдь. Логическим завершением описанных тенденций будет то, что понятие операционной системы для большинства конечных пользователей просто потеряет физический смысл. Действительно, какая разница, что за ОС грузится на данной машине, если все, что от нее, машины, требуется~--- это войти в Сеть и запустить привычный текстовый процессор или электронную таблицу на удаленном сервере. Каковые будут выглядеть и функционировать совершенно одинаково, вне зависимости от того, были ли они запущены через <<подоконный>> Internet Explorer, Linux'овый Konqueror или кросс-платформенный FireFox.

Будет ли внедрение онлайнового стиля работы способствовать вытеснению Windows с пользовательских столов? В какой-то мере~--- да. Разумеется, некоторое количество пользователей покинут эту платформу просто по финансовым соображениям: зачем платить за среду, единственной функцией которой является запуск нескольких онлайновых служб, с чем прекрасно справится и практически бесплатный Linux. Но масштабы этого явления преувеличивать не следует. Ведь и Windows более или менее справится с этой задачей, и уходить с нее помешает просто обычная инерция мышления~--- людей, которые предпочтут заплатить и забыть, не так уж и мало. Тем более, что при таком раскладе Microsoft будет просто вынуждена снизить цены на свою продукцию~--- причем не только на операционки, но и на офисные приложения. Вплоть до того, что официальные диски будут продаваться по вполне пиратским ценам\dots

Не стал бы я, в отличие от Пола, преувеличивать и роль Macintosh'ей в вытеснении Windows~--- по крайней мере, на Руси и в сопредельных странах. И тут причина банальна~--- стоимость. Если в Американщине цена на Маки практически сравнялась с чисто PC'шными брендами, то у нас почему-то до этого далеко. В чем причина, в таможенной ли политике или особенной жадности дистрибьюторов Apple,~--- не знаю и гадать не хочу, но факт остается фактом. И потому круг пользователей этой платформы особенно не расширится~--- тем более за счет интересующихся компьютерами. Да, я лично знаком с юниксоидом старого закала, перешедшим на Mac. Однако знаете, что больше всего его восхитило в MacOS X? Наличие терминала, в котором запускается настоящий bash со всеми его прибамбасами\dots

В общем, пора подвести итоги. Да, смена парадигмы общения пользователей с компьютерами неизбежна, как крах мировой системы социализма. Однако, как и последний, ничего принципиально в мировой расстановке сил это не изменит.

Те, кто понимает, зачем им нужен Linux (и другие свободные UNIX-системы), и кто нуждается в их функциях, будут продолжать их использовать. 

Те, кому по жизни и по службе требуется функциональность Mac'а, будут их покупать и использовать.

Основная же масса пользователей, которым глубоко безразлично не только имя стоящей у них на машине операционной системы, но даже сам факт наличия таковой, будут по прежнему использовать Windows.

И это не хорошо и не плохо: это, как сказал бы Остап Бендер, медицинский факт, с которым надо считаться.

\section{Кое-что о русском Linux'е} 

\begin{timeline}26 декабря 2007 г\end{timeline}

\hfill \begin{minipage}[h]{0.45\textwidth}
Забыл я автора,\\
И Google, бл\dotsин, сломался\\
Распалась связь времен.
\begin{flushright}
\textit{Народное POSIX'ивистское}  
\end{flushright}
\bigskip\end{minipage}

Собственно, проблемы русского Linux'а будут рассматриваться в другой книжке, которую я надеюсь подготовить в обозримом будущем. А для настоящей заметки появился конкретный повод: присланная в редакцию сайта Citkit статья Алексея Дмитриева <<Русский Линукс: первоочередные задачи>>. Которая являет собой пример концентрированного собрания предрассудков, заблуждений и прямых ошибок, связанных с заявленной темой. 

Я, конечно, отдаю себе отчет в том, что написана она, судя по всему, автором, инсталлировавшим свой первый дистрибутив Linux'а пару-тройку месяцев назад. И потому прошу его принять все сказанное мной далее не в обиду, а как информацию к размышлению. Тем не менее, совсем без ответа я её оставить не могу. Ибо практически каждой своей фразой автор дает повод для претензий по существу. А уж поводом для мелких придирок может быть любое слово. 

В связи с этим всё дальнейшее изложение будет представлять собой своего рода виртуальный диалог по схеме <<цитата~$\rightarrow$  комментарий>>. Цитаты даются \textit{as is}. 

Начать можно прямо с первой фразы: 
\begin{shadequote}{}
Итак, Линукс уверенно укоренился на Российской почве.
\end{shadequote}
О, этот сакраментальный оборот \textit{итак}! Создающий впечатление, что только вчера в российскую почву бросили семя Linux'а~--- а сегодня он уже пустил в ней корни. Если это не просто литературный штамп, должен несколько разочаровать автора: Linux укоренился на российской почве\dots ну как минимум года с 94-го. Прошлого тысячелетия, разумеется. А уж времени жизни его предыдущим инкарнациям~--- вариантам UNIX'а,~--- счет идет на десятилетия. 


\begin{shadequote}{}
Последние версии таких дистрибутивов как MopsLinux и AspLinux, похоже, ни у кого не оставили сомнения~--- да, можем!
\end{shadequote}
Хочется спросить автора: а что, предпоследние версии этих дистрибутивов оставляли сомнения? Или повод для сомнений дает дистрибутив Altlinux от одноименной российской компании? Или русская редакция Scientific Linux, выпускаемая питерской фирмой Linux Ink, выглядит в чём-то сомнительной? 

Если уж спуститься совсем в седую древность, то можно вспомнить также KSI и BlackCat с братской Украины. А о пионерах отечественного дистростроения, УрбанСофт и IPLabs Linux Team (ныне они носят упомянутые выше имена Linux Ink и Altlinux), как-то даже и говорить становится неловко. 

В общем, куваевский бич применительно к дистрибутивам Linux свою знаменитую фразу~--- <<Могём, начальник! Кто сказал, что не могём?>>~--- мог бы смело произнести году в 2000, как минимум, а то и раньше. 

Если же вспомнить, что среди разработчиков международных проектов, например, FreeBSD, также немал процент русскоговорящих (хотя и не обязательно российскоподданных или российских резидентов), то отпадут последние сомнения в том, что <<наши люди~--- фигли говорить>>. 
\begin{shadequote}{}
Можем делать дистрибутивы не хуже, а для нас, учитывая все сложности русификации, и лучше западных.
\end{shadequote}
Оставим в стороне вопрос, что такое <<лучше>> и <<хуже>> применительно к дистрибутивам Linux, а также проблему Востока и Запада. Но вот утверждение о сложности русификации последних лежит в области легенд и мифов Древней Греции. По крайней мере наиболее распространенные <<западные>> дистрибутивы, такие, как Red Hat, Suse, Debian, Ubuntu, не говоря уже об изначально интернациональной Mandriva, давно способны работать с языком родных осин ничуть не хуже, нежели самые-рассамые отечественные разработки. 
\begin{shadequote}{}
Настало время определить пути развития русского Линукса\dots
\end{shadequote}
Что же такое случилось, что именно сейчас настало это время? Неужели выход последних версий MopsLinux и AspLinux? Ведь до сих пор русский Linux как-то спокойно развивался естественным образом, без всяких определений его пути посредством чьих бы то ни было директивных указаний\dots 
\begin{shadequote}{}
\dotsчтобы не копировать слепо западное направление развития.
\end{shadequote}
\dotsпричем не копируя слепо западное направление, а в рамках всеобщих тенденций дистростроения. В частности, и в направлении интернационализации~--- например, развития тюркскоязычных локалей. Ведь это вполне естественно~--- в дистрибутивах южноафриканского происхождения развивается и расширяется поддержка, скажем, языка зулусов, а в дистрибутивах России~--- татарская, казахская и другие локализации. 
\begin{shadequote}{}
Линус Торвальдс, с упорством маньяка, печет как блины новые релизы ядра~--- каждые 2-3 месяца новый релиз.
\end{shadequote}
Да уж, сильно сказано. Если Линус~--- маньяк, то я как минимум китайский император. 

И вообще-то Линус не печет релизы, а с упорством\dots не маньяка, а энтузиаста, занимается разработкой тех вещей, которые лично ему интересны. При этом рассматривая и координируя патчи сторонних разработчиков, касающихся тех подсистем ядра, которыми он сам лично не занимается. Причем делает это абсолютно добровольно. 

И релизы ядра выпускаются не с какой-то периодичностью, и не из маниакального стремления к переменам, а тогда и потому, когда таких патчей становится достаточно много и они проходят достаточную проверку. Если бы автор удосужился прочитать, как организована разработка ядра, например, в статье Романа Химова он бы, надеюсь, воздержался от таких заявлений. 
\begin{shadequote}{}
\dotsно может быть лучше довести до ума какой-нибудь один?
\end{shadequote}
Ввиду отмеченной выше добровольности работы Линуса (и других ключевых разработчиков) по координации разработки ядра, предъявлять им какие-либо претензии~--- как минимум, не тактично. 

Вообще, меня просто умиляют заявления типа~--- <<лучше бы разработчики делали то, а не это>>. Разработчики делают то, что считают нужным, что они умеют и что им интересно. Если вы полагаете, что они делают что-то не то или что-то не так~--- вспомните, что это же Open Source. То есть никто не запрещает вам сделать что-то другое, по вашему мнению, более нужное, или что-то, на ваш взгляд, лучшее. Так что флаг в руки, барабан на брюхо~--- и вперед, с песнями, к своей идеальной разработке. 

Маленькое отступление, имеющее, тем не менее, прямое отношение к теме. Наиболее часто заявления указанного выше характера можно слышать в отношении дистрибутивов: зачем разрабатывать их столько, когда лучше довести до ума какой-нибудь один? 

Ну во-первых, как было сказано в предыдущем абзаце, если вы считаете, что это лучше~--- берите какой-нибудь один и доводите его до того, что полагаете <<умом>>. 

А во-вторых, если дистрибутив разрабатывается~--- значит, он кому-то и зачем-то нужен. Как минимум~--- своему разработчику. Если дистрибутив не нужен никому~--- проект просто тихо умирает. Правда, мне известен дистрибутив, который жил, хотя не нужен был даже своим разработчикам~--- но это один единственный пример на без малого пятьсот зарегистрированных на \url{http://distrowatch.com}. Да и он в конце концов умер своей смертью\dots
\begin{shadequote}{}
Ведь что движет группой разработчиков ядра~--- они стремятся поспеть за стремительно развивающимся <<железом>>.
\end{shadequote}
В основном разработчики занимаются совершенствованием подсистемы управления памятью, планировщика задач, файловых систем, виртуализации, а также специализированными задачами типа систем безопасности, реального времени или кластерных вычислений. 

Разработка драйверов устройств для поддержки стремительно развивающегося <<железа>>~--- лишь одна, и далеко не главная, сторона их деятельности. Как раз наоборот~--- желающих заниматься этой рутинной работой из голого энтузиазма не так уж много. И это одна из причин того, что поддержка <<железа>> в Linux'е не идеальна. 
\begin{shadequote}{}
А что побуждает железо стремительно развиваться? Софт. И уж конечно не свободный софт, а проприетарный, и в первую очередь Микрософтовский.
\end{shadequote}
Тут автор просто перевернул все с ног на голову. Как раз наоборот: новый софт, и проприетарный, и свободный, развивается, чтобы использовать возможности стремительно развивающегося <<железа>>. И только тогда, когда развитие этого <<железа>> позволяет. Вы можете представить себе Windows Vista или KDE с трёхмерными эффектами, работающие на процессорах P-166 и видеокартах от S3 с мегабайтом бортовой памяти? То-то же\dots 
\begin{shadequote}{}
Свободные исходники и влившийся в них мощным потоком Линукс с самого начала противопоставляли себя Микрософт и иже с ними.
\end{shadequote}
Для начала, будучи старым бюрократом, не могу не отметить терминологической неточности. Если бы автор, опять же, не поленился ознакомиться с соответствующей литературой, он бы узнал, что никаких свободных исходников не существует. А существуют движение за свободные программы (Free Software~--- на всякий случай подчеркну, что его не следует путать с понятием Freeware, то есть программ бесплатных) и движение за открытые исходники (Open Source). И это не идентичные, хотя и родственные, явления. 

Далее, движение Free Software не могло с самого начала противопоставлять себя Микрософту хотя бы потому, что возникло тогда, когда Гейтсу и Аллену можно было только мечтать не то что о всемирной монополии, но и даже об общем признании. Ибо разрабатываемые компанией продукты в то время предназначались для сугубо маргинальной сферы IT-индустрии. 

Что же до Linux'а~--- так Линусу в 1991 году был глубоко параллелен сам факт существования компании Микрософт. Подозреваю, что в глубине души он ему параллелен и по сей день. 
\begin{shadequote}{}
А в итоге оказались в качестве догоняющего в навязанной им гонке <<железа>>.
\end{shadequote}
Вообще-то, за почти 10 лет работы в Linux'е (а также и BSD), я никакой такой <<гонки железа>> не заметил. Просто на каждом этапе развития оного разрабатываются программы, использующие новые его возможности. Например, в 1998 году KDE~1-й версии прекрасно работал у меня на машине с P-II/266 и 64-мя мегабайтами оперативной памяти. Ныне он столь же прекрасно работает при двухъядерном процессоре AMD с реальной тактовой частотой 3~Ггц и 2-мя гигабайтами <<мозгов>>~--- в своей третьей ипостаси, имеющей, по сравнению с 1-й, массу полезных и удобных, а то и просто красивых, <<фич>>. Появление которых и стало возможным вследствие развития <<железа>>. 
\begin{shadequote}{}
Несомненно, что эта гонка должна вестись, но она не должна быть единственным направлением разработок.
\end{shadequote}
Как уже было сказано выше, это и так далеко не единственное направление разработок. Тем более, что и гонка-то на самом деле не не имеет места быть\dots 
\begin{shadequote}{}
А другим направлением, к сожалению второстепенным, является <<доведение до ума>> уже существующего в мире свободных исходников.
\end{shadequote}
Тут автор высказался весьма лихо. Во-первых, именно <<доведением до ума>>, то есть исправлением ошибок, добавлением мелких полезных <<фич>> и тому подобной рутинной, но необходимой работой занимаются тысячи и тысячи людей во всем мире. И цитированная фраза являет собой вопиющее неуважение к их труду. 
\begin{shadequote}{}
И тут, как мне представляется, особую роль должны сыграть отечественные разработки.
\end{shadequote}
Так что отдавать какой-либо приоритет в этом именно отечественным разработчикам было бы странно. Тем более, что они и так, наряду с прочими народами, выполняют свой интернациональный долг, внося должный клад в процесс <<доведения до ума>>. 

Тем не менее, цитированная фраза знаменует возвращение автора к теме русского Linux'а, от которой он отвлекся рассуждениями о <<гонке железа>>. 
\begin{shadequote}{}
Делать дистрибутивы мы научились.
\end{shadequote}

Как было установлено выше, делать дистрибутивы <<мы>> (вероятно, в смысле~--- <<весь великий советский народ>>? Или в понимании Замятина?) научились давно. И продолжаем учиться~--- то есть совершенствовать уже сделанное. Впрочем, <<не-мы>> поступали и поступают точно также. Более того, нередки случаи, когда <<мы>> и <<не мы>> трудятся плечом к плечу над одними и теми же проектами. 
\begin{shadequote}{}
А что мешает распространению этих дистрибутивов в народные российские массы?
\end{shadequote}

Задается далее автор риторическим вопросом. Должен его огорчить~--- ничего не мешает. Более того, процесс распространения идет очень давно. Только на моей памяти оценочное число только активных пользователей Linux'а возросло на порядок. А если учесть пользователей пассивных~--- тех, кто даже и не знает, какая ОС установлена на его автоматизированном рабочем месте, домашнем десктопе, наладоннике, смартфоне (нужное дописать)~--- не знает, потому что его этот вопрос ни в малейшей степени не колышет,~--- то число таких пользователей возрастет еще как минимум на порядок. 

Тем не менее автор полагает, что распространению Linux'а в массах кто-то или что-то мешает (не происки ли мирового жидо-масонства или проклятых имперьялистов?). И предлагает свои варианты ответа, что мешает именно. 
\begin{shadequote}{}
Во-первых, плохое знание английского языка широкими российскими массами.
\end{shadequote}
Тут автор, подобно герою известного анекдота, как всегда оказался не прав. Широкие российские массы соответствующего возраста владеют английским, как правило, лучше, чем их ровесники по моему поколению (автор этих строк тоже не читает Шекспира в подлиннике, мягко говоря; Гёте и Шиллера, язык которых учил в школе и ВУЗе, в оригинале не читает тоже). Иногда при просмотре форумов создается впечатление, что многие из участники английским владеют лучше, нежели родным. Это раз. 

Два~--- для чтения документации отнюдь не обязательно владеть языком Чосера, как родным. Ибо написаны часто отнюдь не нативными носителями оного, и потому очень просты. 

Три~--- очень большое количество документации переведено и переводится на русский язык, о чем автор, вероятно, не осведомлен. 

И четыре~--- никакой словарь, будь он хоть трижды ABBYY Lingvo, не поможет в понимании специфических терминов, если нет понимания сути терминов этих. 

Так что увы~--- но 
\begin{shadequote}{}
\dotsсоздание современного, удобного, максимально полного, и современного англо-русского электронного словаря
\end{shadequote}

\dotsи на фиг никому не нужно. Всё равно без минимально-технического английского не обойдется ни один активный линуксоид. Ибо даже при стопроцентно русскоязычной документации ему придется читать и понимать системные сообщения, комментарии к конфигурационным и Make-файлам, всякого рода \texttt{README}, переводить которые на русский, хвала Аллаху, еще никто не додумался (хотя в Сети можно найти упоминания о проектах создания исконно русской операционки, в которой даже информация о ходе загрузки выводилась бы на Великом и Могучем). 

Для линуксоида же пассивного чтение документации~--- не его вахта, ибо все, что касается системной части, за него и для него должен сделать линуксоид активный. И ему останется только осваивать необходимые приложения~--- а уж с русификацией большинства их, как и рабочих сред типа KDE и GNOME, дело давно уже налажено. 
\begin{shadequote}{}
Второй сдерживающий фактор освоения Линукса~--- руководства, пресловутые <<маны>>.
\end{shadequote}

Да, это уже переходит всякие границы\dots Каждый, все-таки освоивший Linux (а таких людей, как было отмечено ранее, не так уж и мало), скажет, что смог сделать это благодаря чтению 
\texttt{man}'ов. И невдомек ему, бедняге, что это был сдерживающий его развитие фактор. Тем же, для кого понять man'ы <<задача почти непосильная>>~--- так может быть, оно им и не надо, этого Linux'а? Так что, действительно 
\begin{shadequote}{}
Простым переводом на русский проблему манов не решить
\end{shadequote}
И не потому, что
\begin{shadequote}{}
в переводе непОнятости накладываются на непон\`{я}тности
\end{shadequote}
Тут скорее нужно чего-то поправить в консерватории. 

Да и вообще, чтение \texttt{man}'ов <<в подлиннике>> предпочтительней по одной простой причине: перевод по определению всегда будет отставать от оригинала в отношении версии. И потому новые опции и <<фичи>> могут не найти в нем отражения. 

Кстати, уж что-что, а язык man'ов предельно прост. И не потому, что написаны они не нативными <<англичанами>>, а потому, что вполне сознательно сочинялись <<для всех>>~--- в том числе и не очень твердых в аглицкой мове. Я лично знаю людей, владеющих английским еще хуже меня~--- и тем не менее вполне успешно разбиравших \texttt{man pages} по интересующим их вопросам. 

Для тех же, кто не в ладах с вражьей мовой до степени полного непонимания (и заодно для автора), открою страшную тайну: работа по переводу \texttt{man}'ов ведется уже много лет, историю чего можно прочитать на сайте Виктора Вислобокова. Да и майнтайнеры дистрибутивов не остаются в стороне от этого процесса~--- причем не только наши, родные, но и такие западные акулы, как Novell и Red Hat. 

Ну а уж следующий тезис автора: 
\begin{shadequote}{}
Необходимо организовать народный проект по написанию внятных мануалов, а по-нашему~--- руководств.
\end{shadequote}

\dotsспособен вызвать только взрыв здорового смеха. К коему предлагаю присоединиться и автору. Ибо, как Мальчиш-Плохиш, готов выдать и вторую военную тайну: такой народный проект стихийно, без всяких <<указаний сверх>>, самоорганизовался много лет назад, начиная с первых сетевых публикаций и <<бумажных>> статей Владимира Водолазкого, онлайновых заметок Виктора Вагнера и Владимира Игнатова, проекта под названием <<Пособие для начинающих ковбоев>>, руководства по UNIX Сергея Кузнецова, и многих, многих других. Приношу свои извинения тем основоположникам линуксописания, кого не упомянул поименно~--- ибо имя вам легион, но мы о вас помним, даже если вы давно уже занимаетесь совсем другими делами. 

И этот <<народный проект>> активно развивается и по настоящий момент. Ибо буквально каждый день на многочисленных Linux-сайтах и в еще более обильных блогах линуксоидов появляются материалы по теме~--- и частные, посвященные решению некоей конкретной проблемы, и общие, вплоть до подробных руководств. 

Чтобы убедиться в этом, автору достаточно было бы зайти на сайт Виктора Костромина, на котором собраны ссылки на все русскоязычные материалы по UNIX, Linux и Open Source. Впрочем, все содержание статьи оставляет впечатление, что у автора не только Google сломался, но и вообще весь Интернет. 

Комментировать предлагаемое автором 
\begin{shadequote}{}
третье направление развития отечественного Линукса~--- программы проверки правописания и пунктуации
\end{shadequote}

\dotsне возникает никакого желания. Ибо это ни что иное, как предложение изобретать очередной велосипед. Возможно, даже с квадратными колесами. Вместо того, чтобы, говоря словами автора и его собратьев по классу, развивать и пополнять русскоязычные словари к существующим программам проверки орфографии и приспосабливать их к работе с <<не устаревшими>> кодировками (если под таковыми имеется ввиду UTF-8, то \texttt{aspell} успешно работает с ней уже много лет). 

В заключение хотелось бы призвать автора последовать своим собственным советам. Во-первых, 

Каждый линуксоид, освоивший какую-либо команду, и способный связно изложить свои знания~--- обязан поделиться с народом!

и действительно чего-нибудь освоить (хотя бы \texttt{man}-страницы). А во-вторых, 
\begin{shadequote}{}
главное, начать действовать в этом направлении.
\end{shadequote}

А не заниматься постановкой задач тем, кто давно в этом направлении именно действует. 

Ну и совсем под занавес. 

Linux и Open Source были и остаются явлениями интернациональными, не признающими ни государственных, ни конфессиональных, ни каких-либо иных границ. Свой посильный вклад в них вносят все, могущие это сделать, не зависимо от языка, вероисповедания, подданства и цвета кожи. Также как всем вольно черпать из этого источника то, что нужно для решения его задач. 

Надеюсь, что существующее положение будет сохраняться и впредь. Для меня апогеем развития <<русского Линукса>> было бы не создание еще одного исконно-посконного дистрибутива или приложения. А увидеть нашего соотечественника на посту лидера проекта Debian или FreeBSD, в составе совета директоров компании Mandriva, одним из ключевых разработчиков Linux Kernel\dots 

Занавес опускается, оставляя читателя в недоумении: зачем было разводить столько турусов на колесах вокруг, в общем-то, тривиальных и общеизвестных вещей. Отвечаю.

Увы, они тривиальны и общеизвестны для тех, кто использует Linux (или прочие BSD'и) в своей работе не первый год. Но в каждом новом поколении линуксоидов обязательно появляются первооткрыватели америк, искренне полагающие, что до них никто ничего не делал и ни до чего не додумался. И начинающие формулировать <<очередные задачи Советской власти на данном этапе>>. И комментируемая статья являет собой нечто вроде квинтэссенции формулировки таких задач. Заставляя вспомнить известный анекдот про Вовочку, когда он, придя 1 сентября первый раз из первого класса, радостно закричал родителям: <<Вот вы, козлы, тут сидите и не знаете, а пиписька-то х\dotsм называется!>>. Почему я и не смог оставить её без ответа. 

Дабы связь времен все-таки не разрывалась, и Google не терял бы устойчивости к поломкам, и не банил бы нас таинственный разум, поселившийся в глобальной сети Интернет~--- тот самый, который был описан Роджером Желязны и Фредом Саберхагеном в замечательной повести <<Витки>>.

\section{Три поколения в русском FOSS-движении}

\begin{timeline}Август 11, 2010\end{timeline}

Как ни относись к образу товарища Владимира Ильича Ульянова в скобках Ленина, нельзя отказать в даре предвидения. В частности, в работе <<Памяти Герцена>> (1912 год) он предсказал весь путь грядущего русского FOSS-движения. Что мы, опираясь на память детства (в школе эти строки заучивались наизусть в обязательном порядке), и проиллюстрируем ниже цитатами из указанной работы классика, а также комментариями к ним.

Итак, подобно тому, как
\begin{shadequote}{}
\dotsмы видим ясно три поколения, три класса, действовавшие в русской революции\dots
\end{shadequote}
\dotsтри поколения можно увидеть и в русском FOSS-движении:

\begin{shadequote}{}
Сначала~--- дворяне и помещики, декабристы и Герцен. Узок круг этих революционеров. Страшно далеки они от народа.
\end{shadequote}

С дворянами и помещиками всё ясно. Это~--- юниксоиды первого призыва, обретавшиеся в своих родовых поместьях~--- закрытых КБ, ящичных НИИ и тому подобных режимных предприятиях. Круг их был действительно узок, а с народом они сосуществовали в параллельных мирах.

Но из недр указанных учреждений вышли первые юниксоиды-декабристы, воззвавшие к народу через каналы Курчатника, Демоса, Релкома. Которые, сложись дело в августе 1991-го чуть иначе, могли бы и составить компанию декабристам в соответствующих местах.

Своим призывом они разбудили нашего однозначного Герцена~--- Владимира Водолазкого. Который, правда, революционной агитации не разворачивал. Но зато написал статью в журнале \textit{``Монитор''} (1994 год) под примерным (по памяти) названием: \textit{``Как без головной боли установить Linux''}. Ксерокопии этой статьи, в сопровождении стопки дискет с дистрибутивом древней версии Slackware, годами ходили потом по рукам. Так что можно сказать, что именно\dots

\begin{shadequote}{}
Ее подхватили, расширили, укрепили, закалили революционеры-разночинцы, начиная с Чернышевского и кончая героями ``Народной воли''.
\end{shadequote}
Безусловно, юниксоиды второго призыва (которых уже можно называть FOSSоидами), были в самых разных чинах, званиях, образованиях~--- от учителей и врачей до географов, почвоведов и даже, страшно сказать, писателей. И за кем из них следует закрепить амплуа Чернышевского~--- вопрос спорный. А вот на звание героев Народной воли однозначном может претендовать команда форума Linuxshop'а, составившая позднее ядро Linuxforum'а (нынешнего \texttt{Unixforum.org}). Именно благодаря им\dots
\begin{shadequote}{}
Шире стал круг борцов, ближе их связь с народом.
\end{shadequote}

И действительно, число разночинцев наших росло. И сохранили они связь с народом~--- то есть со своими производственными коллективами, школами, ВУЗами.
\begin{shadequote}{}
Молодые штурманы будущей бури
\end{shadequote}
звал их Герцен. Но это не была ещё сама буря.

Ну как тут не поразиться прозорливости Вождя? Ведь действительно, не линуксоидам-разночинцам суждено было стать штурманами бури. Ибо\dots

\begin{shadequote}{}
Буря~--- это движение самих масс. Пролетариат, единственный до конца революционный класс, поднялся во главе их\dots
\end{shadequote}

Пророчество, достойное Нострадамуса, Глобы и девицы Ленорман. О сугубо пролетарском происхождении вождей Русской революции~--- начиная с самого Ульянова (Ленина) и корифана его Троцкого~--- знают все. Но ведь не меньшими пролетариями оказались и штурманы нынешней FOSS-бури: пролетариями от политики, от бизнеса, от госаппарата. Именно этот пролетариат\dots
\begin{shadequote}{}
\dotsвпервые поднял к открытой революционной борьбе миллионы крестьян.
\end{shadequote}

То есть тех самых новообращённых линуксоидов <<от бороны>>, у которых ожидание доброго барина сменилось вожделением Большой Красной Кнопки.

\begin{shadequote}{}
Первый натиск бури был в 1905 году.
\end{shadequote}
И опять-таки~--- через столетие Вождь угадал с точностью до года: именно в 2005 году по тихой FOSS-заводи пошли барашки, предвещавшие будущие баллы по Бофорту. Правда, мы тогда этого ещё не поняли, приняв первые порывы ветра за освежающий бриз.

\begin{shadequote}{}
Следующий начинает расти на наших глазах.
\end{shadequote}

И продолжает это делать. посмотрим, до чего он вырастет к столетнему юбилею ленинской работы\dots

\section{Применители vs потребители}
\hypertarget{customers}{}

\begin{timeline}Октябрь 19, 2012\end{timeline}

<<Когда машины были большими>> и работали в пакетном режиме, люди, тем или иным образом связанные с компьютерами, делились на две неравные категории: те, кто имели дело непосредственно с вычислительным комплексом, назывались операторами, прочие же, те, кто обеспечивал их работой\dots да вроде никак они специально не назывались. Были, конечно, ещё и программисты~--- но они или каким-то боком примыкали к сословию операторов, либо попадали в категорию <<прочих>>.

Появление систем разделения времени и терминального доступа отменило сословие операторов. Но вместо этого вызвало к жизни кастовое деление~--- на системных администраторов, имеющих доступ к святая святых~--- управляющему терминалу, иначе консоли, и простых людей, вынужденных довольствоваться терминалами обычными. За этими самыми простыми людьми со временем закрепилось имя пользователей (users, они же юзер\'{а}, \'{ю}звери и прочие усер\textit{а}), хотя по жизни они могли быть и программистами.

Казалось бы, персоналки должны были уравнять всех в правах, подобно изобретению полковника Кольта. И поначалу так оно и было, поскольку каждый пользователь ПК поневоле был и сам себе админ, и где-то даже сам себе программер. Особенно начиная с того времени, как для самых демократичных из персональных платформ, x86, появились столь же демократичные UNIX-подобные операционные системы, открытые и свободные~--- сначала BSD-семейства, а затем и Linux. Их пользователи, решая свои профессиональные задачи, занимались и администрированием (хотя бы в масштабах одной отдельно взятой квартиры или даже персоналки).

Впрочем, и Mac'и, и Windows-машины тоже некогда использовались почти исключительно в производственных целях~--- именно в те годы и появилось выражение <<кустарь-одиночка с персональным компьютером>>. Который, в отличие от Виктор Михалыча Полесова, занимался задачами отнюдь не кустарными~--- а теми, что раньше требовали мощи корпоративных или государственных издательств, вычислительных центров и так далее.

А потом, с экспансией сначала мультимедии, а затем и массового Интернета, понятие <<пользователь>> стало размываться: этим словом (иногда с ругательным оттенком) стали называть и конструктора в CAD-системе, и писателя или переводчика, вбивающего в <<цифре>> свою или чужую нетленку, и меломана, и записного игрока, и просто завсегдатая <<одноглазников ффконтакте>>. И ныне, особенно с широким распространением сугубо развлекательных гаджетов, сравнимых по вычислительной мощности с суперкомпьютерами недавнего прошлого, вообще утратило всякую определённость.

Эта тенденция была понятна всем~--- требовалось её только как-то терминологически оформить. И такие попытки были. В частности, распространилось деление всех так называемых пользователей на потребителей контента и его создателей. С первым термином трудно не согласиться: назвать человека, околокомпьютерные интересы которого сводятся к социальным сетям и просмотру Youtube, пользователем персонального компьютера~--- как-то язык не поворачивается.

А вот второй термин представляется мне позорным и преступным. Ибо открытым текстом подразумевает, что все, кто не потребители, только и заняты тем, что клепают контент им на потребу.

Поэтому давеча я пришёл к выводу, что слово <<пользователь>> должно быть изъято из нашего лексикона~--- по крайней мере из своего я постараюсь его изъять (хотя по первости наверняка буду забывать). А взамен собираюсь оперировать терминами <<применитель>> и\dots ну куда же от него денешься, конечно же, потребитель.

С последним всё ясно~--- это тот, кто потребляет всё, что ему предлагают, от игр и фильмов до социальных сетей. И который, в сущности, действительно не нуждается в компьютере, каким бы тот <<потребительским>> не был: его потребности за глаза перекрываются гаджетами типа смартфонов и планшетов.

А вот применитель~--- этот тот, кто применяет компьютер для решения своих задач. И задачи эти могут быть самыми разными~--- от сочинения стихов и романов до проектирования самолётов и даже, страшно сказать, написания кода операционных систем. И который в своём деле обойтись без персоналки не может~--- будь она личной или служебной, настольной или мобильной. Но должна она быть именно полноценной рабочей машиной, а не нишевым гаджетом.

Это не значит, что применителю запрещается применение в своих целях тех же гаджетов. Однако цели его таковы, что гаджет для него~--- инструмент сугубо вспомогательный.

Так что страхи производителей ПК и комплектующих к ним, вызванные ростом продаж потребительских гаджетов на фоне снижения спроса на десктопы и даже ноуты, выглядят несколько наигранными. Да, возможно, что настоящие персоналки будут приобретаться только применителями~--- почти все потребители полностью перейдут на гаджеты. Да, вследствие этого производство ПК снизится, а цены на них, как товар не массовый, возрастут. Но, пока на свете существуют применители, спрос на них будет~--- и спрос платёжеспособный.

Потому что~--- рискну высказать крамольную мысль~--- для профессионала цена его профессионального инструмента большого значения не имеет. Или имеет только на стадии вхождения в профессию. А уж дальше~--- если профессионал не смог заработать на адекватный его задачам инструмент, значит, он ошибся в выборе профессии.

И кто знает, не возродится ли благородное ремесло <<кустаря-одиночки с персональным компьютером>> на новом витке диалектической спирали?

\section{Применители-понахьюсты} 

\begin{timeline}Август 5, 2013\end{timeline}

Те из моих читателей, что застали зарю перестройки, возможно, помнят первый телемост СССР--США, который вели Владимир Познер из Ленинграда и Фил Донахью из Сиэтла. Помнят хотя бы потому, что уже на следующий день половина страны задавалась вопросом:

-- Что общего между Познером и Донахью?

На что вторая половина отвечала:

-- То, что им всё понахью\dots

Давеча мне это вспомнилось в связи с этапами онтогенеза типичного линуксоида-применителя. На первом этапе он с увлечением предаётся настройке системы, в том числе и её интерфейса, благо большинство рабочих сред дают для этого массу возможностей. И изрядную часть времени, свободного от собственно применительства, тратит на подбор обоев своего десктопа (а то и создание собственных), опробование тем и стилей окон и пиктограмм, эксперименты с интерфейсными шрифтами.

Переход ко второму этапу знаменуется тем, что у линуксоида-применителя складываются  устойчивые предпочтения в отношении внешнего вида его рабочего стола. И потому свежеустановленную систему он настраивает несколькими отработанными до рефлекторного уровня действиями. А то и вообще обходится без них, выявив оптимальные для себя методы переноса настроек из старой системы в новую.

И, наконец, в один прекрасный момент линуксоид-применитель обнаруживает, что, по большому счёту, все эти интерфейсные красоты ему становятся по\dotsнахью. Запускается привычный терминал, текстовый редактор, ворд-процессор\dots (нужное вписать, ненужное вычеркнуть)~--- и хорошо, можно начинать заниматься применительством. Лишь бы буковки были читаемы, да цвета глаз не резали. Наступает третий этап онтогенеза~--- понахьический.

Для меня этот этап наступил в момент недавнего возвращения на Ubuntu. Неожиданно  я обнаружил, что меня не раздражает ни её цветовая гамма, ни кнопки-иконки на панели запуска, ни умолчальный вид окон. Тем более что кнопку закрытия окна во всех рабочих средах переносил на левый край строки заголовка~--- но не из эстетических соображений, а для избежания случайного нажатия. Единственно, что шрифты для моих глаз маловаты~--- но это тоже относится не к эстетике, а к медицине.

Тут я и понял окончательно одну из причин популярности Ubuntu не только среди начинающих пользователей, но и среди многоопытных применителей~--- не смотря на всю убогость возможностей её настройки после стандартной инсталляции: она вполне компенсируется их ненужностью. И при этом не создаёт отвлекающих факторов в виде желания в любую свободную минуту подправить в интерфейсе ещё чего-нибудь.

\section{Ещё раз о потребителях и применителях, а также иных} 

\begin{timeline}Август 30, 2013\end{timeline}

В заметке \hyperlink{customers}{<<Применители vs потребители>>} я не затронул несколько важных вопросов, касающихся этой темы. Постараюсь восполнить эти проблеы, черпая вдохновение в трудах классиков.

Во-первых, подозреваю, что у читателя упомянутой заметки могло создастся впечатление, что слово потребители я употребляю в уничижительном смысле, некую серую массу, ни на что не способную.

Прошу поверить, это не так. И никакого оттенка осуждения в слово потребители в общем случае я не вкладываю. Ибо прекрасно понимаю стремление человека, восемь часов в сутки занятого скучной конторской работой, провести свой досуг в просмотре фильмов о приключениях, в которых он никогда не примет участия, или за играми, которые позволят ему представить себя в роли героя~--- спасителя человечества.

Нет, в самом по себе потребительстве как способе проведения досуга ничего худого я не вижу. Страшен не потребитель сам по себе, а потребитель воинствующий, проникнувшийся идеей потребительства и полагающий, что все вокруг должны быть такими же потребителями, каким стал он сам. В литературе эта позиция блестяще иллюстрируется фигурой Глухова из книжки Стругацких <<За миллиард лет до конца света>>:

\begin{shadequote}{}
Меня злит не то, как он сделал свой выбор,~--- проговорил Вечеровский медленно, словно размышляя вслух.~--- Но зачем все время оправдываться? И он не просто оправдывается, он еще пытается завербовать других. Ему стыдно быть слабым среди сильных, ему хочется, чтобы и другие стали слабыми. Он думает, что тогда ему станет легче.
\end{shadequote}

Но это далеко не самое страшное, и воинствующий потребитель в реальной жизни вовсе не являет собой столь апокалиптической личности, какой его показывают братья~--- они ему очень сильно льстят. Ибо самое апокалиптическое, что я могу себе представить~--- это потребитель обыденный. Для которого потребительство стало настолько привычным, что он даже не подозревает: все используемые им для потребительства технологии некогда были придуманы для применительства, а вовсе не для его удовольствия. И до сих пор ещё сохранились люди, которые иногда занимаются применением этих технологий. Что опять же нашло прекрасное отражение в литературе~--- в романе Владимира Савченко <<Открытие себя>>:
\begin{shadequote}{}
Многое было в этих мечтаниях~--- одно сбылось, другое отброшено техникой. Но вот мечты о том, что транзисторы украсят туалеты прыщеватых пижончиков с проспекта, не было.
\end{shadequote}

Далее, опять-таки могло создаться впечатление, что, говоря о применителях, я молчаливо подразумеваю их высокую квалификацию в области информационных технологий. И это в общем случае не так. Вспоминая эпоху доминирования рабочих UNIX-станций, мы видим, что их пользовтаели~--- чистые применители без намёка на потребительство,~--- часто не имели ни малейшего представления об устройстве того UNIX'а, в котором они запускали свои применительские приложения. И ничуть от этого не страдали~--- критичным для них было не знание системы, а владение своим профессиональным инструментарием.

Скорее как раз наоборот~--- некоторым потребителям для удовлетворения своих потребительских потребностей необходимы довольно обширные познания в области компьютерных технологий. Хотя от необходимости иметь некоторые представления о своих потребительских приложениях не избавлен ни один <<чистый>> до стерильности потребитель, обычно они сводятся к набору простых рецептов, в идеале~--- к нажатию Большой Красной Кнопки с надписью \textbf{Сделай мне пе\dotsто}.

Однако тем потребителям контента, которые, подобно персонажу из бригады Венечки Ерофеева <<с претензиями>>, например, записным геймерам или страстным меломанам, приходится прибегать к изощрённым методам потребления, подчас требующих весьма обширных познаний в таких общесистемных материях, о которых квалифицированный применитель даже и не слыхал. Правда, такой потребитель в пределе перестаёт быть чистым потребителем, и тесно примыкает к той категории пользователей, о которых я раньше не говорил. И которую можно назвать гиками.

Количественно группа настоящих гиков немногочисленна. По роду своей основной деятельности они часто никак не связаны с UNIX'ами, ни с Linux'ами, ни вообще с компьютерными технологиями. По жизни они могут заниматься самыми разными делами. Копание во внутренностях операционной системы, сборка, а подчас и сочинение, пакетов для них~--- не более чем хобби. Но на том самом очень серьёзном уровне, когда хобби напрямую смыкается с профессией.

Причины, по которым люди избирают в качестве хобби копание во внутренностях Linux'а и его приложениях, могут быть самыми разными: и неудовлетворённость основной работой с точки зрения творческой самореализации, и стремление к периодической смене деятельности, и самое обычное любопытство, которое затягивает сильнее любого наркотика.

Многие из гиков со временем уходят в области профессионального применения Linux'а и FOSS. Однако это происходит далеко не всегда~--- как по объективным причинам, обусловленным принудительной силой реальности, так и по субъективным, потому что основная сфера деятельности необходима для самореализации не меньше, чем серьёзное хобби.

Роль гиков в развитии FOSS трудно преувеличить~--- достаточно сказать, что Linux изначально возник как побочный результат действий студента-энтузиаста. Правда, для Торвальдса разработка его операционки стала профессией. Однако не менее показателен и пример Кона Коливаса, совмещающего сочинение патчей к ядру, реализующий его собственный планировщик задач, с работой врача-анестезиолога.

И примеров как из первого, так и второго можно привести очень много. Так, для Билла Рейнолдса, также известного как Texstar, разработка и сопровождение дистрибутива PCLinuxOS стали профессией, а Жан-Филипп Гийомен, создатель дистрибутива Zenwalk, продолжает трудиться инженером по безопасности в корпорации Телиндус~--- в области, хоть и связанной с IT, но с разработкой Linux'а пересекающейся слабо.

Однако в последнее время гики, как и применители, становятся всё менее заметны на общем фоне пользователей-потребителей. Не потому, что любопытствующих энтузиастов становится меньше~--- изменились времена и сместились системы приоритетов. Ещё лет 10 назад роль независимого разработчика или применителя могла стать венцом профессиональной карьеры в любой сфере деятельности, то ныне для потенциальных гиков пределом мечтаний является служба в большой и престижной корпорации. Где они вынуждены делать не то, что считают нужным, а то, что прикажут. Прикажут же в большинстве случаев~--- создавать нечто на потребу потребителям. Последние же, согласно словам персонажа романа Куваева, завоюют мир, как раньше
\begin{shadequote}{}
\dotsмир захватывали чингисханы, тамерланы, македонские или орды, допустим, гуннов.
\end{shadequote}

После чего и наступит предсказанный Стругацкими конец света, и до него остался вовсе не миллиард лет\dots

\section{Наступит ли Эра Linux'а?} 

\begin{timeline}2001\end{timeline}

\textsl{Первая версия это заметки была написана (как обычно, по случаю) осенью 2001 г. и опубликована на онлайновой Софтерре (ныне кусок сайта Компьютерры). Поводом для нее послужило как бы интервью, взятое Максимом Отставновым у Евгения Козловского, касаемое опыта общения последнего с Linux'ом (Компьютерра, 2001, \textnumero33 (410), с. 40-41. Прошу понять меня правильно~--- я не ставлю себе задачей доказывать, какой Linux хороший. И убеждать, что все упомянутые в статье проблемы решаются не просто, а очень просто.}\medskip

И вообще, тему проблем я поднял только потому, что затронута честь одного из продуктов издавна (с января 98-го) любимой мной команды Altlinux (ранее~--- IPLabs Linux Team). Каковая всегда отличалась именно беспроблемностью (по Linux'овым стандартом) своих дистрибутивов. Непосредственно с вариантом MSI Edition мне довелось пообщаться только в бета-версии, но и этого было достаточно для заключения: редакция эта в плане беспроблемности ничуть не уступает всем прочим представителям своего поколения (в числе которых~--- Linux Mandrake Spring 2001 и Junior).

Так что проблемы, с которыми столкнулся Евгений, можно отнести только к фатальному невезению. Так, трудновато найти сетевую карту, которая не определялась бы почти любым из развитых (т.~н.~user-oriented) дистрибутивом (вернее, его инсталляционной программой). Программа kudzu, входящая в состав дистрибутивов от Altlinux (как, впрочем, и многих других), обычно вполне успешно выполняет роль P'n'P из Windows: вплоть до того, что успешно определяет телетюнеры на распространенных чипах (типа BTxxxx).

Ситуацию с CD иначе чем недоразумением не назовешь. К сожалению, Евгений не написал, каким образом он пытался получить к нему доступ. Не щелчком ли по иконке на рабочем столе KDE?~--- ни за что не поверю, что ответ типа Permission denied мог быть получен на \texttt{mount /mnt/cdrom} от лица root'а.

Так что, возможно, достаточно было проверить (стандартным и для Windows способом, через пункт \textbf{Свойства} контекстного меню по щелчку правой клавишей мыши), на правильное ли устройство указывает иконка CD-ROM. А на худой конец~--- посмотреть, ссылкой на какое устройство является \texttt{/dev/cdrom}: в системе с пишущим CD-R/RW их может быть два, одно~--- для стандартного IDE-привода, другое~--- для него же в режиме эмуляции SCSI.

А самое главное~--- именно для дистрибутивов Altlinux (не думаю, что MSI Edition в этом отношении отличается) нет необходимости лазать по Сети в поисках недостающих компонентов. Поскольку на их сервере имеет место прибывать метадистрибутив Sysiphus, содержащий немерянно всякого софта (в том числе и мультимедийной направленности), скомпилированного для беспроблемной установки в любых дистрибутивах Altlinux.

Повторяю, все это я написал не в упрек~--- а исключительно к вящей славе Linux вообще и дистрибутивов Altlinux в частности. Тем более, что, по собственному признанию Евгения, он и не пытался решать возникшие проблемы. сославшись на отсутствии мотивации. Вот вопрос мотивации перехода на Linux и является, собственно говоря, предметом настоящей заметки. Но сначала~--- цитата:

\begin{shadequote}{}
98\% известных мне <<офисных>> задач решается под Linux так же, как и под Windows. 
\end{shadequote}

Если знаки препинания в этой фразе расставлены в соответствие с авторским замыслом~--- а я склонен думать, что Евгений, при его опыте, не проглядел бы ошибки набора,~--- это на корню убивает всякое желание переходить на Linux. Но дело в том, что это в корне неверно: задачи решаются под Linux, также как и под Windows. Но~--- совершенно иными методами. 

Действительно, под Linux можно сочинять в StarWriter'е заметки, подобные этой, как и в Word'е под Windows. Однако делать это~--- примерно то же, что ехать по хайвею на вездеходе: <<можно бы, да на фига>>, на сей предмет <<мерседесы>> придуманы. А вот попытка проехать на <<мерседесе>> (или даже джипе) по корякской тундре вряд ли доставит удовольствие, здесь уж ГТС'ка потребна (а лучше~--- ГТТ'уй подавай). 

Так каковы же могут быть мотивы к Linux-миграции у человека творческой, как говорили при соввласти, профессии? Пример первый, предельно элементарный, но наглядно демонстрирующий, тем не менее, различие подходов. 

Дано: десять файлов, составляющих нетленный роман из жизни компьютеров (или любимых кошек, без разницы, рассеянных по разным каталогам (или, как принято говорить в мире Windows, директориям), а то и накопителям. 

Требуется: объединить их в некоем порядке в единый текст, записав последний в новый каталог. Будем считать, сколько погонных метров потребуется накрутить шарику мыши по её подмышнику, и сколько~--- бить указательным пальцем по её кнопке? Или, может, просто выполним команду:\medskip\texttt{\$ cat /path1/file1~\dots~/path10/file10 > /newpath/newfile}\medskip

Напомню, что при этом нет необходимости тупо фигачить по клавишам, вбивая многочленные пути и замысловатые имена: к вашим услугам клавиша автодополнения. А добавив сюда конвейеризацию команд, можно и сформатировать текст в соответствии с его логической структурой, и вывести на печать. Обращение же к скриптам все эти (и многие, многие другие) задачи автоматизирует до любого предела, устанавливаемого только собственной фантазией. 

Я уж не говорю о возможностях, предоставляемых внешне непритязательными текстовыми редакторами: и по созданию той же логической структуры, и по её разметке для дальнейшего форматирования, и по поиску фрагментов, смысл которых помнится лишь приблизительно, и их автоматической замене. 

Разумеется, работа в Windows может показаться комфортнее: щелкать мышью, развалясь в кресле~--- это почти как лететь в МИ-8 над той же корякской тундрой. Разумеется, при наличии погоды, керосина, хороших отношений с отделом перевозок (иначе ведь могут забыть снять~--- бывало и такое), не говоря уже о деньгах. Однако, если вспомнить об авральной загрузке в машину двух-трех тонн груза, столь же авральной разгрузке по посадке, днях и неделях ожидания борта, многокилометровых подходах при неудачном выборе точки заброски~--- чувство комфорта куда-то исчезает. Начинаешь с тоской вспоминать любимую ГТС'ку, на которой, при всех скинутых гусянках, вылетевших пальцах, лопнувших торсионах, проедешь, где угодно. 

Именно стремление к тому, чтобы выполнить ту же работу иначе (быть может~--- более эффективно)~--- и есть, наряду с элементарным любопытством, мотивом к переходу на Linux. Что же касается эффекта <<повелителя мух>>~--- не думаю, что это может послужить стимулом для человека, выросшего при советской власти и не ставшего партийным секретарем. Системой не нужно повелевать, достаточно её любить~--- и она, как женщина, ответит взаимностью\dots 

Так что на вопрос из заголовка заметки я отвечу однозначно отрицательно. И не буду призывать к поголовному переходу на Linux. Тем более, что ждать от него удвоения производительности не следует~--- последний раз такое случилось при переходе с XT'шки на AT'шку. Все дальнейшие upgrade-потуги были направлены (благодаря, к слову сказать, отцу Уындовсу иже с ним) только на сохранение \textit{status quo}. Однако задуматься, а есть ли у него мотивы для такого перехода~--- не повредит любому представителю творческой профессии. Ведь ныне~--- именно он самый медленный компонент компьютерной системы. И разве плохо повысить собственную производительность хотя бы на 5-10 процентов? А дальше уже вопрос скорее upgrade собственной личности, но никак не системы\dots

\section{Феномен Linux~--- в истории и перспективе} 
\begin{timeline}2002\end{timeline}

\textsl{Актуализирующее вступление: эта статья сочинялась для бумажной Компьютерры, где и была опубликована в начале 2002 г. (вероятно, и по сию пору доступна в онлайновой версии). А актуальной она показалась мне в связи с дискуссией об идеальном дистрибутиве на одном из форумов. Текст отредактирован лишь в стилистическом отношении~--- содержательная его часть осталась практически без изменений (исключая пару актуализирующих комментариев). А потому прошу учесть~--- писалось это скоро как три года назад, и кое в чем мои взгляды изменились.}\medskip

В этой заметке рассказ пойдет в особенности об операционной системе Linux~--- именно она символизирует явление, которое вошло в историю как феномен Open Source. Хотя Linux и не одинока на этой стезе, но только она породила множество прямых потомков в лице сотни дистрибутивов и даже послужила объектом для подражания (с точки зрения если не буквы, то~--- духа). В качестве примера можно привести ОС AtheOS. Да и реанимация проекта Hurd после многих лет вялотекущего существования, думается, во многом вызвана популярностью Linux.

В общем, то, что происходит в последние два-три года (\textsl{напоминаю~--- отсчет идет от конца 2001-го}), получило название Linux-бума. Однако ни один бум не может длиться вечно. Чувство восторженной новизны со временем ослабевает, события входят в более спокойное русло, и возникает вопрос: каким окажется будущее Linux? Попробуем представить его, обратившись к истории.

Начну с момента, когда, как мне кажется, ОС Linux впервые обратилась лицом к конечному пользователю. Волею случая, именно тогда началось и мое приобщение к Linux, поэтому я могу говорить о том, что в значительной мере происходило на моих глазах. За точку отсчета приму выход Linux Mandrake 5.1~--- первого произведения Гаэля Дюваля сотоварищи (номер версии тут смущать не должен~--- он соответствует таковому Red Hat, от которого Mandrake отпочковался). Утверждение это может показаться спорным. Ведь к тому времени большинство популярных дистрибутивов далеко отошло от сборки <<с помощью паяльника и осциллографа>>~--- А. Суханов), предоставив достаточно удобные средства установки и конфигурирования. Однако средства эти предназначались все же скорее сисадмину, чем рядовому пользователю.

Mandrake же изначально принял во внимание интересы именно конечного, в том числе и домашнего, пользователя. В него в качестве стандартной была включена первая и тогда единственная работоспособная интегрированная визуальная среда~--- KDE (поклонники Gnome меня, конечно же, не простят, но назвать его работоспособным тогда~--- как, впрочем, и сейчас,~--- можно было с большой натяжкой; а XFce~--- и до сих пор не назовешь по-настоящему интегрированной средой. Именно KDE позволила преодолеть психологический барьер множеству пользователей, не впитавших с молоком матери пристрастия к \texttt{sh} и \texttt{vi}.

Кроме того, Mandrake исходно был интернациональным по сути. Его разработчики пытались, хотя бы теоретически, охватить все изобилие языков нашей маленькой планеты. Последний фактор сыграл решающую роль в судьбе Mandrake на постсоветском пространстве: с локализации пакета (IPLabs Linux Team, ныне Altlinux) по-настоящему началась разработка отечественных дистрибутивов. Предтечами чего выступили и диски УрбанСофта, и Красная Шапочка, и украинские KSI и Black Cat. Однако в начале непрерывной традиции русскоязычия Linux~--- именно Linux Mandrake Russian Edition.

Следуя веяниям времени, по тропе, намеченной Mandrake, двинулись и прочие разработчики. Один за другим дистрибутивы приобретают красивые графические программы инсталляции, удобные средства настройки, расширяют круг поддерживаемых языков, обрастают мультимедийными приложениями. Однако~--- не все: ряд систем не просто сохраняет верность стилю True Linux, но и развивает его.

В итоге к настоящему моменту привычная классификация дистрибутивов на Red Hat-, Slackware-, Debian-клоны оказалась размытой. С одной стороны, появились кросс-дистрибутивные средства управления пакетами, формат которых~--- одна из основ традиции (такие, как исходно Debian'овский 
\texttt{apt}
, ныне успешно прикрученный ко многим rpm-based системам). А благодаря усилиям по стандартизации Linux будет, надеюсь, нивелирован и другой её столп~--- различие логической структуры файловых систем. С другой стороны, клоны основных систем (тот же Mandrake, или, паче того, Corel Linux) настолько отошли от своих прототипов, что об их происхождении подчас не напоминает даже формат пакетов (примером чему~--- Suse, прототипом которой некогда послужила Slackware).

И потому единственный критерий классификации дистрибутивов ныне~--- это их ориентация~--- на конечного ли пользователя или на системного администратора (речь идет о универсальных дистрибутивах. Специализированные системы с нестандартной, так сказать, ориентацией, выступающие под девизом <<Linux на одной дискете>>, здесь не рассматриваются). Под чем отнюдь не подразумевается их назначение~--- настольное (в первом случае) или серверное (во втором): и те и другие могут с успехом использоваться в обоих качествах. Просто в user-ориентированных системах не забыто, что администратор Web- или LAN-сервера~--- тоже пользователь, то есть человек, и ничто человеческое (в частности~--- и здоровая лень в отношении ручных настроек) ему не чуждо. Вторые же напоминают о том, что на настольной машине любой пользователь, как правило,~--- сам себе root и может испытывать желание сделать руками все.

Отсюда вытекают основные различия выделяемых групп. Типичные представители первой~--- современные версии Red Hat, Suse, Caldera OpenLinux, Mandrake, почти все вариации на тему Altlinux, ASPLinux и многие, многие другие. По умолчанию им свойственны: установка в графическом режиме (конечно, порядочные люди, как правило, предусматривают и возможность текстовой инсталляции в ручном режиме, однако это лишь дополнительная фича для нештатных ситуаций), автоматизация создания дисковых разделов и настройки основных параметров, выбор предопределенных групп пакетов~--- скажем, для разработчика, для сервера, для офисного применения. Апофигей этой группы~--- Corel Linux, в котором текстовой установки нет как класса, а многие настройки только автоматические.

Кроме того, user-ориентированные дистрибутивы постепенно обзаводятся едиными системами конфигурирования, функционально подобными 
\textbf{Панели управления}
 из Windows,~--- здесь <<впереди планеты всей>> Mandrake: его 
\textbf{Control Center}
 способен решить почти все пользовательские задачи в автоматическом режиме (включая выдвижение лотка CD-привода для вставки диска, содержащего требуемую библиотеку).

Вторая группа представлена в перую очередь Slackware и Debian, их клонами и, наконец, RockLinux, который можно поднять над ними в качестве штандарта (\textsl{со временем к ним добавились Gentoo, CRUX, Archlinux}). Здесь~--- все иначе: аскетичная установка в текстовом или, в лучшем случае, псевдографическом режиме, ручное создание разделов утилитами \texttt{fdisk} или \texttt{cfdisk} и ручной же выбор пакетов, локализация посредством текстового редактора (он же~--- и главный инструмент конфигурирования вообще). Короче говоря, перечисление отсутствующих возможностей способно вызвать дрожь у Windows-пользователя.

Тем не менее, популярность sysadmin-ориентированных дистрибутивов отнюдь не снижается. Почему? Ответить нетрудно. Именно они дают полную свободу в настройке системы под свои потребности. И пользователь Windows, начавший знакомство с Linux с продукции Corel (в надежде без напряга обрести мощь UNIX), очень быстро понимает, подобно цитируемому товарищем Бендером персонажу, что все в жизни имеет оборотную сторону.

Конечно, приятно, когда сразу после инсталляции можно общаться с системой на родном языке, печатать (в том числе и фотореалистичные изображения), писать CD (не вникая в параметры загрузки ядра) и слушать музыку (не собирая требуемых для того модулей). Однако <<раскошеливание на похороны>> в данном случае заключается в огромном количестве автоматически установленного софта (о назначении большинства представителей которого пользователь рискуете узнать никогда). Или, напротив, в отсутствии чего-то жизненно необходимого~--- так, в Mandrake при выборе офисной установки невозможно скомпилировать простейшую программу~--- никакие инструменты для этого стандартным набором не предусмотрены.

Конечно, все огрехи автоматизации в user-ориентированных дистрибутивах устранимы. Однако подчас это требует не меньших (если не больших) усилий, чем, скажем, установка русской локали, шрифтов и клавиатурных раскладок с нуля. А главное~--- пользователь, понадеявшийся на легкость установки и освоения современных дистрибутивов, оказывается морально не готов к такому повороту событий.

Итак, можно сказать, что если традиционные дистрибутивы пытаются возвысить юзера до сисадмина, то user-ориентированные~--- адаптируют систему до уровня понимания юзера. И расслоение Linux на две эти группы~--- реальность сегодняшнего дня.

Что же ждет нас завтра? Я, честно говоря, не верю в разлив основного Linux-русла по десктопам широких пользовательских масс~--- эта ОС займет место на винчестерах достаточно ограниченного круга юзеров. И~--- специфического, тех, кто в итоге приходит к выводу о необходимости разобраться в той или иной мере в устройстве системы. И потому крайние user-ориентированные возможности современных дистрибутивов останутся невостребованными~--- не в этом ли причина блистательного провала Corel Linux?

Тем не менее, возврата к эпохе паяльника и осциллографа уже не будет~--- к легкой жизни привыкаешь быстро. Так что следующий этап развития Linux видится мне в достижении разумного компромисса между автоматизацией и ручной работой. А вот какова будет форма этого компромисса~--- об этом <<давай хотя бы помечтаем>> (Тимур Шаов).

Рискну предположить, что большинство пользователей, независимо от квалификации, с удовольствием избавятся от необходимости собственноручно прикручивать принтер, сканер и прочее железо, передоверив эти функции программе установки. Не ожидаю активного протеста и против корректной автоматической локализации системы и приложений (включая спеллинг, шрифты и прочее).

А вот что следует оставить сугубо на усмотрение пользователя~--- это комплектацию системы. Тем более что штатные наборы пакетов отражают не только (а подчас и не столько) их взаимозависимости, сколько представления авторов дистрибутива о потребностях типичного разработчика или офисного труженика.

Возможно, это покажется беспринципным, но контроль зависимостей пакетов я, напротив, отдал бы на откуп системы~--- разбираться с этим большинству пользователей лень. Вернее, специальной интегрированной программы, позволяющей доустанавливать или удалять любые программные компоненты. К тому же мы знаем примеры более чем успешной реализации такого подхода~--- что в текстовом режиме (\texttt{pkgtool} в Slackware), что в графическом (тот же \textbf{Control Center} в Mandrake).

А еще лучше, если программа установки софта будет одновременно и общим конфигуратором системы. Желательно~--- идентичным по интерфейсу с программой установки самой ОС.

Если суммировать все сказанное и довести до логического завершения, мы получим. нечто вроде FreeBSD с её \texttt{sysinstall}, не так ли? Внешне, разумеется, а не с точки зрения внутреннего устройства ядра, файловой системы \textit{etc}.

Это я не к тому, что следует сменить Linux на *BSD (в этих системах у пользователя возникнут свои проблемы). Однако почему бы не присмотреться к тому, что делается в стане братьев во Open Source? Братьев по классу, по духу~--- ну и пусть, что не по лицензии.

\textsl{Post Scriptum сего дня: с радостью должен отметить, что, подобно большинству пророков, я также был посрамлен в своем пророчестве~--- о невозможности возврата к эпохе паяльника и осциллографа. Иначе чем объяснить столь широкую популярность Gentoo, не имеющего ни инсталлятора, ни конфигуратора? Или~--- течение Linux from Scratch, время от времени приобретающее характер эпидемии. И если последнее~--- явление сугубо возрастное (не в биологическом, понятно, смысле~--- на определенной стадии своего развития большинство пользователей Linux испытывает непреодолимую потребность собрать собственную систему), то Gentoo прочно занял свое место среди популярных дистрибутивов, потеснив и Debian, и Slackware. А соплеменные ему по духу CRUX и Archlinux обрели пусть и не столь широкую, но вполне сложившуюся пользовательскую базу.}

\section{Еще немного о посетителях и комментариях} 
\begin{timeline}20 марта 2007 г\end{timeline}

\hfill \begin{minipage}[h]{0.45\textwidth}
Скажите, если у нас все так хорошо, а у них все так плохо, то почему у нас все так плохо, а у них все так хорошо?
\begin{flushright}
\textit{Старый советский анекдот}
\end{flushright}
\bigskip\end{minipage}

Прочитал я соображения Владимира Попова в заметке <<Поговорим?>>~--- об особенностях общения on-line, и захотелось мне дополнить их рассмотрением одного частного случая Сетевых комментаторов. 

Время от времени на ресурсах, ориентированных на Unix, Linux и Open Source, появляются пользователи очень интересной категории~--- восхвалители Windows. Которые в обществе линуксоидов начинают доказывать, какой Linux плохой и какая Windows хорошая. Когда же им, сначала вежливо, пытаются объяснить, что они нашли для своих выступлений неподходящее место, очень возмущаются и начинают говорить чуть ли не о нарушении прав человека. И уж во всяком случае не преминут обвинить сообщество в необъективности. Дескать, хвалить <<линух>> и ругать <<венду>> тут можно, а вот критиковать первый и рассказывать о преимуществах второго~--- нельзя.

Что на это ответишь? Да, так оно и есть. И рассказывать на Linux-форумах о достоинствах Windows и недостатках Linux\dots ну не то чтобы нельзя, а просто бессмысленно. Большинство посетителей таких ресурсов давно знают все преимущества Linux и все его недостатки. И столь же давно научились эффективно использовать первые и бороться со вторыми. Или эти самые недостатки их просто не волнуют. Как не расстраиваюсь я, например, из-за того, что моя кофемолка не имеет встроенного аудио-плейера. А несравненные достоинства Windows интересуют действующих линуксоидов ничуть не больше, чем жителя средней полосы России~--- погода в Новой Зеландии.

Ситуация складывается примерно такая же, как если бы в клуб собаководов пришел заядлый кошатник и начал бы читать лекцию об особенностях разведения персидких кошек. Хотя нет, сравнение не точное: большинство собаководов просто любит животных вообще, и потому, вполне вероятно, с интересом послушают и рассказ о кошках. Как, впрочем, и наоборот\dots

Вот более точная аналогия. Представьте себе компанию любителей мяса, разговаривающих о том, как лучше есть сырую лосятину~--- слегка присолив её, или еще и диким лучком присыпав? И вот в ней появляется убежденный вегетарианец, и начинает рассказывать о несравненном вкусе морковных котлет, их целительной пользе для здоровья, делиться тонкостями рецептов приготовления оных. А заодно и объяснять своим собеседникам, что мясо вредно, да и есть его вообще невозможно.

Ясно, что ничего хорошего из этого не выйдет: вегетарианца нашего в лучшем случае вежливо проигнорируют (в худшем~--- могут и послать на известный ресурс, некогда основанный Леонидом Кагановым). И самому ему придется слушать о таких вещах, которые, возможно, вызывают у него физическое отвращение.

Именно в таком положении оказывается любитель Windows на любом Linux-форуме. Собеседники ведут разговор о непонятных и неприятных для него вещах~--- каких-то консолях, шеллах, командах. А он начинает парить им мозги, какие замечательные эффекты есть в Windows Vista и как быстро запускается MS Word. А заодно и доказывать, что консоль и вообще Linux~--- это все плохо, работать там неозможно, и вообще~--- <<туфта этот ваш линух>>.

Причем что интересно: среди моих реальных и виртуальных знакомых~--- действующих линуксоидов, нет ни одного, кто ходил бы по Windows-форумам даже только с целью восхваления своей операционки, не говоря уже о том, чтобы тратить время на критику Windows. А вот обратная картина наблюдается сплошь и рядом. Поневоле приходит на память вопрос, заданный Рабиновичем после лекции о преимуществах социализма над капитализмом. Тот самый, который вынесен в качестве эпиграфа. Не есть ли это свидетельство комплекса неполноценности определенной части Windows-пользователей, развившегося вследствие употребления этой замечательной операционной системы? И подобного тоске вегетарианца по настоящей пище, вкармливавшей Homo Sapiens на протяжении нескольких десятков тысячелетий. Ведь самый простой способ эту тоску заглушить~--- убедить других (да в первую очередь самого себя) в том, что мясо вредно\dots

\section{Какой Linux~--- худший?} 
\begin{timeline}Сентябрь 5, 2009\end{timeline}

\textsl{В соавторстве с Владимиром Поповым, публикуется с его разрешения.}

\hfill \begin{minipage}[h]{0.45\textwidth}
Мещанин понимает~--- пустота не полезна \\
Еда не в прок, и свербит тоска \\
Тогда мещанин подползает к поэзии\\ 
Из чужого огня каштаны таскать
\begin{flushright}
\textit{Михаил Анчаров}
\end{flushright}
\bigskip\end{minipage}

В дискуссиях по вопросу, какой же из дистрибутивов Linux является лучшим, в печати и на форумах сломано больше копий, чем на всех рыцарских турнирах за всю историю Средневековья. А если подойти к проблеме с другой стороны и определить, какой же из Linux'ов~--- худший, и сколько их, таких? Тогда те дистрибутивы, которые не попадут в число худших, автоматически окажутся лучшими, не так ли? Давайте посмотрим.

Поводом для сочинения настоящей заметки послужил материал под таким вот названием: <<2009's 10 Worst Linux Distributions>>. На содержании её задерживаться не будем, как и на рассмотрении прямых фактических ошибок, которых там вдоволь. Важен общий тон, который нас и спровоцировал на данное сочинение.

Автор абсолютно искренне верит, что дистрибутивы Linux создаются с единственной целью: порадовать пользователей IBM PC. Можно сказать больше: основная их масса сегодня уверена, что средства вычислительной техники и развиваются-то с единственной целью: максимального удовлетворения индивидуального\dots нет, даже не пользователя~--- это слово кажется в данном контексте неуместным~--- а потребителя. Ибо пользователь чем-то пользуется, что-то использует или, более того, что-то или кого-то пользует. Как говаривали в старое время, 
\textit{нашу семью пользует доктор имярек}
.

Потребитель же только потребляет. И венцом развития информационных технологий, по его мнению, является электронно-вычислительно-мультимедийно-кофе-в-постель-подавательный комбайн с одной кнопкой, большой и красной: зделайте мну пес\dots хорошо, говоря иначе.

Казалось бы, логично, ага? Всё для человека, всё для блага человека, всё для счастья человека. Не за то ли боролись поколения наших предков, начиная с безымённого кроманьонца плейстоценовой тундростепи и кончая героями информационной революции?

Однако если довести это рассуждение до логического предела, то получится, что основное назначение:

\begin{itemize}
	\item пороха~--- пиротехника,
	\item высокомолекулярного синтеза~--- полиэтиленовые кульки, 
	\item электротехники~--- лампочка в сортире, а 
	\item радиохимии~--- <<светящиеся>> стрелки наручных часов. 
\end{itemize}

Можно согласиться, что всё, что ни делается в мире, делается для <<всё более полного удовлетворения>>. Но, для удовлетворения~--- кого? Не стоит ли при этом отметить, что для некоторых представителей вида (как минимум от Пифагора до Эйнштейна) удовлетворение было накрепко связано с познанием и творчеством?\dots

Впрочем, где-то на уровне подсознания, потребитель догадывается, что кое-что делается и не для него вовсе. И не так, как ему представляется естественным. Он не пытается советовать машиностроителям делать корпуса из бумаги только на том основании, что он попробовал: удобно резать ножницами. Не требует понижения напряжения в ЛЭП до 220 вольт только потому, что ему так привычнее. И не решает: какой тип двигателя должна использовать баллистическая ракета только потому, что один из этих типов (внутреннего сгорания) даже ему известен.

Интересно, что при этом тот же потребитель охотно судит о том, какими должны быть ОС. Наверное, наивно полагая, что разговор затеян исключительно для того, чтобы, в конечном счёте, сделать его эпизодическое общение с компьютером максимально приятным и, по возможности, бездумным. По аналогии с автомобилями это выглядит как требование перехода к автоматической коробке передач абсолютно для всего, что движется на колёсах (один из сочинителей попытался представить себе ГТТ с автоматической коробкой~--- и аж заколдобился).

Однако, помилуйте! Мы вовсе не против комфортности современных седанов, но оставьте в покое карт. И локомотив. И седельный тягач, автобус, болид F-1 и всех прочих представителей семейства <<самоедущих>>. Поверьте: даже ваше личное удовлетворение зависит отнюдь не только от мягкости сидений в вашем авто. Но и от эффективности грузоперевозок, общего технического прогресса и количества энтузиастов, желающих в нём участвовать.

Так и с вычислительной техникой. Не всё, что обсуждается, имеет конечной целью удовлетворение индивидуального пользователя IBM PC. Более того: как раз те, кто делает товар, ориентированный на конечного покупателя, не особенно-то к <<базарному>> общению и склонны: приоритеты другие.

Так и с дистрибутивами Linux. Одни из них~--- просто полигоны или <<тренировочные залы>> для обкатки технологий, которые потом будут использованы в промышленных системах (Fedora, OpenSuse, если выйти за пределы собственно Linux'а~--- OpenSolaris). Другие~--- оригинальные разработки, ориентированные на решение определённых задач, пусть даже и очень широкого круга (RHEL, Suse, Debian и бессчётное количество более иных). Третьи~--- узкоспециализированные системы или конструкторские наборы для их построения. Четвёртые~--- <<начальная школа>> для потенциальных разработчиков или платформы для разработчиков <<кинетических>>.

Разумеется, все они могут применяться (и успешно применяются) в качестве обычных настольных станций. В том числе и пользователями, профессионально с вычислительной техникой не связанными. Но ни одна из этих систем не рассчитана на бездумное потребление: не предусмотрели в них разработчики той самой сакраментальной красной кнопки. Так что с точки зрения потребителя все эти дистрибутивы <<плохи>>. Разница между ними лишь в том, что одни свою <<плохость>> маскируют красивыми инсталляторами, другие же (подобно Gentoo или Slackware) сразу лишают иллюзий относительно своих потребительских качеств.

Лишь единичные дистрибутивы, подобно Ubuntu или Mandriva, ориентированы, казалось бы, приблизительно на тот же круг пользователей, с опорой на который начиналось победное шествие MS Windows. То есть: на использование непрофессиональным потребителем на персональном (не в архитектурном, а в <<правовом>> смысле) компьютере.

Да и в отношении пресловутой Ubuntu надо подчеркнуть, что она создавалась для применения лишь \textbf{в том числе и} на персональном компьютере. Потому что, конечно же, Шаттлворт, начиная этот проект, рассчитывал (и рассчитывает, как можно прочитать между строк его многочисленных заявлений) на выход в сферы всамделишних применений~--- промышленных серверов, мобильных систем \textit{etc}.

Только действует он методами, противоположными, скажем, политике комании Red Hat. Которая отказалась от развития пользовательской линии своего дистрибутива вообще, перевалив это дело на плечи энтузиастов из сообщества. И сосредоточила усилия на внедрение в корпоративный сектор.

Нет, экспансия Ubuntu идёт тем путём, каким шла Microsoft, продвигая Windows~95 как игровую и медиа-платформу бытового назначения. Ведь мало кто помнит нынче, что до 1995 года Windows никем всеръёз не воспринималась~--- ну ещё одна графическая псевдомногозадачная оболочка над DOS, не более.

Но после 1995 года эта оболочка\dots А Windows~95, что бы ни говорили, оставалась оболочкой, что сразу по её выходе блестяще показал Эндрю Шульман в своей книге Неофициальная Windows'95\dots Так вот, эта оболочка пришла в дома. И, как пророчески заметил тогда же Георгий Кузнецов (в то время главный редактор Компьютерры) её неизбежно должны были принести на работу. Что, собственно, и произошло~--- тем более, что промышленная платформа в виде Windows~NT была заготовлена загодя. Только о том, что эта Windows~--- внутри вовсе не та Windows, предпочитали не говорить вслух.

\textsl{Отступление: можно припомнить, что в 1997-1998 года, скажем, хостинг на Windows-серверах предлагал чуть ли не один-единственный провайдер на всю Москву. Это была отдельная услуга, которая стоила вдвое дороже, чем стандартный хостинг на UNIX-машинах. Видимо, с намёком на то, что администрировать сможет любая блондинка. А сколько нынче Web-серверов крутится под разными версиями Windows?}\medskip

Примерно таков же метод действия Canonical: сначала~--- бесплатная рассылка максимально <<юзерофильного>> дистрибутива с целью создания пользовательской базы. И параллельно~--- фоновая разработка специализированных решений: для промышленных серверов, мобильных систем \textit{etc}. Правда, в отличие от случая с Windows, когда в потребительскую и промышленную сферы продвигались две принципиально разные ОС, тут в любом случае речь идёт о Linux, внутренне одной и той же системой, различающейся, в зависимости от назначения, лишь <<обвязкой>>.

Так что и потребителям Ubuntu не следует обольщаться~--- этот дистрибутив для них столь же <<плох>>, как и все остальные.

В итоге сказанного мы приходим к выводу, что с потребительской точки зрения все дистрибутивы Linux одинаково <<плохи>>, будь они столь же разукрашеными, как раскрашенный Штирлиц из <<цветной>> версии <<Семнадцати мгновений весны>>.

Рассужадая же обратным порядком, как кадий Абдуррахман из <<Повести о Ходже Насреддине>>, мы приходим к выводу, что все дистрибутивы Linux одинаково хороши~--- для пользователя профессионального, знающего, что ему нужно от системы. И пользователю этому совсем не обязательно быть профессионалом именно в IT-сфере: достаточно просто не рассчитывать на большую красную кнопку, а создавать её своими руками, применительно к собственным целям и задачам.

\section{Трибализм в мире FOSS} 
\begin{timeline}Август 8, 2010\end{timeline}

\hfill \begin{minipage}[h]{0.45\textwidth}
Был митинг на заводе Лихачёва \\
И на заводе Красный пролетарий:\\
Убил Лумумбу враг народа Чомбе,\\
Ему помог Мобуту с гнусной харей.
\begin{flushright}
\textit{Народное}
\end{flushright}
\bigskip\end{minipage}

Недавно опубликованная заметка Марка Шаттлворта о трибализме в сообществе Open Source~--- <<Tribalism is the enemy within>>~---  довольно активно обсуждается как в мировом масштабе, так и в Рунете. Последнее, в конце концов, вызвало к жизни заметку Владимира Попова <<Трибализм ``от Марка''>>, которая и послужила поводом для заметки настоящей.

Но начнём по порядку~--- с эпиграфа.
 Именно во времена сочинения этой песенки в русском языке и появилось слово трибализм~--- как обозначение идеологии африканских вождей, по мере освобождения Африки от гнёта проклятых колонизаторов постепенно превращавшихся в президентов, а то и императоров.

В более широком смысле трибализм свойственен племенам и народам, долгое время существовавшим в условиях полной изоляции, например, эскимосам и индейцам атапасской группы. И подчас не подозревавших о том, что существуют ещё и другие люди. во всяком случае, у эскимосов Туле (самый север Гренландии) первая встреча с европейцами в начале 19-го века вызвала шок: до тех пор они полагали, что кроме них на свете существуют только белые медведи, моржи, тюлени и прочая живность.

Не случайно общее самоназвание и эскимосов, и атапасков было очень простое~--- люди. Когда же они столкнулись с представителями других племён и народов, они несколько конкретизировали своё имя, назвавшись людьми настоящими.

Ничего не напоминает? Мне~--- так даже очень. А именно~--- сообщество Ubuntu. Для изрядного числа новообращённых убунтоидов
\large\begin{shadequote}{}
Ubuntu == Linux
\end{shadequote}\normalsize

И, подобно эскимосам и атапаскам, они не подозревают о существовании других дистрибутивов Linux (о прочих UNIX-подобных системах FOSS-мира и говорить излишне). Чем и демонстрируют пример племенного мышления (именно так в русскоязычных новостях перевели термин трибализм). Но что понятно и простительно для племён, заброшенных на край ледяной пустыни, в места, где человек выживает на пределе своих возможностей (а подчас~--- и за пределами их), выглядит довольно странно в наш благополучный век тотальной информатизации. Так что Марку следовало бы начинать борьбу с трибализмом с сообщества, сформировавшегося вокруг его собственного дистрибутива.

Причём именно в Ubuntu и его ближайших сородичах трибализм наиболее активно воплощается в жизнь практически~--- путём создания <<не такого, как у всех>>, проявляющегося даже в мелочах. Достаточно вспомнить локализации KDE в Kubuntu, в которых на протяжении долгого времени хладнокровно игнорировались работы команды интернационализации собственно проекта KDE. И таких примеров можно привести ещё много.

Нельзя сказать, что сообщества пользователей более иных дистрибутивов совсем свободны от трибализма. Но обычно он менее бросается в глаза по ряду причин. в частности, и потому, что большинство этих самых более иных пользователей перепробовали более одной системы, имеют представление о мире FOSS в целом, и их нынешний выбор~--- вполне осознан.

Среди пользователей Ubuntu таких тоже немало. Но количественно они просто теряются на фоне новообращённых трибалистов\dots

\section{Унификация интерфейсов: во зло или во благо?} 
\begin{timeline}Октябрь 17, 2011\end{timeline}

Разговоры об унификации пользовательских интерфейсов ведутся очень давно~--- с тех стародавних времён, когда уже забылось, что Венечка Ерофеев был бригадиром на кабельных работах. Но вот вопрос о том, зло это или благо, унификация?~--- уже и не поднимается. Казалось бы, это останется покрыто таким же мраком неизвестности, как то, во зло или во благо пил Венечка между улицей Чехова и неведомым подъездом, в котором он проснулся поутру:

\begin{shadequote}{}
Никто этого не знает, и никогда теперь не узнает. Не знаем же мы вот до сих пор: царь Борис убил царевича Димитрия или же наоборот?
\end{shadequote}
Однако буквально сегодня коллега Hymnazix aka Сергей Голубев неожиданно поднял эту тему в своём блоге в статье: <<Пользовательские интерфейсы: постановка задачи.>> Прочитав её, я вдруг обнаружил, что давненько не брал в руки шашечку\dots то есть не писал пасвилей и памфлетов. И ощутил непреодолимое желание возобновить это занятие. Что и делаю.

Прошу понять правильно: это памфлет не на статью Сергея, с которым я связан давними товарищескими отношениями~--- она послужила только поводом. А на\dots впрочем, проницательный и терпеливый читатель, асиливший мою заметку до конца, сам поймёт, на кого и на что.

Начну я, однако, с цитатирования заключительного абзаца статьи Сергея, так сказать, её квинтэссенции:
\begin{shadequote}{}
С точки зрения пользователя интерфейс должен быть унифицирован\dots если я научился выполнять какую-то операцию в одной системе, то с другими у меня никаких проблем не возникнет.
\end{shadequote}

Точнее, автор подразумевает, что не должно возникнуть. Из чего следует, что унификация интерфейсов понимается им как абсолютное благо. Давайте же посмотрим, чем это обосновывается.

А обосновывается это аналогией с автомобилестроением~--- приём не новый, но от этого не перестающий быть менее жизненным. Так что следующая цитата:
\begin{shadequote}{}
\dotsводитель, один раз освоив управление машиной, запросто меняет модели, не задумываясь о том, где на новой руль, а где педаль газа. Он твердо знает~--- точно там, где им надлежит быть.
\end{shadequote}

Отлично.  А теперь представим, что ему предстоит ездить по горно-тундровой местности~--- ну жизнь так вот сложилась. И потому новая его модель~--- не Лексус или Порш следующего в натуральном ряду чисел номера, а Гусеничный Транспортёр Санитарно-Медицинский ГАЗ-71 (это его паспортное название, в просторечии~--- ГТСка).

Для начала наш водитель  с удивлением обнаружит вместо руля рычаги, а тормоза не найдёт вовсе. А если он сядет не на ГТСку, а на ГГТ, удивление его возрастёт ещё больше -– он не найдёт там рычаг переключения передач, и будет довольно долго смекать, как ему таки включить пониженную передачу.

Столь же верно и обратное: при некотором опыте вождения гусеничного транспорта, усевшись на колёса, остаёшься в недоумении~----куда девались  рычаги, которыми интуитивно понятно, как поворачивать, а толкнув которые от себя одновременно оба два, можно столь же прозрачно для пользователя переключить передачу. И почему эта падла не останавливается на месте, когда я снял ногу с педали газа?

Можно ли унифицировать <<интерфейс>> гусеничных вездеходов и городских легковушек? Вероятно, если поднапрячься, то можно (и такие попытки даже были на практике~--- замена рычагов на штурвал в некоторых гусеничных машинах). Только вот будет ли кому от этого лучше? Ведь гусеничные и колёсные машины приспособлены для решения совершенно разных задач, и ожидать от них одинакового удобства унифицированного <<интерфейса>> было бы нелепо.

В своё время, без малого 20 назад, Джим Сеймур, колумнист журнала PC Magazine, высказался относительно унифицированных интерфейсов примерно так (по памяти, бумажый журнал давно потерян, а в сети номеров тех лет нет и уже, наверное, никогда не будет):
\begin{shadequote}{}
\dotsу моей магнитолы и моего телевизора разные кнопки управления, и это ничуть не портит мне жизнь.
\end{shadequote}

Сказано это было в годы появления первых интегрированных офисных пакетов, каковые начались с унификации интерфейсов электронных таблиц и текстовых процессоров, в них включаемых. Некоторые читатели, возможно, помнят, что до того времени в качестве стандартного интерфейса электронных таблиц рассматривался стиль Lotus 1-2-3~--- Lotus и Borland даже долго судились по этому поводу (пока обе фирмы не накрылись медным тазом~--- в том числе и вследствие собственного сутяжничества). Самый же распространённый текстовый процессор тех лет, WordPerfect, имел интерфейс совершенно отличный. И тем не менее, оба они прекрасно справлялись со своими задачами в руках одних и тех же зачастую пользователей. Которым различие структуры меню и <<горячих>> клавиш мешало ничуть не больше, нежели Джиму Сеймуру~--- разные кнопки/ручки на магнитоле и на телевизоре.

Что же получилось после того, как в рамках MS Office  интерфейс электронных таблиц и текстовых процессоров привели к общему знаменателю? А получилось, как всегда в таких случаях: стало очень неудобно работать и с теми, и с другими, а вслед за ними и с прочими <<унифицированными>> программами. Нынешнее поколение этого уже не осознаёт, потому что, как известно, <<стерпится~--- слюбится>>. Но оценить величину потери могут только те, кто видел бывших секретарей-машинисток, делопроизводительниц и референтш, виртуозно порхающим пальчиками по горячим клавишам Лексикона или WordPerfect'а. Или тот, кому самому довелось обрабатывать огромные электронные таблицы полностью на рефлекторном уровне, нимало не пудря себе мозги теориями интерфейсов.

И это касается интерфейса программ если не родственных, то и не очень далёких~--- условно, офисного назначения (хотя изначально электронные таблицы были скорее орудием инженеров, нежели клерков). Что же доброго может получиться от унификации интерфейса систем твердотельного моделирования и медиапроигрывателей? Не ГТТ ли, управляемый ручками от магнитолы?

Так что единственное добро можно сформулировать опять таки словами Сергея:
\begin{shadequote}{}
Цель унификации~--- минимизация нагрузки на память.
\end{shadequote}
Вот только действительно ли это добро? Как известно, любой орган человеческого, не упражняемый должным образом, атрофируется. А мозг человеческий, управляющий, в том числе, и памятью, этому подвержен чуть ли не в наибольшей степени. И напротив, банальное школьное упражнение по заучиванию стихов наизусть весьма тренировке памяти способствует. Так что всеобщая унификация интерфейсов, помимо чисто технических недостатков, приведёт к дальнейшей деградации человечества.

В этой связи стоит опять обратиться к статье Сергея~--- теперь к самому её началу, в связи с чем~--- очередная цитата:
\begin{shadequote}{}
На пресс-конференции РАСПО <<Создание Национальной программной платформы~--- важный шаг на пути построения информационного общества>> был задан вопрос о пользовательских интерфейсах\dots

По реакции участников пресс-конференции я понял, что представители большого СПО-бизнеса не склонны считать это детским садом, не заслуживающего их внимания.
\end{shadequote}
Как можно понять из всего опубликованного по этому вопросу, НПП предназначена в существенной мере (если не главным образом) для служащих государственных структур. И если при её разработке немалое внимание уделяется <<унификации интерфейсов>>~--- не есть ли это молчаливое признание того, что большинство наших госчиновников не в состоянии запомнить разницу между ручками магнитолы и телевизора? Тех же, кто это способен асилить, не надо ли срочно унифицировать до общего уровня? Путем принудительного внедрения унифицированных интерфейсов, минимизирующих нагрузку на память и тем самым способствующих её атрофии\dots

В своём памфлете я затронул лишь одну из граней проблемы интерфейсов~--- на самом деле их много. Тут и действительно универсальные интерфейсы, хотя и в своей сфере применения~--- это CLI, интерфейс командной строки. И преемственность интерфейсов в рамках одной линии программ~--- точнее, как мы видим в последние годы, ничем не обоснованный отказ от такой преемственности, и многое другое. Но нельзя объять необъятное\dots

\section{О гномах, окнах и МММ, или кредит доверия} 
\begin{timeline}Июнь 7, 2012\end{timeline}

В этой заметке не будет никаких сравнений, и поэтому отнести её к сериалу про сравнение мужей я не могу. Тем не менее она носит чисто этнографический характер. А поскольку дальнейшее развитие событий даст для заметок этой направленности ещё много поводов, видимо, придётся учреждать соответствующую рубрику.

Поводом для этой заметки послужило очередное высказывание Линуса в адрес GNOME~3.  Которое заставило обратиться к истории\dots нет, сначала не GNOME, а совсем другой среды. Той, что когда-то звали просто оболочкой DOS (злые языки добавляли~--- дешёвой). И которая потом стала Операционной Системой (причём весьма дорогой).

Как вы, конечно, догадались, речь идёт о Windows. Появившись в 1-й своей ипостаси в 1985 году как (одна из многих) графических оболочек для DOS, она не умела ничего, кроме как запускать себя и свои приложения, которых тогда не было. Версия 2-я научилась, хотя и не сразу, запускать DOS-программы в квази-многозадачном режиме. Ну а начиная с версии 3.1, она называлась гордым именем Операционной Среды, в которой, несколько приноровившись, можно было работать.

Windows~95 получила уже титул Операционной Системы, хотя по сути своей оставалась всё той же надстройкой над DOS. Как и сменившая её Windows~98. В последнем легко убедиться покупателям ноутбуков, в прайсах стыдливо помеченных <<без ОС (DOS)>>. Далеко не всегда за аббревиатурой в скобках скрывается всем известная FreeDOS: часто это та самая MS DOS 7, которой в отдельном виде никто не наблюдал.

Однако я отвлёкся. Опять же, Windows~95 имела массу недостатков, которые честно признавались производителем (по крайней мере, некоторые из них). Но столь же честно обязалось, что уж в следующей версии они обязательно будут исправлены.

И некоторые действительно исправлялись в Windows~98 и, особенно, её сервис-паках, и по выходе SP2 эта Windows (почему именно эта~--- скоро скажу) стала совсем уже настоящей. Что было неправильно~--- и для коррекции тут же появилась Windows~ME, применение которой наталкивалось на ряд трудностей. Но зато она оказалась последней в своей линейке.

Потому что параллельно развивалась другая Windows, с самого начала представлявшая собой настоящую операционную систему: Windows~NT. Только вот для простого народа она не предназначалась, позиционируясь как серверная, на крайняк~--- как система для рабочих станций.

Но в конце концов обе линии Windows слились воедино~--- в лице сначала Windows~2000, а потом и Windows~XP. Это были последние винды, которые я видел хотя бы краем глаза, и обе были пригодны к использованию. Впрочем, это мнение не только моё, но и многих лично мне знакомых квалифицированных пользователей Windows.

Но потом колесо Фортуны свершило очередной оборот~--- и появилась Windows Vista, о которой те же мои знакомые отзывались не иначе как с использованием обсценной лексики. Видимо, такими слова о ней говорили не только мои личные знакомые, и звучала эта лексика не только на русском языке. Потому что не успело отгреметь горное эхо, поминающее Vista'ему мать, как был выпущен <<психокорректирующий>> релиз в виде Windows~7. Каковой, по словам всё тех же моих знакомых, был лучшей виндой из всех винд в истории виндовства.

Но, как говорится, <<недолго музыка играла, недолго фраер танцевал>>. Потому что грядёт Windows~8, в которой всё опять будет перевёрнуто с ног на голову. Но об этом пусть болит то, на что переворачивают, у тех, кто этим будет пользоваться.

В рамках же этой истории сконцентрирую внимание на двух моментах, которые имеют отношение к этнографии. Первый~--- кредит доверия пользователей. Десять лет, десять дней, десять часов \textit{etc}\dots (с 1985 по 1995). Чёрный Владыка Саурон в Чёрных Горах Мордовии ковал свою супер-мега-ОСь, в которую обещал засадить все свои фичи и все свои баги (ну бэкдоры там и прочие излишества всякие нехорошие). И ему верили~--- потому что каждая фича была обещана ну непременно в следующей версии. Возможно, верили по тому, что по части багов и прочих излишеств он свои обещания выполнял сполна. А подчас и с лихвой.

А второй момент~--- это то, что производители Windows рискнули пускаться на эксперименты типа Vista только тогда, когда окончательно поверили в неограниченность кредита доверия. И теперь, вероятно, с интересом наблюдают~--- схавает ли пипл их <<восьмятое>> порождение? Ничем, кстати, особо не рискуя: схавает~--- хорошо, не схавает~--- <<ах прости, дескать, вышла ошибка>>, ибо старый бронепоезд на запасном пути пары ещё не сбросил.

А теперь вернёмся к истории GNOME. Вряд ли многие помнят, но до-первый GNOME был изделием жутко красивым~--- с Enlightment'ом в качестве оконного менеджера, раздвигающимся при старте занавесом и прочей фанаберией. Недостаток у него был единственный: он достойно умел делать только одно дело. А именно, свопировать. На машинах, где KDE до-первое и первое просто летало, при запуске GNOME индикатор активности винчестера не гас ни на минуту.

Конечно, это компенсировалось всё тем же доверием пользователей~--- ведь GNOME, в отличие от KDE, был истинно свободной рабочей средой (о том, что Qt стало свободной задолго до обретения GNOME хоть какой-нибудь юзабельности, все быстро забыли).

На кредите доверия к свободе GNOME продержался до своей второй версии, в которой в его идеологии произошёл первый крутой перелом: если ранее эта среда позиционировалась как самая совершенная и ни на что не похожая, то теперь её стали выдавать за <<лучшую Windows чем сама Windows>>.

Этого лозунга хватило ненадолго~--- видимо, вследствие одиозности не к ночи помянутого имени в некоторых кругах. И где-то к середине жизненного цикла 2-й ветки (хронологически это примерно 2005 год) происходит метаморфоза: GNOME объявляется средой, самой простой в настройке и использовании. Что, в целом, соответствовало действительности. Правда, простота настройки достигалась тем, что многие из конфигурируемых параметров просто не конфигурировались собственными средствами, а лишь через реестр GNOME Editor.

Интересно, что смена философической парадигмы GNOME совпала по времени и с его широким распространением. Вызванным, правда, не несравненными достоинствами среды, а усилиями фирмы Canonical, распространявшей Ubuntu (с GNOME по умолчанию) бесплатно и <<на каждом километре, здесь и по всему свету>>. А также тем, что Fedora перестала полагать себя <<песочницей Red Hat'а>> и обратилась лицом к пользователю. Возможно, что не последнюю рль сыграл и переход Suse под крыло Novell, незадолго до того прикупившей Mono. Ну и неудачный эксперимент с KDE~4.0, когда очень ранняя тестовая версия получила имя релиза, также сыграл свою роль.

Так или иначе, стечение обстоятельств, не без доброй воли разработчиков, привело к тому, что в конце нулевых годов среда GNOME, во-первых, стала вполне комфортной для работы, и, во-вторых, приобрела сравнимое с KDE обеих веток распространение. Апофеозом чему стала Fedora~14 с GNOME по умолчанию~--- вылизанным и блестяще интегрированным со всеми дистрибутив-специфичными компонентами.

Однако тут наступает время очередного финта ушами~--- или, если угодно, эксперимента над пользователями: GNOME~2 объявляется ретроградным отстоем обскурантов, и на смену ему приходит ультра-революционная, гипер-модернистская и мега-прогрессивная среда GNOME~3, унаследовавшая от предшественницы только название. Ах да, ещё и тенденцию: она была практически лишена собственных средств самого элементарного конфигурирования. Каковые были даже не обещаны в будущем~--- вполне откровенно предлагалось делать их самим.

На такой эксперимент в отношении пользовательского доверия не решалась даже Microsoft\dots Вровень я могу поставить только эпопею с МММ. Впрочем, гражданин Мавроди подошёл к делу более осмотрительно. И выпустил в свет новую свою <<систему>> только тогда, когда значительная часть пользователей первой версии вымерла физически (в том числе и вследствие её <<пользования>>), а потенциальные пользователи версии второй (в те далёкие уже времена ходившие пешком под стол) достигли дееспособного возраста.

Разработчики GNOME~3 уроков Мавроди не учли. Интересно, как это отразится на будущем этой среды? Если никак~--- можно считать, что кредит пользовательского доверия воистину безграничен, и лепить горбатого, почём зря. И не это ли сокровенный смысл проведённого эксперимента~--- как репетиции перед более масштабными экспериментами? Но это~--- тема совсем другого памфлета.

\section{Размышлизм о десктопном Linux’е} 
\begin{timeline}Сентябрь 1, 2013\end{timeline}

Вся история развития десктопного Linux'а, по крайней мере в лице его наиболее известных и распространённых дистрибутивов, несколько напоминает онанизм в библейском (а не вульгарном) понимании этого термина.

Как известно, библейский Онан, вынужденный, согласно обычаю левирата, жениться на вдове своего старшего брата, при общении с ней практиковал один из простейших способов контрацепции~--- прерванный половой акт. Но ведь чем-то подобным занимаются и майнтайнеры ряда дистрибутивов Linux'а, претендундующих на десктопность.

Несколько раз за время своей истории Linux вплотную подходил к той грани оргазма, после которой следует завершить процесс по крайней мере зачатием законченных решений~--- пользовательских и специализированных. И каждый раз процесс прерывался, что обосновывалось неготовностью. И необходимостью перекроить всё~--- на этот раз правильно и окончательно.

Последний по времени пример тому мы видели при всеобщем внедрении \texttt{systemd}'а. Но впереди и пример грядущий~--- Wayland'изация, которая тоже угрожает охватить все дистрибутивы. Или почти все. Потому что есть пример и завершения акта~--- дистрибутивы семейства Ubuntu и возникшие на их базе многочисленные клоны, как общего назначения, так и специализированные для разных сфер применения.

Правда, разработчики тех дистрибутивов, которые гордо именуют себя майнстримом, относят Ubuntu'иодов, вместе со Slackware'щиками и прочими Gentoo'шниками, к категории маргиналов. Однако это именно те маргиналы, суммарное количество которых давно уже превзошло число майнстримщиков~--- но это немного другая тема.

Побуждения библейского Онана понятны: родившийся от его брака с Фамарью сын считался бы сыном его старшего брата Ира~--- со всеми вытекающими из этого правами, в том числе и наследования имущества в обход детей брата младшего, то есть самого Онана. И потому манкирование Онаном своими супружескими обязанностями вполне оправдано в глазах современного здравомыслящего человека. Но не в глазах иудейского бога\dots

Сложнее понять побуждения сторонников гипермодернизма~--- ведь конспирологические версии, что это давно проплачено ребятами из Редмонта, которые, в свою очередь, продались маленьким зелёным человечкам с Марса или сгинувшего Фаэтона, мы отвергаем с негодованием, не так ли? Так что остаётся только допустить панический страх создателей десктопного Linux'а перед его реальной десктопизацией~--- но на эту тему я вроде бы уже писал\dots

Из Библии мы знаем, иудейский бог покарал Онана за его кощунство смертью. Мера наказания гипермоденистов от Linux'а будет куда банальней, и не потребует никакого Бога из Машины. Просто ни один из разрабатываемых ими десктопных Linux'ов никогда не станет по настоящему десктопным\dots

Апологеты \texttt{systemd} и Wayland, пытаясь низвести до маргинального уровня всех, кто не принял генеральную линию их партии, обрекают сами себя на положение маргиналов в глобальном масштабе. И если затея их удалась бы~--- то в число маргиналов попал бы весь десктопный Linux, причём уже навсегда. Хвала Ахурамазде, пока ещё число Linux-маргиналов всякого рода превосходит количество <<твёрдых искровцев>>, и у начинающих пользователей выбор остаётся. Ну а старики его и так сделают.

\section{Всё для блага человека} 
\begin{timeline}Сентябрь 8, 2013\end{timeline}

Время от времени на форумах, в Джуйке и во всяких социальных, с позволения сказать, сетях возникают дискуссии о величии прогресса в FOSS-мире. В них обычно чётко обозначаются две антагонистические стороны:


\begin{enumerate}
	\item прогрессисты~--- как правило, разработчики или сочувствующие, ратующие\dots нет, не за вертикальный прогресс, а за прогресс просто, прогресс всего и вся, и во всех направлениях; 
	\item обычные применители, которые и сейчас неплохо живут, и хотели бы в будущем жить не хуже, а потому, согласно Писареву, немножко консерваторы; однако при этом они не против в  будущем жить и получше~--- а потому немножко прогрессисты, иначе в этих дискуссиях бы не участвовали: их можно назвать консервистами. 
\end{enumerate}


Так вот, основной аргумент корсервистов против  прогрессистов-беспредельщиков таков:
\begin{shadequote}{}
Если старое~--- это в большинстве случаев доброе, хотя бы потому, что проверено временем, то новое может оказаться как добрым, так и не очень. И потому обычному применителю нет никакого смысла в этом новом разбираться, пока не будет доказано, что оно доброе на самом деле.
\end{shadequote}
Очевидно, что при этом  бремя доказательства лежит, в соответствии с юридическими нормами всех стран, претендующих на цивилизованность, на утверждающей стороне, то есть на прогрессистах.

В своих возражениях прогрессисты, как обычно, валят всё с больной головы на здоровую и выдвигают два противоречащих друг другу аргументах.  Первый в простых словах сводится к тому, что
\begin{shadequote}{}
Ты сначала разберись в \texttt{systemd} (GNOME~3, Wayland ~--- нужное дописать), а потом говори, что это плохо.
\end{shadequote}
Нимало при этом не смущаясь тем обстоятельством, что консервист потому и не хочет разбираться в озвученном \texttt{systemd} \textit{etc}., что ни один прогрессист ещё не доказал ему, что это лучше старого, работающего и уже хотя бы потому доброго.

А уж когда консервист напоминает прогрессисту про чайник Рассела, последний вообще встаёт на дыбы и, не забыв про аргумент промежуточный (а ты мне докажи, что \texttt{systemd}~--- это плохо), пускает в ход свой главный  аргумент:
\begin{shadequote}{}
А вообще, все эти \texttt{systemd}'ы с ихними Wayland'ами придуманы не для вас, козлов пользователей, а для нас, разработчиков, чтобы нам было хорошо. А если нам будет хорошо, то и вам, козлам пользователям будет сухо и комфортно.
\end{shadequote}
На логике этого ответа останавливаться не буду~--- она такова, что  от неё  не только переворачивается в урне прах Аристотеля, но и основатели троичной логики всколыхнутся в своих вечных пристанищах (уж не знаю, каковы они у них~--- курганы ямной культуры или воды Ганга).

Не буду вспоминать и том, что суше и комфортней применителям было в те времена, когда разработчики сочиняли свои программы без всяких IDE и прочих веяний прогресса, а напрямую в кодах. Это давно сказано известным лектором про литераторов:
\begin{shadequote}{}
Раньше писатели писали гусиным пером вечные произведения, а ныне~--- вечным пером произведения гусиные.
\end{shadequote}
А остановлюсь только на тезисе: <<Если нам хорошо~--- то хорошо будет и вам>>.

Он напомнил мне одну историю из жизни. В начале августа 1998 года жил я в Ближнем Замкадье, в хорошей такой деревеньке, называлась посёлок Новомосковский, что у города Щербинка, и ныне в его административном подчинении. Рассказывать о нём можно долго, но сейчас разговор не о том.

А вот время действия важно, потому что как раз незадолго до того прошёл ураган, памятный многим москвичам, жителям Подмосковья и гостям тех и других. Прошёл, кстати, в вечер памяти Визбор Иосича, ознаменованный концертом по телевизору. Ради которого отложил я работу, подлежащую сдаче наутро: типа~--- фиг с ней, после концерта докую. В середине концерта ураган и случился, оборвало все провода и посёлок на пару недель (как потом выяснилось) остался без света. Так что наутро мне пришлось брать в рузлак системник и тащить его в город-герой Москву, на службу, к розетке, дабы работу доделать.

Но я опять отвлёкся. К моему сюжету ураган имеет только то отношение, что после него из района приехала комиссия из района, на мурсидесах, в направлении нашего посёлка смогла проехать по, с позволения сказать, дороге, метров пятьсот, и официально признала её находящейся в аварийном состоянии.

Справедливости ради надо сказать, что ураган был тут ни при чём~--- разве что как повод для комиссии. Потому что дорога находилась в аварийном состоянии уже тогда, когда я там поселился, и за прошедшие пять лет её аварийное состояние стало только лучше. То есть аварийней. Вплоть до того, что официальный рейсовый автобус к нам уже года три как не ходил, и старенькие бабуси (а таких на посёлке тогда было немало) ковыляли домой от станции на своих двоих, а то и на троих.

Разумеется, комиссия тут же постановила это безобразие прекратить, и начались ремонтные работы. Их хватило на то, чтобы привести в порядок первую сотню метров. Вот в этот момент и происходит  событие, о котором я веду речь.

Случилось так, что аккурат в это время сгорел у меня монитор (ураган тут опять же был ни при чём, просто годы его вышли) и, после некоторых приключений, купил я новый, трубочный, семнадцатиинчёвый, имени товарища Rolsen'а. И надо было мне его домой доставить.

Как обычно, в нужный момент никого из конных товарищей в пределах досягаемости не было, потому сблатовал я тачку на улице. Честно обрисовал водителю ситуёвину с дорогой ~--- на что он мне со свойственной старым московским бомбилам (они безошибочно по виду и повадке узнавались) лихостью сказал: не дрейф, пацан, прорвёмся. Ну загрузили монитор и поехали.

Долго ли, коротко ли, а под базар за жизнь доехали мы до железнодорожного переезда в городе Щербинка. Откуда вели две дороги: прямо~--- прекрасная (по тогдашним масштабам) шоссейка на Астафьевский аэродром и музей Астафьево, и налево~--- к нам прямо по кривой, на посёлок. Первые сто метров которой, как я уже сказал, блистали свежим асфальтом, как у кота\dots уши.

Лихо свернули мы туда\dots и через сто метров водила мне с тоской в глазах сказал: а давай лучше я тебя по обычному московскому тарифу отвезу в Астафьево\dots

И теперь, когда я слышу разговоры о комфорте разработчиков на благо пользователей, всегда вспоминаю тот эпизод. Давайте вообще будем ехать не туда, куда нам надо, а туда, куда легче. Искать кошельки не там, где потеряли, а там, где светлее. И работать не с тем софтом, который для нас лучше по делу, а с тем, который проще разрабатывать.

Но ведь это именно это нам и предлагают прогресситы~--- создатели \texttt{systemd}'ов и прочих Wayland'ов\dots


\textit{P.S. А история кончилась тем, что довёз меня тот мужик до моей калитки и от предложенной компенсации за качество дороги отказался. Со словами: ты мне всё честно рассказал, а я сам дурак, что не поверил. Старые московские бомбилы~--- это были люди. Как и старые программеры~--- и советские, и анти-советские.}

\section{Размышлизм о графических интерфайсах} 
\begin{timeline}Сентябрь 2, 2013\end{timeline}

Нынешние тенденции развития графических сред блестяще опровергают марксовый тезис о спросе, рождающем предложение. На самом деле всё обстоит с точностью до наоборот: сначала рождается предложение некоторой фичи, после чего нас пытаются убедить, что эта фича нам жизненно необходима.

На самом деле спрос на графические интерфейсы исчерпываются тремя позициями. Первая, наиболее распространённая,~--- ожидание Большой Красной Кнопки с надписью 
\textbf{Сделайте мне пе\dotsто}.

Вторая позиция~--- утверждение: 
\textbf{а мне и так пе\dotsто}.

Третья же, наиболее оригинальная, гласит: 
\textbf{отстаньте от меня со своими предложениями, а пе\dotsто я сделаю себе и сам}.

Однако признать это публично~--- означало бы крах всей индустрии по созданию новых графических интерфейсов и совершенствованию старых. Ибо вожделения искателей Большой Красной Кнопки удовлетворить невозможно по определению~--- разве что в штучном исполнении, для лучших из лучших, за отдельную неприличную мзду. Прочие же вполне удовлетворяются интерфейсами существующими.

Вот и приходится разработчикам убеждать их в том, что, например, GNOMEShell~--- это не только ново и прогрессивно, но также удобно, не только эффектно, но и эффективно. А кто того не понимает~--- старый козёл, ретроград, обскурант и враг прогресса.

\section{Время решений, которое никогда не наступит. Часть 1: зачем нужен десктопный Linux} 
\begin{timeline}Июнь 11, 2012\end{timeline}

Это было\dots нет, не в степях Херсонщины, а в джунглях FOSS-мира. На протяжении многих лет изо всех его закоулков слышались радостные вопли о Linux'е с человеческим лицом, перемещаемые стонами о неготовности его к десктопу.

Зачем надо готовить Linux к десктопу? Процитирую одну из своих заметок исторического цикла (который вскоре составят отдельную онлайновую книжку):
\begin{shadequote}{}
Система, пришедшая в дома на пользовательские десктопы, неизбежно рано или поздно окажется в промышленном секторе.
\end{shadequote}
Это хорошо иллюстрируется случаем с Windows: её 95-я инкарнация, появившись на пользовательских десктопах сначала как платформа для запуска игрушек, быстро проникла на десктопы и офисных работников. А с выходом сестры во интерфейсе~--- Windows~NT~4.0~--- оказалась и достаточно массовой средой для серверов рабочих станций. И с тех пор только укрепляла свои позиции по всем фронтам.

Напротив, enterprise-систему, на пользовательские десктопы не попавшую, столь же неизбежно ждёт увядание и в её основной, промышленной, нише. И тому примером является судьба таких проприетарных UNIX'ов, как True64, HP AUX и Sun Solaris. На протяжении 90-х годов все они пытались вторгнуться в десктопные сферы (причём вместе со своими же аппаратными платформами, по мощности превосходившими настольные писюки многократно)~--- и не смогли это сделать. В том числе и потому, что не уделяли тому должного внимания. А нынче остаётся посмотреть, что с ними делается сейчас в сфере промышленной: первая мертва, две остальные существуют по инерции, на старых корпоративных контактах.

Приводить в пример IBM с её AIX, не обнаруживающей, вроде, признаков угасания, не надо~--- это то самое исключение, которое подтверждает правило. Ибо случай этот исключительный во всех отношениях: IBM~--- это близкий аналог нашего советского ВПК со всеми характерными особенностями, и десктопная нажива для них~--- что немецкий doppel горящей душе русского человека.

Кстати, как показывает история, системам, не преуспевшим на десктопах, мало чего светит и в противоположном секторе пользовательского круга, на всякого рода гаджетах. И тут достаточно вспомнить блистательный взлёт PalmOS и Symbian, не имевших никаких десктопных традиций~--- и их медленное, но неуклонное отступление под натиском Windows~Mobile/CE, перешедшее затем в паническое бегство.

И не говорите мне, что гаджетные версии Windows~--- это совсем не то же самое что Windows для десктопов: мне это известно ничуть не хуже, чем тот факт, что Windows~3.X/9X/ME имеют <<внутре>> мало общего с Windows~NT/2000/XP. Магия имени действует даже на тех, кто об этих отличиях знает~--- как показывает приведённое здесь воспоминание о возникновении web-хостинга на Windows-машинах. Что же говорить о пользователях гаджетов, изрядная часть которых полагает смартфон просто большим мобильником.

В меньшем масштабе история повторилось и при появлении нетбуков: модели с предустановленным Linux'ом быстро исчезли из прайс-листов. И не вследствие козней <<Империи зла>>, а из-за отсутствия спроса. Отдельные же сохранившиеся реликты~--- не более чем платформы для установки пиратской Windows.

Подозреваю, что в ближайшее время следует ожидать и ещё одного витка истории. С появлением Windows~8, в том числе и в мобильной модификации, быстро сдадут свои позиции гаджеты на Android'е. Ибо тут длинные руки Microsoft'а дотянулись до святая святых без-win'ного мира: до ARM-процессоров. И тут магия имени сработает в очередной раз: потребители гаджетной продукции (те, которых пользователями в привычном понимании назвать достаточно затруднительно~--- а их абсолютное большинство) снова предпочтут устройства с системой, хотя бы именем похожей на ту, что стоит на их настольных персоналках. А поскольку Android за всё время своего доминирования на гаджетах так и не порадовал, в отличие от былых PalmOS и Symbian, своими технологическими достоинствами, к ним присоединятся и многие из тех, кого действительно можно назвать пользователями.

Так что десктопизация Linux'а~--- непременное условие сохранения позиций этой ОС в обоих секторах, как в вышележащем, промышленном, так и в нижнем, <<гаджетном>>. И не только её, но и открытого и свободного софта вообще. Ибо из всего изобилия операционок FOSS-мира только она имеет хоть сколько-нибудь заметные позиции в настольном секторе.

\section{Время решений, которое никогда не наступит. Часть 2: время <<Ч>>} 
\begin{timeline}Июнь 14, 2012\end{timeline}

В прошлой заметке я попытался, умышленно утрируя ситуацию, обосновать, почему десктопизация Linux'а необходима. Разумеется, я не полагаю себя самым умным~--- эту необходимость понимали очень многие. Что вызывало, как уже было сказано, с одной стороны, призывы десктопизации Linux'а, с другой~--- победные реляции об успехах в космической области десктопизации, а с третьей~--- горестные стоны:
\begin{shadequote}{}
Linux к десктопу не готов!
\end{shadequote}
Однако не все из понимающих ограничивались словами~--- некоторые претворяли их в дела. И в итоге этих дел многие за призывами, реляциями и стонами не заметили, что тихо и незаметно, как любят писать на одном из известных ресурсов, но наступило время, когда Linux оказался к десктопу готов.

Это случилось, как можно видеть сейчас, спустя достаточное долгое время, примерно в 2005 году. Это не значит, что в одно прекрасное утро пользователи проснулись и увидели, что полки магазинов ломятся от коробок с дистрибутивами, из которых каждому, посредством Большой Красной Кнопки, было обещано сделать п\dots по его хотению. Почему, видимо, никто этой даты и не заметил.

Но, начиная с этого самого условного утра, один за другим начали появляться дистрибутивы (и, главное, их варианты, задумчиво именуемые ремиксами и респинами), которые


\begin{itemize}
	\item во-первых, легко и быстро устанавливались,
	\item во-вторых, при должном подборе варианта, в свежеустановленном виде были пригодны к немедленному использованию рядом категорий пользователей, и 
	\item в-третьих, такой свежеустановленный дистрибутив легко доводился до состояния, адекватного задачам пользователя.
\end{itemize}

 

Такие дистрибутивы, новые, в массовом количестве появлявшиеся на протяжении 2005-2010 годов, или старые, в тот же период времени стремительно эволюционировавшие в этом направлении, и знаменовали пресловутую готовность Linux'а к декстопу. Но именно~--- готовность: ни один из них нельзя было назвать готовым десктопным решением, каждый требовал некоторого допиливания и подтачивания. Пусть не мотопилой и топором, как в прежние времена, а более тонкими инструментами, вплоть до алмазного надфиля, но~--- требовал.

Так что наступило время заняться готовыми решениями: создавать специализированные пользовательские системы с соответствующей косметической отделкой. Однако именно этого сделано и не было. Ибо русло майнстрима переместилось в другую сторону: начали создаваться принципиально <<новые>> (хотя реально~--- просто переделанные старые, некогда забракованные или заброшенные) пользовательские интерфейсы, затем комплексы управления системными службами, а потом на горизонте замаячила и новая графическая система в целом.

Говоря аллегорически, можно описать происходившие процессы так: у пользователей Linux'а появился добротный дом. Он не был воздвигнут единым порывом творческого гения по одному законченному плану, а разрастался, достраивался и перестраивался постепенно. И потому в архитектуре его были пережитки и недостатки. Да, местами неискоренимые без полной перепланировки, но проживанию отнюдь не препятствующие.

В этой ситуации есть три варианта принятия решения~--- если исключить одномоментный снос старого дома и строительство на его месте дома нового:

\begin{itemize}
	\item заняться косметическими улучшениями дома, призванными минимизировать неудобства от его архитектурных недостатков; 
	\item начать строить рядом с существующим домом новый~--- с учётом ошибок в планировке первого, и имея его в качестве надёжного тыла; 
	\item проводить капитальный ремонт частично заселённого дома, ремонт необратимый и делающий невозможным возврат к пренему состоянию. 
\end{itemize}

Именно последний путь регулярно оказывается магистральным последние 10 лет развития Linux'а.

Нельзя сказать, что совсем никто не занимался и не занимается <<косметикой>>~--- то есть готовыми решениями на базе существующего Linux'а. Изрядную часть этой ноши взяла на себя фирма Canonical, развивая вокруг Ubuntu пользовательскую инфраструктуру. В частности, потому на неё и катят бочки с обеих сторон: и консерваторы (которые консервы любят), и ультра-радикалы (которые извлекают квадратный корень из единицы).

Однако желающих заняться этим делом оказалось не очень много. Для энтузиастов-разработчиков это скучно и не сулит славы. Для создателей коммерческих систем~--- не интересно, потому что, во-первых, не обещает немедленной финансовой выгоды, что было прекрасно продемонстрировано судьбой первых коммерческих дистрибутивов, типа Corel Linux. А во-вторых, и, возможно, главных\dots к этому <<во-вторых>> я вернусь через пару абзацев.

Не остался нехоженным и второй путь, отмеченный такими вехами, как AtheOS, практически умершая в своём потомке~--- Syllable, как DragonFlyBSD, тихо и незаметно развиваемая в тени старших сестёр своего семейства, как <<академическая>> MINIX 3. Наконец, как Barrelfish, из которой пока неизвестно что получится (и получится ли вообще).

Однако, чтобы пойти по второму пути, надо обладать либо безбашенностью Курта Скавена, либо, напротив, мудростью и равнодушием к внешней атрибутике Эндрю Таненбаума и Мэтта Дилона. Либо, наконец, просто иметь толику лишних денег, как Microsoft. Тех денег, которые можно бросить на пустую рыбозасолочную бочку в расчёте на то, что рано или поздно в ней заведётся осетрина.

Кроме того, при выборе второго пути нужно помнить, что, каким бы прогрессивным и новаторским он не был, и он не сулит мгновенного успеха. Потому что создания на базе новых систем всё тех же законченных пользовательских решений никто не отменял. А на них, как показывает пример DragonFlyBSD или MINIX 3, у разработчиков просто банально не хватает сил.

Так что, как я уже говорил, магистральным оказался третий путь~--- коренной переделки живого и работоспособного организма. Правда, пока он выражается в выбрасывании старых, добрых, но мещанских фарфоровых слоников и уютных, но вышедших из моды тюлевых занавесок. И заменой из обоями из картин модернистов, кубистов и прочих абстрационистов. Которые, как уже становится ясно, сделают этот дом непригодным для жизни ряда его давних обитателей.

Самое же главное вот в чём: пока одни обитатели дома мирно занимаются своими делами, другие долбят стены перфоратором и сносят несущие конструкции, заменяя их временными подпорками. О комфорте обитателей, то есть пользовательских решениях, все в очередной раз забыли. <<Строителям>> не до них~--- они обещают комфорт по завершении своей титанической работы. Мирные обыватели в них не видят смысла, не очень понимая, что останется от их дома по завершении капитального ремонта.

А сторонние зрители, то есть собственно разработчики решений, уже успели убедиться в том, что пути <<строителей светлого будущего>> неисповедимы: сегодня на стенах они заменяют репродукции Васнецова картинами Пабло Пикассо, а завтра, глядишь, извлекут на свет квадрат Малевича. И ещё в запасе остаётся <<Третья мировая война>> Анастасии Стрелецкой и, наконец, картина, символизирующая голод~--- задница, затянутая паутиной.

Что уже наблюдалось на примерах с devfs, HAL'м и рядом других. И никто не может исключить, что всё в очередной раз не закончится анекдотом про чукчу-хирурга, полосующего внутренности оперируемого большим хирургическим скальпелем с криками <<Опять ничего не получилось>>. Легко ли в такой ситуации найти желающих заниматься развитием пользовательской инфраструктуры? Вот их и не густо в нашем мире.

С год назад Андрей Боровский написал статью под умышленно провокационным заглавием~--- <<Почему Линукса нет и не будет на десктопах>>. Главный её тезис можно сформулировать примерно так:

\begin{shadequote}{}
Linux проиграл битву за десктопы.
\end{shadequote}
Из чего делается вывод, что заниматься декстопным Linux'ом бессмысленно, следует развить те сферы, где позиции его ныне сильны как никогда~--- а) сферу потребительских гаджетов и б) сферу встраиваемых устройств промышленного назначения.

Ни в коем случае не оспариваю фактографическую сторону этой статьи~--- более того, готов подписаться под рядом её утверждений. Например, под тем, что современный Linux имеет всё необходимое для десктопного успеха: драйверное обеспечение, пользовательский интерфейс и информационную поддержку. И по совокупности этих компонентов не уступает ни одной коммерческой системе. Собственно, именно это я и назвал в своей \href{http://alv.me/?p=1532}{предыдущей заметке} потенциальной <<готовностью к десктопу>>. Беда только в том, что эта потенция не реализуется: почти никто не предлагает перечисленные компоненты в совокупности, в виде законченного решения.

Однако с основным тезисом и его следствием я категорически не согласен. И оспаривать их начну как раз со следствия. Конечно, <<промышленный эмбеддинг>> и <<потребительский гаджетизм>>~--- штуки очень важные. Но, как я пытался обосновать в прошлой заметке, без прочного десктопного тыла ориентированная на них система рано или поздно захиреет. Скажем, бодрых рапортов о развитии NetBSD, некогда царя горы встроенных устройств, давненько мне не попадалось\dots

Что же до основного тезиса\dots Нет, Linux не проиграл битву за десктопы. Он от этой битвы уклонился именно в тот момент, когда в эту драку следовало ввязаться. И представится ли ему второй такой шанс~--- ведомо только Ахурамазде.

Впрочем, иногда в драку приходится ввязываться и без надежды на победу. Потому что альтернатива~--- это вечное ожидание того дня, когда всё будет перестроено и достроено до посинения. Дня, который при таком раскладе не наступит никогда.

\section{Камо грядеши?} 
\begin{timeline}Апрель 5, 2013\end{timeline}

Более года минуло с той поры, как я перестал быть пользователем Fedora и даже RFRemix~--- некоторое время назад  именно он обоснованно претендовал на звание умнейшего из медведей лучшего из дистров. Однако за развитием обоих продолжаю следить~--- поскольку в их рамках часто предлагаются решения, применимые и в более иных дистрах. Что вполне логично. Ибо кто более всех подготовлен к борьбе с трудностями, как не тот, кто эти трудности для себя (и для нас для всех) придумал? Смотреть соответствующий пассаж Ильфа и Петрова.

Однако нынче я хочу поговорить немного о другом. Просматривая ленту новостей проекта Russian Fedora, натолкнулся я на сообщение Петра Леменкова о повышении степени интеграции \texttt{systemd} и KDE. На содержании самой заметки останавливаться не буду. А только процитирую вопрос, которым она завершается:
\begin{shadequote}{}
А что вы и ваш дистрибутив будет делать, если когда и GNOME, и KDE перейдут на \texttt{systemd}?
\end{shadequote}
Здесь можно, с одной стороны, в очередной раз удивиться наивности автора (да простит меня Пётр, но иного слова не подберу). Ибо ответ на него очевиден, и давался неоднократно за последние лет 10. Произойдёт следующее:

\begin{itemize}
	\item <<лучшие из лучших>> пользователей Linux (то есть самые богатые) обзаведутся Mac'ами; собственно, уже обзаводятся~--- и первыми с корабля побежали те крысы, которые более всего приложили руку к организации течи; 
	\item <<лучшие из худших>> (то есть достаточно обеспеченные и законопослушные граждане, которых, надеюсь, среди нас большинство) приобретут лицензионную Windows;
	\item а вот <<худшие из худших>>~--- те, что, подобно попам, студентам и офицерам, вистуют на девятерной (соответственно, из жадности, бедности или по пьяни), как юзали Выньду Пераццкую, так и будут её юзать.
\end{itemize}

И это не прогноз онолитегов, и не прорицание Нострадамуса. Это описание сценария, происходившего неоднократно, начиная с 1999 года. Когда Linux в лице Mandrake (а на Руси~--- в её ипостаси Russian Edition) впервые, в реалиях той исторической ситуации, стал готов к выходу на десктоп.

Второй раз сценарий повторился в 2005 году, ознаменованном достижением должного уровня развития Системами Быстрого Развёртывания, начиная с Чернышевского Zenwalk'а и заканчивая героями Народной Воли Ubuntu и его клонами.

Ну а третий раз мы наблюдали этот сценарий сразу после выхода RFREmix 14, когда казалось, что пролетариат, единственный до конца революционный класс майнтайнеры Fedora готовы призвать к открытой, революционной борьбе миллионы крестьян широкие массы применителей IT.

Возникает глубокое подозрение, что большинство разработчиков Linux и майнтайнеров его дистрибутивов панически боятся того, что Linux станет по настоящему десктопным. То есть придёт на машины применителей. Ведь тогда им пришлось бы отвечать за тех, кого приручили. А не с азартом чукчи хирурга восклицать: Опять ничего не получилось! И начинать перекраивать всё заново по-живому. Обрекая тем самым Linux на участь испорченной и урезанной копии потребительских систем.

\section{Куда придеши?} 
\begin{timeline}Апрель 24, 2013\end{timeline}

Прошу прощения за смесь церковнославянского с нижегородским в заголовке, но он хорошо отражает суть дела. Потому как эта заметка представляет собой интегрированный ответ на комментарии к предыдущей~--- дабы не повторяться, ибо большая их часть однотипна. Но для начала попробую ответит на кардинальный вопрос:
\begin{shadequote}{}
чем, собственно, \texttt{systemd} так плох\dots
\end{shadequote}
Сам по себе \texttt{systemd} не плох, и не хорош. Просто это замена существующих схем инициализации, которые работают. Резко от них отличная, существенно более сложная и пока не продемонстрировавшая явных преимуществ с точки зрения пользователя. Но время от времени создающая неожиданные проблемы, давным-давно решённые в классических схемах и их <<мягких>> модернизациях. Ну и про агрессивную её пропаганду забывать не следует.

Большая часть остальных комментариев сводится к следующему положению:
\begin{shadequote}{}
Я работаю в Slackware (Debian, Crux\dots нужное дописать), и эти ваши \texttt{systemd}'ы меня не колышат.
\end{shadequote}
Увы~--- не колышат сейчас, так заколышат со временем. Если основные рабочие среды не будут работать вне \texttt{systemd}, то всем дистрибутивам Linux воленс-неволенс придётся её внедрять~--- иначе они потеряют большую часть своих пользователей.

Вариация на ту же тему:

\begin{shadequote}{}
Я использую WindowMaker (Openbox, IceWM\dots как и в предыдущем случае, нужное дописать), и мне эти ваши гномокеды до лампочки.
\end{shadequote}
Да, но подавляющее большинство пользователей таки используют KDE/GNOME, и майнтайнеры большинства дистрибутивов будут ориентироваться на них. Это во-первых. А во-вторых, следующим шагом той самой железной поступи будет интеграция \texttt{systemd} и Xorg, а для Wayland, скорее всего, эта интеграция будет изначальной. Хотя, конечно, остаётся ещё голая консоль\dots Но и её скоро можно будет русифицировать только средствами \texttt{systemd}.

Следующее положение:
\begin{shadequote}{}
\dotsлюди, которые будут пользовать старые версии дистров, пока это им позволит железо и собственные навыки в обновлении софта.
\end{shadequote}
Да, их есть. И среди действующих \textit{применителей} Linux (см.~\hyperlink{customers}{<<Применители vs потребители>>}) их много. Возможно, даже большинство. Они не нуждаются в регулярном обновлении прикладного софта, потому что нынче не то что давеча~--- принципиально новые (и ими востребованные) фичи появляются крайне редко. Они не нуждаются в обновлении системы~--- потому что установленная обеспечивает функционирование нужных им приложений. Они не нуждаются даже в обновлениях безопасности~--- потому что прекрасно знают цену страшилкам о злобных хацкерах, караулящих доверчивых линуксоидов в подворотнях Интернета.

Не нужен им и перманентный апгрейд <<железа>>~--- потому что они знают: любое <<железо>>, ныне существующее за пределами антикварной лавки или музея электронного мастерства, перекрывает большинство реальных применительских задач сполна. И даже с лихвой. А потому в их руках система проходит естественный жизненный цикл UNIX-машины~--- от покупки и установки ОСи и приложений до полной физической амортизации. Но\dots

\dotsно рано или поздно эта физическая амортизация наступает. И тогда оказывается, что заменить вышедшую из строя железяку или невозможно вообще, или~--- за сумму, в галактических кредо превосходящую цену новой машины. А, вопреки присказке, старый конь-дистр пашет глубоко, но на новую борозду-платформу не встаёт\dots

По поводу того, что
\begin{shadequote}{}
\dotsвы забыли про убунту
\end{shadequote}
Нет. Я о ней никогда не забываю. Но инкорпорация \texttt{systemd} в Xorg/KDE/GNOME/\textit{etc}. отсечёт от неё как минимум Kubuntu и Xubuntu. И останутся они с сиротливым Unity. Да и то, когда и если (если и когда) свой Mir доделают. Насколько я понимаю, там всего и делов~--- начать да кончить.

Так что под занавес:
\begin{shadequote}{}
Расскажите ваши действия.
\end{shadequote}
Мне, подобно многим из ранее высказавшихся, по барабану: я могу работать в любой UNIX-подобной системе. Даже в упомянутой в комментариях по ходу дела DragonFlyBSD, а также FreeBSD и OpenBSD. Возможно, и в NetBSD смог бы, хотя не пробовал. То есть пробовал, но не работал. Но ни одна из BSD-систем (не говоря уже о Solaris и её проблематичных форках) для массового употребления применителями в существующем виде не годится. А я в ответе за тех, кого приручил. И поэтому для них придерживаюсь стратегии, сформулированной \textsl{Viktor~W}.:
\begin{shadequote}{}
Покамест еникеи все на привычных местах расположены и нажимаются очевидным способом\dots
\end{shadequote}

\dotsу них будет openSUSE 12.1. А я составляю шпаргалки по \texttt{systemd} в openSUSE 12.3. А также смотрю новости по FreeBSD и читаю списки рассылки DragonFly\dots

\section{Мастерство работы в Unix} 
\begin{timeline}2006 г\end{timeline}

Это~--- наброски к книге, которая никогда не будет написана до конца. Идея её, как легко догадаться, навеяна замечательной книгой Эрика Реймонда <<The Art of Unix Programming>> (в русском переводе~--- <<Искусство программирования для Unix>>. М.: Издательский дом <<Вильямс>>, 2005, 544~с.). Каковую стоит прочитать каждому~--- даже тем, кто и в мыслях не держал сочинять программы ни для Unix, ни для какой-либо другой операционной системы. В ней идеолог движения Open Source наглядно демонстрирует, как, руководствуясь полутора десятками основополагающих принципов, следует сочинять программы. И, более того, приводит примеры программ, написанных на основе этих принципов.

\subsection{Вступление}
Как известно, программы сочиняются не сами для себя (хотя для разработчиков Open Source сочинение программ~--- в определенной мере самоцель). А в том числе и для того, чтобы их (хоть некоторые) использовали в реальной работе. И тут мы приходим к тому, что искусно (мастерски, артистично~--- к этому я еще вернусь) сочиненная программа предъявляет к своему пользователю определенные требования. Главное из которых~--- умение столь же мастерски (а возможно, и артистично) использовать заложенный в ней создателем потенциал (подобно тому, как кровная и выезженая лошадь требует от своего наездника соответствующего умения верховой езды~--- но и к этой аналогии еще придется обратиться). Можно сказать, что чем больше мастерства (артистизма) вложено в программу разработчиком, тем более высокие требования она предъявляет к тем же качествам пользователя. И вот именно этому я и хотел бы посвятить свои заметки.

Для начала~--- о заглавии. Как назвал свою книгу Эрик~--- я уже говорил. Но следует помнить, что английское \textit{Art} передается русским словом <<искусство>> не вполне адекватно. Также, как и производное от него слово Artist означает вовсе не обязательно актёра на сцене, а скорее художника~--- творца в самом широком смысле этого слова. То есть русским эквивалентом к нему выступил бы Мастер. Точнее, конечно, псевдо-русским~--- но в языках-источниках, опять-таки, этому слову придается совсем другой смысл. Потому я и позволил дать своим заметкам такое заглавие~--- <<Мастерство работы в Unix>>.

Отказался я и от напрашивающегося в заглавие слова~--- <<использование>>. Во-первых, потому, что оно вызывает вполне определенные ассоциации (из старого анекдота~--- <<Встретились я и моя компания. Использовали\dots>>). А во-вторых, потому, что, как, говоря словами Эрика, <<проектирование и реализация программного обеспечения должны быть радостным искусством>>, так не менее радостным должно быть и его применение для решения своих задач~--- и слово <<использование>>, подразумевающее потребительское отношение к деятельности творцов, тут невместно\dots

Другое дело~--- работа в Unix. Работа, как еще встарь отметил великий поэт, есть всегда. И даже тогда, когда предметом её оказывается то, что со стороны показалось бы в лучшем случае развлечением, а то и блажью~--- а со временем мы увидим, что деятельность создателей Unix и Linux могла бы предстать именно в таком качестве. Но: <<Настоящая работа всегда груба и яростна>>~--- эти слова героя <<Территории>> Олега Куваева вовсе не перечеркивают то, что она при этом остается <<радостным искусством>>\dots

Ну вот, вводные слова сказаны, пора переходить к предмету обсуждения. И тут следовало бы начать придется с седой <<бронзовой>> древности~--- для того, чтобы понять откуда пошло быть самое понятие Мастерства. Однако об этом речь пойдёт в совсем другой книжке\dots

А пока вопрос: что требуется от конечного пользователя для достижения мастерства работы в Unix? Попытка ответа на этот вопрос и составит предмет дальнейшего изложения. Однако сначала надо рассмотреть вопрос о том, что это такое~--- Unix, и определиться с тем, кто является его пользователями~--- как <<действующими>>, так и потенциальными.

\subsection{Что такое Unix}
Термин Unix и не вполне эквивалентный ему UNIX используется в разных значениях. Начнем со второго из терминов, как более простого. В двух словах, UNIX (именно в такой форме)~--- зарегистрированная торговая марка, первоначально принадлежавшая корпорации AT\&T, сменившая за свою долгую жизнь много хозяев и ныне являющаяся собственностью организации под названием Open Group. Право на использование имени UNIX достигается путем своего рода <<проверки на вшивость>>~--- прохождения тестов соответствия спецификациям некоей эталонной ОС (Single Unix Standard~--- что в данном случае можно перевести как Единственный Стандарт на Unix). Процедура эта не только сложна, но и очень недёшева, и потому ей подверглись лишь несколько оперционок из ныне здравствующих, и все они являются проприетарными, то есть представляют собой собственность неких корпораций.

В числе корпораций, заслуживших право на имя UNIX п\'{о}том разработчиков/тестировщиков и кровью (точнее, долл\'{а}ром) владельцев, можно назвать следующие:

\begin{itemize}
	\item Sun с её SunOS (более известной в миру под именем Solaris); 
	\item IBM, разработавшая систему AIX; 
	\item Hewlett-Packard~--- владелец системы HP-UX; 
	\item IRIX~--- операционка компании SGI. 
\end{itemize}

Кроме этого, собственно имя UNIX применяется к системам:

\begin{itemize}
	\item True64 Unix, разработанная фирмой DEC, с ликвидацией коей перешедшая к Compaq, а ныне, вместе с последней, ставшая собственностью той же Hewlett-Packard; 
	\item UnixWare~--- собственность компании SCO (продукту слияния фирм Caldera и Santa Cruz Operation). 
\end{itemize}

Будучи проприетарными, все эти системы продаются за немалые американские деньги. Однако это~--- не главное препятствие к распространению собственно UNIX'ов. Ибо общей их особенностью является привязка к определенным аппаратным платформам: AIX работает на серверах и рабочих станциях IBM с процессорами Power, HP-UX~--- на собственных машинах HP-PA (Precission Architecture), IRIX~--- на графических станциях от SGI, несущих процессоры MIPS, True64 Unix~--- предназначена для процессоров Alpha (к сожалению, в бозе почивших). Лишь UnixWare ориентирована на <<демократическую>> платформу PC, а Solaris существует в вариантах для двух архитектур~--- собственной, Sparc, и все той же PC. Что, однако, не сильно поспособствовало их распространенности~--- вследствие относительно слабой поддержки новой PC-периферии.

Таким образом, можно видеть, что UNIX~--- это понятие в первую очередь юридическое. А вот за термином Unix закрепилась технологическая трактовка. Так в обиходе IT-индустрии называют все семейство операционных систем, либо происходящих от <<первозданной>> UNIX компании AT\&T, либо воспроизводящих её функции <<с чистого листа>>, в том числе свободные ОС, такие, как Linux, FreeBSD и другие BSD, никакой проверке на соответствие Single Unix Standard никогда не подвергавшиеся. И потому их часто называют Unix-подобными.

Широко распространен также близкий по смыслу термин <<POSIX-совместимые системы>>, которым объединяется семейство ОС, соответствующих одноименному набору стандартов. Сами по себе стандарты POSIX (Portable Operation System Interface based on uniX) разрабатывались на основе практики, принятой в Unix-системах, и потому последние все являются по определению POSIX-совместимыми. Однако это~--- не вполне синонимы: на совместимость со стандартами POSIX, претендуют операционки, связанные с Unix лишь косвенно (QNX, Syllable), или несвязанные вообще (вплоть до Windows~NT/2000/XP).

Чтобы прояснить вопрос взаимоотношений UNIX, Unix и POSIX, пришлось бы опять немного углубиться в историю. А это, как уже говорилось, будет темой совсем другой книжки. Поэтому далее речь пойдет о работе в Unix-системах в самом широком смысле этого слова, без учета всякого рода торговых марок и прочих юридических заморочек. Хотя основные примеры, относящиеся к приемам работы, будут взяты из области свободных их реализаций~--- Linux, в меньшей степени FreeBSD, и еще в меньшей~--- из прочих BSD-систем.

\subsection{Пользователь Unix~--- кто он?}

\hfill \begin{minipage}[h]{0.45\textwidth}
Есть люди, умеющие пить водку, и есть люди, не умеющие пить водку, но все же пьющие её. 

И вот первые получают удовольствие от горя и от радости, а вторые страдают за всех тех, кто пьет водку, не умея пить её. 
\begin{flushright}
\textit{Исаак Бабель}
\end{flushright}
\bigskip\end{minipage}

Чтобы очертить круг пользователей Unix, вернемся к исторически сложившимся сферам её применения, очерченным в предыдущем разделе. А исторически первыми её пользователями были сами разработчики Unix. И не возьмусь судить, что для них было первичным~--- стремление поиграть в любимую игру, или сама по себе разработка, в качестве игры воспринимавшаяся. Так что одна из групп пользователей Unix очевидна~--- разработчики программного обеспечения, как системного, так и прикладного.

Далее~--- коммуникации и Интернет. Естественно, что в круг пользователей Unix попадают администраторы локальных сетей, web- и ftp-серверов, файловых серверов организаций и серверов баз данных. То есть~--- потенциально все, имеющие отношение к этим областям IT-индустрии.

Столь же очевидна роль Unix-пользователей в образовании и исследованиях в области Computer Science. Однако этим его образовательная роль не исчерпывается. Именно на Unix-машинах обучаются так называемой <<компьютерной грамотности>> студенты естественно-научных и инженерных специальностей, вплоть до геологов, геофизиков, океанологов, во многих множествах университетов мира (к сожалению, наша страна тут часто оказывается несчастливым исключением). Возможно, что в дальнейшем этим <<некомпьютерным>> студентам дела с Unix иметь и не придется~--- что ж, изучение его можно рассматривать как элемент базового инженерного или естественно-научного образования, подобно общей физике или матанализу.

А потом~--- кто знает? Человек предполагает, а судьба располагает. Могу сослаться на свой опыт: университетские курсы по термодинамике и физхимии (каюсь, весьма скверно усвоенные в промежутках между полевыми работами, пьянками с аморалками и изучением спецпредметов~--- примерно таков был ряд приоритетов нормального советского скубента-геолога) для меня оказались востребованными уже через год-два работы. А лет 20 спустя я с удовольствием вспоминал об университетском курсе программирования на Мир-2 с её Алголом~--- который также оказался совсем не лишним.

К Unix-пользователям из сферы образования (точнее, самообразования) можно отнести и легионы просто заинтересовавшихся этой системой~--- школьников, студентов произвольных специальностей, специалистов некомпьютерных профессий~--- вплоть до домохозяек (да-да, среди лично знакомых мне линуксоидов есть и домохозяйки). Стимулом для них является исключительно удовлетворение собственного любопытства и тренировка мозгов. Эрик Реймонд пишет, что программировать для Unix~--- занятие не только радостное, но и очень интересное. От лица простых пользователей (тех самых, о которых скоро пойдет речь) могу утверждать, что просто изучать Unix и приемы работы в нем~--- не менее интересно и захватывающе, даже если поначалу не имеет никакого практического применения. Однако его, этого применения, нет лишь до поры до времени: кто знает, сколько из юных посетителей многочисленных форумов выберут своей профессией программирование или администрирование компьютерных сетей. А если и нет, и их дальнейшая работа никак не будет связана с IT-индустрией, освоение Unix даст им совершенно незабываемый жизненный опыт~--- тот, которого они, в большинстве случаев, недополучают на уроках информатики в средней школе или на формальных курсах этой дисциплины в неспециализированных вузах (впрочем, говорят, что и в специализированных вузах дело подчас обстоит не лучше).

Итак, определилось две категории пользователей Unix, как <<действующих>>, так и потенциальных: люди, профессионально связанные с IT-индустрией, будь то разработчики или системные администраторы, и те, кто использует эту систему для образования (в том числе и самообразования). А как же те самые конечные пользователи настольных машин общего назначения?

Тут впору вспомнить лозунг, активно продвигаемый энтузиастами и их сообществами: 
\textit{Linux~--- на каждый десктоп!}
 То есть поговорить и превращении Linux (в данный момент выступающего как ипостась Unix вообще~--- просто как самая массовая ОС этого семейства) в столь же массовую систему, как MS Windows любого рода. И задаться парой вопросов: возможно ли это? и нужно ли?

Чтобы ответить на них, нужно действовать <<от противного>>: то есть достаточно очертить круг тех, кто 
\textbf{не может}
 стать пользователем Linux.

Это, во-первых, пользователи, категорически не способные к освоению компьютера (но, согласно приведенной в эпиграфе максиме Бени Крика, все-таки его использующие). И это~--- отнюдь не признак их глупости, а, как и способность к питию водки, просто индивидуальная особенность: есть же люди, не отличающие ямба от хорея и \textit{до} от \textit{фа}. И здесь здесь то же самое: мне известно немало пользователей с полуторадесятилетним стажем, так и не освоивших запись на дискету или отключение показа непечатаемых символов в Word. Вынужденные работать на компьютере, они, продолжая словами Бени, страдают: и за себя, и за всех тех, кто на компьютере работать не умеет. И Unix только усугубит их страдания.

Далее, из числа пользователей Unix следует исключить тех, кто испытывает идиосинкразию к чтению~--- а таких, увы, становится все больше даже в нашей стране, некогда бывшей самой читающей в мире. Потому что если в Windows (и тем более в MacOS) кое-какие полезные навыки можно получить методом научного тыка, то в Linux без чтения документации и, возможно, даже толстых книг, обойтись практически невозможно (впрочем, к этой теме мы еще вернемся в ближайших разделах).

На Unix никогда не перейдут запойные игроманы и те, кто использует компьютер исключительно в качестве развлекательного центра. И причины понятны: игр под Linux катастрофически мало, и нет в нем ничего, что оправдывало бы смену ОС для домашней аудио- и видеостанции.

Из пользователей-креативщиков вербовать сторонников Linux также в большинстве случаев бессмысленно: работа в нем профессионалов по созданию мультимедийного контента или спецов высокой полиграфии будет попросту неэффективной. Ну нет в нем инструментов для работы профессионального художника\dots

Аналогично~--- с теми, для кого необходимо работать с векторной, пусть даже технического плана, графикой. А также~--- активными пользователями CAD-систем. Разумеется, в мире Unix и Open Source есть кое-какие векторные рисовалки и CAD-системы, но ни те, ни другие для профессионального использования практически не пригодны.

Так кто же остается в сухом остатке, кроме разработчиков софта и профессиональных администраторов компьютерных сетей, а также занятых в первую очередь образованием и самообразованием энтузиастов? И вот тут пора вспомнить о двух первых, в историческом плане, сферах практического применения Unix: обработке текстов и коммуникациях (включая общение в Глобальной Сетью). Разве не к этому сводятся потребности (в первую очередь сугубо профессиональные) многих и многих пользователей компьютеров? Вынужденных удовлетворять их посредством заведомо избыточного Word'а и однозначно убогих Outloock'ов с Internet Explorer'ом.

И в итоге перед нами всего-навсего одна категория пользователей, потенциально ориентированных на работу в Unix: те, для кого по долгу службы (или велению души) важна эффективность работы с текстовым контентом, дополняемая коммуникационными возможностями. И именно креативщики-текстовики (в любой области~--- от технических писателей и научных работников до поэтов и писателей просто) имеют возможность использовать инструменты Unix максимально эффективно. Остается только продемонстрировать им эту эффективность~--- в чем автор и видит одну из основных задач своего сочинения.

Интересно, что среди <<действующих>> пользователей Linux весьма высок процент профессиональных юристов и переводчиков. И тому можно видеть две причины. Во-первых, и те, и другие, безусловно, входят в сословие креативщиков-текстовиков. А во-вторых, и для юристов, и для переводчиков более, чем для остальных представителей этого сословия, важны аспекты легальности используемого ими софта.

\subsection{Три метода решения пользовательских проблем в Unix}

\hfill \begin{minipage}[h]{0.45\textwidth}
Спасение утопающих~--- дело рук самих утопающих!
\begin{flushright}
\textit{Ильф и Петров}
\end{flushright}
\bigskip\end{minipage}

Нет, наверное, прикладной программы~--- будь она под DOS или Windows, Unix или MacOS, свободной или проприетарной, бесплатной или коммерческой,~--- использование которой не создавало бы время от времени каких либо проблем, требующих разрешения. Однако бытует устойчивое мнение, что наибольшее количество проблем возникает при пользовании именно свободных и бесплатных программ под Unix, конкретно~--- под Linux и BSD. Как, впрочем, чревато проблемами и использование самых этих ОСей. По крайней мере, сообщениями о такого рода проблемах полны форумы соответствующей тематики.

Я не буду вдаваться в дискуссию, насколько это мнение обоснованно, а насколько~--- обязано вековым предрассудкам и заблуждениям. Как бы в скобках рискну высказать свое скромное мнение~--- проблем при использовании Linux или FreeBSD ничуть не больше, чем при работе в Windows любого рода~--- а если уж совсем о личном, так и гораздо меньше. Однако то, что проблемы все же возникают, особенно у начинающих пользователей,~--- это медицинский факт, с которым приходится считаться. И, соответственно, требуется наметить методы их решения.

Собственно, проблемы с ОС или прикладной программой ничем не отличаются от любых других~--- например, с автомобилем, телевизором или стиральной машиной. И за все время существования человечества было придумано лишь три метода их разрешения.

Первый~--- это досконально разобраться в проблеме и все сделать самому. Второе~--- заплатить деньги тому, кто эту проблему может решить, будь то соответствующая сервисная служба или частное лицо, сделавшее решение пользовательских проблем своей профессией (в околокомпьютерном мире за ней закрепилось название~--- эникейщик). Третий~--- прибегнуть к помощи друга-понимальщика, который, соответственно, денег за помощь не возьмет.

Достоинства первого метода очевидны. Разобравшись в проблеме и найдя её решение самостоятельно, пользователь приобретает:

\begin{itemize}
	\item вполне конкретные знания по данному вопросу; 
	\item понимание общих методов решения определенного круга проблем; 
	\item и, самое главное, ничем не заменимую уверенность, что нет таких крепостей, которые не смогли бы взять большевики (пардон, POSIX'ивисты), что любая проблема имеет свое решение, и найти его~--- вполне в его, пользователя, силах. 
\end{itemize}


Недостатки, казалось бы, тоже лежат на поверхности: самостоятельное решение любой проблемы требует а) времени, б) чтения книг, сетевых материалов, документации, а подчас даже~--- о ужас~--- и некоторых размышлений. Однако не это главное~--- эти недостатки я отнес бы скорее к особенностям использования Open Source. Тем самым особенностям, которые могут переходит в достоинства. Согласитесь, вовсе не вредно освежить навыки чтения русских текстов, полученные в начальной школе на уроках родной речи (или как там нынче этот предмет называется?). Чтение же иноязычных (почти равно~--- англоязычных) источников, кроме восстановления навыка разбирать буковки, обеспечивает еще и дополнительную языковую практику, от избытка которой, вроде, еще никто и никогда не страдал. Что же до затраченного времени~--- это сторицей окупится тремя вышеперечисленными положительными факторами.

Главными недостатками первого метода решения проблем (назовем его, вслед за Олегом Куваевым, методом большого болота) мне видятся два. Первый имеет силу в основном для начинающего пользователя. Который просто подчас не знает, с какого конца к своей проблеме подступиться: то ли начинать читать man-страницы, то ли~--- беллетризованные новеллы, подобные тем, что сочиняет автор сих строк, то ли~--- страшно подумать~--- хвататься за <<толстые>> книги, описывающие (или делающие вид, что описывающие) <<основы основ>>.

Безусловно, официальная документация проектов Open Source~--- могучий инструмент познания. Как сказал собеседник бы Бени Крика, оставшийся неизвестным: <<Вы знаете тетю Маню?>>~--- Я знаю тетю Маню>>,~--- ответил ему Беня.~--- <<Вы верите тете Мане?>>~--- <<Я верю тете Мане>>\dots

Так вот, если знать верить тете Мане, она даст ответ почти на любой вопрос. Одна беда~--- спрашивать её нужно правильно, то есть не только верить, но и кое-чего знать~--- в том числе и об этой особе как таковой. Да и иметь общие представления о предмете вопроса~--- также не лишне: man- и info-документация сочиняется в первую очередь как справочник для тех, кто, зная в общих чертах суть дела, не может (или не хочет) обременять свою память подробностями.

А вот за общими представлениями пользователю неизбежно придется обращаться либо к около-линуксовой (условно говоря) беллетристике, либо к <<толстым>> руководствам. К чему именно~--- вопрос, однозначного ответа не имеющий, ибо последний зависит от множества факторов. Первый источник, при доле везения, может поспособствовать решению конкретной проблемы~--- хотя бы на <<рецептурном>> уровне. Тем не менее, и без второго обойтись вряд ли удастся\dots

Однако предположим, что наш начинающий пользователь тем или иным образом прорвался сквозь тернии начальных проблем, неизбежных при освоении <<чужой>> системы. Превратившись, тем самым, в пользователя <<действующего>>. Однако почивать на лаврах ему не придется: ведь проблемы возникают и у <<действующих>> пользователей, в том числе и достаточно опытных. И тут перед ними в полный рост встает второй недостаток самостоятельного их решения. Который легко сформулировать, процитировав бессмертный афоризм Козьмы Пруткова: <<Нельзя объять необъятное>>. Что применительно к нашим условиям можно трактовать как невозможность одинаково глубокого изучения всех аспектов устройства и использования свободных ОС и (особенно) их приложений.

Кроме принципиальной возможности, существенен еще и фактор целесообразности: далеко не всегда есть желание возиться с настройками, скажем, некоей мультимедийной программы, разбираясь попутно в кодеках и движках, если это а) не являет собой предмет профессиональной деятельности и б) по определению будет разовой операцией. И тут время вспомнить о двух оставшихся методах решения проблем.

О втором методе~--- платной поддержке, фирменной или индивидуальной~--- я, к сожалению (или~--- к счастью?) сказать ничего не могу, кроме самых общих соображений. К услугам фирменной техподдержки, обещаемой многими майнтайнерами дистрибутивов, в том числе отечественных, ни разу не довелось обращаться. Что же до индивидуального эникейства~--- похоже, в области софта Open Source это занятие популярности не приобрело. В отличие от Windows-сферы, где установка и настройка системы и программ для нее обеспечивало хлеб насущный (или, по крайней мере, пиво насущное) не одному поколению пользователей с некоторым опытом\dots

Остается третий метод~--- обратиться к другу-понимальщику. Метод идеальный~--- ведь <<друг всегда уступить готов место в лодке и круг>>. Правда, для этого надо, чтобы друг такой был~--- тот самый, с которым пройдены тысячи километров по горам и долам, с кем съедены пуды соли и выпиты цистерны водки. И который сделает для вас все. Правда, и самому нужно быть готовым сделать для него все. Но это~--- лишь один момент. Второй же~--- чтобы друг этот был еще и понимальщиком в той проблеме, которая возникла~--- а вот с этим уже сложнее: ведь возможных проблем немало, а друзей, по определению, много не бывает\dots

Благо, современная действительность в виде Интернета предлагает некий эквивалент дружеской помощи~--- в виде специализированных сайтов и форумов, поддерживаемых энтузиастами и их неформальными объединениями. На сайтах можно поискать ту самую около-линуксовую беллетристику, о которой я, как об источнике сведений, упоминал выше. А на форумах~--- задать вопрос по своей проблеме и, возможно, даже получить ответ. И, вполне вероятно, что это окажется самым простым и быстрым способом разрешения проблемы. Впрочем, только при выполнении некоторых условий.

Сначала~--- о сайтах. Их~--- много, перечислять все было бы затруднительно, а уж охарактеризовать их содержание~--- вообще непосильно. Подчеркну только их общие особенности, о которых следует помнить начинающему пользователю.

Содержание основных сайтов Рунета, посвященных тематике Unix, Linux и Open Source, весьма разнообразно. Там можно найти и переводы официальной документации проектов, и переводные же статьи с зарубежных (точнее, иноязычных) онлайновых источников, и более или менее подробные оригинальные статьи, и краткие советы по частным вопросам. Объединяет их одно: все эти ресурсы создаются и поддерживаются на голом энтузиазме. И потому содержат лишь то, что интересно (или в данный момент нужно) авторам материалов и переводчикам. Так что ожидать, что на сайтах найдутся решения на все случаи жизни~--- было бы несколько опрометчиво.

Конечно, можно написать автору сайта или отдельных его материалов письмо с просьбой осветить интересующую пользователя тему. Однако рассчитывать, что просьба будет удовлетворена~--- также не стоит. Ибо обычно все, что автор сайта знает~--- он уже и так написал. В ненаписанном же он либо не считает себя компетентным, либо этот вопрос ему не интересен. Хотя, конечно, бывают и исключения\dots

Так что вопросы, не освещенные в сетевых материалах, лучше задавать в форумах. Однако и тут хорошо бы придерживаться определенных правил. Правила эти многократно освещены в сетевых источниках, например, в классическом сочинении Эрика Реймонда Как следует задавать вопросы, так что заострю внимание только на некоторых моментах, которые кажутся мне наиболее важными.

Во-первых, следует по возможности избегать так называемых <<дурацких>> вопросов (я никого не хочу обидеть этим определением: <<дурацкие>> вопросы задают и весьма умные люди~--- в тех сферах, в которых они не вполне компетентны). Таковыми традиционно считаются вопросы, лишенные всякой конкретики. Прекрасный образец <<дурацкого>> вопроса был дан классиками советской фантастики: <<Дорогие товарищи ученые! Третий год у меня в подполе раздается какой-то стук. Объясните, пожалуйста, откуда он берётся>> (А. и Б. Стругацкие, <<Понедельник начинается в субботу>>).

Впрочем, не меньшее раздражение вызывают и вопросы, содержащие заведомо избыточную информацию. Например, детальное описание частотных характеристик монитора в вопросе о настройке звуковой карты. Или~--- полное содержание файла xorg.conf, сопровождающее вопрос о включении русской раскладки клавиатуры.

Как добиться баланса между недостаточной или избыточной информацией пользователю, который еще не очень отчетливо понимает, что существенно для решения данного вопроса, и что~--- нет? К сожалению, единственный рецепт, который можно тут дать, будет тривиальным~--- это размышление. Каковое вообще весьма способствует искоренению <<дурацких>> вопросов: в ряде случаев после обдумывания того, как сделать вопрос <<не дурацким>>, ответ на него находится сам, и необходимость в самом вопросе отпадает.

Впрочем, рискну предложить и другой рецепт (возможно, завсегдатаи форумов со мной и не согласятся~--- это чисто личное). Так вот, при сомнении лучше дать несколько меньше информации, чем заведомый её переизбыток. Лично меня многостраничные листинги конфигов, сопровождающие вопросы на форумах, просто вводят в ступор. При недостатке же информации, вполне возможно, последуют уточняющие вопросы, позволяющие сформулировать проблему адекватным образом. Впрочем, только при соблюдении второго непременного условия форумного общения.

Ибо второе условие, которое на самом деле должно быть первым,~--- это вежливость, вежливость, и еще раз вежливость. Самый <<наидурацкий>> из <<дурацких>> вопросов имеет шанс на ответ по делу, если он задан в политкорректной форме, а не как~--- <<все бросайте на фиг, и идите чинить мой велосипед>>. Тут, уверяю вас, реакция будет более чем адекватной, даже на тех форумах, где ненормативная лексика не приветствуется.

Однако не следует вдаваться и в другую крайность~--- начинать вопрос с самоуничижающего описания собственного <<ламерства>>: подчас это уничижение оказывается именно тем, которое паче гордыни. На мой взгляд, достаточно простыми словами обрисовать меру своей компетентности (или некомпетентности) в данном вопросе. Хотя и без этого можно обойтись: ведь по умолчанию понятно, что тот, кто задает вопрос, как минимум, не считает себя экспертом в затрагиваемой теме.

Третье из правил <<хорошего тона>>~--- избегать повторяющихся вопросов: на постоянных посетителей форума пятнадцатый вопрос из серии <<Как подключить win-модем имя рек>>, действует хуже красной тряпки на быка; даже в том случае, если вопрос сопровождается точным указанием на производителя модемного чипа.

Как избежать повторяющихся вопросов? Большинство форумов имеет функции поиска (те же, что таковой не имеют~--- вряд ли заслуживают наименования форума, их практическая польза сведется к нулю после недолгого времени функционирования). Далеко не всегда эти поисковые средства идеальны (чай, не Google), однако с некоторых попыток близкие темы отыскать удается. И лучше задать вопрос в такой, уже существующей, близкой по смыслу, теме, нежели плодить очередной топик: в крайнем случае, модератор перенесет ваш вопрос куда надо, или просто выделит в <<отдельное производство>>. Разумеется, при соблюдении все того же второго условия.

И, пожалуй, последнее. Ваш вопрос имеет тем больше шансов на ответ, чем явственней из него будет следовать~--- предприняли ли вы перед этим хоть какие-то усилия по его разрешению собственными силами, или нет. И обращение к средствам поиска форума~--- самое малое из таких усилий.

Но предположим, что все завершилось благополучно~--- помощь на форуме была получена, и пользовательская проблема была разрешена в порядке дружеской услуги. Что же дальше? А вот дальше-то и начинается самое главное\dots

<<Дар красен отдарком>>~--- эта древняя мудрость не потеряла своего значения и по сей день. И в ответ на дружескую услугу, как я уже говорил, всегда следует быть готовым оказать услугу ответную. О каком же ответе может идти речь в данном случае? Да все проще пареной репы: с вами поделились своими знаниями~--- и вы должны быть готовым ими же поделиться в ответ. Как? Элементарно: описать решение своей проблемы, на форуме ли, в виде ли статьи на сайте, заметки в блоге, и так далее~--- форм представления информации может быть немерянно. В расчете на то, что ваши материалы помогут кому-либо решить свои проблемы.

И тут мы возвращаемся к тому, с чего начали: к первому методу решения пользовательских проблем. Ведь чтобы поделиться способом решения какой-либо проблемы, в ней следует разобраться более или менее досконально. Благо, уже сказано, как: путем поиска в Google и аналогичных службах, чтения сетевых и литературных источников, сообщений на форумах. И, конечно же, изучения документации. Таким образом, способствуя увеличению общего количества информации на тему Open Source. И, рискну добавить словами одного из персонажей куваевской <<Территории>>, увеличивая суммарное количество добра на земле. Ибо поделиться с другими в благодарность за полученное~--- это единственный к тому способ. По крайней мере, я другого не знаю\dots

\subsection{Читать или не читать?}
Итак, мы пришли к выводу, что чтение документации~--- неизбежность для пользователя Unix. Или, иными словами,~--- один из обязательных компонентов мастерства работы в этой системе. Можно сказать, что работать в Unix и быть свободным от него нельзя. А внутренняя система документации~--- неотъемлемый компонент всех представителей этого семейства операционок. И от нее нельзя быть свободным точно так же, как в обществе нельзя быть свободным от его законов.

Правда, у пользователя все равно остается выбор~--- читать документацию, или не читать её. Так же как в обществе у любого его члена есть выбор~--- знать или не знать его законы. Приходится лишь помнить, что незнание законов общества не освобождает от ответственности за их нарушение. Так и не-чтение документации не освобождает от расплаты за ошибки, совершенные по причине её незнания.

Правда, существуют системы~--- так называемые user-ориенти\-ро\-ван\-ные дистрибутивы Linux (их еще называют дружественными к пользователю), которые, казалось бы, позволяют на первых порах не обременять себя изучением документации. Тогда как другие системы (все BSD и дистрибутивы Linux, в свое время метко названные <<дружественными к админу>>) заставляют вникнуть в документацию с первых шагов их освоения. А наиболее яркие их представители (например, один из дистрибутивов Linux~--- Gentoo) в обязательном порядке требуют этого еще до установки системы~--- иначе пользователь, пожалуй, и установить-то её сможет разве что случайно.

Какие из систем лучше для начального освоения системы~--- вопрос очень спорный, многократно обсуждавшийся на всех мыслимых форумах по нашей тематике, и здесь я на нем останавливаться не буду. Тем более, что на самом деле разница~--- чисто количественная: рано или поздно пользователь самого-рассамого юзерофильного дистрибутива Linux читать документацию все равно будет~--- другое дело, что он может заняться этим уже в процессе практической работы, по мере возможности и необходимости.

Можно провести историческую аналогию: различие между <<дружественными>> и <<недружественными>> Unix-системами в отношении документации~--- примерно такое же, как в отношении освоения законов обществ <<цивилизованных>> и <<варварских>> (кавычки уместны, потому что ни тот, ни другой термин не отражают существа явления). В <<цивилизованном>> обществе за первое нарушение закона, скорее всего, мягко пожурят (или там по попе нашлепают). А в обществе <<варварском>> первое же нарушение закона вполне может стать последним: зарежут нафиг\dots

Тем не менее, тысячелетия своей истории человечество существовало в <<варварских>> условиях. И ничего, выжило\dots А отдельные индивидуумы и по сию пору неуютно чувствуют себя в условиях цивилизованных.

Аналогичный случай и с освоением работы в Unix. Конечно, большинство пользователей его с чего-либо юзерофильного~--- результаты опросов и личные наблюдения позволяют предположить, что обычно в этой роли выступает дистрибутив Linux под названием Mandrake (ныне Mandriva), на протяжении многих (в масштабах времени индустрии) лет удерживавший пальму первенства в отношении <<любви к пользователю>>. И, опять-таки, в большинстве случаев это оправданно: Windows-подобие таких систем позволяет отодвинуть постижение законов POSIX-мира (а чтение документации, как уже было сказано, один из них) на неопределенный срок. Однако в конце концов ситуация может обернуться вполне по анекдоту о поручике Ржевском. Помните, как он ответил на вопрос дамы, любят ли гусары своих лошадей?

И потому всегда находились и находятся индивидуумы, которые не хотели бы, даже случайно, оказаться в положении лошади, любимой гусарами Ахтырского полка. И вот для них-то вполне приемлемым вариантом первого выбора может оказаться Linux-дистрибутив типа Gentoo или, скажем, любая из BSD-систем. Только нужно помнить: в этом случае никакого снисхождения к их неопытности от окружающего мира ждать не приходится. И законы его придется постигать сразу. Хорошо это или плохо~--- обсуждать, опять-таки, не будем: главное, чтобы сделанный выбор был осознанным, сопровождаясь пониманием всех возможных последствий.
\chapter{FOSS и СМИ, лицензии, копирастёж и деньги}
\section{Журнал Linux Format: обзор первого номера} 

\begin{timeline}
20 сентября 2005
\end{timeline}
Отечественная пресса, систематически освещающая вопросы Linux, Unix и Open Source, не поражает изобилием наименований. До недавнего времени должное внимание этим темам уделяли внимание журналы Открытые системы и Системный администратор. Хотя на ниве издания тематических Linux-выпусков отметились и Домашний компьютер, и Chip, и Upgrade Special. Однако специализированного Linux-журнала на русском языке до недавнего времени не было.

Минувшей весной эта пустота была, наконец, заполнена ежеквартальным журналом Chip Linux. Однако необходимость в издании ежемесячного журнала, целиком посвященного тематике Open Source в целом от этого не уменьшилась.

И вот~--- свершилось: на только что закончившейся выставке LinuxWorld компания Линуксцентр представила первый в России ежемесячный журнал~--- Linux Format.

Правда, полностью отечественным его пока назвать можно с определенной условностью: это перевод одноименного английского (и, соответственно, англоязычного) журнала. И не просто перевод, а, можно сказать, точное его зеркало, причем выходящее, фактически, без запаздывания. То есть ныне вышедший сентябрьский номер перевода соответствует сентябрьскому же номеру оригинала. И что же мы увидим внутри? Пересказывать содержание не буду, остановлюсь только на статьях, привлекших мое внимание.

Центральный материал номера, безусловно,~--- специальный репортаж Debian на перепутье, посвященный выходу версии 3.1 этого дистрибутива. В нем, кроме особенностей последнего, рассмотрена также история Debian и его взаимоотношения с <<потомками>>, из которых наибольшую популярность завоевал Ubuntu. Не знаю, как для <<действующих>> пользователей Debian, но для тех, кто, подобно автору этих строк, практически не имел с ним дела, в этом материале можно найти много интересной и полезной информации.

К <<материалу номера>> тесно примыкает статья про Xandros~--- один из как бы коммерческих клонов Debian. Не могу сказать, что я~--- поклонник этого дистрибутива (в уже далеком прошлом он был известен как Corel Linux) и самого принципа создания <<окноподобных>> Linux'ов. Однако всегда полезно знать, что делается в <<смежной>> области.

Вторым по важности мне видится материал, посвященный сравнению текстовых редакторов. В статье рассмотрены особенности как классических Vim и Emacs, так и самых мощных редакторов для графического режима~--- NEdit и Kate, освещены также CoolEdit и gedit. Уделено внимание также легко- и средневесам~--- nano и Minimum Profit (о последнем, каюсь, я услышал впервые). Должен заметить, что выводы автора примерно совпали с моими представлениями. В частности, приятно было увидеть высокую оценку моего любимого NEdit\dots

Очень интересно интервью с Жилем Дювалем~--- первым разработчиком дистрибутива Mandrake и основателем компании MandrakeSoft (ныне~--- Mandriva). В нем <<из первых рук>> рассказывается о причинах, по которым компания оказалась на грани банкротства, и о том, как она выпуталась из такого положения.

Это, так сказать, материалы общего назначения. Однако в журнале уделено достаточно места и более специальным проблемам. Так, можно прочитать забавную статью про <<исправление улыбки>> с помощью стандартных инструментов GIMP. Начинающим программистам адресованы циклы статей про Perl и PHP. И, наконец, исчерпывающее руководство по системе коллективной работы над проектом~--- Subversion. Которое видится очень актуальным~--- ведь многие проекты Open Source ныне используют именно его (чему примером KDE).

Крупные материалы номера перемежаются с небольшими заметками, построенными по типу <<Вопрос~--- ответ>>. Они рассчитаны в основном на начинающих пользователей, но и более искушенным там есть что посмотреть. В частности, лично мне очень кстати оказались соображения о выборе CMS (системы управления контентом сайта).

Оценка переводного журнала читателем в значительной мере зависит от исполнения переводов. И тут приятно отметить, что качество таковых~--- вполне на уровне. Хотя в некоторых статьях некоторые пассажи и показались мне спорными, тем не менее случаев явных ошибок мне не встретилось.

На высоте также полиграфическое оформление журнала. Хотя дело и не обошлось без нескольких мелких ошибок верстки~--- их появление в первом номере вполне объяснимо (и, я даже сказал бы, неизбежно). Думаю, в последующих выпусках такого не повторится.

Неотъемлемой частью журнала является сопровождающий его диск~--- двусторонний DVD, все 9 гигабайт которого заполнены разнообразным софтом. В соответствие с центральным материалом номера, и на диске основной объем занимает дистрибутив Debian~3.1 Sarge. Он дополняется последней версией одного из самых популярных дистрибутивов~--- Fedora Core~4. Немало там и всяких приложений, вплоть до игр.

Резюмирую: первый номер Linux Format произвел на меня очень хорошее впечатление, давненько не доводилось мне читать <<бумажный>> журнал с удовольствием. А прилагаемый диск окажется очень полезным для всех граждан нашей страны, не избалованных чрезмерно хорошим коннектом (к сожалению, таковых пока большинство).

Несколько слов о перспективах. По полученным от редакции журнала данным, в планах их~--- не только <<зеркалирование>> англоязычного издания, но и публикация оригинальных материалов отечественных авторов. По своему уже более чем 15-летнему опыту знакомства с компьютерной прессой рискну предположить, что со времнем <<удельный вес>> таких публикаций будет возрастать.

\section{Linux-пресса на Руси: Вопросы истории} 
\begin{timeline}2006, апрель\end{timeline}

На этих страницах я хотел бы проследить, как менялось освещение Linux на страницах печатной компьютерной периодики.

Здесь не будет ни полной библиографии статей, ни даже упоминаний всех изданий, уделивших место для предмета нашего разговора. Моя цель, скорее, выявить тенденции развития Linux-прессы в прошлом и её перспективы~--- в грядущем. Термин <<Linux-пресса>> употребляется здесь для краткости~--- не следует забывать, что речь идет также о Unix и Open Source вообще.

Но сначала~--- несколько слов о том, с чего все началось. У истоков Linux-прессы лежал журнал <<Монитор>> (вскоре прекративший свое существование), который в 1994 году напечатал статью Владимира Водолазкого о том, как легко и без головной боли установить Linux. В этой статье, вероятно, слово Linux впервые прозвучало в русскоязычном окружении. Актуальность же она сохраняла еще годы спустя, будучи не только источником информации об этой ОС, но и руководством к действию: ксерокопии её ходили по рукам в сопровождении горы дискет с дистрибутивом Slackware, и, вероятно, не одно поколение <<линуксоидов первого призыва>> ставило свою первую систему с этого <<комплекта>>. Итогом той давней публикации стала книга Владимира <<Путь к Linux>>~--- первое (1999 год) отечественное издание на эту тему. Но это~--- уже другая история.

Первым журналом, в котором сложилась устойчивая традиция линуксописания, не прерывающаяся и по сей день, стал <<Мир ПК>>. В нем, начиная с 1995 года, регулярно начали публиковаться как переводные, так и оригинальные статьи о самых разных аспектах ОС Linux. Причем отечественные авторы преобладали, и их публикации проявили отчетливую тенденцию к циклизации: до сих пор памятны циклы статей Игоря Хименко (по совместительству~--- одного из соратников Сергея Кубушина, развивавшего тогда на Украине инновационный дистрибутив KSI; однако и это отдельная тема), посвященные файловой системе Linux, процессам, командной оболочке bash. Правда, со временем традиция циклических публикаций в <<Мир ПК>> угасла…

Начиная с 1997 года, многие общекомпьютерные журналы начинают публиковать статьи по тематике Unix, Linux и Open Source~--- не часто, но относительно регулярно такие материалы появляются в журналах <<PC Magazine/RE>>, <<LAN>> и <<Открытые системы>>. Характер этих публикаций различен. Если <<PC Magazine/RE>> печатает почти исключительно переводные статьи весьма случайного содержания, то <<LAN>> и <<Открытые системы>> (тот же издательский дом <<Открытые системы>>, что и <<Мир ПК>>) отдают предпочтение отечественным авторам, и в их материалах можно наметить ту же тенденцию к циклизации, впрочем, также не получившую развития. Упомяну тут и журнал <<СУБД>>, проживший короткую, но яркую жизнь (оборвавшуюся вскоре после приснопамятного августа 1998 года)~--- это был практически единственный компьютерный журнал <<академического>> стиля. И, хотя специальных материалов о Linux в нем не было, его тематика тесно пересекалась с Unix и Open Source.

В общем, разнородность содержания и отсутствие систематического подхода к Linux-тематике характерна для большинства общекомпьютерных изданий и по сей день. Но были и попытки целенаправленного освещения Unix, Linux и Open Source.

Первым в этом ряду должно упомянуть журнал Byte Россия. Начав свое существование в 1998 году под эгидой издательского дома <<Питер>>, он с первых же номеров начал регулярно печатать материалы по Unix и Open Source, причем переводные статьи сразу же составляли меньшинство по сравнению с работами отечественных авторов. Очень скоро эти материалы были выделены в специальный раздел~--- Byte/Unix, своего рода журнал в журнале, редактором которого стал Алексей Выскубов. Это была первая попытка создания специализированного издания по тематике Unix, Linux и Open Source вообще: помимо статей о Linux, здесь уделялось внимание и BSD-системам, и общим вопросам идеологии свободного программного обеспечения. В частности, именно на его страницах увидели свет переводы статей Николая Безрукова об особенностях разработки программ с открытыми исходными текстами, сохраняющие актуальность и по сей день. Благо, они доступны в сети.

Вообще трудно переоценить роль Byte Россия в развитии русской Linux-прессы того времени~--- тираж его в 2000 году достиг 17 тысяч, и не последнюю роль в этом сыграли материалы Unix-раздела. Однако, в конце 2000 года журнал был продан другому издательскому дому, политика его резко изменилась, раздел Byte/Unix был ликвидирован, полностью сменилась редакционная команда. И хотя в начале 2001 года в нем еще по инерции публиковались некоторые материалы по Linux, интерес к нему со стороны линуксописателей (и, подозреваю, линуксочитателей) был утрачен безвозвратно.

Однако развитие Linux-прессы продолжалось, правда, существенно сместившись в сторону онлайновых СМИ. Эстафету подхватила Компьютерра: в начале 2001 года в рамках этого издательского дома возник проект Софтерра (редактор Сергей Scout Кащавцев) со специальным разделом~--- FreeOS (Федор Сорекс), посвященным свободному ПО вообще и ОС Linux (а также Free-, Open- и прочим BSD), в частности. Все материалы проекта публиковалась в Сети, но некоторые статьи раздела попадали также и на страницы <<бумажной>> Компьютерры.

Наибольшим успехом проекта можно считать выход тематического номера журнала <<Домашний компьютер>>, тоже относящегося к издательскому дому Компьютерра (ДК, 2002, \textnumero12). Созданный под идейным руководством Максима Отставнова, он содержал материалы по всем аспектам устройства и использования Linux, в том числе и в домашних условиях. К сожалению, этот успех был последним: в течение первых месяцев следующего года проект FreeOS плавно сошел на нет (да и Софтерра как таковая~--- тоже) , материалы его исчезли из прямого доступа с сайта журнала и, похоже, утрачены безвозвратно.

Следующим номером в эстафетном забеге Linux-прессы оказался Upgrade~--- усилиями Алены Приказчиковой и Сергея Голубева, начиная с середины 2002 года, почти каждый номер в <<софтверном>> разделе содержал материалы про Linux и Open Source, чему способствовал и приток новых авторов. Тенденция к систематическому освещению темы нашла свое воплощение в тематическом выпуске Upgrade Special, посвященном Linux вообще и дистрибутивам Live CD в особенности (середина 2004 года). Однако, как и в случае с ДК, он одновременно стал символом упадка интереса к Linux и Open Source со стороны редакции Upgrade: в последующее время материалы по этой тематике появлялись там лишь эпизодически.

Сказанное не значит, что остальная компьютерная периодика перестала уделять внимание Linux-тематике. С самого дня своего создания (начало 2001 года) значительную активность на этом поприще проявлял журнал Chip~--- вплоть до выпусков спецномеров, целиком посвященных Linux, и сопровождавшихся компактом с тем или иным его дистрибутивом. В 2002 году возник журнал <<Системный администратор>>, коему по титулу положено освещать вопросы, связанные с сетевым администрированием~--- а в этой сфере Linux и свободные BSD-системы доминируют и по сей день (особенно на территории РФ). И материалы о них появляются на его страницах с завидной регулярностью. Время от времени к тематике Linux и BSD обращается <<Хакер>>~--- правда, подчас с материалами весьма спорными.

В ряду общекомпьютерных периодических изданий следует выделить упомянутый ранее журнал <<Открытые системы>>. Его специфика в том, что, наряду со статьями популярного направления он публикует и сугубо научные материалы, посвященные теоретическим вопросам Computer Science (что не удивительно, учитывая его академические корни). К сожалению, популярность его все время падает, а в последние годы он практически пропал из розничной продажи.

Журналы, публикующие материалы по нашей тематике лишь эпизодически, на общую картину Linux-прессы влияли слабо. Её путь, со дня зарождения и почти до сегодняшнего момента, можно описать как серию попыток создать специализированное издание, профилированное на Unix, Linux и Open Source. Ни одно из этих предприятий успехом не увенчалось. Причины неудач можно выискать разные, но в основе каждый раз лежало одно: абсолютное равнодушие руководства общекомпьютерных изданий (даже таких, как <<Открытые системы>>, которых, казалось бы, даже титул обязывал) к Linux-тематике. Дело помощи линуксоидам следовало брать в руки самих линуксоидов.

Таким образом, мы плавно подошли к 2005 году. Я не пророк, но думаю, что он войдет в историю как второй, после 1998 года, переломный рубеж в развитии Linux в России. Если первый ознаменовался выходом прототипа первого российского дистрибутива~--- Mandrake Linux от IPLabs Linux Team (в последующем~--- Altlinux) и попыткой создания первого специализированного издания (Byte/Unix), то в течении 2005 года, во-первых, в России резко активизировалась деятельность лидеров мирового дистростроения (Red Hat и Novell/Suse), и, во-вторых, к нему приурочены первые массовые выставки-конференции, специально посвященные тематике Open Source (Open Source Forum Russia, LinuxWorld Russia, LinuxLand на Софтуле).

А для Linux-прессы год этот памятентем, что в течении его начали выходить первые специализированные периодические издания, полностью посвященные Linux и Open Source~--- Chip Linux Special и Linux Format.

Chip Linux Special~--- ежеквартальное издание, которое берет свое начало с весны 2005 года. Он генетически связан со специальными выпусками журнала Chip, и издается тем же издательским домом <<Бурда>>, что и прародитель. Комплектуется исключительно оригинальными статьями отечественных авторов. Специфика журнала~--- сконцентрированность вокруг материалов по <<титульной>> ОС, расширение тематики в сторону других свободных систем (например, BSD-семейства), насколько я знаю, не планируется.

Эта статья в целом уже была написана, когда поступила печальная весть: журнал Chip Linux Special более издаваться не будет. О причинах этого мне ничего не известно, да и сами источники сложно назвать официальными: помимо личного общения с членами команды Chip Special Linux можно назвать только новость на Linux.org.ru, в которой (со ссылкой на телефонный разговор с издательским домом <<Бурда>>) сообщалось, что журнал было решено закрыть. Хочется надеяться, что <<слухи о его смерти сильно преувеличены>> и Chip Special Linux еще порадует нас своими выпусками~--- благо приобрести его можно было <<от Москвы и до Находки>>.

Linux Format~--- ежемесячный журнал, родившийся в сентябре 2005 года усилиями фирмы Линуксцентр, занимавшейся онлайновой торговлей дистрибутивами свободных ОС и профильной литературой. Он представляет собой перевод одноименного английского журнала, своего рода русское его зеркало: каждый номер по содержанию соответствует аналогичному номеру оригинала (хотя и выходит с некоторым запозданием, обусловленным затратами времени на перевод, верстку и печать). За одним исключением: начиная с 4-го номера, в журнал включаются и оригинальные статьи отечественных авторов. Рискну предположить из исторических аналогий, что с течением времени процент последних будет возрастать. Ведь все патриархи отечественной компьютерной прессы (Мир ПК, Компьютер-Пресс, PC Magzine/RE) начинались некогда как чисто переводные издания.

Отличительная черта Linux Format~---открытость. Все стороны его существования~--- от содержания очередного номера до причин недоставки конкретного экземпляра~--- можно обсудить на его форуме. Члены редакционной команды в обсуждении участвуют~--- и, должен вам заметить, к обоснованным мнениям читателей всегда прислушиваются.

Впрочем, много говорить об этом издании не буду: раз вы читаете материал этого номера, то знакомы с ним в достаточной степени. А потому завершу свое сочинение рассуждениями на тему, каким видится идеальный журнал по тематике Unix, Linux и Open Source.

\section{\dotsИдеальный журнал} 
\begin{timeline}2006, апрель\end{timeline}

Традиционно <<толстые>> компьютерные журналы разделяются на две части: блок новостей и, так сказать, <<тело>> журнала~--- собственно материалы номера. Оправдана ли такая организация в век тотальной интернетизации? В век, когда все, имеющие хоть какое-то подключение к Сети, получают интересующие их последние известия из онлайновых источников, новостные разделы даже компьютерных еженедельников выглядят сборником анекдотов с бородой Карла Маркса. Что же тогда говорить о <<новостях>> ежемесячников?

Так что же, ликвидировать новостные блоки? Отнюдь~--- это было бы политически неправильным. Увы~--- изрядная часть населения постсоветских пространств лишена прелестей Интернета, и Печатное Слово, пусть несколько устаревшее (да и доставленное, силами российской почты, с запозданием), для нее~--- единственный источник информации. А потому предлагается компромиссный вариант: заменить сборники анек… пардон, новостей~--- аналитическими их обзорами. Которые, давая достаточно сведений читателям, не имеющим подключения к Сети, в то же время не вызывали бы раздражения своей <<бородатостью>> у тех, кто таковое имеет. А в идеале~--- были бы просто интересны сами по себе. Конечно, составление таких обзоров~--- дело нелегкое, но оправдается повышением читательского внимания.

Теперь об основной части~--- статьях. Журнал, рассчитывающий на самоокупаемость (а в идеале~--- и на принесение прибыли) обязан ориентироваться на самые широкие пользовательские массы И потому должен содержать материалы нескольких градаций: для совсем начинающих, для <<действующих>> пользователей, и для тех, кто ставит своей задачей углубленное изучение каких-либо частных вопросов. Хорошо это или плохо~--- обсуждать не будем, такое <<смешение жанров>> на данном этапе развития Linux-прессы является необходимостью. Как показала трагическая кончина журнала <<СУБД>> (да и безрадостная судьба аккумулировавших его <<Открытых систем>>), специализированное издание <<для профи>> пока не имеет шансов выжить на постсоветском пространстве. С другой стороны, ориентация издания на <<чайников>> чревата потерей интереса к нему, как только <<чайники>> таковыми быть перестанут (а с помощью хорошего журнала это произойдет очень быстро).

Какой видится компоновка материалов? Возможны варианты: по степени <<продвинутости>> предполагаемого читателя, по тематике, в том числе и с выделением некоего центрального материала и его <<системного окружения>>. Впрочем, я~--- категорический противник <<темы номера>>, что приемлемо для еженедельника, но для ежемесячного издания смертельно: ведь если читателю не интересна именно эта тема номера, он на целый месяц лишается возможности что-либо почерпнуть из любимого журнала.

Формат большинства журналов общего назначения предполагает преимущественно двух- или, реже, четырехполосные статьи, что для глубокого изложения многих животрепещущих проблем явно недостаточно. Конечно, проблема эта для периодических изданий не решаема в принципе: увеличение объема статей повлечет за собой сужение тематики номера и риск утраты читательских симпатий. Однако некий компромисс тут возможен~--- в виде пролонгированных из номера в номер тематических циклов.

Для журнала тематики Unix и Linux очень существенен общий дизайн~--- и здесь положение, в большинстве случаев, не удовлетворительное. Многоколоночная верстка, пришедшая из мира рекламно-развлекательной периодики, и терпимая в периодике, так сказать, литературно-повествовательной, оказывается проклятием в изданиях технического профиля. В нашем случае это относится в первую голову к командам и листингам. Писать без них серьезно про Linux и Unix~--- это все равно, что писать про музыку без нот, или про живопись~--- без репродукций. Но~--- увы~--- длинные команды или содержимое конфигурационных файлов вписывается в облик страницы традиционного современного журнала ничуть не лучше, чем <<парень в джинсах и кожаной куртке>>~--- в интерьер ресторана для новорусского истеблишмента.

Я прекрасно понимаю, что общий дизайн издания определяется множеством привходящих факторов (в том числе и политических). Но и тут возможны варианты. Например, давать необходимые команды и листинги <<внеформатными>> врезками. Или~--- сделать онлайновое дополнение к журналу, которое содержало бы именно ту часть статей, которая подлежит использованию методом Cut\&Paste. Такое дополнение, не дублируя содержание основного материала, будет способствовать его практическому использованию.

\section{К вопросу о торговых марках} 
\begin{timeline}2001\end{timeline}
Интересный вопрос: если система выглядит как UNIX, делает то же, что UNIX, причем почти тем же образом, что UNIX, почему она~--- не UNIX? Ответ прост~--- угадайте с одного раза: потому, что UNIX~--- торговая марка, права собственности на которую принадлежат\dots да поди разберись, кому они нынче принадлежат. А ведь UNIX существует уже больше тридцати лет, и всем доподлинно известно, что это такое, более того~--- имя это будут помнить еще долго, даже если самой системе суждено будет уйти в историю. Но может ли кто-нибудь сходу назвать имен хотя бы пары-тройки правообладателей этой марки, сменившихся за все эти годы?

Поводом для обращения к этой теме послужила реклама очередного компьютера Intel Inside на процессоре Pentium то ли~III, то ли 4, в очередной раз попавшаяся на глаза в очередном журнале, с обычным в таких случаях пояснением (видимо, для особо продвинутых), что Intel Inside\textregistered и Pentium\textregistered~--- зарегистрированные торговые знаки компании Intel Corporation. Таких реклам я видел несчетно, и всегда они по непонятной причине вызывали у меня чувство незавершенности. Но на сей раз меня как озарило: в них не хватает знака Trade Mark или Registered после цифры! А ведь было бы круто, согласитесь:
\begin{shadequote}{}
Intel\textregistered Pentium\textregistered~4\textregistered (или \texttrademark, без разницы)
\end{shadequote}
А технологически (но не морально-юридически) устаревший Pentium~III открывает еще более богатые возможности: знак~\textregistered или \texttrademark можно поставить и после латинской единицы, и после двойки, и после тройки. Закрепив, таким образом, свои права на, по крайней мере, первые три члена натурального ряда чисел. И требовать лицензионные отчисления с тех, кто воспроизведет публично латинские цифры IV (за одну единицу), VII (и за две единицы, и за двойку), а паче всего~--- за~VIII: сразу и за три единицы, и за двойку как компонент последовательности, и за тройку по совокупности.

Это напоминает мне деятельность одного нашего достославного издательства, выпускавшего (и, вроде бы, выпускающего и поныне) книжки про Конана-варвара. Не могу отказать себе в удовольствии процитировать пару фраз с оборота титула:

\begin{shadequote}{}
Конан\texttrademark, Conan\texttrademark~--- зарегистрированные торговые марки\dots запрещается воспроизведение этой книги в любой форме\dots а также использование имени Конан\texttrademark.
\end{shadequote}
Не скрою, увидев это впервые, я был в немалом восхищении и, в меру собственной испорченности, начал размышлять о последствиях. Думается, в Ирландии Конаны ныне встречаются если и не на каждом шагу, то уж не реже, чем на Руси~--- Ярославы или Всеволоды. И вряд ли у каждого из них стоит в паспорте (или чем там они личность удостоверяют) ссылка на собственника торговой марки: дескать, не просто Конан я, 
а Конан\texttrademark, что есть зарегистрированная торговая марка компании Имя рек. И следовательно, каждому можно вчинить иск с требованием компенсации\dots

Но это еще не все. Ведь Артур КОНАН Дойл тоже не удосужился предвидеть, что второе его имя будет зарегистрированной торговой маркой. Сам сэр Артур, к прискорбию правообладателей, ответить по всей строгости закона уже не сможет, но есть ведь потомки~--- а сказано, что грехи отцов падут на детей до третьего и четвертого колена: пока время не ушло, и поколения не сменились, стоит подать на них в суд за присвоение торговой марки.

Да и в многочисленных изданиях книг писателя-правонарушителя нигде не отмечено (по крайней мере мне не попадалось), что написал их Arthur Conan\texttrademark Дойл или хотя бы Артур Конан\texttrademark Дойл. А уж своего брата-издателя засудить~--- милое дело.

Если же вспомнить минимум про двух ирландских королей-Конанов (без~\texttrademark) начала нашей эры,~--- перспективы возникают поистине необъятные. Сами-то они злонамеренно улизнули от ответственности, но ведь, как известно, все ирландцы происходят от королей. Значит, за нарушение прав на торговую марку и ответить должны все. А это уже пахнет иском на государственном уровне~--- ко всей Ирландской Республике. Да и к Северной Ирландии заодно~--- наверняка и там потомки Конанов-королей найдутся. А Северная Ирландия~--- часть Соединенного Королевства, со всеми вытекающими\dots

Однако это все мелочи по сравнению с гениальным достижением известной фирмы, выразившемся в знаменитом словосочетании Microsoft\textregistered Windows\texttrademark. Тут уж и впрямь дух захватывает от полета фантазии~--- окон-то в мире производится ого-го сколько! Получается, что в печати слово <<окно>> во избежание нарушения этого самого~\texttrademark рекомендуется употреблять только в единственном числе, поскольку на X~Window System пока, вроде, никто своей торговой марки не навесил. С первой же половиной символа софтверной индустрии вышла недоработка: можно ведь было и более предусмотрительно значки расставить~--- скажем, так: (Micro{\textregistered}Soft\textregistered)\texttrademark. Тем самым в фирменную собственность можно было бы оприходовать сразу и все мелкое, и все мягкое. А заодно и теплое, с мягким часто путаемое\dots

Post Scriptum: все упомянутые (и не упомянутые) в статье торговые марки являются вроде как собственностью их правообладателей.

\section{Еще раз к вопросу о так называемых пиратах} 
\begin{timeline}2004 г\end{timeline}
На заре своей карьеры чукчи-писателя я несколько раз порывался описать свое видение проблемы т.~н.~пиратов (в первую очередь в IT-сфере, разумеется,~--- но и в аспекте общечеловеческих представлений тоже). Сначала~--- все не мог собраться по разным причинам, потом~--- потому как лень было разбираться с написанным ранее другими (а написано, как все вы знаете, на эту тему немерянно).

А в конце концов пришел к точке зрения великого историка Ключевского. Который, как известно, в свое время (еще в позапрошлом веке) категорически отказывался обсуждать проблему призвания варягов на Русь, мотивируя это тем, что все точки зрения, хоть как-то основанные на фактическом материале, уже высказаны. И ни одну из высказанных точек зрения, базируясь на имеющейся фактуре, доказать невозможно (в скобках заметим~--- и опровергнуть тоже, по крайней мере~--- из числа разумных). Тем более, что личного интереса к проблеме пиратства у меня уже не было по понятным причинам.

Однако времена меняются, и даже в проблеме Рюрика сотоварищи наметились позитивные сдвиги. Так что, может быть, пора вернуться и к нашим IT-флибустьерам, <<братьям по кров\'{и} упругой и густой>>?

Ну и последним (по времени, не по значению) побуждающим мотивом послужила очередная дискуссия на очередном форуме, ныне в Бозе почивше. Которая, с одной стороны, не успела разрастись до полной необозримости, с другой~--- затронула почти весь круг related themes. Так что нижеизложенное будет в значительной мере состоять из цитат с этого самого трейда. Все цитаты даны методом cut and past, без правки.

Историю российского компьютерного пиратства можно исчислять примерно с момента времени минус 10 лет. Когда где-то на исходе 93-го~--- заре 94-го на Руси появились первые сборники под значимым именем All for Windows. Еще на <<золоте>>, без всякого полиграфического оформления (или с лабелами, напечатанными на принтере, подчас матричном). А под сакраментальным в дальнейшем термином Windows имелась в виду версия 3.1 этой системы. К слову~--- были такие диски и под прозванием All for OS/2, а чуть позже, уже в тиражную эпоху, появились и аналогичные подборки All for Linux\dots

Поначалу диски этого типа не то что не тиражировались~--- но даже и не наличествовали в открытой продаже. Писались они в основном под заказ, со стриммерной ленты (происхождение которой не было ведомо верховному божеству ни одной из мировых религий). Для вхождения в <<стриммерную>>, если так можно было выразиться, требовался не просто пропуск~--- а весьма надежная рекомендация общих знакомых знакомых знакомых, как минимум\dots

Если мне не изменяет память, цена да диски эти составляла около 30 уев (впрочем, и слова такого тогда еще не было; хотя ихние доллары уже тогда ничем не отличались от наших баксов). Точного эквивалента уже не помню, но сумма по тем временам~--- очень значительная\dots

Как-то мне довелось услышать, применительно к <<пиратскому>> софту, выражение: это не пиратский софт, а трофейный (копирайт, если не ошибаюсь, принадлежит Алексею Смирнову из Altlinux'а). Так вот, программы с тех дисков в полной мере соответствовали такому определению. А их распространители, следовательно, должны были бы называться вовсе не пиратами, а, скажем, призоньерами или каперами\dots

Хотя тут можно вспомнить и времена почти былинные~--- со второй половины 80-х по самое начало 90-х, эпоху массовой PC'кизации постсоветской России. Когда все (все!) попадавшие на родину мирового социализма программы имели именно трофейное происхождение. И не важно, что брались они не в абордажных боях, а выпрашивались в кулуарах международных совещаний.

Ну и совсем уж седая древность~--- это времена ЕС'ок и СМ'ок, софт к которым добывался в боях в прямом смысле этого слова~--- пусть и бойцами невидимомого фронта. Впрочем, в то героическое время я, в информационном смысле, был крайне мал, и свидетельствовать о том не могу\dots

А в промежутке времени, между классическими буканьерами советской эпохи и их золоторежущими эпигонами середины 90-х, тоже был интересный период времени. Боюсь, многим в это трудно поверить, но это~--- было. В один из первых в Москве отделов, торгующих <<правильным>> софтом (в Доме книги на Арбате, тогда еще имени всероссийского старосты Калинина), заходили и спрашивали: а скоро ли завезут Borland C, или там Borland Pascal, или еще какое FoxPro. И когда сие радостное событие свершалось~--- у отдела буквально выстраивалась очередь, почти как в винный отдел во времена Горбачева. И покупали~--- эти вот коробки, сопровождаемые килограммами документации, кто~--- по безналу на свои конторы (часто госбюджетные, излишними финансами не отягощенные), а кто~--- и за свои кровные, наличные, сопоставимые с месячной зарплатой госбюджетного трудящегося.

Не прошу прощения за столь длинную преамбулу: она призвана проиллюстрировать ошибочность тезиса о врожденной, чуть ли не генетической, тяге советского (и постсоветского) человека к контрафактной продукции.

И к тому же эта преамбула неожиданно подвела меня (и, надеюсь, моих читателей) к первой цитате из помянутого выше топика:
\begin{shadequote}{}
Если закон вынуждены нарушать 99,9\% граждан, то проблема явно не в самих гражданах\dots
\end{shadequote}

Впрочем, на тему эту кто только не высказывался~--- начиная с бессмертного Салтыкова-Щедрина\dots И смешно звучат высказывания, что это началось только сейчас. Ни один руководитель советского производства, начиная с уровня младшеначальственного, без нарушений законов, инструкций, подзаконных актов и прочего просто не в состоянии был бы выполнять свои должностные обязанности. О чем все знали снизу доверху, и что культивировалось совершенно сознательно, дабы в любой нужный момент любого руководителя можно было бы взять\dots ну сами знаете за что. Надеюсь, никто из моих читателей не испытывает иллюзий в отношении того, что с тех времен что-нибудь изменилось.

Будем считать первую цитату~--- \textit{pro} ворованный софт. Теперь~--- \textit{contra}:
\begin{shadequote}{}
Ну поймите же, наконец, что нельзя быть честными только тогда, когда Вас это устраивает.
\end{shadequote}
С одной стороны, все верно: <<Честь у честного не купишь, честный чести не уступит, честь нужна ему как свет>>. Но с другой: можно ли считать нечестным человека, который к тому вынужден? Школьника, которого учат (в классе~--- не в подпольной воровской школе) набивать буковки на ворованной копии Word'а? Студента, учащегося проектированию на ворованной копии AutoCAD, или программированию~--- на столь же контрафактном компиляторе? Да еще под управлением ОС, копия которой, с точки зрения буквы закона, столь же далека от лицензионной чистоты\dots

А можно ли считать нечестным того, кто создает им такие условия? То есть лиц, занимающихся материально-техническим снабжением (или как там это называется) школы/ВУЗа/детсада. Предположим, что такой ответственный дядя~--- для определенности назовем его условно директором,~--- решит быть честным по отношению к правообладателю соответствующего софта. И весь свой бюджет вбухает в легальное программное обеспечение. Как он поступит по отношению к своим воспитанникам? Да, ИМХО, самым бесчестным образом! Потому как он туда поставлен для того, чтобы создавать своим воспитанникам нормальные условия учебы. Включающие, помимо всего прочего, школьные завтраки, удобные парты и еще\dots да мало ли чего еще им требуется\dots

Я, конечно, понимаю, что детей дяди Билла, дяди Ларри и прочих дядей жалко. Но ведь своих-то детей~--- завсегда жалче. И возразить против этого может либо очень циничный, либо очень наивный человек.

Предвижу возражение, очень емко и лаконично сформулированное одним из участником дискуссии:
\begin{shadequote}{}
Тогда Вам к Кремль!
\end{shadequote}

Несмотря на блеск и афористичность, вынужден отмести. Мне (и, надеюсь, всем читателям, имеющим потомство) не очень понятно, почему человек, умеющий учить детей, должен идти в Кремль (или даже в президентский кабинет). Где он в лучшем (для него) случае из хорошего директора школы превратиться в обычного (если уж очень повезет~--- в не самого плохого) чиновника. Столь же неспособного изменить существующее положение вещей, как не смог это сделать ни один из отягощенных благими намерениями ранее\dots

Из чего~--- следующая цитата из серии \textit{contra}:
\begin{shadequote}{}
На мой взгляд главная причина по которой процветает пиратство в России это то что государство не борется с пиратством, точнее делает вид что борется.
\end{shadequote}

Возникает вопрос~--- а почему они только делают вид? Попробуем найти объяснение в следующей цитате:
\begin{shadequote}{}
\dotsтак как пришлось самому столкнуться и увидеть, что за <<пиратами>> стоят настоящие бритоголовые бандиты.
\end{shadequote}
Вполне верю. Однако за каким видом деятельности они не стоят? Разве что за абсолютно убыточным\dots И еще:
\begin{shadequote}{}
\dotsребята, которые рассуждают о высокой миссии пиратов по созданию в России цивилизованного рынка видео (музыки, информатики), просто не понимают, каким харям по концовке они несут бабки.
\end{shadequote}

Однако вспомним~--- на протяжении 70-ти лет советской власти весь советский народ, как один человек, нёс свои бабки, недоплаченные им в виде хлебной пайки, сначала комиссарам в кожанках, потом наркомам в скромных кителях, потом министрам в галстуках не в тон. А ныне отдает их в виде подоходного налога чиновникам в костюмах от Диора. Не они ли маячат за теми самыми бандюганами в цепурах?

И что мы имеем в итоге? А имеем мы то, что имеем. Предположим, наше доблестное правительство\dots Чуть было не написал: <<партия и правительство>>, но вовремя вспомнил, что партий у нас теперь много. А при советской власти все жаловались~--- система у нас однопартийная, потому что больше нам не прокормить; оказалось~--- могем, если захотим\dots

Так вот, предположим, что будет, если правительство наше начнет бескомпромиссную борьбу с пиратством. Как раньше боролись, например, с пьянством. Помните~--- был указ о борьбе с пьянством, потом~--- об усилении борьбы с пьянством, потом~--- об искоренении пьянства\dots Простой соввласть еще несколько лет~--- был бы и указ об усилении искоренения. Так вот на такое предположение отвечаем цитатой:
\begin{shadequote}{}
Например у нас в городе в декабре 2003 началась антипиратская компания, несколько заказных рейдов (заказывали конкурентов) и всё. Большинство фирм как работали на пиратском софте так и работают, ни один магазин пиратских дисков не закрыт. План выполнен, отчёт в Москву отправлен, можно пользоваться пираткой дальше.
\end{shadequote}
Это~--- в масштабах города. А в масштабе всесороссийском имеем еще более яркий пример: тендер на поставку в средние школы изобилия всякого разного Windows. помнится, я тогда сочинил заметку на эту тему, которая нынче показалась мне не лишенной актуальности.

В общем, история Государства Российского учит нас, что наведение порядка <<только кнутом>> всегда давало только одну его разновидность~--- кладбищенско-лагерную. Как сказал незабвенный же Алексей Константинович aka Толстой:
\begin{verse}
Приемами не сладок,\\
Но разумом не хром,\\
Такой завел порядок~---\\
Хоть покати шаром.
\end{verse}
В общем, если вспомнить одну из первых цитат этой заметки, что нельзя быть честным, когда это устраивает. Потому что~--- можно быть честным только тогда, когда это получается.

Все ли так безрадостно? Нет. Ибо на горизонте уже замаячил <<призрак батьки Махно>>. Со времен <<Красных дьяволят>> мы привыкли воспринимать его как персонаж сугубо карикатурный. Однако  неслабого был ума мужик, этот батька, хотя и не отягченный излишним образованием. И не призывал он грабить награбленное, хотя эксцессы и случались (но по сравнению с доблестными красными командирами был он аки ангел, только без крылышек). А хотел он одного~--- как сказали бы нынче, \textit{fuck off, people}, от меня: и красные, и белые, и прочие зеленые). И в своем Гуляй-поле учредил он народную республику~--- так, как это представлялось ему (и его согражданам в большинстве своем). Куда не хотел пускать ни красных, ни белых\dots

Это я веду к тому, что жить в пиратском обществе и быть свободным от него нельзя, как говорил один бородатый пират-теоретик. Но можно~--- по возможности абстрагироваться от его реалий. Красть (в данном случае я имею ввиду конкретно софт)~--- большинству нормальных людей\dots ну скажем так, не приятно (вне зависимости от тяжести и неотвратимости наказания). Не красть~--- казалось бы, невозможно. Да и западло кажется, честно говоря, покупать легальный проприетарный софт, явно своих денег не стоящий. И притом на кабальных условиях, прямо провоцирующих нормального человека к их нарушению.

Но ведь есть~--- Open Source во всех своих проявлениях. И именно он позволяет нам создать свое Гуляй-поле, пусть даже в масштабах отдельно взятого персонального компьютера. Мир в одночасье мы таким образом не переделаем~--- но, верю, суммарное количество добра на земле хоть чуть-чуть, да увеличится. И когда приходишь к такому выводу, споры о вреде или пользе пиратства становятся просто скучными\dots

\section{Еще раз про свободу и халяву} 
\begin{timeline}6 сентября 2005 г\end{timeline}
Эта заметка написана как отклик на статью Виктора Вислобокова <<Ещё раз о свободном ПО, лицензиях и т.д.>> Она поднимает тему свободного и проприетарного (по-русски~--- частнособственнического) программного обеспечения. Казалось бы, тема эта столько раз обсуждалась как в Сети, так и на бумаге, что в ней давно пора бы уже поставить все точки над \textit{i}. Тем более, что документы основоположников движения за свободный софт не оставляют оснований для двусмысленного их толкования.

И, тем не менее, такое толкование, когда теплое смешивается с мягким, встречается сплошь и рядом. И вследствие этого статья Виктора представляется чрезвычайно своевременной и актуальной.Однако начать я хотел бы с воспоминаний о совсем другой статье, достаточно древней в масштабах компьютерной эры. Статье Евгения Рыбникова <<Дюжина ножей в спину Open Source>>. Потому что она и являет собой квинтэссенцию мифов, легенд и предубеждений, сформировавшихся вокруг открытого и свободного программного обеспечения. И уж если такие предубеждения бытуют в среде IT-профессионалов~--- а смею вас уверить, тот, кто скрывается под псевдонимом Евгения Рыбникова, действительно высокий профессионал в этой области\dots Так вот, если такие предубеждения бытуют в среде IT-профессионалов, то что же говорить, о людях, по роду своей основной деятельности лишь вынужденных использовать компьютеры? И потому боюсь, что к теме открытого и свободного программного обеспечения придется обращаться еще не раз.

А теперь~--- к статье Виктора. Для начала, во избежание недоразумений, оговорюсь: я категорически и полностью согласен со всеми её основными положениями. И в настоящей заметке лишь хотел бы а) несколько дополнить её, и б) высказать свое мнение по некоторым частным её моментам, которые представляются мне спорными.

Статья начинается с классификации программного обеспечения по степени, если так можно выразиться, доступности. Выделяются такие группы:

\begin{enumerate}
\item Коммерческое ПО 
\item Условно-бесплатное ПО 
\item Бесплатное ПО 
\item ПО с открытыми исходными текстами 
\item Свободное ПО 
\end{enumerate}

О коммерческом софте говорить особенно нечего~--- на Руси любой ребенок знает, что такое Windows и Word. А также то, что покупать их по коммерческим каналам совсем не обязательно~--- на то есть базары и лотки. Так что~--- замнем для ясности.

В группу условно-бесплатного софта (т.~н.~shareware) объединено несколько различных на первый взгляд способов распространения программ. Однако все они объединены одним: распространяемые на этих условиях программы для полноценного использования требуется тем или иным способом оплатить. Еще мне хотелось бы подчеркнуть здесь, что собственно shareware-программ в настоящее время практически не встречается. Как правило, под этим псевдонимом выступают либо trial-версии, либо варианты коммерческих программ с обрезанной функциональностью.

Бесплатное программное обеспечение (freeware) в полном смысле отвечает своему русскому имени~--- но не английскому, тут возможны разночтения, связанные со знаменитым отличием бесплатного пива от свободного слова. Общее для этой группы программ~--- то, что за их использование платить не нужно или не обязательно. Последняя категория (т.~н.~donationware) по сути своей гораздо больше соответствует shareware в его исконном значении: понравилась программа (или просто захотелось отблагодарить разработчика)~--- заплатил, не понравилась~--- не заплатил\dots

К сказанному здесь Виктором можно добавить одно: по моим наблюдениям, длящимся вот уже почти 15 лет, статус freeware обычно~--- сугубо временный. В случае успеха такой бесплатный проект очень быстро становится shareware (в понимании пункта 2) или даже просто обычным коммерческим. А в случае неудачи~--- тихо и незаметно прекращает свое существование. Исключений из этого правила очень мало. Мне на память приходит только Arachnophilia Пауля Лютуса~--- типичный, кстати сказать, пример donationware.

А вот по поводу ПО с открытыми исходными текстами я бы поспорил. Как показало обсуждение, Виктор имел ввиду лишь одну, совершенно конкретную разновидность софта с открытыми исходниками~--- т.~н.~\textit{comercial with sources}. То есть~--- обычное проприетарное ПО, сопровождаемое (подчас за отдельную плату) исходными текстами. Однако это~--- относительно редкий способ распространения программ широкого назначения. Мне в этой связи вспоминается, пожалуй, только BSD/OS, известная также как BSDi~--- операционная система BSD-семейства, основанная на той же кодовой базе, что и FreeBSD, но, в отличие от последней, распространяемая за деньги. А вообще 
\textit{comercial with sources}
 более характерен для узкоспециализированных программ и заказных разработок.

Обычно же термину Open Source придается гораздо более широкое понятие, подразумевающее просто доступность (физическую и юридическую) исходных текстов некой программы. И таким образом, свободное программное обеспечение, составляющее следующую группу, оказывается лишь подмножеством ПО с открытыми исходниками. Ибо свободный софт подразумевает доступ к исходным текстам~--- очевидно, что без этого невозможно реализовать ни право на изучение программы, ни право на её модификацию.

При этом открытое ПО оказывается почти столь же неразрывно связанным с ПО свободным: ведь мало радости от свободы изучения или модификации, если нет права распространять модифицированную (например, исправленную) версию. И потому на практике обычно оказывается, что все открытые, но не совсем свободные проекты либо постепенно эволюционируют в сторону полного Free Software, либо утрачивают свой свободный статус.

Ну собственно по главной теме статьи~--- свободному ПО~--- также есть о чем подискутировать с Виктором. Правда, большая часть моих вопросов разрешилась в личной переписке. Тем не менее, согласиться с тем, что подлинно свободное ПО~--- это ПО под лицензией GPL, я не могу. Ибо софт, распространяемый под лицензией BSD и родственными (MIT, X-консорциума и другими, обычно объединяемыми понятием <<университетские лицензии>>) предоставляет пользователи все степени свободны~--- свободу использования, изучения и модификации, свободу распространения. Более того, в некотором отношении BSD-лицензия даже <<более свободна>>, так как не требует непременно свободного распространения продукта, созданного с использованием BSD-лицензированного кода.

Конечно, как правильно отметил Виктор, при этом возникает некоторый риск: человек, использовавший 95\% свободного кода, добавляет в него 5\% функциональности и получившийся продукт распространяет как закрытый коммерческий. Однако этим можно пренебречь: ибо по условиям BSD-лицензии он может закрыть только 5\% своего личного вклада, права <<наложить лапу>> на изначально свободные 95\% он не получает. И на практике такие коллизии решаются очень просто: на базе свободного кода создается fork (порождение) со столь же свободным статусом. Примером тому~--- недавние баталии вокруг изменившейся лицензии XFree86: сторонники полной свободы софта просто создали полностью свободный Xorg, который успешно развивается. Хотя, убей меня Бог, я так и не понял, в чем же новая лицензия XFree86 ограничивает свободу\dots

А вообще, именно BSD-лицензированные продукты замечательно демонстрируют мирное сосуществование свободного и проприетарного софта. Примером чему~--- MacOS X, основанная на свободных микроядре Mach и системном обрамлении FreeBSD, надстроенными проприетарным графическим интерфейсом.

Подчеркну~--- я вовсе не утверждаю, что BSD-лицензия однозначно лучше GPL: в некоторых случаях лучше сработает одна, в иных же~--- другая. Более того, на мой взгляд, в 99 случаях из ста вообще не имеет значения, под какой именно из свободных лицензий распространяется тот или иной программный продукт. Тем не менее, можно представить ситуации, когда BSD-лицензия оказывается единственно приемлемым выбором. Правда, практически такие ситуации пока, вроде бы, еще не возникали (и слава Богу), поэтому рискну прибегнуть к несколько утрированной аналогии.

Как известно, принцип расщепления ядра урана~--- общеизвестный научный факт, находящийся, если так можно выразиться, в свободном доступе. Технологическая цепочка создания на его основе ядерной бомбы, также базируется на массе научных принципов, столь же общеизвестных. И что же, руководствуясь логикой GPL, технология создания ядерной бомбы также должна быть открытой и общедоступной?

\section{И обратно о лицензиях: 1. По мотивам RMS} 
\begin{timeline}Март 25, 2009\end{timeline}
Первым поводом для настоящей заметки послужило предостережение Ричарда Столлмена, высказанное им в статье \href{http://www.gnu.org/philosophy/javascript-trap.html}{The Javascript Trap} (см., например, \href{http://citcity.ru/20806/}{соответствующую новость} на CitCity). Суть его в двух словах следующая: Некоторые web-приложения, написанные на свободных языках сценариев (например, на JavaScript), для ускорения загрузки сжимают код таким образом, что он становится не читаемым для пользователя. Таким образом, получается, что на основе свободного софта создаются в сущности проприетарные программы, так как предоставление читаемого исходного кода~--- непременное условие принадлежности к классу FOSS.

В качестве модельного примера приводится Google Docs~--- онлайновый сервис, выполняющий функции офисного пакета, то есть текстового процессора, электронной таблицы и презентационной программы, работающих без установки на локальную машину. Используя набор технологий AJAX, он в конечном счёте основывается на JavaScript (не только на нём, но в данном контексте это не существенно). То есть на свободном языке сценариев, исходники на котором по определению открыты и доступны. То есть сценарий Google Docs можно загрузить на локальную машину и работать с ним в своё удовольствие. Так в чём же проблема?

Оказывается, в том, что Google Docs~--- сценарий на JavaScript, исходный текст которого оптимизирован для максимального ускорения загрузки путём полного отсутствия комментариев, максимальной <<разгрузки>> от пробелов, сокращения имён методов до одного символа. В результате код оказывается не читаемым для пользователя и, следовательно, недоступным для изучения и модификации:


\begin{shadequote}{}Компактный код~--- не исходный код, а исходный код этой программы оказывается недоступным для пользователя.\end{shadequote}

То есть, по мнению Столлмена, пользователю под видом свободного программного обеспечения подсовывается проприетарный софт, о чём он может и не подозревать: хотя в любом браузере имеется возможность отключения исполнения сценариев JavaScript, но функций различения скриптов с <<нормальным>> и <<компактным>> кодом не предусмотрено ни в одном. Так что пурист FOSS может и не подозревать, что он использует классово чуждый ему закрытый софт.

Впрочем, по мнению Столлмена, Google Docs~--- не единственная угроза целомудрию адепта FOSS. Более существенными ему представляются такие технологии, как Flash и Silverlight. Особенно последняя, так как она не только основана на проприетаных (и, следовательно, закрытых) кодеках, но и не предоставляет альтернативных способов их воспроизведения. Тогда как Flash, используя формат, до недавнего времени закрытый (а ныне его можно квалифицировать только как полуоткрытый), всё же допускает использование свободных заменителей фирменному Flash Player.

В качестве методов сохранения девственности FOSS-пользователей предлагаются следующие:

\begin{itemize}
	\item выработка критерия <<отделения зёрен от плевел>>~--- то есть <<кошерного>> кода Javascript от <<трефного>>; 
	\item переделка свободных браузеров таким образом, чтобы, во-первых, сообщать пользователю о степени <<кошерности>> кода web-приложений, а во-вторых, предоставлять ему возможность выбора между кодом <<чистым>> и <<не чистым>>; 
	\item истинно свободный код JavaScript в случае его <<компактизации>> должен внутри себя содержать ссылку на сайт с исходниками и документацией, и сопровождаться специально разработанной на этот случай лицензией. 
\end{itemize}

Казалось бы, как говорил товарищ Саахов,


\begin{shadequote}{}\dotsвсё это конечно так, всё верно, бумага написана правильно, всё хорошо. Это с одной стороны. Но есть и другая сторона медали.\end{shadequote}

Так что с другой стороны, давайте посмотрим, так ли страшен чёрт вместе со всеми его малютками с точки зрения растления непорочных FOSS-душ?

Начнём с того, что с негодованием отметаем растлевающее влияние Flash и Silverlight: каждый пользователь свободного софта, скачивая соответствующие плагины с сайтов Adobe или Microsoft, не может не отдавать себе отчета в том, что он имеет дело с заведомо проприетарными программами. Мифического FOSS-пользователя, невинного до незнания того, что существуют проприетарные программы, можно уподобить Адаму и Еве до их грехопадения.

В случае с Silverlight ситуация более чем ясна: для сохранения своих сердец в чистоте пуристы FOSS должны просто отказаться от его использования. Для Flash же всегда существует альтернатива~--- пользователь, вместо проприетарного Flash Player, может выбрать любой из числа свободных плейеров swf-формата, благо таковые, в виде gnash, klash (возможно, и других) имеются. То есть он ничем не отличается от пользователя, читающего, редактирующего и даже создающего документы в формате Word'а посредством OpenOffice.org.

Более того, наличествуют даже свободные средства для изготовления swf-роликов~--- SWFTools, Spalah и ещё несколько. Не знаю, насколько свободные Flash-ориентированные программы воспроизводят функциональность проприетарных прототипов~--- но чем только ни пожертвуешь ради сохранения непорочности?

Однако вернёмся к главному герою <<Предупреждения Столлмана>>~--- Google Docs с его <<компактизированным>> кодом JavaScript. Не будучи программистом, не могу судить, насколько такая компактизация делает код не читаемым для разработчика. Замечу только, что в принципе разработчику вольно оптимизировать свой код так, как ему видится наилучшим. А уж писать или не писать комментарии~--- это вообще его сугубо личное дело~--- вроде, никакая лицензия (кроме внутренних побуждений) его к этому не обязывает. Что же до того самого пользователя, который постоянно фигурирует в статье Столлмана, то для него в большинстве случаев равно не читаем будет ни реальный, ни компактный код.

Далее, следует подчеркнуть, что в случае с Google Docs пользователь имеет дело не столько с программой самой по себе, сколько с сервисом: пользователю предоставляется не только (и даже не столько) средство для создания своих документов, но и обеспечивается их хранение на сервере, резервирование, доступ по сети, в том числе и защищённый.

Способен ли пользователь, как бы глубоко он ни разобрался в коде, обеспечить аналогичный сервис? Если он располагает такими возможностями, для него не составит труда нанять команду программистов, которая сочинит для него и аналогичные средства его реализации. После чего он может выступить прямым конкурентом Google в мировом масштабе. Кстати, насколько я знаю, такие конкуренты у него имеются и так.

Наконец, пользователь, прибегающий к такого рода сервисам, так или иначе уже отдаётся во власть проприетаризма: ведь никто ещё не спорил, что их обеспечение, такое, как серверы, линии связи и так далее, всегда являются чьей-то частной собственностью.

Так что никакой практической пользы в <<антирастлительных>> мерах, предложенных Столлменом, я не вижу. С задачей различения свободного и проприетарного кода в web-приложениях рассматриваемого типа прекрасно справляются сами пользователи первого; и если они допускают для себя лично использование не вполне свободных плагинов и онлайновых сервисов~--- значит, оно им зачем-нибудь да нужно. Пользователей же второго этот вопрос нимало не волнует: они и так живут в проприетарном мире.

А вот переделка браузеров~--- ИМХО, момент абсолютно отрицательный. Ибо поведёт к их усложнению, которое чревато ошибками. Особенно в таком деле, как распознавание кода, создавая, скажем, лазейки для троянов.

Ну и, наконец, про дополнительную лицензию и говорить нечего: их и так развелось столько, что в них запутались и пользователи, и разработчики.

Вообще, рассмотренная ситуация напоминает мне историю борьбы с предустановкой OEM Windows на ноутбуки : плюсы либо сомнительны, либо копеечны, а вот минусы вполне реальны и в некоторых случаях могут оказаться существенными. Так что в обоих ситуация хорошо бы задуматься~--- а стоит ли овчинка выделки?

\section{И обратно о лицензиях: 2. По мотивам ESR}
\begin{timeline}Март 25, 2009\end{timeline}
\begin{shadequote}[r]{Владимир Высоцкий}
Ведь жизнь таких, как мы, сама накажет строго \\
Тут мы согласны, скажи, Серёга. 
\end{shadequote}


Не успело утихнуть эхо предупреждения Столлмена, как оно сменилось выступлением Эрика Раймонда, сделанным им на собрании LI LUG (Группы пользователей острова Лонг-Айленд). Видеозапись его можно 
\href{http://www.archive.org/details/LILUG_20090310_ESR}{видеть}
 
\href{http://www.archive.org/details/LILUG_20090310_ESR}{здесь}
, а репортаж о заседании, содержащий краткое, с купюрами, содержание собственно выступления~--- 
\href{http://dotcommie.net/feed/index.php?id=160}{прочитать здесь}
  (разумеется, по английски). Суть же этой, без преувеличения, сенсационной, речи, можно передать четырьмя словами:


\begin{shadequote}{}GPL больше не нужна\end{shadequote}

Услышать такое из уст одного из основоположников движения FOSS, казалось бы, равносильно грому с ясного неба. Однако вспомним, что Раймонд никогда не относился к лицензионным вопросам с пуризмом, свойственным Столлмену. Собственно, с этого он и начал своё выступление:


\begin{shadequote}{}Одна из моих еретических идей заключается в том, что мы придаём лицензиям чрезмерное внимание. И в первую очередь, я думаю, что мы не нуждаемся в <<обязывающих>> лицензиях\dots\end{shadequote}

\dotsто есть, продолжу своими словами, в лицензиях обязывающих открывать исходный код любого производного продукта, содержащего GPL-лицензированный код. И на основе которых можно возбудить преследование за нарушение этих правил. Далее он объясняет свою точку зрения, почему такого рода лицензионные дела не нужны. Попробую передать его объяснение так, как его понимаю я.

Опять же кратко суть объяснений Раймонда можно передать словами эпиграфа: жизнь, то есть рынок, сама накажет тех, кто, используя открытый код в проприетарных программах, не желает делиться своими собственными наработками. То есть, попросту говоря, закрытая модель разработки программного обеспечения в конечном счёте менее эффективна, нежели открытая, и потому выполненные в её рамках продукты рано или поздно проиграют в рыночной конкуренции.

Несколько неожиданно, не правда ли? Ведь расхожее мнение гласит, что, например, лицензии BSD-стиля благоприятствуют возникновению закрытых проприетарных программ, в той или иной мере использующих код FOSS без <<взаимоотдачи>> сообществу. То есть, говоря по простому, даёт возможность кому-либо <<наложить лапу>> на изначальный свободный продукт. Правда, ИМХО, мнение это основано на недоразумении. Некогда я уже \href{http://alv.me/?p=279}{писал}, как в реальности может выглядеть это самое <<наложение лапы>>: открытые компоненты законченного программного решения всё равно останутся открытыми, и любому вольно получить к ним доступ и развивать далее.

Однако вернёмся к обоснованию Раймонда. Собственно, в разных формах эту мысль он высказывал неоднократно: закрывая разработки, основанные на открытом коде, производитель лишается поддержки сообщества, и может рассчитывать только на своих сотрудников~--- тех, кого он в состоянии нанять. Число их может быть больше или меньше, но оно всё равно ограничено. Тогда как число разработчиков открытого софта и его пользователей, обеспечивающих обратную связь, потенциально можно считать безграничным. И анализ динамики последних лет показывает, что чисто количественно чаша весов всё больше смещается в сторону последних. Разработчики открытых продуктов могут выпускать их версии более часто, за счёт обратной связи с пользователями эффективней исправлять ошибки, и так далее. В результате производители проприетарных продуктов, базирующихся на открытых исходниках, оказываются в ловушке собственной жадности.

Возникает резонный вопрос: если порок жадности будет наказан рынком, а добродетель поделиться с сообществом~--- им же вознаграждена, то к чему лицензии типа GPL, которые в сущности призваны делать то же самое? Только, скорее всего, менее эффективно.


\begin{shadequote}{}Именно поэтому я думаю, что мы больше не нуждаемся в GPL или подобных ей <<обязывающих>> лицензиях\end{shadequote}

-- примерно такими словами Раймонд подводит итог этой части своего выступления.

В ходе его продолжения Раймонд высказывает мысль, что в качестве альтернативы GPL можно рассматривать BSD-лицензию. И добавил, что стоимость разработки открытых и частично зарытых решений зависит от размера компании: маленькие компании просто не могут позволить себе штат разработчиков для поддержки значительных объемов кодовой базы, хотя это вполне по силам компаниям большим.

И уж совсем в заключение Раймонд отметил, что, хотя FOSS ныне процветает, это не значит, будто GPL пора ликвидировать:


\begin{shadequote}{}Я не думаю, что время, когда ликвидация GPL будет более полезна, чем вредна, настанет и в будущем.\end{shadequote}

Что можно добавить к его словам? Пожалуй, что ничего. Только попросить прощения у читателей за то, что сенсация, обещанная в начале заметки, в конце её так и не состоялась: Раймонд отнюдь не призвал к отмене <<Хованщины>> с оперных подмостков и поголовному переходу на сою. А лишь объяснил, почему, с его точки зрения, без первой можно обойтись, и почему вторая также вполне пригодна к употреблению.

\section{Кто оплачивает банкет?} 
\begin{timeline}15 июня 2005 г\end{timeline}

\hfill \begin{minipage}[h]{0.45\textwidth}
Мой снобизм~--- он как лучик путеводный,\\
Помогает воспринять судьбу как должно:\\
Мол, художник~--- он обязан быть голодным. \\
Он худой, но гордый, он~--- художник.
\begin{flushright}
\textit{Тимур Шаов}
\end{flushright}
\bigskip\end{minipage}

А кто вообще должен (и должен ли?) финансировать такие родственные области человеческой деятельности, как разработка Open Source и фундаментальную науку? И кто их финансирует в настоящее время?

Должно, однако, для начала заметить, что родственность Open Source и фундаментальной науки~--- не мое изобретение. Блестящее обоснование этого тезиса содержится в статье Николая Безрукова Разработка программ с открытыми исходниками как особый вид научных исследований.

У читателя большей части популярных публикаций на тему Open Source вполне может создастся впечатление, что разработка программ этого класса осуществляется на голом энтузиазме, напоминающем ударников-стахановцев и прочих строителей Магнитки. Что, конечно, очень бла-а-родно, но в наш коммерциализованный век вызывает законное недоверие (особенно с учетом аналогии строителей Магнитки и БАМа~--- забайкальских комсомольцев, сокращенно ЗК).

Проблема финансирования Open Source включает в себя три аспекта:


\begin{enumerate}
	\item кто может быть заинтересован (и может ли?) в финансировании разработки Open Source;
	\item в какой форме заинтересованные стороны могут осуществлять финансирование разработки Open Source;
	\item каким образом финансирование это может распределяться среди разработчиков.
\end{enumerate}


Должен сразу признаться~--- ответов на эти вопросы я не знаю. Однако они~--- совершенно те же, что возникают при финансировании т.~н.~фундаментальной науки. А поскольку в этой сфере мировым научным сообществом накоплен весьма большой, хотя и не всегда положительный, опыт, по аналогии рискну высказать некоторые соображения.

Но сначала зададимся иным вопросом: а нужно ли финансирование разработки программ с открытыми исходниками вообще? Ведь теоретически предполагается, что это~--- предприятие самоокупаемое, а то и просто прибыльное. Однако теоретически, как говаривал дед Щукарь, она лошадь, а практически она падает\dots

Напомню, что открытость исходников некоей программы и даже свобода её распространения отнюдь не подразумевает бесплатности: вспомним бессмертное столлменовское <<свободное слово, а не бесплатное пиво>>. Тем не менее практика такова, что подавляющее большинство открытых и свободных ОС вкупе со своими приложениями распространяется фактически бесплатно. То есть пресловутая интеллектуальная собственность источником дохода служить не может (как это имеет место быть в случае проприетарного софта).

Прибыльность разработки Open Source обычно обосновывается туманными фразами о сопровождении и поддержке программных продуктов, распространяемых бесплатно. В смысле~--- по себестоимости носителей и документации. Когда в отрицание бесплатности открытого софта приводят американские цены на дистрибутивы типа Red Hat или Suse, забывают, что 100- (или даже 200-) долларовые их коробки содержат штук пять-шесть изрядной толщины книжек. А в Америке любая специальная книжка меньше полтинника ихних денег, насколько я знаю, сама по себе не стоит. Если же речь идет о дистрибутивах ценой в несколько тысяч долларов~--- то тут уже в стоимость включено именно сопровождение продукта в явном виде.

Тем не менее, достоверных известий о фирмах или персонах, обогатившихся за счет техподдержки Open Source, у меня нет. Что понятно: люди, покупающие дистрибутив Linux (для примера), имеют целью разобраться в системе. И либо этой цели достигают, и тогда в техподдержке не нуждаются. Либо бросают это занятие, и тогда не нуждаются в поддержке тем более. Если же речь идет о корпоративных пользователях~--- не думаю, что разумный руководитель рискнет перевести весь свой документооборот на Linux или BSD, не имея в штате квалифицированного специалиста по одной из этих систем. Так что мечты о заработке на техподдержке~--- столь же наивны, как и мечты о самоокупаемости фундаментальной науки, блестяще-утопично сформулированные на заре перестройки Максимом Максимовым в статье <<Реанимация>> (Знание-сила, 1989, \textnumero11).


\textit{Маленькое отступление. Впрочем, Максимова прославила другая работа~--- <<На грани~--- и за ней>> в \textnumero3 того же журнала за 1988 г. Именно из нее большинство из нас узнало о Бруно Беттельгейме~--- чуть ли не единственном из великих неофрейдистов XX века, чьи труды остаются не переведенными на русский язык по сей день. Кто читал статью Максимова~--- легко догадаются, почему\dots}


Однако вернемся к теме раздела. Конечно, само по себе издание и распространение дистрибутивов прибыль приносить может. Так же, как её приносит, скажем, издание детективов Марининой и их продажа. Однако к аналогии с книжным бизнесом я обращусь несколько позднее.

А пока: кто заинтересован в развитии Open Source и, соответственно, мог бы финансировать его разработку? Кроме его разработчиков и пользователей, разумеется. Но ведь разработчики Open Source, хотя и заняты своим делом в большой степени из любви к искусству, также хотят есть-пить. А пользователи~--- они ведь пользуют Open Source в значительной мере в виду его бесплатности, практической или теоретической. Что удовлетворению означенной потребности разработчиков отнюдь не способствует\dots

Первая сторона, заинтересованная в развитии Open Source~--- пользователи любого коммерческого софта. Впрочем, не смотря на внешнюю парадоксальность, это очевидно: только конкуренция со стороны открытых и свободных программ может подвигнуть коммерческих разработчиков на совершенствование своей продукции.

Как ни странно, в качестве второй стороны, наиболее заинтересованной в развитии открытого софта, видятся производители коммерческих UNIX-систем (хотя, подозреваю, что сами они не всегда понимают свое счастье). Почему~--- обосновать не трудно.

Молодому человеку, сызмальства привыкшему к Linux (FreeBSD, OpenBSD \textit{etc}.) и пришедшему на службу со школьной (или университетской) скамьи, работать в Windows~--- что серпом по\dots сами знаете чему. Гораздо легче ему будет перейти на Solaris или AIX. А достигнув по службе должного положения, он, безусловно, приложит все усилия к тому, чтобы внедрить POSIX-системы в родном трудовом коллективе~--- ведь, по тем или иным причинам, далеко не всегда можно обойтись свободными их реализациями.

За иллюстрацией этого достаточно спуститься в не столь уж далекое прошлое~--- 1995 год. Когда (не заставшим того времени поверить в это трудно~--- но факт имел место быть) в качестве реальной альтернативы для массовых настольных систем рассматривались OS/2 и Windows~95. Причем ни у кого не вызывало сомнений технологическое превосходство первой. Но: Windows~95 пришла в дома, а OS/2~--- нет (о причинах распространяться здесь неуместно). И через несколько лет на работу вышло поколение, вскормленное и вспоенное в <<окнах>>. Результат не замедлил воспоследовать: кто и когда последний раз видел OS/2 на пользовательском десктопе?

Наконец, третья заинтересованная сторона~--- это государство, причем~--- любое (правда, применительно к нашему государству об этом и не подумаешь). Причины этого также достаточно тривиальны~--- здесь и баланс между монополизацией и свободной конкуренцией, и массовый независимый аудит программной продукции, и снижение себестоимости рабочего места госчиновника: список легко продолжить. И не на последнем месте в нем окажутся соображения государственной безопасности: как бы <<подоконная>> кольчужка коротка не оказалась\dots

Теперь посмотрим, горят ли заинтересованные стороны желанием оказать развитию Open Source посильную финансовую помощь?

Конечно, странно было бы ожидать от массы Windows-пользователей благотворительных пожертвований в FSF и аналогичные dot-org'и (а для последних это нередко~--- существенный источник финансирования, примером тому проект OpenBSD). Тем не менее, наиболее продвинутая часть их косвенно в финансировании Open Source участвует. Хотя бы тем, что проявляет интерес к публикациям на эту тему~--- и бумажным, и сетевым. В результате чего компьютерные издания печатают больше статей соответствующего профиля, регулярно выплачивая авторам гонорары, чем и способствует материальному благополучию писателей-POSIX'ивистов (навроде вашего покорного слуги), а также продолжению их скорьного пропагандистского труда.

Теперь о корпорациях~--- производителях UNIX-машин и разработчиках проприетарных версий UNIX и софта для них. Об инвестициях IBM в Linux-компании знают, наверное, все. Однако это лишь одна сторона дела. Меньшее внимание привлекает то, что, скажем, изрядное количество разработчиков свободного браузера Mozilla (из некоторых источников явствует, что~--- большинство) по совместительству являются штатными сотрудниками AOL (хотя скорее~--- наоборот) и работают над проприетарным (хотя и бесплатным, но не свободным) браузером Netscape. А команды разработчиков свободного офисного пакета OpenOffice и проприетарного~--- StarOffice, суть множества пересекающиеся (не знаю уж, насколько точно, но очень значительно). Ну и недавнее приобретение Suse Linux компанией Novell (до некоторого времени правообладателя торговой марки UNIX)~--- также факт показательный.

И, наконец, государство. О прямом госбюджетном финансировании разработки свободного софта в большинстве стран, как будто бы, слышно не очень много. Кроме, разве что, Китая, где курс на внедрение открытого и свободного софта~--- прямо-таки генеральная линия Партии и Правительства. Проскальзывает информация о господдержке Linux (в сфере наробраза, например) в паре-тройке стран Латинской Америки (Мексика, Бразилия). Ну и есть сведения, что французская компания Mandrakesoft (производитель одноименного дистрибутива Linux, ныне именуемого Mandriva) была некогда выведена из затянувшегося кризиса благодаря правительственным ассигнованиям.

Это~--- с одной стороны, Однако~--- вспомним, что BSD (предок свободных операционок Free-, Net- и OpenBSD) почти полтора десятка лет разрабатывалась в Университете Беркли. На денежки, между прочим, оборонного ведомства США~--- то есть прямого госбюджета.

Или~--- история финского студента Линуса Торвальдса. Который благополучно проучился (а потом и проработал) в университете лет семь, за которые свой Linux и разработал. А ведь высшее образование в Финляндии, насколько мне известно, бесплатное~--- сиречь финансируемое из госбюджета. Так что можно сказать, что Linux был создан за счет финской казны (и финских же налогоплательщиков; также как FreeBSD~--- за счет налогоплательщиков американских).

Все это я говорю к тому, что романтическое представление о создателях свободного софта как об энтузиастах-одиночках (со всеми вытекающими из этого следствиями, как положительными, так и отрицательными) подчас весьма далеки от истины. Многие из них~--- и сотрудники коммерческих фирм, и выходцы из академическо-университетской среды,~--- занимаются этим делом в рамках своих прямых должностных обязанностей, за что и получают зарплату.

Значит ли это, что движение Open Source имеет достаточную финансовую базу? Отнюдь. Государственные программы имеют обыкновение рано или поздно заканчиваться по самым разным причинам. Так, не с падением ли мировой системы социализма связано прекращение финансирования проекта BSD в 1991 году?. Вливания от коммерческих фирм зависят от конъюнктуры рынка. А интерес широких народных масс ко всему, связанному с Linux и Open Source, в значительной мере лишь дань моде.

Так что какие-либо дополнительные источники финансирования не помешали бы движению Open Source. Однако уже упоминавшаяся аналогия между ним и фундаментальной наукой наводит меня на мысль: а пойдут ли они ему на пользу? Ибо второй вопрос после получения финансирования~--- как это финансирование будет распределяться среди собственно разработчиков?

Опять же по аналогии с наукой можно предположить три модели распределения финансов. Первая, наиболее эффективная, модель описана в легенде про Папу Римского (не помню уже кого и которого) и знаменитого художника Титиана. Коему Папа просто подарил дворец вместе с ежегодной рентой, необходимой на его содержание. Дабы у мужика голова по пустякам не болела\dots

Модель эта не столь утопична, как может показаться: по доброй традиции, основанной товарищем Сталиным, примерно по такой схеме осуществлялось финансирование науки в Советском Союзе. Правда, у товарища Сталина была некоторая страховка на случай, если товарищ Титиан начнет манкировать своими обязанностями: всегда можно немножечко расстрелять товарища Титиана. Когда такой возможности не стало, советская наука очень быстро пришла к тому, к чему пришла\dots

К слову сказать~--- модель эта минимум однажды эффективно сработала и в мире открытых исходников. Не напоминает ли вам положение Линуса в компании Transmeta то, в котором оказался Титиан Папскою милостью, или товарищ Курчатов~--- волею гения всех времен и народов?

Вторая модель~--- финансирование проектов в форме грантов, как это принято в отношении научных исследований в т.~н.~цивилизованных странах. А снекоторых пор и Россия приобщилась к этой практике. Так что о достоинствах и недостатках такой схемы распространяться не буду~--- наши <<дети капитана Гранта>> (они же~--- <<Джорджа Сороса птенцы>>) испытали их на собственной шкуре.

Хотя и для этой модели есть примеры удачного использования в истории движения Free Software~--- разработка систем линии BSD 4.X в Университете Беркли. Правда, термина Free Software тогда еще не существовало.

Наконец, третья модель~--- гонорарная, аналогичная принятой в книгоиздательской практике. Я уже говорил, что собственно прибыль при распространении открытого софта могут извлекать издатели и продавцы дистрибутивов. Причем прибыль эта в значительной мере определяется сопровождающей печатной продукцией (документацией). Так почему бы им, в соответствии с Божьей заповедью, не поделиться её толикой с теми, кто обеспечивает им предмет издания и продажи?

История показала, что эта модель, пожалуй, наиболее эффективна. Она прекрасно сработала в случае FreeBSD и Linux Slackware. И тот, и другой проект долгое время развивались при поддержки компании Walnut Creek~--- известного продавца компакт-дисков (а потом и дистрибутивов). Доход от продажи дистрибутивов (плюс разнообразной атрибутики~--- маечек, кружечек \textit{etc}.), насколько мне известно,~--- один из основных источников финансирования проекта OpenBSD. Ну и для компаний, собирающих дистрибутивы Linux (типа Red Hat и Suse), это~--- также некоторое подспорье.

Но~--- не более: ибо мир информационных технологий породил и еще одну модель финансирования собственной деятельности~--- ту самую пресловутую техническую поддержку и поставку готовых решений, о которых я уже говорил ранее. И к которой по ряду причин отношусь несколько скептически. Однако недавний отказ Red Hat от официальной поддержки собственного пользовательского дистрибутива показывает, что заинтересованные стороны моего скептицизма не разделяют. Что ж, Бог им в помощь\dots

Впрочем, в нашей стране затронутые здесь вопросы носят чисто теоретический характер. И ныне разработчикам Open Source (как и научным работникам) следует находить утешение в строках Тимура Шаова, приведенных в качестве эпиграфа этого раздела.

\section{Откуда и куда пошел свободный софт} 
\begin{timeline}2004 г\end{timeline}

Эта заметка навеяна чтением статьи Владимира Попова <<Размышления на тему: заработать на Open Source>>. И потому начну её с цитаты из оной:

\begin{shadequote}{}Критерий рентабельности заведомо не приложим к научной, гуманитарной, медицинской и многим другим сферам деятельности человека (не говоря уж о сфере искусства).\end{shadequote}

Возникает вопрос~--- а приложим ли этот критерий собственно к открытому программному обеспечению? Что особенно актуально в свете все более частых попыток <<Заработать на Open Source>>. Конечно, само слово <<заработать>> в этом контексте имеет двойной смысл~--- но к этому я еще вернусь. А пока все же попробую ответит на свой вопрос. Для чего, как обычно, придется обратиться к истории.

Конечно, понятия Unix-way и Open Source~--- далеко не полностью пересекающиеся множества. Но, тем не менее, на протяжении всей своей истории они были тесно, можно сказать~--- неразрывно,~--- связанными. И потому рассмотрим их одним массивом.

С чего началась дорога, получившая потом название Unix-way? Как всем известно, началась она с проекта Multics. Что это было такое? Это была совместная разработка Bell Labs, General Electric и Массачуссетского технологического института (MIT). То есть~--- вполне академический проект. Где работали основоположники Unix? Работали они в той же Bell Labs, хотя и принадлежащей коммерческой компании, но имевшей целью все же научные разработки. А кто сказал, что научная работа не может финансироваться частным капиталом? Только не те, кому посчастливилось получить Нобелевскую премию.

Дальше~--- больше. Чей вклад в становление Unix в современном его виде был наиболее весом (после его основоположников, конечно же)? Университета Беркли, штат Калифорния. Кем был Ричард Столлмен до того, как он начал свой крестовый поход за освобождение гроба Господня (то есть, пардон, за свободу софта)? Был он, как известно, научным сотрудников в лаборатории искусственного интеллекта в том же MIT. Чем занимался Энди Танненбаум, создатель Minix~--- системы, вдохновившей Линуса Торвальдса на его бренный труд по написанию терминальной программки? А занимался Энди преподаванием а Амстердамском университете. Да и Minix свой написал он, собственно говоря, для того, чтобы обучать скубентов основам Unix.

Наконец, кем был сам Линус Торвальдс в то время, когда его терминальная программа медленно, но верно превращалась в операционную систему? Был он студентом, а потом научным сотрудником университета в Хельсинки. И число таких примеров можно умножить до бесконечности.

Из всего сказанного можно сделать вывод: создание Unix как открытой (в смысле~--- соответствующей открытым стандартам) системы и её производных, представленных открытым и свободным софтом во всех его проявлениях, происходило в значительной мере в сугубо академической среде. А если вспомнить о том, что вся софтверная индустрия базируется, в сущности, на математических алгоритмах, развивавшихся в рамках <<чистой>> математики со времен Евклида и Бируни, становится окончательно ясно: Unix-way и Open Source есть порождение той области человеческой деятельности, которую именуют <<чистой>> или фундаментальной наукой. Впрочем, задолго до меня к тому же выводу пришел Николай Безруков в статье <<Разработка программ с открытыми исходниками как особый вид научных исследований>>.

Таким образом, поставленный ранее вопрос сводится к более общему: а можно ли заработать на занятиях фундаментальной наукой? И тут впору вспомнить о двух смыслах русского слова заработать. Первый~--- это получать некоторую плату (в Кодексе законов о труде при соввласти она так и называлась~--- заработная, хотя ниже мы увидим, что это определение не вполне точно) за свою работу. И второй~--- это извлечение прибыли, то есть предпринимательская деятельность.

Если в наш вопрос подставить первый смысл слова заработать, то ответ на него будет сугубо положительным. В обоснование чего можно привести не только немецких профессоров XIX века (весьма обеспеченных по тем меркам людей), но и судьбу Линуса Торвальдса: ведь в сущности на протяжении нескольких лет он получал свою зарплату в университете именно за то, что разрабатывал Linux.

За чей счет выплачивается зарплата научного работника? Я уже затрагивал этот вопрос в своем сочинении <<Кто оплачивает банкет?>> и потому повторю лишь вкратце: научная работа вообще оплачивается обществом в целом. А уж посредством чего и кого эта оплата осуществляется~--- вопрос другой.

Посредником в оплате научной работы может быть государство в целом. Так, можно сказать, что создание Linux'а финансировалось финским налогоплательщиком посредством правительства своей страны, в которой существует бесплатное образование.

Это могут быть отдельные государственные ведомства. Так, агентство DARPA было не более чем передаточным механизмом между государственным бюджетом США (складывающимся из денег тех же налогоплательщиков~--- но теперь уже американских) и Университетом Беркли.

Это могут быть корпорации~--- ведь компании типа AT\&T или IBM суть не что иное, чем социальные объединения, превосходящие по масштабам несредние даже государства.

Наконец, общество (или отдельные, наиболее прогрессивные, его представители) могут выступать финансистами научных работ и непосредственно~--- в виде пожертвований. Не секрет ведь, что это существенный источник средст к существованию многих проектов Open Source.

За века существования науки как важной сферы человеческой деятельности (и за многие уже десятилетия~--- как массовой профессии) было выработано три формы распределения финансовых источников среди заинтересованных лиц. Это~--- прямая выплата так называемой зарплаты (так называемой~--- потому что суть явлеления лучше передается словами жалование или оклад содержания), финансирование в форме безвозмездных кредитов~--- грантов, и гонорарная оплата результата. Каждая из этих форм имеет свои достоинства и недостатки, обсуждать которые здесь не место (об этом опять же~--- в \href{http://alv.me/?p=294}{сочинении о банкете}). Однако никаких иных механизмов пока не придумано и в мире свободного софта: Линус получал оклад содержания в своем университете, создатели BSD жили за счет грантов мирного американского ведомства, а средства к существованию Столлмена вполне подпадают под понятие гонораров.

Так что зарабатывать на кусок хлеба своей работой на ниве Open Source вполне можно (правда, боюсь, не в нашей стране). А вот может ли быть открытый софт объектом предпринимательской деятельности? То есть~--- служить для извлечения прибыли. Здесь однозначный ответ дать трудно.

По аналогии с научной работой можно видеть, что сама по себе разработка открытого софта никакой прибыли принести не в состоянии~--- что-то я не слышал об акциях компаний по разработке специальной теории относительности или квантовой механики. Прибыль начинается на стадии так называемого внедрения научных разработок в жизнь, то есть когда наука перестает быть чистой наукой и становится скорее технологией. В софтверной индустрии с таким внердением можно сопоставить а) тиражирование, б) обучение и в) сопровождение программной продукции. Как тут обстоит дело с прибыльностью?

Очевидно, что норма прибыли при тиражировании свободных программ стремится к нулю. Ведь себестоимость носителей практически нулевая, сами программы~--- общедоступны, и наварить здесь практически невозможно. Вспомним пример из книги Линуса~--- когда стоимость подвоза воды превышает допустимый пользователями уровень, кто-нибудь обязательно протянет водопровод. Так что единственное, что тут остается~--- это пытаться компенсировать норму прибыли её массой. То есть~--- объемом продаж. А объем этот напрямую связан с количеством пользователей Open Source. Причем~--- в первую очередь пользователей-индивидуалов~--- в силу специфики свободных лицензий компания уже в 100 человек вполне может наладить тиражирование своими силами и в потребных масштабах. Но пользователей-индивидуалов за десктопами~--- отнюдь не большинство, и вовсе не все они~--- поклонники свободных разработок. Так что и места под солнцем для Linux-предпринимателей оказывается не так много.

И тут тиражирование тесно смыкается с обучением~--- поскольку существовать без него не может. А под обучением я понимаю не только (и даже не столько) всякого рода сертифицированные курсы, сколько~--- самую элементарную печатную документацию. Ведь каждому, кто видел в книжном магазине США или Европы научные монографии, становится ясным: цена коробочных дистрибутивов Red Hat или Suse на 90\% состоит из стоимости сопутствующей полиграфии. То есть компании-распространители дистрибутивов~--- это не столько разработчики программного обеспечения, сколько~--- издатели книг и сопутствующей продукции (вспомним Ходжу Насреддина~--- <<дом, сад и принадлежащий им водоем>>). Так что структура их доходов в этой части оказывается такой же, как у любого книжного издательства.

Книжные издательства в мире существуют уже веками, и даже те из них, что занимаются издательством исключительно научной литературы (или по преимуществу ее) особо не бедствуют~--- вспомним Эльсивьер. Однако и здесь для Linux-компаний (назовем их так) не так уж все зд\'{о}рово. Ибо обычные книгоиздатели занимаются тиражированием уникальных, то есть безальтернативных, авторских произведений (будь то научные монографии или детективные романы)~--- или по крайней мере тех, которые воспринимаются обществом в этом качестве:-).

Сопровождающая же софт книжная продукция представляет собой лишь один из многих вариантов получения информации об оном софте~--- тут и штатная документация системы (типа man-страниц), и разного рода онлайновые источники. И с развитием Интернета роль печатной документации все более снижается. Так что единственной побудительной причиной к приобретению коробки с руководством может быть только литературный талант написавшего её автора:-)~--- необходимую сумму заний пользователь легко может получить другими путями. Так что особенно раскатывать губы на тиражирование+обучение также не стоит.

Наконец, поддержка. Традиционные формы поддержки можно разбить на две формы: привычное многим эникейство~--- индивидуальное или в масштабе (рабочей) группы товарищей,~--- и поддержка корпоративная. Первая форма, блестяще зарекомендовавшая себя в мире Windows (и обеспечивающая кусок хлеба со стаканом пива не одному уже поколению эникейщиков), в мире Open Source явно не сработает. Ибо начинающий пользователь-индивидуал, скажем, Linux или а) очень быстро перестанет быть начинающим, и в услугах эникейщика нуждаться не будет, или б) бросит это занятие~--- и тогда ему эникейщик не потребуется тем более, или в) получит в свое распоряжение идеально настроенную под его задачи систему, в которой можно будет, ничего не меняя, работать веками~--- и больше обращаться к эникейщику повода у него не будет (разве что по старой дружбе пивком угостить).

Корпоративная поддержка\dots Да, это то, на чем, в основном, делают деньги и Red Hat, и Suse, пытаются~--- IBM и Hewlett-Packard, возможно~--- кто-то еще. Однако уже ограниченность списка~--- списка по настоящему удачных компаний в этой сфере,~--- свидетельствует о том, что ниша эта не столь уж обширна, как кажется. Предвижу возражение~--- с распространением Linux'а в широких массах простых американских (и прочих) миллиардеров ниша эта будет расширяться. Отнюдь~--- возражу я. Потому что с распространением Linux будет расти и количество специалистов в оном (хотя, как показывает практика.~--- не обязательно их качество), и так назваемую поддержку вполне можно будет осуществить в рамках локального предприятия. А в условиях российской дешивизны рабочей силы фактор этот будет особенно весом.

Это я не к тому, что поддержка Linux~--- дело бесперспективное. Хочу лишь подчеркнуть, что сфера эта столь же ограничена, как и область тиражирования/обучения. А с распространением Linux к тому же будет все менее рентальбельной.

Все ли так мрачно в области коммерческого использования Open Source? Не совсем, потому что есть еще и четвертый способ извлечения прибыли из оного, причем~--- находящийся на грани с зарабатыванием денег в первом смысле этого слова. Однако о нем я планирую поговорить в следующей заметке.

А пока приходится констатировать (с сожалением или нет~--- другой вопрос), что все прошлое, настоящее и, в значительной мере, будущее Open Source связано с академической средой (финансируемой из государственных или частных средств~--- в данном случае не важно) и исследовательскими отделами корпораций~--- тех из них, которые могут позволить себе содержание оных. Иными словами~--- за счет все той же общественной поддержки, непосредственной или опосредованной.

А будущее Linux-компаний в их современной форме видится весьма проблематичным~--- по крайней мере, для новых участников этого бизнеса широкой сферы приложения сил (и, соответственно, извлечения прибыли) не просматривается.

И большинство участников Linux-бизнеса хорошо это понимает. Если внимательно пройтись по Distrowatch 'у, легко увидеть, что в каждой стране существует один, максимум~--- два дистрибутива с <<национальным орнаментом>> и претензиями на как бы коммерческий статус. А многие страны и вообще обходятся дистрибутивами интерациональными.

Исключением, как обычно, оказывается Россия, где собственный дистрибутив не собирает только ленивый. Однако и тут~--- большинство из нас занимается этим главным образом в индивидуальном порядке, из любви к искусству, в целях общего образования, и по достижении достаточного уровня коего существенно охладевает к этому занятию. А сфера традиционной Linux-дистрибуции поделена между командами Altlinux и ASPLinux. Каждая из которых имеет собственную субнишу, целевую аудиторию и сложившийся круг пользователей.

Из всего сказанного, казалось бы, должна следовать исключительная жесткость конкуренции между Linux-компаниями~--- особенно на внутринациональных рынках. Однако это не так: в среде базирующегося на Open Source бизнеса не столкнешься (за буквально единичными исключениями:-)) с таким обливанием грязью конкурентов, какое, по словам видевших это людей, практикуют американские производители стиральных порошков. Почему?

Думаю, что руководители Linux-компаний прекрасно понимают, что само их существование обусловлено не столько <<выхватыванием куска изо рта конкурента>>, сколько~--- той самой общественной поддержкой, о которой я только что говорил. И всякого рода скандальные PR-акции приведут только к дискредитации идеи свободного софта. И, как следствие,~--- к потере этой общественной поддержки.

\section{Как же заработать на Open Sources?} 
\begin{timeline}15 июня 2005 г\end{timeline}

Выполняю свое обещание~--- несколько сгладить мрачно апокалиптическую картину, нарисованную мною в прошлой заметке, относительно перспектив коммерческого использования свободного софта. Но для этого нам придется вернуться к тому, кому и зачем нужен Linux (буду говорить так для краткости, но на самом деле все сказанное относится и к FreeBSD, а частично~--- и к другим BSD-системам).

Однажды на одном из форумов я затеял опрос~--- для чего пользователи переходят на Linux и прочие свободные ОС POSIX-семейства. И, как и ожидалось, смысл большей части ответов можно резюмировать так: чтобы получить надежную и устойчивую систему, идеально заточенную под конкретные задачи данного пользователя.

Другое дело~--- что задачи у всех бывают разные. Большинство обращается к Linux сотоварищи или для разработки софта, или для администрирования. Или~--- для того, чтобы в домашних условиях учиться тому или другому занятию. Некоторые авторы полагают, что только для этих целей свободные POSIX-системы и пригодны. Но при этом забывают еще об одной категории, самой многочисленной~--- так называемых простых пользователях. А у них задачи~--- еще более разнообразны. Для кого-то компьютер~--- это гейм-станция, для иного~--- музыкальный центр. А некоторые, как это ни странно, выполняют на компьютере свою непосредственную работу. Обычно никак с компьютерами не связанную. И если перспективы Linux в области игр или мультимедиа не вполне ясны, то как рабочий инструмент для очень многих и многих он оказывается оптимальным.

Будучи, наверное, одним из первых в истории Руси простых (то есть профессионально не связанных ни с разработкой, ни с администрированием) пользователей, целиком перешедших на Open Sources, не могу отказать себе в удовольствии поделиться собственными впечатлениями. Так вот, для меня Linux, а потом и FreeBSD, оказались идеальной средой для работы с текстами~--- не исходными, а обычными, написанными по преимуществу на русском языке. Средой для сочинения текстов новых, их оформления и распространения. А главное~--- для обработки и использования сочиненного ранее~--- а за свою жизнь насочинял я немало.

Не гордыни ради, а токмо представления масштабов для: около ста статей, полторы монографии, отчетов и проектов разного рода~--- несть числа, и это~--- только по геологии; а уж околокомпьютерных сочинений сколько~--- уже и не упомню. Так вот, лишь в средствах шелла и сопутствующих утилитах нашел я методы эффективной борьбы со всем этим изобилием~--- нахождения нужных текстов, извлечения из них фрагментов, компиляции или, напротив, расчленения, и так далее.

И такие задачи возникают перед многими. В годы, предшествовавшие приобщению к POSIX'ивизму (и превращению в чукчу-писателя), одним из источников средств к существованию для меня было эникейство в индивидуальном порядке. И практически все мои клиенты приобретали компьютеры с одной целью~--- выполнять дома ту работу, ради которой ранее им приходилось ходить на службу (а возможность слушать музыку или там в игрушки поиграть~--- рассматривалась в качестве необязательной опции).

А были среди них люди самых разных профессий~--- научные работники и переводчики, редакторы и бухгалтера, даже одна поэтесса и одна певица, сочинявшая собственные аранжировки (или как это называется~--- именно в этом случае я, увы, оказался совершенно некомпетентен). И, конечно, требования к компьютеру (вернее, наличному на нем софту) у них были разные. Но задача обработки текстов (а для финансовой сферы~--- еще и цифр) стояла перед всеми.

Так вот, практически для всех для них (кроме певицы, вероятно) Linux был бы ничуть не худшим решением, нежели DOS/Windows, а для некоторых~--- просто лучшим (особенно если бы за софт пришлось платить всамделишние деньги). При одном условии~--- если бы кто-нибудь его им поставил и заточил адекватно задачам~--- ведь задачи же, как мы уже выяснили, были разными.

И тут впору вспомнить, что любой дистрибутив Linux или BSD-системы представляет собой лишь полуфабрикат готовой индивидуальной системы для практического использования. И любой из них может быть превращен в закаленное и отточенное орудие для выполнения данной задачи, имеющее все необходимые функции~--- и ни одной лишней. Другое дело, что с FreeBSD на это затрачивается часик-другой, с Source Based Linux (типа Gentoo или Archlinux) придется повозиться сутки. А с пакетными дистрибутивами результата не будет, пока не оставишь от них камня на камне (или~--- rpm'а на rpm'е:-)).

Реально ли это? Мой личный опыт (уже не эникейщика, а POSIX'ивиста) показывает~--- более чем. Может ли это выполнить каждый отдельно взятый пользователь (не имеющий, напомню, специального образования и специфических навыков) на отдельно стоящем персональном компьютере? Теоретически~--- ну конечно же, может. Для этого только что и нужно, как прочитать несколько толстых книжек по Linux и, особенно, UNIX (скажем, <<Секреты UNIX>> Джеймса Армстронга, объемом в какие-нибудь тысячу с небольшим страниц), пару сотенок мануальников и HOW-TO'ёв (с побочной пользой~--- практикой в английском:-)), научиться сочинять простенькие шелл-скрипты и макросы для текстового редактора, ну и освоить еще несколько мелких дел.

Правда, возникает вопрос~--- а когда этот пользователь будет заниматься своей непосредственной работой? Ну, это~--- его личное дело (>>Меньше спите. Или больше работайте>>~--- как сказал персонаж из <<Территории>> Олега Куваева). Главная загвоздка тут в другом: в один далеко не прекрасный день такой пользователь с ужасом обнаруживает, что копаться в конфигах или разбираться с опциями компилятора для него стало интереснее, чем переводить контракты, подводить квартальные балансы или даже вычислять P/T-условия выплавления базальтовых магм.

И в этот миг на свете станет еще одним переводчиком, бухгалтером, геологом меньше~--- и одним POSIX'ивистом больше. Что, конечно, прекрасно~--- да вот только если так пойдет дело~--- кто же будет хотя бы начислять им зарплату? Да и за что~--- ведь вся их деятельность превратится в процесс обеспечения самих себя новыми задачами по настройке и совершенствованию собственной системы\dots

И вот тут наступает Час POSIX'ивиста. Именно его может позвать наш бухгалтер/переводчик/геолог для оборудования своего рабочего места. Которое будет включать в себя не просто установку системы, а полный комплекс по её обработке (рашпилем там, или алмазным надфилем~--- это уже зависит от задач и субстрата, сиречь исходного дистрибутива,~--- но к этому мы еще вернемся). Причем в той степени, какая потребуется, чтобы избавить пользователя от необходимости приобретать хоть какие-то знания о внутреннем устройстве системы.

Возникает вопрос~--- а можно ли это выполнить в рамках UNIX-подобной системы? Ведь традиционно считается, что пользователь Linux должен читать горы мануальников, разбираться в правах доступа и т.д. (см. перечисленное выше). Отвечаю: именно в POSIX-системе такое возможно. Потому что обычно индивидуальный пользователь ее~--- не просто пользователь, но и сам себе админ. То есть он вынужден устанавливать и настраивать систему, устанавливать и обновлять прикладной софт, и так далее.

Здесь же речь идет о создании некоего комплекса, того, что именовалось на заре советской компьютеризации Автоматизированным Рабочим Местом~--- АРМом бухгалтера, переводчика, геолога. То есть~--- монофункциональной системы с сознательно урезанными до необходимого уровня возможностями. Пользователь такой системы не должен в сущности даже иметь root-аккаунта: все, что от него требуется~--- это уметь включить питание, элементарным нажатием двух-трех клавиш запустить пару-тройку (или пару-тройку десятков~--- сколько требуется для выполнения произваодственной задачи) приложений или утилит с требуемыми опциями (а создание скриптов, обеспечивающих такую возможность~--- одна из задач нашего POSIX'ивиста) и выйти с сохранением данных и корректным завершением сеанса; например, по нажатию сакраментальной комбинации из трех пальцев, а уж о корректности всего остального должен позаботиться POSIX'ивист.

И устанавливать программы такому пользователю не придется. Весь комплекс необходимого для его задач софта будет установлен единовременно. Обновления? А нужны ли они, если комплекс этот будет тщательно продуман изначально, подогнан как под задачи, так и под железо? Ведь классические программы в стиле UNIX way меняются мало (в смысле качества~--- давно уже лучше некуда, а в смысле функциональности~--- в том-то и суть UNIX way, чтобы не прикручивать к утилите find функции заварки кофе). По настоящему (не удовлетворения любопытства для) необходимость в обновлениях связана а) с обеспечением безопасности и б) появлением нового оборудования, не поддерживаемого софтом. Но в данном случае ни то, ни другое не актуально: root'овых прав у пользователя нет вообще, а оборудование в таком АРМе не меняется до полной физической амортизации.

А, вообще говоря, все описанное в предыдущих абзацах,~--- не более чем изобретение велосипеда. Именно по такой схеме начиналась всеобщая PC'фикация все страны (тогда еще~--- Советов). То есть: IBM PC/XT с <<черным>> DOS'ом (необходимости в NC <<по делу>>~--- не возникает) и программой бухучета (или там учета кадров), запускаемой batch-файлом, вызываемым нажатием Any Key:-). Правда, реализация этого, как правило, оставляла желать лучшего, но речь сейчас~--- об идее. И представьте, как это может быть реализовано на базе современного <<железа>>~--- раз, и с полной возможностью лишить пользователя возможности (пардон за тавтологию) совершить потенциально опасное действие в принципе~--- два.

Конечно, возникает вопрос. Создание такой системы (при качестве реализации выше среднего уровня)~--- весьма кропотливая работа. Её, в сущности, можно сравнить с ружьем штучного разбора (да еще и с ручной высокохудожественной гравировкой). И много ли заработает наш POSIX'ивист-индивидуал на столь же индивидуальных пользователях?

Скорее всего, не очень. Потому что вопрос этот адресован не ко мне. И упирается в повышение материального благосостояния советского (пардон, российского) гражданина~--- а тут уже UNIX way бессилен. Однако\dots

Однако все сказанное относится не только к обеспечению трудящегося-индивидуала. А имеет силу и для любого трудового коллектива~--- будь то частная фирма или госпредприятие. И даже, я бы сказал, большую силу. Потому что функционально ограниченные АРМы (на которых, в частности, невозможно резаться в tetris или linе, смотреть порнографию по Сети и заниматься прочими увлекательными занятиями в рабочее время) востребованы скорее в служебной, нежели домашней, обстановке. А тут уже:


\begin{itemize}
	\item совершенно другие масштабы~--- это понятно; 
	\item соврешенно другие объемы кропотливой ручной работы~--- ведь АРМы для ста банковских операционисток, выполняющих одинаковую работу, будут практически идентичными, и достаточно отрихтовать руками один экземпляр~--- а дальше просто тиражировать его;
	\item совершенно иные условия работы~--- ведь все эти сто АРМов можно единовременно инсталлировать по локалке;
	\item и, как следствие совершенно иные соотношения трудозатрат/трудоооплат.
\end{itemize}


Впрочем, для последних более существенна проблема спроса~--- есть ли он? Ну, во-первых, отсутствие спроса в настоящее время прямо связано с отсутствием предложения (часто ли в советских магазинах спрашивали черную икру?~--- спросом не пользовалась\dots). А во-вторых, даже не смотря на отсутствие предложения, спрос есть (свидетельством чему~--- <<Заметки тетки-бухгалтера>>, опубликованные еще пару-тройку лет назад в онлайновой Компьютерре).

Предвижу возражение и с другого фланга: а не является ли идея таких АРМов профанацией идеи свободного софта? Одним из краеугольных камней которого, как известно, является свобода выбора. В какой-то мере~--- да, но по сути~--- нет. Потому что начинающий пользователь свободной POSIX-системы все равно свободы выбора не имеет: он просто в силу отсутствия знаний не в силах выбрать адекватный почтовый клиент или текстовый редактор из того легиона программ, который лежит на полудюжине сидюков любого т.~н.~user-ориентированного дистрибутива Linux. Свободу эту он получит только тогда, когда изучит их все, то есть превратится в POSIX'ивиста,~--- но этот вопрос мы уже обсудили, не так ли?

Тем не менее, у него есть свобода выбора другого~--- заниматься всем этим самому или предоставить это дело тому, кто может это осуществить по уровню знаний и должностной инструкции, то есть тому же нашему POSIX'ивисту. А вот у последнего эта самая свобода выбора сохраняется в полном объеме~--- и, более того, он имеет не только право, но и реальную возможность ею воспользоваться.

Более того, рискну высказать крамольную с точки зрения абстрактного свободолюбия мысль: существует немало сфер человеческой деятельности, где свобода выбора не только не полезна, но и просто противопоказана. И пример с банковскими операционистками тут далеко не единственный\dots

Итак, <<кратко резюмирую сегодняшний базар>>:

Будущее коммерческого использования свободных POSIX-систем~--- не в дистрибуции Linux на десктоп каждой секретарши на смену Windows-десктопу, дабы она могла им управлять, как кухарка~--- государством. А в создании специализированных монофункциональных систем на базе некоего дистрибутива общего назначения.


Я хотел еще тут поговорить о том, каким должен быть дистрибутив, выбираемый в качестве базы для наших гипотетических АРМов, и какими качествами должен обладать их создатель. Но понял, что эта тема столь широка, многогранна и необъятна, что являет собой совсем другую историю.

\section{Вот и вспомнил я старые истины\dots}
\begin{timeline}Август 15, 2012\end{timeline}

В заглавие заметки вынесены слова из песни Бориса Алмазова~--- очень мало и незаслуженно мало известного автора и исполнителя, одного из тех первых, кого когда-то безграмотно окрестили бардами. И текст, и исполнение можно найти в Сети.  А я продолжу цитату:

\begin{shadequote}{}Вот и вспомнил я старые истины Пожар, мол, не страшен нищему Одна голова не больна А если больна~--- всё одна.\end{shadequote}

А начал я всё это с того, что есть вечные истины. Что нельзя заработать денег на образовании. На медицинском обслуживании. На фундаментальной науке. В конечном счёте~--- нельзя заработать их и на открытом и свободном софте.  Всё это~--- общее достояние человечества. И дотироваться должно~--- человечеством же. Иначе только опять придётся Визбор Иосича вспомнить~--- про руины великих пожарищ и питекантропа с копьём.  Хотя, похоже, что питекантропы, <<умные и красивые>> (с) Алиса Деева, понимали эти истины куды лучше многих людей современного физиологического типа~--- иначе не выжили бы. А они выжили~--- и стали предками многих из нас.

Ну так вот, врач, поклявшийся именем товарища Гиппократа (это ведь типа воиской присяги), не может не оказать помощь тому, кто в ней нуждается. Учитель, взявшийся учить, не может не учить детей. Научный работник не может не заниматься наукой~--- просто потому, что мозги так у него устроены. И никто из них прибыли извлекать из своей работы не может: иначе мы имеем то, что имеем. Когда врачи не лечат, а калечат, педагоги не учат, а мучат\dots Ну а научные работники умотали в те страны, где наука давно стала разновидностью шоу-бизнеса.

А вот из результатов их работы вполне можно извлечь прибыль. Но извлекать её будут совсем другие люди. С другим устройством мозгов (в терминах <<хорошо>> и <<плохо>> не оцениваем).

И в нормально устроенном обществе они будут её извлекать. И будут платить налоги. Правительству~--- а в конечном счёте обществу. Тому самому, которое, в лице лучших своих представителей, придумает, как дотировать тех, кто никакой прибыли не извлекает. Но от которых в конечном счёте зависит прибыль всех других.

Казалось бы, очевидные вещи, понятные каждому, да?

А вот почему все делают вид, что их не понимают?

Почему RMS призывает разработчиков свободного софта зарабатывать на всякой фигне типа саппорта?  Почему Тео де Раадт зарабатывает деньги на свой проект продажей кружек и маечек? Почему, в конце концов, ребята, реально чего-то делающие, вешают у себя на сайтах номера своих вебманевых кошельков?

\section{Копирайт и научно-технический прогресс: три стороны одной медали}
\begin{timeline}Август 30, 2010\end{timeline}
Случайно наткнулся на такую вот статью: <<Отсутствие копирайта способствовало технологической революции в Германии 19 века>>, представляющую собой перевод статьи Франка Тадеуша (Frank Thadeusz) <<The Real Reason for Germany's Industrial Expansion?>> из онлайнового <<Шпигеля>>.

Статья не бесспорна, но интересна и вызывает разные мысли, частью из которых я и хотел бы здесь поделиться. Для начала остановлюсь как раз на спорных положениях.

Прямые сопоставления Германии и Англии середины позапрошлого века с точки зрения количества <<печатной продукции>>, в частности, научной, представляются не правомерными. Не будем забывать, что в Англии (точнее, в Великобритании вообще~--- Шотландии это касалось не в меньшей степени) профессорствование в университете было <<джентльменским>> занятием. А поскольку истинный джентльмен чужд всякого меркантилизма, оплачивалось это дело крайне плохо.

И потому профессора, не отягощенные собственным  состоянием, в свободное от преподавания время занимались совсем другими, нежели написание книг, делами. Например, в 19-м веке подавляющее большинство профессоров Оксфорда и Кембриджа принимало сан~--- это давало ощутимую прибавку к жалованию. Ну а о частных уроках и тому подобных занятиях и говорить не приходится~--- достаточно вспомнить биографии Льюиса Кэролла или Толкиена.

Кстати, и основной целью системы университетского образования Великобритании было в первую голову воспитание джентльменов, а отнюдь не воспроизведение научных работников или подготовка инженеров для промышленности.

В Германии (опять же, точнее, в германоязычных странах) преподавание и в университетах, и даже в сельских школах рассматривалось как государственная служба (возможно, даже служение). И профессора ведущих университетов обладали не меньшим общественным престижем, нежели высокие чины администрации или военные. Что сопровождалось получением соответствующего жалования. Так что свободное время они вполне могли посвящать сочинению не только чисто научных трудов, но и научно-популярных работ <<для народа>>.

Напрашивается аналогия с русской классической литературой: наличие поместья, достаточного количества крепостных и не очень вороватого управляющего создавало предпосылки для занятий изящной словесностью~--- с соответствующим результатом.

Далее, в Англии не только беллетристика, но и научные работы издавались в значительной мере силами профессиональных издателей, то есть самых обычных бизнесменов, озабоченных в первую очередь тем, чтобы <<отбить бабло>>. В Германии научные книги, те самые, премиум-издания, о которых говорится в обсуждаемой статье, издавались в основном университетами, в которых работали их авторы. Это обеспечивало престиж университету и автору, а иногда даже приличные гонорары последнему. Так что дальнейшее <<распирачивание>> ни тех, ни других уже особенно не волновало. Более того, было фактором положительным, способствуя известности автора (а вместе с ним и его университета) в широких массах.

В общем, можно констатировать, что жесткая копирайтная модель, о которой говорится применительно к Англии, отнюдь не способствовала научно-техническому прогрессу. Но и в Германии таковой осуществлялся не благодаря отсутствию копирайта, а в силу совершенно других факторов.

Однако, проецируя германскую ситуацию на современность, можно разглядеть и третью сторону копирастической медали. Которая явочным порядком ныне реализуется благодаря Интернету. То есть книга, изданная первично в бумажном исполнении на гонорарной основе (хотя нынче речь обычно идёт не о гонорарах, а об отчислениях с продаж), в случае проявленного к ней интереса широко тиражируется на сайтах и блогах, в самых разных форматах, от отсканированного DjVu до PDF. Причём что интересно: такое, вроде бы <<пиратское>>, тиражирование вовсе не вызывает снижения продаж <<бумажного>> издания. Скорее наоборот: читатель, хотя бы приблизительно ознакомившийся с содержанием книги по полуслепому скану, купит её с большей вероятностью, нежели если бы покупал кота в мешке.

Это не голословное утверждение, и проверено на собственном опыте. Так, моя книжка про FreeBSD, которая в виде DjVu- и прочих файлов годами валялась  на десятках, если не сотнях, ресурсв, была, тем не менее, продана фантастическим для такой тематики тиражом. И допечатки тиража, собственно, прекратились только полного устаревания её контента. Совпавшим, кстати, с пресловутым кризисом, вообще сильно ударившим по продажам толстых бумажных книг.

\section{Копирастёж vs антикопирастёж: монета обрастает гранями}
\begin{timeline}Сентябрь 16, 2010\end{timeline}

\hfill \begin{minipage}[h]{0.45\textwidth}
Вроде не пиратили \\
И могли бы жить \\
Им бы копирайты \\
Взять~--- и отменить
\bigskip\end{minipage}

В предыдущей заметке я говорил о трёх сторонах медали, выявленных в модели копирайта. Однако с тех пор она стала стремительно обрастать дополнительными гранями, превращаясь в какой-то ромбододекаэдр. Грани эти содержат вопросы:

\begin{itemize}
	\item должна ли информация распространяться свободно? 
	\item должны ли создатели контента и его распространители получать за это деньги? 
	\item и если должны, то как? 
	\item эффективна ли нынешняя копирайтная модель? 
	\item нужны ли в наш цифровой век какие-либо законы о копирайте вообще? 
	\item и если нужны, то какие?
\end{itemize}

Это основные вопросы, которые обрастают серией дополнительных. Например, что первично:

\begin{itemize}
	\item паршивость нынешних схем распространения контента или поганость большей части контента, распространяемого по этим схемам? 
	\item можно ли обеспечить сочетание легальности, эффективности и справедливости при онлайновом распространении контента (вопрос, что такое справедливость, пока замнём для ясности)? 
	\item и какими методами должно это осуществляться~--- законодательно-административными или технологическими? 
\end{itemize}

Сформулировав положения, которые показались мне первоочередными, я понял, что рассмотреть их все в рамках одной заметки мне слабо. Поэтому буду делать это постепенно, по мере сил и возможностей~--- уверен, что при этом возникнут и другие дополнительные вопросы. Но \href{http://alv.me/?p=1174}{сначала посмотрим}, что такое копирайт, вокруг которого и пойдёт весь сыр-бор. И что такое~--- его противоположность.

\section{Копирастёж vs антикопирастёж: что это?}
\begin{timeline}Сентябрь 16, 2010\end{timeline}

Копирайт в обыденной жизни в последнее время рассматривают как синоним понятия авторского права. Так ли это?

Для начала вспомним, что существуют неимущественные и имущественные авторские права. Первые (право считаться автором, право на имя~--- как ни странно, это не совсем одно и то же, право на публикацию или её отзыв, право на защиту контента, в том числе его названия, от искажений) возникают в силу создания автором этого контента и являются неотчуждаемыми, то есть принадлежат автору во веки веков. Нарушение их называют плагиатом: обычно он выражается в приписывании себе авторства контента или включении чужого контента в собственный без указания настоящего автора.

Имущественные авторские права~--- это в первую очередь право на распространение контента, в том числе исключительное. От рождения оно также принадлежит автору, однако может быть передано другому лицу или любому предприятию на условиях, определяемых авторским договором.

Совокупность неимущественных и имущественных авторских прав символизируется знаком охраны авторского права~--- пресловутой латинской c в кружочке. При этом, как ни парадоксально, на самом деле этот знак ни от чего не охраняет, а только указывает лицо, группу лиц или предприятие, которым принадлежат какие-либо из авторских прав на контент, размещённый на маркированном им носителем.

По английски знак охраны авторских прав называется copyright~--- и именно в данном смысле это слово первоначально вошло в русский типографский жаргон.

Однако основное значение данного английского слова~--- это право на копирование. То есть близкое к тому, что у нас называется имущественными авторскими правами. Каковые, принадлежа изначально автору, очень редко за ним остаются: лишь немногие авторы занимались самоизданием, и ещё меньше~--- в этом деле преуспело. Почему разумные люди и передают право на копирование и распространение людям, у которых это получается лучше. В случае книг это будут издатели, в случае всякого рода медиа-продукции~--- какие-нибудь антрепренёры или там продюсеры.

Впрочем, в медиа-бизнесе я не копенгаген, поэтому дальше речь пойдёт в основном о книгоиздании. При привлечении примеров из смежных сфер (в том числе и софтверной индустрии) это будет оговариваться особо.

В общем случае владельцы имущественных авторских прав так и называются~--- правообладатели. Вот к ним и был привязан термин копирайт уже в совсем обиходной трактовке. Разумеется, в народе их стали звать ласковым словом копирасты, а их деятельность~--- копирастёжем. В их число попадают медиа-магнаты, книгоиздатели и проприетарщики от софта. Впрочем, ввиду контаминации понятий авторского права и копирайта эти слова часто стали применять и к собственно авторам, то есть создателям контента, подпадающего под действие копирайта. Хотя некоторые, как мы увидим ниже, полагают их скорее жертвами копирастёжа.

Разумеется, достаточно быстро появилась и оппозиция. Первой ласточкой тут стало движение FreeSoftware, исповедовавшее принцип copyleft. Впрочем, его влияние сначала ограничивалось почти исключительно областью разработки программ, поначалу оказывая мало влияния на другие сферы человеческой деятельности, и потому пока о нём говориться не будет. Тем более, что на эту тему достаточно сказано в более иных материалах, в том числе и автора этих строк.

А вот более массовый класс-антагонист, охвативший области издательской и медиа-деятельности, возник в лице антикопирастов. Их основным занятием стала борьба за полную отмену копирастёжа, либо, хотя бы, за изменение модели копирайта. Правда, каким образом эту модель следует менять, как правило, не уточняется~--- но к этому вопросу нам придётся возвращаться ещё неоднократно в последующих заметках.

Как и всякое оппозиционное движение, антикопирастёж быстро разделился на два уклона, которые можно условно назвать правым и левым (к взаимоотношениям copyright'а и copyleft'а они имеют некоторое отношение, хотя и косвенное).

Представители первого уклона полагают, что от копирастёжа страдают в первую очередь авторы контента. И именно их интересы надо защищать от злобных копирастов. Для чего следует сильно модернизировать механизмы копирайта, но сам институт всё-таки сохранить (почему я условно и назвал их правыми). Как и все правые, конкретной программы для этого они, однако, не предлагают.

Левые уклонисты, напротив, причисляют авторов к кулакам-мироедам, стоящим на службе копирастов, и не желающих даром отдавать своё добро (то есть авторские права любого рода) народу: потребителям книжной, медиа- и софтверной продукции. Как и все левые, они имеют чёткую, простую и понятную программу действий: копирайт~--- зло, поэтому его надо отменить напрочь, а весь контент сделать свободным во всех отношениях (каких именно~--- рассмотрим в следующей заметке).

А недавно возникло и крайне-радикальное крыло, которое можно сопоставить с одним из течений анархистов, хотя его представители называют себя Пиратской партией. Их программа ещё проще и понятней


\begin{shadequote}{}Будьте счастливы, копируйте все!\end{shadequote}

Такова примерно расстановка сил на копирастическом фронте сегодняшенего дня.

\section{Сравнение мужей: журналист vs блоггер.}
\begin{timeline}Май, 2012\end{timeline}

Данный материал в наибольшей степени соответствует  тематике этнографического исследования, ибо в ней будут сравниваться не некие программы, а группы вполне конкретных людей. А уж что из этого сравнения получится~--- я и сам пока не знаю.

\subsection{Вступление}
Для начала~--- о жанре. Это~--- \textbf{блогометка}. Я решил реанимировать этот термин, некогда изобретённый мной в далёкие те годы для одного из проектов. На копирайт не претендую~--- термин настолько напрашивающийся, что странно было бы, если кто-то ещё не изобрёл его независимо и, возможно, раньше меня. В скобках замечу, что большинство написанных мной материалов принадлежат именно к данному жанру. 

Поводом для этой блогометки послужила блогометка другая~--- моего давнего товарища и коллеги Сергея Голубева <<Журналисты и блогеры>>. В которой, в свою очередь, рассматриваются соображения Андрея Колесова о различии журналистики и блоггерства. Именно на этом акцентируется авторский заголовок: \textbf{<<Публикации и блоги~--- в чем разница?>>}. Однако Сергей назвал эту


\begin{shadequote}{}<<подстраницу>> о различии журналиста и блогера\end{shadequote}

иначе. Не знаю, умышленно он это сделал, или так вышло случайно, но в этих немногих словах ему удалось уцепить суть вопроса.

Ибо речь идёт не различии журналистики и блоггерства как сфер деятельности, а о <<сравнении мужей>>, ими занимающихся. Да простят меня дамы, занимающиеся одним из этих занятий, но это просто этнографический термин. Отнюдь не мешающий прекрасной половине человечества участвовать в этом сравнении.

Так вот, я начал было писать комментарий к блогометке Сергея, но быстро понял, что он разрастается в отдельный материал. Который и предлагается вниманию читателей.

\subsection{Цитаты и комментарии}
Дав только что вводную установку, должен сразу определить свою позицию: в данном вопросе я не последовал завету Шарикова~--- не соглашаться с обоими. Хотя с мнением Андрея Колесова я действительно не согласен категорически. А с мнением Сергея~--- согласен более или менее. Хотя большинство его соображений я высказал бы в более категорической форме. Чем сейчас и займусь, чередуя цитаты из обоих <<исходников>> и свои соображения.

Итак, слово Андрею Колесову (далее~--- \textit{АК}):


\begin{shadequote}{}Для многих людей, блоги~--- это возможность просто высказаться. Журналист и без того, постоянно пишет.\end{shadequote}

Комментарий Сергея Голубева (далее~--- \textit{СГ}):


\begin{shadequote}{}\dotsблоги, в которых нет ничего, кроме самовыражения автора, долго не живут.
\end{shadequote}

Вообще-то, любой автор начинает писать исключительно ради самовыражения. Если он самовыражается так, что его выражения интересны не только ему~--- его читают. Если нет~--- не читают. Для некоторых авторов это превращается в профессию~--- то есть основной источник средств к существованию. При этом прямой корреляцией между <<профориентацией>> и интересностью с читаемостью тут нет. И это, как мы увидим дальше, не значит, что блоггер становится журналистом.


\begin{shadequote}[r]{АК}
Самореклама журналисту не очень нужна\dots
\end{shadequote}


\begin{shadequote}[r]{СГ}
Опять же, все не так линейно.
\end{shadequote}

Настолько нелинейно, что с точностью <<до наоборот>>. Журналиста читают постольку, поскольку он представляет некое издание. Это имело место ещё в бумажные времена, когда в ходу был оборот: <<А ты читал в газете \textit{``Правда''} про\dots?>> Правда, возможно, он возник от того, что никто в точности не знал, кто же на самом деле написал эту статью в газете \textit{``Правда''}\dots

В сетевых же СМИ это выражено ещё более ярко. Обычно говорят: <<на сайте таком-то было написано про\dots>>. И очень мало кто знает имя или хотя бы ник того, кем это было написано.

А поскольку журналисты~--- люди, более того~--- авторы, и ничто человеческое им не чуждо (в том числе и авторское честолюбие, в котором нет ничего худого), они вынуждены заниматься самопеарастёжем~--- просто чтобы не затеряться на фоне своего издания. Одни делают это активней, другие~--- пассивней, но делают почти все. И мера активности зависит не от личного честолюбия, а скорее от системы приоритетов и условий оплаты: что важнее~--- слава или деньги? Кстати, между славой и деньгами корреляция тоже далеко не прямая.

Блоггера же читают таким, каков он есть. Или не читают. Но если уж читают~--- это сам по себе факт признания его лично, а не представляемого им издания. И особых усилий к самопеарастёжу ему уже не требуется. К <<продвижению>> своего блога~--- да, но это совсем другая история.

То есть сказанное совсем не значит, что блоггеры самопеарастёжем не занимаются. Занимаются, и ещё как. Но для них это подчинено главной цели~--- продвижению ресурса. Тогда как известное СМИ и так достаточно <<продвинуто>>, и для журналиста оно~--- в том числе и способ <<продвинуть>> себя лично. Хотя бы для строчки в резюме: <<Работал для таких-то изданий>>.

\begin{shadequote}[r]{АК}
\dotsон и без того публикуется, причем в более популярных местах.
\end{shadequote}


\begin{shadequote}[r]{СГ}
Есть блоги, которые у целевой аудитории популярней некоторых <<профильных>> СМИ. Причем, без всякой рекламы, участия в мероприятиях и п.т.
\end{shadequote}

И тут я не могу не согласиться с мнением Сергея: среди людей, читающих материалы <<по делу>>, блоги и личные сайты (границу между ними провести невозможно) обычно куда более востребованы, чем <<бумажные>> СМИ и даже их электронные версии. Более того, большинство ныне популярных <<безбумажных>> порталов, которые уже можно отнести к категории СМИ (вне зависимости от того, имеют ли они этот статус официально), выросли из <<хомяков>> 90-х годов~--- предтечи нынешних блогов.

\begin{shadequote}[r]{АК}
Блоги~--- это <<потеря времени>>. Это~--- время, когда ты не пишешь статьи, т.е. не зарабатываешь деньги.
\end{shadequote}


\begin{shadequote}[r]{СГ}
Вообще-то заработки некоторых блогеров ничуть не уступают заработкам журналистов. Полагаю, такой блогер скажет, что писать статьи~--- это потеря времени.
\end{shadequote}

И опять-таки должен присоединиться к мнению Сергея. С той только банальной оговоркой, что тут вовсе <<не в деньгах счастье>>. Ибо:

\begin{itemize}
	\item статья для любого издания должна подчиняться его правилам, хотя бы чисто оформительским (врезки, картинки \textit{etc}.), тогда как у пишущего для себя руки полностью развязаны; 
	\item автор, публикующийся в любом издании, должен вписываться в некий объём, ограниченный как сверху, так и, как ни странно, снизу; и добро бы речь шла только о первой границе~--- поставить предел растеканию мыслию по древу можно, хотя и трудно (ибо плохо отражается на презренных килознаках); но обратная ситуация~--- когда тема исчерпана, а до конца полосы ещё тысяча символов~--- куда мучительней; и, к слову, мне известны издания, которые практикуют оплату не за килознаки, а за факт раскрытия темы~--- но они как раз и выросли из <<хомяков>> и блогов; 
	\item самое главное, что написанное для издания увидят только его читатели (а в случае издания бумажного~--- когда материал, возможно, потеряет актуальность); написанное для блога увидят все, кто интересуется темой; а это, даже для тематических изданий, множества весьма слабо пересекающиеся. 
\end{itemize}

В один прекрасный день я осознал это (на самом деле этот список можно было бы продолжать ещё долго)~--- и (почти) перестал писать для изданий.

А возвращаясь к вопросу о заработках, приведу цитату из другого раздела той же страницы Андрея Колесова:


\begin{shadequote}{}
время, которые ты не стучишь по клавишам,~--- это время, в которое ты не зарабатываешь деньги.
\end{shadequote}

Что, опять же безотносительно к деньгам, заставляет вспомнить слова Резерфорда, сказанные им сотруднику, которого постоянно заставал в лаборатории в любое время суток:


\begin{shadequote}{}
Если вы всё время работаете~--- то когда же вы думаете?
\end{shadequote}

Не в обиду моим коллегам, материалы которых я сейчас обсуждаю, но чтение ряда СМИ, вне зависимости от профильности или непрофильности, вызывает в памяти другие слова, теперь уже попугая Флобера:


\begin{shadequote}{}
Jamais, jamais, jamais\dots
\end{shadequote}

\noindentи заставляет если не заплакать по французски, то заржать <<испацтуло>>.

Но поехал дальше. Слова


\begin{shadequote}[r]{АК}
Статьи и блоги~--- это совершенно разные жанры.
\end{shadequote}

пока пропустим~--- к этому вопросу я вернусь на странице, завершающей блогометку. А вот на цитате


\begin{shadequote}[r]{АК}
Блоги в чем-то проще, но во многом~--- намного сложнее и более трудоемко
\end{shadequote}
остановлюсь сейчас. Трудоёмкость что статьи, что блогометки определяется исключительно количеством вложенного в неё труда. Работать, как известно, вообще довольно трудно. И потому <<статей>> в СМИ, написанных левой задней ногой журналиста, куда больше, чем аналогичных по исполнению блогометок в блогах читаемых авторов. Может быть, это следствие предыдущего высказывания? Сводящегося к народному


\begin{shadequote}{}
Фигли тут думать~--- трясти надо!
\end{shadequote}

И не отсюда ли идёт мнение, что вести блоги~--- более трудоёмко? Ибо блоггер, не связанный ни сроками, ни объёмами, имеет таки время подумать~--- а это не самая лёгкая работа.

Но АК причину видит в другом:


\begin{shadequote}{}
основная проблема~--- дискуссии, это большим затраты времени и сил
\end{shadequote}

На что СГ резонно отвечает, что


\begin{shadequote}{}
\dotsнет прямой зависимости от интенсивности дискуссии и успешностью блога.
\end{shadequote}

К его примеру с комильфошками можно добавить и пример нашего с Сергеем старого товарища~--- Владимира Попова: его материалы, размещённые на <<блогоподобных>> ресурсах, всегда имели высокий PR (подчас больший, чем на <<морде>> сайта), хотя в дискуссиях он почти никогда не участвовал.

И последнее~--- об ответственности:



\begin{shadequote}[r]{АК}
Принципиально иная модель \textit{отвественности} (это~--- очень важно!)\dots За \textit{граматические} ошибки в моих статья несет \textit{отвественность} выпускающий редактор, а не я!
\end{shadequote}

Курсив мой~--- пардон, но так в цитируемом фрагменте. Это я не крючкотворства ради (грешен, своих опечаток тоже не замечаю), а для иллюстрирования к высказыванию Сергея,


\begin{shadequote}{}
что в действительно полезном материале на грамматические ошибки и некоторую корявость стиля читатель уже не обращает внимания.
\end{shadequote}

А по существу вопроса об ответственности\dots Юридическую сторону оставим юристам. Но, Андрей, Вам ли с Вашим журналистским стажем не знать, что читателем любое лыко в материале будет поставлено в строку не главному редактору (которого читатель, как правило, не знает), и не выпускающему редактору (о его существовании читатель и не подозревает), а именно автору~--- будь он журналистом или блоггером. Правда, для этого автор должен быть достаточно известным, а лыко -– достаточно <<липовым>>. Иначе его просто никто не заметит.

Ну а на вопрос Сергея


\begin{shadequote}{}
Так в чем же главное отличие журналиста от блогера?
\end{shadequote}

я постараюсь ответить в следующем раздельчике.

\subsection{Так в чём же разница?}

В конце предыдущего раздельчика я обещал ответить на вопрос, поставленный Сергеем. И постараюсь нынче исполнить своё обещание.

Первое отличие банально, лежит на поверхности и бросается в глаза~--- может быть, именно поэтому на нём не было акцентировано внимание: журналист работает журналистом. Блоггер блоггером обычно не работает~--- он может работать кем угодно.

Предвижу возражения по первому тезису: журналист не обязательно должен быть штатным сотрудником некоего издания и получать в нём жалованье. Согласен. Но все профессиональные (то есть зарабатывающие себе этим на хлеб насущный) журналисты, по крайней мере в околокомпьютерной сфере, с каким-либо изданием (или кругом изданий) связаны обязательно~--- формальными или неформальными узами. Настоящих вольных стрелков~--- ландскнехтов от компьютерной журналистики, тех, кто сегодня предлагает свой меч компьютер в одно издание, завтра в другое, нынче нет. Да, пожалуй, что и не было на просторах постсоветской Руси. Ибо это занятие не могло служить гарантированным источником хлеба насущного.

Так что, даже если журналист-профессионал не расписывается в ведомости издания за фиксированный оклад содержания, а получает свои шекели покилознаково~--- считать его свободным художником не больше оснований, чем проходчика или буровика, сидящего на сдельщине: долговременные, пусть и неформальные, отношения с изданием предполагают, что он вписывается в его график.

Блоггер же, повторяю, может работать где угодно. И, как правило, где-то и работает. Даже в том случае, если доход его от блогерства многократно превышает формальную зарплату, некое место постоянной службы он обычно имеет. Но даже если и нет~--- это не значит, что он блоггер по профессии. Потому что удачливое блоггерство неразрывно связано с SEO. А вот это~--- уже профессия. И потому всякий удачливый блоггер~--- неизбежно немного SEO'шник и немного рекламный (или саморекламный) агент.

С сущности, блоггер как раз и выступает в роли того самого вольного стрелка-ландснехта. Только торгует он не непосредственно своим мечом, а военной добычей, которую ему этот меч обеспечил: рекламными площадками на своих ресурсах. И действительно имеет возможность, по крайней мере теоретическую, выбирать, кому их продать. То есть, кроме того, удачливый блоггер ещё немного и маркитант.

Тут впору затронуть вопрос о свободе слова, печати \textit{etc}.~--- одну из вечных тем, поднимаемых, когда речь заходит об информации в самом широком смысле слова, и путях информирования общества в частности: ведь и журналисты, и блоггеры выполняют эту функцию: первые~--- по долгу службы, вторые~--- по призванию (или по тому чувству, которое они призванием полагают).

Казалось бы, тут всё ясно. Прожжённый журналюга, связанный постоянными отношениями с каким-либо изданием (пусть, повторяю, совсем не формальными), даже теоретически не может считаться независимым. Ибо публиковать что-либо, несоответствующее профилю издания, ему не позволят. Говоря <<профилю>>, я вкладываю в это понятие самый глубокий смысл, например: в глянцевом журнале для кошатников не будут публиковать статьи про методы дрессировки служебных собак, в атеистическом издании~--- религиозные проповеди, в издании религиозной тематики, напротив,~--- сочинения записных безбожников\dots эту цепочку можно продолжать долго.

С другой стороны, блоггер~--- часть того самого общества, которое он призван информировать, и которое так в информации нуждается, плоть от плоти его. А потому сочинения его будут восприниматься этим обществом с большим доверием, нежели сочинения профессионального журналиста, которого обществе \textit{a priori} полагает продажным (вне зависимости от того, насколько это обвинение обосновано). И это доверие~--- одна из составных частей успеха блоггерства вообще и удачливости отдельных блоггеров в частности.

На самом деле, как мы скоро увидим, всё не так просто. Но к этому я вернусь после рассмотрения второй составляющей успеха блоггерства и удачливости блоггера.

И тут на память приходят слова Евстигнеева, произносимые им в качестве режиссёра самодеятельности в бессмертном советском фильме \textit{<<Берегись автомобиля>>}:


\begin{shadequote}{}
\dotsестественно, что актёр, не получающий зарплаты, будет играть с большим вдохновением. Ведь тогда актёр должен где-то работать\dots
\end{shadequote}

Мы с советских времён привыкли, и не без оснований, иронизировать над этими словами. Однако в свете темы \textit{Журналиста и Блоггера} они неожиданно обретают новый смысл. Чтобы прояснить его, позволю маленькое отступление.

Когда я уже почти четверть века назад впервые приобщился к той сфере, которую потом назовут IT, я, вследствие острого чувства информационного голода, читал всю компьютерную прессу: сначала совсем <<ненашу>>, потом~--- <<нашу>> переводную и смешанную, потом оригинальную. И по прессе <<ненашей>> и <<нашей>> переводной заметил: записной колумнист <<толстого>> компьютерного журнала занимается сочинением колонок без перерывов года два-три.

Потом года на два-три же его имя исчезает со страниц журнала. Из косвенных источников часто можно было узнать, то это время он посвятил бизнесу, консультационной деятельности (не обязательно по тематике его бывших колонок) или даже возвращался к началу своих начал~--- практической работе в IT-сфере.

А потом, как феникс из пепла, снова появлялся в журнале в амплуа колумниста~--- и, возможно, по совсем другой тематике.

Исключения из этого правила были редки, и однозначно показывали, что штатный колумнист за три-пять лет банально исписывается. Если, конечно, он не занимается ничем другим, кроме текущей редакционной работы журналиста-штатника. Но~--- продолжу цитирование:


\begin{shadequote}{}
Нехорошо, неправильно, если он целый день, понимаете ли, болтается в театре
\end{shadequote}

Ибо периодическая смена рода деятельности для получения нового опыта и новых впечатлений~--- насущная необходимость для творческого работника. А сочинительство в любом жанре, в том числе и в журналистике~--- работа творческая, что бы ни говорили по этому поводу.

Но ведь блоггера, напрямую не получающего гонораров за свои сочинения, сама жизнь ставит в условия, когда он должен временами отвлекаться от своего уютного бложика и окунаться в реальную жизнь ради куска хлеба с маслом. Окружающий же нас реал способен дать массу впечатлений даже самому не впечатлительному человеку.

Участь сия не минует и того блоггера, который зарабатывает своим блогом больше иного журналиста: потому что, повторяю, для этого ему приходится постоянно переходить в амплуа маркитанта-сеошника, а это хороший источник впечатлений и эмоций, столь необходимых для творчества.

Конечно, журналист тоже живёт не на облаке. И тоже получает массу впечатлений от своей профессиональной деятельности. Однако они и ограничены его профессиональной сферой, тогда как у блоггера~--- далеко выходят за её рамки.

И тут становится ясно, что в словах (продолжаю цитирование любимого фильма)


\begin{shadequote}{}
Насколько Ермолова лучше играла бы вечером, если бы днём она стояла у шлифовального станка
\end{shadequote}

\noindentкроется глубокое философское содержание. Результатом является то, что сочинения блоггера более разнообразны и несут на себе более яркий отпечаток действительности. И это, наряду с представляющейся объективностью, вторая причина, почему ярких неординарных блоггеров народ читает охотнее, чем журналистов, возможно, столь же ярких и неординарных.

Поэтому~--- прошу прощения, но ещё одна цитата из того же кладезя. На этот раз последняя:


\begin{shadequote}{}
Есть мнение, что народные театры вскоре вытеснят, наконец, театр профессиональный. И это правильно.
\end{shadequote}

Вариант, озвученный в первой фразе, я не исключаю: вполне может быть, что традиционные СМИ отомрут уже на нашей памяти. Или изменятся до неузнаваемости.

А вот со второй фразой согласиться не могу. Почему~--- ответ кроется во вскользь затронутом выше вопросе о видимой зависимости журналиста и независимости блоггера. Но это~--- уже в следующем раздельчике.

\subsection{Следствия}
А теперь посмотрим, какие следствия может повлечь за собой массовая <<блоггеризация>> источников информации для общества. Для чего необходимо сначала рассмотреть вопрос о <<продажности>> журналистов и <<независимости>> блоггеров.

Первый тезис в нашем народе сомнений не вызывает: все журналюги, а тем более журналасты и журнализды, давно продались на корню. Продались президенту и правительству, коммунистам и олигархам, Госдепу и Моссаду, мировому монополисту или отечественному производителю\dots моя фантазия исчерпана, предлагаю читателю дописать нужное.

Так что, не зная точно, кому именно продалась вся журналистская братия оптом и в розницу, и, тем более, не имея никакой инсайдерской информации о котировках на этом рынке, этот вопрос оставлю на обсуждение тем Неуловимым Джо от журналистики, которых никто не купил. А поговорить предлагаю чуть о другом.

Как уже было отмечено на одной из предыдущих страниц, профессиональный журналист теми или иными узами связан с каким-либо изданием. И должен писать в рамках, в этом издании очерченном. Точно так же, как вольный бич-проходчик, сидящий на сдельщине, копает канаву в направлении, указанном ему начальником партии (нет, не коммунистической, а геологической). Точно так же, как столь же вольный бичара, но промывальщик, сидящий на повремёнке, тоже моет шлихи не там, где ему почесалось под левой подмышкой.

Интересно, что никто не обвиняет в продажности проходчика или промывальщика, хотя они получают своего прямого или тарифную ставку за выполнение приказов начальника. А хорошие промывальщики и хорошие проходчики~--- за хорошее их выполнение. Как не обвиняют себя в этом и сами обвинители, будь они слесарями, чиновниками, офисными работниками и даже начальниками чином пониже. Ибо и они делают то же самое.

Как раз у журналиста степеней свободы в отношениях с начальством несколько больше, чем у офисной секретарши. По крайней мере, как мы скоро увидим, пока больше.

А вот блоггер на этом фоне выглядит как д'Артаньян весь в белом. Или как батька Махно в тачанке под чёрным знаменем. Короче, символом независимости от всех и вся. Ибо начальство над ним если и есть~--- то за пределами блогосферы, там, где он получает зарплату. Или, если зарабатывает на блоге, в силу специфики далёк от всяких политических, идеологических и прочих привходящих моментов.

Казалось бы, существование блоггерства, и блоггерства успешного, напрочь опровергает слова некоего 
В.\,И.\,Ульянова в скобках Ленина:


\begin{shadequote}{}
Жить в обществе и быть свободным от общества нельзя. Свобода буржуазного писателя, художника, актрисы есть лишь замаскированная (или лицемерно маскируемая) зависимость от денежного мешка, от подкупа, от содержания\dots
\end{shadequote}

Ибо блоггер действительно свободен от некоего конкретного денежного мешка, подкупать его никто не собирается, и брать на содержание~--- тем более. Но всё это не делает его независимым от общества, в котором он живёт, как вне блогосферы, так и внутри неё. И, будучи его членом, он также не свободен от манипулирования, целенаправленного или случайного, этим самым обществом.

Более того, он может служить инструментом для манипулирования. Потому что энергичный и удачливый блоггер действительно способен донести некую мысль до множества своих читателей~--- как я уже говорил, в народе веры блоггеру куда больше, чем любому иному пропагандисту и агитатору. Так что со стороны манипуляторов дело остаётся за малым: эту самую мысль блоггеру подкинуть в той или иной форме. Например, в форме борьбы за свободу выбора или технологический прогресс.

Если не подниматься до вершин обществоведения или, напротив, не погружаться в клоаку <<настоящей>> политики, а ограничиться рамками даже не IT-сферы, а той её части, которая имеет отношение к открытому и свободному софту, то совсем недавно мы видели ряд примеров тому. Здесь можно вспомнить

\begin{itemize}
	\item и кампанию против OEM'ной Windows, трактовавшуюся чуть ли не как борьба со вселенским злом; 
	\item и агрессивное продвижение <<ультра-прогрессивных>> интерфейсов вроде GNOME~3 с его GNOMEShell'ом; 
	\item и протекающая буквально сейчас эпопея с внедрением \texttt{systemd} сотоварищи. 
\end{itemize}


Всё это в значительной мере осуществлялось (и осуществляется) силами блоггеров.

Впрочем, не будем лезть в проблемы глобальные и всемирно-общественные, а ограничимся более узкими вопросами сообщества. Ибо блоггеры народ всё больше молодой и увлекающийся. В том числе и идеями улучшения мира. А потому легко ведутся на фразы, обёрнутые в идеи свободы, выбора и прогресса. А то, что за этими фразами стоит желание неудачливого политика чуть подправить своё реноме, или стремление некоей фирмы к доминированию в своём сегменте рынка~--- это ведь на лозунгах не написано\dots

Да, до сих пор организация таких кампаний носила мелко-дилетантский характер. Но это пока~--- от отсутствия опыта и наработок. И то, и другое~--- дело наживное, и манипуляторы быстро ликвидируют эту прореху. Собственно, процесс их обучения можно видеть при сопоставлении кампаний \textit{pro} GNOME~3 и 
\textit{pro} \texttt{systemd}. Первая~--- это грубый лобовой напор в расхваливании новой системы, увенчавшийся лишь относительным успехом.

А вот последняя кампания готовилась более тонко и загодя, начиная с превентивного клеймения потенциальных противников как ретроградов, обскурантов, консерваторов и вообще врагов прогресса; разве что в несчастьях Галилео Галилея их лично не обвиняли. Плюс были учтены уроки великих манипуляторов пошлого. В частности, такой: если постоянно повторять некий набор тривиальных истин, то в конце концов блоггеры, а за ними и их читатели поверят, что они изобретены повторяющим их. Впрочем, это тема совсем другого памфлета.

А пока подведу итог. Если технология манипулирования через <<независимых>> блоггеров получит развитие (а к тому дело и идёт), то действительно, народные театры вытеснят театр профессиональный. Ибо в организованных СМИ и профессиональной журналистике просто не будет необходимости: те же цели будут достигаться с куда меньшими издержками. Как говорится, добровольно и с песнями.

И тогда останется только с грустью вспомнить тех самых прожжённых журналюг прошлого, которых нынче обвиняют в продажности.

Так что не в интересах блоггеров вытеснять журналистов. Как и не в интересах последних~--- вытеснять блоггеров. Ибо тогда степеней свободы в выборе источников информации станет одной меньше. А ведь их всего две\dots

\section{Конец бумажного книгоиздания?} 
\begin{timeline}Ноябрь, 2011\end{timeline}
Заметка о судьбе книгоиздания и книготорговли, планировавшаяся первоначально на несколько абзацев, по логике повествования разрослась в цикл, который и предлагается вниманию читателей.

В цикле этом речь пойдёт исключительно о художественной, исторической, публицистической и тому подобной литературе. И ещё, всё сказанное в данном цикле, за исключением единичных особо оговоренных моментов, касается русскоязычной литературы и её распространения на просторах бывшего Союза~--- там, где она востребована. О проблемах книгоиздания и книготорговли в мировом масштабе пусть рассуждают мировые\dots нужное вписать в меру своей испорченности или политической грамотности.

\subsection{Часть 1}
Несколько лет назад я вдруг обратил внимание, что народ в московском метрополитене почти прекратил читать. Ну или количество читающей публики резко сократилось. Поначалу я не придал этому значения. А потом поймал себя на том, что не вижу в книжных магазинах ничего, что хотел бы приобрести. Да со временем и в книжные стал заходить существенно реже\dots

Одновременно, поскольку я продолжал поддерживать связи с книгоиздательским миром, до меня стали доходить слухи об остром кризисе жанра в этой отрасли. Но поскольку та часть книгоиздательского мира, с которой я общался, была достаточно специфической, масштабов этого кризиса я не представлял. Точнее, перестав писать бумажные книжки, о нём не задумывался до тех пор\dots

До тех пор, пока на Джуйке с подачи Евгения Чайкина aka StraNNicK не началось весьма активное обсуждение статьи Елены Прудниковой <<Радостный взлет цунами>>. Которая, в свою очередь, некоторым образом представляет собой обсуждение статьи Анастасии Якоревой <<Конец одной книги>>.

Пересказывать их содержание я не буду~--- интересующийся темой легко прочтёт сам. Скажу только, что в них констатируется падение оборота книготорговли~--- до 15-20\% в год. За точность цифры ничего не скажу, но то, что книг покупают меньше~--- это факт, подтверждаемый наблюдениями в книжных магазинах.

Во второй статье факт этот объясняется, с одной стороны, конкуренцией с более иными формами проведения досуга, типа компьютерных игр и Интернета, с другой --конкуренцией с электронными читалками.

По первому пункту Елена Прудникова резонно возражает:


\begin{shadequote}{}
Люди, которые проводят досуг перед телевизором или в развлекательных центрах, не читали книг раньше и не будут читать их впредь.
\end{shadequote}

И с этим нельзя не согласиться. Более того, как я уже говорил, впервые о кризисе книгоиздания я услышал от тех, кто издаёт вовсе не <<досугозаполнительную>> литературу, а естественно-научную, например. Которую в Интернете заменить нечем.

А вот со вторым её возражением,


\begin{shadequote}{}
что все больше людей пользуются интернет-версиями изданий только для того, чтобы определить, стоит ли книга того, чтобы её прочесть
\end{shadequote}

\noindentвсё не так просто. Особенно если не ограничивать проблему только онлайновым чтением, но не забывать и о тех самых электронных читалках. Но к этому вопросу мы вернёмся после того, как рассмотрим объяснения Елены Прудниковой, коих она предлагает четыре:
\begin{enumerate}
	\item неадекватные цены; 
	\item качество издаваемой продукции;
	\item неквалифицированный маркетинг; 
	\item дискомфортные условия продажи.
\end{enumerate}

Рассмотрим эти пункты последовательно, пропустив пока второй~--- до выяснения общей обстановки.
\subsubsection{Цена}
Да уж, что есть, то есть, цены на книги ни в~\dots~ухо, ни в Красную Армию. Причём обусловлены они вовсе не жадностью книгоиздателей и тем более авторов, как часто думают люди, далёкие от знания реалий. А многоступенчатостью распространения, с накруткой на каждом этапе, и накрутками конечных продавцов. Цитирую:


\begin{shadequote}{}
за время путешествия от типографии к прилавку её цена до последнего времени ухитрялась вырасти в три-четыре раза
\end{shadequote}

И подтверждаю, что это действительно так, и даже хуже. Одна из моих книжек, отнюдь не боевик-блокбастер (а вполне такая техническая), в некоторых регионах продавали в 10 (!) раз дороже отпускной цены издательства, которую я, по понятным причинам, знаю совершенно точно. А уж если речь идёт о продаже в Ближнем Зарубежье~--- тут вообще туши свет. А ведь, что бы ни вещали наши политики, информационное пространство России, Украины и Белоруссии одно. По крайней мере, если не выходить за пределы IT-сферы. Тем не менее, ей богу, в Израиле мои книжки продавались дешевле, чем в Украине.


\textit{Отступление специально для любителей считать деньги в чужих карманах: автору с этих накруток не обламывается ни копейки. Он получает фиксированный процент с отпускной цены издательства. Которому, в свою очередь, тоже ничего не перепадает от цены книжки на прилавке.}


Всё это верно,

\begin{shadequote}[r]{Александр Галич}
\dotsно, однако, Не подходит это дело к моменту. 
\end{shadequote}

Во-первых, читатель, как правило, понятия не имеет о том, из чего слагается конечная цена книжки на прилавке магазина. Так что все приведённые выше расчёты его, в отличие от нас, многогрешных, нимало не волнуют.

Во-вторых, книги всегда были относительно дороги:

\begin{itemize}
	\item и при социализме, потому что их часто приходилось приобретать отнюдь не в магазине, а, скажем, на толкучке в столешниковом переулке, и вовсе не по цене, напечатанной сзади, 
	\item и при <<перестройке>>, когда книгоиздатели и книготорговцы бросились снимать сливки, пенки и тому подобную сметану со своего высококультурного начинания, 
	\item и при <<пост-советском режиме>>, когда окончательно воцарился бездушный и бездуховный дух чистогана. 
\end{itemize}


В-третьих, не смотря на это, я не знаю ни одного настоящего книгочия, которого остановила бы цена приобретения желанной книги. Разве что чуть притормозила~--- до получки, скажем.

В-четвёртых и главных, на самом деле для настоящего книгочия реальные затраты на приобретение книг не так уж и велики~--- но это уже относится к пункту второму, который будет рассмотрен в следующей заметке.

Так что, хотя цена~--- конечно, весомый фактор, но он имел место быть всегда. Но почему же именно в последние пять лет он оказался таким критичным? Так что объяснение по первому пункту принимается лишь частично. Переходим, в соответствие с принятой ранее договорённостью, к пункту третьему --

\subsubsection{Неквалифицированный маркетинг}
И с этим не поспоришь~--- так себе маркетинг книжной продукции. В большинстве случаев он сводится к аннотациям на обороте титула или форзаце. Причём часто они настолько не соответствуют содержанию книги, что возникает сомнение~--- а открывал ли её автор, с позволения сказать, аннотации?

Однако и тут возникает очередное однако: а когда этот самый книжный маркетинг был лучше. Ссылка на то, что


\begin{shadequote}{}
при социализме книги издавались и раскупались миллионными тиражами
\end{shadequote}

не проходит: тиражи определялись в соответствие с директивными указаниями (вспомним бессмертную работу Брежнева Ленинским курсом, изданную на татарском, башкирском, каракалпакском, якутском и многих других языках массовым тиражом). А раскупались не благодаря <<маркетингу>> (функции которого выполняли тогда штатные рецензенты литературных журналов), а часто вопреки ему. Хрестоматийный пример~--- первое книжное издание \textit{Двенадцати стульев}, разошедшееся мгновенно, не смотря на молчание критиков.

На моей памяти лучшим книжным маркетологом Советского Союза был старик, продававший с рук книги в подземном переходе от станции метро <<Площадь Свердлова>> к улице Горького. Главный его маркетинговый тезис звучал так:


\begin{shadequote}{}
Книга, которую ищет вся читающая Москва!
\end{shadequote}

И ведь обычно это соответствовало действительности, так что покупали.

В отношении маркетинга ничего не изменилось и ныне. Если некая книга (или, скорее, автор) удостаивается специального продвижения, то делается это так, что лучше бы не делалось вовсе. Опять же цитирую автора статьи:


\begin{shadequote}{}
Десять лет такой работы, и читатель начал плеваться при виде рекламы, а издательства~--- бояться её.
\end{shadequote}

В этом отношении резонней поступают издатели научно-технической, в частности, околокомпьютерной, литературы, возлагая маркетинг книг на их авторов. Что же, лозунг


\begin{shadequote}{}
Дело помощи утопающим~--- дело рук самих утопающих
\end{shadequote}

\noindentкак заметил один из героев Бориса Акунина, давно уже должен быть внесён в Конституцию Российской Федерации отдельной статьёй. Возможно, самой первой\dots

Ну а если вернуться к нашим книгочиям~--- всякого рода официальный (ныне фирменный) маркетинг не колыхал их в социалистические времена точно так же, как не колышет сейчас. Прежде информация о книгах, заслуживающих прочтения, передавалась из уст в уста. Ныне~--- через форумы, блоги, ЖЖ, социальные сети. А поскольку возможности оных гораздо шире, чем изустной молвы, можно сказать, что <<народный>> маркетинг только окреп и закалился.

Так что и третий пункт объяснения Елены Прудниковой на самом деле не объясняет книгоиздательского кризиса последних лет.

\subsubsection{Дискомфортные условия продажи}
А вот по этому пункту позволю себе не согласиться с автором целиком и полностью. Да, есть магазины, где пробираешься промеж книжных стеллажей, как гротах сьяновских каменоломен, да ещё заполненных людьми, как в московском метро в часы <<пик>>. Есть магазины, где книги рассыпаны по полкам так, как будто их раскидывал сеятель облигаций в исполнении Остапа Бендера.

Но встречаются и магазины достаточно просторные, где книги тщательно рассортированы по тематике, авторам \textit{etc}. А бывают даже такие, в которых стоят компьютеры, выдающие квиточек с местоположением искомой книжки с точность до зала, ряда стеллажей, конкретного стеллажа, полки в нём и даже места на полке.

Опять же, книгочии, регулярно посещающие определённые магазины, ориентируются в них, даже самых бардачных, гораздо лучше сотрудников. И весь дискомфорт им не помеха~--- они целеустремлённо следуют сквозь толпу случайных\dots не могу назвать их покупателями, скорее посетителей. И без особых проблем находят то, что им нужно.

Но и это не самое главное. Потому что существует институт Интернет-торговли. И как раз в отношении книжек он нынче вполне отлажен: можно заказать книги, в том числе и те, что давно прошли в оффлайновых магазинах, с куроерской доставкой по данному городу или почтой по всей стране, практически с любой формой оплаты~--- от наличными курьеру до web money и даже, иногда, наложенного платежа. Цены на книги в Интернет-магазинах~--- как правило, не выше, чем в магазинах обычных, а подчас и ниже. Стоимость курьерской доставки~--- вполне пристойная, и при заказе нескольких наименований практически незаметна на общем фоне, сроки~--- в пределах двух-трёх рабочих дней.

Конечно, в случае доставки почтой в другой город в дело вступают почтовые тарифы и почтовые же сроки. Но, даже с учётом пересылки, книги оказываются дешевле, нежели в обычных магазинах того же региона, куда они попадают через тридцать три посредника. Или не попадают вообще~--- и тогда Интернет оказывается единственным источником книжной премудрости.

Что характерно~--- книжная Интернет-торговля существует у нас уже полтора десятка лет. Но именно за последние годы она приобрела свой нынешний отлаженный вид, это во-первых. А во-вторых, на эти же годы пришлась и интернетизация множества городов и весей нашей необъятной Родины, в том числе и тех, где до недавнего времени о Сети не приходилось и мечтать.

Так что, вместе с развитием <<народного>> маркетинга, как раз в последние пять лет создались благоприятные условия для расцвета книжной торговли. И почему же они оказались реализованы в виде блистательного провала? Ведь, как мы убедились перед этим, ценовой фактор существовал всегда, и не мог оказать какого-то особо тлетворного влияния именно в это время.

А вот тут самое время вернуться к пункту второму.  О чём~--- в следующем раздельчике.

\subsection{Часть 2}
Итак, в прошлой заметке мы выяснили, что причиной нынешнего падения оборота книготорговли не могут быть ни цены, потому что они всегда были завышенными, ни маркетинг, потому что он всегда был плох, ни условия продажи. Более того, последние два фактора, казалось бы, должны играть на руку продажам книг~--- за счёт сетевого <<народного>> маркетинга и торговли через Интернет, соответственно. Однако этого не наблюдается.

Почему? Чтобы ответить на этот вопрос, обратимся ко второму объяснению, предлагаемому Еленой Прудниковой~--- качеству выпускаемой продукции. Рассмотрев вопрос цен и придя к выводу, что они завышены (мы его тоже рассмотрели, и пришли к выводу, что такими они были всегда), она пишет:

\begin{shadequote}{}
\dotsза сумму, обозначенную на ценнике, читатель вправе потребовать если не шедевр, то добротную вещь, которую он будет читать не один раз. Однако добротной вещи он не получит.
\end{shadequote}

Вот тут-то и зарыта собака! Как я уже говорил, настоящий книгочий за стоящую книгу готов заплатить настоящую цену. Но нету их, стоящих книг, просто не стало. Куда же они делись? Ведь совсем-совсем недавно казалось, что их~--- море израилеванное. Чтобы ответить на этот вопрос, Елена Прудникова предлагает вспомнить, с чего всё начиналось. Вспомним это и мы~--- только спустившись немного глубже.

Сейчас уже как-то забывается, что в начале перестроечного книгоиздания был выпуск <<диссидентской>>~--- произведений, написанных в советское время, однако имеющих <<антисоветский>> характер, и потому в своё время не печатавшихся. Народ поначалу набросился на них, ожидая обнаружить там раскрытие тайн мадридского двора кремлёвских кабинетов и лубянских подвалов.

Однако прочтение одной-двух, много трёх таких книжек вызывало чувство искреннего почтения к литературному вкусу экспертов Главлита и кураторов КГБ, в сущности проделывавших работу литредакторов: вне зависимости от меры <<антисоветскости>>, это были, за редким исключением, просто плохие повести, рассказы, романы. Настолько плохие, что сейчас не припомнить ни их авторов, ни названий. Те же, что не были плохими, интересующиеся уже давно прочли в самиздате или тамиздате (это опять таки к вопросу о <<народном>> маркетинге). Так что интерес к такого рода литературе пропал очень быстро.

Почти одновременно опустели прилавки больших книжных магазинов~--- нечего стало продавать. И вы думаете, это был кризис книгоиздания? Ни фига подобного. Ибо поднялся мутный вал переводной литературы~--- сначала детективов и боевиков, потом \textit{science fiction}, а там и до \textit{fantasy} дело дошло.

Стоило всё это хозяйство несусветные по тем временам (сравнительно с зарплатой госслужащего) деньги, при том, что качество изданий варьировало от плохого до хуже некуда.  А продавалось с лотков в переходах метро, подземных переходах, просто на улице, часто на морозе и пронизывающем ветру~--- то есть в местах, для торговли, мягко говоря, не очень приспособленных.

И, разумеется, ни малейшего маркетинга, кроме <<народного>>: значительная часть авторов была знакома старым книгочеям по единичным публикациям в советских издательствах и самиздату, и знаниями своими они делились прямо у лотков. Часто в обсуждении активно участвовали продавцы~--- среди них встречались очень литературно подкованные люди.

Помню, как на станции Царицыно время в ожидании электрички можно было провести за очень приятной беседой с девушками, торговавшими книжками с лотка. Например, обсудить вклад, который Спрэгг де Камп как редактор внёс в рассказы Роберта Говарда. Наверняка, не все нынешние любители фэнтези это знают\dots


\textit{Отступление. Между прочим, с лотков же торговали и компьютерной литературой~--- тогда она практически сводилась к двум позициям, книгам Брябрина сотоварищи и Фигурнова. Которые, кстати, стоили дороже Чейза и Агаты Кристи, и найти их было гораздо труднее. Но об этом~--- в другой раз и в другом месте.}


Так вот, не смотря на цены, качество полиграфии (точнее, отсутствие такового), никакой маркетинг и и более чем никакие условия продажи, кризисом книготорговли и книгоиздания не пахло. Ибо, хотя далеко не всё, что продавалось с лотков, было шедеврами, но и откровенного дерьма среди переводной литературы встречалось не много.

Ну и плюс, конечно, фактор новизны~--- советская власть не особо баловала нас всеми перечисленными жанрами, а фэнтези тогда многие открывали для себя впервые, как та девушка, что открыла прокладки\dots не помню имени кого.

Однако постепенно чувство новизны притупилось. Да и поток качественной переводной литературы начал иссякать~--- ибо и на западе вовсе не каждый день клепают нетленку. Но и тут кризиса не случилось, потому что подоспел третий вал произведений~--- в тех же жанрах, но уже отечественных авторов.

Правда, как правило, сюжеты их были вторичными. Даже в жанре славянского (или, скорее, славяно-варяжского) фэнтези, претендующего на оригинальность, большая часть оригинального шла как бы от оппозиции ихним западным эльфам и прочим оркам. Тем не менее, произведения той волны в основном были вполне читабельны.

Однако на горизонте собирался уже четвёртый вал~--- вал сериалов. Количество книг одних и тех же авторов стало исчисляться уже не томами, а погонными метрами заполненных ими стеллажей или кубометрами развалов. Авторы были как те же самые, так и новые, причём среди последних в изобилии оказались представлены и переводные, фамилий которых ранее не было слышно.

Прогресс? Казалось бы, да. Однако, мысленно прикидывая погонный метраж, созданный за столько короткий срок, поневоле приходит мысль, некогда высказанная Резерфордом в адрес своего сотрудника: если они столько работают, то когда же они думают? И желание знакомиться с их творчеством, даже путём пролистывания у книжной полки, пропадало напрочь.

Да, я забыл сказать, что параллельно, и даже опережающими темпами, шло исчезновение книжных лотков с их подчас очень интеллигентными продавцами. Постепенно они свелись к развалам типа 
\textit{``Любая книга за \#\# рублей''}
, уцелевшим и поныне. Книжная торговля возвращалась в благоустроенные магазины.

Появился тот самый пресловутый маркетинг~--- хотя бы в виде плакатиков или объявлений по матюгальнику в больших магазинах. Да и цены\dotsну не то чтобы снизились. Но по крайней мере в Москве и Питере не стало столь резкой диспропорции между стоимостью книги и средними заработками активной части населения. Хотя в Замкадье эта диспропорция не только сохранилась, но местами даже и выросла~--- однако тут, как я уже говорил, на помощь начал приходить Интернет.

И ещё обозначилась тенденция к вымыванию из ассортимента изданий прошлых лет, вне зависимости от их удачности или не удачности с точки зрения литературной (про коммерческую не говорю за отсутствием информации).

Нет, отдельных авторов, раскрученных (или успевших раскрутиться) продолжали переиздавать с регулярностью, ненамного уступающей бессмертной трилогии нашего дорого и незабвенного товарища Леонида Ильича Брежнева.

Не оскудевали и полки с переводной классикой, как правило, в виде полупудовых глыб типа <<Весь Толкин в одном томе>>, читать которые могли только те интеллигенты, которые в молодости достаточно позанимались спортом или поработали пролетариями. Да и с классикой отечественной дело несколько наладилось.

Но пропущенную книгу автора средней раскрученности (повторяю, вне зависимости от её литературных достоинств) найти стало возможно только на упомянутых выше развалах \textit{``Любая книга за''\dots}
 Да и то не всегда, в основном там концентрировались как раз книги, литературных достоинств бесспорно лишённые.

Все эти процессы завершились примерно одновременно~--- где-то году к 2005-му. Вот тут-то и начался тот самый кризис в околокнижном бизнесе, промежуточный результат которого мы наблюдаем сейчас (промежуточный~--- потому что итоговый может быть ещё хуже, но об этом в следующей серии). Народ перестал покупать книги, народ перестал читать книги. По крайней мере бумажные новоизданные. Не потому что он перестал читать вообще~--- а потому что почти все новые книги, которые лежали на полках магазинов, стали плохими. Как опять же резонно замечает Елена Прудникова, люди


\begin{shadequote}{}
\dotsне желают читать ту литературу, которую предлагают им магазины
\end{shadequote}

Помнится такой случай. В одном из больших книжных магазинов почти в центре Москвы воздвигли, как у них было в обычае (возможно, в обычае и сейчас~--- давно там не был), пирамиду из только что поступившей в продажу книги. В тот день это были очередные похождения Гарри Поттера. И я был невольным свидетелем разговора двух пацанов лет 10-12~--- то есть, казалось бы, целевой аудитории. Так вот, один из них говорит другому:


\begin{shadequote}{}
Смотри, опять Потного Гарри навалили.
\end{shadequote}

Мне это очень понравилось, и с тех пор я означенного персонажа иначе, чем 
\textit{Потным Гарри}, не называю.

Казалось бы, Русь Великая утрачивает славу самой читающей страны мира? Нет, от этой позорной участи мы оказались избавлены. Чем, как и благодаря кому~--- об этом далее.

\subsection{Часть 3}
Предыдущую заметку я закончил на оптимистической ноте: одновременно с тем, как угроза книжного голода обозначилась более чем реально, появилось и средство борьбы с ней. Надеюсь обосновать, что это действительно так. Но перед этим должен остановиться на нескольких пессимистических моментах.

Для начала вернёмся немного назад~--- к низкому качеству литературы, отдыхающей на книжных полках, всех магазинов, и от любых издательство. То есть коллег-книгочиев, казалось бы, поставили в безвыходные условия сложившейся многоголовой монополии, причём естественно сложившейся (интересно, описано ли такое явление? памятный с марксистких времён картельный сговор тут явно не к месту). А в таких условиях невольно вспоминается про без-рыбьи и без-птичьи, а также про дворника в отсутствие горничной. К счастью, обращаться к столь крайним мерам нет необходимости, ибо (опять цитирую Елену Прудникову)


\begin{shadequote}{}
Каждая вновь выпущенная книга вылетает из типографской машины не на пустую полку, а конкурирует со всеми книгами, выпущенными ранее.
\end{shadequote}

И это не может не радовать, особенно с учётом того, что


\begin{shadequote}{}
За годы социализма в стране были напечатаны даже не миллионы, а миллиарды книг\dots
\end{shadequote}

Здесь прервём цитату и для начала вспомним, какой процент от этих миллиардов составляют бессмертные труды незабвенного товарища Леонида Ильича\dots А можете мне поверить, очень немаленький, особенно с учётом бывших союзных и автономных республик. Причём, что характерно, труды классиков марксизма, ленинизма и тем более сталинизма такими тиражами не издавались никогда.

Далее, не забудем и классиков великой многонациональной советской литературы~--- их тоже было ох как немало. С учётом аналогичных классиков из братских стран социализма~--- так ещё больше. Ну а если сюда приплюсовать просто откровенную халтуру и макулатуру~--- а таковой и при советской власти тоже было вдоволь, например, производственные романы из жизни не сталеваров, и не плотников, и даже не монтажников-высотников, а вообще хрен кого\dots Таким образом обозначенные выше миллиарды сокращаются если не на порядки, то во многие разы~--- точно.

Продолжаю цитату:


\begin{shadequote}{}
\dotsи большая часть этого книжного моря где-то существует.
\end{shadequote}

И высказываю категорическое с ней несогласие. Если исключить перечисленное в предыдущих абзацах, то и из оставшегося многое ныне уже не существует. На счёт рукописей, которые не горят, не знаю~--- хотя подозреваю, что горят, и очень даже неплохо. А что уже вышедшая из типографии продукция с песнями тонет, да простят меня читающие это дамы, в дерьме, вытекающем из прорванной в книгохранилище канализации. Автор этих строк сие наблюдал неоднократно, и даже участвовал в эвакуационных работах. На прочих стихийных бедствиях, а также том, сколько библиотечных фондов оказалось ныне за чертой нашего современного государства, останавливаться не буду.

А перейду к следующей цитате, ещё более оптимистичной


\begin{shadequote}{}
В любой читающей семье в шкафу стоит не одна сотня томов, в нечитающей~--- несколько десятков.
\end{shadequote}

И вызывающей ещё более пессимистичные возражения. Поскольку это верно только в отношении читающих семей\dots эээ\dots среднего, так скажем, возраста, большую часть жизни проживших на одном месте (и, скорее всего, если это место~--- один из крупных городов).

Молодёжь же, в силу ряда причин, на которых здесь неуместно останавливаться, ныне существенно более мобильна. И перевозить им, даже очень читающим, эти самые сотни томов за сотни, а то и тысячи километров не всегда с руки. Не потому, что означенные тома не нужны~--- просто, увы, часто приходится выбирать между нужным, очень нужным и жизненно необходимым. И в число последних книги не всегда попадают~--- в том числе и потому, что до не столь уж далёкого времени личные книжные запасы казались столь же легко восполнимыми, как одежда или домашний инвентарь.

Но, как мы уже видели, нынче это стало совсем не так. Если ещё недавно при необходимости перечитать, скажем, Геродота, можно было просто пойти в магазин и купить его, то нынче это не так однозначно (даже не принимая в расчёт ценовой фактор, который в отношении такого рода литературы стал очень весомым). И если с Геродотом вопрос ещё как-то решается, то уж в поиска Арриана точно придётся побегать до пропотения, а Диодора Сицилийского или Курция Руфа так просто не найти. О научных монографиях исторического направления, случайно затесавшихся в стройные шеренги фоменковцев или пристроившихся к колоннам гумилёвцев, и говорить не приходится.

Так вот, процессе перемещения по стране (и тем более за её пределами) утрачивается изрядная часть тех самых миллиардов книг. Утрачивается не вообще~--- но для данного конкретного индивидуума безвозвратно. Это я вам говорю со всей ответственностью, поскольку носило меня\dots ну не от Амура до Туркестана, как товарища Сухова, а всего только от Кригизии до Корякии и от Кореи до Корсики. И возобновить их оказывается негде.

Потому что те самые миллиарды томов, за вычетом перечисленного ранее, конечно, где-то продолжают существовать. Однако для нашего конкретного индивидуума они недоступны.

Почти недоступны. Потому что, переходя, наконец, к оптимистической части своего повествования, позволю себе ещё одну цитату\dots


\begin{shadequote}{}
\dotsв современном городе даже страстный библиофил вполне способен прожить, вообще не заходя в книжный магазин.
\end{shadequote}

И вот тут я подписываюсь обеими руками и ногами. Но не потому, что у него дома стоят сотни томов, а по совершенно другой причине, о которой~--- на следующей странице.

\subsection{Часть 4}
Ну вот, после очередной порции уныния в предыдущей заметке мы наконец вышли на финишную прямую к светлому оптимистическому будущему. Так что же спасёт русских книгочиев? Вы будете очень громко смеяться, но спасут их\dots китайцы.

Выше мы обозначили роковой для книжного бизнеса рубеж~--- 2005-2006 годы. Странным образом он совпал по времени с распространением так называемых электронных книг, в просторечии именуемых, да простят меня читающие это дамы, e-book'ами, reader'ами или попросту читалками (последний термин и будет использоваться в дальнейшем).

Нет, читалки существовали и раньше, ещё в прошлом тысячелетии, однако на протяжении долгого времени широкой популярностью не пользовались. О причинах судить трудно~--- можно предположить, что главное была высокая стоимость. Не последним являлось та, что вывод текста осуществлялся на LCD-дисплей, то есть нагрузка на зрение была точно такой же, как и при чтении с обычного компьютерного монитора. Хотя на одном из форумов меня поправили, что были и вполне комфортные модели, но это очень индивидуально: я знаю людей, которые спокойно читают с двухдюймового экрана дебильника.

Однако на рубеже 2005-2006 годов в продаже появляются читалки принципиально иного типа, основанные на технологии так называемых электронных чернил (E-ink). Собственно на технологии я здесь останавливаться не буду. В рамках настоящего повествования достаточно сказать, что чтение с такого рода устройств создавало те же ощущения, что и чтение с бумажного листа (почему эту технологию называют также электронной бумагой). И, соответственно, нагрузка на глаза сильно снижалась. Правда, в отличие от LCD-читалок, читалки <<электробумажные>> (в дальнейшем~--- ЭБ) требовали внешнего источника освещения~--- читать с них в темноте (например, под одеялом) невозможно.

Кроме того, энергия аккумулятора расходовалась только при <<пролистывании>> страниц, благодаря чему время работы без подзарядки составляло очень многие часы. Собственно, автономия этих устройств измеряется не временем, а количеством просмотренных страниц, число которых обычно начинается с 10 тысяч.

Читалки на e-ink, точнее, их программное обеспечение (а изрядная их часть работает под управлением ОС Linux) способны выводить текст во многих общеупотребимых форматах~--- PDF, HTML, TXT, RTF, иногда DjVu, а также в форматах специализированных~--- ePUB и FB2. Последний, разработанный специально для использования в читалках нашими соотечественниками во главе с Дмитрием Грибовым и Михаилом Мацневым, снискал наибольшую популярность для русскоязычных текстов. Именно в нём по умолчанию доступны произведения нашей литературы на крупнейших Интернет-порталах соответствующего направления (парадоксально, что один из них, до некоторого времени самый известный, располагается в Эквадоре).

На протяжении первых лет своего существования ЭБ-читалки не могли похвастаться широким кругом пользователей. И причиной тому было сочетание двух факторов. Если LCD-читалки, в сущности, представляли собой облегчённые КПК, обеспечивающие ряд дополнительных возможностей, то ЭБ-читалки~--- в сущности, монофункциональные устройства. Кроме вывода текста, они способны, разве что, отображать картинки и воспроизводить звук~--- даже сетевые средства, как правило, напрочь отсутствуют. То есть электронные книги должны скачиваться из Интернета на компьютер и уже с него тем или иным способом (обычно~--- по USB-кабелю или посредством SD-карточек) переноситься на читалку.

И при этом цена ЭБ-читалок была совершенно запредельной для столь ограниченных по возможностям устройств: она начиналась, если мне не изменяет память, с 400 примерно уёв, нарастая, в зависимости от диагонали экрана (а типичные её значения укладывались в диапазон 5-10 дюймов) и <<брендовости>> (по простому~--- жадности) производителя до\dots да собственно до пределов жадности и возрастая.

Причём, в отличие от всей прочей компьютерной техники, надежд на снижение цены с ростом массовсти этих изделий не было. Ибо изрядная часть стоимости ЭБ-читалок приходилась на отчисления владельцу патента на технологию E-ink (он так и зовётся~--- E Ink Corporation). И отчисления эти, по агентурным сплетням из осведомлённых кругов, составляли около ста бкасов с каждого устройства. Производителю которого, кроме погашения расходов, надо было и себя не обидеть.

Так что, казалось бы, ЭБ-читалки не могли составить конкуренции чтению с компьютерного монитора для оцифрованного контента, с одной стороны, и не могли восполнить дефицит качественной бумажной литературы~--- с другой. Именно к этим тяжким годам и относится наблюдение, с которого я начал этот цикл заметок: народ перестал читать в метро.

Неужели для книгочиев обозначился тупик? А вот фиг, простите за невольную рифму. Потому что в течении последнего года чётко обозначилась тенденция к подешевлению ЭБ-читалок. Сначала это было ползучее снижение, то есть младшие (с точки зрения диагонали) модели дошли до терпимых 200 с копейками ихних тугриков. А в последние\dots даже не месяцы, а буквально недели произошёл настоящий обвал цен. И ныне стоимость моделей с экраном 5>> стартует с 4 тысяч рублей~--- обычных, постсоветских. Есть и устройства с вообще демпинговыми ценами, менее трёх тысяч, но они функционально ограничены, в частности, не понимают наш любимый формат FB2.

Правда, говоря о снижении и тем более обвале цен, я выразился не совсем правильно (или даже совсем неправильно). Цены на <<брендовые>> ЭБ-читалки почти не изменились, как и уже давно продающиеся полу-бренды (в число последних можно отнести существенно украинский PocketBook). Но появилось много моделей от фирм, ранее на нашем рынке не представленных, таких, как WEXLER и Hanvon. И сразу по низким (первая) или бросовым (вторая) ценам. Правда, и ползучее  снижение цен на продаваемые модели также имело место быть~--- но оно затронуло не очень именитых (правильней сказать~--- не зазнавшихся) производителей, вроде Gmini~--- его читалки продаются по 5-6 тысяч.

Чем это обусловлено~--- трудно сказать. Самое простое объяснение, конечно~--- фирма E Ink Corporation резко снизила патентные отчисления, движимая альтруизмом и любовью к книгочиям всех стран и народов. Поскольку мы с вами верим в лучшие чувства людей, даже тех, которые представляют бездушный и бездуховный мир чистогана, примем это объяснение в качестве основного. Однако элементарная аккуратность бывшего научного работника заставляет меня рассмотреть и другие варианты.

Первый~--- конкуренция со стороны LCD-читалок. Как уже было сказано ранее, они появились задолго до читалок ЭБ и популярности не снискали. Однако жизнь не стоит на месте~--- и в последнее время читалки с жидкокристаллическими дисплеями очень сильно усовершенствовались: и воспроизведение текста на них стало более чем приличным, и время автономии существенно возросло. К тому же они обладают функциями, ЭБ-читалкам в принципе недоступными: возможностью чтения в абсолютно тёмной обстановке и способностью воспроизводить не только звук, но и видео. И всё это~--- при цене, в среднем на треть меньше, чем у ЭБ-читалок. А LCD-модели начального уровня, с экраном 4>>, вообще можно приобрести за 2 тысячи рублей. И, однако, текст они выводят так, что он воспринимается даже остатками моих глаз.

А второе объяснение, почти конспирологическое, было высказано в обсуждении на Джуйке участником, известным как @Zmeyko: китайским производителям, способным обеспечить массовое производство читающих устройств, удалось свести патентную составляющую их цены практически к нулю. Разумеется, исключительно с помощью доброго слова (типа~--- или берите \$5, или сосите и худейте).

Возможно, будущее прояснит, какое из предложенных объяснений правильно. Пока же можно констатировать массовое распространение ЭБ-читалок. По крайней мере, в Москве. В первой заметке цикла я уже говорил, что почти полное отсутствие читающей публики в столичном метро в текущем году сменилось большим количеством людей, скрашивающих переезды читалками. А в последнее время я вижу ребят и девчат с читалками даже в переполненных маршрутках.

Ну а какие следствия это может иметь для книгоиздательского дела~--- поговорим в чуть позже.

\subsection{Часть 5}
Таким образом, в прошлой заметке мы выяснили, что смерть от книжного голода нашим книгочиям не грозит, и спасут от неё китайцы. Вернее, обеспечат материальную базу для спасения~--- собственно, уже обеспечили. Однако кроме инструмента для чтения необходимо иметь и объект оного~--- а тут уж не помогут ни китайцы, ни даже марсиане.

И тут самое время вспомнить про те миллиарды томов, о которых Елена Прудникова говорила как о <<где-то существующих>>. Мы с вами тоже обсудили этот вопрос и пришли к выводу (>>я думаю, что мы обсудим, и я, понимаешь, решим>>~--- Тимур Шаов), что если они и существуют, то та часть из этих миллиардов, которая могла бы заинтересовать читателя, в существующем виде, то есть в виде томов, для него практически недоступна.

Чтобы обладатели китайских читалок получили доступ ко всему океану русской (точнее, русскоязычной, в том числе и переводной) литературы, океан этот надо разделить на цифровые ручейки. То есть отсканировать тексты, <<распознать>> их, претворить в какой-либо машинно-читаемый формат и разместить результаты на общедоступных сетевых ресурсах.

И такая работа ведётся очень давно, как минимум со времён знаменитой Библиотеки Мошкова. Правда, до недавнего времени такие онлайновые фонды были предназначены для чтения или в распечатках, или в онлайне~--- с компьютерных мониторов.

Однако, как я уже говорил, одновременно с широким распространением читалок появились и библиотеки, специально (или преимущественно) ориентированные на их обладателей. То есть с преобладанием текстов в форматах fb2 и epub, хотя, как правило, они включают и html-версии для предварительного ознакомления в онлайне, а часто также версии в PDF, иногда~--- в DjVu и более иных форматах.

Объемы наиболее крупных таких фондов ныне исчисляются сотнями гигабайт. И там можно найти\dots ну, конечно же, не всё, что издано на русском языке, но 
\textit{buono parte}
: от литературных памятников Древнего Мира до современных боевиков, детективов и прочей фантастики. По поводу последних время от времени возникают конфликты с правообладателями (или теми, кто себя таковыми считает), но на этой теме мы сейчас останавливаться не будем.

Замечу только, что подобно магазинам, торгующим через Интернет бумажнымии книгами, существуют и системы онлайновой продажи электронных книг. Правда, как у них самих дело обстоит с правообладанием и, главное, обламывается ли что с этого ныне живущим авторам~--- для меня не вполне ясно. Был бы признателен за комментарии от лиц, которые в теме.

Пока же наша цель~--- поглядеть, как появление читалок сказалось (и, главное, может сказаться) на книгоиздательском деле. И тут впору опять вспомнить те две статьи, которые, собственно, и послужили поводом для сочинения начала цикла. Так, Анастасия Якорева считает, что упадок книжного дела объясняется


\begin{shadequote}{}
конкуренцией книг с электронными носителями и интернетом
\end{shadequote}

Про <<другие формы проведения досуга>> из продолжения этой фразы умолчим: мы, вроде, выяснили, что те, кто проводит его за


\begin{shadequote}{}
интернетом, торгово-развлекательными центрами, телевидением
\end{shadequote}

\noindentпри отсутствии оных не бросились бы читать книга, а лузгали бы семки, пили портвейн по подворотням и били морды друг другу на танцплощадках.

Впрочем, и мнение Елены Прудниковой о неконкурентоспособности цифрового контента вследствие технической отсталости вынужден поставить под сомнение. Особенно когда она ссылается на такого большого авторитета в этой области, как Дарья Донцова. Ну и на слова


\begin{shadequote}{}
Кто не верит~--- пусть попробует отправить электронную почту из любого российского райцентра.
\end{shadequote}

\noindentмогу возразить: те, кому это действительно надо, умудряются отправлять электронную почту (и даже активно участвовать в форумах) не то что из райцентров, но даже из глубины корякских тундр.

А вот слова

\begin{shadequote}{}
\dotsчто все больше людей пользуются интернет-версиями изданий только для того, чтобы определить, стоит ли книга того, чтобы её прочесть? И если стоит~--- покупают бумажный вариант.
\end{shadequote}

\noindentзаслуживают более подробного рассмотрения. Потому что по результатам <<партитурного чтения>> (оказывается, я всю жизнь по первости читал книги именно <<партитурно>>, только, подобно герою Мольера, не подозревал об этом) html-файла действительно определяется, стоит ли книгу прочесть. И если стоит~--- она скачивается с одной из упомянутых выше библиотек и переносится на читалку. А вот если оказывается, что её стоит не только прочесть, но и перечитывать~--- тут-то дело и доходит до покупки бумажного варианта.

Или~--- не доходит. Потому что, как обычно, возникает два варианта:

\begin{enumerate}
	\item книгу можно прочесть (а подчас и не дочитать до конца), но уж перечитывать её точно не захочется~--- и тогда покупать её бумажную версию никто не будет; 
	\item книгу нужно прочесть, и потом неоднократно перечитывать, может быть, на протяжении всей оставшейся жизни~--- но тогда её бумажной версии, скорее всего, купить не удастся, в силу причин, озвученных в одной из прошлых заметок. 
\end{enumerate}

То есть, в сущности, никакой конкуренции между между бумажными и электронными книгами не существует. В первом случае, скорее всего, книга была бы отвергнута и в бумажном виде~--- на стадии её пролистывания у стеллажа. Возможно, необоснованно, но вероятность купить бумажную халтуру намного порядков превышает возможность случайно нарваться\dots даже не на шедевр, а просто на приличное произведение.

Во втором же случае, как неоднократно повторялось на протяжении всего цикла, читателю всё равно не удастся купить понравившуюся книгу. Просто потому, что издатели не переиздают хороших книжек, которые, что называется, <<прошли>>. То есть они Сами Себе Злобные Буратины (синдром ССЗБ).

Так что же, со временем народные театры электронные книги вытеснят театры профессиональные книги бумажные? Есть такое мнение, и оно не лишено оснований. Но это, товарищи, не правильно. Рискну предположить, что многие, если не большинство, из читающих эти строки любят книгу не только как источник знаний (вместилище информации), а саму по себе, как явление, не побоюсь громкого слова, культуры (иначе они бы это не читали). Любят её в единстве содержания и формы. И было бы очень грустно, если бы это явление исчезло: значит, одним явлением культуры будет меньше. А их нынче и так не густо.

По моему глубокому убеждению, онлайновое чтение, электронные читалки и бумажные книги~--- вещи не взаимоисключающие, а взаимодополняющие. По крайней мере, должны ими быть. При одном необходимом условии: книгоиздатели и книгопродавцы это осознают, исцелятся от синдрома ССЗБ и смогут перестроиться соответственно изменившимся реалиям жизни.

В противном случае им просто настанет кирдык. То есть кирдык, конечно, настанет не всем: многие книгоиздатели смогут переключиться на производство туалетной бумаги (а что, тоже целлюлоза), а иные книгопродавцы займутся торговлей презервативами и памперсами. Но мы-то бумажные книжки потеряем напрочь. И, кстати, потеряем вместе со всеми потенциальным Пушкиными и Шекспирами, но это~--- совершенно отдельная песня.

А пока, в следующем номере нашей программы, рассмотрим, чисто гипотетически, вопрос, можно ли этого избежать. И если можно~--- то как.

\subsection{Часть 6}
На протяжении предыдущих пяти заметок мы долго и упорно пытались разрешить первый из вековых русских вопросов~--- кто виноват? Так что теперь нам осталась сущая мелочь~--- ответить на вопрос: что делать, блин? И главное, кому?

Понятное дело, что книгочиям ничего делать не надо, кроме выполнения своей боевой задачи~--- читать книги, бумажные или, за отсутствием таковых, электронные. Книгопродавцам тоже беспокоиться особо не о чем~--- на крайняк они могут превратить книжный магазин в салон по продаже элитной мебели или гаванских сигар имени Че Гевары (кроме шуток, было такое в Нерезиновой).

А вот книгоиздателям явно надо что-то делать~--- если они, конечно, не желают переключиться на производство туалетной бумаги.

Первое деяние очевидно~--- развивать онлайновую торговлю. Практика продажи компьютеров и комплектующих показывает, что удачливы в онлайновой торговле те, кто ранее удачно занимался торговлей оффлайновой. Потому что за кадром и той, и другой~--- одна и та же логистика. И с этим у давно существующих издательств всё нормально: склады есть, складские запасы~--- тоже, остаётся только наладить и приём заказов и своевременную доставку конечному пользователю.

Второе деяние~--- очевидно не менее: перестать печатать плохие книжки, а печатать только хорошие. Остаётся решить, какие книжки считать хорошими, а какие~--- плохими.

С классикой мировой и отечественной литературы всё понятно. Скажем, наше фсио, ас наш Пушкин, может нравиться или не нравиться. Но никто, даже такой его нелюбитель, как ваш покорный слуга, не может сказать, что книжки, написанные им~--- плохие.

Однако одной классикой жив не будешь. Что-то не знаю я людей, коротающих досуг исключительно за чтением Гомера, чередуемого с Шекспиром и даже Львом Толстым. разве что в тюрьме или в восемнадцатимесячном рейсе по Индийскому океану~--- в обоих случаях ничего другого просто нету. В нормальных же условиях у людей есть привычка читать что-нибудь новое (если, конечно, у них вообще есть привычка читать).

Утверждая на одной из предыдущих страниц, что все ныне издаваемые книги плохие, я, разумеется, преувеличивал: среди написанных за двадцать лет постсоветской власти встречаются и неплохие произведения, и даже откровенно хорошие. Вот только распознать их среди многометровых серий очень трудно. И в итоге читатели не покупают ни плохие книги, ни хорошие. По меткому выражению Елены Прудниковой, серии


\begin{shadequote}{}
\dotsстали братской могилой для десятков перспективных книг.
\end{shadequote}

Как избежать этого? И вот тут на помощь могут прийти электронные книги как средство первичного ознакомления как с конкретными произведениями, так и, главным образом, с образцами творчества данного автора вообще. Вероятность того, что читатель, знакомый с романами Имя рек по электронным версиям, и составивший о нём благоприятное впечатление, приобретёт его новую книгу в бумажном виде, возрастает многократно.

При этом издательству вовсе не обязательно полагаться на <<народные>> библиотеки электронных книг~--- они вполне могут заниматься их распространением и сами. И даже не обязательно бесплатно~--- я знаю немало людей, которые готовы покупать их за деньги. И даже делают это~--- в случае, если цена представляется адекватной, а форма оплаты~--- удобной.

Что до цены~--- поскольку себестоимость электронной версии изданной книги стремится к нулю (ведь она уже существует в машинно-читаемом виде и не нуждается в редподготовке, верстке и так далее), то и продавать их можно очень дёшево. И, при больших объёмах продаж, это дело вполне может оказаться прибыльным для издательства и приносить некоторый доход авторам. Кроме того, изучение спроса на электронные книги позволит выявить те из них, которые заслуживают переиздания в бумажном виде.

Ну а удобных форм оплаты нынче существует сколько угодно~--- от банковских карточек до Web Money и Яндекс-денег. Разве что оплаты путём отправки СМС, как мне не раз попадалось в онлайновых книжных магазинах, не надо: эта форма в народе стала прочно ассоциироваться с мошенничеством\dots

Правда, для выполнения описанных выше действий требуется два необходимых условия: предварительно распространяемые книги должны быть написаны, во-первых, и изданы~--- во вторых. То есть они касаются уже действующих авторов. А вот где взять новых? И каким образом издатель может выявить авторов, потенциальные  книги которых заслуживают публикации и читательского внимания? Рискну предложить один из вариантов по аналогии с изданием научной литературы.

Традиционно, ещё с советских времён, считается, что научное книгоиздание~--- занятие сугубо убыточное. Хотя мой старый товарищ Валентин Зинкевич, директор издательства Научный мир, издаёт научную литературу уже более 15 лет~--- и, судя по нашей недавней поездке в Ростов Великий и Кострому, помирать с голоду пока не собирается. А на проклятом Западе существуют научные издательства, успешно функционирующие уже не первое столетие.

С четверть века назад в журнале \textit{<<Природа>>} была опубликована статья~--- к сожалению, не помню ни автора, ни названия, и поиски её в онлайне успехом не увенчались. Написана она была сотрудником издательства \textit{<<Наука>>}, посетившим США в перестроечные времена, на заре эпохи всякого рода телемостов и обменов, на предмет постановки научного книгоиздания в этой стране.

И начинается эта статья с утверждения, что в США издание научной литературы~--- дело прибыльное. Не приносящее баснословных доходов, окупаемое в течении длительного времени, но, тем не менее, прибыльное. И достигается это тем, что соответствующие специализированные издантельства публикую, во-первых, хорошие книги, и во-вторых, книги долговременного спроса.

Каким образом осуществляется фильтрация? Как пишет автор статьи, в те далёкие, ещё доинтернетовские, времена в крупных специализированных издательствах существовали своего рода научно-литературные агенты. Они ездили по научным конференциям и всякого рода мероприятиям, ошивались в кулуарах, вступали в контакты с участниками, слушали околонаучные разговоры. И, таким образом, составляли представление о том, книги какой тематики могут быть интересны читателям, и кто из действующих научных работников мог бы такие книги написать. После чего издательство заключало с <<отфильтрованным>> автором договор~--- и, по прошествии должного времени, книга шла в производство и появлялась на прилавках.

Насколько я понимаю, агентам этим вовсе не обязательно было иметь специальное образование~--- да и при обилии научных направлений никакое образование не помогло бы в них ориентироваться. Скорее от них требовалась журналистская хватка и интуиция~--- это примерно такая же работа, какой занимается главный герой романа Саймака 
\textit{Почти как люди}
.

В наши дни изрядную часть такого рода работы можно выполнять через Интернет~--- и так, собственно, и делается. Причём не только <<у них>>, но и <<у нас>>. Так, издательство, с которым я сотрудничал на протяжении ряда лет~--- 
\textit{БХВ-Петергбург}
, специализирующееся преимущественно на выпуске компьютерной литературы, активно занимается мониторингом тематических ресурсов~--- сайтов, блогов \textit{etc}.~--- и, выявив авторов, работы которых могут заинтересовать читателей, обращается к ним с предложением написать книгу по мотивам их онлайновых материалов. А поскольку потенциальные авторы, как правило, уже достаточно известны в сети, некий предварительный <<народный>> маркетинг будущей книге уже обеспечен.

Ни о чём подобном в области художественной и публицистической литературы я не слышал. А ведь литературных сайтов и блогов нынче ничуть не меньше, чем околокомпьютерных ресурсов\dots

Мы рассмотрели проблему книжного дела по одну сторону баррикады~--- книгоиздательскую. Однако есть и другая сторона медали~--- авторская, о которой до сих пор практически не было. А ведь, как уже было сказано, прежде чем мнигу можно будет издать и, тем более, распространять, она должна быть написана.

Правда, говоря о проблеме авторов книг, нам придётся вступить на зыбкую почву копирастии, чего я хотел всячески избежать~--- уж

\begin{shadequote}[r]{Александр Галич}
Больно тема какая-то склизкая, \\
Не марксистская, ох не марксисткая 
\end{shadequote}

Да и жёвана и пережёвана в последнее время до такой степени, что остаётся её только выплюнуть.

Однако логика сюжета заставляет сказать несколько слов и по копирастическому вопросу~--- но уже в следующей заметке.

\subsection{Часть 7}
Последняя заметка была посвящена одному из гипотетических вариантов, по которому книгоиздатели гипотетически смогли бы выбраться из вовсе не гипотетического, а более чем реального кризиса. Однако для этого требуется некоторое участи и второй стороны книжного дела, а именно авторов.

Конечно, за время существования мировой литературы написано столько, что даже русскоязычную (оригинальную и переводную) её часть~--- читать не перечитать. Однако без притока <<свежей сочинительской крови>> действительно есть риск (опять-таки цитирую Елену Прудникову из обсуждения к упомянутой статье), что


\begin{shadequote}{}
народ начнет переходить на читалки, пусть читает старые книги. Не думаю, что кто-то станет задаром писать новые.
\end{shadequote}

И с этим нельзя не согласиться. Разумеется, автор будет писать и задаром, для типа самоутверждения~--- некоторое время. Пока очень кушать не захочется. А когда захочется~--- займётся чем-то другим. Например, переквалифицируется в управдомы. И управдомом будет скорее всего плохим. И таким образом общество потеряет потенциально хорошего автора. Но приобретёт кинетически плохого управдома. Который сможет проесть плешь всем тем, кто не платил за его книги.

Так что подумайте, ребята: когда вам пудрят мозги в домуправлении~--- может быть, это тот автор, за книжки которого вы некогда пожлобились заплатить?

В общем, скучно мне стало писать на эту тему. Любителям острых шахматных ощущений рекомендую ознакомиться вот с обсуждениями её на Юниксфоруме. Откуда вырву с мясом цитату~--- высказывание моего старинного знакомого, хорошего товарища, но непримиримого идеологического противника sash-kan'а:


\begin{shadequote}{}
здесь как раз выплывает на повестку дня порочная связь с трупом копирайта и именно в процессе насилования этого трупа такая торговля (и онлайн- и оффлайн-) и хиреет не по дням, а по часам.
\end{shadequote}

Трудно с этим не согласиться. Но~--- позволю процитировать сам себя, в чуть отредактированном виде:


\begin{shadequote}{}
Прежде чем прекратить насиловать труп копирайта, надо найти где-нибудь симпатичную живую девушку, которая давала бы сама. 
\end{shadequote}

Да ещё и на взаимовыгодных условиях~--- в том числе взаимовыгодных финансово. А, как поётся песне


\begin{shadequote}{}
Где ж её нам найти, Где ж её взять? Ведь её не найдёшь, Девятью пять.
\end{shadequote}

Так что пока, увы, альтернативой трупу копирайта является --


\begin{shadequote}{}
тихо, сам с собою, правою рукою.
\end{shadequote}

Ну и разумеется, как


\begin{shadequote}{}
устал работать правой~--- работай левой.
\end{shadequote}

В общем, тема мне осточертела, и я с ней завязываю. По крайней мере, на ближайшее время.

Тем не менее, я очень доволен~--- косвенно тема эта послужила поводом для знакомства с сочинениями Елены Прудниковой. Каковые, при полном идеологическом несогласии, читаю с большим удовольствием.

\section{Будущее линуксописательства} 
\begin{timeline}Февраль 9, 2012\end{timeline}
Последнее время ловлю себя на том, что, встречая некий вопрос на одном из посещаемых мной форумов FOSS-тематики, всё чаще отвечаю: да прочтите же вы, наконец, какую-нибудь толстую книжку про UNIX или Linux. Согласен, ответ не вполне политкорректный. Но за без малого десять лет обретания на окололинуксовых форумах реально стали раздражать вопросы, которые в качестве ответа требуют пересказа нескольких десятков страниц из любой книжки указанной тематики.

Однако со временем меня начала грызть совесть. Потому как такой ответ подразумевает, что не худо бы добавить, какую конкретно книжку следует прочесть. Или, хотя бы, обозначить диапазон рекомендуемых книжек. Желательно, конечно, на русском языке. И тут я призадумался\dots

Разумеется, в запасе всегда был вариант, потибренный у Бернарда Шоу. Который на вопрос одной юной девушки, что бы мне почитать умного, ответил:


\begin{shadequote}{}
Читайте меня
\end{shadequote}

Но увы~--- во-первых, я не Бернард Шоу. Во-вторых, рекомендовать читать себя всем и каждому я бы не стал: всё-таки я пишу скорее не man'ы, а романы, и рассчитаны они на тех, кто может отличить беллетристику от документалистики. А в-третьих и главных, последняя из моих книжек была издана в 2006 году. А в динамично развивающемся IT-мире пять лет~--- срок изрядный. Впрочем, к этому вопросу мы ещё вернёмся.

Ну ладно, оставляю себя, любимого, и начинаю вспоминать, что же я ещё рекомендовал в аналогичных случаях? В произвольном порядке из подсознания всплывают:

\begin{itemize}
	\item Кай Петцке, \textit{Линукс: от понимания к применению};
	\item Мэтт Уэлш сотоварищи, \textit{Запускаем Линукс}; 
	\item Виктор Костромин, \textit{LINUX для пользователя}; 
	\item Джеймс Армстронг, \textit{Секреты UNIX}. 
\end{itemize}


Поскольку ложная скромность никогда не была в числе моих многочисленных недостатков, рискну в этот список добавить и свой \textit{Доступный UNIX} (с указанной выше оговоркой). А вот свой 
\textit{Гуманистический Linux}~--- не включил бы. И не из скромности, опять-таки, а потому что это книжка совсем иного жанра.

Специфика всех книжек моего <<золотого списка>> в том, что они в принципе рассчитаны на совсем неподготовленного пользователя, так называемого ньюба. И в то же время это не книжки из серии \textit{``\dotsдля чайников''} или \textit{``Освой~\dots~за 5 минут''}. Нет, они требуют вдумчивого чтения, и за пять минут их <<ниасилить>>. Они предназначены для тех начинающих пользователей, которые как можно скорее хотят избавиться от своего <<начинального>> статуса и перейти в категорию пользователей действующих. Говоря словами сэра Артура Конана Дойла, они предназначены


\begin{shadequote}{}
Мальчикам, которые уже наполовину мужчины, и мужчинам, которые ещё наполовину мальчики.
\end{shadequote}

Причем предназначены они именно для пользователей~--- грезящим о карьере сисадмина или системного UNIX-программера следует читать более иные книжки. Первым, например, любую из книжек Эви Немет, вторым~--- 
\textit{UNIX изнутри}
 Юреша Вахалии и (или) 
\textit{Современные операционные системы}
 Эндрю Танненбаума.

Маленькое отступление: меня всегда умиляет, когда на форумах книжки Эви Немет рекомендуют по всякому поводу и без всякого повода. Не потому что они плохи, напротив. Просто в них содержится масса материала, абсолютно ненужного конечному пользователю. И в то же время нет очень многого, что ему жизненно необходимо.

Однако вернёмся к нашему (пардон, моему) <<Золотому списку>>. Сверяю годы издания входящих в него книжек. Получается соответственно:

\begin{itemize}
	\item 2000
	\item 2000
	\item 2002
	\item 2001
\end{itemize}



Да, кажется, я что-то упустил в этой жизни. Ведь наверняка за минувшее десятилетие было издано что-то ещё, кроме меня. Так что отправляюсь на поиски по онлайновым книжным магазинам. И что же там вижу?

А в том-то и дело, что в рамках поставленной задачи~--- практически ничего нового не вижу. Есть переиздание книжки Стахнова 
\textit{Linux}
~--- первое издание я не читал, но просматривал: оно не столько о Linux вообще, сколько конкретно о тогдашнем Red Hat. Так что нынешнее, подозреваю, в основном о Fedora. И изобилие книжек, посвящённых популярным (или активно популяризируемым) дистрибутивам, в первую очередь, конечно же, Ubuntu, далее, с большим отрывом~--- Mandriva, ну а всё остальное~--- мелочи.

Есть книжки, как отечественные, так и переводные, преследующие ограниченный круг задач, типа: 99 советов по Linux или 100\% самоучитель Linux. А есть, наоборот, книги, претендующие на всеохватность, подобно 
\textit{Всему Linux}
 и 
\textit{Полному руководству}
 Михаэля Кофлера (подозреваю, что это одна и та же книга, изданная в разных издательствах под разными названиями; если так~--- то это всё дериваты, идущие от начала нулевых, как минимум).

Наконец, есть многие множества книжек Колисниченко на самые разные темы~--- от новичка к профессионалу в любой системе, включая FreeBSD. И в любой сфере приложения сил~--- от shell-программирования до администрирования серверов.

Нет в нашем печальном списке только одного: свежих книжек, подобных тем, что я перечислил в <<списке золотом>>.

Хотя нет, ура! Вижу два приятных сюрприза~--- переиздание 
\textit{Запускаем Linux}
, теперь за авторством Дальхаймера и Уэлша, и 
\textit{Самоучитель. Linux для пользователя}
 Виктора Костромина, 2008 и 2011 года издания, соответственно. Однако при внимательном рассмотрении всё оказывается не так уж и замечательно. Первая книга~--- перевод достаточно старого 5-го издания (в оригинале с тех пор вышло как минимум 6-е). А вторая~--- вообще просто допечатка тиража одноимённой книжки десятилетней давности, той самой, что была включена в указанный выше список. Утверждаю это со всей ответственностью, потому что из общения с Виктором точно знаю, что он второго, модернизированного, издания не готовил.

Таким образом можно констатировать, что в рамках интересующей нас темы за последние 5 лет не было издано

\begin{itemize}
	\item ни одной новой книжки отечественных авторов; 
	\item ни одной новой переводной книжки; 
	\item ни одной <<осовремененной>> редакции книжек из моего <<золотого списка>> (как я уже говорил книжки Уэлша с Дальхаймеров и Костромина не в счёт). 
\end{itemize}



В то же время было издано достаточно большое количество книжек категории <<для чайников>> и дистрибутив-специфических руководств. Не могу утверждать, что они плохие, ибо ни одной не читал. Но дело в том, что они, в отличие от аналогичных книжек Windows-тематики, практически не имеют своего читателя. Ибо книжки тематики UNIX/Linux читают те, кто нуждается в углублённых знаниях. А минимума, необходимого для элементарного практического использования, легко нахватать в Интернете.

Для очистки совести я зашёл в свой близлежащий оффлайновый магазин, который являет собой нечто вроде слепка московского книжного рынка~--- в нём не найти раритетов, но и лавкой по распродаже дамских детективов он также не является.

Раньше в нём была цельная полка, заставленная книжками про UNIX, Linux и современных материи самой разной ориентации~--- от меня до Танненбаума. Ныне же я не обнаружил там ни одной книжки <<нашего круга>> вообще. Правда, в более <<масштабных>> оффлайновых магазинах (а их на всю Москву, грубо говоря, четыре) таковые обнаружить можно~--- но опять-таки в реализации <<для чайников>>. И ажиотажа вокруг них не наблюдается.

То есть мы имеем ситуацию, описанную \href{http://expert.ru/2011/11/15/radostnyij-vzlet-tsunami/}{Еленой Прудниковой}  и \href{http://gistoria.info/?p=160}{мной}  для литературы художественной:

\begin{itemize}
	\item книжки <<для чайников>> не читают, потому что Linux-чайники вообще не читают книжек; 
	\item ни одной новой переводной книжки; 
	\item ни одной <<осовремененной>> редакции книжек из моего <<золотого списка>> (как я уже говорил книжки Уэлша с Дальхаймеров и Костромина не в счёт). 
\end{itemize}

Книжки, аналогичные таковым из <<золотого списка>>, не читают, потому что их нынче просто нет. 

Надо учесть ещё, что из пяти причин <<нечтения>>, рассмотренных в статьях по указанным ссылкам, первая~--- завышенность цены~--- для компьютерной литературы вообще много более весома, нежели для литературы художественной: компьютерная книжка стоимостью от тысячи и более рублей на полках оффлайновых магазинов и в прайсах магазинов сетевых~--- дело обычное.

И не надо думать, что дело тут в жадности книгоиздателей. Копаясь в сети при подготовке этой заметки, в одном онлайновом магазине, не из последних, я обнаружил свою книжку 
\textit{Гуманистический Linux или Ubuntu и сородичи}  по цене\dots не буду называть абсолютных цифр, скажу только, что она втрое превышала отпускную цену издательства. Которую, как вы понимаете, я знаю абсолютно точно~--- мне с неё ройялти платят.

Так что обвинения авторов и издателей в жадности ни на чём не основано: ни тем, ни другим с этой книготорговой накрутки не перепадает ни копейки. Хотя именно они несут основные затраты~--- моральные и материальные, соответственно. Затраты же онлайновых книготорговцев минимальны. Страшно даже представить, сколько эта книжка стоила в магазине оффлайновом, которому надо ещё отбивать аренду торговых площадей.

В общем, сочетание завышенной цены с проблематичностью оправданности ожиданий в отношении содержания приводит к тому, что Linux-книжки не читают. Тем более, что фактор конкуренции со стороны онлайновых материалов тут ещё больше, чем в литературе художественной. Причём ни о какой контрафактности их не может быть и речи: все они распространяются абсолютно свободно, под рядом свободных лицензий типа 
\textit{Creative Commons} и, главное, под лицензией \textit{Человеческой Порядочности и Здравого Смысла}
.

Так что вымирание компьютерной литературы вообще и книжек тематики UNIX/Linux будет идти и дальше по нарастающей. Причём, в отличие от художественной литературы, никаких путей к исправлению ситуации не видно. Если бумажное производство изящной словесности может сколь угодно долго существовать на произведениях, сочинявшихся веками, то в нашей сфере самая-рассамая классика жанра нуждается в постоянном обновлении.

А для этого авторы книжек должны быть заинтересованы в обновлении, иногда коренном, своих старых книжек и в сочинении новых. А такой заинтересованности нет~--- ни материальной, ни моральной. Материальной~--- потому что ввиду маленьких тиражей ройялти автора составляет сущие гроши. Моральной~--- потому что книжки становятся никому не интересными. И большинство линуксописателей либо бросают это занятие вообще, либо, если они уже стали сочинителями-наркоманами, развивают собственные онлайновые проекты.

А теперь вернёмся к началу заметки и попробуем ответить на вопрос: что же читать начинающему линуксоиду, если новых ярких книжек по этой тематике мы уже никогда не увидим? Ответ прост~--- читайте всё те же книжки из <<золотого списка>>. Да, многое в них потеряло актуальность. Многие новшества, появившиеся в последние годы, в них не описаны. Но специфика UNIX и Linux такова, что их непреходящие ценности не подвержены старению. И книжка Кернигана и Пайка 
\textit{UNIX: универсальная среда программирования}
, впервые изданная на русском языке в 1992 году, в этом отношении не менее актуальна, чем 20 лет назад. Не случайно она была издана в 2003 году в новом переводе с того же оригинального издания 1984 года (под названием \textit{Unix. Программное окружение}).
\chapter{События FOSS-мира}

\section{Девять дней, которые потрясли Open Source} 
\begin{timeline}
14 ноября 2006
\end{timeline}

\hfill \begin{minipage}[h]{0.45\textwidth}
— У меня есть все основания думать, что я и один справлюсь со своим делом.

— \dotsВ таком случае\dots у меня есть не меньшие основания предполагать, что и я один смогу справиться с вашим делом.
\begin{flushright}
\textit{Ильф и Петров, Двенадцать стульев}
\end{flushright}
\bigskip\end{minipage}

Как известно, большевикам, чтобы потрясти устои старого мира, потребовалось десять дней. По крайней мере, таковы были результаты подсчетов американского писателя Джона Рида, очевидца, а возможно, и участника тех событий, сочинившего по сему поводу соответствующую книгу, которая так и называется: <<Десять дней, которые потрясли мир>>.

Для того, чтобы потрясти мир Open Source, достаточно оказалось 9 дней. Которые уложились в интервал с 25 октября по 2 ноября. Впрочем, точными вычислениями с учетом времени суток и часовых поясов я не занимался, вполне возможно, что хватило всего восьми астрономических дней\dots

Итак, первый из этих дней выпал на 25 октября, когда компания Oracle объявила о выпуске собственного дистрибутива, предназначенного для работы с её собственной же СУБД. Впрочем, если принадлежность СУБД Oracle одноименной фирме никем не оспаривается, то относительно дистрибутива возникают вполне обоснованные сомнения. Ибо являет он собой ни что иное, как пересобранный и несколько разгруженный Red Hat.

Как известно, после расщепления Red Hat на ветвь чистого Open Source~--- Fedora Core, и корпоративный RHEL, полностью свободное распространение последнего прекратилось. Разумеется, дистрибутив этот, в соответствии с лицензией GPL, по прежнему можно безвозмездно (то есть даром) скачать с официальных серверов~--- в том числе и в виде исходников. Из которых, впрочем, можно собрать точный функциональный аналог <<настоящего>> RHEL~--- только без фирменной техподдержки. Процедура сборки таких клонов и их распространение, в соответствие с буквой и духом GPL, абсолютно законна, чем уже давно пользуются майнтайнеры, выпуская своего рода <<нишевые>> продукты~--- Scientific Linux, ориентированный на использование в сфере науки и образования, и CentOS~--- своего рода <<RHEL для бедных>>, то есть всякого рода малого бизнеса и пользователей-индивидуалов.

Так что, казалось бы, и Oracle не свершила ничего крамольного, просто использовав возможность разжиться готовой системой на законных основаниях. Почему же её действия получили столь громкий резонанс в кругах, тем или иным образом связанных с движением Open Source?

<<Что позволено Юпитеру, не позволено быку>>~--- говаривали древнеримские греки. К нашему случаю эта фраза применима в инвертированном виде: что вполне естественно для <<бюджетных>> майнтайнеров, выглядит несколько странно в исполнении гиганта софтверной индустрии. От которого можно было бы ожидать если не создания дистрибутива <<с нуля>> (в наши дни это просто нецелесообразно), то по крайней мере существенной доработки существующего. Ну и вообще обратной отдачи на благо сообщества.

Нельзя сказать, что Oracle совсем не приложили руку к своему дистрибутиву, получившему имя 
\textit{Unbreakable Linux}
 (от перевода воздержусь~--- смысл интуитивно понятен, а точный эквивалент подобрать затрудняюсь). Но приложение это было, так сказать, со знаком <<минус>>: из их фирменной сборки был изъят ряд компонентов, в частности, <<конкурирующие>> программы~--- Postgres и MySQL. Впрочем, ознакомиться с этим дистрибутивом <<вживе>> мне пока не удалось. Хотя он и доступен для свободного скачивания, однако ему предшествует очень нудная процедура регистрации.

Предлагает Oracle и техническую поддержку для корпоративных пользователей~--- причем по ценам, чуть ли не вдвое более низким, чем Red Hat.

Собственно, последнее и было вторым фактом, вызвавшим возмущение общественности, каковая усмотрела желание <<подсидеть>> фирму-производителя родительского дистрибутива. Не случайно ответным шагом компании Red Hat стал лозунг~--- 
\textit{Unfakeable Linux}
 (в переводе, думаю, также не нуждается). Хотя мне первый фактор~--- то есть отказ от собственного вклада в развитие Open Source,~--- представлялся более важным.

Однако, когда отгорели костры первых эмоций, настало время задуматься: а так ли все суицидально, как показалось сначала? По здравом размышлении, можно представить себе два сюжета развития дальнейших событий. Первый вытекает из факта предоставления Oracle технической поддержки своего дистрибутива в целом. А это неизбежно вызовет создание соответствующей инфраструктуры и её развитие, в том числе и финансирование разработчиков открытого софта примерно в тех же формах, как это ныне делают Red Hat и Novell. То есть, в конечном счете, того самого обратного вклада в движение Open Source, которым не пренебрег еще ни один коммерческий пользователь его продукции. И в итоге на поле открытого софта появится просто еще один корпоративный игрок~--- и игрок далеко не последней категории.

Второй сюжет~--- это превращение <<Несгибаемого Linux'а>> от Oracle просто в довесок к их же СУБД, своего рода <<стартер>> для её запуска и среду функционирования. И в этом случае мы получаем просто еще один <<нишевый>> продукт~--- правда, ниша его оказывается ох какой глубокой (в финансовом выражении). Но на развитии Open Source это скажется, по моему, весьма мало~--- как мало заметно влияние на него Linux'ов для встроенных устройств и тому подобных узко специализированных систем.

Другой вопрос, как тот или иной сценарий скажутся на положении старейшего коммерческого Linux-дистрибьютора~--- компании Red Hat. В первом случае она оказывается в положении Кисы Воробьянинова из процитированного в эпиграфе диалога. И, вероятно, ей не останется иного выхода, кроме как, подобно Ипполиту Матвеевичу, в той или иной форме вступить в число пайщиков-концессионеров.

При <<нишевом>> развитии событий Red Hat тоже ожидает не так уж много хорошего. По данным от Линуксцентра~--- крупнейшего онлайнового магазина России, распространяющего дистрибутивы, до 80\% покупок RHEL (вкупе с фирменной технической поддержкой) выполняется с целью обеспечения работы Oracle. Ведь именно RHEL долгое время был одной из двух официально сертифицированных Linux-платформ для этой СУБД. Но тут у Red Hat возникает возможность <<обратиться к истокам>>~--- и вплотную заняться внедрением Linux в десктопную сферу. Не возьмусь судить, насколько это может быть коммерчески выгодно~--- но вот польза сообществу Open Source от этого была бы несомненно.

Так что первое потрясение Open Source при ближайшем рассмотрении оказывается не столь уж фатальным~--- хотя на мир свободного софта оно влияние окажет безусловно~--- и чуть позже я еще вернусь к вопросу, как именно это влияние может проявиться. А вот масштабы второго пока оценить трудно.

Второе потрясение оказалось приуроченным ко второму ноября, когда компания Novell~--- уже более двух лет являющаяся владельцем второго из <<корпоративных>> дистрибутивов Linux, Suse,~--- и корпорация Microsoft объявили о начале сотрудничества в технической, маркетинговой и патентной сферах.

Ну, с техническим сотрудничеством все более или менее понятно: оно направлено на обеспечение совместимости Windows и Linux (точнее, конкретно SLES) в гетерогенных средах, совместимости форматов документов (впрочем, далеко не все из них фигурируют в опубликованных материалах), увязку фирменных служб каталогов (ActiveDirectory и eDirectory).

Маркетинговое сотрудничество также вопросов не вызывает: отныне Microsoft <<дает добро>> тем своим пользователям, которые нуждаются еще и в Linux'е, на применение SLES от Novell. Конкретно этот <<одобрямс>> пока выражается в том, что Microsoft будет распространять купоны на техническую поддержку SLES~--- как это будет выглядеть в реале, я, честно говоря, представляю с трудом.

Так что наибольшее внимание широких народных масс привлекла патентная сторона соглашения. Согласно которой Microsoft предоставляет своего рода индульгенцию разработчикам Novell на использование в Linux'е своих патентованных технологий. А также дает иммунитет пользователям Linux-дистрибутивов Novell от судебных преследований по патентным искам. Более того, индульгенция распространяется, как будто бы, и на независимых разработчиков открытого софта, не используемого в коммерческих целях.

Все это, конечно, очень благородно и должно только приветствоваться. Вот только о том, что патентованные технологии Microsoft тем самым станут открытыми в плане доступности кода и свободными с точки зрения условий распространения, не говорится ни слова. И потому сразу же возникает тот же самый вопрос, который встал после предположения Эрика Реймонда о возможности включения в ядро Linux фрагментов проприетарного кода, в частности, драйверов устройств.

С этим вопросом тесно связан и другой момент, который можно назвать идеологическим или, если угодно, психологическим: не есть ли допуск разработчиков Open Source к патентованным технологиям попыткой вовлечения их в дебри проприетаризма? И не будет ли в дальнейшем найдена юридическая лазейка для требования оплаты патентных отчислений? Или, того паче, для <<прихватизации>> свободного кода, созданного с использование кода патентованного (или тесно с ним интегрированного). Существующие законы о патентном и авторском праве, а также свободные, GPL-совместимые, лицензии, как будто бы не могут такого допустить. Однако, зная изощренность Microsft в юридическом крючкотворстве, помноженную на финансовую мощь корпорации, кто рискнет утверждать невозможность подобного исхода?

И если такой исход будет иметь место~--- последствия его предсказуемы с трудом. Опыт давнишней истории с System V и BSD учит, что избавиться от самой малости проприетарного кода гораздо сложнее, чем его включить. И это в свое время сыграло роковую роль в судьбе всех BSD-систем вообще и FreeBSD в частности.

Кстати: в ответ на соглашение Microsoft и Novell компания Red Hat в буквальном смысле грудью бросается на амбразуру дзота. Предлагая своим клиентам, в случае патентных претензий третьей стороны (предлагается угадать с трех раз, какой именно), <<встать на их место>> и принять удар на себя. Юридически или технологически~--- это другой вопрос. Судя по всему, предполагается переписывать код, вызвавший патентные трения. Вот только насколько легко это будет сделать? И не окажется ли это той самой соломинкой, которая ломает хребет верблюду?

Есть и еще один осложняющий фактор. Не секрет, что многие разработчики Open Source давно уже не являются энтузиастами-любителями, сочиняющими код в качестве хобби (вспоминается фраза из давнишней статьи в ныне не существующем компьютерном еженедельнике <<Софтмаркет>>: <<В свободное от службы время поручик был программером и сочинял разные программки>>). Нет, многие из них состоят в штате IT-компаний и получают зарплату именно за разработку открытого софта. В частности, кое-кто из ключевых разработчиков ядра Linux расписывается в платежных ведомостях компаний Novell и Red Hat. И как они поведут себя в том случае, если встраивание закрытого кода в ядро будет их служебным заданием? Если положительно~--- то корпорации, выплачивающие им зарплату, вполне могут полагать, что подобно сыну турецкоподданного, и сами прекрасно справятся с делом дальнейшего развития Linux.

С другой стороны, те же самые ключевые разработчики ядра не только непосредственно программируют сами~--- кроме того, они еще и аккумулируют код, написанный множеством независимых разработчиков, которые получают зарплату совсем в других местах и совсем за другое (тут на память приходит Кон Коливас, врач-анестезиолог~--- но имя таким разработчикам легион). И возможность такой аккумуляции зиждилась и зиждется исключительно на доверии: на убежденности широких программирующих масс в том, что ключевые разработчики действуют в интересах сообщества, а не той или иной корпорации. Если это доверие будет поколеблено (обоснованно или необоснованно~--- это другой вопрос), распадется вся <<вертикаль власти>>, связывающая сообщество. И последствия этого могут быть весьма печальными. В частности, многочисленные форки ядра Linux, что, при наложении на уже имеющееся изобилие его дистрибутивов, даст картину воистину апокалиптическую. В лучшем случае, это будет откат <<чистого>> Open Source на позиции до 1999 года, когда в прессе и народе впервые заговорили о Linux-буме. 

Да и <<коммерческому>> Linux'у это счастья не принесет~--- база его разработчиков резко сузится. Причем за счет наиболее квалифицированных энтузиастов~--- и наибольших энтузиастов среди квалифицированных.

И, наконец, нельзя забывать о <<несгибаемых>>~--- Ричарде Столлмене, GNU/FSF и разработчиках Debian. Которые заведомо не пойдут на компромисс ни с какими проявлениями проприетаризма. И они могут оказаться третьим центром кристаллизации среды Open Source.

В общем, не буду выступать в роли пророка. Одно ясно~--- после этих <<Девяти дней>> мир Open Source изменится~--- и весьма сильно. К добру это будет, или к худу, мы узнаем уже через несколько месяцев\dots

\section{Урожай сенсаций} 
\begin{timeline}18 Сентябрь 2007 г\end{timeline}

Осень~--- пора сбора урожая, в том числе и урожая сенсаций. Вспомним осень минувшую, когда бурно обсуждались сначала альянс Novell и Microsoft, а потом~--- антитеза Oracle vs Red Hat.

Но похоже, что нынешняя осень будет особенно урожайной на сенсации. Судите сами: на дворе еще только сентябрь, а уже два неожиданных известия дают повод для обсуждения.

Во-первых, это объявление об открытии исходных текстов ОС реального времени QNX. Во-вторых~--- сообщение о банкротстве SCO Group, фирмы, пусть и одиозной в глазах всех сторонников Open Source, но являвшейся не последним игроком в ряду Unix-ориентированных компаний.

До конца осени еще далеко. Что ждет нас впереди?

\section{QNX, открой личико!} 
\begin{timeline}18 Сентябрь 2007 г\end{timeline}
QNX\dots Таинственная, несмотря на прекрасную документированность (в том числе и на русском) система, о которой многие слышали, но мало кто видел. Микроядерная ОС реального времени, с собственным, весьма своеобразным, графическим интерфейсом, именуемым Photon (и действительно работающим с субсветовой скоростью). Система промышленного назначения, вовсе не рассчитанная на десктопы (хотя при определенных условиях могла бы использоваться и в этом качестве). Которой приписывают управление ядерными реакторами и ракетными комплексами. Быстрая и компактная. Бесплатная при индивидуальном применении и более чем дорогая при использовании коммерческом. И~--- до недавнего времени~--- система сугубо закрытая.

И вот~--- сенсация: производители~--- а в этом качестве традиционно рассматривается фирма QNX Software Systems, которая с некоторых пор (точнее, вот уже почти три года) фактически принадлежит автомобильному концерну Harman\dots Так вот, производители начали поэтапное открытие исходных текстов QNX.

Первым этапом было предоставление доступа к исходным текстам знаменитого микроядра QNX Neutrino, главной библиотеки C, некоторых драйверов для взаимодействия оборудования. Получить это богачество можно уже сейчас.

В последующем планируется обречь этой же участи исходники графического интерфейса пользователя~--- не менее знаменитого Photon'а, файловой системы, базовых утилит. После чего эту систему можно будет считать почти столь же открытой, как Linux или любую BSD.

Почти~--- потому что лицензия, под которой открываются исходники, весьма своеобразна. Это не BSD-лицензия, и тем более не GPL любого рода. Имя ей~--- QNX hybrid software model. На деталях её пока задерживаться не буду. Замечу только, что она напомнила мне лицензию, под которой Trolltech распространяет Qt: возможность сторонним разработчикам вносить изменения в код, бесплатность для некоммерческого использования и платность~--- для использования коммерческого. Причем разработчики не обязаны делать достоянием общественности свои достижения, а вполне могут сохранять их в составе закрытых систем (и здесь мы видим влияние скорее лицензии BSD, нежели GPL).

О причинах, толкнувших на такой шаг разработчиков, не берусь даже гадать. А вот о его возможных последствиях порассуждать интересно.

В первую очередь, каких последствий точно не воспоследует.

А именно, не следует ожидать, что разработчики Open Source всё бросят и кинуться писать драйверы для QNX, призванные поддерживать все изобилие PC'шного оборудования. То есть <<десктопизации>> этой ОС не произойдет. Хотя приток независимых разработчиков в областях традиционного использования QNXС, конечно будет~--- вероятно, это и есть один из резонов производителя открыть исходники.

Во-вторых, наивно было бы ожидать и того, что конечные пользователи Linux или BSD будут сносить свои годами проверенные и привычные системы, устанавливая вместо них (или даже вместе с ними) QNX.

В-третьих, не следует думать, что своим актом разработчики превратили QNX в систему открытую и свободную (в понимании ли FSF или движения Open Source). Нет, она остается собственностью соответствующей компании (а в конечно счете, вероятно, концерна Harman). И последней вольно как открыть исходники, так и закрыть их~--- если нынешнее мероприятие почему-либо не оправдает ожиданий собственника. В 90-х годах это проделывали и Sun со своим Solaris'ом, и DEC с True64 Unix (хотя исходники последней закрывал, кажется, уже Compaq).

Нет, значение открытия исходников QNX видится мне в другом. А именно~--- во взаимном обмене идеями. Как известно, все хорошее, что появляется в Linux'а, очень быстро перекочевывает в BSD-системы, и наоборот. Так что теперь и многое хорошее из QNX теоретически может перекочевать в открытые Unix-подобные системы. И не обязательно на уровне кода~--- возможно, на уровне идей.

А что хороших идей в QNX заложено немало~--- думаю, спорить не будет никто из тех, кто хоть раз видел эту систему. Одна идея сверхлегкого и сверхбыстрого Photon'а дорого стоит. Кто знает, а вдруг она найдет свое предназначение, например, в Minix3? Или~--- в DragonFlyBSD? Ведь из всех монолитно-ядерных систем она идеологически наиболее близка <<микроядерщикам>>.

\section{История одного банкротства. Конец SCO?} 
\begin{timeline}18 сентября 2007 г\end{timeline}

\hfill \begin{minipage}[h]{0.45\textwidth}
This is the end of Solomon Grundy
\begin{flushright}
\textit{Древнеанглийский стишок}
\end{flushright}
\bigskip\end{minipage}

Очередная сенсация~--- объявление о банкротстве SCO согласно главе 11 Кодекса о банкротстве США. Это не значит, что деятельность компании прекращается, а имущество её идет с молотка. Глава 11 предусматривает для фирмы, объявившей себя банкротом, нечто вроде финансовой <<передышки>>, в частности, запрещая кредиторам на опредленный срок взыскивать с нее долги (с деталями можно ознакомиться здесь).

Эта новость на форумах тематики Open Source обсуждается повсеместно, активно и злорадно. Действительно, компания SCO <<прославилась>> своей тяжбой с IBM относительно незаконного раскрытия той использования исходного кода UNIX, а в дальнейшем~--- откровенным шантажом майнтайнеров Linux-дистрибутивов и их пользователей. Чем и снискала законную <<любовь>> всех сторонников свободного софта. Однако в этих обсуждения проскальзывает немало неточностей, во многом обусловленных запутанностью наименований, которые я и хотел бы прояснить в этой заметке. 

Для начала вкратце напомню историю вопроса. Потом посмотрим, может ли это событие быть основанием для радости (и тем более злорадства). А под конец попробуем извлечь мораль из всей этой истории.

Собственно, компания Santa Cruz Operations (известная также под аббревиатурой SCO~--- к современной SCO Group она имеет косвенное отношение)~--- одна из первых фирм, которая начала выпускать UNIX для PC, практически сразу после появления i386 (другой такой фирмой была, как ни странно, Microsoft). Название её происходит от местечка Санта Крус (Калифорния), где располагалась её штаб-квартира.

Разрабатывавшаяся в SCO система изначально называлась SCO UNIX (ныне она известна под именем SCO OpenServer). Она никогда не блистала технологическими изысками и новинками, но считалась очень устойчивой и надежной. И потому получила довольно широкое распространение в банковской сфере. В частности, начиная с прошлого века, версии этой ОС на протяжении многих лет используются в Российском Сбербанке, где под нее было написано немало специализированного софта (например, система коммунальных платежей).

Это одна линия истории. Другая же связана с именем Novell. В первой половине 90- годов проглого века эта компания была одержима хватательным рвением~--- ею были приобретены DR DOS, WordPerfect, QuattroPro, и в числе прочего~--- также права на исходный код и торговую марку UNIX (оставшиеся как бы бесхозными после разделения их прародителя~--- AT\&T). На базе чего была создана система UNIXWare.

>>Офисное>> направление деятельности Novell особого успеха не имело, как и её UNIX-бизнес. В результате один из основателей фирмы, Рэй Нурда (Ray Noorda), ушел в оставку, а все новоприобретения~--- распроданы в розницу: офисные пакеты~--- фирме Corel, где они составили пакет Corel Office, агония которого продолжается чуть не по сей день, UNIXWare~--- SCO, дополнив её собственную систему, получившую отныне имя SCO OpenServer.

Рэй же Нурда основал новую компанию~--- Caldera Systems, которая, в частности, приобрела у Novell DR DOS. Однако основным направлением её деятельности стала Linux-дистрибуция. Первоначально Caldera Linux представляла собой цельнотянутый Red Hat, дополненный некоторыми не вполне свободными компонентами, типа рабочего стола и средств интеграции с сетями Novell Netware. Однако вскоре этот дистрибутив обзавелся собственным инсталлятором (ИМХО, одним из лучших графических инсталляторов своего времени), средой Caldera Network Desktop. И вообще приобрел своеобразие, отмеченное именем Caldera OpenLinux.

Свободная версия Caldera OpenLinux~--- представляла собой очень компактный, но аккуратно укомплектованный дистрибутив, широкому использованию которого препятствовали только слабые средства интернационализации. А коммерческая версия включала в себя немало проприетарных продуктов, вплоть до Wabi (средства эмуляции Windows от фирмы Sun), WordPerfect и CorelDRAW~--- квази-портов (квази~--- потому что на самом деле они работали в режиме эмуляции), StarOffice, и так далее.

Наконец, Caldera становится одним из соучредителей альянса United Linux~--- наряду с Suse (тогда это была Европа), Turbolinux (Япония) и Conectiva (Бразилия); таким образом, альянс этот охватил чуть не все континенты (африканской Ubuntu тогда еще и в проекте не было).

Целью альянса была стандартизация дистрибутивов Linux. Тем не менее, никакими славными делами он не отметился, тихо и незаметно прекратив свое существование года через два. Что и знаменовало собой начало конца.

Однако это еще впереди. А пока, на рубеже тысячелетий, Caldera выглядит процветающей и преуспевающей фирмой. Тогда как у SCO, несмотря на наличие такого могучего клиента, как Сбербанк, дела, судя по всему, идут не блестяще: видимо, две UNIX-системы оказались ей не под силу. И она продает оба своих флагманских продукта~--- SCO OpenServer и SCO UNIXWare~--- фирме Caldera.

Та же, в свою очередь, по-видимому, испытывает разочарование в своем Linux-бизнесе: первый Linux-бум рубежа тысячелетий схлынул. И Caldera начинает переориентироваться на UNIX-бизнес~--- во главу угла ставятся новоприобретенные SCO OpenServer и SCO UNIXWare. В соответствие с чем меняется и название~--- отныне фирма называется SCO Group, и SCO здесь уже условное буквосочетание, а не аббревиатура (базируясь в штате Юта, она ни малейшего отношения к местечку Санта Крус не имеет).

Прежняя же SCO, оставшись без своих титульных систем, переименовывается по названию последнего своего продукта~--- Tarantella. Впрочем, жизнь её была недолгой: остатки древней Santa Crus Operations были приобретены Sun'ом, и ныне это просто подразделение последнего.

Однако вернемся к героине нынешней сенсации~--- SCO Group (далее вторую часть названия я буду опускать, но прошу помнить, что к почившей в Бозе Santa Cruz Operations она имеет очень опосредованное отношение).

Непосредственно после приобретения двух новых UNIX'ов развитие Caldera OpenLinux прекратилось: последняя её версия датируется концом 2002 года (и выступает уже под именем SCO Linux). Однако, судя по всему, и UNIX-бизнес у свеже-переименованной компании тоже не особо заладился. И она предпринимает отчаянный шаг~--- начинает тяжбу с IBM о нарушении той условий лицензирования исходного кода UNIX.

Нормальному советскому человеку понять суть претензий SCO довольно сложно; да и антисоветскому человеку~--- тоже, если он в должной мере не владеет юридическим английским. В меру своего понимания попробую их изложить:

IBM, будучи обладателем лицензии на исходники первозданного UNIX (полученной в лохматые годы еще от AT\&T), выложила часть их в открытый доступ, откуда они были заимствованы разработчиками ядра Linux. Тем самым потенциально подвергнуться судебному преследования могли и разработчики этой ОС, и майнтайнеры дистрибутивов, и даже конечные пользователи (в первую очередь, конечно, корпоративные).

Правда, аргументы, выдвинутые SCO в подтверждение своих притязаний, смотрелись смехотворно: это был исходный текст функции \texttt{malloc}, опубликованный в открытой (<<бумажной>>) печати отцами-основателями UNIX еще во времена былинные. Да и вообще весь иск в глазах любого здравомыслящего человека выглядел смешным. Однако сутяжничество~--- американский национальный вид спорта, и потому процесс продолжался около двух лет.

Правда, SCO тут же нарвалась на встречный план (пардон, иск) со стороны Novell. Которая, со своей стороны, оспорила права SCO и на исходный текст первозданного UNIX, и на саму торговую марку.

Перипетии обоих процессов пересказывать не буду. Во-первых, о них не писали только те СМИ, обозреватели новостей в которых страдали патологической ленью. Во-вторых, оба процесса SCO в первом приближении проиграла. В первом приближении~--- потому что теоретически, по правилам этого спортивного состязания, может оспорить оба судебных судебных решения. Насколько в нынешних условиях она имеет возможность сделать это практически~--- вопрос.

И вот~--- закономерный итог всех предшествующих событий, сообщение о банкротстве. Как я уже говорил, в соответствие с главой 11 американского кодекса, это вовсе не значит, что SCO немедленно прекращает своё существование. Так что в строке из эпиграфа к заметке стоило бы поставить вопросительный знак, однако в оригинале его нет, а мы ведь свято чтим настоящее авторское право~--- даже если оно принадлежит народу.

Финансовая <<передышка>> часто позволяет обанкротившимся компаниям выпутаться из трудной ситуации~--- примером чему может служить SGI. Сможет ли это сделать SCO~--- поживем, увидим. Но мне представляется, что в любом случае с юридическими претензиями её на UNIX и Linux можно считать поконченным~--- <<не до грибов, Петька>>, как сказал Василий Иванович в ответ на сообщение~--- <<Белые в лесу>>.

Так что же, следует радоваться и кидать в воздух шапки? Ведь бабло (пардон, добро) в очередной раз победило зло, и восторжествовала справедливость. Не будем спешить.

Во-первых, радоваться несчастью ближнего, даже если он сам был его кузнецом,~--- не в нашем, товарищи, духе. А во-вторых, окончательное банкротство SCO может иметь негативные последствия, которые не в последнюю очередь затронут Россию.

Вспомним, что весь Российский Сбербанк работает под управление SCO UNIX. Более того, в конце июля текущего года Сбербанк объявил о результатах тендера по тотальному обновлению системного программного обеспечения до версии OpenServer 6. Он был выигран Премьер-Партнером SCO в России~--- компанией Бизнес-Консоль.

Впрочем, широкие массы IT-общественности, как и положено, узнали об этом из сообщения буржуазной прессы, датированного 30 августа. И успел ли состояться этот самый тотальный апгрейд~--- остается не вполне ясным. И если нет~--- то состоится ли он теперь?

Да впрочем, это и не важно. Конечно, такого рода системы способны долго существовать и без всяких апгрейдов. Однако возникает вопрос с текущей технической поддержкой. Разумеется, для Сбербанка она осуществляется не напрямую SCO, а какими-нибудь её российскими авторизованными партнерами (скорее всего, фигурирующей выше Бизнес-Консолью). Однако очевидно, что в сложившейся обстановке эти партнеры (или этот партнер) будут постепенно переориентироваться на другие сферы деятельности или просто прекратят свое существование. Да и физический износ парка компьютеров и, соответственно, необходимость в новых инсталляциях, никто еще не отменял.

Предвижу возражение: вместо апгрейда продукции SCO Сбербанк может просто сменить ОС. Да, может. Но, во-первых, это потребует времени~--- представьте себе количество его отделений по городам и весям Руси. А во-вторых, как уже упоминалось, именно под SCO UNIX написаны многие специализированные банковские системы. Конечно, их, скорее всего, можно перекомпилировать под какой-либо иной UNIX~--- но и это, с учетом отладки, не сделать в одночасье. В любом случае, процесс смены ОС не пройдет бессбойно и безболезненно.

В общем, мне уже видятся бабушки и дедушки, отстаивающие очереди в тщетной надежде получить свои пенсии, ошибки в счетах за коммунальные услуги (которые по какой-то роковой случайности никогда не бывают в пользу потребителей оных), и прочие апокалиптические картины. И все это, между прочим, в преддверии грядущих выборов. Так что от злорадства по поводу судьбы SCO я бы воздержался\dots

Остается извлечь из произошедшего события мораль, свежую и оригинальную: вот к чему может привести ориентация государственных структур на закрытые проприетарные продукты. А Сбербанк, несмотря на свое формально акционированное положение, в значительной степени остается структурой государственной, в том числе и в глазах широких масс электората.

Действительно, представим себе, что на месте SCO UNIX в банковских системах используется какой-либо коммерческий, но основанный на открытых исходниках продукт, например Red Hat или Suse. И соответствующую компанию в одночасье постигает банкротство (от чего Господь борони и ту, и другую). Что же~--- свято место пусто не бывает: немедленно будет создан форк, неотличимый от исходной системы. Теми же останутся и люди, оказывающие техподдержку. Конечно, сбои неизбежны и в этом случае~--- но масштабы их не сопоставимы с теми, которые можно ожидать при умирании продукта проприетарного.

Так может быть, произошедшее заставит наконец задуматься наши власти предержащие о роли свободного софта в современных условиях?

\section{Третья сенсация осени} 
\begin{timeline}23 октября 2007 г\end{timeline}
Начинает осуществляться предсказание, что нынешняя осень будет богата сенсациями: Microsoft предложила миру две свободные лицензии. Нет ли у нее намерения на корню скупить мир открытого софта?

Позор на мою седую репортерскую голову~--- на эту сенсацию я реагирую с изрядным запозданием. Дело в том, что новости, касающиеся Microsoft, я обычно пропускаю~--- даже если они попадаются на FOSS-ресурсах: мы с этой компанией существуем в параллельных мирах, и все, что касается Microsoft, Windows и тому подобных материй, интересует меня не больше, чем геолога, работающего в корякской тундре~--- погода где-нибудь на Багамах. Но тут, проглядывая как-то новости на нескольких сайтах, случайно наткнулся на известие о свободных лицензиях от Microsoft~--- свидетельстве того, что жители солнечной Багамии неожиданно заинтересовались жизнью тружеников тундры. А пролистав несколько сопряженных новостей, понял, что это и есть следующая сенсация текущей осени из числа предсказанных в одной из предшествующих заметок.

Давным давно, лет 10 назад, Сергей Леонов в Компьютерре написал нечто вроде следующего: я поверю в серьезность Linux'а, когда не линуксоиды будут ругать Microsoft, а Microsoft начнет ругать Linux. Момент это наступил пару-тройку лет назад~--- с публикаций о совокупной стоимости владения Linux и Windows. Разумеется, с выводами в пользу второй~--- кто же будет говорить, что у него все плохо, а рядом все хорошо.

С этого момента и можно говорить об интересе багамцев к тундровикам (пардон, Microsoft к Linux'у). Правда, поначалу он носил сугубо негативный характер, и сводился к утверждениям того, что, мол, у нас на Багамах тепло и дешево, а у вас в тундре холодно и дорого. Цифры и графики сравнения совокупной стоимости владения с web-страниц специализированных сайтов перекочевали на рекламные страницы компьютерных журналов~--- кажется, не было такого, кто бы не отметился таким образом.

Интересно, а как подобная реклама согласуется с соответствующими российскими законами? Не случайно ведь производители стиральных порошков или моющих средств, говоря о их несравненных достоинств в сравнении с конкурентами, никогда не называют конкретных имен, а используют термин <<обычный стиральный порошок>>, <<обычный подгузник>>, и так далее. Представляете себе рекламу Microsoft в таком контексте:


\begin{shadequote}{}
Совокупная стоимость владения для системы Windows на столько-то процентов ниже стоимости владения обычной операционной системой.
\end{shadequote}

Улыбает, правда?

Ну да бог с ней, стоимостью владения. Тем паче, что на стадии развития негативной ветви своего интереса к Linux'у Microsoft выдала перл не хуже. Я имею ввиду баннер с предложением дешево продать Linux-сервер.

Кстати, Microsoft, как с ней нередко бывает, в своей рекламе оказалась\dots ну скажем мягко, не совсем точна. И когда лица, заинтересованные в покупке дешевого Linux-сервера, стали обращаться по этому поводу в Российское представительство компании, выяснилось, что Linux-серверов на продажу у них и нет. Впрочем, обещание рассмотреть запросы было дано~--- правда, насколько мне известно, пока безрезультатно. Или опять Linux-серверов на всех не хватило, как сладких пряников в песне Булата Окуджавы?

Однако постепенно интерес Microsoft к Linux'у приобрел более позитивный характер. В итоге это выразилось в прошлогоденм соглашении с Novell о начале сотрудничества в технической, маркетинговой и патентной сферах. Оно вызвало бурную реакцию сообщества FOSS~--- по преимуществу отрицательную. Однако пока, по прошествии года, ничего сверхъестественно фатального в результате него не произошло. Попытки шантажа Linux-компаний угрозами патентных исков никем всерьез восприняты не были. А некоторые положительные последствия~--- например, улучшение взаимодействия между службами каталогов Active Directory и eDirectory~--- имеют место быть. Но давайте посмотрим дальше\dots

А дальше происходит вот что~--- неожиданное объявление о том, что Microsoft предложила сразу две лицензии распространения открытого софта, Microsoft Public License (Ms-PL) и Microsoft Reciprocal License (Ms-RL). И более того, обе они были признаны Open Source Initiative (OSI)~--- организацией, определяющей степень свободы лицензий. Так вот, рассмотрев предложения от Microsoft, OSI признала, что они соответствуют всем десяти критериям лицензий открытых лицензий~--- Open Source Definition. И следовательно, код, распространяемый под Ms-PL и Ms-RL, может использоваться в проектах, лицензируемых, например, под GPL, и наоборот.

Убедиться в том, что Microsoft отныне приобщилась к миру открытого софта, можно, ознакомившись с полным списком <<совместимых>> лицензий в алфавитном порядке. Хотя любопытно, что в списке лицензий по категориям ни Ms-PL, Ms-RL нет. Почему? Не успели внести? Или OSI затрудняется в определении того, к какой категории их приписать?

Интересно также, что Microsoft подала свою заявку в обычном режиме, как это делают любые другие организации и частные лица. И в первой редакции~--- как Microsoft Permissive License,~--- она была отвергнута из-за несовместимости с другими Open Source-лицензиями. И лишь вторая попытка редмонтцев внедриться в ряды последователей FOSS увенчалась успехом.

Вникать в тексты лицензий от Microsoft полагаю излишним. Уж если пуристы из OSI признали их свободными и совместимыми (в они в этом праве отказывают т.~н.~<<старой>> лицензии BSD или, например, последней редакции лицензии проекта XFree86), то простые трудящиеся, не имеющие юридического образования, вполне могут положиться на их мнение.

Интересней другой вопрос~--- а какова тайная цель, преследуемая компанией при создании своих открытых лицензий? Ведь сообщество FOSS всегда ожидает подвоха с её стороны. И, возможно, не без оснований. Ибо буквально через несколько дней Стив Балмер в интервью, данном им на конференции по Web-2, заявил о намерении его компании начать приобретение фирм Open Source направленности.

И тут поневоле приходит на память старая истина советских начальников:


\begin{shadequote}{}
Если пьянку нельзя пресечь~--- её надо организовать и возглавить.
\end{shadequote}

Не убедилась ли Microsoft в результате своих предыдущих действий, что движение FOSS ей пресечь не удастся? И значит, пора принимать меры к тому, чтобы его возглавить. Ну а потом уже организовать~--- в меру своего понимания, <<как надо>>~--- ведь тайное знание того, что именно надо конечному пользователю, во все времена отличало эту компанию. Даже тогда, когда сам пользователь еще и не подозревал о своих потребностях\dots

И в этом аспекте события последних лет укладываются в стройную и логичную цепочку. Первым звеном в которой можно считать приглашение на работу Дэниэля Роббинса~--- создателя дистрибутива Gentoo,~--- в качестве эксперта по свободному программному обеспечению (июнь 2005 года).

Не склалось~--- разрулив свои финансовые проблемы, DRobbins покинул Microsoft. Но на протяжении более чем полугода (вплоть до февраля 2006-го) ему ведь за что-то платили зарплату? Ту самую, что позволила ему избавиться от финансовых трудностей (интересно, а почему ни одна компания, бизнес которой был связан с Open Source, во времена оных не предложила ему достойной работы?). То есть~--- чего-то рассматривал, давал экспертные заключения, и так далее~--- то есть выполнял действия, вполне понятные любому, кому приходилось быть или бывать экспертом в любой сфере.

Следующее звено~--- осень 2006 года, заключение уже упоминавшегося соглашения с Novell. Которое, как я уже говорил, вопреки всем ожиданиям, ничего плохого, кроме хорошего, второй брачующейся стороне пока не принесло (за исключением, разве что, осуждения коммьюнити, на что Novell ответила прекращением контактов с ним). Что вполне укладывается в модель <<возглавить и организовать>>: заманивают. Особенно с учетом просочившихся сведений о попытках заключить аналогичное соглашение с Red Hat, Mandriva и Canonical (спонсором разработки Ubuntu)~--- попытках, успехом не увенчавшихся. Ответом на что были угрозы патентными исками к Linux-компаниям. То есть мы видим оборотную сторону любого заманивания в виде политики кнута и пряника.

И, наконец, последние события: открытые лицензии и намерения по покупке компаний FOSS-ориентации,~--- лежат все в том же русле. Сначала втереться в ряды FOSS-сообщества, с целью использования его потенциала как разработчиков, а потом~--- скупка на корню всего и вся и ликвидация движения FOSS как такового. После чего никаких угроз монополии Microsoft уже не предвидится даже в отдаленной перспективе.

Вероятен ли такой сценарий? Если обратиться к истории и проследить творческий путь компании~--- более чем. Но вот реализуется ли он? Конечно, там, у них (а теперь и у нас)~--- бездушный и бездуховный мир чистогана, где всё покупается и все продается. А поскольку все мы люди, все мы человеки, Microsoft может предложить такие условия, от которых не откажутся многие приверженцы FOSS~--- тем более, что имеет к тому все возможности.

И все же\dots Пример Дэна Роббинса показывает, что разработчику свободного софта ужиться в мире софта проприетарного не так просто. Ну а о том, что Брут-RMS никогда не продастся большевикам (пардон, проприетарщикам), думаю, не сомневается никто.

Однако старая шутка линуксоидов о скором выходе Microsoft Linux вполне может стать реальностью. Осуществится ли такое, и какие будет иметь последствия для сообщества FOSS~--- покажет время.

\section{Мир без солнца} 
\begin{timeline}Июнь 2009\end{timeline}

Разговоря о продажи фирмы Sun циркулируют в Сети давно. А ныне факт покупки её компанией Oracle можно считать почти свершившимся: юридические вопросы с иском акционеров, недополучивших, как им кажется, своего бабла, по мнению знающих людей, будут улажены легко (и очевидным для нас способом). Какие следствия для мира FOSS будет иметь исчезновение старейшей UNIX-компании? Напомню, что на её иждивении находится ряд крупных свободных проектов~--- Openoffice.org, MySQL, VitrualBox, не говоря уже о собственно ОС~--- OpenSolaris, и ряде средств разработки. Не загнутся ли они под чутким руководством Ларри Эллисона? 

Наибольшие опасения вызывает судьба OpenSolaris: а нужна ли будет Oracle ещё одна ОС, в добавление к собственному клону RHEL? ОС, за время своего <<свободного плавания>> не достигшая ни полностью работоспособного состояния, ни критической массы комьюнити? Мне кажется, что ответ будет отрицательным. Но так ли это страшно? Все здоровые инновации OpenSolaris (а их немало) могут быть легко инкорпорированы в Linux. И, чем чёрт не шутит, вдруг новые хозяева изменят лицензию на ZFS? После чего она легко впишется в Linux-ядро.

А за остальные свободные проекты Sun'а волноваться нечего: MySQL выступит <<легковеным>> дополнением к собственно Oracle, Openoffice.org не бросят, как востребованный конечным пользователем, VirtualBox, Sun Studio \textit{etc}.~--- как интересные для всех разработчиков.

И как знать, не увидим ли мы вскоре нового монополиста~--- теперь уже в сфере свободного софта? Да ещё в сцепке с собственной аппаратной платформой: не зря ведь Ларри обмолвился, что <<Sparc не бросим, потому что он хороший>>.
\chapter{FOSS и Windows}

\section{Пора ли кричать караул?} 
\begin{timeline}
2001, лето
\end{timeline}

\hfill \begin{minipage}[h]{0.45\textwidth}
Мал-мал ошибка давал~--- \\
вместо ура караул кричал
\begin{flushright}
\textit{Международная присказка}  
\end{flushright}
\bigskip\end{minipage}

\textsl{Первая версия этой статьи была опубликована на \url{www.softerra.ru} где-то году в 2001-м. Однако она кажется мне не лишенной актуальности и поныне.}\medskip

Не знаю, привлекла бы такое внимание статья Дмитрия Коровина <<Потому что без \textquotedblleftВорда\textquotedblright ни туда и ни сюда>> (Компьютерра, \textnumero25 (402) от 03.07.01, с. 42-44), если бы не комментарий к ней Сергея Леонова (как и положено, на с.1). А главное~--- не рубрика, под которой статья была опубликована. Именуемая просто и тревожно: Караул!

Впечатление по первом прочтении~--- действительно, впору <<Караул!>> кричать. Это же надо~--- всю нашу Русь сельскую, исконную, о'Windows'ить и об'Office'ить, да еще со младых ногтей начиная. То есть со средней школы. Да еще в приказном порядке~--- по указанию сверху. И притом~--- за большие (называется сумма в 5 млн долларов) деньги, взятые, как это принято у нас, из бюджета. То есть из кармана граждан, все последние годы добровольно и добросовестно отстегивавших 13 процентов своих, с позволения сказать, доходов (а куда денешься~--- хочешь не хочешь, а в бухгалтерии все рано вычтут) на укрепление казны в лице отдельных её представителей (помните призыв: <<Пожалуйста, заплатите налоги!>>).

Факт, конечно, возмутительный во всех отношениях. И особенно~--- с точки зрения воздействия на неокрепшее сознание подрастающего поколения. Которое, как правильно заметил Сергей, будет ассоциировать компьютер со всем известной заставкой при его загрузке.

Воздействие на сознание школьников тем более преступно, что оно (то есть сознание их), в отличие от сознания чиновников, не закалено годами пропаганды преимуществ социализма. Хотя, кто знает, может, сникерсы и прокладки обеспечивают иммунитет не хуже. Вспоминая Hugges, который растет и развивается вместе с~\dots~(ну, сами знаете, с чем~--- тем, что становится все здоровее), готов в это поверить.

Стоит пожалеть и учителей. Ведь (цитирую приводимый в приложении к статье Дмитрия документ, из-за которого весь сыр-бор разгорелся) <<В комплект поставки каждого рабочего места учителя один (подозреваю, пропущено ``должен входить''---~А.\,Ф.) лицензионный установочный OEM CD-ROM с операционной системой и документацией по её использованию>>.

Судя по контексту, под операционной системой имеется ввиду (опять цитирую) <<Microsoft Windows2000/98Professional>>. Я, конечно, не знаток продукции Microsoft, но об ОС с таким названием слышать не доводилось. Так какой же ОС должны обучать учителя своих питомцев? Да еще по документации с OEM CD-ROM~--- покажите мне дорожку диска, на которой эта документация притулилась.

И с точки зрения отношения к отечественному производителю~--- поступок хамский. Ведь ранее наш производитель с успехом (правда, переменным) невозбранно внедрял свои Lexicon'ы на десктопы российских госчиновников. Тех самых, принявших пресловутое решение (не под воздействием ли несравненных достоинств Lexicon'а, к слову сказать?).

И те же госчиновники, ответственные за пресловутое решение, тоже поступили антигуманно~--- по отношению к самим себе, вроде унтер-офицерской вдовы: им же, беднягам, теперь придется оправдываться: мол, не брали мы взяток у Microsoft'а, вот-те крест, не брали!

Единственно, за кого можно, казалось бы, порадоваться, так это за Microsoft, даже и допуская, что той пришлось раскошелиться на взятки (но мы такой клеветы~--- не допустим!). Ведь если фундамент популярности её продукции на пост-советских просторах был заложен российскими пиратами, то крыша над ним (или для него~--- фундамента, то есть) окажется делом рук трудящихся Минобразования Российской Федерации. <<Я думаю~--- так>>~--- сказал бы умнейший из медведей.

Однако я пишу свои заметки не ради возвеличивания Microsoft, советских (пардон, российских) чиновников и даже не в заботе об отечественных производителях. А исключительно к вящей славе Linux'а и всяких прочих POSIX-совместимых операционок. И потому хотел бы взглянуть на это дело с другой стороны. Для чего придется обратиться к недавней истории.

Давайте вспомним, чему учили школьников в советской средней школе. Кому на Руси жить <<Хорошо!>>, и как <<гвозди бы делать из этих людей>>, и не для того ли <<закалялась сталь>>, чтобы забивать их десятью сталинскими ударами, детектив <<Кто же разбудил Герцена>>, эротические эссе о любви Партии к народу, приключенческие романы об Острове Свободы в флибустьерском море.

Чем отвечал на это благонамеренный советский школьник? Бодрыми пионерскими песнями типа <<Куба, отдай наш хлеб, Куба, возьми свой сахар!>> или о Гагарине и Хрущеве, сентиментальными загадками о съеденных шишках, народными частушками о гранате в окне дома Советов и лучшем стрелке маршала Блюхера (фамилия которого, как показало исследование Василия Ивановича Чапаева, с английского не переводится). И во всем этом~--- ни малейшей политики.

Теперь представим себе, каково будет отношение нынешних школьников к ненавязчивому, в рамках обязательной школьной программы, приобщению к мощи Microsoft Windows (будь она даже четырежды 2000 по 98Professional), да еще помноженной на достижения советской педагогической науки и (особенно) практики? Лично мне, из-за строгого старорежимного воспитания, приходят в голову только стихи вроде <<Мимо дома педагога без Excel'я не хожу: то в Windows Word засуну, то Access'а сынсталлю>>. А у нынешних школьников, в силу, повторяю, раннего знакомства с тампаксами и хаггисами, фантазия куда богаче. Так что нужно верить в нашу молодежь~--- реакция будет адекватной\dots

И потому каждый приверженец идеи Open Sources должен быть благодарным нашему Минобразу за то, что в качестве базовой платформы для обучения школьников основам компьютерной грамотности он не принял Linux или FreeBSD. Не согласны? А припомните-ка, какой срок карантина по окончании школы вам потребовался, чтобы прочитать, наконец, <<Капитанскую дочку>> или <<Мцыри>>?

Так что и Microsoft, на самом деле, завидовать не с чего. Вообще говоря, повод к настоящей заметке ложится в один ряд с изменением лицензионной политики и условиями регистрации грядущих программных продуктов, призванными, как вы помните, по возможности затруднить жизнь их легального пользователя, или заявленными аппаратными требованиями для Windows~XP.

Что наводит на размышления: а кто определяет политику гиганта софтверной индустрии? Не внедрился ли в её стройные ряды агент из движения Open Source? Если да~--- он воистину заслуживает недавно организованной премии проекта GNU. Если же это сознательная политика фирмы~--- не зря ли из нее пытаются вылепить образ врага свободы? Ведь ни Ричард Столлмен, ни Эрик Раймонд не смогли бы сделать больше для популяризации альтернативных открытых решений\dots

\section{О злокозненности Билла Гейтса, или почему я не люблю Windows} 
\begin{timeline}2003 г\end{timeline}
Написать эту заметку меня подвигло обсуждение на одном из форумов~--- зачем нужен Linux на домашней машине, постоянная переписка с моим коллегой и старым товарищем Владимиром Поповым, а также наблюдения и размышления. Название же возникло по аналогии с произведением известного древнеримского грека~--- писателя и моралиста Плутарха <<О злокозненности Геродота>>. Каковую Плутарх усмотрел в чрезмерном, по его мнению, возвеличивании Геродотом древних персов и прочих варваров.

Действительно, если вчитаться не в оценки Геродота, а в описание поступков, становится ясным, что всякие там Киры, Дарии и Ксерксы были ребятами в высшей степени разумными, справедливыми и ответственными. Чего далеко не всегда можно сказать о Фемистоклах там и Мильтиадах. И вообще, неизвестно, как бы мы относились к оплоту древней демократии, если бы до нас дошли какие-либо не-греческие источники о Греко-Персидских войнах\dots 

В этих не претендующих на систематичность заметках я не буду кричать <<Виндовс мастдай>> или призывать <<Линукс форева>>. А просто попробую объяснить, что же именно в Windows мне не нравится. Как известно, Линус Торвальдс в свое время назвал Windows объективно плохой системой. Основываясь, вероятно, на своем опыте разработчика. Я же скажу, почему солидарен с ним с позиций пользователя. 

Основываясь на своем более чем десятилетнем опыте эникейщика, все компьютерных пользователей я разделил бы на три категории. Первая~--- это те, которые хотели бы разобраться в системе до упора. Причем~--- вне зависимости от образования, опыта предыдущей работы, характера основной деятельности. С этой, немногочисленной, категорией все ясно~--- рано или поздно эти люди приходят в мир Linux/UNIX и Open Source. Как клиенты, эти пользователи поначалу способны доставить немало хлопот. Однако потом~--- потом общение становится двусторонним, и они превращаются в коллег и товарищей. 

Вторая категория, преобладающая количественно,~--- напротив, те, кто об устройстве компьютера, его операционной системе и прочих материях знать ничего не желают, им достаточно было бы того, что комп был кем-то запущен и как-то работает. Я не говорю, что это плохо~--- амбиции таких пользователей лежат совсем в других сферах. Более того, как клиенты такие пользователи очень приятны~--- они делают, что им сказано, и не занимаются никакой отсебятиной в тех вопросах, в которых они не разбираются и разбираться не хотят. 

Идеальной системой для пользователей второй категории был бы Мак. Опять же по моим наблюдениям, пользователь Мака вполне реально может не иметь представления не только об устройстве, скажем, файловой системы, но даже и слова-то такого, <<файл>>, не знать. Что вовсе не мешает ему очень эффективно использовать Мак в своей профессиональной деятельности\dots

Агитировать таких пользователей за Linux~--- ни один здравомыслящий человек не будет. Однако Маков у нас мало, они дороги, софт для них не очень доступен. И в итоге такой пользователь оказывается безальтернативно привязан к самой великой и дружественной операционной системе. Которая обещает решение всех пользовательских проблем. Если не сейчас, то уж в следующей версии~--- точно. 

Однако насколько оправданны эти обещания? Конечно, на элементарном уровне разобраться в Windows легко~--- достаточно взять книжку вроде <<Освой самостоятельно за 24 часа>>, чтобы научиться запускать Word или выходить в Интернет с Explorer'ом и Outlook'ом (можно и вовсе без книжки обойтись, методом научного тыка). 

Однако дальше начинаются проблемы. Оказывается, что для мало-мальски эффективной работы в Windows мало щелкать мышкой по иконкам и менюшкам. Приходится узнавать не только о том, что такое файл, но и (страшно подумать) о драйверах устройств, прерываниях и многом другом. Однако разобраться в этом досконально а) нет желания, и б) нет возможности. И в итоге пользователь второй группы волей неволей превращается в пользователя группы третьей.

А именно: вынужденно нахватавшись слов <<файл>>, <<драйвер>> и т.д., научившись менять разрешение экрана и обои рабочего стола, он проникается чувством собственного величия. 

Давным-давно в одной научно-популярной книжке по физике (это была <<Занимательная ядерная физика>> Константина Мухина) я прочитал, почему вредны научно-популярные книжки по физике: <<потому что создают иллюзию понимания там, где о настоящем понимании не может быть и речи>>. Так вот, эти слова в полной мере приложимы к Windows.

Действительно, пользователь Linux/FreeBSD \textit{etc}. просто вынужден идти до конца в освоении системы. Начиная лет пять назад свое знакомство с Linux'ом для того, чтобы иметь простую и удобную среду для писания научных статей про геологию (тогда я еще тешился иллюзиями, что это кому-то, в том числе и мне самому, нужно), я и помыслить не мог, что логика этого знакомства занесет меня в сочинение скриптов для обработки текстов или в дебри устройства файловых систем. А уж что я буду собирать собственную систему~--- такое мне и в кошмарном сне привидеться не могло. Как правильно заметил в своей известной книге Владимир Водолазкий, быть просто пользователем Linux'а скучно\dots Я бы добавил~--- пожалуй, что и невозможно. 

С другой стороны, как уже отмечалось, пользователь Мака может обойтись без знаний о системе вообще~--- ему довольно будет чисто практических навыков работы со своими программами. Относительно незнания слова файл~--- это я не выдумал. В нашем институте как-то одна девушка сдавала кандидатский минимум по информатике. Так вот, экзаминаторы задали ей вопрос~--- какие расширения имеют графические файлы. Последовал ответ: она работает на Маке, а там ни расширений нет, ни файлов\dots Конечно, впорос был поставлен некорректно, но и ответ, согласитесь, характерен. Причем ничего плохого как о специалисте я об этой девушке сказать не могу.

Windows же оказывается в данном случае той самой золотой серединой, которая~--- хуже всего (<<если ты первый, это говорит само за себя, если последний~--- можешь себя первым вообразить>>). Пользователя Windows просто против его воли толкают к компьютерной полуграмотности. Совсем без знаний ему обойтись невозможно (разве что по первому требованию вызывать домашнего эникейщика). Получать же эти знания часто нет желания, это во-первых. Во-вторых же и главных, разобраться в Windows до конца практически невозможно даже при наличии пламенного энтузиазма~--- столько в ней необъяснимого и непредсказуемого. 

Случай из собственной практики, послуживший последним толчком к искоренению Windows на моей домашней машине. Сижу себе, никого не трогаю, книжку верстаю, про геологию (если кому интересно~--- сборник <<Минеральные ресурсы России>>). Сроки поджимают~--- принесли мне текст вчера, срок сдачи в типографию, как водится, позавчера. Текст~--- в Word'е, в нем же, экономии времени ради, и верстаю (благо не произведения полиграфического искусства от меня ждут). Все нормально, как вдруг~--- по всему тексту закрывающие кавычки из французских самопроизвольно становятся английскими. Пытаюсь выполнить глобальную замену~--- ничего не получается. Заменяю глобально, вбивая номера кодов~--- с тем же эффектом. Причина непонятна, как бороться~--- не ясно. В итоге, как обычно, пришлось действовать через \texttt{/dev/ass}, но это уже не очень интересно.

Можете представить себе такую ситуацию в LyX или в OpenOffice? Даже если и можете, тут же и метод борьбы придумаете~--- открыть в текстовом редакторе исходник и поправить, что нужно (благо первый~--- почти \TeX, а второй~--- XML). 

Предположим, однако, что пользователь Windows не пасует перед трудностями. И готов к любым тяготам и лишениям, дабы досконально разобраться в любимой системе. И тут он обнаруживает, что затраченные им усилия и время таковы, что стоят освоения трех Linux'ов. Особенно если речь идет о всамделишней системе, то есть Windows~NT/2000/XP, в которых есть все то же самое, что в UNIX'е, только непонятнее. Так как глубоко закопано в недрах дружественного графического интерфейса. 

Как обычно, происходит типичная подмена понятий. Святая правда, что существует Windows, легкая в освоении, и что существует Windows устойчивая и надежная. Да вот только никто еще не доказал, что это~--- одна и та же Windows. А номенклатура типа Windows~98 или ME только способствует тому, чтобы пользователь путал теплое с мягким. 

Предвижу возражение: но зато Windows, при всех её недостатках, способствовало массовому внедрению компьютерных технологий. Если раньше на Маках работали, скажем так, представители творческих профессий, а на PC'ках, например, научные работники в <<черном DOS'е>> считали геохимические коэффициенты, то сейчас, под Windows, все достижения компьютерной мысли в руках любого юзера. 

Крыть нечем, так оно и есть. Однако зададимся вопросом, однозначно ли это хорошо, и хорошо ли для всех? Как сказали классики отечественной фантастики, <<медведя можно научить ездить на велосипеде, да только будет ли от этого медведю удовольствие или польза?>> Я уж не говорю о том, что распространение <<народного компьютера>> привело к невиданной бюрократизации.

Не секрет же, что все усилия по внедрению <<безбумажного документооборота>> привели пока только к лавинообразному росту документооборота бумажного . Раньше, чтобы создать руководящий циркуляр в четырех экземплярах, его сделовало написать от руки, отдать секретарше, той~--- аккуратно переложить листы копиркой, отпечатать, с помощью бритвы/забивки/замазки исправить ошибки. Теперь~--- врубил на PC'шке из Word'а печать хоть ста копий~--- <<и сидишь себе, болтаешь ножками, сам сачкуешь, а она работает>>. Задача, вполне посильная любому руководителю средней руки. А мы потом эти циркуляры~--- читай, да еще и расписывайся, что ознакомился. 

Я не призываю к луддизму. Потому что массовые компьютеры действительно сделали возможным для широких масс трудящихся-индивидуалов то, что раньше было по силам только госучреждениям и корпорациям. И причем на принципиально ином уровне, чем, так сказать, вручную. Однако это отнюдь не значит, что компьютеры сделали жизнь (и работу) легче~--- они сделали её другой. И для того, чтобы эффективно использовать их возможности, требуются не графические интерфейсы~--- требуется перестройка мышления, как это ни высокопарно звучит. И вот этой-то перестройке мышления Windows ни в малейшей мере не способствует. 

По моему глубокому убеждению, большинство пользователей применяет компьютер неправильно. И это не вина их, а беда: ничто в окружающем мире не толкает их к пониманию этой неправильности. К пониманию того, что с компьютером можно работать просто как с пишущей машинкой, но это все равно, что забивать микроскопом гвозди.

Парадоксальный, но хорошо знакомый мне пример: подавляющее большинство научных работников не использует компьютер как инструмент научного исследования, а только как средство представления его результатов. И не потому, что компьютер не нашел бы применения в исследовательском процессе, скажем, в геологии: просто, за исключением нескольких очевидных случаев (расчета тех же геохимических параметров, например) применения эти не лежат на поверхности. И, кроме всего прочего, требуют еще и изменения подхода к самому исследовательскому процессу. А вот этого-то никто не объясняет на курсах по подготовке к сдаче кандидатского минимума\dots 

В свое время, пытаясь объяснить сказанное выше своим коллегам, я, дабы они не приняли это в обиду, придумал такой пример, который и не устаю повторять. Представьте себе, что блестящую кавалерийскую дивизию в одночасье пересаживают на танки. Не только не переучивая их, но напротив, пытаясь убедить в том, что все приобретенные ранее навыки без всяких изменений можно использовать в новых условия. Например, скажем, рычаги~--- все равно что трензеля, педаль газа~--- что твое стремя, а переключатель передач~--- та же нагайка. 

А ведь это именно то, в чем пытаются, и небезуспешно, убедить пользователей апологеты Windows.

И в этом, на мой взгляд, главная злокозненность этой системы. Вместо ясного понимания того, что <<река это не дорога>>, пользователя тешат иллюзиями, что ему ничего не нужно менять в своих навыках и привычках, достаточно регулярно обновлять версии Windows (заодно с железом)~--- и жить будет легко и весело. И если пока его десктоп еще не точная копия письменного стола, то уж в следующей версии это обязательно будет исправлено\dots 

С массовым внедрением компьютеров мир изменился, и таким, как прежде, уже никогда не будет. И чем скорее мы все это осознаем, тем лучше. А потому системы, такому осознанию способствующие, должны развиваться, популяризироваться и распространяться. Чем, надеюсь, мы с вами по мере сил и занимаемся.

\section{Еще раз о доблести и злокозненности} 
\begin{timeline}17 мая 2005 г\end{timeline}

\hfill \begin{minipage}[h]{0.45\textwidth}
Не бойтесь сумы, не бойтесь тюрьмы,\\
Не бойтесь пекла и ада\dots\\
А бойтесь единственно только того,\\
Кто скажет: <<Я знаю, как надо>>
\begin{flushright}
\textit{Александр Галич}
\end{flushright}
\bigskip\end{minipage}


Сочинив пару лет назад заметку о причинах, по которым я не люблю Windows (по случаю, под впечатлением дискуссии на одном из форумов), я никак не ожидал, что спустя пару  лет она станет предметом активного обсуждения. В каковое поначалу вмешиваться не хотел~--- все, что я мог сказать по предмету разговора, было сказано мною ранее. Однако ввиду столь разветвленного обсуждения~--- не выдержал, ретивое взыграло.

Для начала в конспективной форме изложу причины, по которым я не люблю Windows (или, если угодно, по которым лично мне, как обычному пользователю, эта система не нравится).

Первая и главная из них~--- такова: обещая пользователю избавление от необходимости приобретения специфически компьютерных знаний, она своего обещания не выполняет.

Вторая причина вытекает из первой. Требуя от пользователя некого минимального объема знаний, Windows, в отличие от UNIX-подобных систем, ни в коей мере не подталкивает пользователя к их приобретению. Более того, просто провоцирует его на <<полузнание>> на уровне набора готовых рецептов, без понимания сути производимых действий. И добро бы ему, как, скажем, пользователю Mac'а, этого действительно хватало бы. Ан нет~--- рано или поздно ему приходится разбираться и в правах доступа, и в структуре реестра, и еще во многих вещах\dots Впрочем, тут мы возвращаемся к первой причине.

Наконец, третья причина кроется в проприетарной природе как самой системы, так и, главное, наиболее распространенного прикладного софта для нее. Разработчики которого, дабы побудить пользователя к <<смене вех>> (пардон, версий), вынуждены отягощать свои продукты новыми <<продвинутыми фичами>>, 90\% которых оказываются невостребованными пользователями. Что ведет к утяжелению программ без адекватного увеличения производительности и стабильности (а подчас наоборот~--- с падением и той, и другой).

Сравните монстроидальность любой современной программы <<малого базарного набора>> для Windows с отточенностью классических UNIX-утилит, функционально доведенных до немыслимого совершенства чуть не десятилетия назад.


\textit{Оговорка к пункту третьему: я отнюдь не считаю, что все программы вообще должны быть свободными и, тем более, бесплатными. Как и за какие деньги распространять плоды своего творчества~--- сугубо личное дело каждого, сподобившегося что-то сотворить\dots}


А теперь~--- несколько комментариев к мнениям, высказанным участниками дискуссии. Начну, разумеется, со статьи Вадима Валерьевича Монахова, которая и послоужила поводом для этой заметки. И, конечно же, с его программного утверждения:


\begin{shadequote}{}
Если судить объективно~--- MS~Windows~XP сейчас самая лучшая из имеющихся для пользователей систем.  
\end{shadequote}

Что ж, готов был бы поверить автору на слово. Однако доказательствами этого утверждения автор себя не обременяет. И потому возникает вопрос~--- а чем его субъективное мнение более весомо, чем мнение простого разработчика~--- Линуса Торвальдса, или простого пользователя~--- вашего покорного слуги. Только не подумайте, что я равняю себя с именами, упомянутыми выше. Однако мне видится весьма странным утверждение, что мой оппонент, позиционирующий себя в качестве разработчика, рискует определять, что лучше мне, пользователю. Может быть, я и сам справлюсь с этой нелегкой задачей?

Кстати, из сказанного выше можно сформулировать и четвертую причину моей нелюбви к Windows: уверенность её создателей в знании потребностей пользователей можно сравнить только с убежденностью вождей мирового коммунизма в понимании потребностей трудящихся (развивать эту тему можно было долго, но это~--- отдельная история, а потому~--- см. эпиграф).

Ну вот, в главном, вроде, высказался, остальное~--- мелочи.


\begin{shadequote}{}
У меня есть знакомый в фирме, где работа идёт на Макинтошах. Люди страдают!  
\end{shadequote}

Думаю, каждый, чьи амбиции лежат вне сферы IT~--- те, кого называют content creator'ами (и кто когда-либо видел Mac и его OS),~--- с удовольствием согласился бы пострадать вместо тех далеких ближних\dots

Случай из жизни. Лет эдак пять назад делал я сайт для нашей лаборатории. В силу специфики контента, требовал он масштабируемой графики. Для представления которой черт меня дернул использовать формат DjVu (незадолго перед тем появившийся). После чего со всех концов мира посыпались письма~--- мол, не видим мы твоей графики. Хотя на индексной странице русским (а также английским) по белому было написано~--- для просмотра графики скачать соответствующий plug-in. И указано, откуда именно\dots

И каким же бальзамом на мою душу был приезд одного мужика с Австралийщины. Который на аккуратный вопрос о том, видит ли он наши карты, ответил: конечно. Далее еще более аккуратно было спрошено: ты что, plug-in скачал, откуда было сказано? Да нет, отвечает он, ничего такого я не качал~--- да и знать не знаю, что это. Тут-то меня и надоумило спросить~--- а на чем работаешь-то? Да на Маке, ответил он.

Это что касается отношения пользователя (и его страданий). Однако оппонент затрагивает и другой вопрос~--- 


\begin{shadequote}{}
Почему Windows лучше для программиста  
\end{shadequote}

Хотя исходная заметка, вроде бы, не давала к поднятию этой темы ни малейшего повода. Поскольку я не программист~--- а, как неоднократно подчеркивалось, тот самый обычный пользователь, на страже интересов которого стоит Windows,~--- по существу дела мало чего могу возразить. Хотя (опять слово Вадиму):


\begin{shadequote}{}
Под LINUX так не сделать. Нет единой системы взаимодействия программ друг с другом!  
\end{shadequote}

Боясь показаться профаном, спрошу: а что, каналы (pipe) и сокеты (sockets)~--- это не средства взаимодействия программ друг с другом? И не отсчитывается ли их возраст с середины 70-х? Ах да, понял: это не единая система. Каковой являются, видимо, DDE, OLE в купе с COM'ом\dots

Что же касается <<Под LINUX так не сделать>>~--- это тот момент, где я радостно соглашусь с моим оппонентом: да, того, что он приводит по ссылке, не сделать ни одному линуксоиду. Видимо, и понятия о\dots да нет, не дизайне даже, а элементарном вкусе, у Windows-разработчиков совсем свои. Простому пользователю такой цветовой гаммы не осознать (о прочих элементах декора умолчим).

А дальше оппонент снова возвращается к теме страданий пользователя:


\begin{shadequote}{}
Под Windows можно написать программу, работающую с каким-нибудь форматом файлов, к примеру~--- видео. А после выхода нового формата пользоваться той же программой, просто догрузив кодеки. И будут показываться файлы тех форматов, которые еще не были придуманы на момент написания вашей программы.  
\end{shadequote}

Завершая абзац уже цитированной фразой:


\begin{shadequote}{}
Под LINUX так не сделать.
\end{shadequote}

Недостаточная информированность оппонента в этом вопросе объясняется, видимо, отсутствием общения с детьми школьного возраста, активно обменивающимися видеодисками разного рода. Каждый из которых несет на себе собственный кодек, упорно пытающийся установиться в автоматическом режиме (после чего ранее установленные кодеки, как правило, работать отказываются). А вот в Linux (и, замечу в скобках, во всех прочих свободных UNIX-подобных системах) можно использовать сквозной набор кодеков (например, от текущей версии mplayer'а) во всех программах, в которых таковые требуются. По крайней мере, это имеет место быть на машине автора этих строк (или я что-то сделал неправильно?).

От обзора дальнейшей дискуссии воздержусь: с высказавшимися в пользу Linux'а мне дискутировать особо не о чем (частные заморочки и непонятки можно решить в рабочем порядке). Тем же, чей выбор склоняется в пользу Windows, могу сказать: это ваш выбор, и, вероятно, у вас к тому есть веская мотивация. Напомню, что в исходной статье не содержалось утверждения: Linux лучше, чем Windows (Чем?~--- Чем Windows). А говорилось лишь о причинах, по коим я лично не люблю вторую из именованных операционок. Все люди разные, и некоторые, возможно, любят Windows именно за то, что я перечислил, как его недостатки. Однако, как сказал Д'Артаньян кардиналу Ришелье, <<по роковой случайности все мои друзья состоят в королевских мушкетерах, а все враги, по стечению обстоятельств, служат Вашему Преосвященству>>.

И уж совсем в заключение~--- именно для пользователей Windows. Не надо представлять себе пользователя Linux (или иной другой UNIX-подобной операционки) как некое существо, только и делающее, что настраивающее систему, или перекомпилирующее ядра. Отнюдь~--- подавляющее большинство из них занято практической работой, причем в самых разных сферах человеческой деятельности. Среди моих личных знакомых->>линуксоидов>> есть историки, юристы, экономисты, переводчики, в том числе с весьма экзотических языков. А также, простите, те самые секретарши и домохозяйки, интересы которых столь ревностно отстаивают разработчики Windows. И уж поверьте~--- мотивация для своего выбора у них была не менее веской, чем у вас. Возможно, следовало бы сочинить заметку~--- <<За что и почему я люблю Linux>> (хотя на самом деле Free- и прочие BSD я люблю больше:-)). Кто знает~--- может быть, и вы изменили бы свое мнение. Но это~--- совсем-совсем другая история\dots

\section{Чем удобна Вынь-да?} 
\begin{timeline}4 Июль 2007 г\end{timeline}

\hfill \begin{minipage}[h]{0.45\textwidth}
Все говорят: Кремль, Кремль. Ото всех я слышал про него, а сам ни разу не видел. Сколько раз уже (тысячу раз), напившись, или с похмелюги, проходил по Москве с севера на юг, с запада на восток, из конца в конец и как попало~--- и ни разу не видел Кремля.
\begin{flushright}
\textit{Венедикт Ерофеев, <<Москва-Петушки>>}
\end{flushright}
\bigskip\end{minipage}

Подобно лирическому герою Венечки, ото всех я слышал, как удобна Вынь-да для конечного пользователя. Слышать~--- слышал, а сам ни разу не видел. Были, конечно, старые смутные воспоминания~--- сначала о версии 1.0 (это был шедевр сюрреализма и абстракто-кубизма с примесью садо-мазохизма), потом о~3.0, потом~3.1 и~3.11. А там и всякие~95-е попадались, и прочие Линолеумы с Миллениумами\dots

Вот и вчера опять не увидел. Я, как только в нее, Вынь-ду эту (Windows~XP она нынче называется, кто не в курсе) загрузился, давай виртуальные консоли искать. Жму \keystroke{Alt}+\keystroke{Control}+\keystroke{F2}~--- нету. На \keystroke{Alt}+\keystroke{Control}+\keystroke{F3}~--- опять нету. И так до девятого стакана (пардон, до клавиши \keystroke{F9}). Но и на клавише \keystroke{F10}~--- ни малейшей виртуальной консоли не обнаружилось. Обидно, да? Досадно? Ну да ладно, обойдусь виртуальными рабочими столами в графическом режиме.

Не тут-то было~--- и виртуальных рабочих столов в этой самой Вынь-де нетути, один он единственный, на который предлагается все окошки потребные грузить. А мне их грузить надо~--- ох много, штук 6-10, да еще желательно чуть ни каждое в полноэкранном режиме.

Хорошо, а просто виртуальный декстоп, с разрешением больше физического, сделать-то можно? Чтобы хоть четыре окна размером в экран в него вписать? Нет, говорит мне Вынь-да, нельзя.

Ну опять-таки ладно, перетерпим, терминальное-то окно тут есть? Есть, ура, и командная оболочка в нем запускается. Только вот с командами не густо: 
\texttt{find} ~--- нет, 
\texttt{grep} ~--- нет, 
\texttt{cat} ~--- нет, 
\texttt{split} ~--- и то нетути. 

Что получается? Нехорошо получается. Ну хоть telnet какой-никакой запущу. Набираю~--- \texttt{tel}, жму табулятор. И вы думаете, у меня хоть что-то автодополнилось? Фиг с маслицем, как сказали бы в Одессе (правда, конечно, в Одессе покруче бы выразились, но не будем оскорблять нравственное чувство молоденьких девушек, возможно, читающих эти строки). Экранного буфера терминала~--- нет. Даже истории команд путёвой~--- и то нет.

Ну что же, а хоть текстовый редактор имеется? Имеется, Notepad~--- погоняло его. Но боже, если это~--- текстовый редактор~--- то я не иначе как китайский император и Папа Римский в одном лице. Это же то самое, что некий сексуально озабоченный юноша назвал жалким подобием левой руки: ни кейбиндингов, ни регэскпов, ни макросов (о протоколировании действий я даже и заикаться побоялся).

Да, приходится признать, что графа Монте-Кристо, сиречь пользователя Вынь-ды в рабочих целях, из меня не вышло. Но ведь глаголят, что несравненна эта система в целях развлекательных. Так что переквалифицируюсь-ка я в управдомы, лягу на диван и муз\'{ы}ку послушаю. Например, любимого Вертинского, взятого с сайта \href{http://bards.ru}{bards.ru}, очень успокаивает и на философский лад настраивает.

Опять незадача: все, что на \href{http://bards.ru}{bards.ru} лежит, лежит там в формате Real Audio. А медиаплейер Вынь-довый о таком формате и не подозревает, оказывается. Не беда, Ынтырнет, хвала Аллаху, все-таки есть (хотя как появился~--- не иначе попустительством Шайтана), так что скачаем с сайта производителя плейер соответствующий.

Скачать-то оказалось несложно~--- да вот только нынешняя его версия не играет те древние Real Audio, что были на \href{http://bards.ru}{bards.ru} (хотя, заметим в скобках, Linux'овые кодеки, например, от mplayer'а, справляются с ними шутя).

В общем, ни для чего Вынь-да оказалась не пригодной~--- ни для работы, ни для развлечений. Так что если кто скажет вам, что она удобна для конечного пользователя, то, вслед за Ходжой Насреддином, плюньте этому человеку в лицо, назовите лгуном и выгоните из своего дома. Наверняка человек этот ничего, кроме Вынь-ды, в жизни своей не видел~--- даже Кремля. Потому что каждый раз оказывался на Курском вокзале.

P.S. Автор этой басни, в отличие от своего лирического героя, знает, что для Windows существует POSIX-шелл с набором классических утилит, соответствующих вышеуказанному стандарту. Слышал он и о том, что для него можно скачать дополнения, превращающие этот шелл в полноценный bash, а POSIX-версии утилит равняющие по функционалу с GNU- и BSD-версиями. Есть у него подозрения, что виртуальный декстоп можно получить средствами драйверов от производителей видеокарт, а множественные рабочие столы учреждаются благодаря сторонним программам, типа Aston. Да и mplayer под Windows, вроде, никто не запрещает собрать.

То есть, если как следует постараться, Windows можно превратить в некое подобие Linux'а или *BSD. Но, как сказал известный советский поэт,


\begin{shadequote}{}
Можно бы~--- да на фига?
\end{shadequote}

Если в Linux'е или любой BSD можно иметь все это из коробки\dots

\section{Без-Win’ные машины: несколько крамольных соображений}
\begin{timeline}Март 5, 2009\end{timeline}
В последнее время обострился вопрос о возврате денег за предустановленную на новые компьютеры Windows и о продаже машин без предустановленной ОС вообще. Довольно давно он обсуждается \href{http://forum.posix.ru/viewtopic.php?id=1174}{здесь}. А недавно в кампанию за без-win'ные компьютеры активно включился ЦЕСТ, который \href{http://www.centercest.ru/archive/2009/02241.return.vista/}{намерен бороться с навязыванием OEM-программ}. В связи с чем возникла и ещё одна \href{http://linuxforum.ru/index.php?showtopic=87019}{аналогичная тема}. В этой связи и мне захотелось высказать несколько соображений, возможно, вполне крамольных. Почему эта проблема обострилась именно сейчас? Ведь на самом деле лет ей очень немало. Напомню, что первый в истории случай возврата денег за OEM-операционку имел место году в 1996-м. Когда один австралийский парень, пользователь Linux'а, приобретя ноут от Toshiba с предустановленной Windows~95, нафиг ему ненужной, добился соответствующего решения в суде. Точно не помню, сколько времени он на это затратил, но, кажется, не очень мало.

А ещё до этого, опять же должен напомнить, все компьютеры <<в сборе>> поставлялись с предустановленной DOS текущей или предпоследней версии. И тоже никто особо не жаловался. Правда, DOS в розничном (дискеточном) варианте стоила баксов 30-50~--- но в начале 90-х для постсоветского трудящегося это были вполне большие деньги.

С тех пор такие случаи происходили регулярно, но всегда были единичны, касались исключительно индивидуалов, и борьба проходила с переменным успехом. Если порыться в старых темах на форумах FOSS-тематики, можно найти как жалобы на отказ возврата денег фирмой имя рек, так и радостные сообщения о том, что после длительных и упорных препирательств пользователю удалось отстоять свою без-win'ную непорочность.

Начать надо с того, что под стыдливым эвфемизмом~--- предустановленное OEM-программное обеспечение прячется Windows. Причём в последнее время не всякого рода, а исключительно Windows Vista. Именно её в обеих редакциях~--- Home Basic или Home Premium~--- и предустанавливают нынче на машины. И именно она вызвала такой взрыв протестов, причём не только со стороны записных линуксоидов, но и стойких вендузяднегов. Повторяю, до недавнего времени такие протесты были единичны, а предустановленная Windows~XP обычно не вызывала жалоб даже от убеждённых адептов FOSS.

Так что первопричина~--- не в самом факте предустановки некоей ОС, а в общепризнанных <<высоких>> потребительских качествах одного из их представителей. Добавлю, что мнение о качествах Windows Vista~--- отнюдь не моё (я её в глаза не видел и, скорее всего, не увижу никогда), а исходит от лично мне знакомых пользователей Windows всякого рода (не <<вендузяднегов>>) с большим стажем и высокой профессиональной квалификацией в своей области, будь то администрирование, разработка, обработка изображений \textit{etc}.

Далее, протесты эти возникают почти исключительно при покупке ноутбуков. Хотя предустановленная ОС может быть обнаружена на <<брендовых>> десктопах и даже на персоналках отечественной сборки, возмущений по этому поводу я не слышал. И понятно, почему. <<Брендовые>> машины покупаются почти исключительно крупными корпоративными заказчиками в ипостасях производственных серверов и рабочих станций, а в таких случаях разговор о предустановленных операционках~--- совершенно отдельный, и к данной теме не относящийся. Что же касается отечественных сборщиков настольных компьютеров~--- с ними всегда можно договориться. И каждый, кто покупал в своей жизни более одного компьютера, имеет на примете одну или ряд дружественных фирм, где ему завсегда пойдут навстречу в этом щекотливом вопросе.

А теперь посмотрим, такое ли уж зло предустановленная Vista на ноутбуках.

С одной стороны, если рассуждать с точки зрения абстрактной справедливости <<для всех>>~--- безусловное зло. Во-первых, покупателя грубо решают свободы выбора~--- одного из величайших достижений мира FOSS. Во-вторых, злостно нарушается закон о защите прав потребителя, запрещающего продавать <<товар в нагрузку>>. В-третьих, покупатель теряет на этом свои собственные деньги, если он не намерен пользоваться <<свистой>>, вне зависимости от того, на какую операционку~--- Windows~XP, Linux или FreeBSD,~--- он её сменит.

А с другой стороны, если посмотреть на дело с позиций справедливости суровой, но конкретной, все эти во-энных следует разобрать по пунктам. О лишении свободы выбора можно говорить только в отношении тех покупателей, которые этой самой свободой в состоянии воспользоваться. Смею заявить с полной ответственностью, что доля их в общем количестве людей, покупающих ноутбук, ничтожна. А для всех остальных свобода выбора~--- это лишь свобода выбрать более дешёвое решение (да и то~--- очень условно дешёвое), которым они всё равно не смогут воспользоваться. Почему~--- скажу несколько позже.

Нарушение закона~--- это, конечно, плохо. В юридические тонкости вдаваться не буду за некомпетентностью. Но если исходить из здравого смысла~--- а неужели это самое страшное нарушение закона из тех, с какими нам приходилось сталкиваться? Если же подойти с позиций сугубо технологических~--- с помощью несложной софистики можно доказать, что никакого нарушения тут нет. И это я также надеюсь продемонстрировать в дальнейшем.

Наконец, третий вопрос~--- о затратах на товар, насильственно навязанный и покупателю не нужный. Этот вопрос~--- самый сложный, и потому распадается на множество подвопросов.

Для начала оценим масштаб затрат. До недавнего времени долю предустановленной ОС Windows в общей цене машины (повторю, речь идёт практически исключительно о ноутах) определить было практически невозможно. Практически все ноуты от средней и выше категории продавались с OEM'ной виндой <<в комплекте>>. Некоторые бюджетные и большинство супербюджетных моделей так называемых отечественных производителей продавались с FreeDOS, которая теоретически не должна была бы добавлять к цене ничего. Но конфигурационно модели с Windows и с FreeDOS не пересекались. Что же касается немногочисленных моделей с предустановленным Linux'ом~--- этим делом также баловались <<отечественные>> производители,~--- ценообразование на них вообще было (да и по сей день остаётся) необъяснимым ни с марксисткой, ни с анти-марксисткой платформы.

Почему я говорю~--- так называемых отечественных производителей? Потому что платформы всех ноутов (в том числе и почти всех зарубежных <<супербрендов>>) изготавливаются на Малой Арнаутской улице острова Тайваня и её продолжении в континентальном Китае. Это на Руси знает любой ребёнок, изучивший во времена Венечки Ерофеева способы очистки политуры. А вот где изготовляются всякого рода дополнительные комплектующие конкретной модели~--- не знает даже он.

Ныне ситуация изменилась. В прайс-листах многих фирм можно видеть модели отечественных брендов, зарубежных полубрендов и супербрендов, лежащих в вышесредней ценовой категории, укомплектованных абсолютно одинаково (то есть ближе к верхней планке сегодняшних стандартов), с двумя вариантами предустановленных ОС~--- FreeDOS и Windows~XP. Так вот, разница в цене между ними составляет\dots~40 баксов. Конечно, можно найти и идентичные цены, и цены, различающиеся на~5000 рублей (в <<пользу винды>>, так сказать), но это можно отнести к естественной человеческой жадности и перегибам на местах.

Однако, если ориентироваться на компании типа HP (40 баксов были вычислены именно по её продукции) и около, у которых рекомендованные розничные цены одинаковы по всему миру, и <<белой>> накрутке при ввозе, разница получается именно такая. Что, кстати, я, не имея прямых данных, высчитал ещё пять лет назад, когда на ноуты предустанавливалась XP, а я писал заметку с эпиграфом~--- Квинтилий Вар, верни легионы. Кстати, эта самая гипотетическая (или расчётная) разница в, грубо говоря, полтинник <<бакинских>>, не зависела от редакции: просто считалось приличным комплектовать почти бюджетные модели <<хоумовой>> редакцией, а заведомо не-бюджетные~--- <<профессиональной>>. Возможно, просто в соответствии с реалиями страны, в которой 600-й Pentium обязательно должен сопровождаться цветным монитором~--- малиновым, с золотыми кнопками.

Много это или мало? Опять-таки, очень как посмотреть. С одной стороны, со стороны человека, покупающего ноут за полторы штуки (вне зависимости от того, покупает он его <<по делу>> или <<прикиду для>>), разницы нет просто. Потому что она будет скорректирована в соседнем салоне при пересчёте уёв в без-уи по внутреннему курсу. Причём скорректирована в любую сторону~--- очень хитрожопый человек придумал понятие уёв сразу после августа 98-го.

А само слово, замечу в скобках, придумал автор этих строк сразу после явления понятия. Вполне возможно, что кто-то придумал этот термин независимо от меня и практически одновременно~--- уж очень он напрашивающийся. Но спорить о праве первородства не буду.

С другой стороны, у покупателя возникает вопрос~--- а какого такого зелёного я должен платить свои кровные сорок уёв, пусть даже для меня это и не деньги, за то, чем я пользоваться не буду? Для этого посмотрим, а что же он получает за свои кровные уи.

Давайте подумаем. Продавец должен продемонстрировать покупателю работоспособность приобретаемой машины. В самом первом приближении, для машины без предустановленной ОС, это достигается нажатием кнопки Power и демонстрацией картинки BIOS Setup~--- типа машина таки включилась. Кстати, на машине с предустановленной FreeDOS результат будет тот же самый~--- разве что вместо картинки BIOS вылезет ещё и приглашение командной строки. Может быть, его это устроит.

Но скорее всего, наш зловредный покупатель захочет, чтобы в машине функционировало всё: встроенный модем модулировал и демодулировал, WiFi~--- коннектился, кард-ридер~--- читал всякие разные карты памяти, в том числе и те, которые дают в качестве сдачи на кассах больших гастрономов. А это уже потребует некоторой ОС, поддерживающей соответствующие устройства как таковые, и имеющей драйвера под устройства конкретные. А если он ещё покупает не голимый ноут, а комплекс вместе с притером/сканером, а то ещё и, страшно сказать, МФУ, у него могут возникнуть гнусные и неприличные желания, чтобы принтер~--- печатал, а сканер~--- сканировал.

Спрашивается, каким образом продавец может удовлетворить столь наглые и беспочвенные притязания покупателя? Ответ очевиден: только посредством предустановки текущей на данный момент ОС, которая поддерживает набор текущего, опять-таки, <<железа>>. Или~--- честно предупредить покупателя, что ты, приятель, за полторы штуки уёв купил железяку, которая может включаться от сети, а всё остальное~--- твои трудности. Только вот много ли он продаст при таком подходе к делу?

Таким образом, у продавца есть единственный выход: продавать машину с той ОС, которую предустановил на неё производитель, которая была протестирована на совместимость со всеми комплектующими, в том числе и второстепенными, типа тех же WiFi'ев и card-rider'ов, о происхождении которых не знает даже советский ребёнок, собаку съевший за очищением политуры.

Предвижу возражение: Windows~XP поддерживает практически тот же набор оборудования, а, по мнению большинства пользователей, превосходит Vista по потребительским качествам (повторяю, это не моё мнение). Так почему бы не предустанавливать её?

Ответ готов: большинство серьёзных продавцов компьютерной (не бытовой) техники готовы предоставить эту услугу. При условии выполнения пользователем лицензионных требований. То есть приобретения ими у продавца <<коробочной>> версии указанной ОС или предоставлении собственного экземпляра таковой, заведомо не контрафактной. В противном случае речь будет идти о банальном воровстве, что, с точки зрения законов людских и божьих (не говоря уже о государственных) гораздо хуже нарушения закона о правах потребителя. Если же припомнить цену на <<коробочную>> XP~--- покупателю впору решить, что~40 баксов за предустановленную Vista~--- это всё равно, что <<дешевле только даром>>.

Вопрос ценообразования на <<вынь-ды>> различного рода мы тут обсуждать не будем. Равно как и соответствие их цен идеалам абстрактной справедливости. Замечу только, что в сложившемся положении вещей немалая вина и тех, кто ныне за эту самую абстрактную справедливость ратует.

А как же предустановленный Linux, спросите вы меня? Тот самый Linux, который, по уверениям его адептов, во-первых, бесплатен, во-вторых, поддерживает весь спектр более или менее современного оборудования. Отвечаю.

Да, Linux бесплатен почти во всех своих проявлениях. Да, ядро этой ОС поддерживает почти всё распространённое оборудование. Но\dots Чтобы выполнялось второе условие, ядро должно быть сконфигурировано должным образом. А этим нужно заниматься <<всерьёз и надолго>>, как говаривал дедушка Ленин. Ибо предустановленный <<в лоб>> Linux, без <<заточки>> под конкретное <<железо>>, может только вызвать у пользователя стойкое отвращение к этой операционной системе. Вроде как рыбий жир, спасающий от авитаминоза, которым пичкали в детсадах людей моего поколения, привил стойкое отвращение ко всему, хотя бы отдалённо напоминающему рыбу.

Впрочем, и история о том, как нужно (точнее, как не нужно) предустанавливать Linux~--- совершенно особая. Когда-нибудь я к ней вернусь.

Так что~--- увы и ах, но у продавца компьютерной техники есть только два пути: или продавать эту технику с ОС, предустановленной производителем, или без операционки вообще (или с вариациями на тему FreeDOS, что однофигственно). Можно ли его за это осуждать?

Можно. Если встать на позицию пользователя, который ходит за покупками с собственным LiveCD с дистрибутивом Linux, содержащим ядро и initrd, заточенные под всё, что только способно запускаться при включении питания. Вы знаете много таких покупателей? Я~--- так ни одного (в том числе даже и себя, хотя набор Live CD, благодаря наводкам Владимира Попова, у меня очень обширный).

Нельзя. Если исходить из потребностей того самого покупателя, за свободу выбора которого ратуют сторонники запрещения в законодательном порядке продажи машин с предустановленной ОС. И который в результате всякого рода законодательных инициатив лишится того, что есть, не приобретя ничего взамен. Я далёк от мысли, что кто-то из законодателей или инициаторов лично займётся подгонкой хоть какого дистрибутива под спектр продаваемого <<железа>>~--- не царское это дело, в ядре ковыряться\dots

Вообще о законодательных инциативах можно говорить очень долго. Я этого делать не буду. Скажу только одно: когда и если (или если и когда) какие-либо законодательные инициативы по этому поводу будут приняты~--- нам останется только вспомнить слова д'Артаньяна по поводу почившего Мазарини: <<\dotsмы еще пожалеем о Мазарини>>.

Так что же, Лёха, спросите вы меня, бороться против предустановленных ОС (не называя имён) не следует? Что я вам на это скажу? Следует. Но только одним способом: по возможности не покупать там, где эти самые предустановленные ОС впаривают насильно. А покупать лишь там, где вам готовы продать либо <<голую>> машину, либо заменить предустановленную ОС по возможностям и потребностям.

Опять же предвижу возражение: вы там в своей Москалии зажрались. Потому как, если вам не понравилось в данной фирме, то всегда можно перейти через дорогу и зайти в фирму соседнюю. А как быть нам в Тьмутараканске, на который на весь есть одна-единственная фирма, торгующая компьютерами? Которая что хотит, то и творит?

Увы~--- на это я ничего не могу возразить, такова объективная реальность: у жителей зажравшейся Москалии выбор больше. Можно лишь вспомнить слова Линуса Торвальдса о цистернах и водопроводе. Кто-нибудь когда-нибудь его проведёт.

Нескоро? Да, может быть. Но, как говорили классики,


\begin{shadequote}{}
\dotsдо конца света еще миллиард лет\dots Можно много, очень много успеть за миллиард лет, если не сдаваться и понимать, понимать и не сдаваться.
\end{shadequote}

Давайте попробуем?

\section{Без-Win’ные машины: продолжение} 

\begin{timeline}Март 28, 2009\end{timeline}

Высказав в прошлой заметке почти всё, что я думаю о компании против предустановленной Windows на ноутбуках в технологическом аспекте, я полагал эту тему закрытой (для себя лично, разумеется). Однако продолжающееся обсуждение заставиля меня возвратиться к ней.

Тем более, что тема эта получила продолжение: в газете \textbf{Коммерсантъ}  \textnumero53(4108) от 26.03.2009 была опубликована  статья под названием <<ФАС облегчит отказ от Windows>>. Следом подобные же сообщения прошли по многим онлайновым изданиям, однако ничего принципиально нового, по сравнению с ней, они не содержат.

Прежде чем перейти к сути вопроса, процитирую фрагмент из комментария Владимира Попова к моей предыдущей заметке:


\begin{shadequote}{}
Комплектация СВТ~--- как полигон для утверждения стандартов гражданского общества? А с остальным у нас всё уже о.к., ага?
\end{shadequote}

Так что нынче я ни слова не скажу о технологии, а только о пресловутом человеческом факторе. Вдохновляемый, опять же, цитатой из комментария Владимира Попова:


\begin{shadequote}{}
А как вам такое предположение: успешная “борьба” в этом направлении среди, прочих следствий, будет иметь: 1) неприязнь массового продавца/покупателя к FOSS и всему, что с этим как-то ассоциируется, и 2) укрепление позиций MS, как “единственного” гаранта того, что купленный современный ноут будет таки работать в соответствии с его спецификацией (что, вообще-то неудивительно, если производитель разрабатывает модель под определённую ОС).
\end{shadequote}



Пересказывать сообщения Коммерсанта и других СМИ я не буду. Напомню лишь, что в той же газете несколько ранее~--- № 25(4080) от 12.02.2009~--- прошло сообщение: <<ФАС протестирует Windows>>. Из которого следовало, что ФАС (Федеральная антимонопольная служба) начала проверку компаний Acer, Asus, HP, Samsung, Dell и Toshiba на предмет того, предусмотрена ли для их ноутов возможность отказа от предустановленной Windows и существует ли штатная процедура возврата денег за неё покупателю.

Приводятся и первые результаты этой проверки. Которые показали, что из всех проверяемых лишь ASUS смог представить документы, описывающие процедуру возврата денег (запомним на будущее~--- представить документы, а не примеры возврата), у всех остальных таковая не предусмотрена вовсе.

Обо всём прочем~--- дальнейших планах ФАС и отсутствии у них претензий непосредственно к фирме Microsoft, объяснениях представителей проверяемых фирм, всякого рода численных расчётах и так далее~--- заинтересованные лица могут прочитать сами (да, скорее всего, уже и прочитали). Остановимся только на карательных мерах, которые могут последовать в отношении возможных нарушителей, ибо именно они наиболее интересны для нас~--- конечных пользователей.



Но сначала зададим себе ещё один вопрос, которым не озаботилось ни одно из СМИ, обращавшихся к этой теме: почему именно эти шесть фирм стали предметом проверки? Ведь в прайсах наших крупных фирм, торгующих ноутбуками, можно найти и Lenovo, и Sony, и LG, и MSI, и Fujitsu-Siemens и BenQ. Не говоря уже об отечественных Roverbook'ах и неких ноутах от Эльдорадо.

Почему они не попали в список? Положим, отечественного производителя решили пока не трогать. Для остальных же~--- возможно, вследствие маленькой доли на рынке. Хотя в отношении Lenovo, Sony и Fujitsu-Siemens я сомневаюсь, что она столь уж мала. Или, по данным ФАС, у них заведомо всё в порядке с возвратом денег за OEM Windows? Тоже сомнительно. Например, Lenovo не так давно публично отказалась от предустановки Linux на свои машины~--- не с DOS же они их поставляют? А Sony, сколько я помню, всегда столь же публично заявляла о том, что их машины принципиально совместимы только с Windows Vista и ни с чем иным. И в любом случае, если речь идёт о борьбе за вселенскую справедливость в данном секторе Мироздания, проверять надо бы всех. Разве что для BenQ сделать исключение~--- именно для этой фирмы документально зафиксирован факт возврата денег за отвергнутую предустановленную Windows.

Сразу скажу, ответа на этот вопрос у меня нет.

Вернёмся, однако, к нашим баранам, точнее, к их шерсти. А именно~--- к тем карательным мерам, которые будут применены к возможным нарушителям, когда и если (или если и когда) их действия будут признаны нарушающими антимонопольное законодательство, по этому факту будут возбуждены соответствующие дела, эти дела завершатся нашей победой\dots ну, подумав, можно найти ещё несколько и <<когда>> и <<если>>. В отношении их последуют штрафные санкции в весьма крупных размерах. Плюс к тому им таки придётся возвращать деньги всем, пожелавшим обратиться с подобным требованием.

Чем ответят на это производители? Есть несколько вариантов ответа:

\begin{itemize}
	\item установление специальных дилерских цен для России, включающих в себя вероятные штрафные санкции и процент возврата (такое уже бывало в истории индустрии); 
	\item официальный ввоз в Россию только машин с предустановленными FreeDOS (а это исключительно бюджетные модели) и Linux (а спектр этих моделей крайне ограничен); 
	\item передачу всего дела на откуп <<серым дилерам>>~--- и такое тоже было на нашей памяти. 
\end{itemize}



Кто от этого выиграет? Не знаю. А вот круг проигравших очертить не сложно: это будут все покупатели ноутбуков. Ибо первый вариант повлечёт за собой рост цен, второй~--- сужение ассортимента, третий~--- отсутствие гарантии на аппаратную часть.

Кстати, при втором сценарии можно ожидать того, что машины с предустановленной Windows Vista будут доставать по блату~--- ибо только они могут оказаться, как сказано Владимиром в приведённой выше цитате а)  мощными, б) современными и при этом в) гарантированно работоспособными для простого пользователя, не имеющего сертификата от сибирских шаманов. Поневоле приходит на память хрестоматийная сцена из фильма <<Берегись автомобиля>>:


\begin{shadequote}{}
-- Мне нужен магнитофон заграничный, американский или немецкий -- Вот есть очень хороший, отечественный\dots -- Нет, спасибо, отечественный не подойдёт. -- Заграничный надо изыскивать\dots
\end{shadequote}

Кстати, пока суть да дело, непосредственные продавцы компьютерной техники на нашей Родине тоже дремать не будут: ведь по закону о правах потребителя конечный пользователь вправе подать в суд именно на них. Пользователь же, начитавшись статей о том, что <<предустановленная винда~--- это плохо>>, да к тому же, подчас, заменивший штатную Vista на пиратскую XP, в массовом порядке бросится отстаивать свои законные права. Суды же, состоящие из таких же пользователей, могут в столь же массовых масштабах начать выносить решения в пользу истцов. Куды же податься бедному барыге? Правильно, он будет тихо и спокойно возвращать деньги за предустановленную Windows. Предварительно прикинув возможный процент возврата, округлив в большую сторону и включив в цену, чтобы покрыть свои возможные потери сполна. А может быть, и с лихвой. И это всё мы уже не раз видели за двадцать лет существования российского капитализма.

И на него найдётся управа в виде тех же антимонопольных органов, скажете вы мне. Отнюдь.  Страховаться от возможных рисков~--- право продавца. Просто вряд ли найдётся мудрец, который включит в цену страховку от цунами в центре Москвы, обрушившегося непосредственно на его магазин. А от риска возврата денег за OEM Windows будут страховаться все~--- так что никакой монополии.

Если же ещё вспомнить, что всё это время будет развиваться бурная деятельность, собираться и заседать комиссии, привлекаться эксперты, и вся эта деятельность будет вестись на деньги налогоплательщиков~--- в проигрыше окажутся не только покупатели ноутов, но и десктопов, сборщики машин из копмлектующих, и даже те, кто прекрасно живёт без компьютера и не думает его себе покупать.

Разумеется, никакого апокалипсиса не произойдёт: всё это, в разное время и в разных сферах,  мы уже проходили и переживали~--- переживём и это. Вот только рядовой пользователь компьютера (и даже безкомпьютерный гражданин) будет думать о пользователях FOSS: это те самые\dots ээээ\dots нехорошие люди, которые заварили всю эту бучу. И поминать их добрым и ласковым Русским Словом. А с ними заодно~--- и пресловутю свободу выбора вместе со стандартами гражданского общества.

\section{Еще раз о без-win’ных машинах: кому и зачем они нужны?} 

\begin{timeline}Апрель 7, 2009\end{timeline}

Не утихающее обсуждение предустановленной Windows вынуждает меня ещё раз обратиться к этой теме. Кое-где мне придётся по ходу дела повторить сказанное в предыдущих заметках~--- в крамольных соображениях и их продолжении, однако в несколько другом аспекте.

Но для начала~--- несколько слов о том, зачем я всё это пишу. Ибо лично меня проблема предустановленной Windows не волнует ни с какого бока. Поскольку:

\begin{enumerate}
	\item во-первых, сам по себе факт наличия оной, вместе с наклейками, меня совершенно не трогает, и я не считаю машину этим фактом осквернённой; 
	\item во-вторых, стоимость предустановленной Windows я действительно полагаю стремящейся к нулю, поскольку она всегда может быть скомпенсирована подбором фирмы с более низкой ценой и (или) более выгодным для покупателя курсом внутреннего пересчёта уёв в буи (то есть единиц условных~--- в безусловные рубли); 
	\item в-третьих, если и когда (или когда и если) мне так уж захочется приобрести ноутбук без Windows~--- я знаю, где и как это сделать;
	\item наконец, в четвёртых~--- очень немаловажный для меня момент: всем моим ноутам уготована судьба рано или поздно быть проданными или подаренными при замене на новый; и вовсе не факт, что новые владельцы их будут пользователями Linux'а; тут-то и наступает психологический момент для восстановления <<родной>> системы с комплектного дистрибутива или имиджа; хотя сам я после покупки первым делом сношу предустановленную Windows к чертям собачьим, но все штатные диски храню вместе с руководствами.
\end{enumerate}


Собственно, к данной проблеме меня побуждают обратиться следующие моменты. В первую очередь, психологический. Что заставляет пользователей участвовать в акции, польза от которой лично для них будет сомнительна?

Второй момент~--- мне хочется поддержать тех здравомыслящих пользователей, которые, понимая не очень большой смысл данной акции, тем не менее, стесняются говорить об этом вслух: как же так, все за свободу, а я, как Баба-Яга, оказываюсь против?

И третий момент~--- уже технологический, поскольку я категорически не согласен с поставленной целью, которую преследуют организаторы акции и, так сказать, группа поддержки. Но об этом подробнее я скажу позже.

И так, давайте посмотрим, кому таки нужен отказ от предустановленной Windows?

Напрашивающийся ответ: пользователям более иных операционных систем, которые Windows не использовали, не используют и использовать никогда не будут~--- к их числу принадлежит и ваш покорный слуга. Казалось бы, уж они-то от этой акции должны выиграть?

Самое парадоксальное в этой истории, что напрашивающийся ответ~--- не верен. У меня достаточно широкий круг знакомств в этой сфере, как реальных, так и виртуальных. И подавляющее большинство из них к проблеме предустановки Windows более чем равнодушны~--- и примерно по тем же причинам, что и автор этих строк. Разумеется, есть исключения, обусловленные либо чисто религиозными соображениями (дабы скверна Windows не коснулась их машины), либо стремлением вернуть таки деньги за ненужный товар. Хотя у меня есть глубокое подозрение, что в последнем случае людьми движет чисто спортивный азарт~--- что же, это чувство я хорошо понимаю и уважаю. Только я бы свой азарт обратил всё-таки на что-нибудь более конструктивное.

Наконец, судя по многочисленным обсуждениям, есть пользователи, которые в такой ситуации действуют чисто из принципа~--- дабы мой полтинник не достался акуле капитализма Биллу Гейтсу. Что же, и это чувство понятно, но уж тут никаким конструктивом и не пахнет вообще.

Повторяю, все три рассмотренных случая~--- буквально единичны. Подавляющее же большинство лично знакомых мне пользователей Linux'а (или BSD~--- в данном случае это не важно) не просто индифферентно относятся к предустановленной Windows, но и крамольно не удаляют ей со свежеприобретённых ноутбуков, и даже временами используют в тех или иных целях. Чаще всего~--- для игр, но кое-кому эта ОС требуется и в производственных целях, по долгу службы.

Следующая категория пользователей, интересы которых призвана блюсти описываемая акция~--- это пользователи начинающие, возможно покупающие первый свой ноутбук в жизни. И которым просто жизненно необходимо предоставить свободу выбора~--- между предустановленной Windows, FreeDOS, Linux или машиной без всякой ОС вообще. Благородная цель, не правда ли? Однако пойдёт ли на пользу этой категории поупателей пресловутая свобода выбора?

Рискну предположить~--- нет: машина без ОС для начинающего пользователя не более чем кусок пастика, текстолита и прочих материалов. Современный ноут с предустановленной FreeDOS, эквивалентен машине без ОС вообще, да и не проблема это~--- без всяких акций найти подходящую модель с FreeDOS. Что же касается Linux'а~--- то тут с силу вступает технологический фактор, о котором речь пойдёт дальше. Так что фактически вся свобода выбора для начинающего пользователя сводится всё к той же предустановленной Windows.

Третья категория покупателей~--- опытные пользователи Windows, привыкшие к интерфейсу Windows~2000/XP, не желающие менять его на новомодные рюшечки и бантики Windows Vista. Насколько я знаю по рассказам, это примерно то же самое, что для KDE'шника со стажем сменить версию KDE~3.5.X на 4.X, так что и их я хорошо понимаю. Как и резоны избавиться от ненужной ОС, заменив её любимой и привычной.

Однако их положение и сейчас вовсе не безвыходно. Во-первых, как только что было отмечено, можно приобрести машину с предустановленной FreeDOS, после чего заменить её на любимую Windows~XP. Разумеется, лицензионную и коробочную~--- к случаю контрафактного софта мы вернёмся чуть позже.

Далее, существуют модели ноутбуков, несущие на винте предустановленную Vista, но комплектуемые Recovery DVD Windows~XP Professional (рус.)~--- специально для тех, кто с претензиями, и не желает пить лосьон <<Свежесть>> вместо коньяка в международном аэропорту <<Щереметьево>>.

Наконец, ряд фирм штатно предлагают такую услугу~--- замену предустановленной Vista на Windows~XP. Разумеется, опять же гарантированно лицензионную.

Получается, что акция по отмене предустановки Windows Vista не нужна ни записным линуксоидам, ни убеждённым пользователям Windows, ни даже информационно неполовозрелым начинающим пользователям, не знающим даже ключевых слов~--- ОС, предустановка, OEM и тому подобных. Кто же остаётся в сухом остатке как целевая аудитория акции?

Методом исключения ответить легко: это отпетые вендузяднеги (не путать с настоящими пользователями Windows, это четыре большие разницы), принципиально применяющими контрафактный софт. Единственной целью которых является получение компенсации за предустановленную ОС и инсталляцию пиратской Windows любого рода~--- вполне возможно, то же самой Vista. Да-да, господа организаторы акции~--- как бы вы от этого не открещивались, но объективно вы защищаете их интересы, и больше ничьи.

Внимательный читатель наверняка обратил внимание, что при перечислении объектов (или субъектов?~--- трактуйте, как угодно) акции я забыл категорию самую главную: её организаторов. Нет, не забыл. Более того, рискну утверждать, что это~--- единственная категория, за исключением пиратов, которой эта акция нужна по настоящему.

Зачем? Вариант личного пиара рассмотрим лишь теоретически. Хотя, после блистательного провала идеи Национальной Операционной Системы (НОС), почва для него представляется вполне благодарной. Действительно, если создание НОС~--- дело сложное, хлопотное, неблагодарное и не поддержанное никем, то идея возврата кровных денег за OEM Windows заведомо привлечёт к себе стронников~--- хотя бы тех, кто, как я говорил, готов приплатить, лишь бы Биллу Гейтсу не досталось. Так что популярность акции в соответствующих кругах была гарантирована.

По сему поводу не могу отказать себе в удовольствии процитировать меткое высказывание моего старого товарища Евгения Чайкина aka StraNNicK, правда, по более иному поводу:


\begin{shadequote}{}
В общем, прежде чем присоединиться к какой-либо акции~--- подумайте. В большинстве случаев, заявленных целей можно достичь сугубо своими силами, причём куда эффективнее.
\end{shadequote}

Однако я не настолько циничен, каким мне следовало бы быть. И хочу верить, что по крайней мере некоторыми из организаторов акции движут иные побуждения~--- например, всё тот же спортивный азарт. Как я говорил, чувство понятное и достойное уважения~--- но, ИМХО, направленное не в то русло.

Какое же русло то? Для этого вспомним, каков главная цель акции: добиться отчуждения аппаратных и программных средств друг от друга. И вот это тот самый момент, который побудил меня продолжить цикл настоящих заметок.

Потому что время отчуждения прошло, и мы по спирали вернулись к тому, с чего начиналась IT-индустрия~--- развитию в сторону единства программных и аппаратных средств .

Современный ноутбук~--- это действительно программно-аппаратный комплекс, пригодность к применению компонентов которого обеспечивается софтом не в меньшей степени, нежели <<железом>>. Да, ныне это гарантируется только одной-единственной (лучшей и величайшей) ОС. Но тут мы вступаем на технологическое поле, на котором происходят совсем другие истории.

\section{Без-win’ные машины: закон есть закон?} 

\begin{timeline}Апрель 8, 2009\end{timeline}

Следующую заметку без-win'ного цикла я хотел было посвятить технологическим аспектам предустановки ОС. Однако дискуссии по теме как-то резко свернули в юридическую сторону, так что появилась необходимость высказаться сначала по этому поводу.

В прошлой заметке я не упомянул ещё одну категорию людей, заинтересованных в акции ЦеСТ: абстрактных правдоискателей, искренне полагающих, что законы должны выполняться. Чисто теоретически я с этим согласен. Но вот практически\dots

Для начала, не очень ясно, какой именно закон нарушает предустановка Windows на ноутбуки. Закон о защите прав потребителя? Казалось бы~--- безусловно. Но позволю себе процитировать представителя Общества защиты прав потребителей Дмитрия Лесняка из комментария к статье <<ФАС проверяет Acer, Asus, HP, Samsung, Dell и Toshiba>>:


\begin{shadequote}{}
Многое в данной ситуации зависит от того, как составлены товаросопроводительные документы. Так, если операционная система указана в качестве принадлежности устройства (условно говоря, как комплектующее изделие), комплектация сформирована изготовителем и отдельная её стоимость не указана~--- потребитель не вправе требовать продать ему компьютер без программного обеспечения. В то же время, если программное обеспечение (в терминологии ГК~--- лицензионное право) продаётся как отдельный товар с собственной ценой, но продавец навязывает его приобретение вместе с компьютером (кстати, возможна и обратная ситуация~--- навязывание покупки оборудования с софтом), такие действия, безусловно, противоречат Закону <<О защите прав потребителей>>.
\end{shadequote}

Похоже, что большинство участников вышеупомянутого обсуждения в юридической стороне дела не очень сильны. Признаюсь, я тоже, так как изрядную часть жизни прожил там, где закон~--- тайга, медведь~--- судья. Вследствие чего стал, по выражению одного из моих знакомых юристов ярко выраженным правовым нигилистом. И потому обращусь в авторитетному мнению Елены Тяпкиной~--- профессионального (и практикующего) юриста, не чуждого миру FOSS; помимо всего прочего, именно ей принадлежит адекватный перевод GPL v.2 на русский язык.

Так вот, в далёком для многих 2002 году на одном из заседаний Семинара Altlinux Елена делала доклад под названием <<Сравнительный анализ основных лицензий Open Source>>. Сам по себе он был очень интересным, однако в основном лежит за рамками нашей сегодняшней темы. А вот в процессе обсуждения была поднята тема о правовом статусе лицензий не свободных.

Так вот, Елена трактовала лицензии на программное обеспечение, как свободное, так и проприетарное, в рамках юридических понятий российского гражданского права как договор присоединения. Подробнее об этом говорится в другой её работе~--- <<Правовой статус GPL в России>>. Начну с цитаты из неё, касающейся GPL:


\begin{shadequote}{}
В соответствии с нормами Гражданского кодекса РФ, GPL~--- это договор присоединения, условия которого определены одной из сторон в стандартной форме и могут быть приняты другой стороной не иначе, как путем присоединения к предложенному договору в целом.
\end{shadequote}

Далее, Елена отмечает явное сходство её с пресловутой EULA, которая\dots


\begin{shadequote}{}
\dotsвступает в силу и становится обязательным для пользователя, если он вскрыл упаковку программного продукта или установил программное обеспечение на свой компьютер.
\end{shadequote}

То есть пользователя никто не обязывает присоединяться к договору, предложенному производителем конкретной софтины. Но уж если он совершил


\begin{shadequote}{}
<<конклюдентные действия>>, то есть действия, выражающие волю лица заключить сделку, но не в форме устного или письменного волеизъявления, а поведением, по которому можно сделать заключение о таком намерении\dots
\end{shadequote}

то тем самым он принимает условия производителя, каковы бы они ни были.

Применительно к нашему случаю это можно интерпретировать так: производитель ноутбуков, очевидно, заключает договор присоединения к лицензии Microsoft в силу самого факта предустановки Windows на свои машины, не так ли?

Далее, из высказывания Дмитрия Лесняка очевидно, что производителю вольно объявить предустановленную Windows неотъемлемой частью своего продукта. Хотя бы потому, что все компоненты ноутбука тестировались на совместимость только с этой операционной системой~--- ну не хотелось ему возиться с настройкой для них Linux'а. И, замечу в скобках, правильно не хотелось: все, кто видел Linux'ы, предустановленные на ноутбуки, вспоминают о них не иначе как о ночном кошмаре.

А далее производитель ноутбуков уже со своей стороны предлагает покупателю договор присоединения. Пусть не напрямую, а через своих дилеров, дистрибьюторов и прочих партнёров, а те, в свою очередь, через розничных продавцов. Причём заметим, что насильно товар ему не всучивается. Но сам по себе факт покупки машины, в спецификации которой русским (английским, зулусским или маорийским~--- нужное дописать) языком сказано, что она несёт на себе предустановленную Windows, можно рассматривать как то самое конклюдентное действие. В противном случае покупателю вольно отказаться от покупки машины данного производителя или данного продавца, и обратиться к тому продавцу или производителю, который предалагает ему иной договор присоединения~--- например, на условиях GPL в случае предустановки FreeDOS или Linux. Но без договора присоединения покупатель не обходится и в этом случае.

По моему скромному мнению (и это мнение не только моё), ни производитель, ни все промежуточные звенья в цепи распространения не делают ничего, что противоречило бы законам людским и божьим, равно как и здравому смыслу. А вот почему в качестве предустановленной ОС на большей части ноутбуков предустановленной оказывается именно MS Windows, и лишь на подавляющем меньшинстве~--- FreeDOS или Linux, вопрос совершенно иной.

Ситуация прекрасно описывается в терминах преферанса: взявший прикуп объявляет игру, партнёрам вольно пасовать или вистовать. Но уж если кто-либо из партнёров вистует~--- он обязан взять положенное количество взяток, иначе получит в гору. Причём правила при торговле ноутбуками гуманней по отношению к покупателю: никакого <<обязона>> (иначе говоря, <<сталинграда>>), так как даже покупатель супербюджетного ноута (каковой можно приравнять к <<шести пикам>>) имеет возможность выбрать между минимум тремя предустановленными операционками (иначе говоря, <<уйти за свои>>), или даже отказаться от вистов (сиречь покупки) вообще.

Из сказанного, ИМХО, ясно, что пытаться изменить существующее положение вещей законодательными или полицейскими мерами бесполезно и бессмысленно. Изменить его можно только комплексом мер технологических и информационно-просветительских.

P.S.В заключение авансом отвечу на возражение, неоднократно звучавшее в ряде обсуждений: если производитель рассматривает ОС (любую, но в данном случае Windows) как часть программно-аппаратного комплекса, то пусть будет любезен предоставлять гарантию не только на <<железо>>, но и на предустановленный софт. В пример чему с удивительным постоянством приводится Apple, который якобы именно так и поступает в отношении своих Mac'ов.

Для начала развею последнее заблуждение. Я специально проконсультировался с одним из немногочисленных лично мне знакомых пользователей Mac'а~--- Леонидом Уточкиным aka Lentux (который по совместительству является старым линуксоидом). В своём письме он процитировал Гарантийный сертификат, который прилагался кего машине. Цитирую вслед за ним:

\begin{shadequote}{}
Программное обеспечение, поставляемое вместе с продукцией Apple Computer, Inc., не подлежит гарантийному обслуживанию.

Apple Computer, Inc. и сервисные центры не несут ответственности за сохранность Вашей информации и программного обеспечения. Переустановка и восстановление программного обеспечения не входит в список работ, предусмотренных гарантийным обслуживанием.
\end{shadequote}


Что, исходя из здравого смысла, опять-таки не может вызвать возражений: ведь подавляющее большинство <<повреждений>> предустановленного софта связаны либо с заражением вирусами, либо с шаловливым рукоблудием пользователя. А уж брать на себя обязательства в том, что у последнего руки растут непременно оттуда, откуда надо, не станет ни один здравомыслящий человек\dots

\section{Так Pro или Contra?} 

\begin{timeline}Август 28, 2009\end{timeline}

Как ни надоела мне тема предустановленной Windows и сопуствующие ей материи~--- а время от времени приходится к ним возвращаться. Правда, в данный момент~--- с позиций консенсуса, а не конфронтации, что не может не радовать.

Эта заметка сочинилась под впечатлением материала Сергея Голубева <<Pro et Contra>> в его блоге. Собственно, она представляет собой нечто вроде развития итогового её вывода, каковой позволю себе процитировать:

\begin{shadequote}{}
\dotsстранная у людей маркетинговая стратегия. Товар, вроде, есть. Но ни описаний, ни обзоров, ни тестов\dots Кот в мешке, однако. Традиционные изделия так продать можно, а вот пробиваются на рынок обычно более агрессивно.Ну допустим~--- запретят Микрософту ставить OEM. Неужто кто-то думает, что от этого будут лучше продаваться <<патриоты>> и <<инфрабуки>>.
\end{shadequote}

Что напомнило мне старый анекдот. Год 1917. Дама просыпается от шума и спрашивает свою горничную:

\begin{shadequote}{}
-- Что случилось? \\
-- Революция, народ вышел на улицы. \\
-- И чего они добиваются? \\
-- Чтобы не было богатых. \\
-- Странные люди. Мой прадед вышел на Сенатскую площадь для того, чтобы не стало бедных\dots
\end{shadequote}

Что ж, таково поведение большинства революционеров~--- и OEM-революция тут не исключение. Я бы сформулировал это примерно так:

Можно тратить время и энергию на \textbf{убеждение} в том, что машины с предустановленной Windows~--- это плохо. А можно~--- на \textbf{объяснение} того, что машины с Linux'ом~--- это хорошо.

Второй способ не просто конструктивней~--- он дешевле. Так как требует только личных усилий. А всякая революция начинается с создания инфраструктуры. И рано или поздно обрастает генсеками, членами политбюро, активистами на местах. Я уж не говорю о бухгалтерии для обеспечения жизнедеятельности всех перечисленных. По сравнению с этой Батыевой ратью Гейтс с Балмером покажутся скромнягами-бессребрениками.

\section{FOSS на Руси: революционная ситуация?} 

\begin{timeline}Апрель 11, 2009\end{timeline}

В приснопамятном обсуждении OEM'ной Вынь-ды от одного из организаторов акции ЦеСТа некогда прозвучал призыв~--- говорить на чистоту. Я долго воздерживался от этого предложения, дабы остаться в рамках пресловутой политкорректности~--- что делать, заразителен дурной пример некоего государства, в котором лопату предпочитают не называть лопатой, а для простого русского слова, которое дети любят писать на заборах, могут придумать такое количество эвфемизмов, перед которым блекнет лексикон классической арабской литературы и <<Тысячи и одной ночи>>.

Однако тон обсуждения адептов акции становится всё более агрессивным, что вынуждает меня таки поговорить на чистоту~--- заодно расставив все точки над i в моём к ней отношении. И, как показывают отклики, не только моём.

Интересно, что таковой тон принимают даже организаторы дискуссии~--- причём местами на олпанской мове, не являющейся родной ни для одного из участнегов диалога.

Но сначала~--- немного истории. Каковая, как известно, в силу категорического неусвоения её уроков, имеет обыкновение повторяться: сначала~--- в виде трагедии, затем~--- в виде фарса (возможно, также вполне трагического), а потом уже~--- в форме откровенной травестии. Что же, нам остаётся ответить на это бурлеском.

Революции в истории человечества происходили неоднократно. Классики вечно живого всепобеждающего (потому что верного) учения утверждали, что они происходят тогда, когда создаётся революционная ситуация: верхи не могут, низы не хотят. Но, как метко отметил творческий гений советского народа (подозреваю, что и анти-советского тоже), это ситуация импотента на фригидной женщине: одна сторона не могёт, другой всё пофигу. И потому никакой революцией данная <<революционная ситуация>> не чревата. А чревата она вялотекущим процессом, также описанным коллективным творческим гением в диалоге босса и секретарши:

\begin{shadequote}{}
-- Иван Петрович, Вы уже ввели? \\
-- Кажется, да\dots \\
-- Ой как хорошо, ой как приятно!
\end{shadequote}

Читатель вправе спросить меня~--- опровергая общепризнанных классиков марксизма (величие которых признают даже признанные анти-марксисты), что же ты предложишь взамен? Отвечаю: это проще показать на примере, нежели сформулировать в научных терминах. Может быть, социологи грядущих столетий, если таковые будут отпущены человечеству, это сделают~--- уже без гнева и пристрастия; впрочем, беспристрастие маловероятно и по прошествии тысячелетий\dots

Итак, пример: 1917-й год, канун Октябрьского переворота, который потом назовут ВОСРом. Что мы наблюдаем в контексте революционной ситуации? А мы наблюдаем примерно следующее:

\begin{itemize}
	\item кучку будущих комиссаров в кожанках с маузерами в деревянных кобурах-прикладах; правда, повторяю, всё это в будущем~--- в кожанки с маузерами они обрядятся, когда захватят армейские склады, подготовленные для летнего наступления 17-го года; 
	\item братишек-матросов, просидевших всю войну на кораблях на приколе, мающихся от безделия, но, за отсутствием конструктива, не способных ни к чему, кроме деструктива откровенного; 
	\item серой вооружённой массы, которой на самом деле всё до лапочки Ильича (каковая ещё только вызревает в светлых мозгах комиссаров. 
\end{itemize}



Комиссарам очень хочется власти~--- собственно, ради того они в комиссары и подались. Братишкам флотским~--- хочется развернуть руку, да раззудить плечо, цель рояля не играет. Ну а серой вооружённой массе хочется одного. Того самого, что выразил Юлий Ким aka Михайлов:

\begin{shadequote}{}
Наплявать, наплявать\\ 
Надоело воевать\dots \\
Были мы солдаты, А теперь до хаты-ы-ы-ы\dots
\end{shadequote}

Что дальше? Дальше~--- Братишки-матросы рвут на груди тельники, подымая в атаки те самые серые массы. Ну а комиссары в кожанках и с маузерами сидят в глубоком тылу (по возможности под охраной латышских стрелков или китайских <<интернационалистов>>). Употребляя свои маузеры разве что для постреливания в затылок тем из серой массы, кто почему-то не испытывает большого желания воевать за светлое будущее всего человечества.

Чем закончилось дело~--- общеизвестно: водворением социализма на одной шестой части суши. Так что об этом не будем. А обратимся к более близкому примеру, который памятен некоторым из здесь присутствующих~--- смены парадигмы: от построения светлого коммунистического будущего в масштабах всего человечества к слиянию со всем прогрессивным капиталистическим человечеством. Что наблюдалось нами уже воочию~--- на рубеже 80-х и 90-х годов последнего века прошлого тысячелетия.

Промежуточные стадии~--- процесс превращения комиссаров в парт- и госчиновников, сменивших кожанки на номенклатурные костюмы, братишек~--- в павших смертью храбрых национальных героев, а также сезоны отстрела возомнивших о себе тех и других~--- опустим, как не имеющие отношения к теме. И обратимся ко второй революционной ситуации минувшего века~--- рубежу 80-х и 90-х годов. Когда комиссарам захотелось покоя~--- а именно, смены номенклатурных <<Чаек>> и <<Волг>>, которые могли в любой момент отобрать вместе с должностью (а то и головой, как порою случалось), на частнособственные <<Мерседесы>> и <<БМВ>>, номенклатурных окладов~--- на личные счета в зарубежных банках, номенклатурных дач~--- на собственные коттеджи в Подмосковье или, ещё лучше, на виллы в Дальнем Забугорье.

И потому из парторгов ставшие банкирами и коммерсантами.

Нашлись и братишки~--- получившие имя братков, а затем и брателл, они тоже изменили форму одежды: вместо клешей и тельников одели <<Адидасы>> в фирменном или китайском исполнении.

Их теперь называли бандитами~--- и вовсе не всегда в ругательном смысле.

Не было серой массы: с одной стороны, массы эти были безоружны, с другой, выработали не сформулированное официально, но самое передовое в мире учение пох\dots фигизма. Потому революционная ситуация как бы рассосалась сама собой. Не совсем без жертв, конечно, но всё-таки относительно мирно. Иными словами, ситуация фарса, хотя иногда и вполне трагического.

Это я всё к чему? Ныне в российском секторе общемирового FOSS-сообщества наблюдаем уже совсем комичное воплощение всего того, что предшествовавшие поколения и поколение наше пережило в течении двадцатого века.

До недавнего времени FOSS-сообщество России жило и развивалось само по себе~--- подобно чистой фундаментальной науке, не до конца про-официозной литературе и прочим сферам культуры и искусства, равно как и их поп-ответвлениям. Развивалось вполне успешно и целенаправленно~--- не смотря на отсутствие внимания к нему со стороны власти предержазих. Или, может быть, как раз благодаря отсуствию такового.

И вдруг в одночасье случилось~--- государство задумалось о том, что великая держава не имеет ничего своего, кроме нефти, газа и ещё кое-каких природных ресурсов. То есть: работники госаппарата, отказавшись от своих номенклатурных машин от отечественного производителя, пересели на иномарки, исчисляют свои доходы не в родимых рублях и копейках, а в каких-то импортных баксах (подозрительно похожих на доллары недавнего идеологического противника), пользуются ноутбуками, изготовленными\dots да кто его знает, где. И в довершение ко всему, на ноутбуках этих предустановлена операционная система вполне определённого происхождения.

Разумеется, менять иномарки на продукицю АЗЛК или Тольятти бывшим товарищам, ныне господам, не хотелось, отказываться от ноутов Sony~--- тоже, да и менять предустановленную Windows на непонятный Linux~--- тоже. И тогда в действие вступает оппозиция~--- те из психологических парторгов, которые не попали в первую обойму и потому всеми вышеперечисленными аксессуарами не обзавелись.

Началось всё, естественно, с самого слабого звена~--- с операционных систем. Благо сложившаяся ситуация~--- доминирование в этой сфере продукции самой великой софтверной компании~--- давала к тому массу поводов.

И раздаются призывы, с одной стороны, к внердрению существующего свободного программного обеспечения в сферах, финансируемых государством, с другой~--- к созданию Национальной Операционной Системы (сокращённо НОС~--- копирайт на эту аббревиатуру оставляю за собой).

Правда, быстро выяснилось, что дело это~--- сложное, хлопотное, вполне бессмысленное (ибо создать что-то принципиально более иное, ввиду развала настоящей фундаментальной науки, в том числе и той, что называется Computer Science, было не реально) и неблагодарное: вследствие перечисленных причин немедленных политических дивидендов оно не сулило.

И потому принялись за акцию, обещающую таковые: борьбу за свободу выбора ОС для конечного пользователя.

Разумеется, занялась всем этим кучка комиссаров, не вполне удачливых на своём комиссарском поприще. Далее~--- всё по многократно апробированному сценарию: вербовка по многочисленным форумам группы братишек, по преимуществу своему~--- тех, кто, как говорилось в предыдущей заметке, только недавно слез с подоконника и потому словом и делом готов доказывать свою верность новому учению. Логическое завершение чего~--- та самая пресловутая акция. Но это лишь одна сторона вопроса.

Для освещения другой стороны придётся таки вернуться к вопросу о засилье заморской софтверной корпорации во всех сферах, подведоственных государственному бюджету. И, соответственно, декларации соответствующими структурами поддержки свободного софта. И тут в секторе отечественного FOSS-сообщества впервые ощутимо запахло баблом. Не очень большим~--- но на запах которого потянулся отечественный бузинес. В том числе и тот, который принято называть большим.

Вообще удивительно, сколько за последнее время обнаружилось радетелей свободного софта. Причём ещё с древних времён. И как тут не вспомнить персонажа одного из рассказов Георгия Фёдорова~--- деда, который после прослушки оперы <<Иван Сусанин>> задумчиво сказал:


\begin{shadequote}{}
Теперь много таких находят, которые ещё встарь за советскую власть стояли.
\end{shadequote}

Радетелй свободного софта, стоявших до него со времён Очакова и покорения Крыма (а то и Куликовской битвы), можно отыскать ничуть не меньше. Кроме братишек-матросиков с Линуксфорума, в это число можно записать и отмеченных выше крупных бизнесмеев. Не так давно был сильно умилён фразой из интервью с Дмитрием Комиссаровым. Она настолько показательна, что процитирую:


\begin{shadequote}{}
Мы довольно давно идем в сторону СПО. В частности, еще в конце 90-х была попытка портировать купленный у компании <<Микроинформ>> Лексикон под Linux и открыть коды этой программы. К сожалению, после кризиса 98-го года отрасль впала в некое депрессивное состояние и проект был закрыт.
\end{shadequote}

Очень улыбает: так что же было~--- попытка портировать или попытка открыть? Да и Lexicon на фоне Linux'а образца 1998 года выглядит достаточно забавно.

Кстати, что интересно~--- на те же около-дефолтные годы падает активность IPLabs Linux Team (ныне Altlinux); им, почему-то, депрессивное состояние не помешало. При всём моём уважении к деятельности Урбансофта и его <<Открытому ядру>>~--- именно в это время и именно IPLabs Linux Team заложил предпосылки для массового применения Linux'а. Среди тех масс, которым он по настоящему нужен, разумеется\dots

Ну да бог с ними, с крупными бизнесмеями~--- работа у них такая, извлекать чистоган из всего. В том числе и из открытого софта. Тем более, что от их бизнесмейской деятельности вроде как даже польза бывает~--- налогами, толика которых косвенно обламывается и для Open Source: в виде грантов академическим учреждениям, проектов по Linux'изации школ и ВУЗов, и так далее. Конечно, забавно, когда телегу ставят впереди лошади, и проект портирования Lexicon'а полагают соизмеримым со свободными Vim или Joe, юзаемым поколениями юниксоидов\dots Ну да, повторяю, господь им судья.

Вернёмся к нашим политикам от Open Source. Тем более, что эти господа являются не столько налогоплательщиками, сколько налогопользователями. Причём не только политики действующие~--- но и, так сказать, теневые. Потому как эти самые теневые сочиняют, дабы эмулировать свою политическую активность, всякого рода запросы и обращения. На которые действующие политики, для поддержания своего статуса, время от времени должны реагировать~--- примерно так, как мы видим в настоящий момент с проверкой ФАС.

Политиков тоже можно понять~--- это тоже издержки профессии, радеть за счастье всего человечества. В первую очередь, конечно, за счастье лучших его представителей~--- себя, любимых. Но это реализовать в полной мере могут только политики действующие. А вот теневым политикам нужно отрабатывать всеобщее счастье по полной программе. Желательно, разумеется, с наименьшими накладными расходами~--- вроде возврата денег за предустановленную Windows\dots

Тоже, в общем-то, не смертельно: как говорится, чем бы дитя ни тешилось, лишь бы с ножом на дорогу не выходило. Или не звало выходить с гранатомётом на баррикады. А только отметиться в опросе\dots

Но только вот не надо бы представлять всё ту же самую акцию как нечто общеполезное~--- раз, нечто способствующее процветанию свободного софта~--- два, и вообще нечто, способствующее чему-либо, кроме личной популярности организаторов~--- три. Каковые и хотят произвести маленькую, локальную революцию в мире свободного софта в одной отдельно взятой стране.

Сапиенсам~--- надеюсь, sat?
\chapter{Разное}

\textsl{В этой рубрике собраны заметки разных лет, тематическую принадлежность которых я определить затрудняюсь.}

\section{Апокалипсис для СПО?} 
\begin{timeline}
Июнь 15, 2009
\end{timeline}

Настоящая заметка посвящена прискорбному событию, а потому я начну её цитатой, оное описывающей:

\begin{shadequote}{}
Во все российские школы исполнителями госконтракта о поставке СБППО (<<Первая ПОмощь>>) рассылаются диски ПСПО, по крайней мере два из которых, после добавления на них материалов без ведома <<Армады>> и <<Альт Линукс>>, непригодны к использованию.
\end{shadequote}

Источник цитаты~--- сообщение в блоге Алексея Новодворского <<О рассылке ПСПО внутри коробки ``Первой Помощи''>>, по ряду причин вызвавшее большой резонанс на всех ресурсах, связанных с FOSS и Linux. В частности, и обсуждение на форумах, которые и заставили меня написать настоящую заметку.

Сразу скажу, что она

\begin{itemize}
	\item резюмирует всё сказанное мною ранее в отдельных форумных трейдах; 
	\item выражает моё личное мнение, и потому не является приглашением к \href{http://linuxforum.ru/index.php?showtopic=94518}{дискуссии};
	\item не касается ни собственно проекта внедрения СПО в школах (далее~--- Школьного проекта) как такового, ни способов его реализации. 
\end{itemize}


А касается она преимущественно двух вопросов:

\begin{enumerate}
	\item методов обсуждения вопроса, послужившего его первопричиной, и 
	\item возможных последствий. 
\end{enumerate}

Начнём с методов, для чего придётся рассмотреть историю вопроса~--- благо, на сегодняшний момент она укладывается в несколько дней.

Итак, всё началось с указанного сообщения в блоге Алексея Новодворского, к коему и отсылаю тех, кто ещё <<не в теме>>. Для остальных сформулирую свое отношение к событию.

Да, поставка дефективных дисков~--- это плохо. Причём кем бы то ни было, кому бы то ни было и любых по содержанию, хоть порнушных (если они были заказаны через соответствующие службы). Не менее плохо, чем продажа тухлой колбасы или плохой водки. Но и не более плохо.

Поэтому возмущение Алексея и некоторая эмоциональность в его описании события более чем понятны. Тем не менее, итоговый вывод сообщения вполне сдержан:


\begin{shadequote}{}
Жадность и непрофессионализм.
\end{shadequote}

Что же, за годы пост-советской действительности к этому нам не привыкать.

Однако тут в действие вступает тяжелая артиллерия: не прошло и полусуток, как в блоге Виктора Алксниса появляется (2009-06-12 09:53:00) сообщение: <<Вести с полей: два непригодных к использованию диска за 17 миллионов рублей>>. Где, после изложения канвы событий в несколько расширенном виде к определению Алексея добавляется,


\begin{shadequote}{}
\dotsчто это типичное проявление КОРРУПЦИИ, царящей в коридорах власти.
\end{shadequote}

А завершается всё обращением:


\begin{shadequote}{}
Уважаемые друзья! Помогите вывести этот пост и пост Алексея Новодворского в ТОП, дайте у себя в блоге ссылку на них! Кроме того, по возможности, прошу оставить комментарий о происходящем в блоге у Д.~Медведева\dots
\end{shadequote}

Что же, призыв был услышан мгновенно. И реализован постом с устрашающим названием: <<Внедрение СПО в школах под угрозой>>. Где уже рисуется грядущая апокалиптическая картина:


\begin{shadequote}{}
Поймите, если школьный проект провалится (к чему идет дело), то будут провалены и планы перехода нашей страны на СПО.
\end{shadequote}

Ну и, разумеется прозвучавший ранее призыв Виктора Алксниса был подхвачен (а также расширен и угл\'{у}блен:


\begin{shadequote}{}
Вне зависимости от наших разногласий по тем или иным вопросам, сегодня самое главное спасти школьный проект. Спасем его, значит Linux в России победит!
\end{shadequote}

Который почти немедленно был <<ретранслирован>> (определение топикстартера) на более ином форуме: <<Внедрение СПО в школах под угрозой>>.

И реакция определённой части читателей была вполне предсказуема. И лучше всего может быть выражена сакраментальным


\begin{shadequote}{}
Пионеры наших бьют!
\end{shadequote}
из <<Республики Шкид>>. Что и было целью этой, по выражению \textsl{vinny}, тестовой акции.

Хорошо, спросите вы меня устами одного из участников обсуждения, sash-kan'а


\begin{shadequote}{}
как с твоей точки зрения должен вести себя в этой ситуации тот самый all? тихонько похихать в уголке?
\end{shadequote}

Если я сам же признаю, что ситуация с рассылкой дисков, мягко говоря, нехорошая? Отвечаю: для здравомыслящего человека есть два варианта поведения.

Вариант первый: глубоко вникнуть в тему~--- а она имеет обширную ретроспективу; причём вникнуть с разных сторон~--- ибо противостоянием героев и злодеев эта история не ограничивается. И уже после этого принять взвешенное решение и поступать соответственно своим убеждениям и темпераменту.

Вариант второй: если на изучение темы нет времени, сил или возможностей, просто спокойно заниматься своим делом. В каковом каждый из нас, смею надеяться, разбирается лучше, чем в организации сетевых митингов и тому подобных акций протеста. Особенно если точно не знаешь, против чего именно следует протестовать.

Впрочем, результатом первого варианта будет скорее всего переход ко второму. Потому что ничего из ряда вон выходящего описанная история, не смотря на всю свою плохость, не содержит. Ибо в ретроспективе своей имеет и иные не вполне хорошие случаи. Например, беззастенчивое <<использование>> материалов Тихона Тарнавского.

Почему же именно дело о запорченных дисках получило такой резонанс? И тут мы переходим ко второму аспекту настоящей заметки~--- о возможных последствиях.

Да вследствие того, что делу этому придали апокалиптический оттенок. Вспомним цитату, где провал Школьного проекта приравнивается чуть ли не к гибели FOSS в масштабе страны. Так ли это? Для ответа на этот вопрос опять придётся обратиться к истории, теперь уже более давней, охватывающей уже более пятнадцати лет.

А история эта свидетельствует, что на протяжении более чем пятнадцати лет FOSS и Linux на Руси (в это понятие я включаю также Украину и Белоруссию) без всяких Школьных и иных проектов и внедрений Linux'а куда бы то ни было. UNIX был, FOSS был, Linux был, FreeBSD крутилась на серверах бессчётного числа Интернет-провайдеров, а Школьного проекта не было. Из этого рискну предположить, то если его снова не станет~--- всё вернётся на круги своя, только и всего.

За счёт чего или кого? Да всё того же сообщества. Того самого пресловутого, аморфного, неорганизованного. На которое со всех сторон сыпятся жалобы, что оно ничего никуда не внедряет и никому ничем не помогает.

Но ведь это не его, сообщества, собачье дело~--- чего-то куда-то внедрять или кому-то в чём-то помогать. Сообщество~--- оно или есть, или его нет. И то, и другое~--- констатация медицинского факта.

В нашей стране и некоторых сопредельных сообщество, хвала Аллаху, есть. Благодаря усилиям многих поколений его членов. В числе которых, причём первых, напомню, был и Алексей Новодворский. Так неужели оно исчезнет в случае провала Школьного проекта?

А пострадают ли от его провала ученики, ради грядущего счастья которых всё затевается? Позволю себе высказать очередное крамольное соображение: нет, не пострадают. В обоснование чего процитирую самого себя:


\begin{shadequote}{}
И потому каждый приверженец идеи Open Sources должен быть благодарным нашему Минобразу за то, что в качестве базовой платформы для обучения школьников основам компьютерной грамотности он не принял Linux или FreeBSD. Не согласны? А припомните-ка, какой срок карантина по окончании школы вам потребовался, чтобы прочитать, наконец, ``Капитанскую дочку'' или ``Мцыри''?
\end{shadequote}

Это~--- из заметки <<Пора ли кричать караул?>>, написанной в 2001 году по поводу начала Windows'изации российской школы. Много воды утекло с тех пор, но главный её вывод я готов повторить: нерадивые учителя могут привить ученикам отвращение к Linux'у столь же эффективно, как их предшественникам в наше время удавалось привить нам отвращение к Пушкину или Гоголю.

Так что в повсеместном внедрении Linux'а в школы, да ещё в приказном порядке, я не вижу ничего хорошего. Там же, где ученикам это интересно, а учителя радивые, FOSS и Linux будет внедряться, вне зависимости от всех проектов. В обоснование позволю себе процитировать один из постов vinny всё из того же обсужения на Linuxforum'е:


\begin{shadequote}{}
Год назад на <<Инфокоме>> в рамках конкурса <<Цифровой маршрут>> я протестировала на знание Gimp около 1000 школьников. Задание было простое: найти в Линуксе графический пакет и что-нибудь в нем нарисовать. На компах стояла Mandriva и Runtu. Половина детей справлялась за пару минут, потому что <<У нас в школе везде стоит Линукс>>\dots ситуация не так плоха, как её малюют на ЛОРе, исходя из общения с одним, двумя школьниками. Процесс перехода идёт, в не зависимости от <<общей линии партийного руководства>>, взяток, откатов, головотяпства с дисками и прочих проблем.
\end{shadequote}

И опять же можно спросить~--- чьими силами движим этот процесс? Да силами учителей и учеников, составляющих сообщество. И потому он будет идти, не смотря ни на что. Потому что это нужно людям. А наше дело, дело тех, кто относит себя к сообществу~--- обеспечить, чтобы те, кому это нужно, узнали о том, что им нужно. А не митинговать в Сети и наяву. И не участвовать в кампаниях, кем-то и зачем-то затеянных.

\section{Тётя Ася приехала\dots} 
\textsl{В соавторстве с Алисой Деевой, при участии творческого гения всего прогрессивного человечества}

\hfill \begin{minipage}[h]{0.45\textwidth}
Сарынь на кичку!\\
Ядреный лапоть\\
Пошел шататься \\
По берегам.  \\
Сарынь на кичку!\\
В Казань!  В Саратов!
\begin{flushright}
\textit{Василий Каменски}
\end{flushright}
\bigskip\end{minipage}
  
\subsection{Преамбула}

Настоящий материал представляет собой контаминацию отдельных заметок, размещавшихся на нашем Блогосайте, с учётом комментариев к ним и обсуждения на форумах. Авторы выражают признательность всем комментаторам и участникам. 

Исходные заметки сочинялись на злобу дня~--- в связи с очередным бесчинством тёти Аси, меняющей протоколы, 


\begin{shadequote}{}
\dotsкак, терья-терьям, перчатки
\end{shadequote}

И по горячим следам событий. Ныне они все собраны, причёсаны и уложены в одну сюжетную линию. 

Настоящий материал адресован не только линуксоидам и прочим POSIX'ивистам, но и пользователям более иной операционной системы. Даже, пожалуй, скорее последним: ведь все линуксоиды со стажем, использующие службы мгновенных сообщений, давно уже заготовили себе запасные ВПП в виде аккаунтов Jabber, а некоторые даже превратили их в основные аэродромы. 

Пользователи же Windows подчас оказываются лишенными связи~--- когда AOL, как всегда неожиданно (подобно наступлению морозов на Руси) меняет протоколы, делая службу ICQ недоступной для <<альтернативных>> её клиентов. Ниже мы постараемся показать, что свет клином не сошёлся на тёте Асе. И для альтернативных клиентов есть и протоколы, и службы, обеспечивающие к ним доступ. 

Надеемся также, что данный материал будет полезен совсем начинающим линуксоидам, которые ещё не вполне адаптировались в новой для себя среде. 

Наконец, хотелось бы верить, что он поможет многоопытным линуксоидам, для которых важные и дорогие записи в контакт-листе представлены ICQ-аккаунтами. И которые, подобно Д'Артаньяну, давно забыли даже то, чего никогда в виндах и не знали. Но которых жизнь ставит перед необходимостью объяснять это своим корреспондентам\dots 

\subsection{Как это делалось в Одессе}

Если бы всё делалось так, \textit{Как это делалось в Одессе}, то перед очередной сменой протокола ICQ можно было бы реконструировать следующий диалог: 

\begin{quotation}
---~Слушайте, Хакер,~--- сказал Молодой Юзер,~--- я имею вам сказать пару слов. Меня послала тетя Ася с AOL'а\dots 

---~Ну, хорошо,~--- ответил Гуря Кряк, по прозвищу Хакер,~--- что это за пара слов? 

---~В AOL вчера пришёл новый ман\'{а}гер, велела вам сказать тетя Ася\dots 

---~Я знал об этом позавчера,~--- ответил Гуря Кряк.~--- Дальше. 

---~Ман\'{а}гер собрал девелопёров и оказал девелопёрам речь\dots 

---~Новая метла чисто метет,~--- ответил Гуря Кряк.~--- Он хочет сменить протокол. Дальше\dots 

---~А когда будет смена протокола, вы знаете, Хакер? 

---~Она будет завтра. 

---~Хакер, она будет сегодня. 

---~Кто сказал тебе это, мальчик? 

---~Это сказала тетя Ася. Вы знаете тетю Ася? 

---~Я знаю тетю Асю. Дальше. 

---~\dotsМан\'{а}гер собрал девелопёров и сказал им речь. <<Мы должны задушить Гурю Кряка,~--- сказал он,~--- потому что там, где есть государь Проприетатор, там нет Хакера. Сегодня, когда Кряк закончил очередной патч к ядру и все бросятся поздравлять его по ICQ, сегодня нужно сменить протокол>>\dots

---~Дальше. 

---~\dotsТогда девелопёры начали бояться. Они сказали: если мы сделаем сегодня смену протокола, когда у него праздник, так Гуря рассерчает, и много клиентов уйдут на Jabber. Так ман\'{а}гер сказал~--- самолюбие мне дороже\dots 

---~Ну, иди,~--- ответил Хакер. 

---~Что сказать тете Асе за смену протокола? 

---~Скажи: Гуря знает за смену протокола.
\end{quotation}



Думаю, все пользователи так называемых альтернативных ICQ-клиентов\dots прежде чем продолжать, давайте спросим: а почему~--- альтернативных? Для этого надо обратиться к истории. 

Совсем-совсем недавно, лет сто назад\dots хотя в масштабах времени IT-индустрии с 1996 года прошли, казалось бы века. 

Так вот, именно эти века назад четверо израильских школьников сочинили систему обмена мгновенными сообщениями ICQ, клиентская часть которой распространялась бесплатно. И которая мгновенно завоевала предпочтения сначала пользователей-индивидуалов, а затем~--- и так называемых бизнес-пользователей. Хотя последние поначалу относились к ней настороженно, а некоторые полагали ICQ даже разновидностью компьютерных вирусов. Что, однако, не помешало триумфальному шествию нового явления по просторам Интернета. 

Короче говоря, ребята вместо тривиальной золотой жилы раскопали целый Витватерсранд. Окучить который собственными силами не могли~--- и потому успешно продали свой бизнес компании AOL. Которая и взвалила ношу поддержки ICQ на свои могучие плечи. В качестве доли своей малой получая плату за рекламу, более или менее ненавязчиво впариваемую пользователям их клиента, за которым так и закрепилось название~--- ICQ. Что на просторах нашей необъятной Родины трансформировалось в аську. Ибо здесь хорошо помнили ещё, наряду с незабвенным Лёней Голубковым и очень простой фирмой Сэлдом, также и заботливую тётю Асю, никогда не приезжавшую в гости без поллитры\dots не подумайте плохого, отбеливателя; да ещё она и стиральный порошок на закусь прихватывала. 

Дурной пример заразителен. И по стопам тех самых ребят пошли многие другие, начавшие сочинять программы, способные работать с протоколом ICQ~--- такие же бесплатные, в чём-то более удобные или более функциональные. А главное~--- способные работать не только под Windows, как первозданная аська, но и под более иными операционками, конкретно~--- свободными и Unix-подобными. И, что было ещё главнее с точки зрения нынешнего владельца службы ICQ, не показывавшие их рекламы и, соответственно, не приносившие в мошну AOL ни единого шекеля. 

Кроме того, широкие народные массы вспомнили и о существовании иных протоколов передачи мгновенных сообщений~--- во-первых (а аська на этом поприще была далеко не первой), и занялись разработкой новых, в том числе и свободных. И потому большинство <<альтернативных>> программ для работы с ICQ, кроме удобства и функционала, предлагали своим пользователям ещё и мультипротокольность. 

В частности, могу сказать за себя. Если лет 12 назад один из авторов этой заметки, как и все его знакомые, пользовал каноническую ICQ, то ныне в круге его общения (а он довольно обширен) нет никого, юзавшего бы родную аську. Даже среди самых отпетых вендузяднегов и записных пользователей Windows\dots 

Сначала доля <<альтернативников>> была не велика. Но в силу указанных причин, подкреплённых ещё и отсутствием рекламы, она, как поросёнок из оперетты, росла и росла. Пока не выросла в большую-пребольшую свинью, подложенную AOL'у. И тут AOL'овские 
ман\'{а}геры, подобно приставу из бабелевской истории, забили в набат и начали собирать свой участок на предмет борьбы с таким безобразием. 

Шпики из участка, в отличие от бабелевских, бояться не начали. А предложили главному приставу радикальный, казалось бы, метод борьбы: время от времени устраивать облавы\dots пардон, смены протокола. С тем, чтобы новый протокол, поддерживаемый, разумеется, каноническим ICQ-клиентом, служил барьером, отсекавшим <<альтернативников>> со старыми версиями протоколов. 

Метод оказался не эффективным: разработчики <<альтернативных>>, особенно свободных, программ передачи мгновенных сообщений реагировали на это в соответствие со своим титулом, то есть~--- мгновенно. И единственным действенным ответом им было~--- учащать и учащать смены протокола. 

Сначала таковая происходила не периодически, а просто время от времени~--- ну как меняются версии программ. И пользователи ещё не просекли грядущего. Хотя наиболее дальновидные уже тогда начали переносить свои контакты, особенно важные по делу или дорогие по жизни, на Jabber. 

Потом более или менее устаканилась периодичность смены раз в полгода. И постепенно все к этому привыкли: патчи, добавляющие поддержку новой версии протокола, для всех свободных клиентов ICQ сотоварищи появлялись в считанные часы после облавы, а спустя сутки-другие выходили и новые версии этих самых клиентов, в том числе, и в бинарных сборках для наиболее распространённых дистрибутивов. 

В общем, пошла нормальная цивилизованная жизнь: разработчики канонической ICQ делали вид, что они борются с <<альтернативниками>>, а последние~--- делали вид, что они этой борьбы страх как боятся. Но, тем не менее, постепенно оттягивались на Jabber. 

Хотя и не все: кому-то было жаль аськиных контактов с отсталыми ретроградами, кому-то таковые требовались по долгу службы, а гейт между Jabber'ом и ICQ налаживать ленились\dots 

Однако, наконец, в AOL'овском участке появился новый пристав~--- та самая новая метла, которая чисто метёт. Это (и то, что будет говориться далее) исключительно наша интерпретация событий~--- никаких агентурных источников, в виде молодого человека от тёти Ханы, у нас не имелось. А поскольку новому приставу самолюбие оказалось\dots ну не дороже денег, конечно, просто он, наивняк, думал своё самолюбие прямым путём пересчитать в СКВ\dots 

Так вот, решил он облавы участить. И смены протоколов начали происходить сначала раз в год, потом~--- раз в полгода, потом внедрили ежеквартальный график. И, наконец, кажется, перешли целиком на самую прогрессивную, раз в две недели, систему. 

Чем ответил на это народ? Очевидно, что народ опенсорсный, давно знающий о свободных протоколах, столь же давно подготовил себе и зап\'{а}сные пути в виде Jabber-аккаунтов, о чём говорилось в преамбуле. 

Но и на простых советских граждан надежда плоха: в ответ на наглое бесчинство бухгалтера Кукушкинда широкие народные массы неожиданно вспомнили о Google Talk, который являет собой самый обычный Jabber, действующий по тому же самому протоколу XMPP. А поскольку почтовый адрес в системе \texttt{gmail.com}~--- это ни что иное, как логин для любого Jabber-клиента, все пользователи указанной службы автоматически оказываются Jabber'истами. И \texttt{gmail.com}'ом список нечаянных Jabber'истов далеко не исчерпывается, в чём мы скоро убедимся. 

Так что пусть AOL скрежещет злобно и зубовно. Мы ответим ему исконно и кондово, словами волжских атаманов, вынесенными в эпиграф этой статьи: 


\begin{shadequote}{}
Сарынь на кичку! Айда на Jabber!
\end{shadequote}

И пусть даже светлый его жабий лик явится нам в ипостаси Google, которая, как предсказывают пессимисты, когда-нибудь потом, при поддержке длинной ЦРУ'шной руки, станет мировым жандармом-монополистом. Но ведь это будет потом. А сейчас эта длинная рука протягивается нам в виде помощи. И, повторяю, рука эта не одна. 

Правда, меня, как и всякого порядочного пост-советского человека, терзают смутные сомнения: а не внедрился ли в тесные ряды AOL'а тот самый длиннорукий Google'вец? И не он ли исполняет роль нового чисто метущего пристава? Если да~--- имя его должно занести в скрижали мировой разведки. Наряду с именами Ланового, Бониониса, Тихонова\dots 

И потому завершить этот раздел уместно будет словами Таманцева из <<Августа 44-го>>: 


\begin{shadequote}{}
Бабушка приехала! Бабулечка!
\end{shadequote}

Ибо тётя Ася за минувшие годы наверняка стала бабушкой. И, похоже, она действительно приехала~--- дальше некуда. Конечная остановка. Поезд следует в депо. Просьба срочно освободить вагоны. Пользуйтесь услугами наземного транспорта. 

\section{Дело Ханса Рейзера} 
\begin{timeline}2006, октябрь~--- 2008, июль\end{timeline}

Эти материалы собирались и писались  все полтора года вяло двигавшегося процесса, на разных его стадиях. Здесь я попытался уложить их в единую сюжетную линию. Некоторые фрагменты были написаны по моей просьбе Алексеем Жбановым ala allez, в прошлой жизни профессиональным криминалистом (публикуется с его разрешения), и Uncle\_Theodore, в миру Олегом Свидерским, знакомым с американским правосудием не понаслышке. Увы. Олег ушёл к верхним людям, но, думаю, он бы не возражал.

\subsection{Вступление}

Материалы для этих заметок я начал собирать практически сразу, как узнал о поводе для них~--- аресте Ханса Рейзера, известного программиста Open Source, разработчика одной из самых совершенных файловых систем в истории IT-индустрии. Аресте по обвинению в убийстве собственной жены, хотя практически и бывшей. 

Первое время я регулярно излагал фактографическую сторону дела, не особенно это афишируя~--- в расчете на то, что ясность появится буквально в ближайшие если не дни, то недели. 

Однако время шло, а никакой ясности не наступало: судебные заседания либо ничего не решали, либо просто откладывались, иногда по смехотворным причинам. 

Но вот теперь в этом деле поставлена очень жирная точка~--- вердикт присяжных: виновен в убийстве первой степени. В нашей терминологии это нечто вроде убийства при отягчающих обстоятельствах~--- с заранее обдуманными намерениями. Это при отсутствии тела и недоказанности самого факта убийства (\textit{на момент вынесения вердикта}). 

Потому я и решил собрать все свои заметки~--- именно в том виде, в каком они сочинялись по ходу дела~--- лишь с минимальной правкой и, местами, с комментариями. В них неизбежны противоречия~--- я сознательно даже не пытался их устранить: ведь противоречия суть не менее весомые косвенные свидетельства, которые, возможно, когда-нибудь послужат делу выяснения истины. 

Впрочем, я и сам рассчитываю дожить до того момента, когда истина будет установлена. Если нет~--- возможно, мои заметки послужат сюжетом какого-нибудь детектива. Только не след забывать, что речь в них идет не о литературных героях или злодеях, а о реальных людях, наших современниках.  И по сей день нет в этом деле никакой ясности. Ибо это вовсе не детектив Гарднера с адвокатом Перри Мейсоном в главной роли. 

\subsection{Часть 1}

Эта детективная история разворачивается сейчас, можно сказать, на наших глазах. Причем пока достоверно неизвестно, что же собственно произошло. А уж чем дело кончится~--- и подавно. 

Сначала кратко, чисто для справки, о ком идет речь. Ханс Рейзер~--- очень известный человек в мире разработчиков и пользователей открытого программного обеспечения, создатель одной из лучших файловых систем для Linux~--- ReiserFS, и её принципиально новой версии~--- Reiser4, роль которой далеко выходит за рамки файловых систем вообще. В описываемо время проживал в городе Окленд штата Калифорния. Чуть позднее я вернусь к биографическим подробностям~--- они важны сейчас и, возможно окажутся еще более важными в будущем. 

Традиционный disclaimer. Я постараюсь писать, излагая только фактографическую сторону дела, как говорил старина Тацит, без гнева и пристрастия, по возможности избегая всякого рода моральных оценок (и оценок вообще). Получится или нет~--- судить читателю. Ибо относиться к Хансу равнодушно я не могу, слишком уважая результаты его работы. 

Итак, начало истории. Я, как и большинство моих коллег, узнал о ней ранним утром 11 октября (другое дело, что понятие об утре, и тем более раннем, у всех нас разное). Когда по всем сайтам Рунета, связанным с тематикой Linux и Open Source, прошла новость: Ханс Рейзер арестован по подозрению в убийстве своей жены в городе Окленд, 10 октября, в 11 часов по местному времени. 

В первый момент это сообщение было воспринято как утка или глупая шутка~--- кое-кто задумался даже, не 1-е ли на дворе апреля. Однако в таких случаях каждый IT-шник по привычке обращается к Google. И гугление (кстати, в английском языке глагол \textit{to google} недавно утвержден официально) показало, что ни об утке, ни о шутке говорить не приходится~--- история эта началась не вчера и широко освещалась в региональной прессе и Интернет-изданиях, находя отражение даже на русскоязычных форумах тех краев. Более того, уже некоторое время существовал сайт, специально посвященный одному из аспектов этого дела. 

Видимое начало истории падает на 3 сентября~--- день, когда последний раз видели жену Ханса, Нину Рейзер. С тех пор, и по сей день (12 октября, 19:15 московского времени) её не видел никто~--- ни живой, ни мертвой. 

Добавлю с скобках, что на момент, когда редактируется эта заметка (20 июня 2007 года), никакой информации о её судьбе нет по прежнему. 

Но тут мне придется прервать повествование, чтобы вернуться к биографическим подробностям~--- как уже было сказано, они могут оказаться важными. 

Итак, Ханс Рейзер\dots Родился в 1964 году в Калифорнии, там же поучился в средней школе, которую бросил в возрасте 14 лет, по причине несогласия с программой. Тем не менее, в 15 лет его приняли в Калифорнийский университет, Беркли (тот самый, где была создана основа современного Интернета в виде протокола TCP/IP, операционная система BSD и множество других штуковин). И, насколько я понимаю специфику ихнего образования, обучался там по некоей индивидуальной программе (физика, математика и соплеменные науки). 

Сколько Ханс проучился в университете~--- я пока не выяснил. Известно лишь, что диссертации (PhD) он так и не защитил, видимо, посчитав это ненужным. 

Начиная с 1988 года, деятельность Ханса протекает в сфере IT и Computer Science. С 1988 по 1996 год он занимает различные должности, связанные с программированием и администрированием, в ряде компаний, от малоизвестных (по крайней мере, мне) до Исследовательского центра IBM (IBM Almaden Research Cente). 

В 1997 году Ханс основывает собственную компанию~--- Namesys; собственно, это небольшая команда разработчиков принципиально новой файловой системы которая получает имя своего основателя~--- ReiserFS. 

К 1999 году относится событие, прямо с IT не связанное, но зато связанное с нашим сюжетом~--- женитьба. Жена~--- Нина, жительница Санкт-Петербурга, врач по образованию, 1974 года рождения. Об обстоятельствах знакомства и брака говорить не буду~--- поскольку точной информации не имею. 

Примерно в то же время в Рунете циркулировали упорные слухи, что штаб-квартира Namesys то ли переносится, то ли уже перенесена в Россию. На чем они основывались (и основывались ли вообще на чем-то)~--- не знаю. Но, что интересно, сайт компании обслуживается DNS-серверами зоны RU\dots 

Впрочем, дальнейшая профессиональная деятельность Ханса к сюжету имеет мало отношения~--- интересующиеся могут ознакомиться с ней на сайте компании и в многочисленных сетевых и оффлайновых источниках. Для нашей же истории важно, что брачный союз Нины и Ханса, проживавших в вышеупомянутом Окленде, имел результатом рождение двух детей~--- сына Рори (Rory) и дочери Niorline (транскрибировать не берусь). Ныне им 7 и 5 лет, соответственно (точнее, уже скорее 8 и 6). 

Однако брак Нины и Ханса оказался не таким долгим. В 2004 году они становятся <<раздельно проживающими супругами>>. В качестве одной из причин расставания Нина позднее, в ходе бракоразводного процесса, назовет <<трудоголизм>> мужа~--- знающие результаты работ Ханса согласятся с этим определением. 

Бракоразводный процесс, однако, начался вроде бы лишь в 2006 году. По имеющимся данным, он протекал отнюдь не полюбовно. Англо-американскую юридическую терминологию я не понимаю (впрочем, и русскую-то с трудом), но, судя по всему, Нина, в частности, настаивала на отстранении отца от участия в воспитании детей (насколько я понял за время знакомства с этой историей, в США нет понятия лишения родительских прав в нашем смысле). Тут-то и всплыло определение <<трудоголик>> (workoholic), его частые поездки в Россию (где работала большая часть его команды), а также ряд других подробностей, на которых я не буду задерживаться, ибо почерпнуты они из не вполне достоверного источника (какого~--- скажу несколько позже). 

Тем не менее, какое-то общение между (почти) бывшими супругами продолжалось. В частности, Нина привозила детей к Хансу и оставляла на несколько дней. 

Так было и в день, с которого формально начинается эта история~--- 3 сентября 2006 года. Нина привезла детей к отцу и уехала. Чтобы более не появляться. Или не уехала. Но с того момента о ней ничего не известно. По крайней мере, по сей день\dots 

И тут нужно сделать отступление об источниках. Почти все, о чем говорилось ранее, основывается на источниках узкопрофессиональных, которые\dots ну не то чтобы достовернее всех прочих, но легко проверяются IT'шниками по независимым каналам. Ныне же мы вступаем в мир массовых СМИ~--- новостных сайтов локально-регионального масштаба (нечто типа районных многотиражек старых советских времен), со всеми вытекающими последствиями. И, естественно, англоязычными. 

Благо, в этом море у нас будет и русскоязычный путеводитель~--- сайт \url{http://privet.com}, точнее~--- отдельный трейд его форума, посвященный исчезновению Нины Рейзер, Созданный первоначально исключительно для организации помощи в поисках её (или её тела), он постепенно, как и любой другой форум, оброс флеймом, флудом, оффтопиком и тому подобными, неизбежно сопутствующими, явлениями. В частности, на нем не только даются ссылки на местные новостные ресурсы, но и часто пересказывается их содержание~--- на русском языке (правда, с вкраплениями английских юридических терминов). Ныне этот трейд заглох и представляет собой чисто исторический интерес. 

Так вот, согласно указанным источникам, Нина направлялась на вечеринку к подруге. И по дороге должна была заехать в магазин за продуктами. Судя по дальнейшему, в магазин она заезжала. Но потом~--- все\dots 
Поиски начались через несколько дней (насколько я понял, по инициативе подруги, а не мужа, хотя и бывшего). Завершились на восьмой (или девятый, как считать сутки) день~--- находкой машины Нины. Аккуратно запаркованной, внутри~--- дамская сумочка с деньгами и тому подобными кредитками, мобильный телефон. И более ничего. 

\subsection{Часть 2}

Итак, продолжаю. Третьего сентября Нина (вроде бы) покинула дом Ханса, оставив там детей, и отправилась на вечеринку к подруге~--- это по одной, ранней, версии. Позднее же утверждалось, что она отправилась поужинать с бойфрендом~--- неким Энтони Зографусом (Anthony Zografos). между прочим, тоже работающим в сфере информационных технологий. Как бы то ни было, ни там, ни там она не появилась. 

Поиски Нины начались через несколько дней, по чьей инициативе~--- не очень понятно, но во всяком случае~--- не по инициативе (почти) бывшего мужа. И велись без особого энтузиазма до тех пор, пока на 8-й (или 9-й, как считать часовые пояса) день не была обнаружена её машина. 

Аккуратно припаркованная, запертая, внутри она содержала, однако, дамскую сумочку с документами, деньгами и кредитками, а также мобильник. Были в машине и продукты, купленные для вечеринки~--- вне зависимости от того, с кем она планировала её провести. 

После этого за поиски принимаются более серьезно. На сайте \texttt{privet.ru} возникает упомянутая выше тема, посвященная исчезновению Нины Рейзер, затем возникает сайт <<Помогите найти Нину Рейзер>> (\textit{он давно не работает}), где за предоставление информации о ней объявляется награда в~15 тысяч американских рублей. 

Одновременно с этим начинает витать мысль о причастности Ханса к исчезновению жены. На чем она основывается~--- непонятно, никаких сообщений в прессе или заявлений официальных лиц не обнаруживается. 

Наконец, в конце сентября (23-24) осуществляются массовые поиски в окрестных лесных массивах~--- силами полиции, с собаками и привлечением добровольцев. Поиски оказываются безрезультатными~--- никаких следов Нины не обнаружено. 

Сам Ханс всячески избегает встреч с полицией~--- по выражению прессы, отказывается от сотрудничества. Дети переходят на попечение общества~--- то есть, насколько я могу судить, помещаются в нечто вроде приюта для сирот. Хотя в наличии имеется бабушка по отцу~--- Беверли Палмер, а из Петербурга приезжает бабушка по матери~--- Ирина Шаранова. 

Хансу все же не удается избежать общения с правоохранительными органами~--- в начале октября у него берут кровь для анализа (как задумчиво сообщалось в прессе, для анализа на ДНК). Из чего можно было заключить, что где-то найдены какие-то т.~н.~биологические улики (кровь, волосы и т.п.). 

Затем в доме Рейзера производится обыск, давший такие улики, которые позволили санкционировать 10 октября его арест. По сообщениям прессы, вместе с ним было арестовано еще два человека, но об их личностях никакой информации в известных мне источниках не имеется (и в дальнейшем о них не упоминается вообще). Теперь, по законам штата Калифорнии, ему в течении 48 часов должно быть предъявлено обвинение\dots 

Каковы были улики, обнаруженные в доме Рейзера? В разных вариантах эта тема муссировалась в местной интернет-прессе, но я от ответа на этот вопрос пока воздержусь. Ибо жду заключения одного из членов нашего комьюнити~--- в недавнем прошлом эксперта-криминалиста по профессии. 

А пока сделаю маленькое технологическое отступление. Я уже говорил, что Рейзер был создателем и основным разработчиком файловой системы для ОС Linux~--- ReiserFS, давно (с 2001 года) и успешно используемой многими и многими пользователями, в том числе и в индустриальных масштабах~--- например, на критически важных серверах Интернета и интранета. 

В последние же годы им разрабатывалась файловая система Reiser4~--- с одной стороны, логическое продолжение своей предшественницы, с другой~--- принципиально новая разработка, функции которой далеко выходят за рамки просто файловой системы. В августе 2004 года она обретает статус релиза~--- то есть считается вполне надежной и годной к индустриальному применению. Однако поддержка её не включается в официальное ядро Linux. В результате те пользователи Linux, которые хотели использовать Reiser4, должны были обращаться к патчам ядра сторонних разработчиков. Впрочем, в некоторых (хотя и единичных) дистрибутивах поддержка Reiser4 была штатной функцией (ныне ни в одном из дистрибутивов её не найти). 

Причины отсутствия официальной поддержки ReiserFS выдвигались разные, иногда достаточно формальные~--- например, несоответствие форматирования кода драйверов Reiser4 стандарту, принятому в коде ядра; это примерно то же самое, что (для литературных текстов) различие в величине абзацных отступов. 

И так было на протяжении более чем двух лет~--- срок для Open Source проектов просто астрономический. Пока Эндрю Мортон, один из главных разработчиков ядра Linux, в своем интервью не обмолвился (правда, между строк)~--- если поддержка Reiser4 будет включена в ядро, а разработка её по каким-либо причинам прекратится, это доставит много сложностей остальным разработчикам. Было это 22 сентября 2006 года\dots 

Что же до ReiserFS~--- то, как я говорил, она использовалась очень широко, во многих дистрибутивах Linux выступая в качестве файловой системы по умолчанию. В частности~--- в дистрибутиве Suse, второй по распространенности Linux-системе, в том числе и в корпоративной сфере. Дистрибутив этот, начиная с 1994 года, разрабатывался одноименной немецкой фирмой, активно участвующей в финансировании проектов по разработке открытого софта, в частности~--- и фирмы Namesys, принадлежащей Рейзеру. 

В скобках заметим, что вторым основным источником финансирования работ Ханса (а может быть, и первым) было агентство DARPA (Defense Advanced Research Projects Agency~--- агентство передовых оборонных исследовательских проектов), находящее в подчинении Министерства обороны США. И это при том, что большинство разработчиков команды Namesys проживают в России и на Украине\dots 

В 2004 году фирма Suse была куплена компанией Novell~--- для людей, тесно не связанных с IT-индустрией, имя это, в отличие от Microsoft, ничего не говорит. Однако для IT-специалистов, особенно нашей страны и сопредельных стран оно~--- нечто типа живой легенды, выступая своего рода синонимом локальной сети, как Ленин и Партия (говоришь сеть~--- подразумеваешь Novell, говоришь Novell~--- подразумеваешь сеть). 

До того времени фирма Novell в сфере свободного софта никак не отметилась, и потому поначалу продолжались традиции Suse. Однако постепенно они тем или иным образом модифицировались, пока наконец в выступлении одного из руководителей разработки Suse не прозвучал отказ от использования ReiserFS в качестве умолчальной. Резонов к тому он привел много, повторять их не буду, но поверьте~--- все они были более чем смешны. Это произошло 4 октября 2006 года. До ареста Ханса Рейзера оставалась еще неделя\dots 

PS Здесь позвольте мне взять тайм-аут. Конечно, в добрых традициях журналистских расследований следовало бы развести турусы на колесах со ссылками на компетентные мнения. Но мне вдруг стало по настоящему страшно. Ссылки на интернет-треп никому ничем по настоящему не опасны~--- и то в первой части своего очерка я их избегал. А тут ребятам реально может грозить потеря работы~--- мир тесен, и все носители компетентных мнений вычисляются очень легко\dots 

Все равно до окончания истории еще далеко. В следующей части я вернусь к уликам и обвинительному заключению, которое и было предъявлено Хансу Рейзеру. 

\subsection{Часть 3}

Теперь пришла пора обратиться к уликам, послужившим основанием для ареста Ханса Рейзера. О них писалось уже накануне судебного заседания, но сколько-нибудь целостную картину стало возможным представить только по его результатам. Да и то, в картине этой зияющие прорехи видны невооруженным глазом. 

Я уже говорил, что у Ханса брали кровь на анализ. Теперь выясняется, зачем: в доме и в машине Ханса (не Нины!) были обнаружены какие-то следы крови (в сетевом репортаже сказано гораздо экспрессивней~--- брызги, но, как мы увидим дальше, это Интернет-издание нельзя упрекнуть в излишней сдержанности в отношении оценки фактов). Так вот, в этом самом репортаже утверждалось, что это кровь Нины. На чем это утверждение основывалось, за отсутствием тела, не очень ясно. Конечно, данные о крови Нины, как дважды рожавшей женщины, должны были где-то наличествовать~--- но достаточно ли их для однозначной идентификации? 

Далее, в машине Рейзера отсутствовало пассажирское кресло. Причем свою машину он пытался скрывать от полиции~--- ездил на прокатной. 

И, наконец, третья улика~--- две книги о расследовании преступлений, которые он приобрел в оффлайновом книжном магазине через несколько дней после исчезновения жены. В источниках можно найти их названия~--- но вряд ли они что-то скажут нашему читателю, так что не буду копаться\dots 

Все это в принципе было опубликовано еще до судебного заседания (в том числе и утверждение о том, что обнаруженная в доме и машине кровь принадлежит Нине Рейзер!). А вот что новое прозвучало на заседании~--- так это сведения о показаниях детей. Самих их на суде, естественно, не было. Но, по сведениям полиции, по приезде в дом Рейзера дети играли в компьютерные игры и слышали разговор родителей. 

Впрочем, и сообщение об их показаниях основывается не на стенограмме суда, а на словах одного из офицеров полиции. Вот как оно выглядит (цитирую фрагмент интернет-сообщения в переводе allez'а, кстати говоря, криминалиста по профессии): 


\begin{shadequote}{}
Один из детей указал, что <<Ганс Райзер и Нина Райзер, возможно, спорили>>,~--- написал полицейский следователь по розыску пропавших без вести Райан Джилл. <<Ребенок указал, что его родители говорили на <<средних>> тонах и что они использовали <<нехорошие слова>>,~--- написал Джилл.
\end{shadequote}

Один из детей позже сказал следователю, что он пошел наверх, в комнату, где были его мать и отец. Ребенок сказал, что Ганс Райзер велел ему спуститься вниз и не возвращаться, согласно заключению. 

Сообщаются и кое-какие подробности об автомобиле Нины~--- том самом, с обнаружения которого всерьез начались её поиски. По нему были раскиданы продукты~--- купленные вместе с детьми в большом универмаге до приезда к дому Рейзера~--- и это подтверждается просмотром видеозаписей магазинных камер наблюдения. 

А еще~--- сотовый телефон, о котором говорилось ранее, был разобран, то есть вынут аккумулятор. О наличии или отсутствии SIM-карты ни единого слова не говорится (правда, в ходе одного из обсуждений мне подсказали, что в телефонах CDMA, применяемых на Калифорнийщине, SIM-карты может и не быть вообще). 

И еще не могу удержаться от цитаты: 


\begin{shadequote}{}
Служебные собаки отработали место вокруг автомобиля, но след не взяли. Это привело следователей к выводу, что Нина Райзер не была в том месте, где был найден её автомобиль. Окрестные свидетели сообщили полиции, что фургон был припаркован там 5 сентября.
\end{shadequote}

Правда, тактично не сообщается, когда там побывали служебные собаки~--- в день обнаружения автомобиля, через неделю или через месяц. 

Вот и всё, что было известно на момент сочинения этой заметки. 


\textit{Интермедия. Мнение Алексея Жбанова aka allez}


С точки зрения криминалиста могу сказать одно: а практически нечего тут сказать. Если кого-то разочаровал, прошу прощения. Но это только книжный детектив Ниро Вульф распутывает сложнейшие дела, не выходя из дома. И то лишь потому, что верный помощник Арчи Гудвин выполняет нелегкую работу по снабжению своего босса максимально полной и достоверной информацией. По поводу полноты имеющейся у меня информации можно лишь тихо выматериться, о достоверности же оной, я думаю, даже говорить не стоит~--- источники те еще\dots 

Да и американская экспертная практика, надо сказать, несколько озадачивает. За время своей службы в милиции я твердо усвоил, что вывод в заключении или справке эксперта должен быть категорическим, то есть содержать один из всего лишь двух вариантов ответа на поставленный вопрос~--- <<да>> или <<нет>>, в противном случае~--- <<НПВ>> (жаргонизм от слов <<не представляется возможным>>; третьим вариантом не является, так как <<НПВ>>~--- это отсутствие ответа как такового). У американцев же кроме <<да>> и <<нет>> допускается также <<может быть>>\dots 

В общем, имеется следующее: 

\begin{enumerate}
	\item Пропавшая без вести женщина. 
	\item Находящийся с ней в разводе и натянутых отношениях муж. 
	\item Свидетельство малолетнего ребенка о факте ссоры родителей в день исчезновения матери. 
	\item Брошенный автомобиль Нины без признаков ограбления, но с косвенными и недостоверными признаками борьбы в нем (разбросанные продукты, разобранный мобильник). 
	\item Пятна крови в доме Ганса и на чехле спального мешка в его автомобиле, которая, может быть, принадлежит Нине. 
	\item Вымытый автомобиль Ганса без правого переднего сиденья с двумя книгами о расследовании убийств, початым рулоном мешков для мусора, скотчем, торцевыми ключами и сифонным насосом.
	\item Неприветливое отношение Ганса к полиции (отказ сотрудничать, попытки уйти от наружного наблюдения). 
\end{enumerate}


Какой из вышеизложенного можно сделать вывод? Со своей стороны могу сказать лишь одно: НПВ. Во-первых, нет никаких данных о том, какие производились экспертизы и по каким методикам. Во-вторых, все улики косвенные. Я, прошу заметить, служил не следователем и даже не дознавателем, поэтому УПК РФ знал лишь в части, меня касающейся (да и то уже подзабыл, каюсь), но, по-моему, с такой <<доказухой>> у нас в России человека можно лишь задержать на срок до трех суток, что ли. А уж об аресте и тем более предъявлении обвинения и суде вовсе речи быть не может~--- только оперативная разработка. Что будет дальше там, в Окленде, я не хочу и не имею никакого права даже прогнозировать. 

Резюмируя, хочу написать следующее: сколько бы мы ни сказали в этой теме слов, какими бы они ни были, а словом делу не поможешь и на положение вещей (по крайней мере, в данном конкретном случае) не повлияешь. Что же тут можно сделать? Не знаю точно, но на мой взгляд, тем из нас, кто верует, стоит помолиться за благополучный исход дела, неверующим же~--- изо всех сил надеяться на этот самый благополучный исход. Я, как неверующий, буду надеяться и ждать. Молча. 

\subsubsection{Мнение Uncle\_Theodore}

Американское правосудие еще никогда не было настолько опозорено со времен суда над О.~Джей Симпсоном, когда черные присяжные вопреки уликам оправдали черного актёра, убившего свою белую жену. Рейзера признали виновным в убийстве первой степени, то есть в предумышленном спланированном убийстве. Ему грозит от 25 лет тюрьмы до пожизненного заключения. 

Основной причиной такого решения присяжных адвокат Рейзера Дюбуа назвал отрицательное впечатление, которое Рейзер произвел на присяжных. Если учесть, что Дюбуа сам заставил Рейзера давать показания, а потом произнес совершенно дебильную заключительную речь, жуя сопли, я думаю, дело не в Рейзере, а в самом Дюбуа. 

В общем, дебилизм сплошной\dots Грустно мне. 

Прокурор произнес великолепную речь, растянувшуюся на несколько дней. Кровь на столбе в доме Рейзера, кровь на чехле от спального мешка в машине Рейзера. И тот факт, что он вел себя так, как будто знал, что Нина мертва. И все. 

Правду говорят, неужели можно отдать решение судьбы человека в руки 12 человек, у которых даже не хватило ума отмазаться от обязанности заседать в жюри\dots 

Прокурор произнес выдающуюся речь. Ну так еще бы, учитывая, из какого г\dotsа ему пришлось конфетку делать\dots А адвокат наоборот, сказал, по-моему, самую дебильную речь за всю историю своей работы\dots 

\subsection{Часть 4}

Предыдущие заметки писались по следам событий, по мере появления информации~--- или по мере того, как я её находил. Внимательный читатель наверняка нашел в них немало противоречий. Готов поклясться~--- я ничего не придумывал, и все противоречия имели место быть в моих источниках, уж больно они были различны. Тут и сообщения Интернет-прессы, и сплетни с форумов\dots 

На упрек~--- а зачем же я пользовался заведомо недостоверными источниками,~--- отвечу с легкостью: а какие источники для этой истории могут считаться достоверными? Неужели за таковые можно считать сообщения провинциального издания Inside Bay Area, освещавшие историю пропажи Нины Рейзер, её поисков и ареста её мужа? Издания, которое, ничтоже сумняшеся объявляет принадлежность найденных следов крови, не удосужившись ни малейшей аргументацией. А ведь все (ВСЕ!) остальные Интернет-публикации по делу Рейзера представляют собой простое переразмещение материалов Inside Bay Area\dots 

Посты же с \href{http://forum.privet.com}{forum.privet.com}, если тщательно <<фильтровать базар>> (а не это ли~--- основное призвание IT-шников, фильтровать сигналы от шума?), дают массу информации~--- во-первых от людей, лично знавших Нину, и о них самих, во-вторых, о царящих в обществе настроениях. Следует только учитывать, что пост этот создан личной знакомой Нины, выступающей под ником JU, и первоначально предназначался исключительно для организации её поисков, и лишь позднее (хотя и очень быстро) оброс всякого рода флеймом и флудом. Посему полагаю, что указанный трейд скоро будет закрыт, а возможно, и уничтожен вообще. 

Короче говоря, настало время свести воедино все наличные факты в их хронологической последовательности и многообразии версий. 

Итак, начинаем с 3 сентября, 13-55 местного времени~--- это первая и чуть ли не последняя точно установленная последняя временная метка: в это время Нина с детьми выходит из магазина <<Беркли Боул Маркет>>, где закупала продукты; это зафиксировано камерами видеонаблюдения. Согласно сообщению считается, что около 14 часов она подъехала к дому Рейзера. Точнее, его матери, Беверли Палмер, где он жил после фактического развода. Сама Беверли Палмер в это время отсутствовала, находясь на фестивале искусств <<Burning Man>> в Неваде (тут источники ссылаются на её собственные слова). 

Тут начинаются первые расхождения. Согласно ранним источникам, Нина высаживает детей у дома и отправляется дальше~--- мы еще вернемся к вопросу, куда. Поздние же сообщения, основанные на показаниях детей Рейзеров, гласят, что она некоторое время была в доме. 

Как и почему она там оказалась, причем не где-нибудь в прихожей, а в гостиной на втором этаже (позднее, по показаниям одного из детей, оттуда будут доноситься голоса бывших супругов)~--- это первый вопрос. Ведь, согласно многочисленным сообщениям прессы, Нина относилась к бывшему супругу настороженно, поскольку он после расхождения угрожал ей физическим насилием. Правда, основанием для этих сообщений были исключительно свидетельства личных знакомых Нины (кого именно, в источниках не сообщается). В обсуждении на \texttt{privet.com} намеков такого рода вдоволь~--- вплоть до того, что в полиции ей предлагали обзавестись оружием для защиты от посягательств бывшего супруга. Впрочем, эти сведения исходят не от полиции, а от одного из участников обсуждения на форуме. 

Тем не менее, игнорировать прямое свидетельство прессы нельзя: дети играют на компьютере, на первом этаже дома, и слышат доносящийся со второго разговор родителей, ведущийся на повышенных тонах, с употреблением <<нехороших>> выражений. Затем один из детей (кто именно~--- источники умалчивают) поднимается наверх, но отец отсылает его обратно. Далее~--- лакуна в источниках, до 5 сентября, когда Нина не забирает детей из школы. 

Тут возникает целая серия вопросов. Первый: как уже сказано, в тот день, 3 сентября, Нина, после того как завести детей к Хансу, должна была ехать~--- то ли на вечеринку к подруге (свидетельство JU с форума \texttt{privet.com}), то ли на ужин с другом, сиречь Энтони Заграфусом (все более поздние сообщения прессы). Ни у безымянной подруги, ни у Энтони она не появилась. Тем не менее, беспокойство о её отсутствии начало проявляться только 5-го числа~--- когда, как было сказано, она должна была забрать детей из школы, и не сделала этого (вопрос в скобках~--- а как дети попали в школу? отвезены отцом?). Кто проявляет беспокойство~--- подруга, друг или школьная администрация,~--- источники хранят молчание. По ряду косвенных признаков можно заключить, что во всяком случае не Ханс: на \texttt{privet.ru} можно найти намеки на то, что даже своей матери, вернувшейся из Невады, он ничего не говорит об исчезновении Нины, Беверли узнает об этом от полиции. 

Первоначально полиция особого рвения в поисках не проявляет, что и понятно: истории с загулявшими мужьями-женами, думаю, для Америки столь же обычны, как и для России. Однако 9 сентября обнаруживается автомобиль Нины, и дело принимает несколько иной оборот. 

Автомобиль стоял, аккуратно запаркованным и запертым. По ранним свидетельствам, внутри были: дамская сумочка, кредитные карточки, наличные деньги, всякого рода документы, в том числе и финансовые, типа чеков, и мобильный телефон. Были в машине и продукты, закупленные для вечеринки или ужина. В общем, из ранних описаний создается впечатление, что хозяйка оставила машину на минуту~--- о более в нее не вернулась. 

Позднее оказывается, что телефон был вскрыт и приведен в нерабочее состояние~--- из него был вынут аккумулятор (что интересно, о SIM-карте ни в одном сообщении не было сказано ни слова). Что же до продуктов~--- то они оказались разбросанными по всему фургону. 

Когда появился автомобиль на месте его обнаружения~--- также не ясно. В одном из сообщений проскочило, что, по свидетельству соседей, не ранее 5 сентября. 

Сведения о месте нахождения автомобиля ничего не говорят тому, кто не знаком с тамошней топографией. Однако JU в сообщении на форуме утверждает, что он был обнаружен примерно между домом Рейзера и домом той самой подруги, к которой Нина, по ранней версии, направлялась на вечеринку, но ближе к последнему (невольно хочется задать провокационный вопрос~--- а не JU была ли той подругой?) 

Потому что позднее во всех сообщения прессы утверждалось, что она собиралась ехать на ужин с Заграфосом~--- к нему домой? к себе?~--- учитывая наличие продуктов, последний вариант более вероятен. 

Осмотр местности вокруг автомобиля со служебными собаками дал отрицательный результат~--- след Нины собаки не взяли. Однако не ясно, когда это происходило~--- в день находки машины или позднее, и если позднее, то насколько. 

Хронология дальнейших событий не ясна совершенно. Можно лишь предположить, что с активизацией поисков Нины полиция первым делом обратилась к Хансу~--- и как к мужу, пусть и бывшему, и как к человеку, видевшему её последним. По сообщениям прессы, каких-либо объяснений на этот счет Рейзер не дал, и вообще, как пишет пресса, отказался от сотрудничества с полицией и всячески избегал контактов с ней. В результате этого он был взят под наблюдение. Видимо, в эти же дни дети Рейзеров помещаются в государственный приют и допрашиваются полицией. 

Следующая дата~--- 19 сентября, когда полиция обнаруживает автомобиль Рейзера, надо полагать, брошенный~--- есть сведения что сам Рейзер ездил на машине, взятой в прокат. В автомобиле отсутствовало переднее пассажирское кресло, которое так до сих пор и не найдено. Зато в нем присутствовали: 

\begin{itemize}
	\item рулон больших мешков для мусора, причем нескольких мешков не хватало; 
	\item две книги, <<Убийство: год на смертельных улицах>> Дэвида Саймона и <<Шедевры Убийства>> Джонатана Гудмана, которые были куплены 8 сентября; 
	\item свидетельства помыва машины, в виде остатков воды под ковриком и сифонного насоса; 
	\item штрафная квитанция за нарушение правил дорожного движения, от 12 сентября; выдавший её полицейский вспомнил, что переднее сиденье <<определенно было на месте, и он бы запомнил, если бы оно отсутствовало>>; 
	\item наконец, самое главное~--- спальный мешок с пятном крови. 
\end{itemize}


В связи с последним обстоятельством Рейзер 28 сентября задерживается полицией для взятия крови на анализ. При этом у него, кроме денег (более 8 тысяч баксов) и документов обнаруживается чек на сифонный насос (к сожалению, о дате чека ничего не сообщается). 

А тем временем, 22 сентября, в интервью Эндрю Мортона, одного из основных разработчиков ядра Linux, проскальзывает причина, по которой поддержка файловой системы Reiser4 до сих пор не включена в ядро: это~--- возможность прекращения её разработки\dots 

В эти же примерно дни (23-24 сентября далекого уже 2006 года) осуществляются массовые поиски в окрестных лесных массивах~--- силами полиции, с собаками и привлечением добровольцев. Поиски оказываются безрезультатными~--- никаких следов Нины не обнаружено. 

Вероятно, анализ крови Рейзера показал, что кровь на спальном мешке из машины~--- не его. Что, видимо, и дало основание для обыска в его доме. Где в гостиной были обнаружены следы (в одном из сообщений говорится~--- брызги) крови. Поскольку именно это и позволило санционировать затем арест Ханса, можно предположить, что она оказалась идентичной найденной на спальном мешке. Однако поведение служебных собак и тут дало отрицательный результат. 

К 4 октября относится заявление одного из руководителей фирмы Novell о том, что отныне файловая система ReiserFS не будет использоваться в дистрибутиве Suse в качестве устанавливаемой по умолчанию. 

И, наконец, 10 октября полиция производит арест Ханса Рейзера и его заключение в тюрьму. 12 октября на заседании суда, где Рейзер присутствует в наручниках и арестантской робе, ему предъявляют обвинение в убийстве. По свидетельству JU (видимо, на суде присутствовавшей), он выглядит испуганным (интересно, а как будет выглядеть любой человек, впервые в жизни оказавшийся в такой ситуации?). Собственно суд должен состояться 28 ноября в 9 часов местного времени. До суда Рейзер будет содержаться в тюрьме без права выхода под залог. Тем не менее, его адвокаты не выражают сомнения в том, что на суде Рейзер будет освобожден. 

Я не могу и не хочу делать каких-либо выводов о виновности или невиновности Ханса Рейзера. Однако обращает на себя внимание отношение общественности к его делу. Оно проявляется в местных СМИ~--- фактически с самых первых публикаций читатель упорно подталкивается к мнению о несомненной виновности Ханса. В частности, в одном из сообщений прямо утверждается, что кровь, обнаруженная в машине и доме Рейзера, принадлежит Нине. Хотя уже в следующей публикации автор поправился~--- эксперты не исключают её принадлежности Нине Рейзер. 

Показательно также настроение, царящее среди участников упомянутого выше трейда на сайте \texttt{privet.com}~--- большинство из них вину Рейзера под сомнение не ставят~--- и эта позиция поддерживается администрацией сайта. Редкие же участники, напоминавшие о презумпции невиновности, либо подвергались давлению, либо просто банились (подобно автору этих строк). 

Я не случайно привел тут данные о позиции разработчиков ядра Linux и компании Novell~--- оба они были сделаны до ареста Рейзера (интервью Мортона~--- фактически сразу после обнаружения его брошенной машины). 

Вот пока и все сведения по делу Ханса Рейзера. Полагаю, что новой информации до суда не появится: адвокаты Ханса, скорее всего, будут молчать, а полиция, вероятно, сказала прессе все, что считала нужным; адвокаты полагают, что даже более, чем нужно, и выразили недовольство утечкой информации в прессу, подчеркнув, что к настоящему времени сказано гораздо больше, чем выявлено. 

\subsubsection{Версия Uncle\_Theodore}

Нина не сбегала в Россию, она умерла в США. Согласны? Уехав от Ханса, она отправилась к Хелен, чтобы расслабиться без детей и встретилась там с кем-то из своих друзей-выдумщиков по части развлечений, я думаю, с Шоном Стерженом. Шон либо увез её куда-то примерно в шесть тридцать третьего сентября, либо с ней случился несчастный случай прямо в доме Хелен, примерно в это же время. 

Другой вариант~--- романтический вечер с Зографусом, который кончился плохо. Опять-таки, например, несчастным случаем в горах, например. Причем несчастным случаем такого плана, который привел бы участников этого случая в тюрьму. Садо-мазохистический акт, приведший к смерти Нины, например. Или стрельба в горах, в результате которой она была случайно убитой. 

Как бы то ни было, во вторник пятого Хелен уже знала, что произошло, и начала активно обвинять в убийстве Нины Ханса, которого давно ненавидела. Возможно, в понедельник четвертого она попыталась каким-то образом повлиять на паранойю Ханса (в моих фантазиях это выглядит так, что она могла подкинуть тело Нины на переднее сидение машины Ханса, но это уж очень проблематично) Тем не менее, четвертого, кто бы ни был убийцей, ждет чего-то. Но ничего не происходит, все идет своим чередом, и пятого в 2:30 Хелен несется в школу забирать детей Ханса. Она уже знает, что Нина их не заберет. Это было ошибкой, конечно. Её подвела русская сентиментальность. Но ей повезло в этот раз. 

Потом она опять начинает окучивать Ханса. Подговаривает русских знакомых попугать его, погуляв вокруг него в виде мафиози на rest area, начинает глобальную компанию на русском форуме, лжет в интервью про пунктуальность Нины и <<горячую любовь>> Шона к Нине. Шон, впрочем, срывается однажды. Как я уже писал на ЛОРе, его фраза <<Я убил восемь человек, но Нину я не убивал\dots>> так и хочет продолжения, что-то типа <<\dotsэто был несчастный случай!>> Впрочем, полиции уже все было ясно и так. Ханс уходит в жуткую паранойю. Ему мерещатся русские мафиози, подброшенные улики, кровь повсюду, сговор\dots И для суда его поведение выглядит доказательством убийства. 

Вывод: Хелен Дорен знает, что произошло. Хелен Дорен~--- прекрасный психолог. Несмотря на то, что она принадлежит к самому презираемому классу русской иммиграции~--- почтовым невестам~--- она использовала и американскую бюрократическую систему, и предрассудки русскоязычного сообщества, и даже паранойю Ханса, чтобы повесить на него убийство, в котором сама была косвенной соучастницей. 

\subsection{Послесловие: вопросы есть, будут ли ответы?}

Настоящая заметка не преследует целью ни доказать невиновность брата-линуксоида, ни, напротив, откреститься от изверга и убицы, позорящего наш славный клан. Я просто собрал тут вопросы, возникшие у меня (неужели только у меня?) при чтении последних материалов дела~--- начиная с первых чисел июля. Вопросы и противоречия в более старых материалах, также до сих пор не нашедшие адекватного объяснения, были уже сформулированы или в настоящем цикле, или в соответствующих темах ряда форумов. Здесь я их повторять не буду. 

Как можно узнать из последних материалов, в минувший понедельник, 7 июля 2008 года, около 16 часов по местному времени (хотя имеются сведения, что поздно вечером), Ханс Рейзер, в сопровождении своего адвоката Уильяма Дюбуа, отвел власти к месту захоронения тела своей жены, Нины Рейзер, исчезнувшей 3 октября 2006 года, в убийстве которой он обвинялся. И был признан жюри присяжных виновным в убийстве первой степени (по нашим законам этому примерно соответствует предумышленное убийство при отягчающих обстоятельствах). 

По результату этого вердикта Рейзеру грозило 25 лет тюремного заключения. Приговор, всеми ожидаемый и прогнозируемый, должен был вынести в минувшую среду, 9 июля 2006 года, судьёй Верховного суда графства Аламейда Ларри Гудманом. Однако вследствие похода властей и Рейзера к месту захоронения, завершившегося обнаружением тела и его идентификацией, заседание по вынесению приговора было отложено (по некоторым данным, до 13 августа, хотя и эта дата не окончательна и зависит от хода расследования, в частности, экспертизы тела). 

Как единогласно сообщают все источники, в обмен на указание места захоронения вина Рейзера будет переквалифицирована в убийство второй степени, что повлечёт снижение срока заключения до 15 лет. Из детективной литературы и политических сочинений (например, по Уотергейтскому делу~--- кажется, все его участники за время своей отсидки успели отметиться в мемуарном жанре) нам известно, что такого рода торговля, типа признания вины по таким-то пунктам обвинения в обмен на снятие пунктов таких-то, широко используется в американской судебной практике. Однако о такой форме~--- тело в обмен на минус 10 лет срока,~--- мне читать еще не приходилось (это, впрочем, не значит, что прецедентов не было). 

Впрочем, эту тему обсуждать еще рано: Рейзер свою роль сыграл, посмотрим, какова будет реакция судебных властей не на словах, а на деле. Соображения же, высказываемые в настоящей заметке, будут касаться только обстоятельств обнаружения тела. 

Далее пойдут свидетельства участников событий, взятые из различных интернет-источников, сопровождаемые моими комментариями. 

Итак\dots 

Эрси Джойнер (Ersie Joyner), начальник отдела по расследованию убийств Оклендской полиции 


\begin{shadequote}{}
\dotsсказал, что тело могло быть обнаружено только человеком, который вырыл могилу. 
\end{shadequote}

А по свидетельству Ричарда Теймора (Richard Tamor), одного из защитников Рейзера, 


\begin{shadequote}{}
Рейзер без труда определил место. Поверенный сказал: <<Он пошел прямо к этому>> (месту). 
\end{shadequote}

И тут возникает первый вопрос: а мог ли Рейзер определить место без труда? Для чего надо учитывать очень многие факторы. 

Между предполагаемым убийством с последующим захоронением и определением места последнего прошли год и девять месяцев; а редко ли нам приходилось искать вещь, которую мы положили куда-нибудь только вчера, и не в лесу, а в собственной комнате? 

Можно, конечно, допустить, что это место всё время заключения являлось Рейзеру в ночных кошмарах, подобно годуновским мальчикам кровавым. Но следует помнить и то, что сидение в тюрьме, да еще и в одиночке, отнюдь не способствует укреплению памяти (и психологической устойчивости вообще). А Рейзер~--- не Дерсу Узала, и не Улукиткан, что способны были отыскать в тайге\dots ну, что они были способны там отыскать, можно прочитать у Арсеньева и Федосеева. 

Напротив, Рейзер~--- сугубо городской человек, навыков лесной жизни вроде не имевший, глазомерной съемке не обучавшийся, сведений о занятиях его спортивным ориентированием также не имеется. 

Так что пришлось бы признать, что Рейзер обладает феноменальной зрительной памятью. Это объяснение сгодилось бы, но\dots 

\dotsзахоронение тела происходило осенью, а его обнаружение~--- в разгар лета. Как меняется пейзаж в зависимости от времени года, знают все, не только прирожденные таёжники. Так что то, что\dots 


\begin{shadequote}{}
Рейзер сказал властям, что надеется найти место могилы по вишневому дереву\dots 
\end{shadequote}

\dotsвыглядит довольно странно. Насколько я представляю себе этот тип людей, они не обратили бы внимания даже на то, что <<листья тополя летят с ясеня>>. И к тому же, по озвученной в Сети версии, захоронение происходило ночью\dots 

Вот и поставьте себя на место Рейзера: отыскать без малого через два года (проведённых в достаточно специфической обстановке) место, которое видел недолгое время, ночью, при иных сезонных условиях и совершенно иных обстоятельствах~--- легко ли? А если учесть еще и состояние стресса, каковой наверняка имел место быть~--- всё же Рейзер не Синяя Борода, и мочил своих жён отнюдь не ежедневно, чисто для тренировки, дабы обрести должную психологическую устойчивость\dots 

Иными словами, более естественным для него была не уверенность в указании места захоронения, а метание по всей округе в попытках вспомнить, где же именно оно было. 

Дальше~--- больше. Слово опять Джойнеру, который\dots 


\begin{shadequote}{}
\dotsсказал, что власти должны были <<удалить деревья и кустарники, чтобы добраться до могилы>>\dots добавив, что в одном месте им пришлось спускаться по веревке вниз по склону. 
\end{shadequote}

Из этих слов можно заключить, что и Рейзеру пришлось проделать тот же путь~--- напоминаю, ночью и находясь в состоянии стресса. При этом <<удалять деревья и кустарники, чтобы добраться до могилы>>, у Рейзера, в отличие от властей, вряд ли была возможность. 

Учитываем при этом, что обвиняемый нес на себе мёртвое тело. Каким бы ни был Рейзер здоровяком, и как ни субтильна была Нина, каждый, кому, увы, приходилось это делать, знает, что тащить мёртвое тело~--- совсем не то же самое, что живого человека, пусть даже раненого и находящегося без сознания. Это даже лошади понимают, не то что люди\dots 

Непосредственно о месте захоронения. Согласно одному из Сетевых СМИ, Рейзер 


\begin{shadequote}{}
\dotsпривёл власти к месту захоронения Нины, похороненной в яме 4 фута глубиной. 
\end{shadequote}

Согласно же Теймору, 

\begin{shadequote}{}
Тело было найдено в могиле приблизительно четыре на четыре фута 
\end{shadequote}

То есть мы имеем дело с небольшим таким шурфиком примерно $1,2\times1,2\times1,2$~м~--- более кубометра вынутого грунта, точнее, 1,733. О характере грунта можно только гадать, но, судя по тому, что властям пришлось <<удалить деревья и кустарники, чтобы добраться до могилы>>, как минимум, грунт был пронизан корнями растений. А приводимая в одном из источников фотография заставляет предполагать, что грунт был еще и каменистым. 

Я уже не помню норм времени на проходку поверхностных выработок при проведении горных работ, но поверьте, такой шурф за пять минут не выкопать. Да и малой пехотной лопаткой или даже обычной штыковой лопатой тут справиться трудно. Требуется наличие соответствующих инструментов~--- большой саперной лопаты (на гражданке именуемой <<лопатой проходимца>>), лома, кайла. Водилось ли всё это (или нечто функционально аналогичное) в хозяйстве Рейзера или его матери? И если водилось~--- то он должен был тащить этот шанцевый инструмент вместе с мёртвым телом. Ведь совершить две ходки значило многократно увеличить риск. Да и затраты времени возрастают: полмили (около 800 метров) по сильно пересеченной и заросшей местности (а она такова там и есть, судя по всем свидетельствам)~--- это отнюдь не 10 минут на километр шоссе. И, кстати, а куда этот инструмент делся потом? Спрятан, уничтожен, тщательно почищен и возвращён в сарай при доме? 

Наконец, последнее~--- о предшествовавших поисках тела Нины. 

По свидетельству одного из источников, 


\begin{shadequote}{}
В течении недель после исчезновения Нины Рейзер, полиция вела поиск трупа с помощью собак в тех же самых холмах, где в понедельник было найдено тело. В то же время добровольцы прочесывали эту область 
\end{shadequote}

По данным с форума Привет, где одно время была заведена специальная тема, посвященная исчезновению Нины Рейзер, в этих поисках участвовало до нескольких сот добровольцев. Конечно, вряд ли большинство из них обладало специальными навыками поиска чего бы то ни было. Да и среди полицейских, кажется, не наблюдалось ни Алёхина, ни Таманцева, ни Блинова. Но ведь и район поисков~--- отнюдь не Белорусские леса\dots 

Таковы вопросы. Читатель уже догадался, я думаю, что на них можно дать два не противоречащих логике ответа: 


\begin{enumerate}
	\item Место захоронения было подготовлено Рейзером заранее, ориентиры к нему~--- сохранены в памяти, чтобы найти его, что называется, с закрытыми глазами, инструмент, с помощью которого копалась будущая могила, либо спрятан, либо приведен в первозданное состояние, все следы <<земляных работ>>~--- подготовлены к быстрому уничтожению. Но в этом случае ни о каком непреднамеренном убийстве или убийстве в состоянии аффекта не может быть и речи~--- мы имеем дело с тщательно спланированным злодейством. Я не буду касаться психологической стороны вопроса~--- способен ли был Рейзер на такое злодеяние. А задам другой~--- могли ли в этом случае власти пойти на компромисс со столь отпетым преступником? Впрочем, это вопрос к американскому правосудию и совести участников процесса. 
	\item Рейзер не прятал тело мертвой супруги там, где оно было с его помощью обнаружено. А про место захоронения ему популярно объяснили. Но в этом случае вся версия обвинения летит в тартарары, вызывая град дополнительных вопросов, суть которых тотально можно сформулировать так: каким способом всё это могло быть осуществлено практически? И кому и зачем это было нужно? 
\end{enumerate}

Не берусь судить, какое из этих ответов правильный, тем более, что ими, возможно, варианты не исчерпываются. Требует же ответа самый главный, результирующий вопрос: неужели всех этих противоречий не заметили полицейские, прокурор и судья? <<Не верю>>~--- сказал бы Станиславский\dots 

Короче говоря, вопросы заданы. Получим ли мы на них ответы? 

\subsection{Точка в деле?}

Итак, дело Ханса Рейзера, создателя файловой системы ReiserFS, обвинявшегося в убийстве своей жены, можно считать законченным. Вердикт присяжных провозглашён~--- убийство первой степени. Приговор судьи вынесен: пожизненное заключение с возможностью досрочного освобождения через 15 лет, в случае положительного решения комиссии по помилованиям.

Тем не менее, более чем за полтора года никакой ясности в деле так и не появилось.

С самого начала дела меня преследовало ощущение deja vu~--- где-то я с этим уже сталкивался. А потом понял~--- это же повторение сюжета недописанного романа Чарлза Диккенса~--- <<Тайна Эдвина Друда>>. Со всеми его атрибутами: исчезновением человека, отсутствием трупа, косвенными уликами и явным подозреваемым, против которого нагнетается общественное мнение. Разница лишь в том, что в романе речь идет о литературных героях и злодеях, а в нашей истории~--- о реальных людях, наших современниках.

Я не собираюсь никого оправдывать и никого обвинять~--- просто потому, что не имею для этого достаточно данных. Но надеюсь, что ответы на вопросы, поставленные в послесловии, когда-нибудь будут получены. Пусть не в юридических документах, а в журналистских реконструкциях или даже в детективных романах\dots
\chapter{Старое}

\textsl{Под этой рубрикой я собрал свои старые заметки, написанные на рубеже тысячелетий и даже раньше. В основном они представляют интерес как исторические свидетельства. Хотя некоторые местами сохранили актуальность}.

\section{ОС в свободном исполнении} 
\begin{timeline}1998, весна\end{timeline}

\textsl{Это мой первый опыт сочинительства на Linux-темы. Опубликовано в виде письма в редакцию журнала Мир ПК, 1998, №4, с. 8. Размещается в виде, распознанном со скана (за что спасибо @Taciturn'у с Джуйки), с минимальной орфографический и стилистической правкой.}\medskip

Поводом для этого письма послужила <<Колонка редактора>> из №11/97. Стиль мне понравился: сдержанная ирония без наскучивших выпадов в адрес злобного монстра. Поэтому хочу поделиться некоторыми соображениями. Я не принадлежу ни к поклонникам, ни к ненавистникам Microsoft. Каковы бы ни были недостатки её продуктов, их доминирование~--- принудительная сила реальности (О. Куваев), особенно в России. И доминирование это нарастает со временем.

Ещё пару-тройку лет назад в каждом компьютерном журнале (и в Вашем в том числе) обсуждались сравнительные характеристики разных ОС для ПК. И список их (ОС) состоял не из одного и не из двух названий. Где же ныне NextStep для Intel? А OS/2? Как система для настольных персоналок она мертва, a Solaris вряд ли кто всерьёз рассматривал в качестве таковой. Да и Mac~--- если не умирает, то уж по крайней мере чувствует себя плохо.

Примерно то же происходит в области приложений. Несколько лет назад мой любимый текстовый процессор AmiPro на равных конкурировал с Word, a WordPerfect был уже недосягаемой вершиной. Где они теперь? У WordPerfect шансов на нашем рынке нет: последняя русская версия, если не ошибаюсь, 6.1 (ещё под Win 3.1). А AmiPro, этот <<Друг профессионала>> (или <<Подруга>>?) утерял(а) всё, даже собственное имя, сменив его на безликий WordPro. Мортиролог можно продолжить. Но безальтернативность убивает свободу. Ведь свобода~--- это свобода выбора, пусть и неправильного.

И тем не менее альтернатива существует, хотя компьютерная пресса и обходит её дружным молчанием. Ваш журнал~--- почти единственный, который регулярно, пусть и редко, упоминает о ней. Это~--- Linux, FreeBSD и другие некоммерческие Unix-подобные системы.

Однако все прочитанные мной книги и статьи о Linux имеют одну характерную особенность. Подробно описывается процесс установки системы. А что дальше? Перекомпилировать ядро? А потом ещё раз, добиваясь высшего совершенства~--- полного соответствия своим устройствам. А ещё можно написать программу для какого-нибудь своего уникального устройства, включить её в ядро и быть единственным в мире обладателем операционной системы, специально поддерживающей\dots ну, не знаю что. А что же все-таки можно под Linux делать?

Обычно пишут, что <<софта>> под Linux <<немерено>>, но упоминают, как правило, опять-таки компиляторы. Но не компилятором же единым жив человек! Утверждают, что всякого freeware и shareware очень много в Internet. И это действительно так. Но скачивать все подряд, наугад, руководствуясь лишь невнятным указанием, что это <<программа для модема>>, или <<хорошая программа для модема>>, или <<очень популярная (интересно, среди кого?) программа для модема>>, по меньшей мере, неразумно: это потребует времени, сопоставимого с возрастом Метагалактики. А рассчитывать на такое долголетие трудно, да ещё и поработать с этими программами хочется\dots

Пройдемся еще по Internet. Сколько Web-серверов в России построено на Linux или FreeBSD? А сколько, для сравнения, на WinNT или коммерческих Unix? А еще на подходе HURD\dots Вот и получается:

\begin{itemize}
	\item Linux \textit{etc}.~--- единственная в настоящее время реальная альтернатива Windows для настольных персоналок. 
	\item Неплохо бы иметь представление о параллельном мире (по Евклиду), это не помешает каждому грамотному пользователю ПК. Ведь главное все же~--- знать, что возможность выбора есть, даже если никогда ею не воспользуешься. 
	\item Освещение темы Linux в печати явно недостаточно. Особенно это касается приложений, не относящихся к средствам разработки. А ведь для Linux есть и StarOffice, и ApplixWare, и аналоги Photoshop и многое другое (сам убедился). 
\end{itemize}

Поэтому предлагаю рассмотреть возможность создания в Вашем журнале раздела, посвящённого Linux, FreeBSD, HURD, X~Window в свободном исполнении, а также приложениям для них. Есть же у вас раздел Macworld\dots А значение Linux сейчас если и ниже Mac'а, то не намного, особенно в Сети.


\textit{Post Scriptum от сего дня: не смотря на то, что отдельные статьи по Linux'у в журнале 
Мир ПК
 (а затем и некоторых других) стали периодически появляться, состояние с русскоязычными печатными материалами продолжало оставаться неудовлетворительным, а с Интернетом тогда были напряги не только в городах и весях нашей необъятной Родины, но даже местами и в её столице. Да и там сайтов Linux- и UNIX-тематики было не густо. Вот я и решил восполнить пробелы, сначала на бумаге, в журнале 
Byte/Россия
 (первом компьютерном журнале на Руси, в котором появился специализированный раздел 
Byte/UNIX
), а потом и в Сети. С чего и началась моя карьера линуксописателя, но об этом~--- в другой раз и в другом месте.}

\section{Куда падают камни?} 
\begin{timeline}1998, январь\end{timeline}

\textsl{Написано немало лет назад по поводу статьи Михаила Ваннаха <<Камень, падающий вверх\dots>>, опубликованной в журнале <<Компьютерра>>, \textnumero1, 1998}\medskip

Традиционно статьи преподобного Михаила Ваннаха~--- в Компьютерре одни из самых интересных среди всех, прямо не связанных с комьютерной тематикой. Они отличаются нестандартостью подхода и парадоксальностью выводов, тем не менее, как правило, достаточно обоснованных. Не является исключением и его последняя статья. Тем не менее она вызывает вызывает определенные\dots не столько замечания, сколько соображения, вызванные дискуссионностью ряда её положений.

Начну по порядку, с первой гипотезы Михаила Ваннаха и её обоснования (автор называет это иллюстрацией). 

Сначала~--- цитата:

\begin{shadequote}{}
Гражданские технологии почти всегда обладают более высоким качеством и более высоким отношением цена/качество, чем технологии военные.
\end{shadequote}

Сразу скажу, что с известными оговорками, о которых я скажу ниже, с преимуществом гражданских технологий перед военными нельзя не согласиться. Однако обоснование этого представляется мне совершенно иным. 

Утверждение, что война не есть нормальное состояние общества, может быть распространено только на настоящий исторический момент (\textit{да и то~--- со множеством оговорок~--- вряд ли, начиная с августа 1914 года, был хоть один момент, когда в какой-либо точке земного шара не происходило локальных конфликтов}). Поскольку только для него можно (и то достаточно условно) использовать слово <<общество>> в единственно числе. А до этого обществ было достаточно много. К стати, постоянно подчеркиваемое Георгием Кузнецовым (
\textit{тогдашний главред Компьютерры}
.) различие менталитета русского (или российского? или советского?) человека от человека американского~--- не более чем иллюстрация положения Л.Гумилева, что этнические различия~--- это различия стереотипа поведения. А ведь как ни относись к концепции Гумилева в целом, в точности эмпирических наблюдений ему не откажешь. 

Во всяком случае, в истории очень многих обществ неоднократно бывали эпохи, когда война всех против всех, война как средство производства была именно нормальным их состоянием. В качестве примеров достаточно вспомнить кельтско-германское общество Европы или кочевые общества Евразийских степей. И они, как и всякие традиционные общества, способны были существовать и существовали на протяжении длительного времени. По крайней мере, гораздо дольше, чем современная цивилизация евро-американского типа. Общий стиль жизни евразийских кочевников, не смотря на смену языков и религий, почти не менялся по крайней мере с VII-VIII вв. до н.э. до XVI-XVII вв.~--- эры нашей, варварских обществ Европы~--- с VI в. до н.э. и до конца эпохи Великого переселения. А отголоски его сохранялись в т.~н.~феодальной Европе~--- и до XVIII в. 

Кстати, т.~н.~буржуазная демократия, на которой основана современная евро-американская цивилизация и, следовательно, компьютерные технологии, в основе своей имеет не демократию Древней Греции (менее демократического, с современной точки зрения, общества трудно себе представить) или общехристианскую мораль, а именно стремление полудиких готских или франкских баронов отстоять свою личную независимость от столь нелюбимого преподобным Михаилом (да и многими другими) государства. 

Таким образом, нельзя согласиться с тем, что слабость военных технологий определяется узостью рынка. И не в отсутствии конкуренции~--- ведь в условиях перманентной войны, распадающейся на множество столь же постоянных локальных конфликтов, техническое превосходство должно было бы сказаться исторически мгновенно. Однако этого почти никогда не происходило. На мой взгляд, объяснение отставания военных технологий почти во все эпохи (подчеркиваю, почти~--- но об этом позднее) объясняется чисто практической причиной: эффективность ведения боевых действий связана не с технологическим превосходством, не с индивидуальным боевым мастерством (каковое можно рассматривать как частный случай мастерства технологического~--- в Японии, например, мастер рукопашного боя или фехтования, почитался наравне с прославленными поэтами, художниками, кузнецами или мастерами чайной церемонии). А, как это ни прискорбно для людей полагающих себя интеллигенцией, почти исключительно (и почти всегда) с превосходством в живой силе и её организации. 

Известный Н. Буонопарте однажды произнес фразу, варьирующую в разных переводах, о том, что счастье (или правда) всегда на стороне 
б\'{о}льших батальонов. А он кое-что в этом понимал. Военная машина революционной Франции была создана до Бонапарта (в том числе и теми, кого потом революция полечила от головной боли самым эффективным средством~--- гильотиной), и, впервые в истории, основывалась на всеобщей воинской повинности. Каковую не следует путать со всеобщей вооруженностью варварского общества~--- там владение оружием и участие в боевых действиях были просто атрибутом полноправного свободного человека: не хочешь~--- переходи в личную зависимость от другого, кто, в том числе, обязуется и защищать тебя. 

Прочие же европейские армии комплектовались на основе рекрутского набора или (как в Англии) на квазидобровольной основе. Что и дало решающее преимущество генералам революционной Франции. И легло в основу их тактики (создание которой тоже приписывается будущему императору)~--- атакам глубоких пехотных колонн, поддержанных столь же глубокими колоннами тяжелой кавалерии. Атакам под ядра и картечь, не считаясь с потерями (кое на что похоже, не правда ли?). Но того, что доходило, хватало, чтобы проломить линейные порядки противника. 

Так зачем же, спрашивается, совершенствовать технологии? Не надо, чтобы штыки были совершенней или порох~--- лучше. Лишь бы штык колол, и пули летели. А главное~--- что бы и того, и другого хватило на всех загнанных под ружье\dots 

Так что причина отставания военных технологий не узости рынка, а в его массовости. Зачем делать штык из высокосортной стали, если вооруженный им солдат с вероятностью, близкой к 50\%, будет убит, пустив его в ход один раз. И приклад его ружья, будь он хоть чудом деревообработки, будет сломан о вражескую голову. Напротив, все, что предназначено для ведения боевых действий, должно быть максимально простым и дешевым при массовом производстве. Именно поэтому всю Вторую Мировую войну в немецкой армии продержался технически несовершенный, мало пригодный для полевой войны, ориентированный первоначально только на вооружение мотоциклетных команд и танковых экипажей, пистолет-пулемет ERMA (в простонаречии именуемый <<шмайссером>>), хотя в опытных образцах существовало более совершенное оружие~--- штурмовая винтовка Шмайссера (настоящего)~--- примерный аналог и предтеча автомата Калашникова и винтовок серии M. 

Что же касается эффективности массовой военной продукции в гражданской жизни\dots Будучи геологом, неоднократно наблюдал попытки использования таковой в мирных целях. Не знаю, как комплект полевой одежды армии США, о котором говорит преподобный Михаил, но нет ничего менее пригодного для горно-таежных или горно-тундровых условий, чем любое обмундирование армии советской. 

То же и для стрелкового оружия. Оно (и наше, основанное на т.~н.~патроне Мосина калибра 7,62~мм, и американское серии M, с очень сходным по баллистике патроном, и немецкое, с патроном имени братавьев Маузер калибра 7,92) приспособлено для вывода из строя (не убийства, а именно потери способности выполнять боевую задачу) одного-единственного зверя~--- человека. В результате такое оружие слишком сильно для, скажем мелких копытных, и безнадёжно слабо для крупного и опасного зверя. А уж подранков\dots Ведь раненого солдата проигравшей армии можно дострелить и потом, в спокойной обстановке. А зверь ждать не будет. Лично видел лося, раненого четырьмя карабинными пулями (из кавалерийского карабина калибра 7,62), причем одна пробила желудок, а другая~--- превратила легкие в нежный мясной фарш), который сохранил способность не только убегать, но и драться. И, в конце концов, был убит пулей Майера 12 калибра из гражданской двустволки не высшего разбора. 

Если же говорить об оружии штучном~--- никакое армейское оружие с ним и рядом не лежало. Можно возразить, что оно несопоставимо дороже, и вспомнить пресловутое соотношение качество/цена. Первое~--- безусловно верно, а вот второе\dots Если штуцер фирмы Голланд и Голланд, стоимостью в некую астрономическую сумму, спас вам жизнь при столкновении с медведем, какую цифру вы поставите в числитель этой дроби? 

Но это лирика, напрямую не связанная с технологиями. А потому вернусь к вопросу, могут ли военные технологии быть лучше гражданских. Несколько примеров такого рода мне известно. 

Выше я уже упоминал о двух крупнейших <<варварских>> сообществах Евразии~---  кельто-германском (правильнее сказать, древнеевропейском, поскольку в него, по крайней мере временами, или балты, славяне, иллирийцы и т.~н.~<<народы между кельтами и германцами>>) и степном кочевническом. Так вот, вопреки расхожему мнению, и те, и другие достигли весьма высоких для своего времени технологических высот, во многом превосходя своих так называемых цивилизованных современников. И проявилось это технологическое совершенство в первую очередь в военной области. Скажем, кельтское оружие далеко превосходило по качеству обработки металла и, так сказать, тактико-техническим данным, оружие римское. Чем подтверждается ранее сформулированное положение, что для массовой армии, каковой была римская, нужно не хорошее оружие, а много оружия. Правда, и в гражданских областях производства, скажем землеобработке, древнеевропейцы опережали греко-римский мир: достаточно сказать, что конская плужная запряжка~--- их изобретение. И позднее, вплоть до высокого средневековья, франкские и сделанные по их технологии скандинавские и славянские мечи широко экспортировались в <<высокотехнологичные>> страны Ближнего и Среднего Востока. 

Аналогично и в степном мире. Гунны первыми на Дальнем Востоке начали производство высокосортной стали. Собственно, с этого и начался здесь железный век. Позднее победы тюркютов Первого каганата над окружающими народами, такими, как жужани, были во многом определены их технологическим превосходством~--- слава тюркютов как железоплавильщиков и кузнецов докатилась и до Византии. Но, повторю, в этом случае технологическое превосходство оказалось определяющим только благодаря равенству прочих условий~--- численности, организации, способу ведения войны. 

В чем же причина этого исключения из общего правила? В высокой роли войны в жизни этих обществ? Но для победы в войне, как я пытался показать, важно не хорошее оружие, а много оружия. А ведь даже стрелы кочевников, т.е., в сущности, расходный материал, отличает чрезвычайная тщательность выделки, обусловленная не только повышением баллистических качеств. В том, что война была средством производства и оружие, в сущности, было продуктом не военных, а гражданских технологий, пользующихся массовым спросом? В какой-то мере это так. Однако численность варварских сообществ была ничтожной по сравнению с так называемым цивилизованным миром, и о действительно массовом спросе говорить трудно. 

Думается, объяснение в том, что в варварском обществе оружие было необходимым атрибутом свободного, полноправного человека; не имевшие же оружия находились в разных формах личной зависимости. И поэтому оружие было необходимо для достижения максимально возможной в данных условиях личной независимости. А тут уже качество выходило на первые роли. Какой толк в многочисленных армиях, если ты был убит в поединке с равным из-за того, что у тебя сломался меч? Ну и, конечно, вопрос личного престижа, тесно связанного с личной независимостью. Как бы независим ни был человек, его самооценка должна подтверждаться оценкой равных:

\begin{shadequote}[r]{О.~Куваев, <<Территория>>}
Если тебе суждено быть лидером~--- равные выдвинут тебя лидером.
\end{shadequote}

Напрашивается такая аналогия. При советской власти, как мне кажется, максимально возможной личной независимости (или, если угодно, минимальной степени личной зависимости) и от людей, и от обстоятельств можно было достигнуть, работая научным сотрудником академического института, и достигнув строго определенного положения (не ниже, но и ни в коем случае не выше)~--- кандидата наук, старшего научного сотрудника: ниже ты обычно был мальчиком на подхвате, выше~--- на тебя возлагались обязанности, имеющие мало отношения к науке; бывали и исключения, были вариации в разных институтах, но тенденция именно такова. А что было атрибутом этого положения? Два диплома~--- об ученой степени и ученом звании. Так вот, дипломы эти в ВАКе заполнял один из лучших каллиграфов Советского Союза. А помните, с чем была связана задержка с их выдачей? Правильно, каллиграф впадал в запой. Равного ему не было, но на снижение качества не шли. И возмущения широкой научной общественности это не вызывало. 

Как вывод из всего изложенного: наиболее передовые технологии развиваются не в связи с областью приложения (военной, гражданской или какой еще), а там, где они способствуют достижению максимально возможной в данных условиях личной независимости. И современное высокотехнологическое евро-американское общество~--- лучший тому пример. Особенно компьютерные технологии. Это, по-моему, уже трюизм, но, честное слово, я его изобрел сам, независимо. Появление убогого, нелюбимого и ругаемого писюка аналогично патентованию полковником Кольтом механизма проворота барабана при спуске курка~--- и то, и другое уравняло людей в правах. 

Вернее, предоставило им более или менее равные возможности. Чем и дало им возможность добиваться личной независимости. Виртуоз револьвера Дикого Запада~--- такая же реальная фигура второй половины прошлого века, как и кустарь-одиночка с персональным компьютером~--- нынешнего (
\textit{теперь уже~--- прошедшего, в новом тысячелетии фигура эта становится почти анахронизмом}
). и определяющим в поведении для того и другого было стремление к личной независимости. 

И ведь первые добились её. Когда прошло время, выжившие ганфайтеры свою независимость сохранили: их приглашали шерифами и маршалами с высокими окладами и широкими полномочиями. Время вторых тоже проходит. Но тех, кто достиг определенного положения, приглашают на хорошие должности в крупные компании. И они выбирают, какое предложение принять. А что такое свобода, как не возможность выбора, и что такое независимость, как не возможность уйти, не боясь уморить голодом детей? 

Вот и все, что я хотел сказать по поводу первой гипотезы Михаила Ваннаха. 

В заключении~--- несколько разрозненных замечаний. 

Первое~--- относительно широко известного прецедента в Нюрнберге. Ни в коем случае не могу согласиться с его оценкой преподобным Михаилом. Это такой же акт государственного насилия, как и любой другой. Причем насилия одного государства, замаскированного под суд мировой общественности. Вспомните: Дёница, топившего англо-американские транспорты и пассажирские суда, оставили в живых. Советскому Союзу судьба его была в лучшем случае безразлична. Если не сказать больше~--- будущий потенциальный противник был уже определен. А Кейтеля и Йодля повесили. Это были последние, кто точно знал, кто, когда и как начал Вторую Мировую войну. 

Да и судьи кто? Самый страшный преступник из судимых Нюрнбергским трибуналом в палаческом ремесле был не более чем подмастерьем по сравнению с заурядным следователем ГПУ-НКВД\dots 

А на счет приказов\dots Армия, солдаты которой обсуждают приказ, проигрывает сражение до его начала. И это правило всех времен и народов. Другое дело, что погибнуть, сохранив независимость, или сохранить жизнь, приняв государство~--- вопрос личного выбора. Следует только помнить, что ничего нельзя приобрести, ничего не теряя. И отдавать себе отчет: без номинальной защиты государства каждый изрядную часть времени должен посвящать не работе за компьютером, а отработке приемов боя без оружия, стрельбе или фехтованию. Но это тоже каждый выбирает сам. 

И наконец, собственно по основной теме статьи. Не знаю, чью точку зрения это подтверждает, но является историческим фактом. Самый эффективный способ криптографической защиты был применен американцами во время военных действий на Тихом океане во Второй Мировой. Во флот было призвано примерно сорок тысяч индейцев-навахов (за точность цифры не ручаюсь~--- но много). Обучены на радистов. Направлены на каждый из боевых кораблей. Откуда и вели передачи. Открытым текстом. На родном языке. А в этноцентричной Японии не нашлось не только ни одного наваха, но и ни одного профессора навахского языка\dots 

\section{О Linux'е, католиках и гугенотах} 
\begin{timeline}1998.06.22\end{timeline}

\textsl{Эта заметка была сочинена мной в 1998 году и является одним из первых продуктов моего линуксописательства. Размещается здесь ввиду утраты доступа к оригиналу. Статьи, послужившие поводом для этой заметки, увы, также недоступны.}\medskip

Ответ на: \textbf{Алексей Костарев. <<ОС Linux, религия, движение Реформации XV века~--- аналогии>>. Невод, декабрь 1998 г.}

С большим интересом прочел эту заметку (как, впрочем, и большинство материалов НЕВОДа~--- одного из лучших в русскоязычном Интернете Linux-сайта). Поскольку она представляет, в свою очередь, реплику на статью Андрея Шипилова~--- пришлось прочитать и её. 

Относительно исходного материала (А.Шипилов. <<Религиозные войны. 20 век. Линукс на Comdeх.>> Компьютерра, N 48 от 11.12.98) многого не скажешь~--- ничего нового там не обнаружил, кроме обычного для данного журнала иронизирования над всем (чувствуется, что автор в анамнезе, как сказано,~--- писатель-сатирик). Аналогии с мусульманством (<<нет Бога, кроме\dots>>) притянуты за уши и свидетельствуют о незнании предмета. Каковы бы ни были разногласия между мусульманскими течениями, ислам всегда ощущал себя единым целым (Исключение, конечно, исмаилиты, карматы и тому подобные секты, но в своем крайнем выражении они к мусульманству не относятся). Не случайно, скажем, один из известных людоедов XX века Иди Амин, будучи номинальным мусульманином, нашел политическое убежище в Саудовской Аравии~--- гостей-единоверцев мусульмане не выдают (сравните судьбу другого людоеда, Бокассы).

И в теологическом плане мусульмане всегда ощущали свое генетическое родство с иудеями и христианами, называя тех и других, как и себя, людьми Книги, то есть имеющими Священное Писание и чтущие его, в отличие от зороастристов, маздеистов, митраитов, богословная традиция которых (не менее развитаятая, кстати, и оказавшая определяющее влияние на иудаизм в отношении эсхатологии и космогонии) была преимущественно устной (письменная фиксация Авесты ортодоксальных зороастристов~--- явление очень позднее, а, скажем, митраизм стал мировой религией вообще без таковой). 

Другое дело~--- реплика Алексея Костарева. Построенная на методе аналогии, базирующаяся на знании фактического материала (не смотря на скромное заявление автора, что оно~--- <<на уровне общеобразовательного>>), логично написанная, она вполне может послужить <<руководством к действию>>. Поэтому и счел своим долгом написать несколько строк. 

Во первых, будучи не профессионалом-историком, но историком-любителем (профессионал~--- тот, кто, владея соответствующими методиками, например, источниковедческими, археологическими и т.д., извлекает факты и создает концепции, любитель~--- тот, кто эти факты и концепции знает, ибо интересуется), должен отметить некоторые фактические неточности.

Во-первых, дело происходило не в XV, а в XVI веке~--- Виттенбергские тезисы Мартина Лютера были прибиты на двери Замковой церкви этого города в 1517 году.

Во вторых, учение Жана Кальвина~--- ни в коем случае не развитие взглядов Лютера, а совершенно самостоятельная концепция, восходящая в первооснове к Блаженному Августину (тогда как у Лютера можно уловить влияние Пелагия). И вообще, протестантсво, лютеранство, кальвинизм и т.д., как названия конфессий, в значительной мере порождение советской историографии. Последователи Лютера, насколько я знаю, называют свою веру Евангелической, а последователи Кальвина и концептуально связанные с ними течения (гугеноты Франции, пресвитериане Шотландии, утраквисты Чехии и другие) именуют свою церковь Реформированной. И друг с другом себя не смешивают. Известно, что избрание Фридриха Пфальцского (реформата) королем Богемии (т.е. Чехии) вызвало негодование евангелических князей Германии. В последовавшей за этим Тридцатилетней Войне (1618-1648 гг.) ни один из них не поддержал Фридриха. Так как не относился к нему как к единоверцу. 

Англиканство же, <<созданное>> Генрихом VIII Тюдором, имеет к Лютеру и Калькину весьма отдаленное отношение. Догматически и, особенно, ритуально т.~н.~Высокая (государственная) англиканская церковь очень близка к католицизму. И происхождение ее~--- сугубо политическое, аналогично Галликанской церкви Франции, которая, обособившись политически, в религиозном отношении осталась в системе католицизма. Так что, не будь Генрих VIII столь любвеобильным, а тогдашний папа~--- столь бескомпромиссным, ни о каком англиканстве мы бы и не слышали. Потому что Низкая Церковь и разнообразные реформатские течения типа пуританства, пресвитерианства, всякого рода индепендентов и т.д. развивались под влиянием взглядов Кальвина. 

Теперь по существу вопроса. Проводимые автором аналогии между противостоянием католицизма и протестанства в начале Нового Времени и софтверных монополий (в лице Microsoft \textit{etc}.) и сообщества разработчиков открытого софта (не знаю, как правильно можно назвать это очень разнородное множество) очень интересны и, действительно, напрашиваются. Я, во всяком случае, прочитал их с большим удовольствием. Очевидно, что автор статьи, поскольку он пишет на Linux-сайте, является не только пользователем, но и приверженцем этой операционной системы, как и всей концепции открытого программного обеспечения во всех его видах. И поэтому можно не спрашивать, на чьей стороне находятся его симпатии в этом противостоянии. Поэтому в своей исторической аналогии он эту свою симпатию переносит на оппонентов католицизма, рассматривая их, насколько я понял, в качестве носителей всякого рода добродетелей. Тогда как католиков~--- как аналог Империи Зла. А вот это уже~--- вопрос очень неоднозначный. 

Нельзя отрицать роль протестантизма (в широком смысле слова) в создании того, что принято называть современной буржуазной демократией, либерализмом и прочими словами, употребляемыми в хвалебном или ругательном смысле. В общем, современного общества европейского типа, во многом основанного на святости частной собственности. И определяемой этим независимости отдельного индивидуума. Однако заслуга ли это протестантизма? Скорее~--- того индоевропейского варварского субстрата, уцелевшего в Западной Европе, как в заповеднике (поскольку Западная Европа~--- не более, чем один из полуостровов Евразии). Именно в варварском понятии о Законе, как том, что повелось от века, о неотчуждаемой земельной собственности (одаль Норвегии, с некоторыми оговорками~--- аллод Франкского государства) лежит исток чувства независимости и личностное начало и средневекового рыцаря, и городского ремесленника, и современного частного предпринимателя. А протестанская идеология просто наложилась на этот субстрат. Там, где его не было или он не сохранился, как на Востоке Европы, реформатство прекрасно обходилось без демократических институтов. Ведь сейчас трудно представить, но в конце XVI- начале XVII веков дворянство Литвы, Чехии, Венгрии, Австрии было почти поголовно реформатским. А назвать эти страны столпами демократии того времени~--- несколько затруднительно. Достаточно вспомнить хотя бы Януша Радзивилла\dots 

И вообще преставление об общинах протестантов как о демократических институтах основано, по моему, только на выборности пресвитеров, существовавших в некоторых его течениях. Но ведь мы тоже выбирали народных депутатов (и не так давно). Правда, называлось это не просто демократией, а советской демократией, основанной на демократическом централизме (кто застал~--- знает, что это такое, а кто не застал~--- тому и знать не надо). К тому же в условиях замкнутости любой общины (а обособленность ранних протестантов от иноверцев~--- первооснова их поведения), да еще внешней угрозы, любое всенародное (в масштабах этой общины) собрание приобретает характер воровской сходки~--- тому в истории мы тьму примеров сыщем. Достаточно вспомнить тех же большевиков. 

К стати, о большевиках. Подсознательно они прекрасно понимали свою психологическую близость к ранним реформатам. Не зря в трудах советских историков постоянно проглядывает симпатия к последним. И при том тем большая, чем радикальнее и, следовательно, нетерпимее, те были~--- пуритане~--- лучше пресвитериан, а индепенденты Кромвеля~--- еще лучше. 

Относительно терпимости протестантов~--- это даже не миф, а прямая ошибка автора. Так, испанцы-католики, не смотря ни на что, пытались искать компромиссов с язычниками: конкиста Кортеса или Писсаро~--- не более чем высутпление горстки авантюристов (кстати, плохо вооруженных, почти без пороха, в проржавевших латах и с иззубренными мечами) на одной из враждующих сторон аборигенов. А то, что в результате они оказались хозяевами огромных стран~--- всего лишь превратность военной судьбы (не таково ли происхождение варяжской династии на Руси?). И после этого индейская аристократия~--- та, на стороне которой оказались испанцы,~--- заняла высокое положение в новых вице-королевствах. Немало родовитых мексиканских (и даже испанских) семей возводили свой род к табасским или тлашкаланским вождям. 

Реформаты~--- квакеры Новой Англии, буры Южной Африки~--- дали обратный пример полного отсутсвия не только возможности, но и желания сосуществования с аборигенами. Следствие~--- судьба восточных алгонкинов и готтентотов, коих ныне можно пересчитать по пальцам. 

По поводу особенных зверств католиков~--- также не вполне корректное замечание. История Религиозных войн во Франции пестрит свидетельствами о сожженных гугенотами монастырях (разумеется, со всеми обитателями), расправах над пленными и т.д. И Варфоломеевская ночь~--- отнюдь не коварный заговор католических фанатиков Екатерины Медичи и герцога Гиза, а стихийное возмущение парижан засильем чужаков-гугенотов (большая часть которых происходила с юга и в то время французами не являлась) и северо-германских наемников. Именно потому резня и приняла массовый характер~--- для решения политической цели достаточно было перебить несколько десятков, в крайнем случае сотен человек~--- Колиньи, Генриха Навварского и их приближенных. Достаточно прочитать Проспера Мериме~--- вступление к <<Хроникам времен Карла IX>>. 

Об испанской инквизиции~--- она была учреждена в конце как раз XV века, при Фердинанде и Изабелле, то есть до возникновения протестантизма как такового. И целью её была борьба с крещеными маврами и иудеями, оставшимися в Испании после завершения Реконкисты: тем и другим был преложен выбор~--- покинуть страну или креститься. И не потому она была создана, что большинство и мавров, и иудеев продолжало тайно исповедовать веру отцов, а потому, что финансовые группировки моранов финансировали предприятия магрибских пиратов~--- врагов испанской короны и союзников Франции во время Итальянских войн конца XV~--- начала XVI веков (тоже о нетерпимости католиков~--- французский король был католиком и носил титул Христианнейшего; другое дело, что политически он располагался в блоке, противостоявшем папе и Испании). Да и масштабы деятельности инквизиции сильно преувеличены вражеской пропагандой (в руках реформатов оказалось то, что послужило в дальнейшем основой свободной прессы, а также полиграфические мощности) и позднейшей (англо-американской и советской) историографией: документально известно, что за все время существования инквизиции (с девяностых годов XV века до взятия французами Мадрида в 1807, если не ошибаюсь, году, то есть за два с лишним века) репрессиям подверглось несколько тысяч (точного числа, к сожалению, не помню) человек: по сравнению с известными нам примерами масштабы просто смешные. 

К чему я все это написал? Не к тому, чтобы укорить автора в ошибках или неточностях. А потому, что проведенная им аналогия очень показательна. И вызывает другую, возможную, аналогию~--- а не повторит ли сообщество сторонников свободного софта судьбу реформатов в плане нетерпимости к инакомыслящим? Ведь, как реформаты взывали к авторитету Ветхого Завета (Израиль, поражающий филистимлян~--- излюбленный образ времен Религиозных войн), так и <<свободософтовцы>> (прошу прощения за неизящество выражения) аппелируют к первозданному UNIX'у. А обращение к корням всегда чревато тем, что каждый, как сказал патер Браун~--- естественно, католик,~--- <<\dotsчитает свою библию.>>\dots А <<\dotsбесполезно читать свою библию и не читать при этом библии других людей>>. И примеры этого можно наблюдать в нашей жизни. Лично мне не доводилось видеть фанатиков Windows или Word'а, а вот фанатиков протестантстких операционок~--- встречать приходилось. 
\textit{Напомню, что это написано 15 лет назад, с тех пор ситуация сильно изменилась.}


Так пусть же столь тонко подмеченная Алексеем Костаревым аналогия послужит предупрежедением протестантам-компьютерщикам~--- дабы не уподобились они генералу Сент-Клэру~--- старому англо-индийскому солдату протестантского толка из цитировавшейся выше <<Сломанной шпаги>>. Читавшие Честертона (тоже, кстати, католика, хотя и неофита) помнят, чем он кончил\dots 

\section{Криптосвобода} 
\begin{timeline}2001, осень\end{timeline}

В этой заметке не будет почти ни слова ни о Linux, ни об Open Source. События 11 сентября (2001 года) заставляют по иному взглянуть на ряд вещей, окраска которых до сего времени казалась вполне определенной. Одна из них~--- проблема криптографии в частной переписке: является ли злом возможность доступа к ней государственных спецслужб, или таковое может быть оправдано ситуацией.

До недавнего времени для всех людей, относящихся к внутренне цивилизованным, ответ был более чем однозначным. Ныне же раздаются голоса, что, может быть, это не так уж и плохо, если поможет изловить террористов? И, может быть, даже мочить их (в том числе и в сортирах).

На технической стороне криптографического дела останавливаться не буду (тем более что не очень-то в ней и разбираюсь). Но предлагаю задержать внимание на двух аспектах проблемы:

\begin{enumerate}
	\item Способствует ли криптография как таковая свободе? 
	\item Какие последствия могут иметь ограничения, накладываемые на криптографические технологии? 
\end{enumerate}

Должен сказать, что рассуждения о криптосредствах как одном из инструментов защиты свободы личности вызывают у меня, к сожалению, только ироничную усмешку. Наивно ожидать, что власти предержащие (кстати, наиболее точный перевод этого выражения на современный русский язык~--- держащиеся за власть) для подавления свободы в Сети (или любой иной свободы) будут ломать коды пользователей.

Нет, при необходимости у них в распоряжении имеются более действенные средства. Например, роты (полки, дивизии~--- сколько потребуется) автоматчиков, ходящих по домам и штык-ножами обрезающих витые пары, оптоволокно, телефонные кабели. А заодно~--- конфисковывающих модемы, сетевые карты, да и сами компьютеры~--- тоже.

За примерами далеко ходить не нужно~--- можно даже не говорить о талибах и 
\textit{офлайнистане}
. Но многие помнят, по крайней мере по рассказам, что одним из первых распоряжений после 22 июня 1941 года было~--- сдать радиоприемники. Дабы вражеской пропаганде не подвергаться. И ведь шли, и сдавали. А на особо несознательных (или дюже продвинутых) известно какое средство воздействия имелось.

Это было давно и неправда, скажете вы мне. Не согласен. История, когда её забывают, мстит тем, что имеет обыкновение повторяться. Да и вообще все понятия свободы и независимости (применительно к личности) могут рухнуть в одночасье. Хотите пример? Ну что же, пожалуйста~--- Германия, 1933-й.

У нас годами создавали представление о Германии как вотчине затянувшегося средневековья. А ведь это страна с вековыми традициями личной и общественной свободы. Германская сельская община~--- марка,~--- и магдебургское (поищите этот город на карте) городское право были \textbf{самыми} демократичными институтами средневековой Европы.

Да и вообще, именно германские (уже в широком смысле) представления~--- о неотчуждаемой собственности, обычном праве и т.д.,~--- легли в основу развития всей западной буржуазной демократии. Вся англо-американская юридическая система базируется на германском праве. Чтобы осознать это, достаточно прочитать подряд Сагу о Ньяле, а потом (в любом порядке) пару романов Гарднера о Перри Мейсоне.

Так что надеяться, что ничего подобного не может произойти в стране, где общеевропейские демократические традиции отсутствовали вовсе, где даже князья нехилых родов (от Гедиминовичей до Рюриковичей) величали себя не иначе как государевыми холопами (и отнюдь не фигурально говоря), было бы вдвойне наивно. Тем более что мы-то знаем~--- было, и не раз.

Это не к тому, что я против защиты приватности, в том числе и посредством криптографических технологий. Более того, я очень даже 
\textbf{за}. Просто нужно отдавать себе отчет, что в действительно критической ситуации не помогут ни PGP, ни файрволлы.

В свое время (на закате советской власти) мне довелось провести несколько полевых сезонов в Северной Корее (она же~--- КНДР, хотя сами они называют свою страну просто: Корея). Так вот, там в каждом гостиничном номере имело место быть подслушивающее устройство. Положение его было строго фиксировано: между портретами Великого Вождя и Дорогого Руководителя (для ясности замечу, что речь идет о 80-х годах прошлого века). Да оно, собственно, и не скрывается.

Глубоко убежден, что 99 и 9 десятых в периоде процентов этих устройств не работало за практической ненадобностью: ведь нужно не только подслушать, но и прослушать. А это, учитывая долю ненормативной лексики в речи советских специалистов любого ранга, задача нетривиальная и вызывающая в памяти историю о загадочности не только русской души, но и анатомии: <<Вася, надень шапку на~\dots, а то уши отморозишь>>.

Так что действие этих устройств~--- сугубо психологическое, призванное выработать у прослушиваемого (чуть было не написал~--- у пользователя) 
\textit{синдром Большого Брата}
. И бороться с такой практикой, на мой взгляд, можно только одним способом: не позволять этому самому синдрому развиться.

Таким образом, плавно переходим ко второму из заявленных к рассмотрению аспектов~--- что же даст законодательное ограничение криптотехнологий? В плане отлова террористов (видимо, на предмет дальнейшего замачивания в сортирах)~--- сказать трудно. Я тоже испытываю по этому поводу изрядные сомнения. В меру своей некомпетентности рискну предположить, что нецеленаправленная (=тотальная) перлюстрация может быть действенной только в том случае, если фильтруются \textbf{абсолютно все} почтовые сообщения. Что а) практически неосуществимо и б) потребует такого роста компетентных, как говорили у нас встарь, органов, количественного и качественного, что система ГПУ-НКВД покажется районным детсадом супротив Артека.

А вот то, что наличие <<черных ходов>> и прочих back doors в системах криптозащиты, вне зависимости, используются они на самом деле или нет (и даже существуют ли вообще~--- достаточно слуха об их наличии) будет способствовать <<синдрому Большого Брата>>~--- ясно любому, заставшему времена <<развит\'{о}го социализма>>. Вспомните историю про майора, которому <<ваша шутка понравилась>>.

Цель любого террора~--- не устранение врагов: как учит, опять-таки, история, их стараются убрать тихо и незаметно. Нет, она~--- в запугивании всех остальных, особенно~--- абсолютно посторонних, людей. И если законодательные ограничения криптотехнологий будут приняты, террористов с Боингов можно только поздравить~--- цели своей они добились на двести процентов.

И, кстати, стоит только начать. А далее ничто не помешает запретить в законодательном порядке и все проявления открытого и свободного софта. На том, например, основании, что его адепты в состоянии разработать криптосистемы, недоступные для т.~н.~<<профи>> из компетентных органов, и распространяемые безо всяких ограничений.

Заодно не нужно надеяться, что это затронет только Америку: как я уже говорил, в любой стране может случиться все, что угодно. А уж в нашей~--- и подавно. Ведь в ней уже происходило все, что угодно, плюс еще кое-что\dots

\section{О судьбе ресурсов Open Source в Сети} 
\begin{timeline}2001\end{timeline}

\textsl{От автора: эта заметка была написана много лет назад для онлайновой Софтерры. Наткнувшись на нее при разборке архивов, я подумал, что и нынче она не совсем потеряла актуальность.}\medskip

За прошедшие несколько лет мне довелось написать не один десяток заметок, связанных с Linux и открытым софтом, большая часть из которых размещалась на моих сайтах, кочевавших, в силу ряда причин, с хоста на хост. И, начиная с некоторого времени, я стал довольно часто получать просьбы о разрешить переразмещение своих материалов. Именно это и заставило меня задуматься о судьбе сетевых ресурсов вообще и ресурсов Open Source в частности. Наконец, собравшись с силами, я и решился сочинить данную заметку, в которой постараюсь максимально абстрагироваться от сугубо личных аспектов. 

Однако сначала нужно остановиться на общей характеристике сетевых ресурсов по обозначенной тематике и их особенностях в условиях постсоветской действительности. То есть ресурсах в пространстве, именуемом Рунетом, Префикс <<Ру>> в данном контексте символизирует лишь языковую принадлежность, но не физическое размещение сайтов или, паче того, не подданство или пятый пункт анкеты их авторов. 

Все Linux-ресурсы (буду для краткости пользоваться этим термином, но не следует забывать, что он подразумевает все, связанное с Open Source во всех его проявлениях), и не только Рунтовские, я разделил бы на три группы. Первая~--- это сайты крупных и долгоживущих проектов. Этой группе свойственна стабильность во времени и (сетевом) пространстве: история многих из них исчисляется уже десятилетиями.

Вторая группа~--- сайты более или менее крупных разработчиков дистрибутивов и иных, иногда даже коммерческих, программных комплексов. Их стабильность тесно коррелирует с положением создателей, в качестве коих могут выступать фирмы, отдельные лица или их объединения. Если одни сайты (Red Hat, скажем, или Suse) существуют и развиваются уже скоро десять лет, то другие~--- исчезают в одночасье или годами влачат вялотекущее существование. 

Наконец, третья группа, самая многочисленная~--- сайты Linux-энтузиастов, каковыми являются не только разработчики софта, разного рода группы пользователей, но и все более крепнущая группировка Linux-описателей и обозревателей новостей. Именно о них-то в основном и пойдет речь дальше. 

Сразу оговорюсь, что я не преследую дать какую-либо оценку рассматриваемым ресурсам с точки зрения объема контента, его полезности и т.д. В качестве примеров мной привлекаются сайты, которые в силу тех или иных причин я посещаю регулярно и за развитием которых слежу. Это отнюдь не значит, что прочие сайты~--- плохи или не заслуживаю упоминания: просто нельзя ведь объять необъятное\dots 

Так вот, отличительной особенностью таких ресурсов являются условия их хостинга. Насколько я могу судить по имеющимся (обычно косвенным) данным, они размещаются а) на серверах фирм или госучреждений (любого профиля, но, в общем случае, с Linux'ом никак не связанных), где работают их создатели, и б) на всякого рода серверах, предоставляющих услуги бесплатного хостинга. 

В первом случае размещение сайта может осуществляться с официального (или молчаливого) разрешения начальства. Или~--- просто явочным порядком, пользуясь его, начальства, некомпетентностью (или обычным пофигизмом). Случаи активной поддержки таких сайтов руководством, хотя и имеют место быть, вряд ли типичны и многочисленны. 

Как следствие, стабильность таких <<квази-фирменных>> сайтов может быть далекой от идеала. Что особенно характерно для госбюджетных контор: выгорание хаба или развал винчестера могут вызвать их паралич на неопределенный срок (это я по собственному опыту говорю). А всякого рода внешние обстоятельства могут привести к исчезновению ресурса. Примером чему~--- судьба замечательного (некогда лучшего в своей группе) html-редактора WebMaker (автор~--- Алексей Дець), сайт которого молчит уже который год (что отнимает шансы на его реанимацию). 

С сайтами, пользующимися бесплатным хостингом, другая сложность. В свое время я посвятил немало времени изучению проблемы бесплатных хостов, в результате чего вывел нехитрую закономерность их жизненного цикла. Сначала, сразу после создания, такой сервер функционирует просто замечательно, налагая лишь минимальные обязанности рекламного характера типа показа баннеров (а то и вовсе обходясь без таковых). 

Такая лафа имеет естественным следствием наплыв пользователей. По достижении некоей критической массы которых качество сервиса столь же естественным образом падает. Далее возможно несколько сценариев развития: 

\begin{enumerate}
	\item хроническая агония бесплатного хоста, вплоть до летального исхода; 
	\item ужесточение рекламных обязательств пользователей (к баннерам добавляются принудительные pop-up'ы и прочие прелести) на предмет отсечения халявщиков (или максимального затруднения жизни для тех, кому на халяву и уксус сладок); 
	\item беззастенчивое урезание пользователей бесплатного хостинга под любыми предлогами (типа~--- необновления в течении такого-то срока), в том числе и просто выдуманными; 
	\item наращивание его ресурсов (пропускной способности каналов, дисковых объемов и прочего) с одновременными мерами по компенсации понесенных затрат. 
\end{enumerate}

Компенсация эта может выражаться как теми же рекламными акциями, так и просто переходом к платному хостингу. Стоимость которого может быть вполне символической (10-50~у.~е.) в масштабах вечности, но более чем реальной для отечественных Linux-энтузиастов (большинство из которых излишними доходами не отягощены). Да и, по моим наблюдениям, <<малоплатный>> хостинг, в большинстве случаев, отличается от чисто бесплатного тем, что предоставляет услуги того же качества и количества, что и последний, но~--- уже за некоторые деньги. 

Я не хочу сказать худого про действия хостеров. Во-первых, <<халява, сэр>>~--- она халява и есть, грех на нее обижаться. Во-вторых, даже не лучший по качеству бесплатный хостинг все же лучше, чем никакого: вспомним, сколько интересных проектов выросло из бесплатной странички на \url{geosites.com} и подобных серверах. Перефразируя известное высказывание, можно сказать, что бесплатный хостинг~--- как секс: если он хорош, это здорово, если же плох, то это все же лучше, чем\dots эээ\dots~--- <<сам с собою одною рукою>>. И наконец, в третьих: существует некоторая категория бесплатных хостеров (хотя их скорее следует отнести к системам поддержки проектов), на удивление ответственно относящихся к добровольно принятым на себя обязательствам (и тут я не могу не вспомнить добрым словом Чертовы Кулички~--- недавно я там с удивлением обнаружил свою страницу давностью в несколько лет; воистину~--- <<хранить вечно>>). 

Однако факт остается фактом: в рассматриваемом контексте следствием является частая миграция Linux-ресурсов, а то и просто их, казалось бы, бесследное исчезновение: подсчитайте количество <<мертвых>> ссылок в любой их коллекции. Например, на \url{rus-linux.net}, почти исчерпывающей подборке русскоязычных Linux-ресурсов. И сколько интересных материалов скрылось с горизонта за последние три-четыре года. 

Все это было бы очень грустно, если бы не феномен эпохи Интернета~--- фактор переразмещения. На котором я хотел бы остановиться подробнее. 

Как было сказано в преамбуле, я довольно часто получаю письма с просьбой о таковом. И еще чаще~--- письма вроде: <<прочитал твою статью на \texttt{http://www.khren-znaet-gde.ru} и хочу сказать\dots>>. С одной стороны, доставляет мне известные неудобства: после многоточия обычно следует сообщение о допущенной мной там или здесь ошибке. 

Поймите меня правильно: я ни в коем случае не против таких писем . более того, очень даже <<за>>. Ведь именно таким образом я узнаю массу нового и для меня интересного. Однако ошибки, на которые в них указывается, с большой долей вероятности, исправлены в оригинальном материале. А иногда таковой и просто снят за утратой актуальности\dots 

С другой стороны, мне весьма лестно, что мои писания переразмещаются на многих сайтах. Это вселяет надежду~--- что бы ни случилось со мной или моим сайтом, они будут существовать, пока к ним сохраняется интерес. Хотя, должен заметить, что последнее время я по умолчанию разрешения на переразмещение не даю. По причинам, упомянутым в предыдущем абзаце, и еще некоторым, важнейшую из которых я назвал бы <<политкорректностью>>.

На вопросе <<политкорректности>> хотелось бы остановиться отдельно. Он включает несколько аспектов. Первое~--- ссылки на первоисточник, от которых не освобождает никакие представления о свободе и свободные лицензии. Почему-то во многих случаях считается достаточным привести e-mail автора~--- хотя, повторяю, автору это ничего не дает, кроме лишней головной боли (в виде писем <<как бы по делу>>, не говоря уже об откровенном спаме. 

И даже ссылка на конкретный материал с авторского сайта~--- не вполне корректный способ цитирования. Ибо настоящий авторский сайт~--- это цельное произведение, подобное книге. А ведь никому не приходит в голову ссылаться на главу из книги, а не на книгу как на целостность\dots 

И главное: переразместитель, по моему глубокому убеждению, должен не только пользоваться плодами чужого труда, но и принимать на себя некоторые обязанности. Из которых важнейшая~--- отслеживание изменений в оригинале переразмещенной статьи. Ведь сетевые материалы по определению не есть нечто неизменное\dots 

В общем, если культура цитирования бумажных изданий, складываясь веками, к настоящему времени устоялась (по крайней мере, среди образованной части населения, то культура онлайнового цитирования находится в зародыше. Хотя роль её в настоящее время трудно переоценить. 

Ведь именно практика свободного переразмещения в Сети спасла от забвения множество вещей, забвения отнюдь не заслуживающего. Для примера вспомним авторскую песню в своем классическом выражении, казалось бы, прочно забытую к середине 80-х. И вдруг волшебным, как из небытия, образом возникшую снова во второй половине 90-х. 

И в лице не только своих всенародно известных представителей. Но и тех, кто и в старые-то времена пользовались очень широкой известностью лишь в очень узких кругах. Ну кто, скажите на милость, лет десять-пятнадцать назад помнил, что Михаил Анчаров был автором всенародно известной <<Баллады о воздушном десанте>>, Владимир Ланцберг~--- культовых, как говорят нынче, для моего круга и поколения <<Пора в дорогу, старина>> и <<Мы условимся, трупов не будет>>. А уж <<Юнкера>>, <<Кони вороные>>, <<И падал я в душные травы\dots>> Бориса Алмазова, казалось, вообще канули в Реку Забвения\dots 

Впрочем, это~--- совсем отдельная тема. Как-нибудь, выкроив время и собравшись с мужеством, я сочиню статью о внутреннем родстве между явлениями советской авторской песни и движением Open Source\dots 

Можно сказать, что <<попавшее в Сеть . не вырубишь топором>>. Однако не нужно иллюзий на тему <<рукописи не горят>>. Мы-то знаем, что как раз они и горят, да еще как. Правда, и попавшее в Сеть вырубить можно~--- не топором, но рубильником в мировом масштабе\dots Тем не менее, Сетевые материалы я сравнил бы скорее не с печатной, но с устной традицией. 

Каковая, при всей своей кажущейся хрупкости, гораздо устойчивее того самого, <<написанного пером>>. Не зря же в седой и бронзовой индоевропейской древности только устная традиция считалась священной. Так было и у кельтов, и у иранцев, и у индоариев. Хотите примеры? Их есть у меня\dots 

О скольких авторах античной древности, начиная с Гекатея Милетского, зафиксированных письменно, знаем мы только то, что они были авторами чего-то (в лучшем случае . авторами чего именно они были)? А ирландские эпические сюжеты, именуемые в русских переводах ирландскими~--- не путать с исландскими~--- сагами (по ирландски они назывались просто scela, то бишь повесть), зафиксированные письменной традиции не раньше X в. н. э. (от Р. Х., как стало модно говорить в последнее время), восходят часто к рубежу эр (а то и к III-II вв. до н.э.. И все это время сохранялись исключительно в традиции устной. То есть устная традиция существует вплоть до поголовного физического истребления её носителей. Или пока они не потеряют к своей традиции интереса окончательно~--- такое тоже подчас случается. 

Итак, какова же судьба сетевых Linux-ресурсов? Да будут они существовать вечно, не на сайте автора . так на ином, где к ним был проявлен интерес. И появляться, после своего кажущегося исчезновения, снова и снова, в самых неожиданных местах. И так будет, пока тема материала сохраняет интерес для сообщества Open Source (хотя бы и исторический), пока существует само явление Open Source и пока существует Сеть как наследница устной традиции наших предков.

\section{О причинах, по коим геокомпьютинг должен стать основным полем приложения сил сообщества Open Source} 
\begin{timeline}2000 г, апрель\end{timeline}

\textsl{Первая версия этой заметки была сочинена в ночь с 1 апреля на день геолога 2000~г. и размещена на моем старом сайте на Чертовых Куличках. Давеча, разбирая свои архивы, я перечитал это сочинение~--- и вдруг понял, что оно отнюдь не потеряло актуальности. Поскольку намеченная в нем тенденция не только не ослабла, но и продолжает прогрессировать. Впрочем, судите сами. Ссылки, по причине лени, не актуализированы.}

\subsection{Что же такое геокомпьютинг}

Геокомпьютинг, с моей точки зрения,~--- это все, имеющее отношение в раной мере к геологии и компьютером. Утвердившийся термин <<геоинформатика>> мне не нравится, так как плотно ассоциирован с геоинформационными системами. Коими отнюдь не исчерпывается применение компьютеров в науках о Земле. Более того, именно здесь геоинформационные системы наименее применимы. По крайней мере, примеров успешной реализации ГИС-проектов в геологии пока немного. Почему? Причины, думается, в следующем.

Во первых, геология - наука кривая в прямом и переносном смысле (у верблюда спросили: почему у тебя шея кривая? - а что у меня прямое? - ответил тот). Так вот, все геометрическое в геологии - криво. То есть: стратирафические границы, осевые поверхности складок, линии надвигов; и даже трансформные разломы только аппроксимируются более или менее прямыми линиями. ГИС же эффективно работают при условии прямизны границ. Не зря же самые удачные примеры их использования относятся к лесопосадкам и избирательным участкам\dots

Вторая причина - геология по сути своей наука индуктивная. А индукция - это то, что великий Шерлок Холмс называл дедукцией: не иначе как его создатель Артур Конан Дойл, будучи студентом-медиком, не отличался прилежанием в области классической логики. И всякое геологическое действо (а геокартирование - особенно) происходит от частного к общему. Любой съемщик понимает, что для того, чтобы нарисовать карту в двухсоттысячном масштабе, площадь надо отходить как минимум со стотысячной детальностью. В основе же ГИС - принцип прямо противоположный: детализация генерализованного изображения.

Ну и в третьих, ГИС - это инструмент не столько исследования, сколько представления уже, так сказать, на-исследованного. Не умаляя важности этого, скажу: если я знаю, что нужно, представить, то уж как - соображу по возможности. Но проблема-то обычно упирается именно в незнание того, что\dots

Так что, не умаляя роли ГИС, скажу, что как инструмент исследования в науках о Земле более подходящим представляется то, что именуют задумчиво image processor. Применительно к геологии, это, в первую очередь, цифровая картография во всех её проявлениях. Поскольку геологическая картография~--- это основа любых построений в области наук о Земле. А потому возникает вопрос, существует ли для открытых и свободных платформ

\subsection{Приложения для геокомпьютинга}

Как оказывается, существует. В первую здесь следует упомянуть GMT - The Generic Mapping Tools. Это пакет программ, созданный профессорами-геофизиками Гавайского университета Паулем Весселем (Paul Wessel) и Уолтером Смитом (Walter H. F. Smith) еще в 80-х годах.

Эта программа распространяется бесплатно в исходных текстах (в полном виде около 50 Мбайт), компилируемых для любой Unix- или Unix-подобной системы. В настоящее время её можно обнаружить в системе портов FreeBSD и в портежах Gentoo. Она включает более 50 отдельных модулей для обработки двухмерных и трёхмерных картографических данных, построения на этой основе контурных карт, shadow map и истиннно трёхмерных блок-диаграмм. Которые могут быть записаны в формате EPS.

Помимо собственно пакета, на сайте авторов доступны многочисленные примеры его применения (главным образом для геологических объектов Гавайских островов), а также очень подробная документация в форматах PDF и PS.

Все модули GMT работают исключительно в режиме командной строки. Однако для этого пакета разработана и интерактивная оболочка iGMT, написанная на TclTk и работающая в графическом режиме. Её можно обнаружить на сайте Сейсмологического факультета Гарвардского университета.

Следующий пакет, заслуживающий упоминания~--- это GRASS, представляющий собой нечто среднее между ГИС и имидж-процессором. Он предназначается как для построения векторных карт, так и для обработки растровых изображений~--- космо- и аэрофотоматериалов, результатов спектрозональных съемок и тому подобного. Распространяется бесплатно, как в исходных текстах, так и в виде бинарных пакетов. Имеются также дополнительные базы картографических данных и примеры применения.

Кроме того, на ряде сайтов американских университетов и правительственных служб можно обнаружить упоминания о таких системах для геокомпьютинга, как SPRING, Xmap8 (нынешняя версия которого носит название Geotoush) и еще нескольких, теоретически заявленные как свободные и доступные для бесплатного скачивания. Практически, однако, скачиванию предшествует длительная и сложная процедура онлайновой регистрации. Которая подчас заканчивается предложением подождать письма с идентификатором и паролем для доступа на ftp-сервер. Возможно, это мое личное везение, но ожидание это оказывается, как правило, тщетным:-)

Таким образом, список работоспособного инструментария для геокомпьютинга под Linux сводится к двум позициям. К тому же ни GMT, ни GRASS не удовлетворяют в полной мере требованиям к таковому: первая~--- как чрезвычайно сложная в использовании, вторая~--- как имеющая ограниченные возможности работы с форматом DEM (Digital Elevation Modelling), основным для анализа геологического строения в региональном масштабе. И потому~--- оправдана ли постановка вопроса, вынесенного в заголовок статьи?

Правда, а есть ли к тому причины? Думаю, есть, и даже две. Первая - и, скажу честно, главная, - мне бы очень этого хотелось, потому что это то, чем я пытаюсь заниматься и что мне интересно.

Вторая же - более общего характера, и именно к ней я хотел бы привлечь внимание сообщества Open Sources, в первую очередь, конечно, нашего, российского.

\subsection{Технологическая перспектива}

Не секрет, что ныне - один из редких периодов в истории информационных технологий, когда программная индустрия не в состоянии выполнить своего сакрального предназначения. Каковым является (помимо, конечно, обеспечения работой нас, писателей в жанре технологической новеллы) выколачивание у пользователей денег на развитие аппаратных средств. Путем создания все более ресурсопожирающих программных комплексов~--- без этого остановится развитие индустрии хардверной.

Действительно, ныне практически с любым программным обеспечением общего назначения (и даже со многими узкопрофессиональными пакетами) с успехом справилась бы система на базе Celeron-400 с 64 мегабайтами памяти, любой (!) реально доступной видеокартой и самым маленьким, какой только удастся достать по блату, жестким диском. И даже (страшно сказать) большинство игр (за исключением уж самых навороченных) можно запустить на любой машине, каковую можно обнаружить в прайсах двух-трехлетней давности.

Но ведь развитие компьютерного железа не стоит на месте: частоты массовых (то есть недорогих) процессоров (и от Intel, и от AMD) давно перевалили за 2 гигагерца, память своеобычным ныне объемом обойдется в сущие копейки, а о жестких дисках и говорить не приходится: за разумные деньги можно приобрести их только такого объема, который не по силам заполнить даже при тотальном скачивании mpeg-музыки (и потому, дабы свято место пусто не осталось, приходится качать фильмы, как правило, посредственные:-)).

В перспективе же~--- тактовые частоты <<камней>> от Intel переваливают за 4 Ghz (и его эквивалент от AMD), 64-битные вычисления, память DDR2 с фантастической (и совершенно незадействованной) пропускной способностью\dots Как же убедить пользователя в необходимости приобретения этого богачества?

Появление все новых и новых версий Windows с их нарастающими хардверными аппетитами положения уже не спасает: даже адепты её признают, что в корпоративной среде <<нет никакой необходимости устанавливать Windows~2000 (\textit{не говоря уже о XP и тем более Longhorn}~--- А.~Ф.) на каждую рабочую станцию и каждый сервер сети>> (Шен Дэйли. 10 шагов для перехода к Windows~2000. Windows~2000 Magazine/RE, \textnumero1(4), 2000, с.~48). А значит~--- нет и необходимости в upgrade каждой машины. А это - бандитизм и бесчинство со стороны пользователя, не так ли?

Надежды, возлагаемые ранее на бурное развитие мультимедиа-технологий, похоже, себя тоже не оправдывают. И звук, и видео вполне успешно прокручиваются на современной машине средне-офисного ранга, а массового внедрения трёхмерной графики ожидать уже не приходится. В частности, по чисто психологическим причинам: на протяжении всей истории человек стремился аппроксимировать три измерения реального мира его двухмерными представлениями. Почему, в частности, я не думаю, что 3D-интерфейсы обречены на успех: первый такой интерфейс я увидел году в 92-м (3D-Room от Hewlett-Packard, если мне не изменяет память), и очереди пользователей за ними с тех пор не наблюдается.

То же касается и игровых приложений. Круг фанатичных геймеров, видимо, замкнулся, количественного роста его ожидать не приходится. А возможности аппаратуры ныне перекрывают с лихвой потребности самых крутых игр. Не говоря уж о том, что гейминг окучивает в первую очередь видеосистемы, но никак не комплекс аппаратных средств современной персоналки.

Так спрашивается, куда же девать лишние мегагерцы и мегабайты (а то и гигагерцы с гигабайтами)? Ответствую, аки отрок Феодосий: в технологии геокомпьютинга!

Обоснуй! - резонно скажете вы мне на это.

Что ж, попробую. Для начала~--- первая к тому посылка,

\subsection{Ресурсы}

Обработка картографической и аэро-космофотографической информации - в числе немногих действительно (а не искуственно, как текстовые процессоры) ресурсоемких задач, с которыми сталкивалась компьютерная индустрия за все время своего существования. Рядом (и даже впереди) я поставил бы (если не считать профессиональной индустрии развлечений) только метеорологию и всякого рода аэрокосмические приложения. Однако трудно представить себе, что рынок метеорологов-любителей или слесарей-надомников с персональным <<Бураном>> станет когда-либо массовым, и каждому потребуется суперкомпьютер с трехзначным числом процессоров. А вот рынок геосистем - может, о чем речь пойдет чуть ниже, в посылке второй.

Так вот, решение задач геокомпьютинга посредством ГИС и имидж-процессоров требует изрядной вычислительной мощности, оперативной памяти и дискового пространства (во времена оны, когда я активно этим занимался, у меня за неделю работы иногда набиралось по несколько гигабайт данных). А для визуализации результатов, особенно трёхмерной, весьма желательна могучая видеосистема, включающая и качественный (а значит - дорогой!) монитор. До недавнего времени эти задачи были вообще недоступны для настольных персоналок и решались (да и решаются) на рабочих станциях стоимостью в десятки тысяч условных единиц.

Ныне, начиная со времени появления процессоров класса \mbox{P-III/Athlon} (не говоря уже о Pentium-4 и AMD64), положение немного изменилось. И мощные ГИС и имидж-процессоры вполне могут функционировать на PC среднедомашнего (то есть игрового!) уровня. Однако и геософт ведь не стоит на месте: взамен аскетической командной строки GMT приходят развитые графические интерфейсы. Да и аппетиты растут: просто построения карты кажется недостаточным, хочется и 3D-моделей, и реалистичной генерации ландшафтов, и виртуальных облетов территории\dots

И ведь ГИС, хотя они и <<ГЕО>>, к Земле отнюдь не привязаны: есть материалы и по Марсу, и по Венере, и по старушке Луне, наконец. А там - свои проблемы обработки, требующие своих решений. И, соответственно, ресурсов.

Промежуточный вывод: геокомпьютерные технологии на сегодняшний день могут утилизировать вычислительные ресурсы почти любого масштаба. И почти во всем спектре производимых аппаратных средств. Вот только кому это нужно, кроме фанатиков от геологии? На сей предмет у меня заготовлена вторая посылка~---

\subsection{Перспективы массовости}

На все сказанное в предыдущем пункте можно возразить: это удел узких профессионалов, каких - единицы на всю нашу планету, народу это не нужно (поскольку то, что нужно народу - не это).

Сейчас это действительно так. Однако: дело идет к тому, что скоро GPS'ками высокой прецизионности будут оснащаться не только транспортные средства специального назначения, не только каждый катер и прогулочная яхта, но и автомобиль, мобильный телефон и прочие носимые устройства. Не говоря уже о доступности просто карманных систем позиционирования. Собственно говоря, процесс уже пошел: мой друг, выезжая на дачу на своем мотороллере, без GPS'ки не обходится.

И сейчас легко представить себе геолога (или представителя любой другой относительно массовой полевой профессии), привязывающего точку наблюдения не по трем засечкам компаса (или, паче того, по лаптям правее солнца), а по GPS тех же габаритов и веса; но - немыслимой при привязке в лаптях точности. У них это давно уже не мечты, да и у нас становится реальностью.

Точная йифровая привязка полевых наблюдений сразу вызовет астрономический рост объема данных. А цифровые данные, не подвергнутые количественной обработке, все равно что не существуют. И обработка эта требует соответствующих аппаратных ресурсов.

Но это - не все. Неизбежен рост прецизионности позиционирующей аппаратуры - ведь законы конкуренции действенны не только в компьютерной индустрии. Это приведет к тому, что морально устареют топографические карты, выполненные посредством мензулы и кипрегеля (и даже фотограмметрии аэроснимков).

Новые средства позиционирования потребуют разработки картографических основ принципиально нового вида и наполнения. Каких - даже не могу себе представить. Но что разработка их задействует достаточно вычислительных ресурсов - ясно. А учитывая потребность экспедиционщиков в количестве таких материалов - производство должно быть массовым. Я уж не говорю о таких мирных пользователях картографической продукции, как вооруженные силы: читавшие Виктора Суворова представляют себе, сколько экземпляров карт требуется для обеспечения боеспособности артиллерийского полка. Да и любого другого - тоже.

Дальше - больше. Ведь помимо профессиональных экспедиционщиков, существуют, так сказать, экспедиционщики-любители. Именуемые обычно туристами. Они - разные: от спортсменов-первопроходимцев маршрутов до отпускников-автомобилистов, прокладывающих маршрут из пункта Б1 в пункт Б2 (>>как известно, в Петушках нет ни пунктов А, ни пунктов Ц, а есть только пункты Б>>). И всем им потребуются карты нового поколения, программные средства для работы с ними и с данными прецизионного позиционирования, мощные компьютеры для запуска этих программ.

А все это вместе будет требовать вычислительных ресурсов, вычислительных ресурсов и вычислительных ресурсов (для подготовки карт, подготовки программного обеспечения и для его использования, соответственно).

Промежуточный вывод: развитие средств глобального позиционирования создает предпосылки для массового спроса на технологии геокомпьютинга самого разного уровня: если провести аналогию с системами обработки текстов (с чего и начался массовый спрос на персональные компьютеры), то это будут системы класса QuarkPress и Framemaker, во первых, класса Word - во вторых, и класса Lexicon - в третьих.

\subsection{Общий вывод}

Представляется, что колоссы компьютерной индустрии не могут не осознавать, что ноги у них имеют шанс из кремниевых неожиданно превратиться в глиняные. И потому не искать сферы приложения для своих mips'ов и гигагерцов. А таковой, как я пытался обосновать, в массовом масштабе ныне может быть только геокомпьютинг в широком смысле этого слова.

Потэому можно прогнозировать всплеск инвестиций в развитие ГИС, имидж-процессоров и ассоциирующих с ними продуктов (тех же генераторов ландшафтов, например). Что вызовет разработку новых и активизацию существующих коммерческих продуктов этого направления. Причем - не только узкоспециализированных, глубоко профессиональных систем, но и систем более или менее массового использования.

Поскольку именно предложение, вопреки Марксу, рождает (вернее, по-рождает, при соответствующих усилиях) спрос, таковой и будет порожден: должны же софверщики окупить свои производственные затраты. Что вызовет приток в эту нишу новых средств и новых участников. И - так далее, то есть система станет саморазвивающейся.

\subsection{Следствия для сообщества Open Sources}

Все это я написал не для того, чтобы дать совет хардверным и софтверным фирмам - они и без него обойдутся. Моя цель - обосновать тезис: впервые за всю историю всему миру Open Source дается шанс: не стоять в позиции для парада, отражая рипосты коммерческих производителей. Типа - на удар с кварты PhotoShop'ом мы ответим с терции GIMP'ом, на укол MSOffce - отводом Koffice, и т. д.

Нет, Linux -сообщество может оказаться на острие, что называется, прогресса. Ведь коммерческие ГИС сотоварищи развиваются уже давно. И неизбежно несут на себе (и долго еще будут нести) груз тяжкого наследия. В виде ориентации на профессиональные применения, архаических черт интерфейса, привязки к традиционной картографической базе и прочее, прочее, прочее. А кто будет спорить с фактом, что перестроить поточную линию сложнее, чем штучное производство?

Ныне работоспособных ГИС-подобных систем под Linux - раз, два - и обчелся (почти буквально). И это, товарищи, правильно - сложившиеся стереотипы, готовые наработки и тому подобные тормоза не будут мешать при создании новых продуктов.

А продукты эти, безусловно, будут конкурентоспособны. Даже не обсуждая вопросы качества и функциональности - просто по цене. Современные коммерческие ГИС - огромные программные комплексы с массой опциональных (но обычно - необходимых!) модулей; суммарная цена их (не по прайс-листам, а в реальности) достигает десятков тысяч долларов на рабочее место. Конечно, с выпуском массовых коммерческих продуктов цена падать будет, но - далеко не сразу. Помните, не так давно стандартный текстовый процессор стоил с полштуки американских рублей. Да и сейчас - почти столько же, потому как покупается в составе офисного комплекта.

Даже при худшей функциональности очевидна перспектива бесплатного и открытого софта для геокомпьютинга. А, как я пытался показать, функциональность его в данном случае может быть и выше: нет необходимости копировать чужие решения и обеспечивать совместимость с ними. Как нет и памяти прошлых решений - не всегда удачных и уж точно принятых в других условиях.

При этом я не призываю устроить всепланетную богадельню от Open Sources. Поскольку системы для геокомпьютинга идеально вписываются в модель его распространения. То есть сам софт распространяется бесплатно, деньги же берутся за установку, обучение, адаптацию под задачу; в общем, за то, что называется звучным заграничным словом support или не вполне адекватным нашим - поддержка.

Так вот ГИС, какими бы дружественными к пользователю они ни были, не тот софт, с которым сможет управляться любая кухарка. Просто по самой своей природе он требует некоторых специальных знаний и умений. И здесь поле для support'а - практически неограниченное.

Более того, Земля дана нам в единственном экземпляре, но объектов на ней - уж очень много. И каждый класс объектов потребует своих программных решений. И что сложнее - адаптировать готовую систему с открытыми исходниками под вашу задачу или ждать, пока кто-нибудь напишет коммерческую программу для её решения?

Ну а уж за это сам бог велел денег взять. Как сказал кто-то из великих инженеров прошлого: за то, что переключил контакты~--- с Вас~1 доллар, за то, что знаю, как это сделать~--- \$999\dots

\section{Потому и не любят} 
\begin{timeline}2002, весна\end{timeline}

\textsl{Эта заметка сочинялась, как своего рода открытое письмо госпоже SKV в ответ на её статью <<Почему они нас не любят или записки тетки-бухгалтера>>, опубликованную в онлайновой <<Компьютерре>>, если не изменяет память, весной 2002 года (ныне недоступна). Сочинялась по горячим следам, и так нигде не была опубликована. Давеча, копаясь в архивах, я на нее наткнулся - и подумал, что некоторую актуальность она сохраняет. Хотя многие реалии изменились~--- в частности, открытые и свободные программы разрабатывают нынче далеко не только из любви к искусству.}\medskip

Поскольку статья, послужившая предметом дальнейшего разговора, ныне недоступна, в двух словах перескажу её сюжет. Он сводится к жалобам на то, что компьютерщики~--- очень нехорошие люди, которые очень не любят бухгалтеров и не желают облегчить их нелёгкий труд разработкой специальных бухгалтерских программ под Linux. Поскольку соответствующие программы под Windows~--- платны и весьма дороги. Что и подвигла госпожу SKV ознакомиться с положением дел на альтернативной платформе.

Должен заметить, что автор этой заметки при своей попытке знакомства с Linux'ом действовала очень разумно и логично. Рассудив, что если первоисточник информации о Linux'е вообще~---  сайт \url{linux.org}, то первоисточником сведений о Linux'е русском должен быть его аналог в зоне ru. На форум которого она в поисках истины и отправилась. Продолжать, вероятно, не надо~--- там и у закаленного мужика уши подчас завянут. Остается только пожалеть, что она не пришла на один из более иных ресурсов, где джентльмены простили бы ей даже и <<социальный статус человека успешного и состоятельного>>.

А в целом обсуждаемая заметка показалась мне весьма интересной~--- как своего рода социологический показатель. Тем не менее, я не стал бы писать настоящие строки, если бы не пара-тройка толстых в ней намеков, показавшихся мне~--- ну, не то чтобы оскорбительными, но вполне дающими повод к ответу в том же духе.

Для начала хотелось бы рассмотреть вопрос из заголовка обсуждаемой статьи~--- <<Почему они нас не любят>>. Где под ОНИ подразумеваются, по контексту, трудящиеся информационной сферы вообще, адепты Open Source в частности и линуксоиды в особенности. А под словом НАС - работники финансовой сферы. Далее по тексту я для обозначения первых и вторах, соответственно, буду использовать идеограммы ОНИ и ВЫ, в требуемых грамматических формах. Ответить на этот вопрос не сложно, однако~--- не в одну фразу, а как минимум в две.

Первая~--- проста, это вопрос на вопрос: <<А кто ВАС любит?>> Упаси Боже, здесь нет ничего личного - местоимение, повторяю, относится не к Вам, а к ВАШЕЙ, сударыня, профессиональной касте.

Действительно, ВАС не любят все. Не любят руководители любых предприятий любого ранга~--- за требования того, что при соввласти называлось финансовой дисциплиной. То есть, по простому,~--- соблюдение внешних приличий при окучивании госбюджетных (или каких-либо иных) средств. Впрочем, как показывает практика, на этом уровне взаимопонимание легко достижимо~--- за исключением случаев клинической жадности.

Не любят ВАС и начальники подразделений (также вне зависимости от масштаба)~--- потому что, по искреннему убеждению бухгалтерии, вся их деятельность должна иметь целью только соблюдение финансовой отчетности.

И уж, разумеется, ВАС не любят рядовые сотрудники-исполнители~---  например, за маленькую и несвоевременно выплачиваемую зарплату. Объясняемую обычно финансовыми трудностями предприятия вследствие внешних условий. К слову сказать, сотрудников финслужб эти внешние условия почему-то обычно не затрагивают\dots

Честно говоря, я тоже не любил бы ВАС, ибо большую часть своей трудовой карьеры провел во втором качестве. Если бы не встреча на заре юности с неким бухгалтером, который умел внятно объяснить начинающему начальнику отряда суть требований финансовой отчетности. А главное~--- как и где их можно нарушать в интересах дела (а никакое дело при соблюдении всех финансовых требований делаться просто не могло, и это знали все), чтобы тебя не взяли за~\dots~сами знаете что. Чтобы понять меру его гражданского мужества, добавлю~--- было это более тридцати лет назад, когда нарушение финансовой дисциплины легкой словесной эквилибристикой могло быть приравнено к хищению социалистической собственности. Со всеми вытекающими последствиями\dots

Следствие сказанного выше~--- ИМ не за что любить ВАС. Поскольку все ОНИ с неизбежность попадают в одну из перечисленных категорий. И это, скорее всего, для Вас, сударыня, не секрет. Однако, судя по вынесенной в заголовок фразе, именно ОНИ, в отличие от иных категорий трудящихся, не любят ВАС каким-то особенным, нетрадиционным, способом.

Вероятно, это действительно так. И причину Вы интуитивно поняли сами. Она - в том, что ВЫ совершенно искренне убеждены в том, что ОНИ ВАМ что-то должны. Цитирую: 


\begin{shadequote}{}
Сделайте кто-нибудь консервативный офисный дистрибутив с тремя опциями\dots, напишите к нему хорошую документацию\dots
\end{shadequote}
и при этом
\begin{shadequote}{}
ознакомьте меня с ним.
\end{shadequote}

И все это, судя по контексту~--- на халяву, то есть по себестоимости носителя и доставки - в явном виде это не сказано, но подразумевается из жалоб на дороговизну легального проприетарного ПО для трудящихся бухгалтерий всех стран. А уж 


\begin{shadequote}{}
\dotsесли я посчитаю достойным
\end{shadequote}

… может быть и соблагоизволю купить (по той же цене).

Да неужели ВЫ всерьез думаете, что это не лениво? Ведь те, кто клепает открытые и свободные программы, чем бы они свою деятельность ни мотивировали на словах, на деле занимаются ею (в очень существенной мере) из любви к искусству. И, по большому счету, им глубоко безразлично, что ВЫ думаете по этому поводу. ВАМ они ничем не обязаны и не обязаны ничего. Им сугубо параллельны ВАШИ заботы о том, чтобы девушки-операционистки не имели в рабочее время возможности раскладывать Солитера или резаться в~\dots~ну, во что нынче там в офисах режутся? Как, впрочем, и ВАМ сугубо до лампочки, 


\begin{shadequote}{}
как конструктор рисует свои чертежи, а дизайнер - свои Гауссовые кривые.
\end{shadequote}

Это нужно ВАМ~--- чтобы ВАШИ девушки-операционистки при деле были. Так и флаг ВАМ в руки~--- найдите на просторах Рунета очередного молодого и в меру голодного гения, и он за умеренную мзду склепает ВАМ систему учета именно для ВАШЕГО предприятия. И даже две~--- одну вместо 1С, для фискальных органов, и другую~--- всамделишнюю, для гендиректора. А заодно уж и третью~--- лично для ВАС, на предмет дальнейшего ВАШЕГО успеха и состоятельности.

А теперь попробуем отделить зёрна от плевел~--- ибо если бы первых не было, не стоило и писать ответ. Я всегда скептически относился к идее офисного использования открытого софта. Однако при чтении рецензируемой заметки подумалось, что и здесь для него есть сфера применения~--- та, где человек выступает в качестве придатка к калькулятору (закаленного и отточенного орудия, как сказал герой известной новеллы Альфреда Бестера).

Конечно, сие есть профанация идеи Open Sources (основа которой - в том, что любой исполнитель \textbf{понимает}, что делает). Однако я не настолько идеалист, чтобы не знать~--- есть области деятельности, где шаг влево/вправо равнозначен побегу, а прыжок на месте~--- провокации. И именно там~--- место предельно \textit{закрытым}, то есть изолированным от пользователя, решениям, основанным на \textit{открытых} системах. Нечто подобное описал в свое время Владимир Попов в статье, которая увидела свет на страницах <<Системного администратора>>. 

Только ВЫ не должны думать, что кто-нибудь сделает это для ВАС, в расчете на ВАШЕ благоизволение~--- спасение утопающих, как известно, есть дело рук самих утопающих. Да еще и не рассчитывайте, что при этом обойдетесь без квалифицированного сисадмина (в дополнение к тем, которые <<честно говоря, не очень-то в курсе>>).

Разумеется, ВАС никто не неволит~--- сообществу Open Source как-то без разницы, выбросите ВЫ самортизированную 1С или купите свежую её версию (вкупе с Windows'ами, Microsoft Office'ами и прочими прибамбасами). Но иначе уж будьте добры мириться с играющими в Тетрис операционистками под ОС, с коей <<время, надежность и безопасность>> суть вещи несовместные.

Не извиняюсь, если <<получилось\dots не по доброму как-то, зло>>~--- материал к добру не располагает.

И ещё о зёрнах: в здравости самой высказанной автором идеи~--- о закрытой пользователю монофункциональной системе для выполнения нескольких предопределенных действий,~--- я за прошедшее время убеждаюсь все больше и больше. Время от времени с форумов доносятся жалобы~--- почему, например, на компьютере под Linux фильмы нельзя смотреть так же просто, как на бытовом плейере. Так давно уже~--- можно. Легким движением руки - всовыванием в CD-привод диска с Movix,~---  тысячедолларовый компьютер нечувствительно превращается в телевизор баксов за 200 (о чём говорится в заметке <<Почему компьютер не видка>>). Так неужели превратить его в кассовый терминал (или что там у НИХ в бухгалтериях)~--- сложнее? И, к слову сказать, FreeBSD как платформа такого терминала будет смотреться ничуть не хуже, нежели Linux\dots
\chapter{Притчи и легенды}

\textsl{Эти притчи и легенды были услышаны в разное время, от разных людей и в разных местах. Может быть, в этой книжке они и ни при чём~--- да в них намёк: линуксоидам урок.}

\section{Батыр и мамлюк} 
\begin{timeline}Июль 19, 2012\end{timeline}

\textsl{Эту древнюю красивую легенду рассказал некогда Шурик: после приключений с <<Кавказской пленницей>> на следующие каникулы он поехал не на Кавказ, а в Азиатские степи.}\medskip

Мерялись силами два джигита. Один~--- степной батыр, сам взрастивший своего коня, саблю которому отковал его кровный брат-кузнец, а чепрак под седло вышили руки любимой женщины. И второй~--- мамлюк от двора великого подышаха, при дворе его выросший и вскормленный, и амуницию свою получивший с казённых складов. Не из рук подышаха, конечно~--- тот такими мелочами не занимался. А перепоручал их своему сорок пятому визирю, потому как вороват тот был и ни на что больше не годился.

Результат предсказуем: победа осталась за степным батыром. Сразу и окончательно.

А вот если чуть изменить условия задачи. И против батыра нашего поставить\dots да, мамлюка с подышахова двора. Но двора того подышаха, который знал, что сила его~--- на мамлюках держится. И потому ни обучение, ни снаряжение их не перепоручал проворовавшимся визирям, а всем занимался лично.

И тут результат был бы уже не столь однозначен. Да и не допустил бы умный подышах такого поединка. Потому что и батыра, и мамлюка к делу приставил бы. К тому, для которого каждый лучше подходит. Первого~--- коней у соседей угонять, чтобы свежая кровь в табуны текла. А второго~--- в строю стоять, когда дело до серьёзной заварушки дойдёт.

Правда, в какой же стране такой подышах найдётся? Я такой страны не знаю. И в этом суть притчи, рассказанной нам Шуриком\dots

\section{Шашлык и люля} 
\begin{timeline}Август 31, 2012\end{timeline}

\textsl{Это тоже одна из рассказанных Шуриком легенд, которая объясняет, почему мало какая из открытых систем была доведена до законченного решения для конечных пользователей.}\medskip

В одной далёкой стране бегали по горам дикие бараны, архары они называются. И были они свободны, а охота на них~--- доступна для всех, без различия рода, звания, пола и даже сексуальной ориентации. Разумеется, обладавших должными умениями.

И жил в тех горах один вольный джигит, умениями такими обладавший. Ходил он по горам со своим карамультуком, и баранов тех стрелял. Потому что ему это интересно было. Ну и свежую печёнку архаров очень он уважал. А с остальным мясом не знал, что делать. Большую часть соседям по аилу раздавал, что-то продавал задёшево, чтобы на порох да свинец хватало. Не интересно ему было коммерцией заниматься~--- ему лишь бы по горам походить\dots

Так и было, пока не встретил он в кишлаке, в долине, одного мясника. Тот и предложил ему~--- а продавай мне все излишки мяса, я его на шашлык разделаю, отвезу в город и продам знакомому, владельцу ошхоны (столовки по нашему). И завернём мы на свободных архарах вот такой бизнес~--- тебе не только на порох и свинец хватит, но и на подарки для твоей пэри.

На том и порешили. Продолжает джигит охотиться, мясо, что не съедает, отвозит в кишлак и мяснику продаёт. Не дорого продаёт~--- не нужно ему было много денег: что с ними в горах делать? А мясник то мясо на шашлык разделывает и в город отвозит, да своему приятелю продаёт. А тот уже из него шашлык жарит~--- да такой, что к нему в ошхону весь город сбегается, да ещё и из других городов приезжают~--- специально шашлыка из архара отведать.

И всё было замечательно. Да вот однажды решил наш мясник, что неправильно это, туши на шашлык разделывать. Почему~--- только Аллаху ведомо. Может, подумал, что не нужно народу шашлык грызть, для зубов это вредно. А может, жадность его обуяла~--- ведь в шашлык требухи какой с жилами не подмешаем. Взял он, и порубил всё мясо-свежатинку в нежный мясной фарш. Привёз в город своему приятелю и говорит~--- ты с шашлычком-то завязывай, а давай торгуй теперь люля-кебабом.

Ошханщику делать нечего~--- к тому времени весь его бизнес на шашлыке из архаров строился. Ну и попробовал люля-кеба делать. А народ собрался и говорит: не нужен нам твой люля-кебаб, без вкуса и запаха, мы такой же в любой другой ошхоне поесть можем. И перестал он ходить в ту ошхону.

Ошханщик наш и подумал: как можно иметь дело с человеком, который сегодня мне шашлык из свежих архаров привозит, завтра~--- фарш на люля-кебаб, в который неизвестно что подмешано, а послезавтра вообще собачатину за мясо впаривать попробует. И в чайханщика переквалифицировался. А может быть, как злые языки говорили, в базарного сводника.

И ведь действительно, мясник сначала пробовал собачатиной да мертвечиной торговать. Пока однокишлачники про то не прознали и крепко его не побили. И тогда уехал он в очень далёкий город, в чужой стране. И стал там шаурмой приторговывать, да такой, что непонятно было, лаяла его шаурма раньше, или мяукала. А может, и вопросы глупые задавала\dots

Ну а джигиту уже на свинец с порохом не хватает, не то что на подарки~--- инфляция. А ни к чему другому, кроме как по горам ходить, у него душа не лежала. И пошёл он в соседние горы, к тамошнему беку в охотники наниматься. А беку охотники были без надобности~--- для него специально фазанов фисташками выкармливали. А надобные были ему аскеры, солдаты то есть. И нанялся наш джигит в войско бека, стал по его горам ходить, да врагов бека стрелять и резать.

Вот что бывает, когда свежее мясо в мясной фарш рубят\dots

\section{Легенда о советском рудознатце} 
\begin{timeline}Май 22, 2011\end{timeline}

\textsl{Эту старинную легенду я услышал много лет назад в одном рудничном посёлке. Я не скажу, в каком именно~--- чтобы не быть несправедливым к другим рудничным посёлкам, где вы могли бы услышать точно такую же легенду\dots}\medskip

Подобно историям Шурика и Юрия Деточкина, неизвестно, в каком именно из уголков нашей некогда необъятной Родины происходили описанные события, и происходили они ли вообще. Как я уже сказал, мне довелось услышать её на рубеже 60-х и 70-х годов прошлого столетия. И услышать от нескольких людей, вполне информированных, но, тем не менее, более чем заслуживающих доверия.

Все услышанные мной версии совпадали в главном, хотя и расходились в деталях, что позволяет предполагать, что в основе этой легенды лежит событие, действительно имевшее место быть. Расхождения же между версиями, как мы увидим ниже, не менее показательны, чем совпадения.

Судя по оценкам времени, началась эта история не позднее начала 60-х годов. Жил да был в одном из населённых пунктов бывшего Советского Союза (как я уже говорил, неизвестно какого) некий мужик. Населённый сей пункт, как уже говорилось, скорее всего, был рудничным посёлком.

А работал тот мужик, скажем условно, слесарем~--- хотя, возможно, столяром или фрезеровщиком. Важно то, что никакого специального образования он не имел, и профессия его не имела никакого отношения ни к геологии, ни к горно-рудному делу. Хотя из дальнейшего можно предположить, что работал где-то в этой сфере или около. Вероятно также, что профессия его была скорее по металлу, нежели по дереву.

И в один прекрасный день он чисто из любопытсва занинтересовался~--- а где же добывают те полезные металлы, с которыми он имеет дело каждый день. И начал он собирать сведения (инфу, как сказали бы мы сейчас) по этой теме. Гугла тогда не было, как и Интернета, и сведения свои он черпал из газет. Сначала~--- из поселковой многотиражки (что подкрепляет гипотезу, что дело происходило таки не просто в посёлке, а в посёлке рудничном). Не пренебрегал он и газетами центральными, типа \textit{<<Известий>>} и \textit{<<Труда>>}
.

Получив, таким образом, общее представление о горно-рудной базе во всесоюзном, так сказать, масштабе, он перешёл к газетам республиканского, областного и районного уровней. А там добрался до поселковых газет и многотиражек более иных, нежели собственное, предприятий.

Почему я и думаю, что какое-то отношение к горно-рудной промышленности он всё-таки имел: много ли простых советских трудящихся знали о самом факте существования Хайдарканской Геолого-разведочной партии или Навоинского Горно-обогатительного комбината? Не говоря уже о знании того, что практически каждая ГРП, рудник или ГОК выпускают свои газеты. В которых бодро рапортуют о своих трудовых успехах~--- приросте запасов, рекордам по добыче, и так далее. И на которые, тем не менее, может подписаться любой советский трудящийся.

Разумеется, никто, кроме сотрудников соответствующих предприятий, на них не подписывался~--- хватало добровольно-принудительного подписного минимума для коммунистов и беспартийных. А читало эти многотиражки~--- и того меньше народу, преимущественно те, кому это было положено по должности. Даже от отсутствия чтива они популярностью не пользовались: вопреки мифам нашего времени, как раз на заброшенных в тайгу или пустыню рудниках проблемы с печатным словом не было, можно было подписаться и на толстые журналы, и на собрания сочинений классиков, и даже на детективы из библиотеки \textit{<<Подвиг>>}.

Ну а наш герой изучал все эти газеты доскональнейшим образом. И извлечённую информацию сводил воедино. Сначала в терадках, потом, для пущей наглядности, начал выносить на карту Советского Союза. Масштаба и размера имевшихся в продаже карт ему скоро стало не хватать, и он собственноручно изготовил нечто вроде макета такой карты, на которую нанёс множество горно-рудных объектов (легенда гласит, что чуть ли не все). И не только местоположение самих рудников и ГОКов, но и каким-то образом умудрился привязать к ним всю текстовую и численную информацию, щедро черпаемую им из газет. Создав, таким образом, нечто вроде рукотворной ГИС~--- во времена, когда таких систем в компьютерном исполнении не было ещё и в проекте.

Вот так герой наш много лет жил-поживал, газеты читал, да ГИС апгрейдил с регулярностью, до которой современным софтостроителям~--- что до Пекина раком. Хотя и делал это в свободное от основной работы время. И был, по-видимому, счастлив. Но в один прекрасный момент заинтересовались им в высших сферах.

Почему? Тут начинаются первые расхождения. По одной из версий, кому-то из соседей показалось странным, что простой советский слесарь (повторяю, профессиональная принадлежность его определена условно) кипами и пачками получает газеты со всего Союза. И он сигнализировал о столь необычном поведении в компетентные органы~--- мол, не шпион ли?

Мне это кажется маловероятным: ведь это были времена далеко не сталинские, а как минимум позднехрущовские, а то и раннебрежневские, когда энтузиазм миллионов угасал на глазах. Да и у компетентных органов другие заботы появлялись во множестве.

Согласно второй версии, герой наш случайно познакомился в пивной с заезжим геологом и показал ему результаты своей многолетней работы. Геолог восхитился (субверсия~--- перепугался) и доложил об этом руководству. Это уже больше похоже на правду, но также не очень вероятно. Мощь геологов по части выпивки общеизвестна, но сколько же надо принять на грудь, чтобы забросить свои непосредственные дела и бежать докладывать руководству о каком-то слесаре. Мне, например, столько не выпить\dots

Третья версия~--- наиболее вероятна: герой наш не захотел быть собакой на сене, а решил поделиться своими результатами с народом. Каким образом мог поступить человек в его положении? Только написав куда следует. И вот в вопросе~--- куда именно следует,~--- существует, опять-таки, две субверсии.

Субверсия первая~--- что написал он в Министрество геологии СССР. Резонно предположив, что Минчермет занимается чёрными металлами, Минцветмет~--- цветными, Средмаш~--- радиоактивными, а в его ГИС охвачено всё горно-рудное богатство страны (горючими полезными ископаемыми он почему-то брезговал~--- вероятно, инстинктивно), а это должно проходить по ведомству Мингео. Логично и правдоподобно, но в сущности сводится к предыдущей версии, ибо ожидать реакции от министерских чиновников~--- ещё более наивно, нежели от заезжего полевика-геолога. Да и выпить они могут существенно меньше

И потому самое вероятное, что наш герой написал в ЦК КПСС. Тем более, что столь дотошный и аккуратный человек наверняка и на своей основной работе был передовиком производства, ударником коммунистического труда, а следовательно, вполне мог быть принят в Партию по разнарядке.

Как это ни покажется странным нынешнему поколению, в ЦК КПСС на письма трудящихся, особенно коммунистов, реагировали: сам был свидетелем одного такого случая и практически участником~--- другого, но об этом не сейчас. Реагировали, спуская вниз по партийной и ведомственной линиям предписания~--- разобраться и об исполнении доложить. После чего бюрократическая машина начинала крутиться~--- не очень быстро, но уже необратимо.

В нашем случае эта машина докрутилась до того, что приехала полномочная комиссия для ознакомления с результатами работы нашего героя (суб-субверсия~--- его самого вызвали в Москву вместе с его ГИС на заседание такой комиссии). И вот тут комиссия действительно восхитилась и испугалась: повторяю, на его карте были нанесены практически все действующие горнорудные объекты страны~--- открытые, не совсем открытые и вовсе секретные, то есть золото-урановые. Да ещё с пространственно привязанными сведениями о запасах, бортовых содержаниях, объемах добычи и так далее.

И, естественно, рукотворная ГИС была реквизирована~--- или, говоря политкорректней, взята для проверки в соответствующей организации.

Относительно финала истории также мнения расходятся. Согласно одним свидетельствам, мужика этого определили куда следует~--- не по договору, а по приговору. Я в это не верю~--- времена, повторяю, были уже не те.

Вторая версия гласит, что нашему герою без защиты диссертации (и, повторяю, без высшего образования) присвоили учёную степень кандидата геолого-минералогических наук, забрали в Москву и определили старшим научным сотрудником в один из закрытых институтов, дабы он продолжал там свой скорбный труд на благо Отчизны.

В самом по себе факте такого карьерного взлёта нет ничего невероятного, и примеры тому известны. Однако геологический мир тесен, и в этом случае, не смотря на всю секретность предполагаемого НИИ, о его дальнейшей судьбе имелись бы вполне достоверные сведения (как и обо всей этой истории в целом).

Наконец, третья версия, наиболее вероятная: нашего героя сердечно поблагодарили за проделанную работу и выписали ему премию в размере годового (условно) оклада. После чего он, лишённый дела своей жизни, благополучно спился и умер.

Слабым местом любой из этих версий (как и всей истории) является отсутствие сведений о дальнейшей судьбе первой в истории горно-рудной ГИС. Наиболее вероятно, что эту рукотворную ГИС задвинули дальний угол какого-нибудь ведомства или института. И со временем выкинули~--- при первой же инвентаризации, как не числящуюся на балансе.

Как говаривали древние, если эта легенда и не правда~--- то хорошо придумана. Хотя и немного грустно. Но ведь все древние красивые легенды немного грустны, не так ли?

\textsl{Post Scriptum. Читатель вправе спросить: а какое отношение эта легенда имеет к миру FOSS? Самое непосредственное~--- отвечу я. Ибо показывает, что потребность в информации, буде она у человека возникнет, удовлетворяется всегда~--- вне зависимости от того, открыты или закрыты её <<исходники>>. Нужно только желание информацию получить и умение с ней работать.}


\section{Притча о верхних людях} 

Как-то подумалось: всё больше и больше старых моих товарищей к верхним людям уходит. И попробовал я представить~--- какие они, верхние люди? Да так представить, чтобы и принципами своими атеистическими не поступиться, и про естественно-научное образование не забыть. Ну и чтоб братву от агностизма и деизма не обидеть.

И увиделась мне такая картина.

Ночь. Тайга глухая, непролазная. Прогалинка. Посреди неё костерок дымокурный. Вкруг него верхние люди сидят. Те, что когда-то в нашем мире жили, и кого ты уважал. И чьему суду подчинишься~--- без прокурора и судебного исполнителя, сам.

Водку они там у костерка пьют, и о делах своих верхних разговаривают. И ничего им больше не надо\dots

Но случается так, что кто из них устанет, от костра отойдёт, на отдых. И тогда они из нашего мира зовут. Не абы кого~--- зовут тех, кто для их компании подойдёт. Верхние люди -- они ведь мудрые, и много чего им ведомо\dots

А когда позванный приходит, они его очень спрашивают -- куды там гестапам всяким и прочим\dots По твоим понятиям спрашивают. И всё-всё тебе припомнят -- где и когда слабину дал, что был сделать должен, а не сделал, кому помочь надо было -- а не помог.

Много чего они вспомнят, и оправданий, типа что ну никак иначе не получалось, не примут. Потому что они верхние люди, и много чего им ведомо\dots

А потом за всё отчехвостят тебя в хвост и в гриву, и делом, и словом -- тем словом, что сильнее дела бьёт. Так что мало не покажется\dots

А потом скажут~--- да, сукин ты кот, много за тобой хрени всякой, и про всю мы знаем, и отмазки все твои знаем, что иначе, мол, не мог. Но ведь подлянок ты не делал, а за всю хрень сполна сам расплатился\dots

И подвинутся, и место у костерка дадут, и стопаря накатят, и оставят средь своих~--- тех, что уже там, и кто из нашего мира ещё придёт.

А если суда их не пройдёшь~--- оправят в рай, или там ад какой, пускай, мол, там с тобой разбираются, по ихним законам.

Если бы верхние люди хотели, назначили бы в мире нашем пророка, и религью такую учредили. Мол, в верхних людей веришь -- к ним попадёшь. А не веришь -- невесть где сгинешь. Но они не хотят\dots

И потому зовут они тех только, кто и так к ним придёт, нету кому пути ни в ады, ни в раи, ни в прочие кущеря.

Тех, кто судит себя сам, без бога и чёрта, и без прочего глюкавого. А они ведь только приговор формулируют\dots

\end{document}
