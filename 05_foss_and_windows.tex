\chapter{FOSS и Windows}

\section{Пора ли кричать караул?} 
\begin{timeline}
2001, лето
\end{timeline}

\hfill \begin{minipage}[h]{0.45\textwidth}
Мал-мал ошибка давал~--- \\
вместо ура караул кричал
\begin{flushright}
\textit{Международная присказка}  
\end{flushright}
\bigskip\end{minipage}

\textsl{Первая версия этой статьи была опубликована на \url{www.softerra.ru} где-то году в 2001-м. Однако она кажется мне не лишенной актуальности и поныне.}\medskip

Не знаю, привлекла бы такое внимание статья Дмитрия Коровина <<Потому что без \textquotedblleftВорда\textquotedblright ни туда и ни сюда>> (Компьютерра, \textnumero25 (402) от 03.07.01, с. 42-44), если бы не комментарий к ней Сергея Леонова (как и положено, на с.1). А главное~--- не рубрика, под которой статья была опубликована. Именуемая просто и тревожно: Караул!

Впечатление по первом прочтении~--- действительно, впору <<Караул!>> кричать. Это же надо~--- всю нашу Русь сельскую, исконную, о'Windows'ить и об'Office'ить, да еще со младых ногтей начиная. То есть со средней школы. Да еще в приказном порядке~--- по указанию сверху. И притом~--- за большие (называется сумма в 5 млн долларов) деньги, взятые, как это принято у нас, из бюджета. То есть из кармана граждан, все последние годы добровольно и добросовестно отстегивавших 13 процентов своих, с позволения сказать, доходов (а куда денешься~--- хочешь не хочешь, а в бухгалтерии все рано вычтут) на укрепление казны в лице отдельных её представителей (помните призыв: <<Пожалуйста, заплатите налоги!>>).

Факт, конечно, возмутительный во всех отношениях. И особенно~--- с точки зрения воздействия на неокрепшее сознание подрастающего поколения. Которое, как правильно заметил Сергей, будет ассоциировать компьютер со всем известной заставкой при его загрузке.

Воздействие на сознание школьников тем более преступно, что оно (то есть сознание их), в отличие от сознания чиновников, не закалено годами пропаганды преимуществ социализма. Хотя, кто знает, может, сникерсы и прокладки обеспечивают иммунитет не хуже. Вспоминая Hugges, который растет и развивается вместе с~\dots~(ну, сами знаете, с чем~--- тем, что становится все здоровее), готов в это поверить.

Стоит пожалеть и учителей. Ведь (цитирую приводимый в приложении к статье Дмитрия документ, из-за которого весь сыр-бор разгорелся) <<В комплект поставки каждого рабочего места учителя один (подозреваю, пропущено ``должен входить''---~А.\,Ф.) лицензионный установочный OEM CD-ROM с операционной системой и документацией по её использованию>>.

Судя по контексту, под операционной системой имеется ввиду (опять цитирую) <<Microsoft Windows2000/98Professional>>. Я, конечно, не знаток продукции Microsoft, но об ОС с таким названием слышать не доводилось. Так какой же ОС должны обучать учителя своих питомцев? Да еще по документации с OEM CD-ROM~--- покажите мне дорожку диска, на которой эта документация притулилась.

И с точки зрения отношения к отечественному производителю~--- поступок хамский. Ведь ранее наш производитель с успехом (правда, переменным) невозбранно внедрял свои Lexicon'ы на десктопы российских госчиновников. Тех самых, принявших пресловутое решение (не под воздействием ли несравненных достоинств Lexicon'а, к слову сказать?).

И те же госчиновники, ответственные за пресловутое решение, тоже поступили антигуманно~--- по отношению к самим себе, вроде унтер-офицерской вдовы: им же, беднягам, теперь придется оправдываться: мол, не брали мы взяток у Microsoft'а, вот-те крест, не брали!

Единственно, за кого можно, казалось бы, порадоваться, так это за Microsoft, даже и допуская, что той пришлось раскошелиться на взятки (но мы такой клеветы~--- не допустим!). Ведь если фундамент популярности её продукции на пост-советских просторах был заложен российскими пиратами, то крыша над ним (или для него~--- фундамента, то есть) окажется делом рук трудящихся Минобразования Российской Федерации. <<Я думаю~--- так>>~--- сказал бы умнейший из медведей.

Однако я пишу свои заметки не ради возвеличивания Microsoft, советских (пардон, российских) чиновников и даже не в заботе об отечественных производителях. А исключительно к вящей славе Linux'а и всяких прочих POSIX-совместимых операционок. И потому хотел бы взглянуть на это дело с другой стороны. Для чего придется обратиться к недавней истории.

Давайте вспомним, чему учили школьников в советской средней школе. Кому на Руси жить <<Хорошо!>>, и как <<гвозди бы делать из этих людей>>, и не для того ли <<закалялась сталь>>, чтобы забивать их десятью сталинскими ударами, детектив <<Кто же разбудил Герцена>>, эротические эссе о любви Партии к народу, приключенческие романы об Острове Свободы в флибустьерском море.

Чем отвечал на это благонамеренный советский школьник? Бодрыми пионерскими песнями типа <<Куба, отдай наш хлеб, Куба, возьми свой сахар!>> или о Гагарине и Хрущеве, сентиментальными загадками о съеденных шишках, народными частушками о гранате в окне дома Советов и лучшем стрелке маршала Блюхера (фамилия которого, как показало исследование Василия Ивановича Чапаева, с английского не переводится). И во всем этом~--- ни малейшей политики.

Теперь представим себе, каково будет отношение нынешних школьников к ненавязчивому, в рамках обязательной школьной программы, приобщению к мощи Microsoft Windows (будь она даже четырежды 2000 по 98Professional), да еще помноженной на достижения советской педагогической науки и (особенно) практики? Лично мне, из-за строгого старорежимного воспитания, приходят в голову только стихи вроде <<Мимо дома педагога без Excel'я не хожу: то в Windows Word засуну, то Access'а сынсталлю>>. А у нынешних школьников, в силу, повторяю, раннего знакомства с тампаксами и хаггисами, фантазия куда богаче. Так что нужно верить в нашу молодежь~--- реакция будет адекватной\dots

И потому каждый приверженец идеи Open Sources должен быть благодарным нашему Минобразу за то, что в качестве базовой платформы для обучения школьников основам компьютерной грамотности он не принял Linux или FreeBSD. Не согласны? А припомните-ка, какой срок карантина по окончании школы вам потребовался, чтобы прочитать, наконец, <<Капитанскую дочку>> или <<Мцыри>>?

Так что и Microsoft, на самом деле, завидовать не с чего. Вообще говоря, повод к настоящей заметке ложится в один ряд с изменением лицензионной политики и условиями регистрации грядущих программных продуктов, призванными, как вы помните, по возможности затруднить жизнь их легального пользователя, или заявленными аппаратными требованиями для Windows~XP.

Что наводит на размышления: а кто определяет политику гиганта софтверной индустрии? Не внедрился ли в её стройные ряды агент из движения Open Source? Если да~--- он воистину заслуживает недавно организованной премии проекта GNU. Если же это сознательная политика фирмы~--- не зря ли из нее пытаются вылепить образ врага свободы? Ведь ни Ричард Столлмен, ни Эрик Раймонд не смогли бы сделать больше для популяризации альтернативных открытых решений\dots

\section{О злокозненности Билла Гейтса, или почему я не люблю Windows} 
\begin{timeline}2003 г\end{timeline}
Написать эту заметку меня подвигло обсуждение на одном из форумов~--- зачем нужен Linux на домашней машине, постоянная переписка с моим коллегой и старым товарищем Владимиром Поповым, а также наблюдения и размышления. Название же возникло по аналогии с произведением известного древнеримского грека~--- писателя и моралиста Плутарха <<О злокозненности Геродота>>. Каковую Плутарх усмотрел в чрезмерном, по его мнению, возвеличивании Геродотом древних персов и прочих варваров.

Действительно, если вчитаться не в оценки Геродота, а в описание поступков, становится ясным, что всякие там Киры, Дарии и Ксерксы были ребятами в высшей степени разумными, справедливыми и ответственными. Чего далеко не всегда можно сказать о Фемистоклах там и Мильтиадах. И вообще, неизвестно, как бы мы относились к оплоту древней демократии, если бы до нас дошли какие-либо не-греческие источники о Греко-Персидских войнах\dots 

В этих не претендующих на систематичность заметках я не буду кричать <<Виндовс мастдай>> или призывать <<Линукс форева>>. А просто попробую объяснить, что же именно в Windows мне не нравится. Как известно, Линус Торвальдс в свое время назвал Windows объективно плохой системой. Основываясь, вероятно, на своем опыте разработчика. Я же скажу, почему солидарен с ним с позиций пользователя. 

Основываясь на своем более чем десятилетнем опыте эникейщика, все компьютерных пользователей я разделил бы на три категории. Первая~--- это те, которые хотели бы разобраться в системе до упора. Причем~--- вне зависимости от образования, опыта предыдущей работы, характера основной деятельности. С этой, немногочисленной, категорией все ясно~--- рано или поздно эти люди приходят в мир Linux/UNIX и Open Source. Как клиенты, эти пользователи поначалу способны доставить немало хлопот. Однако потом~--- потом общение становится двусторонним, и они превращаются в коллег и товарищей. 

Вторая категория, преобладающая количественно,~--- напротив, те, кто об устройстве компьютера, его операционной системе и прочих материях знать ничего не желают, им достаточно было бы того, что комп был кем-то запущен и как-то работает. Я не говорю, что это плохо~--- амбиции таких пользователей лежат совсем в других сферах. Более того, как клиенты такие пользователи очень приятны~--- они делают, что им сказано, и не занимаются никакой отсебятиной в тех вопросах, в которых они не разбираются и разбираться не хотят. 

Идеальной системой для пользователей второй категории был бы Мак. Опять же по моим наблюдениям, пользователь Мака вполне реально может не иметь представления не только об устройстве, скажем, файловой системы, но даже и слова-то такого, <<файл>>, не знать. Что вовсе не мешает ему очень эффективно использовать Мак в своей профессиональной деятельности\dots

Агитировать таких пользователей за Linux~--- ни один здравомыслящий человек не будет. Однако Маков у нас мало, они дороги, софт для них не очень доступен. И в итоге такой пользователь оказывается безальтернативно привязан к самой великой и дружественной операционной системе. Которая обещает решение всех пользовательских проблем. Если не сейчас, то уж в следующей версии~--- точно. 

Однако насколько оправданны эти обещания? Конечно, на элементарном уровне разобраться в Windows легко~--- достаточно взять книжку вроде <<Освой самостоятельно за 24 часа>>, чтобы научиться запускать Word или выходить в Интернет с Explorer'ом и Outlook'ом (можно и вовсе без книжки обойтись, методом научного тыка). 

Однако дальше начинаются проблемы. Оказывается, что для мало-мальски эффективной работы в Windows мало щелкать мышкой по иконкам и менюшкам. Приходится узнавать не только о том, что такое файл, но и (страшно подумать) о драйверах устройств, прерываниях и многом другом. Однако разобраться в этом досконально а) нет желания, и б) нет возможности. И в итоге пользователь второй группы волей неволей превращается в пользователя группы третьей.

А именно: вынужденно нахватавшись слов <<файл>>, <<драйвер>> и т.д., научившись менять разрешение экрана и обои рабочего стола, он проникается чувством собственного величия. 

Давным-давно в одной научно-популярной книжке по физике (это была <<Занимательная ядерная физика>> Константина Мухина) я прочитал, почему вредны научно-популярные книжки по физике: <<потому что создают иллюзию понимания там, где о настоящем понимании не может быть и речи>>. Так вот, эти слова в полной мере приложимы к Windows.

Действительно, пользователь Linux/FreeBSD \textit{etc}. просто вынужден идти до конца в освоении системы. Начиная лет пять назад свое знакомство с Linux'ом для того, чтобы иметь простую и удобную среду для писания научных статей про геологию (тогда я еще тешился иллюзиями, что это кому-то, в том числе и мне самому, нужно), я и помыслить не мог, что логика этого знакомства занесет меня в сочинение скриптов для обработки текстов или в дебри устройства файловых систем. А уж что я буду собирать собственную систему~--- такое мне и в кошмарном сне привидеться не могло. Как правильно заметил в своей известной книге Владимир Водолазкий, быть просто пользователем Linux'а скучно\dots Я бы добавил~--- пожалуй, что и невозможно. 

С другой стороны, как уже отмечалось, пользователь Мака может обойтись без знаний о системе вообще~--- ему довольно будет чисто практических навыков работы со своими программами. Относительно незнания слова файл~--- это я не выдумал. В нашем институте как-то одна девушка сдавала кандидатский минимум по информатике. Так вот, экзаминаторы задали ей вопрос~--- какие расширения имеют графические файлы. Последовал ответ: она работает на Маке, а там ни расширений нет, ни файлов\dots Конечно, впорос был поставлен некорректно, но и ответ, согласитесь, характерен. Причем ничего плохого как о специалисте я об этой девушке сказать не могу.

Windows же оказывается в данном случае той самой золотой серединой, которая~--- хуже всего (<<если ты первый, это говорит само за себя, если последний~--- можешь себя первым вообразить>>). Пользователя Windows просто против его воли толкают к компьютерной полуграмотности. Совсем без знаний ему обойтись невозможно (разве что по первому требованию вызывать домашнего эникейщика). Получать же эти знания часто нет желания, это во-первых. Во-вторых же и главных, разобраться в Windows до конца практически невозможно даже при наличии пламенного энтузиазма~--- столько в ней необъяснимого и непредсказуемого. 

Случай из собственной практики, послуживший последним толчком к искоренению Windows на моей домашней машине. Сижу себе, никого не трогаю, книжку верстаю, про геологию (если кому интересно~--- сборник <<Минеральные ресурсы России>>). Сроки поджимают~--- принесли мне текст вчера, срок сдачи в типографию, как водится, позавчера. Текст~--- в Word'е, в нем же, экономии времени ради, и верстаю (благо не произведения полиграфического искусства от меня ждут). Все нормально, как вдруг~--- по всему тексту закрывающие кавычки из французских самопроизвольно становятся английскими. Пытаюсь выполнить глобальную замену~--- ничего не получается. Заменяю глобально, вбивая номера кодов~--- с тем же эффектом. Причина непонятна, как бороться~--- не ясно. В итоге, как обычно, пришлось действовать через \texttt{/dev/ass}, но это уже не очень интересно.

Можете представить себе такую ситуацию в LyX или в OpenOffice? Даже если и можете, тут же и метод борьбы придумаете~--- открыть в текстовом редакторе исходник и поправить, что нужно (благо первый~--- почти \TeX, а второй~--- XML). 

Предположим, однако, что пользователь Windows не пасует перед трудностями. И готов к любым тяготам и лишениям, дабы досконально разобраться в любимой системе. И тут он обнаруживает, что затраченные им усилия и время таковы, что стоят освоения трех Linux'ов. Особенно если речь идет о всамделишней системе, то есть Windows~NT/2000/XP, в которых есть все то же самое, что в UNIX'е, только непонятнее. Так как глубоко закопано в недрах дружественного графического интерфейса. 

Как обычно, происходит типичная подмена понятий. Святая правда, что существует Windows, легкая в освоении, и что существует Windows устойчивая и надежная. Да вот только никто еще не доказал, что это~--- одна и та же Windows. А номенклатура типа Windows~98 или ME только способствует тому, чтобы пользователь путал теплое с мягким. 

Предвижу возражение: но зато Windows, при всех её недостатках, способствовало массовому внедрению компьютерных технологий. Если раньше на Маках работали, скажем так, представители творческих профессий, а на PC'ках, например, научные работники в <<черном DOS'е>> считали геохимические коэффициенты, то сейчас, под Windows, все достижения компьютерной мысли в руках любого юзера. 

Крыть нечем, так оно и есть. Однако зададимся вопросом, однозначно ли это хорошо, и хорошо ли для всех? Как сказали классики отечественной фантастики, <<медведя можно научить ездить на велосипеде, да только будет ли от этого медведю удовольствие или польза?>> Я уж не говорю о том, что распространение <<народного компьютера>> привело к невиданной бюрократизации.

Не секрет же, что все усилия по внедрению <<безбумажного документооборота>> привели пока только к лавинообразному росту документооборота бумажного . Раньше, чтобы создать руководящий циркуляр в четырех экземплярах, его сделовало написать от руки, отдать секретарше, той~--- аккуратно переложить листы копиркой, отпечатать, с помощью бритвы/забивки/замазки исправить ошибки. Теперь~--- врубил на PC'шке из Word'а печать хоть ста копий~--- <<и сидишь себе, болтаешь ножками, сам сачкуешь, а она работает>>. Задача, вполне посильная любому руководителю средней руки. А мы потом эти циркуляры~--- читай, да еще и расписывайся, что ознакомился. 

Я не призываю к луддизму. Потому что массовые компьютеры действительно сделали возможным для широких масс трудящихся-индивидуалов то, что раньше было по силам только госучреждениям и корпорациям. И причем на принципиально ином уровне, чем, так сказать, вручную. Однако это отнюдь не значит, что компьютеры сделали жизнь (и работу) легче~--- они сделали её другой. И для того, чтобы эффективно использовать их возможности, требуются не графические интерфейсы~--- требуется перестройка мышления, как это ни высокопарно звучит. И вот этой-то перестройке мышления Windows ни в малейшей мере не способствует. 

По моему глубокому убеждению, большинство пользователей применяет компьютер неправильно. И это не вина их, а беда: ничто в окружающем мире не толкает их к пониманию этой неправильности. К пониманию того, что с компьютером можно работать просто как с пишущей машинкой, но это все равно, что забивать микроскопом гвозди.

Парадоксальный, но хорошо знакомый мне пример: подавляющее большинство научных работников не использует компьютер как инструмент научного исследования, а только как средство представления его результатов. И не потому, что компьютер не нашел бы применения в исследовательском процессе, скажем, в геологии: просто, за исключением нескольких очевидных случаев (расчета тех же геохимических параметров, например) применения эти не лежат на поверхности. И, кроме всего прочего, требуют еще и изменения подхода к самому исследовательскому процессу. А вот этого-то никто не объясняет на курсах по подготовке к сдаче кандидатского минимума\dots 

В свое время, пытаясь объяснить сказанное выше своим коллегам, я, дабы они не приняли это в обиду, придумал такой пример, который и не устаю повторять. Представьте себе, что блестящую кавалерийскую дивизию в одночасье пересаживают на танки. Не только не переучивая их, но напротив, пытаясь убедить в том, что все приобретенные ранее навыки без всяких изменений можно использовать в новых условия. Например, скажем, рычаги~--- все равно что трензеля, педаль газа~--- что твое стремя, а переключатель передач~--- та же нагайка. 

А ведь это именно то, в чем пытаются, и небезуспешно, убедить пользователей апологеты Windows.

И в этом, на мой взгляд, главная злокозненность этой системы. Вместо ясного понимания того, что <<река это не дорога>>, пользователя тешат иллюзиями, что ему ничего не нужно менять в своих навыках и привычках, достаточно регулярно обновлять версии Windows (заодно с железом)~--- и жить будет легко и весело. И если пока его десктоп еще не точная копия письменного стола, то уж в следующей версии это обязательно будет исправлено\dots 

С массовым внедрением компьютеров мир изменился, и таким, как прежде, уже никогда не будет. И чем скорее мы все это осознаем, тем лучше. А потому системы, такому осознанию способствующие, должны развиваться, популяризироваться и распространяться. Чем, надеюсь, мы с вами по мере сил и занимаемся.

\section{Еще раз о доблести и злокозненности} 
\begin{timeline}17 мая 2005 г\end{timeline}

\hfill \begin{minipage}[h]{0.45\textwidth}
Не бойтесь сумы, не бойтесь тюрьмы,\\
Не бойтесь пекла и ада\dots\\
А бойтесь единственно только того,\\
Кто скажет: <<Я знаю, как надо>>
\begin{flushright}
\textit{Александр Галич}
\end{flushright}
\bigskip\end{minipage}


Сочинив пару лет назад заметку о причинах, по которым я не люблю Windows (по случаю, под впечатлением дискуссии на одном из форумов), я никак не ожидал, что спустя пару  лет она станет предметом активного обсуждения. В каковое поначалу вмешиваться не хотел~--- все, что я мог сказать по предмету разговора, было сказано мною ранее. Однако ввиду столь разветвленного обсуждения~--- не выдержал, ретивое взыграло.

Для начала в конспективной форме изложу причины, по которым я не люблю Windows (или, если угодно, по которым лично мне, как обычному пользователю, эта система не нравится).

Первая и главная из них~--- такова: обещая пользователю избавление от необходимости приобретения специфически компьютерных знаний, она своего обещания не выполняет.

Вторая причина вытекает из первой. Требуя от пользователя некого минимального объема знаний, Windows, в отличие от UNIX-подобных систем, ни в коей мере не подталкивает пользователя к их приобретению. Более того, просто провоцирует его на <<полузнание>> на уровне набора готовых рецептов, без понимания сути производимых действий. И добро бы ему, как, скажем, пользователю Mac'а, этого действительно хватало бы. Ан нет~--- рано или поздно ему приходится разбираться и в правах доступа, и в структуре реестра, и еще во многих вещах\dots Впрочем, тут мы возвращаемся к первой причине.

Наконец, третья причина кроется в проприетарной природе как самой системы, так и, главное, наиболее распространенного прикладного софта для нее. Разработчики которого, дабы побудить пользователя к <<смене вех>> (пардон, версий), вынуждены отягощать свои продукты новыми <<продвинутыми фичами>>, 90\% которых оказываются невостребованными пользователями. Что ведет к утяжелению программ без адекватного увеличения производительности и стабильности (а подчас наоборот~--- с падением и той, и другой).

Сравните монстроидальность любой современной программы <<малого базарного набора>> для Windows с отточенностью классических UNIX-утилит, функционально доведенных до немыслимого совершенства чуть не десятилетия назад.


\textit{Оговорка к пункту третьему: я отнюдь не считаю, что все программы вообще должны быть свободными и, тем более, бесплатными. Как и за какие деньги распространять плоды своего творчества~--- сугубо личное дело каждого, сподобившегося что-то сотворить\dots}


А теперь~--- несколько комментариев к мнениям, высказанным участниками дискуссии. Начну, разумеется, со статьи Вадима Валерьевича Монахова, которая и послоужила поводом для этой заметки. И, конечно же, с его программного утверждения:


\begin{shadequote}{}
Если судить объективно~--- MS~Windows~XP сейчас самая лучшая из имеющихся для пользователей систем.  
\end{shadequote}

Что ж, готов был бы поверить автору на слово. Однако доказательствами этого утверждения автор себя не обременяет. И потому возникает вопрос~--- а чем его субъективное мнение более весомо, чем мнение простого разработчика~--- Линуса Торвальдса, или простого пользователя~--- вашего покорного слуги. Только не подумайте, что я равняю себя с именами, упомянутыми выше. Однако мне видится весьма странным утверждение, что мой оппонент, позиционирующий себя в качестве разработчика, рискует определять, что лучше мне, пользователю. Может быть, я и сам справлюсь с этой нелегкой задачей?

Кстати, из сказанного выше можно сформулировать и четвертую причину моей нелюбви к Windows: уверенность её создателей в знании потребностей пользователей можно сравнить только с убежденностью вождей мирового коммунизма в понимании потребностей трудящихся (развивать эту тему можно было долго, но это~--- отдельная история, а потому~--- см. эпиграф).

Ну вот, в главном, вроде, высказался, остальное~--- мелочи.


\begin{shadequote}{}
У меня есть знакомый в фирме, где работа идёт на Макинтошах. Люди страдают!  
\end{shadequote}

Думаю, каждый, чьи амбиции лежат вне сферы IT~--- те, кого называют content creator'ами (и кто когда-либо видел Mac и его OS),~--- с удовольствием согласился бы пострадать вместо тех далеких ближних\dots

Случай из жизни. Лет эдак пять назад делал я сайт для нашей лаборатории. В силу специфики контента, требовал он масштабируемой графики. Для представления которой черт меня дернул использовать формат DjVu (незадолго перед тем появившийся). После чего со всех концов мира посыпались письма~--- мол, не видим мы твоей графики. Хотя на индексной странице русским (а также английским) по белому было написано~--- для просмотра графики скачать соответствующий plug-in. И указано, откуда именно\dots

И каким же бальзамом на мою душу был приезд одного мужика с Австралийщины. Который на аккуратный вопрос о том, видит ли он наши карты, ответил: конечно. Далее еще более аккуратно было спрошено: ты что, plug-in скачал, откуда было сказано? Да нет, отвечает он, ничего такого я не качал~--- да и знать не знаю, что это. Тут-то меня и надоумило спросить~--- а на чем работаешь-то? Да на Маке, ответил он.

Это что касается отношения пользователя (и его страданий). Однако оппонент затрагивает и другой вопрос~--- 


\begin{shadequote}{}
Почему Windows лучше для программиста  
\end{shadequote}

Хотя исходная заметка, вроде бы, не давала к поднятию этой темы ни малейшего повода. Поскольку я не программист~--- а, как неоднократно подчеркивалось, тот самый обычный пользователь, на страже интересов которого стоит Windows,~--- по существу дела мало чего могу возразить. Хотя (опять слово Вадиму):


\begin{shadequote}{}
Под LINUX так не сделать. Нет единой системы взаимодействия программ друг с другом!  
\end{shadequote}

Боясь показаться профаном, спрошу: а что, каналы (pipe) и сокеты (sockets)~--- это не средства взаимодействия программ друг с другом? И не отсчитывается ли их возраст с середины 70-х? Ах да, понял: это не единая система. Каковой являются, видимо, DDE, OLE в купе с COM'ом\dots

Что же касается <<Под LINUX так не сделать>>~--- это тот момент, где я радостно соглашусь с моим оппонентом: да, того, что он приводит по ссылке, не сделать ни одному линуксоиду. Видимо, и понятия о\dots да нет, не дизайне даже, а элементарном вкусе, у Windows-разработчиков совсем свои. Простому пользователю такой цветовой гаммы не осознать (о прочих элементах декора умолчим).

А дальше оппонент снова возвращается к теме страданий пользователя:


\begin{shadequote}{}
Под Windows можно написать программу, работающую с каким-нибудь форматом файлов, к примеру~--- видео. А после выхода нового формата пользоваться той же программой, просто догрузив кодеки. И будут показываться файлы тех форматов, которые еще не были придуманы на момент написания вашей программы.  
\end{shadequote}

Завершая абзац уже цитированной фразой:


\begin{shadequote}{}
Под LINUX так не сделать.
\end{shadequote}

Недостаточная информированность оппонента в этом вопросе объясняется, видимо, отсутствием общения с детьми школьного возраста, активно обменивающимися видеодисками разного рода. Каждый из которых несет на себе собственный кодек, упорно пытающийся установиться в автоматическом режиме (после чего ранее установленные кодеки, как правило, работать отказываются). А вот в Linux (и, замечу в скобках, во всех прочих свободных UNIX-подобных системах) можно использовать сквозной набор кодеков (например, от текущей версии mplayer'а) во всех программах, в которых таковые требуются. По крайней мере, это имеет место быть на машине автора этих строк (или я что-то сделал неправильно?).

От обзора дальнейшей дискуссии воздержусь: с высказавшимися в пользу Linux'а мне дискутировать особо не о чем (частные заморочки и непонятки можно решить в рабочем порядке). Тем же, чей выбор склоняется в пользу Windows, могу сказать: это ваш выбор, и, вероятно, у вас к тому есть веская мотивация. Напомню, что в исходной статье не содержалось утверждения: Linux лучше, чем Windows (Чем?~--- Чем Windows). А говорилось лишь о причинах, по коим я лично не люблю вторую из именованных операционок. Все люди разные, и некоторые, возможно, любят Windows именно за то, что я перечислил, как его недостатки. Однако, как сказал Д'Артаньян кардиналу Ришелье, <<по роковой случайности все мои друзья состоят в королевских мушкетерах, а все враги, по стечению обстоятельств, служат Вашему Преосвященству>>.

И уж совсем в заключение~--- именно для пользователей Windows. Не надо представлять себе пользователя Linux (или иной другой UNIX-подобной операционки) как некое существо, только и делающее, что настраивающее систему, или перекомпилирующее ядра. Отнюдь~--- подавляющее большинство из них занято практической работой, причем в самых разных сферах человеческой деятельности. Среди моих личных знакомых->>линуксоидов>> есть историки, юристы, экономисты, переводчики, в том числе с весьма экзотических языков. А также, простите, те самые секретарши и домохозяйки, интересы которых столь ревностно отстаивают разработчики Windows. И уж поверьте~--- мотивация для своего выбора у них была не менее веской, чем у вас. Возможно, следовало бы сочинить заметку~--- <<За что и почему я люблю Linux>> (хотя на самом деле Free- и прочие BSD я люблю больше:-)). Кто знает~--- может быть, и вы изменили бы свое мнение. Но это~--- совсем-совсем другая история\dots

\section{Чем удобна Вынь-да?} 
\begin{timeline}4 Июль 2007 г\end{timeline}

\hfill \begin{minipage}[h]{0.45\textwidth}
Все говорят: Кремль, Кремль. Ото всех я слышал про него, а сам ни разу не видел. Сколько раз уже (тысячу раз), напившись, или с похмелюги, проходил по Москве с севера на юг, с запада на восток, из конца в конец и как попало~--- и ни разу не видел Кремля.
\begin{flushright}
\textit{Венедикт Ерофеев, <<Москва-Петушки>>}
\end{flushright}
\bigskip\end{minipage}

Подобно лирическому герою Венечки, ото всех я слышал, как удобна Вынь-да для конечного пользователя. Слышать~--- слышал, а сам ни разу не видел. Были, конечно, старые смутные воспоминания~--- сначала о версии 1.0 (это был шедевр сюрреализма и абстракто-кубизма с примесью садо-мазохизма), потом о~3.0, потом~3.1 и~3.11. А там и всякие~95-е попадались, и прочие Линолеумы с Миллениумами\dots

Вот и вчера опять не увидел. Я, как только в нее, Вынь-ду эту (Windows~XP она нынче называется, кто не в курсе) загрузился, давай виртуальные консоли искать. Жму \keystroke{Alt}+\keystroke{Control}+\keystroke{F2}~--- нету. На \keystroke{Alt}+\keystroke{Control}+\keystroke{F3}~--- опять нету. И так до девятого стакана (пардон, до клавиши \keystroke{F9}). Но и на клавише \keystroke{F10}~--- ни малейшей виртуальной консоли не обнаружилось. Обидно, да? Досадно? Ну да ладно, обойдусь виртуальными рабочими столами в графическом режиме.

Не тут-то было~--- и виртуальных рабочих столов в этой самой Вынь-де нетути, один он единственный, на который предлагается все окошки потребные грузить. А мне их грузить надо~--- ох много, штук 6-10, да еще желательно чуть ни каждое в полноэкранном режиме.

Хорошо, а просто виртуальный декстоп, с разрешением больше физического, сделать-то можно? Чтобы хоть четыре окна размером в экран в него вписать? Нет, говорит мне Вынь-да, нельзя.

Ну опять-таки ладно, перетерпим, терминальное-то окно тут есть? Есть, ура, и командная оболочка в нем запускается. Только вот с командами не густо: 
\texttt{find} ~--- нет, 
\texttt{grep} ~--- нет, 
\texttt{cat} ~--- нет, 
\texttt{split} ~--- и то нетути. 

Что получается? Нехорошо получается. Ну хоть telnet какой-никакой запущу. Набираю~--- \texttt{tel}, жму табулятор. И вы думаете, у меня хоть что-то автодополнилось? Фиг с маслицем, как сказали бы в Одессе (правда, конечно, в Одессе покруче бы выразились, но не будем оскорблять нравственное чувство молоденьких девушек, возможно, читающих эти строки). Экранного буфера терминала~--- нет. Даже истории команд путёвой~--- и то нет.

Ну что же, а хоть текстовый редактор имеется? Имеется, Notepad~--- погоняло его. Но боже, если это~--- текстовый редактор~--- то я не иначе как китайский император и Папа Римский в одном лице. Это же то самое, что некий сексуально озабоченный юноша назвал жалким подобием левой руки: ни кейбиндингов, ни регэскпов, ни макросов (о протоколировании действий я даже и заикаться побоялся).

Да, приходится признать, что графа Монте-Кристо, сиречь пользователя Вынь-ды в рабочих целях, из меня не вышло. Но ведь глаголят, что несравненна эта система в целях развлекательных. Так что переквалифицируюсь-ка я в управдомы, лягу на диван и муз\'{ы}ку послушаю. Например, любимого Вертинского, взятого с сайта \href{http://bards.ru}{bards.ru}, очень успокаивает и на философский лад настраивает.

Опять незадача: все, что на \href{http://bards.ru}{bards.ru} лежит, лежит там в формате Real Audio. А медиаплейер Вынь-довый о таком формате и не подозревает, оказывается. Не беда, Ынтырнет, хвала Аллаху, все-таки есть (хотя как появился~--- не иначе попустительством Шайтана), так что скачаем с сайта производителя плейер соответствующий.

Скачать-то оказалось несложно~--- да вот только нынешняя его версия не играет те древние Real Audio, что были на \href{http://bards.ru}{bards.ru} (хотя, заметим в скобках, Linux'овые кодеки, например, от mplayer'а, справляются с ними шутя).

В общем, ни для чего Вынь-да оказалась не пригодной~--- ни для работы, ни для развлечений. Так что если кто скажет вам, что она удобна для конечного пользователя, то, вслед за Ходжой Насреддином, плюньте этому человеку в лицо, назовите лгуном и выгоните из своего дома. Наверняка человек этот ничего, кроме Вынь-ды, в жизни своей не видел~--- даже Кремля. Потому что каждый раз оказывался на Курском вокзале.

P.S. Автор этой басни, в отличие от своего лирического героя, знает, что для Windows существует POSIX-шелл с набором классических утилит, соответствующих вышеуказанному стандарту. Слышал он и о том, что для него можно скачать дополнения, превращающие этот шелл в полноценный bash, а POSIX-версии утилит равняющие по функционалу с GNU- и BSD-версиями. Есть у него подозрения, что виртуальный декстоп можно получить средствами драйверов от производителей видеокарт, а множественные рабочие столы учреждаются благодаря сторонним программам, типа Aston. Да и mplayer под Windows, вроде, никто не запрещает собрать.

То есть, если как следует постараться, Windows можно превратить в некое подобие Linux'а или *BSD. Но, как сказал известный советский поэт,


\begin{shadequote}{}
Можно бы~--- да на фига?
\end{shadequote}

Если в Linux'е или любой BSD можно иметь все это из коробки\dots

\section{Без-Win’ные машины: несколько крамольных соображений}
\begin{timeline}Март 5, 2009\end{timeline}
В последнее время обострился вопрос о возврате денег за предустановленную на новые компьютеры Windows и о продаже машин без предустановленной ОС вообще. Довольно давно он обсуждается \href{http://forum.posix.ru/viewtopic.php?id=1174}{здесь}. А недавно в кампанию за без-win'ные компьютеры активно включился ЦЕСТ, который \href{http://www.centercest.ru/archive/2009/02241.return.vista/}{намерен бороться с навязыванием OEM-программ}. В связи с чем возникла и ещё одна \href{http://linuxforum.ru/index.php?showtopic=87019}{аналогичная тема}. В этой связи и мне захотелось высказать несколько соображений, возможно, вполне крамольных. Почему эта проблема обострилась именно сейчас? Ведь на самом деле лет ей очень немало. Напомню, что первый в истории случай возврата денег за OEM-операционку имел место году в 1996-м. Когда один австралийский парень, пользователь Linux'а, приобретя ноут от Toshiba с предустановленной Windows~95, нафиг ему ненужной, добился соответствующего решения в суде. Точно не помню, сколько времени он на это затратил, но, кажется, не очень мало.

А ещё до этого, опять же должен напомнить, все компьютеры <<в сборе>> поставлялись с предустановленной DOS текущей или предпоследней версии. И тоже никто особо не жаловался. Правда, DOS в розничном (дискеточном) варианте стоила баксов 30-50~--- но в начале 90-х для постсоветского трудящегося это были вполне большие деньги.

С тех пор такие случаи происходили регулярно, но всегда были единичны, касались исключительно индивидуалов, и борьба проходила с переменным успехом. Если порыться в старых темах на форумах FOSS-тематики, можно найти как жалобы на отказ возврата денег фирмой имя рек, так и радостные сообщения о том, что после длительных и упорных препирательств пользователю удалось отстоять свою без-win'ную непорочность.

Начать надо с того, что под стыдливым эвфемизмом~--- предустановленное OEM-программное обеспечение прячется Windows. Причём в последнее время не всякого рода, а исключительно Windows Vista. Именно её в обеих редакциях~--- Home Basic или Home Premium~--- и предустанавливают нынче на машины. И именно она вызвала такой взрыв протестов, причём не только со стороны записных линуксоидов, но и стойких вендузяднегов. Повторяю, до недавнего времени такие протесты были единичны, а предустановленная Windows~XP обычно не вызывала жалоб даже от убеждённых адептов FOSS.

Так что первопричина~--- не в самом факте предустановки некоей ОС, а в общепризнанных <<высоких>> потребительских качествах одного из их представителей. Добавлю, что мнение о качествах Windows Vista~--- отнюдь не моё (я её в глаза не видел и, скорее всего, не увижу никогда), а исходит от лично мне знакомых пользователей Windows всякого рода (не <<вендузяднегов>>) с большим стажем и высокой профессиональной квалификацией в своей области, будь то администрирование, разработка, обработка изображений \textit{etc}.

Далее, протесты эти возникают почти исключительно при покупке ноутбуков. Хотя предустановленная ОС может быть обнаружена на <<брендовых>> десктопах и даже на персоналках отечественной сборки, возмущений по этому поводу я не слышал. И понятно, почему. <<Брендовые>> машины покупаются почти исключительно крупными корпоративными заказчиками в ипостасях производственных серверов и рабочих станций, а в таких случаях разговор о предустановленных операционках~--- совершенно отдельный, и к данной теме не относящийся. Что же касается отечественных сборщиков настольных компьютеров~--- с ними всегда можно договориться. И каждый, кто покупал в своей жизни более одного компьютера, имеет на примете одну или ряд дружественных фирм, где ему завсегда пойдут навстречу в этом щекотливом вопросе.

А теперь посмотрим, такое ли уж зло предустановленная Vista на ноутбуках.

С одной стороны, если рассуждать с точки зрения абстрактной справедливости <<для всех>>~--- безусловное зло. Во-первых, покупателя грубо решают свободы выбора~--- одного из величайших достижений мира FOSS. Во-вторых, злостно нарушается закон о защите прав потребителя, запрещающего продавать <<товар в нагрузку>>. В-третьих, покупатель теряет на этом свои собственные деньги, если он не намерен пользоваться <<свистой>>, вне зависимости от того, на какую операционку~--- Windows~XP, Linux или FreeBSD,~--- он её сменит.

А с другой стороны, если посмотреть на дело с позиций справедливости суровой, но конкретной, все эти во-энных следует разобрать по пунктам. О лишении свободы выбора можно говорить только в отношении тех покупателей, которые этой самой свободой в состоянии воспользоваться. Смею заявить с полной ответственностью, что доля их в общем количестве людей, покупающих ноутбук, ничтожна. А для всех остальных свобода выбора~--- это лишь свобода выбрать более дешёвое решение (да и то~--- очень условно дешёвое), которым они всё равно не смогут воспользоваться. Почему~--- скажу несколько позже.

Нарушение закона~--- это, конечно, плохо. В юридические тонкости вдаваться не буду за некомпетентностью. Но если исходить из здравого смысла~--- а неужели это самое страшное нарушение закона из тех, с какими нам приходилось сталкиваться? Если же подойти с позиций сугубо технологических~--- с помощью несложной софистики можно доказать, что никакого нарушения тут нет. И это я также надеюсь продемонстрировать в дальнейшем.

Наконец, третий вопрос~--- о затратах на товар, насильственно навязанный и покупателю не нужный. Этот вопрос~--- самый сложный, и потому распадается на множество подвопросов.

Для начала оценим масштаб затрат. До недавнего времени долю предустановленной ОС Windows в общей цене машины (повторю, речь идёт практически исключительно о ноутах) определить было практически невозможно. Практически все ноуты от средней и выше категории продавались с OEM'ной виндой <<в комплекте>>. Некоторые бюджетные и большинство супербюджетных моделей так называемых отечественных производителей продавались с FreeDOS, которая теоретически не должна была бы добавлять к цене ничего. Но конфигурационно модели с Windows и с FreeDOS не пересекались. Что же касается немногочисленных моделей с предустановленным Linux'ом~--- этим делом также баловались <<отечественные>> производители,~--- ценообразование на них вообще было (да и по сей день остаётся) необъяснимым ни с марксисткой, ни с анти-марксисткой платформы.

Почему я говорю~--- так называемых отечественных производителей? Потому что платформы всех ноутов (в том числе и почти всех зарубежных <<супербрендов>>) изготавливаются на Малой Арнаутской улице острова Тайваня и её продолжении в континентальном Китае. Это на Руси знает любой ребёнок, изучивший во времена Венечки Ерофеева способы очистки политуры. А вот где изготовляются всякого рода дополнительные комплектующие конкретной модели~--- не знает даже он.

Ныне ситуация изменилась. В прайс-листах многих фирм можно видеть модели отечественных брендов, зарубежных полубрендов и супербрендов, лежащих в вышесредней ценовой категории, укомплектованных абсолютно одинаково (то есть ближе к верхней планке сегодняшних стандартов), с двумя вариантами предустановленных ОС~--- FreeDOS и Windows~XP. Так вот, разница в цене между ними составляет\dots~40 баксов. Конечно, можно найти и идентичные цены, и цены, различающиеся на~5000 рублей (в <<пользу винды>>, так сказать), но это можно отнести к естественной человеческой жадности и перегибам на местах.

Однако, если ориентироваться на компании типа HP (40 баксов были вычислены именно по её продукции) и около, у которых рекомендованные розничные цены одинаковы по всему миру, и <<белой>> накрутке при ввозе, разница получается именно такая. Что, кстати, я, не имея прямых данных, высчитал ещё пять лет назад, когда на ноуты предустанавливалась XP, а я писал заметку с эпиграфом~--- Квинтилий Вар, верни легионы. Кстати, эта самая гипотетическая (или расчётная) разница в, грубо говоря, полтинник <<бакинских>>, не зависела от редакции: просто считалось приличным комплектовать почти бюджетные модели <<хоумовой>> редакцией, а заведомо не-бюджетные~--- <<профессиональной>>. Возможно, просто в соответствии с реалиями страны, в которой 600-й Pentium обязательно должен сопровождаться цветным монитором~--- малиновым, с золотыми кнопками.

Много это или мало? Опять-таки, очень как посмотреть. С одной стороны, со стороны человека, покупающего ноут за полторы штуки (вне зависимости от того, покупает он его <<по делу>> или <<прикиду для>>), разницы нет просто. Потому что она будет скорректирована в соседнем салоне при пересчёте уёв в без-уи по внутреннему курсу. Причём скорректирована в любую сторону~--- очень хитрожопый человек придумал понятие уёв сразу после августа 98-го.

А само слово, замечу в скобках, придумал автор этих строк сразу после явления понятия. Вполне возможно, что кто-то придумал этот термин независимо от меня и практически одновременно~--- уж очень он напрашивающийся. Но спорить о праве первородства не буду.

С другой стороны, у покупателя возникает вопрос~--- а какого такого зелёного я должен платить свои кровные сорок уёв, пусть даже для меня это и не деньги, за то, чем я пользоваться не буду? Для этого посмотрим, а что же он получает за свои кровные уи.

Давайте подумаем. Продавец должен продемонстрировать покупателю работоспособность приобретаемой машины. В самом первом приближении, для машины без предустановленной ОС, это достигается нажатием кнопки Power и демонстрацией картинки BIOS Setup~--- типа машина таки включилась. Кстати, на машине с предустановленной FreeDOS результат будет тот же самый~--- разве что вместо картинки BIOS вылезет ещё и приглашение командной строки. Может быть, его это устроит.

Но скорее всего, наш зловредный покупатель захочет, чтобы в машине функционировало всё: встроенный модем модулировал и демодулировал, WiFi~--- коннектился, кард-ридер~--- читал всякие разные карты памяти, в том числе и те, которые дают в качестве сдачи на кассах больших гастрономов. А это уже потребует некоторой ОС, поддерживающей соответствующие устройства как таковые, и имеющей драйвера под устройства конкретные. А если он ещё покупает не голимый ноут, а комплекс вместе с притером/сканером, а то ещё и, страшно сказать, МФУ, у него могут возникнуть гнусные и неприличные желания, чтобы принтер~--- печатал, а сканер~--- сканировал.

Спрашивается, каким образом продавец может удовлетворить столь наглые и беспочвенные притязания покупателя? Ответ очевиден: только посредством предустановки текущей на данный момент ОС, которая поддерживает набор текущего, опять-таки, <<железа>>. Или~--- честно предупредить покупателя, что ты, приятель, за полторы штуки уёв купил железяку, которая может включаться от сети, а всё остальное~--- твои трудности. Только вот много ли он продаст при таком подходе к делу?

Таким образом, у продавца есть единственный выход: продавать машину с той ОС, которую предустановил на неё производитель, которая была протестирована на совместимость со всеми комплектующими, в том числе и второстепенными, типа тех же WiFi'ев и card-rider'ов, о происхождении которых не знает даже советский ребёнок, собаку съевший за очищением политуры.

Предвижу возражение: Windows~XP поддерживает практически тот же набор оборудования, а, по мнению большинства пользователей, превосходит Vista по потребительским качествам (повторяю, это не моё мнение). Так почему бы не предустанавливать её?

Ответ готов: большинство серьёзных продавцов компьютерной (не бытовой) техники готовы предоставить эту услугу. При условии выполнения пользователем лицензионных требований. То есть приобретения ими у продавца <<коробочной>> версии указанной ОС или предоставлении собственного экземпляра таковой, заведомо не контрафактной. В противном случае речь будет идти о банальном воровстве, что, с точки зрения законов людских и божьих (не говоря уже о государственных) гораздо хуже нарушения закона о правах потребителя. Если же припомнить цену на <<коробочную>> XP~--- покупателю впору решить, что~40 баксов за предустановленную Vista~--- это всё равно, что <<дешевле только даром>>.

Вопрос ценообразования на <<вынь-ды>> различного рода мы тут обсуждать не будем. Равно как и соответствие их цен идеалам абстрактной справедливости. Замечу только, что в сложившемся положении вещей немалая вина и тех, кто ныне за эту самую абстрактную справедливость ратует.

А как же предустановленный Linux, спросите вы меня? Тот самый Linux, который, по уверениям его адептов, во-первых, бесплатен, во-вторых, поддерживает весь спектр более или менее современного оборудования. Отвечаю.

Да, Linux бесплатен почти во всех своих проявлениях. Да, ядро этой ОС поддерживает почти всё распространённое оборудование. Но\dots Чтобы выполнялось второе условие, ядро должно быть сконфигурировано должным образом. А этим нужно заниматься <<всерьёз и надолго>>, как говаривал дедушка Ленин. Ибо предустановленный <<в лоб>> Linux, без <<заточки>> под конкретное <<железо>>, может только вызвать у пользователя стойкое отвращение к этой операционной системе. Вроде как рыбий жир, спасающий от авитаминоза, которым пичкали в детсадах людей моего поколения, привил стойкое отвращение ко всему, хотя бы отдалённо напоминающему рыбу.

Впрочем, и история о том, как нужно (точнее, как не нужно) предустанавливать Linux~--- совершенно особая. Когда-нибудь я к ней вернусь.

Так что~--- увы и ах, но у продавца компьютерной техники есть только два пути: или продавать эту технику с ОС, предустановленной производителем, или без операционки вообще (или с вариациями на тему FreeDOS, что однофигственно). Можно ли его за это осуждать?

Можно. Если встать на позицию пользователя, который ходит за покупками с собственным LiveCD с дистрибутивом Linux, содержащим ядро и initrd, заточенные под всё, что только способно запускаться при включении питания. Вы знаете много таких покупателей? Я~--- так ни одного (в том числе даже и себя, хотя набор Live CD, благодаря наводкам Владимира Попова, у меня очень обширный).

Нельзя. Если исходить из потребностей того самого покупателя, за свободу выбора которого ратуют сторонники запрещения в законодательном порядке продажи машин с предустановленной ОС. И который в результате всякого рода законодательных инициатив лишится того, что есть, не приобретя ничего взамен. Я далёк от мысли, что кто-то из законодателей или инициаторов лично займётся подгонкой хоть какого дистрибутива под спектр продаваемого <<железа>>~--- не царское это дело, в ядре ковыряться\dots

Вообще о законодательных инциативах можно говорить очень долго. Я этого делать не буду. Скажу только одно: когда и если (или если и когда) какие-либо законодательные инициативы по этому поводу будут приняты~--- нам останется только вспомнить слова д'Артаньяна по поводу почившего Мазарини: <<\dotsмы еще пожалеем о Мазарини>>.

Так что же, Лёха, спросите вы меня, бороться против предустановленных ОС (не называя имён) не следует? Что я вам на это скажу? Следует. Но только одним способом: по возможности не покупать там, где эти самые предустановленные ОС впаривают насильно. А покупать лишь там, где вам готовы продать либо <<голую>> машину, либо заменить предустановленную ОС по возможностям и потребностям.

Опять же предвижу возражение: вы там в своей Москалии зажрались. Потому как, если вам не понравилось в данной фирме, то всегда можно перейти через дорогу и зайти в фирму соседнюю. А как быть нам в Тьмутараканске, на который на весь есть одна-единственная фирма, торгующая компьютерами? Которая что хотит, то и творит?

Увы~--- на это я ничего не могу возразить, такова объективная реальность: у жителей зажравшейся Москалии выбор больше. Можно лишь вспомнить слова Линуса Торвальдса о цистернах и водопроводе. Кто-нибудь когда-нибудь его проведёт.

Нескоро? Да, может быть. Но, как говорили классики,


\begin{shadequote}{}
\dotsдо конца света еще миллиард лет\dots Можно много, очень много успеть за миллиард лет, если не сдаваться и понимать, понимать и не сдаваться.
\end{shadequote}

Давайте попробуем?

\section{Без-Win’ные машины: продолжение} 

\begin{timeline}Март 28, 2009\end{timeline}

Высказав в прошлой заметке почти всё, что я думаю о компании против предустановленной Windows на ноутбуках в технологическом аспекте, я полагал эту тему закрытой (для себя лично, разумеется). Однако продолжающееся обсуждение заставиля меня возвратиться к ней.

Тем более, что тема эта получила продолжение: в газете \textbf{Коммерсантъ}  \textnumero53(4108) от 26.03.2009 была опубликована  статья под названием <<ФАС облегчит отказ от Windows>>. Следом подобные же сообщения прошли по многим онлайновым изданиям, однако ничего принципиально нового, по сравнению с ней, они не содержат.

Прежде чем перейти к сути вопроса, процитирую фрагмент из комментария Владимира Попова к моей предыдущей заметке:


\begin{shadequote}{}
Комплектация СВТ~--- как полигон для утверждения стандартов гражданского общества? А с остальным у нас всё уже о.к., ага?
\end{shadequote}

Так что нынче я ни слова не скажу о технологии, а только о пресловутом человеческом факторе. Вдохновляемый, опять же, цитатой из комментария Владимира Попова:


\begin{shadequote}{}
А как вам такое предположение: успешная “борьба” в этом направлении среди, прочих следствий, будет иметь: 1) неприязнь массового продавца/покупателя к FOSS и всему, что с этим как-то ассоциируется, и 2) укрепление позиций MS, как “единственного” гаранта того, что купленный современный ноут будет таки работать в соответствии с его спецификацией (что, вообще-то неудивительно, если производитель разрабатывает модель под определённую ОС).
\end{shadequote}



Пересказывать сообщения Коммерсанта и других СМИ я не буду. Напомню лишь, что в той же газете несколько ранее~--- № 25(4080) от 12.02.2009~--- прошло сообщение: <<ФАС протестирует Windows>>. Из которого следовало, что ФАС (Федеральная антимонопольная служба) начала проверку компаний Acer, Asus, HP, Samsung, Dell и Toshiba на предмет того, предусмотрена ли для их ноутов возможность отказа от предустановленной Windows и существует ли штатная процедура возврата денег за неё покупателю.

Приводятся и первые результаты этой проверки. Которые показали, что из всех проверяемых лишь ASUS смог представить документы, описывающие процедуру возврата денег (запомним на будущее~--- представить документы, а не примеры возврата), у всех остальных таковая не предусмотрена вовсе.

Обо всём прочем~--- дальнейших планах ФАС и отсутствии у них претензий непосредственно к фирме Microsoft, объяснениях представителей проверяемых фирм, всякого рода численных расчётах и так далее~--- заинтересованные лица могут прочитать сами (да, скорее всего, уже и прочитали). Остановимся только на карательных мерах, которые могут последовать в отношении возможных нарушителей, ибо именно они наиболее интересны для нас~--- конечных пользователей.



Но сначала зададим себе ещё один вопрос, которым не озаботилось ни одно из СМИ, обращавшихся к этой теме: почему именно эти шесть фирм стали предметом проверки? Ведь в прайсах наших крупных фирм, торгующих ноутбуками, можно найти и Lenovo, и Sony, и LG, и MSI, и Fujitsu-Siemens и BenQ. Не говоря уже об отечественных Roverbook'ах и неких ноутах от Эльдорадо.

Почему они не попали в список? Положим, отечественного производителя решили пока не трогать. Для остальных же~--- возможно, вследствие маленькой доли на рынке. Хотя в отношении Lenovo, Sony и Fujitsu-Siemens я сомневаюсь, что она столь уж мала. Или, по данным ФАС, у них заведомо всё в порядке с возвратом денег за OEM Windows? Тоже сомнительно. Например, Lenovo не так давно публично отказалась от предустановки Linux на свои машины~--- не с DOS же они их поставляют? А Sony, сколько я помню, всегда столь же публично заявляла о том, что их машины принципиально совместимы только с Windows Vista и ни с чем иным. И в любом случае, если речь идёт о борьбе за вселенскую справедливость в данном секторе Мироздания, проверять надо бы всех. Разве что для BenQ сделать исключение~--- именно для этой фирмы документально зафиксирован факт возврата денег за отвергнутую предустановленную Windows.

Сразу скажу, ответа на этот вопрос у меня нет.

Вернёмся, однако, к нашим баранам, точнее, к их шерсти. А именно~--- к тем карательным мерам, которые будут применены к возможным нарушителям, когда и если (или если и когда) их действия будут признаны нарушающими антимонопольное законодательство, по этому факту будут возбуждены соответствующие дела, эти дела завершатся нашей победой\dots ну, подумав, можно найти ещё несколько и <<когда>> и <<если>>. В отношении их последуют штрафные санкции в весьма крупных размерах. Плюс к тому им таки придётся возвращать деньги всем, пожелавшим обратиться с подобным требованием.

Чем ответят на это производители? Есть несколько вариантов ответа:

\begin{itemize}
	\item установление специальных дилерских цен для России, включающих в себя вероятные штрафные санкции и процент возврата (такое уже бывало в истории индустрии); 
	\item официальный ввоз в Россию только машин с предустановленными FreeDOS (а это исключительно бюджетные модели) и Linux (а спектр этих моделей крайне ограничен); 
	\item передачу всего дела на откуп <<серым дилерам>>~--- и такое тоже было на нашей памяти. 
\end{itemize}



Кто от этого выиграет? Не знаю. А вот круг проигравших очертить не сложно: это будут все покупатели ноутбуков. Ибо первый вариант повлечёт за собой рост цен, второй~--- сужение ассортимента, третий~--- отсутствие гарантии на аппаратную часть.

Кстати, при втором сценарии можно ожидать того, что машины с предустановленной Windows Vista будут доставать по блату~--- ибо только они могут оказаться, как сказано Владимиром в приведённой выше цитате а)  мощными, б) современными и при этом в) гарантированно работоспособными для простого пользователя, не имеющего сертификата от сибирских шаманов. Поневоле приходит на память хрестоматийная сцена из фильма <<Берегись автомобиля>>:


\begin{shadequote}{}
-- Мне нужен магнитофон заграничный, американский или немецкий -- Вот есть очень хороший, отечественный\dots -- Нет, спасибо, отечественный не подойдёт. -- Заграничный надо изыскивать\dots
\end{shadequote}

Кстати, пока суть да дело, непосредственные продавцы компьютерной техники на нашей Родине тоже дремать не будут: ведь по закону о правах потребителя конечный пользователь вправе подать в суд именно на них. Пользователь же, начитавшись статей о том, что <<предустановленная винда~--- это плохо>>, да к тому же, подчас, заменивший штатную Vista на пиратскую XP, в массовом порядке бросится отстаивать свои законные права. Суды же, состоящие из таких же пользователей, могут в столь же массовых масштабах начать выносить решения в пользу истцов. Куды же податься бедному барыге? Правильно, он будет тихо и спокойно возвращать деньги за предустановленную Windows. Предварительно прикинув возможный процент возврата, округлив в большую сторону и включив в цену, чтобы покрыть свои возможные потери сполна. А может быть, и с лихвой. И это всё мы уже не раз видели за двадцать лет существования российского капитализма.

И на него найдётся управа в виде тех же антимонопольных органов, скажете вы мне. Отнюдь.  Страховаться от возможных рисков~--- право продавца. Просто вряд ли найдётся мудрец, который включит в цену страховку от цунами в центре Москвы, обрушившегося непосредственно на его магазин. А от риска возврата денег за OEM Windows будут страховаться все~--- так что никакой монополии.

Если же ещё вспомнить, что всё это время будет развиваться бурная деятельность, собираться и заседать комиссии, привлекаться эксперты, и вся эта деятельность будет вестись на деньги налогоплательщиков~--- в проигрыше окажутся не только покупатели ноутов, но и десктопов, сборщики машин из копмлектующих, и даже те, кто прекрасно живёт без компьютера и не думает его себе покупать.

Разумеется, никакого апокалипсиса не произойдёт: всё это, в разное время и в разных сферах,  мы уже проходили и переживали~--- переживём и это. Вот только рядовой пользователь компьютера (и даже безкомпьютерный гражданин) будет думать о пользователях FOSS: это те самые\dots ээээ\dots нехорошие люди, которые заварили всю эту бучу. И поминать их добрым и ласковым Русским Словом. А с ними заодно~--- и пресловутю свободу выбора вместе со стандартами гражданского общества.

\section{Еще раз о без-win’ных машинах: кому и зачем они нужны?} 

\begin{timeline}Апрель 7, 2009\end{timeline}

Не утихающее обсуждение предустановленной Windows вынуждает меня ещё раз обратиться к этой теме. Кое-где мне придётся по ходу дела повторить сказанное в предыдущих заметках~--- в крамольных соображениях и их продолжении, однако в несколько другом аспекте.

Но для начала~--- несколько слов о том, зачем я всё это пишу. Ибо лично меня проблема предустановленной Windows не волнует ни с какого бока. Поскольку:

\begin{enumerate}
	\item во-первых, сам по себе факт наличия оной, вместе с наклейками, меня совершенно не трогает, и я не считаю машину этим фактом осквернённой; 
	\item во-вторых, стоимость предустановленной Windows я действительно полагаю стремящейся к нулю, поскольку она всегда может быть скомпенсирована подбором фирмы с более низкой ценой и (или) более выгодным для покупателя курсом внутреннего пересчёта уёв в буи (то есть единиц условных~--- в безусловные рубли); 
	\item в-третьих, если и когда (или когда и если) мне так уж захочется приобрести ноутбук без Windows~--- я знаю, где и как это сделать;
	\item наконец, в четвёртых~--- очень немаловажный для меня момент: всем моим ноутам уготована судьба рано или поздно быть проданными или подаренными при замене на новый; и вовсе не факт, что новые владельцы их будут пользователями Linux'а; тут-то и наступает психологический момент для восстановления <<родной>> системы с комплектного дистрибутива или имиджа; хотя сам я после покупки первым делом сношу предустановленную Windows к чертям собачьим, но все штатные диски храню вместе с руководствами.
\end{enumerate}


Собственно, к данной проблеме меня побуждают обратиться следующие моменты. В первую очередь, психологический. Что заставляет пользователей участвовать в акции, польза от которой лично для них будет сомнительна?

Второй момент~--- мне хочется поддержать тех здравомыслящих пользователей, которые, понимая не очень большой смысл данной акции, тем не менее, стесняются говорить об этом вслух: как же так, все за свободу, а я, как Баба-Яга, оказываюсь против?

И третий момент~--- уже технологический, поскольку я категорически не согласен с поставленной целью, которую преследуют организаторы акции и, так сказать, группа поддержки. Но об этом подробнее я скажу позже.

И так, давайте посмотрим, кому таки нужен отказ от предустановленной Windows?

Напрашивающийся ответ: пользователям более иных операционных систем, которые Windows не использовали, не используют и использовать никогда не будут~--- к их числу принадлежит и ваш покорный слуга. Казалось бы, уж они-то от этой акции должны выиграть?

Самое парадоксальное в этой истории, что напрашивающийся ответ~--- не верен. У меня достаточно широкий круг знакомств в этой сфере, как реальных, так и виртуальных. И подавляющее большинство из них к проблеме предустановки Windows более чем равнодушны~--- и примерно по тем же причинам, что и автор этих строк. Разумеется, есть исключения, обусловленные либо чисто религиозными соображениями (дабы скверна Windows не коснулась их машины), либо стремлением вернуть таки деньги за ненужный товар. Хотя у меня есть глубокое подозрение, что в последнем случае людьми движет чисто спортивный азарт~--- что же, это чувство я хорошо понимаю и уважаю. Только я бы свой азарт обратил всё-таки на что-нибудь более конструктивное.

Наконец, судя по многочисленным обсуждениям, есть пользователи, которые в такой ситуации действуют чисто из принципа~--- дабы мой полтинник не достался акуле капитализма Биллу Гейтсу. Что же, и это чувство понятно, но уж тут никаким конструктивом и не пахнет вообще.

Повторяю, все три рассмотренных случая~--- буквально единичны. Подавляющее же большинство лично знакомых мне пользователей Linux'а (или BSD~--- в данном случае это не важно) не просто индифферентно относятся к предустановленной Windows, но и крамольно не удаляют ей со свежеприобретённых ноутбуков, и даже временами используют в тех или иных целях. Чаще всего~--- для игр, но кое-кому эта ОС требуется и в производственных целях, по долгу службы.

Следующая категория пользователей, интересы которых призвана блюсти описываемая акция~--- это пользователи начинающие, возможно покупающие первый свой ноутбук в жизни. И которым просто жизненно необходимо предоставить свободу выбора~--- между предустановленной Windows, FreeDOS, Linux или машиной без всякой ОС вообще. Благородная цель, не правда ли? Однако пойдёт ли на пользу этой категории поупателей пресловутая свобода выбора?

Рискну предположить~--- нет: машина без ОС для начинающего пользователя не более чем кусок пастика, текстолита и прочих материалов. Современный ноут с предустановленной FreeDOS, эквивалентен машине без ОС вообще, да и не проблема это~--- без всяких акций найти подходящую модель с FreeDOS. Что же касается Linux'а~--- то тут с силу вступает технологический фактор, о котором речь пойдёт дальше. Так что фактически вся свобода выбора для начинающего пользователя сводится всё к той же предустановленной Windows.

Третья категория покупателей~--- опытные пользователи Windows, привыкшие к интерфейсу Windows~2000/XP, не желающие менять его на новомодные рюшечки и бантики Windows Vista. Насколько я знаю по рассказам, это примерно то же самое, что для KDE'шника со стажем сменить версию KDE~3.5.X на 4.X, так что и их я хорошо понимаю. Как и резоны избавиться от ненужной ОС, заменив её любимой и привычной.

Однако их положение и сейчас вовсе не безвыходно. Во-первых, как только что было отмечено, можно приобрести машину с предустановленной FreeDOS, после чего заменить её на любимую Windows~XP. Разумеется, лицензионную и коробочную~--- к случаю контрафактного софта мы вернёмся чуть позже.

Далее, существуют модели ноутбуков, несущие на винте предустановленную Vista, но комплектуемые Recovery DVD Windows~XP Professional (рус.)~--- специально для тех, кто с претензиями, и не желает пить лосьон <<Свежесть>> вместо коньяка в международном аэропорту <<Щереметьево>>.

Наконец, ряд фирм штатно предлагают такую услугу~--- замену предустановленной Vista на Windows~XP. Разумеется, опять же гарантированно лицензионную.

Получается, что акция по отмене предустановки Windows Vista не нужна ни записным линуксоидам, ни убеждённым пользователям Windows, ни даже информационно неполовозрелым начинающим пользователям, не знающим даже ключевых слов~--- ОС, предустановка, OEM и тому подобных. Кто же остаётся в сухом остатке как целевая аудитория акции?

Методом исключения ответить легко: это отпетые вендузяднеги (не путать с настоящими пользователями Windows, это четыре большие разницы), принципиально применяющими контрафактный софт. Единственной целью которых является получение компенсации за предустановленную ОС и инсталляцию пиратской Windows любого рода~--- вполне возможно, то же самой Vista. Да-да, господа организаторы акции~--- как бы вы от этого не открещивались, но объективно вы защищаете их интересы, и больше ничьи.

Внимательный читатель наверняка обратил внимание, что при перечислении объектов (или субъектов?~--- трактуйте, как угодно) акции я забыл категорию самую главную: её организаторов. Нет, не забыл. Более того, рискну утверждать, что это~--- единственная категория, за исключением пиратов, которой эта акция нужна по настоящему.

Зачем? Вариант личного пиара рассмотрим лишь теоретически. Хотя, после блистательного провала идеи Национальной Операционной Системы (НОС), почва для него представляется вполне благодарной. Действительно, если создание НОС~--- дело сложное, хлопотное, неблагодарное и не поддержанное никем, то идея возврата кровных денег за OEM Windows заведомо привлечёт к себе стронников~--- хотя бы тех, кто, как я говорил, готов приплатить, лишь бы Биллу Гейтсу не досталось. Так что популярность акции в соответствующих кругах была гарантирована.

По сему поводу не могу отказать себе в удовольствии процитировать меткое высказывание моего старого товарища Евгения Чайкина aka StraNNicK, правда, по более иному поводу:


\begin{shadequote}{}
В общем, прежде чем присоединиться к какой-либо акции~--- подумайте. В большинстве случаев, заявленных целей можно достичь сугубо своими силами, причём куда эффективнее.
\end{shadequote}

Однако я не настолько циничен, каким мне следовало бы быть. И хочу верить, что по крайней мере некоторыми из организаторов акции движут иные побуждения~--- например, всё тот же спортивный азарт. Как я говорил, чувство понятное и достойное уважения~--- но, ИМХО, направленное не в то русло.

Какое же русло то? Для этого вспомним, каков главная цель акции: добиться отчуждения аппаратных и программных средств друг от друга. И вот это тот самый момент, который побудил меня продолжить цикл настоящих заметок.

Потому что время отчуждения прошло, и мы по спирали вернулись к тому, с чего начиналась IT-индустрия~--- развитию в сторону единства программных и аппаратных средств .

Современный ноутбук~--- это действительно программно-аппаратный комплекс, пригодность к применению компонентов которого обеспечивается софтом не в меньшей степени, нежели <<железом>>. Да, ныне это гарантируется только одной-единственной (лучшей и величайшей) ОС. Но тут мы вступаем на технологическое поле, на котором происходят совсем другие истории.

\section{Без-win’ные машины: закон есть закон?} 

\begin{timeline}Апрель 8, 2009\end{timeline}

Следующую заметку без-win'ного цикла я хотел было посвятить технологическим аспектам предустановки ОС. Однако дискуссии по теме как-то резко свернули в юридическую сторону, так что появилась необходимость высказаться сначала по этому поводу.

В прошлой заметке я не упомянул ещё одну категорию людей, заинтересованных в акции ЦеСТ: абстрактных правдоискателей, искренне полагающих, что законы должны выполняться. Чисто теоретически я с этим согласен. Но вот практически\dots

Для начала, не очень ясно, какой именно закон нарушает предустановка Windows на ноутбуки. Закон о защите прав потребителя? Казалось бы~--- безусловно. Но позволю себе процитировать представителя Общества защиты прав потребителей Дмитрия Лесняка из комментария к статье <<ФАС проверяет Acer, Asus, HP, Samsung, Dell и Toshiba>>:


\begin{shadequote}{}
Многое в данной ситуации зависит от того, как составлены товаросопроводительные документы. Так, если операционная система указана в качестве принадлежности устройства (условно говоря, как комплектующее изделие), комплектация сформирована изготовителем и отдельная её стоимость не указана~--- потребитель не вправе требовать продать ему компьютер без программного обеспечения. В то же время, если программное обеспечение (в терминологии ГК~--- лицензионное право) продаётся как отдельный товар с собственной ценой, но продавец навязывает его приобретение вместе с компьютером (кстати, возможна и обратная ситуация~--- навязывание покупки оборудования с софтом), такие действия, безусловно, противоречат Закону <<О защите прав потребителей>>.
\end{shadequote}

Похоже, что большинство участников вышеупомянутого обсуждения в юридической стороне дела не очень сильны. Признаюсь, я тоже, так как изрядную часть жизни прожил там, где закон~--- тайга, медведь~--- судья. Вследствие чего стал, по выражению одного из моих знакомых юристов ярко выраженным правовым нигилистом. И потому обращусь в авторитетному мнению Елены Тяпкиной~--- профессионального (и практикующего) юриста, не чуждого миру FOSS; помимо всего прочего, именно ей принадлежит адекватный перевод GPL v.2 на русский язык.

Так вот, в далёком для многих 2002 году на одном из заседаний Семинара Altlinux Елена делала доклад под названием <<Сравнительный анализ основных лицензий Open Source>>. Сам по себе он был очень интересным, однако в основном лежит за рамками нашей сегодняшней темы. А вот в процессе обсуждения была поднята тема о правовом статусе лицензий не свободных.

Так вот, Елена трактовала лицензии на программное обеспечение, как свободное, так и проприетарное, в рамках юридических понятий российского гражданского права как договор присоединения. Подробнее об этом говорится в другой её работе~--- <<Правовой статус GPL в России>>. Начну с цитаты из неё, касающейся GPL:


\begin{shadequote}{}
В соответствии с нормами Гражданского кодекса РФ, GPL~--- это договор присоединения, условия которого определены одной из сторон в стандартной форме и могут быть приняты другой стороной не иначе, как путем присоединения к предложенному договору в целом.
\end{shadequote}

Далее, Елена отмечает явное сходство её с пресловутой EULA, которая\dots


\begin{shadequote}{}
\dotsвступает в силу и становится обязательным для пользователя, если он вскрыл упаковку программного продукта или установил программное обеспечение на свой компьютер.
\end{shadequote}

То есть пользователя никто не обязывает присоединяться к договору, предложенному производителем конкретной софтины. Но уж если он совершил


\begin{shadequote}{}
<<конклюдентные действия>>, то есть действия, выражающие волю лица заключить сделку, но не в форме устного или письменного волеизъявления, а поведением, по которому можно сделать заключение о таком намерении\dots
\end{shadequote}

то тем самым он принимает условия производителя, каковы бы они ни были.

Применительно к нашему случаю это можно интерпретировать так: производитель ноутбуков, очевидно, заключает договор присоединения к лицензии Microsoft в силу самого факта предустановки Windows на свои машины, не так ли?

Далее, из высказывания Дмитрия Лесняка очевидно, что производителю вольно объявить предустановленную Windows неотъемлемой частью своего продукта. Хотя бы потому, что все компоненты ноутбука тестировались на совместимость только с этой операционной системой~--- ну не хотелось ему возиться с настройкой для них Linux'а. И, замечу в скобках, правильно не хотелось: все, кто видел Linux'ы, предустановленные на ноутбуки, вспоминают о них не иначе как о ночном кошмаре.

А далее производитель ноутбуков уже со своей стороны предлагает покупателю договор присоединения. Пусть не напрямую, а через своих дилеров, дистрибьюторов и прочих партнёров, а те, в свою очередь, через розничных продавцов. Причём заметим, что насильно товар ему не всучивается. Но сам по себе факт покупки машины, в спецификации которой русским (английским, зулусским или маорийским~--- нужное дописать) языком сказано, что она несёт на себе предустановленную Windows, можно рассматривать как то самое конклюдентное действие. В противном случае покупателю вольно отказаться от покупки машины данного производителя или данного продавца, и обратиться к тому продавцу или производителю, который предалагает ему иной договор присоединения~--- например, на условиях GPL в случае предустановки FreeDOS или Linux. Но без договора присоединения покупатель не обходится и в этом случае.

По моему скромному мнению (и это мнение не только моё), ни производитель, ни все промежуточные звенья в цепи распространения не делают ничего, что противоречило бы законам людским и божьим, равно как и здравому смыслу. А вот почему в качестве предустановленной ОС на большей части ноутбуков предустановленной оказывается именно MS Windows, и лишь на подавляющем меньшинстве~--- FreeDOS или Linux, вопрос совершенно иной.

Ситуация прекрасно описывается в терминах преферанса: взявший прикуп объявляет игру, партнёрам вольно пасовать или вистовать. Но уж если кто-либо из партнёров вистует~--- он обязан взять положенное количество взяток, иначе получит в гору. Причём правила при торговле ноутбуками гуманней по отношению к покупателю: никакого <<обязона>> (иначе говоря, <<сталинграда>>), так как даже покупатель супербюджетного ноута (каковой можно приравнять к <<шести пикам>>) имеет возможность выбрать между минимум тремя предустановленными операционками (иначе говоря, <<уйти за свои>>), или даже отказаться от вистов (сиречь покупки) вообще.

Из сказанного, ИМХО, ясно, что пытаться изменить существующее положение вещей законодательными или полицейскими мерами бесполезно и бессмысленно. Изменить его можно только комплексом мер технологических и информационно-просветительских.

P.S.В заключение авансом отвечу на возражение, неоднократно звучавшее в ряде обсуждений: если производитель рассматривает ОС (любую, но в данном случае Windows) как часть программно-аппаратного комплекса, то пусть будет любезен предоставлять гарантию не только на <<железо>>, но и на предустановленный софт. В пример чему с удивительным постоянством приводится Apple, который якобы именно так и поступает в отношении своих Mac'ов.

Для начала развею последнее заблуждение. Я специально проконсультировался с одним из немногочисленных лично мне знакомых пользователей Mac'а~--- Леонидом Уточкиным aka Lentux (который по совместительству является старым линуксоидом). В своём письме он процитировал Гарантийный сертификат, который прилагался кего машине. Цитирую вслед за ним:

\begin{shadequote}{}
Программное обеспечение, поставляемое вместе с продукцией Apple Computer, Inc., не подлежит гарантийному обслуживанию.

Apple Computer, Inc. и сервисные центры не несут ответственности за сохранность Вашей информации и программного обеспечения. Переустановка и восстановление программного обеспечения не входит в список работ, предусмотренных гарантийным обслуживанием.
\end{shadequote}


Что, исходя из здравого смысла, опять-таки не может вызвать возражений: ведь подавляющее большинство <<повреждений>> предустановленного софта связаны либо с заражением вирусами, либо с шаловливым рукоблудием пользователя. А уж брать на себя обязательства в том, что у последнего руки растут непременно оттуда, откуда надо, не станет ни один здравомыслящий человек\dots

\section{Так Pro или Contra?} 

\begin{timeline}Август 28, 2009\end{timeline}

Как ни надоела мне тема предустановленной Windows и сопуствующие ей материи~--- а время от времени приходится к ним возвращаться. Правда, в данный момент~--- с позиций консенсуса, а не конфронтации, что не может не радовать.

Эта заметка сочинилась под впечатлением материала Сергея Голубева <<Pro et Contra>> в его блоге. Собственно, она представляет собой нечто вроде развития итогового её вывода, каковой позволю себе процитировать:

\begin{shadequote}{}
\dotsстранная у людей маркетинговая стратегия. Товар, вроде, есть. Но ни описаний, ни обзоров, ни тестов\dots Кот в мешке, однако. Традиционные изделия так продать можно, а вот пробиваются на рынок обычно более агрессивно.Ну допустим~--- запретят Микрософту ставить OEM. Неужто кто-то думает, что от этого будут лучше продаваться <<патриоты>> и <<инфрабуки>>.
\end{shadequote}

Что напомнило мне старый анекдот. Год 1917. Дама просыпается от шума и спрашивает свою горничную:

\begin{shadequote}{}
-- Что случилось? \\
-- Революция, народ вышел на улицы. \\
-- И чего они добиваются? \\
-- Чтобы не было богатых. \\
-- Странные люди. Мой прадед вышел на Сенатскую площадь для того, чтобы не стало бедных\dots
\end{shadequote}

Что ж, таково поведение большинства революционеров~--- и OEM-революция тут не исключение. Я бы сформулировал это примерно так:

Можно тратить время и энергию на \textbf{убеждение} в том, что машины с предустановленной Windows~--- это плохо. А можно~--- на \textbf{объяснение} того, что машины с Linux'ом~--- это хорошо.

Второй способ не просто конструктивней~--- он дешевле. Так как требует только личных усилий. А всякая революция начинается с создания инфраструктуры. И рано или поздно обрастает генсеками, членами политбюро, активистами на местах. Я уж не говорю о бухгалтерии для обеспечения жизнедеятельности всех перечисленных. По сравнению с этой Батыевой ратью Гейтс с Балмером покажутся скромнягами-бессребрениками.

\section{FOSS на Руси: революционная ситуация?} 

\begin{timeline}Апрель 11, 2009\end{timeline}

В приснопамятном обсуждении OEM'ной Вынь-ды от одного из организаторов акции ЦеСТа некогда прозвучал призыв~--- говорить на чистоту. Я долго воздерживался от этого предложения, дабы остаться в рамках пресловутой политкорректности~--- что делать, заразителен дурной пример некоего государства, в котором лопату предпочитают не называть лопатой, а для простого русского слова, которое дети любят писать на заборах, могут придумать такое количество эвфемизмов, перед которым блекнет лексикон классической арабской литературы и <<Тысячи и одной ночи>>.

Однако тон обсуждения адептов акции становится всё более агрессивным, что вынуждает меня таки поговорить на чистоту~--- заодно расставив все точки над i в моём к ней отношении. И, как показывают отклики, не только моём.

Интересно, что таковой тон принимают даже организаторы дискуссии~--- причём местами на олпанской мове, не являющейся родной ни для одного из участнегов диалога.

Но сначала~--- немного истории. Каковая, как известно, в силу категорического неусвоения её уроков, имеет обыкновение повторяться: сначала~--- в виде трагедии, затем~--- в виде фарса (возможно, также вполне трагического), а потом уже~--- в форме откровенной травестии. Что же, нам остаётся ответить на это бурлеском.

Революции в истории человечества происходили неоднократно. Классики вечно живого всепобеждающего (потому что верного) учения утверждали, что они происходят тогда, когда создаётся революционная ситуация: верхи не могут, низы не хотят. Но, как метко отметил творческий гений советского народа (подозреваю, что и анти-советского тоже), это ситуация импотента на фригидной женщине: одна сторона не могёт, другой всё пофигу. И потому никакой революцией данная <<революционная ситуация>> не чревата. А чревата она вялотекущим процессом, также описанным коллективным творческим гением в диалоге босса и секретарши:

\begin{shadequote}{}
-- Иван Петрович, Вы уже ввели? \\
-- Кажется, да\dots \\
-- Ой как хорошо, ой как приятно!
\end{shadequote}

Читатель вправе спросить меня~--- опровергая общепризнанных классиков марксизма (величие которых признают даже признанные анти-марксисты), что же ты предложишь взамен? Отвечаю: это проще показать на примере, нежели сформулировать в научных терминах. Может быть, социологи грядущих столетий, если таковые будут отпущены человечеству, это сделают~--- уже без гнева и пристрастия; впрочем, беспристрастие маловероятно и по прошествии тысячелетий\dots

Итак, пример: 1917-й год, канун Октябрьского переворота, который потом назовут ВОСРом. Что мы наблюдаем в контексте революционной ситуации? А мы наблюдаем примерно следующее:

\begin{itemize}
	\item кучку будущих комиссаров в кожанках с маузерами в деревянных кобурах-прикладах; правда, повторяю, всё это в будущем~--- в кожанки с маузерами они обрядятся, когда захватят армейские склады, подготовленные для летнего наступления 17-го года; 
	\item братишек-матросов, просидевших всю войну на кораблях на приколе, мающихся от безделия, но, за отсутствием конструктива, не способных ни к чему, кроме деструктива откровенного; 
	\item серой вооружённой массы, которой на самом деле всё до лапочки Ильича (каковая ещё только вызревает в светлых мозгах комиссаров. 
\end{itemize}



Комиссарам очень хочется власти~--- собственно, ради того они в комиссары и подались. Братишкам флотским~--- хочется развернуть руку, да раззудить плечо, цель рояля не играет. Ну а серой вооружённой массе хочется одного. Того самого, что выразил Юлий Ким aka Михайлов:

\begin{shadequote}{}
Наплявать, наплявать\\ 
Надоело воевать\dots \\
Были мы солдаты, А теперь до хаты-ы-ы-ы\dots
\end{shadequote}

Что дальше? Дальше~--- Братишки-матросы рвут на груди тельники, подымая в атаки те самые серые массы. Ну а комиссары в кожанках и с маузерами сидят в глубоком тылу (по возможности под охраной латышских стрелков или китайских <<интернационалистов>>). Употребляя свои маузеры разве что для постреливания в затылок тем из серой массы, кто почему-то не испытывает большого желания воевать за светлое будущее всего человечества.

Чем закончилось дело~--- общеизвестно: водворением социализма на одной шестой части суши. Так что об этом не будем. А обратимся к более близкому примеру, который памятен некоторым из здесь присутствующих~--- смены парадигмы: от построения светлого коммунистического будущего в масштабах всего человечества к слиянию со всем прогрессивным капиталистическим человечеством. Что наблюдалось нами уже воочию~--- на рубеже 80-х и 90-х годов последнего века прошлого тысячелетия.

Промежуточные стадии~--- процесс превращения комиссаров в парт- и госчиновников, сменивших кожанки на номенклатурные костюмы, братишек~--- в павших смертью храбрых национальных героев, а также сезоны отстрела возомнивших о себе тех и других~--- опустим, как не имеющие отношения к теме. И обратимся ко второй революционной ситуации минувшего века~--- рубежу 80-х и 90-х годов. Когда комиссарам захотелось покоя~--- а именно, смены номенклатурных <<Чаек>> и <<Волг>>, которые могли в любой момент отобрать вместе с должностью (а то и головой, как порою случалось), на частнособственные <<Мерседесы>> и <<БМВ>>, номенклатурных окладов~--- на личные счета в зарубежных банках, номенклатурных дач~--- на собственные коттеджи в Подмосковье или, ещё лучше, на виллы в Дальнем Забугорье.

И потому из парторгов ставшие банкирами и коммерсантами.

Нашлись и братишки~--- получившие имя братков, а затем и брателл, они тоже изменили форму одежды: вместо клешей и тельников одели <<Адидасы>> в фирменном или китайском исполнении.

Их теперь называли бандитами~--- и вовсе не всегда в ругательном смысле.

Не было серой массы: с одной стороны, массы эти были безоружны, с другой, выработали не сформулированное официально, но самое передовое в мире учение пох\dots фигизма. Потому революционная ситуация как бы рассосалась сама собой. Не совсем без жертв, конечно, но всё-таки относительно мирно. Иными словами, ситуация фарса, хотя иногда и вполне трагического.

Это я всё к чему? Ныне в российском секторе общемирового FOSS-сообщества наблюдаем уже совсем комичное воплощение всего того, что предшествовавшие поколения и поколение наше пережило в течении двадцатого века.

До недавнего времени FOSS-сообщество России жило и развивалось само по себе~--- подобно чистой фундаментальной науке, не до конца про-официозной литературе и прочим сферам культуры и искусства, равно как и их поп-ответвлениям. Развивалось вполне успешно и целенаправленно~--- не смотря на отсутствие внимания к нему со стороны власти предержазих. Или, может быть, как раз благодаря отсуствию такового.

И вдруг в одночасье случилось~--- государство задумалось о том, что великая держава не имеет ничего своего, кроме нефти, газа и ещё кое-каких природных ресурсов. То есть: работники госаппарата, отказавшись от своих номенклатурных машин от отечественного производителя, пересели на иномарки, исчисляют свои доходы не в родимых рублях и копейках, а в каких-то импортных баксах (подозрительно похожих на доллары недавнего идеологического противника), пользуются ноутбуками, изготовленными\dots да кто его знает, где. И в довершение ко всему, на ноутбуках этих предустановлена операционная система вполне определённого происхождения.

Разумеется, менять иномарки на продукицю АЗЛК или Тольятти бывшим товарищам, ныне господам, не хотелось, отказываться от ноутов Sony~--- тоже, да и менять предустановленную Windows на непонятный Linux~--- тоже. И тогда в действие вступает оппозиция~--- те из психологических парторгов, которые не попали в первую обойму и потому всеми вышеперечисленными аксессуарами не обзавелись.

Началось всё, естественно, с самого слабого звена~--- с операционных систем. Благо сложившаяся ситуация~--- доминирование в этой сфере продукции самой великой софтверной компании~--- давала к тому массу поводов.

И раздаются призывы, с одной стороны, к внердрению существующего свободного программного обеспечения в сферах, финансируемых государством, с другой~--- к созданию Национальной Операционной Системы (сокращённо НОС~--- копирайт на эту аббревиатуру оставляю за собой).

Правда, быстро выяснилось, что дело это~--- сложное, хлопотное, вполне бессмысленное (ибо создать что-то принципиально более иное, ввиду развала настоящей фундаментальной науки, в том числе и той, что называется Computer Science, было не реально) и неблагодарное: вследствие перечисленных причин немедленных политических дивидендов оно не сулило.

И потому принялись за акцию, обещающую таковые: борьбу за свободу выбора ОС для конечного пользователя.

Разумеется, занялась всем этим кучка комиссаров, не вполне удачливых на своём комиссарском поприще. Далее~--- всё по многократно апробированному сценарию: вербовка по многочисленным форумам группы братишек, по преимуществу своему~--- тех, кто, как говорилось в предыдущей заметке, только недавно слез с подоконника и потому словом и делом готов доказывать свою верность новому учению. Логическое завершение чего~--- та самая пресловутая акция. Но это лишь одна сторона вопроса.

Для освещения другой стороны придётся таки вернуться к вопросу о засилье заморской софтверной корпорации во всех сферах, подведоственных государственному бюджету. И, соответственно, декларации соответствующими структурами поддержки свободного софта. И тут в секторе отечественного FOSS-сообщества впервые ощутимо запахло баблом. Не очень большим~--- но на запах которого потянулся отечественный бузинес. В том числе и тот, который принято называть большим.

Вообще удивительно, сколько за последнее время обнаружилось радетелей свободного софта. Причём ещё с древних времён. И как тут не вспомнить персонажа одного из рассказов Георгия Фёдорова~--- деда, который после прослушки оперы <<Иван Сусанин>> задумчиво сказал:


\begin{shadequote}{}
Теперь много таких находят, которые ещё встарь за советскую власть стояли.
\end{shadequote}

Радетелй свободного софта, стоявших до него со времён Очакова и покорения Крыма (а то и Куликовской битвы), можно отыскать ничуть не меньше. Кроме братишек-матросиков с Линуксфорума, в это число можно записать и отмеченных выше крупных бизнесмеев. Не так давно был сильно умилён фразой из интервью с Дмитрием Комиссаровым. Она настолько показательна, что процитирую:


\begin{shadequote}{}
Мы довольно давно идем в сторону СПО. В частности, еще в конце 90-х была попытка портировать купленный у компании <<Микроинформ>> Лексикон под Linux и открыть коды этой программы. К сожалению, после кризиса 98-го года отрасль впала в некое депрессивное состояние и проект был закрыт.
\end{shadequote}

Очень улыбает: так что же было~--- попытка портировать или попытка открыть? Да и Lexicon на фоне Linux'а образца 1998 года выглядит достаточно забавно.

Кстати, что интересно~--- на те же около-дефолтные годы падает активность IPLabs Linux Team (ныне Altlinux); им, почему-то, депрессивное состояние не помешало. При всём моём уважении к деятельности Урбансофта и его <<Открытому ядру>>~--- именно в это время и именно IPLabs Linux Team заложил предпосылки для массового применения Linux'а. Среди тех масс, которым он по настоящему нужен, разумеется\dots

Ну да бог с ними, с крупными бизнесмеями~--- работа у них такая, извлекать чистоган из всего. В том числе и из открытого софта. Тем более, что от их бизнесмейской деятельности вроде как даже польза бывает~--- налогами, толика которых косвенно обламывается и для Open Source: в виде грантов академическим учреждениям, проектов по Linux'изации школ и ВУЗов, и так далее. Конечно, забавно, когда телегу ставят впереди лошади, и проект портирования Lexicon'а полагают соизмеримым со свободными Vim или Joe, юзаемым поколениями юниксоидов\dots Ну да, повторяю, господь им судья.

Вернёмся к нашим политикам от Open Source. Тем более, что эти господа являются не столько налогоплательщиками, сколько налогопользователями. Причём не только политики действующие~--- но и, так сказать, теневые. Потому как эти самые теневые сочиняют, дабы эмулировать свою политическую активность, всякого рода запросы и обращения. На которые действующие политики, для поддержания своего статуса, время от времени должны реагировать~--- примерно так, как мы видим в настоящий момент с проверкой ФАС.

Политиков тоже можно понять~--- это тоже издержки профессии, радеть за счастье всего человечества. В первую очередь, конечно, за счастье лучших его представителей~--- себя, любимых. Но это реализовать в полной мере могут только политики действующие. А вот теневым политикам нужно отрабатывать всеобщее счастье по полной программе. Желательно, разумеется, с наименьшими накладными расходами~--- вроде возврата денег за предустановленную Windows\dots

Тоже, в общем-то, не смертельно: как говорится, чем бы дитя ни тешилось, лишь бы с ножом на дорогу не выходило. Или не звало выходить с гранатомётом на баррикады. А только отметиться в опросе\dots

Но только вот не надо бы представлять всё ту же самую акцию как нечто общеполезное~--- раз, нечто способствующее процветанию свободного софта~--- два, и вообще нечто, способствующее чему-либо, кроме личной популярности организаторов~--- три. Каковые и хотят произвести маленькую, локальную революцию в мире свободного софта в одной отдельно взятой стране.

Сапиенсам~--- надеюсь, sat?