\chapter{Разное}

\textsl{В этой рубрике собраны заметки разных лет, тематическую принадлежность которых я определить затрудняюсь.}

\section{Апокалипсис для СПО?} 
\begin{timeline}
Июнь 15, 2009
\end{timeline}

Настоящая заметка посвящена прискорбному событию, а потому я начну её цитатой, оное описывающей:

\begin{shadequote}{}
Во все российские школы исполнителями госконтракта о поставке СБППО (<<Первая ПОмощь>>) рассылаются диски ПСПО, по крайней мере два из которых, после добавления на них материалов без ведома <<Армады>> и <<Альт Линукс>>, непригодны к использованию.
\end{shadequote}

Источник цитаты~--- сообщение в блоге Алексея Новодворского <<О рассылке ПСПО внутри коробки ``Первой Помощи''>>, по ряду причин вызвавшее большой резонанс на всех ресурсах, связанных с FOSS и Linux. В частности, и обсуждение на форумах, которые и заставили меня написать настоящую заметку.

Сразу скажу, что она

\begin{itemize}
	\item резюмирует всё сказанное мною ранее в отдельных форумных трейдах; 
	\item выражает моё личное мнение, и потому не является приглашением к \href{http://linuxforum.ru/index.php?showtopic=94518}{дискуссии};
	\item не касается ни собственно проекта внедрения СПО в школах (далее~--- Школьного проекта) как такового, ни способов его реализации. 
\end{itemize}


А касается она преимущественно двух вопросов:

\begin{enumerate}
	\item методов обсуждения вопроса, послужившего его первопричиной, и 
	\item возможных последствий. 
\end{enumerate}

Начнём с методов, для чего придётся рассмотреть историю вопроса~--- благо, на сегодняшний момент она укладывается в несколько дней.

Итак, всё началось с указанного сообщения в блоге Алексея Новодворского, к коему и отсылаю тех, кто ещё <<не в теме>>. Для остальных сформулирую свое отношение к событию.

Да, поставка дефективных дисков~--- это плохо. Причём кем бы то ни было, кому бы то ни было и любых по содержанию, хоть порнушных (если они были заказаны через соответствующие службы). Не менее плохо, чем продажа тухлой колбасы или плохой водки. Но и не более плохо.

Поэтому возмущение Алексея и некоторая эмоциональность в его описании события более чем понятны. Тем не менее, итоговый вывод сообщения вполне сдержан:


\begin{shadequote}{}
Жадность и непрофессионализм.
\end{shadequote}

Что же, за годы пост-советской действительности к этому нам не привыкать.

Однако тут в действие вступает тяжелая артиллерия: не прошло и полусуток, как в блоге Виктора Алксниса появляется (2009-06-12 09:53:00) сообщение: <<Вести с полей: два непригодных к использованию диска за 17 миллионов рублей>>. Где, после изложения канвы событий в несколько расширенном виде к определению Алексея добавляется,


\begin{shadequote}{}
\dotsчто это типичное проявление КОРРУПЦИИ, царящей в коридорах власти.
\end{shadequote}

А завершается всё обращением:


\begin{shadequote}{}
Уважаемые друзья! Помогите вывести этот пост и пост Алексея Новодворского в ТОП, дайте у себя в блоге ссылку на них! Кроме того, по возможности, прошу оставить комментарий о происходящем в блоге у Д.~Медведева\dots
\end{shadequote}

Что же, призыв был услышан мгновенно. И реализован постом с устрашающим названием: <<Внедрение СПО в школах под угрозой>>. Где уже рисуется грядущая апокалиптическая картина:


\begin{shadequote}{}
Поймите, если школьный проект провалится (к чему идет дело), то будут провалены и планы перехода нашей страны на СПО.
\end{shadequote}

Ну и, разумеется прозвучавший ранее призыв Виктора Алксниса был подхвачен (а также расширен и угл\'{у}блен:


\begin{shadequote}{}
Вне зависимости от наших разногласий по тем или иным вопросам, сегодня самое главное спасти школьный проект. Спасем его, значит Linux в России победит!
\end{shadequote}

Который почти немедленно был <<ретранслирован>> (определение топикстартера) на более ином форуме: <<Внедрение СПО в школах под угрозой>>.

И реакция определённой части читателей была вполне предсказуема. И лучше всего может быть выражена сакраментальным


\begin{shadequote}{}
Пионеры наших бьют!
\end{shadequote}
из <<Республики Шкид>>. Что и было целью этой, по выражению \textsl{vinny}, тестовой акции.

Хорошо, спросите вы меня устами одного из участников обсуждения, sash-kan'а


\begin{shadequote}{}
как с твоей точки зрения должен вести себя в этой ситуации тот самый all? тихонько похихать в уголке?
\end{shadequote}

Если я сам же признаю, что ситуация с рассылкой дисков, мягко говоря, нехорошая? Отвечаю: для здравомыслящего человека есть два варианта поведения.

Вариант первый: глубоко вникнуть в тему~--- а она имеет обширную ретроспективу; причём вникнуть с разных сторон~--- ибо противостоянием героев и злодеев эта история не ограничивается. И уже после этого принять взвешенное решение и поступать соответственно своим убеждениям и темпераменту.

Вариант второй: если на изучение темы нет времени, сил или возможностей, просто спокойно заниматься своим делом. В каковом каждый из нас, смею надеяться, разбирается лучше, чем в организации сетевых митингов и тому подобных акций протеста. Особенно если точно не знаешь, против чего именно следует протестовать.

Впрочем, результатом первого варианта будет скорее всего переход ко второму. Потому что ничего из ряда вон выходящего описанная история, не смотря на всю свою плохость, не содержит. Ибо в ретроспективе своей имеет и иные не вполне хорошие случаи. Например, беззастенчивое <<использование>> материалов Тихона Тарнавского.

Почему же именно дело о запорченных дисках получило такой резонанс? И тут мы переходим ко второму аспекту настоящей заметки~--- о возможных последствиях.

Да вследствие того, что делу этому придали апокалиптический оттенок. Вспомним цитату, где провал Школьного проекта приравнивается чуть ли не к гибели FOSS в масштабе страны. Так ли это? Для ответа на этот вопрос опять придётся обратиться к истории, теперь уже более давней, охватывающей уже более пятнадцати лет.

А история эта свидетельствует, что на протяжении более чем пятнадцати лет FOSS и Linux на Руси (в это понятие я включаю также Украину и Белоруссию) без всяких Школьных и иных проектов и внедрений Linux'а куда бы то ни было. UNIX был, FOSS был, Linux был, FreeBSD крутилась на серверах бессчётного числа Интернет-провайдеров, а Школьного проекта не было. Из этого рискну предположить, то если его снова не станет~--- всё вернётся на круги своя, только и всего.

За счёт чего или кого? Да всё того же сообщества. Того самого пресловутого, аморфного, неорганизованного. На которое со всех сторон сыпятся жалобы, что оно ничего никуда не внедряет и никому ничем не помогает.

Но ведь это не его, сообщества, собачье дело~--- чего-то куда-то внедрять или кому-то в чём-то помогать. Сообщество~--- оно или есть, или его нет. И то, и другое~--- констатация медицинского факта.

В нашей стране и некоторых сопредельных сообщество, хвала Аллаху, есть. Благодаря усилиям многих поколений его членов. В числе которых, причём первых, напомню, был и Алексей Новодворский. Так неужели оно исчезнет в случае провала Школьного проекта?

А пострадают ли от его провала ученики, ради грядущего счастья которых всё затевается? Позволю себе высказать очередное крамольное соображение: нет, не пострадают. В обоснование чего процитирую самого себя:


\begin{shadequote}{}
И потому каждый приверженец идеи Open Sources должен быть благодарным нашему Минобразу за то, что в качестве базовой платформы для обучения школьников основам компьютерной грамотности он не принял Linux или FreeBSD. Не согласны? А припомните-ка, какой срок карантина по окончании школы вам потребовался, чтобы прочитать, наконец, ``Капитанскую дочку'' или ``Мцыри''?
\end{shadequote}

Это~--- из заметки <<Пора ли кричать караул?>>, написанной в 2001 году по поводу начала Windows'изации российской школы. Много воды утекло с тех пор, но главный её вывод я готов повторить: нерадивые учителя могут привить ученикам отвращение к Linux'у столь же эффективно, как их предшественникам в наше время удавалось привить нам отвращение к Пушкину или Гоголю.

Так что в повсеместном внедрении Linux'а в школы, да ещё в приказном порядке, я не вижу ничего хорошего. Там же, где ученикам это интересно, а учителя радивые, FOSS и Linux будет внедряться, вне зависимости от всех проектов. В обоснование позволю себе процитировать один из постов vinny всё из того же обсужения на Linuxforum'е:


\begin{shadequote}{}
Год назад на <<Инфокоме>> в рамках конкурса <<Цифровой маршрут>> я протестировала на знание Gimp около 1000 школьников. Задание было простое: найти в Линуксе графический пакет и что-нибудь в нем нарисовать. На компах стояла Mandriva и Runtu. Половина детей справлялась за пару минут, потому что <<У нас в школе везде стоит Линукс>>\dots ситуация не так плоха, как её малюют на ЛОРе, исходя из общения с одним, двумя школьниками. Процесс перехода идёт, в не зависимости от <<общей линии партийного руководства>>, взяток, откатов, головотяпства с дисками и прочих проблем.
\end{shadequote}

И опять же можно спросить~--- чьими силами движим этот процесс? Да силами учителей и учеников, составляющих сообщество. И потому он будет идти, не смотря ни на что. Потому что это нужно людям. А наше дело, дело тех, кто относит себя к сообществу~--- обеспечить, чтобы те, кому это нужно, узнали о том, что им нужно. А не митинговать в Сети и наяву. И не участвовать в кампаниях, кем-то и зачем-то затеянных.

\section{Тётя Ася приехала\dots} 
\textsl{В соавторстве с Алисой Деевой, при участии творческого гения всего прогрессивного человечества}

\hfill \begin{minipage}[h]{0.45\textwidth}
Сарынь на кичку!\\
Ядреный лапоть\\
Пошел шататься \\
По берегам.  \\
Сарынь на кичку!\\
В Казань!  В Саратов!
\begin{flushright}
\textit{Василий Каменски}
\end{flushright}
\bigskip\end{minipage}
  
\subsection{Преамбула}

Настоящий материал представляет собой контаминацию отдельных заметок, размещавшихся на нашем Блогосайте, с учётом комментариев к ним и обсуждения на форумах. Авторы выражают признательность всем комментаторам и участникам. 

Исходные заметки сочинялись на злобу дня~--- в связи с очередным бесчинством тёти Аси, меняющей протоколы, 


\begin{shadequote}{}
\dotsкак, терья-терьям, перчатки
\end{shadequote}

И по горячим следам событий. Ныне они все собраны, причёсаны и уложены в одну сюжетную линию. 

Настоящий материал адресован не только линуксоидам и прочим POSIX'ивистам, но и пользователям более иной операционной системы. Даже, пожалуй, скорее последним: ведь все линуксоиды со стажем, использующие службы мгновенных сообщений, давно уже заготовили себе запасные ВПП в виде аккаунтов Jabber, а некоторые даже превратили их в основные аэродромы. 

Пользователи же Windows подчас оказываются лишенными связи~--- когда AOL, как всегда неожиданно (подобно наступлению морозов на Руси) меняет протоколы, делая службу ICQ недоступной для <<альтернативных>> её клиентов. Ниже мы постараемся показать, что свет клином не сошёлся на тёте Асе. И для альтернативных клиентов есть и протоколы, и службы, обеспечивающие к ним доступ. 

Надеемся также, что данный материал будет полезен совсем начинающим линуксоидам, которые ещё не вполне адаптировались в новой для себя среде. 

Наконец, хотелось бы верить, что он поможет многоопытным линуксоидам, для которых важные и дорогие записи в контакт-листе представлены ICQ-аккаунтами. И которые, подобно Д'Артаньяну, давно забыли даже то, чего никогда в виндах и не знали. Но которых жизнь ставит перед необходимостью объяснять это своим корреспондентам\dots 

\subsection{Как это делалось в Одессе}

Если бы всё делалось так, \textit{Как это делалось в Одессе}, то перед очередной сменой протокола ICQ можно было бы реконструировать следующий диалог: 

\begin{quotation}
---~Слушайте, Хакер,~--- сказал Молодой Юзер,~--- я имею вам сказать пару слов. Меня послала тетя Ася с AOL'а\dots 

---~Ну, хорошо,~--- ответил Гуря Кряк, по прозвищу Хакер,~--- что это за пара слов? 

---~В AOL вчера пришёл новый ман\'{а}гер, велела вам сказать тетя Ася\dots 

---~Я знал об этом позавчера,~--- ответил Гуря Кряк.~--- Дальше. 

---~Ман\'{а}гер собрал девелопёров и оказал девелопёрам речь\dots 

---~Новая метла чисто метет,~--- ответил Гуря Кряк.~--- Он хочет сменить протокол. Дальше\dots 

---~А когда будет смена протокола, вы знаете, Хакер? 

---~Она будет завтра. 

---~Хакер, она будет сегодня. 

---~Кто сказал тебе это, мальчик? 

---~Это сказала тетя Ася. Вы знаете тетю Ася? 

---~Я знаю тетю Асю. Дальше. 

---~\dotsМан\'{а}гер собрал девелопёров и сказал им речь. <<Мы должны задушить Гурю Кряка,~--- сказал он,~--- потому что там, где есть государь Проприетатор, там нет Хакера. Сегодня, когда Кряк закончил очередной патч к ядру и все бросятся поздравлять его по ICQ, сегодня нужно сменить протокол>>\dots

---~Дальше. 

---~\dotsТогда девелопёры начали бояться. Они сказали: если мы сделаем сегодня смену протокола, когда у него праздник, так Гуря рассерчает, и много клиентов уйдут на Jabber. Так ман\'{а}гер сказал~--- самолюбие мне дороже\dots 

---~Ну, иди,~--- ответил Хакер. 

---~Что сказать тете Асе за смену протокола? 

---~Скажи: Гуря знает за смену протокола.
\end{quotation}



Думаю, все пользователи так называемых альтернативных ICQ-клиентов\dots прежде чем продолжать, давайте спросим: а почему~--- альтернативных? Для этого надо обратиться к истории. 

Совсем-совсем недавно, лет сто назад\dots хотя в масштабах времени IT-индустрии с 1996 года прошли, казалось бы века. 

Так вот, именно эти века назад четверо израильских школьников сочинили систему обмена мгновенными сообщениями ICQ, клиентская часть которой распространялась бесплатно. И которая мгновенно завоевала предпочтения сначала пользователей-индивидуалов, а затем~--- и так называемых бизнес-пользователей. Хотя последние поначалу относились к ней настороженно, а некоторые полагали ICQ даже разновидностью компьютерных вирусов. Что, однако, не помешало триумфальному шествию нового явления по просторам Интернета. 

Короче говоря, ребята вместо тривиальной золотой жилы раскопали целый Витватерсранд. Окучить который собственными силами не могли~--- и потому успешно продали свой бизнес компании AOL. Которая и взвалила ношу поддержки ICQ на свои могучие плечи. В качестве доли своей малой получая плату за рекламу, более или менее ненавязчиво впариваемую пользователям их клиента, за которым так и закрепилось название~--- ICQ. Что на просторах нашей необъятной Родины трансформировалось в аську. Ибо здесь хорошо помнили ещё, наряду с незабвенным Лёней Голубковым и очень простой фирмой Сэлдом, также и заботливую тётю Асю, никогда не приезжавшую в гости без поллитры\dots не подумайте плохого, отбеливателя; да ещё она и стиральный порошок на закусь прихватывала. 

Дурной пример заразителен. И по стопам тех самых ребят пошли многие другие, начавшие сочинять программы, способные работать с протоколом ICQ~--- такие же бесплатные, в чём-то более удобные или более функциональные. А главное~--- способные работать не только под Windows, как первозданная аська, но и под более иными операционками, конкретно~--- свободными и Unix-подобными. И, что было ещё главнее с точки зрения нынешнего владельца службы ICQ, не показывавшие их рекламы и, соответственно, не приносившие в мошну AOL ни единого шекеля. 

Кроме того, широкие народные массы вспомнили и о существовании иных протоколов передачи мгновенных сообщений~--- во-первых (а аська на этом поприще была далеко не первой), и занялись разработкой новых, в том числе и свободных. И потому большинство <<альтернативных>> программ для работы с ICQ, кроме удобства и функционала, предлагали своим пользователям ещё и мультипротокольность. 

В частности, могу сказать за себя. Если лет 12 назад один из авторов этой заметки, как и все его знакомые, пользовал каноническую ICQ, то ныне в круге его общения (а он довольно обширен) нет никого, юзавшего бы родную аську. Даже среди самых отпетых вендузяднегов и записных пользователей Windows\dots 

Сначала доля <<альтернативников>> была не велика. Но в силу указанных причин, подкреплённых ещё и отсутствием рекламы, она, как поросёнок из оперетты, росла и росла. Пока не выросла в большую-пребольшую свинью, подложенную AOL'у. И тут AOL'овские 
ман\'{а}геры, подобно приставу из бабелевской истории, забили в набат и начали собирать свой участок на предмет борьбы с таким безобразием. 

Шпики из участка, в отличие от бабелевских, бояться не начали. А предложили главному приставу радикальный, казалось бы, метод борьбы: время от времени устраивать облавы\dots пардон, смены протокола. С тем, чтобы новый протокол, поддерживаемый, разумеется, каноническим ICQ-клиентом, служил барьером, отсекавшим <<альтернативников>> со старыми версиями протоколов. 

Метод оказался не эффективным: разработчики <<альтернативных>>, особенно свободных, программ передачи мгновенных сообщений реагировали на это в соответствие со своим титулом, то есть~--- мгновенно. И единственным действенным ответом им было~--- учащать и учащать смены протокола. 

Сначала таковая происходила не периодически, а просто время от времени~--- ну как меняются версии программ. И пользователи ещё не просекли грядущего. Хотя наиболее дальновидные уже тогда начали переносить свои контакты, особенно важные по делу или дорогие по жизни, на Jabber. 

Потом более или менее устаканилась периодичность смены раз в полгода. И постепенно все к этому привыкли: патчи, добавляющие поддержку новой версии протокола, для всех свободных клиентов ICQ сотоварищи появлялись в считанные часы после облавы, а спустя сутки-другие выходили и новые версии этих самых клиентов, в том числе, и в бинарных сборках для наиболее распространённых дистрибутивов. 

В общем, пошла нормальная цивилизованная жизнь: разработчики канонической ICQ делали вид, что они борются с <<альтернативниками>>, а последние~--- делали вид, что они этой борьбы страх как боятся. Но, тем не менее, постепенно оттягивались на Jabber. 

Хотя и не все: кому-то было жаль аськиных контактов с отсталыми ретроградами, кому-то таковые требовались по долгу службы, а гейт между Jabber'ом и ICQ налаживать ленились\dots 

Однако, наконец, в AOL'овском участке появился новый пристав~--- та самая новая метла, которая чисто метёт. Это (и то, что будет говориться далее) исключительно наша интерпретация событий~--- никаких агентурных источников, в виде молодого человека от тёти Ханы, у нас не имелось. А поскольку новому приставу самолюбие оказалось\dots ну не дороже денег, конечно, просто он, наивняк, думал своё самолюбие прямым путём пересчитать в СКВ\dots 

Так вот, решил он облавы участить. И смены протоколов начали происходить сначала раз в год, потом~--- раз в полгода, потом внедрили ежеквартальный график. И, наконец, кажется, перешли целиком на самую прогрессивную, раз в две недели, систему. 

Чем ответил на это народ? Очевидно, что народ опенсорсный, давно знающий о свободных протоколах, столь же давно подготовил себе и зап\'{а}сные пути в виде Jabber-аккаунтов, о чём говорилось в преамбуле. 

Но и на простых советских граждан надежда плоха: в ответ на наглое бесчинство бухгалтера Кукушкинда широкие народные массы неожиданно вспомнили о Google Talk, который являет собой самый обычный Jabber, действующий по тому же самому протоколу XMPP. А поскольку почтовый адрес в системе \texttt{gmail.com}~--- это ни что иное, как логин для любого Jabber-клиента, все пользователи указанной службы автоматически оказываются Jabber'истами. И \texttt{gmail.com}'ом список нечаянных Jabber'истов далеко не исчерпывается, в чём мы скоро убедимся. 

Так что пусть AOL скрежещет злобно и зубовно. Мы ответим ему исконно и кондово, словами волжских атаманов, вынесенными в эпиграф этой статьи: 


\begin{shadequote}{}
Сарынь на кичку! Айда на Jabber!
\end{shadequote}

И пусть даже светлый его жабий лик явится нам в ипостаси Google, которая, как предсказывают пессимисты, когда-нибудь потом, при поддержке длинной ЦРУ'шной руки, станет мировым жандармом-монополистом. Но ведь это будет потом. А сейчас эта длинная рука протягивается нам в виде помощи. И, повторяю, рука эта не одна. 

Правда, меня, как и всякого порядочного пост-советского человека, терзают смутные сомнения: а не внедрился ли в тесные ряды AOL'а тот самый длиннорукий Google'вец? И не он ли исполняет роль нового чисто метущего пристава? Если да~--- имя его должно занести в скрижали мировой разведки. Наряду с именами Ланового, Бониониса, Тихонова\dots 

И потому завершить этот раздел уместно будет словами Таманцева из <<Августа 44-го>>: 


\begin{shadequote}{}
Бабушка приехала! Бабулечка!
\end{shadequote}

Ибо тётя Ася за минувшие годы наверняка стала бабушкой. И, похоже, она действительно приехала~--- дальше некуда. Конечная остановка. Поезд следует в депо. Просьба срочно освободить вагоны. Пользуйтесь услугами наземного транспорта. 

\section{Дело Ханса Рейзера} 
\begin{timeline}2006, октябрь~--- 2008, июль\end{timeline}

Эти материалы собирались и писались  все полтора года вяло двигавшегося процесса, на разных его стадиях. Здесь я попытался уложить их в единую сюжетную линию. Некоторые фрагменты были написаны по моей просьбе Алексеем Жбановым ala allez, в прошлой жизни профессиональным криминалистом (публикуется с его разрешения), и Uncle\_Theodore, в миру Олегом Свидерским, знакомым с американским правосудием не понаслышке. Увы. Олег ушёл к верхним людям, но, думаю, он бы не возражал.

\subsection{Вступление}

Материалы для этих заметок я начал собирать практически сразу, как узнал о поводе для них~--- аресте Ханса Рейзера, известного программиста Open Source, разработчика одной из самых совершенных файловых систем в истории IT-индустрии. Аресте по обвинению в убийстве собственной жены, хотя практически и бывшей. 

Первое время я регулярно излагал фактографическую сторону дела, не особенно это афишируя~--- в расчете на то, что ясность появится буквально в ближайшие если не дни, то недели. 

Однако время шло, а никакой ясности не наступало: судебные заседания либо ничего не решали, либо просто откладывались, иногда по смехотворным причинам. 

Но вот теперь в этом деле поставлена очень жирная точка~--- вердикт присяжных: виновен в убийстве первой степени. В нашей терминологии это нечто вроде убийства при отягчающих обстоятельствах~--- с заранее обдуманными намерениями. Это при отсутствии тела и недоказанности самого факта убийства (\textit{на момент вынесения вердикта}). 

Потому я и решил собрать все свои заметки~--- именно в том виде, в каком они сочинялись по ходу дела~--- лишь с минимальной правкой и, местами, с комментариями. В них неизбежны противоречия~--- я сознательно даже не пытался их устранить: ведь противоречия суть не менее весомые косвенные свидетельства, которые, возможно, когда-нибудь послужат делу выяснения истины. 

Впрочем, я и сам рассчитываю дожить до того момента, когда истина будет установлена. Если нет~--- возможно, мои заметки послужат сюжетом какого-нибудь детектива. Только не след забывать, что речь в них идет не о литературных героях или злодеях, а о реальных людях, наших современниках.  И по сей день нет в этом деле никакой ясности. Ибо это вовсе не детектив Гарднера с адвокатом Перри Мейсоном в главной роли. 

\subsection{Часть 1}

Эта детективная история разворачивается сейчас, можно сказать, на наших глазах. Причем пока достоверно неизвестно, что же собственно произошло. А уж чем дело кончится~--- и подавно. 

Сначала кратко, чисто для справки, о ком идет речь. Ханс Рейзер~--- очень известный человек в мире разработчиков и пользователей открытого программного обеспечения, создатель одной из лучших файловых систем для Linux~--- ReiserFS, и её принципиально новой версии~--- Reiser4, роль которой далеко выходит за рамки файловых систем вообще. В описываемо время проживал в городе Окленд штата Калифорния. Чуть позднее я вернусь к биографическим подробностям~--- они важны сейчас и, возможно окажутся еще более важными в будущем. 

Традиционный disclaimer. Я постараюсь писать, излагая только фактографическую сторону дела, как говорил старина Тацит, без гнева и пристрастия, по возможности избегая всякого рода моральных оценок (и оценок вообще). Получится или нет~--- судить читателю. Ибо относиться к Хансу равнодушно я не могу, слишком уважая результаты его работы. 

Итак, начало истории. Я, как и большинство моих коллег, узнал о ней ранним утром 11 октября (другое дело, что понятие об утре, и тем более раннем, у всех нас разное). Когда по всем сайтам Рунета, связанным с тематикой Linux и Open Source, прошла новость: Ханс Рейзер арестован по подозрению в убийстве своей жены в городе Окленд, 10 октября, в 11 часов по местному времени. 

В первый момент это сообщение было воспринято как утка или глупая шутка~--- кое-кто задумался даже, не 1-е ли на дворе апреля. Однако в таких случаях каждый IT-шник по привычке обращается к Google. И гугление (кстати, в английском языке глагол \textit{to google} недавно утвержден официально) показало, что ни об утке, ни о шутке говорить не приходится~--- история эта началась не вчера и широко освещалась в региональной прессе и Интернет-изданиях, находя отражение даже на русскоязычных форумах тех краев. Более того, уже некоторое время существовал сайт, специально посвященный одному из аспектов этого дела. 

Видимое начало истории падает на 3 сентября~--- день, когда последний раз видели жену Ханса, Нину Рейзер. С тех пор, и по сей день (12 октября, 19:15 московского времени) её не видел никто~--- ни живой, ни мертвой. 

Добавлю с скобках, что на момент, когда редактируется эта заметка (20 июня 2007 года), никакой информации о её судьбе нет по прежнему. 

Но тут мне придется прервать повествование, чтобы вернуться к биографическим подробностям~--- как уже было сказано, они могут оказаться важными. 

Итак, Ханс Рейзер\dots Родился в 1964 году в Калифорнии, там же поучился в средней школе, которую бросил в возрасте 14 лет, по причине несогласия с программой. Тем не менее, в 15 лет его приняли в Калифорнийский университет, Беркли (тот самый, где была создана основа современного Интернета в виде протокола TCP/IP, операционная система BSD и множество других штуковин). И, насколько я понимаю специфику ихнего образования, обучался там по некоей индивидуальной программе (физика, математика и соплеменные науки). 

Сколько Ханс проучился в университете~--- я пока не выяснил. Известно лишь, что диссертации (PhD) он так и не защитил, видимо, посчитав это ненужным. 

Начиная с 1988 года, деятельность Ханса протекает в сфере IT и Computer Science. С 1988 по 1996 год он занимает различные должности, связанные с программированием и администрированием, в ряде компаний, от малоизвестных (по крайней мере, мне) до Исследовательского центра IBM (IBM Almaden Research Cente). 

В 1997 году Ханс основывает собственную компанию~--- Namesys; собственно, это небольшая команда разработчиков принципиально новой файловой системы которая получает имя своего основателя~--- ReiserFS. 

К 1999 году относится событие, прямо с IT не связанное, но зато связанное с нашим сюжетом~--- женитьба. Жена~--- Нина, жительница Санкт-Петербурга, врач по образованию, 1974 года рождения. Об обстоятельствах знакомства и брака говорить не буду~--- поскольку точной информации не имею. 

Примерно в то же время в Рунете циркулировали упорные слухи, что штаб-квартира Namesys то ли переносится, то ли уже перенесена в Россию. На чем они основывались (и основывались ли вообще на чем-то)~--- не знаю. Но, что интересно, сайт компании обслуживается DNS-серверами зоны RU\dots 

Впрочем, дальнейшая профессиональная деятельность Ханса к сюжету имеет мало отношения~--- интересующиеся могут ознакомиться с ней на сайте компании и в многочисленных сетевых и оффлайновых источниках. Для нашей же истории важно, что брачный союз Нины и Ханса, проживавших в вышеупомянутом Окленде, имел результатом рождение двух детей~--- сына Рори (Rory) и дочери Niorline (транскрибировать не берусь). Ныне им 7 и 5 лет, соответственно (точнее, уже скорее 8 и 6). 

Однако брак Нины и Ханса оказался не таким долгим. В 2004 году они становятся <<раздельно проживающими супругами>>. В качестве одной из причин расставания Нина позднее, в ходе бракоразводного процесса, назовет <<трудоголизм>> мужа~--- знающие результаты работ Ханса согласятся с этим определением. 

Бракоразводный процесс, однако, начался вроде бы лишь в 2006 году. По имеющимся данным, он протекал отнюдь не полюбовно. Англо-американскую юридическую терминологию я не понимаю (впрочем, и русскую-то с трудом), но, судя по всему, Нина, в частности, настаивала на отстранении отца от участия в воспитании детей (насколько я понял за время знакомства с этой историей, в США нет понятия лишения родительских прав в нашем смысле). Тут-то и всплыло определение <<трудоголик>> (workoholic), его частые поездки в Россию (где работала большая часть его команды), а также ряд других подробностей, на которых я не буду задерживаться, ибо почерпнуты они из не вполне достоверного источника (какого~--- скажу несколько позже). 

Тем не менее, какое-то общение между (почти) бывшими супругами продолжалось. В частности, Нина привозила детей к Хансу и оставляла на несколько дней. 

Так было и в день, с которого формально начинается эта история~--- 3 сентября 2006 года. Нина привезла детей к отцу и уехала. Чтобы более не появляться. Или не уехала. Но с того момента о ней ничего не известно. По крайней мере, по сей день\dots 

И тут нужно сделать отступление об источниках. Почти все, о чем говорилось ранее, основывается на источниках узкопрофессиональных, которые\dots ну не то чтобы достовернее всех прочих, но легко проверяются IT'шниками по независимым каналам. Ныне же мы вступаем в мир массовых СМИ~--- новостных сайтов локально-регионального масштаба (нечто типа районных многотиражек старых советских времен), со всеми вытекающими последствиями. И, естественно, англоязычными. 

Благо, в этом море у нас будет и русскоязычный путеводитель~--- сайт \url{http://privet.com}, точнее~--- отдельный трейд его форума, посвященный исчезновению Нины Рейзер, Созданный первоначально исключительно для организации помощи в поисках её (или её тела), он постепенно, как и любой другой форум, оброс флеймом, флудом, оффтопиком и тому подобными, неизбежно сопутствующими, явлениями. В частности, на нем не только даются ссылки на местные новостные ресурсы, но и часто пересказывается их содержание~--- на русском языке (правда, с вкраплениями английских юридических терминов). Ныне этот трейд заглох и представляет собой чисто исторический интерес. 

Так вот, согласно указанным источникам, Нина направлялась на вечеринку к подруге. И по дороге должна была заехать в магазин за продуктами. Судя по дальнейшему, в магазин она заезжала. Но потом~--- все\dots 
Поиски начались через несколько дней (насколько я понял, по инициативе подруги, а не мужа, хотя и бывшего). Завершились на восьмой (или девятый, как считать сутки) день~--- находкой машины Нины. Аккуратно запаркованной, внутри~--- дамская сумочка с деньгами и тому подобными кредитками, мобильный телефон. И более ничего. 

\subsection{Часть 2}

Итак, продолжаю. Третьего сентября Нина (вроде бы) покинула дом Ханса, оставив там детей, и отправилась на вечеринку к подруге~--- это по одной, ранней, версии. Позднее же утверждалось, что она отправилась поужинать с бойфрендом~--- неким Энтони Зографусом (Anthony Zografos). между прочим, тоже работающим в сфере информационных технологий. Как бы то ни было, ни там, ни там она не появилась. 

Поиски Нины начались через несколько дней, по чьей инициативе~--- не очень понятно, но во всяком случае~--- не по инициативе (почти) бывшего мужа. И велись без особого энтузиазма до тех пор, пока на 8-й (или 9-й, как считать часовые пояса) день не была обнаружена её машина. 

Аккуратно припаркованная, запертая, внутри она содержала, однако, дамскую сумочку с документами, деньгами и кредитками, а также мобильник. Были в машине и продукты, купленные для вечеринки~--- вне зависимости от того, с кем она планировала её провести. 

После этого за поиски принимаются более серьезно. На сайте \texttt{privet.ru} возникает упомянутая выше тема, посвященная исчезновению Нины Рейзер, затем возникает сайт <<Помогите найти Нину Рейзер>> (\textit{он давно не работает}), где за предоставление информации о ней объявляется награда в~15 тысяч американских рублей. 

Одновременно с этим начинает витать мысль о причастности Ханса к исчезновению жены. На чем она основывается~--- непонятно, никаких сообщений в прессе или заявлений официальных лиц не обнаруживается. 

Наконец, в конце сентября (23-24) осуществляются массовые поиски в окрестных лесных массивах~--- силами полиции, с собаками и привлечением добровольцев. Поиски оказываются безрезультатными~--- никаких следов Нины не обнаружено. 

Сам Ханс всячески избегает встреч с полицией~--- по выражению прессы, отказывается от сотрудничества. Дети переходят на попечение общества~--- то есть, насколько я могу судить, помещаются в нечто вроде приюта для сирот. Хотя в наличии имеется бабушка по отцу~--- Беверли Палмер, а из Петербурга приезжает бабушка по матери~--- Ирина Шаранова. 

Хансу все же не удается избежать общения с правоохранительными органами~--- в начале октября у него берут кровь для анализа (как задумчиво сообщалось в прессе, для анализа на ДНК). Из чего можно было заключить, что где-то найдены какие-то т.~н.~биологические улики (кровь, волосы и т.п.). 

Затем в доме Рейзера производится обыск, давший такие улики, которые позволили санкционировать 10 октября его арест. По сообщениям прессы, вместе с ним было арестовано еще два человека, но об их личностях никакой информации в известных мне источниках не имеется (и в дальнейшем о них не упоминается вообще). Теперь, по законам штата Калифорнии, ему в течении 48 часов должно быть предъявлено обвинение\dots 

Каковы были улики, обнаруженные в доме Рейзера? В разных вариантах эта тема муссировалась в местной интернет-прессе, но я от ответа на этот вопрос пока воздержусь. Ибо жду заключения одного из членов нашего комьюнити~--- в недавнем прошлом эксперта-криминалиста по профессии. 

А пока сделаю маленькое технологическое отступление. Я уже говорил, что Рейзер был создателем и основным разработчиком файловой системы для ОС Linux~--- ReiserFS, давно (с 2001 года) и успешно используемой многими и многими пользователями, в том числе и в индустриальных масштабах~--- например, на критически важных серверах Интернета и интранета. 

В последние же годы им разрабатывалась файловая система Reiser4~--- с одной стороны, логическое продолжение своей предшественницы, с другой~--- принципиально новая разработка, функции которой далеко выходят за рамки просто файловой системы. В августе 2004 года она обретает статус релиза~--- то есть считается вполне надежной и годной к индустриальному применению. Однако поддержка её не включается в официальное ядро Linux. В результате те пользователи Linux, которые хотели использовать Reiser4, должны были обращаться к патчам ядра сторонних разработчиков. Впрочем, в некоторых (хотя и единичных) дистрибутивах поддержка Reiser4 была штатной функцией (ныне ни в одном из дистрибутивов её не найти). 

Причины отсутствия официальной поддержки ReiserFS выдвигались разные, иногда достаточно формальные~--- например, несоответствие форматирования кода драйверов Reiser4 стандарту, принятому в коде ядра; это примерно то же самое, что (для литературных текстов) различие в величине абзацных отступов. 

И так было на протяжении более чем двух лет~--- срок для Open Source проектов просто астрономический. Пока Эндрю Мортон, один из главных разработчиков ядра Linux, в своем интервью не обмолвился (правда, между строк)~--- если поддержка Reiser4 будет включена в ядро, а разработка её по каким-либо причинам прекратится, это доставит много сложностей остальным разработчикам. Было это 22 сентября 2006 года\dots 

Что же до ReiserFS~--- то, как я говорил, она использовалась очень широко, во многих дистрибутивах Linux выступая в качестве файловой системы по умолчанию. В частности~--- в дистрибутиве Suse, второй по распространенности Linux-системе, в том числе и в корпоративной сфере. Дистрибутив этот, начиная с 1994 года, разрабатывался одноименной немецкой фирмой, активно участвующей в финансировании проектов по разработке открытого софта, в частности~--- и фирмы Namesys, принадлежащей Рейзеру. 

В скобках заметим, что вторым основным источником финансирования работ Ханса (а может быть, и первым) было агентство DARPA (Defense Advanced Research Projects Agency~--- агентство передовых оборонных исследовательских проектов), находящее в подчинении Министерства обороны США. И это при том, что большинство разработчиков команды Namesys проживают в России и на Украине\dots 

В 2004 году фирма Suse была куплена компанией Novell~--- для людей, тесно не связанных с IT-индустрией, имя это, в отличие от Microsoft, ничего не говорит. Однако для IT-специалистов, особенно нашей страны и сопредельных стран оно~--- нечто типа живой легенды, выступая своего рода синонимом локальной сети, как Ленин и Партия (говоришь сеть~--- подразумеваешь Novell, говоришь Novell~--- подразумеваешь сеть). 

До того времени фирма Novell в сфере свободного софта никак не отметилась, и потому поначалу продолжались традиции Suse. Однако постепенно они тем или иным образом модифицировались, пока наконец в выступлении одного из руководителей разработки Suse не прозвучал отказ от использования ReiserFS в качестве умолчальной. Резонов к тому он привел много, повторять их не буду, но поверьте~--- все они были более чем смешны. Это произошло 4 октября 2006 года. До ареста Ханса Рейзера оставалась еще неделя\dots 

PS Здесь позвольте мне взять тайм-аут. Конечно, в добрых традициях журналистских расследований следовало бы развести турусы на колесах со ссылками на компетентные мнения. Но мне вдруг стало по настоящему страшно. Ссылки на интернет-треп никому ничем по настоящему не опасны~--- и то в первой части своего очерка я их избегал. А тут ребятам реально может грозить потеря работы~--- мир тесен, и все носители компетентных мнений вычисляются очень легко\dots 

Все равно до окончания истории еще далеко. В следующей части я вернусь к уликам и обвинительному заключению, которое и было предъявлено Хансу Рейзеру. 

\subsection{Часть 3}

Теперь пришла пора обратиться к уликам, послужившим основанием для ареста Ханса Рейзера. О них писалось уже накануне судебного заседания, но сколько-нибудь целостную картину стало возможным представить только по его результатам. Да и то, в картине этой зияющие прорехи видны невооруженным глазом. 

Я уже говорил, что у Ханса брали кровь на анализ. Теперь выясняется, зачем: в доме и в машине Ханса (не Нины!) были обнаружены какие-то следы крови (в сетевом репортаже сказано гораздо экспрессивней~--- брызги, но, как мы увидим дальше, это Интернет-издание нельзя упрекнуть в излишней сдержанности в отношении оценки фактов). Так вот, в этом самом репортаже утверждалось, что это кровь Нины. На чем это утверждение основывалось, за отсутствием тела, не очень ясно. Конечно, данные о крови Нины, как дважды рожавшей женщины, должны были где-то наличествовать~--- но достаточно ли их для однозначной идентификации? 

Далее, в машине Рейзера отсутствовало пассажирское кресло. Причем свою машину он пытался скрывать от полиции~--- ездил на прокатной. 

И, наконец, третья улика~--- две книги о расследовании преступлений, которые он приобрел в оффлайновом книжном магазине через несколько дней после исчезновения жены. В источниках можно найти их названия~--- но вряд ли они что-то скажут нашему читателю, так что не буду копаться\dots 

Все это в принципе было опубликовано еще до судебного заседания (в том числе и утверждение о том, что обнаруженная в доме и машине кровь принадлежит Нине Рейзер!). А вот что новое прозвучало на заседании~--- так это сведения о показаниях детей. Самих их на суде, естественно, не было. Но, по сведениям полиции, по приезде в дом Рейзера дети играли в компьютерные игры и слышали разговор родителей. 

Впрочем, и сообщение об их показаниях основывается не на стенограмме суда, а на словах одного из офицеров полиции. Вот как оно выглядит (цитирую фрагмент интернет-сообщения в переводе allez'а, кстати говоря, криминалиста по профессии): 


\begin{shadequote}{}
Один из детей указал, что <<Ганс Райзер и Нина Райзер, возможно, спорили>>,~--- написал полицейский следователь по розыску пропавших без вести Райан Джилл. <<Ребенок указал, что его родители говорили на <<средних>> тонах и что они использовали <<нехорошие слова>>,~--- написал Джилл.
\end{shadequote}

Один из детей позже сказал следователю, что он пошел наверх, в комнату, где были его мать и отец. Ребенок сказал, что Ганс Райзер велел ему спуститься вниз и не возвращаться, согласно заключению. 

Сообщаются и кое-какие подробности об автомобиле Нины~--- том самом, с обнаружения которого всерьез начались её поиски. По нему были раскиданы продукты~--- купленные вместе с детьми в большом универмаге до приезда к дому Рейзера~--- и это подтверждается просмотром видеозаписей магазинных камер наблюдения. 

А еще~--- сотовый телефон, о котором говорилось ранее, был разобран, то есть вынут аккумулятор. О наличии или отсутствии SIM-карты ни единого слова не говорится (правда, в ходе одного из обсуждений мне подсказали, что в телефонах CDMA, применяемых на Калифорнийщине, SIM-карты может и не быть вообще). 

И еще не могу удержаться от цитаты: 


\begin{shadequote}{}
Служебные собаки отработали место вокруг автомобиля, но след не взяли. Это привело следователей к выводу, что Нина Райзер не была в том месте, где был найден её автомобиль. Окрестные свидетели сообщили полиции, что фургон был припаркован там 5 сентября.
\end{shadequote}

Правда, тактично не сообщается, когда там побывали служебные собаки~--- в день обнаружения автомобиля, через неделю или через месяц. 

Вот и всё, что было известно на момент сочинения этой заметки. 


\textit{Интермедия. Мнение Алексея Жбанова aka allez}


С точки зрения криминалиста могу сказать одно: а практически нечего тут сказать. Если кого-то разочаровал, прошу прощения. Но это только книжный детектив Ниро Вульф распутывает сложнейшие дела, не выходя из дома. И то лишь потому, что верный помощник Арчи Гудвин выполняет нелегкую работу по снабжению своего босса максимально полной и достоверной информацией. По поводу полноты имеющейся у меня информации можно лишь тихо выматериться, о достоверности же оной, я думаю, даже говорить не стоит~--- источники те еще\dots 

Да и американская экспертная практика, надо сказать, несколько озадачивает. За время своей службы в милиции я твердо усвоил, что вывод в заключении или справке эксперта должен быть категорическим, то есть содержать один из всего лишь двух вариантов ответа на поставленный вопрос~--- <<да>> или <<нет>>, в противном случае~--- <<НПВ>> (жаргонизм от слов <<не представляется возможным>>; третьим вариантом не является, так как <<НПВ>>~--- это отсутствие ответа как такового). У американцев же кроме <<да>> и <<нет>> допускается также <<может быть>>\dots 

В общем, имеется следующее: 

\begin{enumerate}
	\item Пропавшая без вести женщина. 
	\item Находящийся с ней в разводе и натянутых отношениях муж. 
	\item Свидетельство малолетнего ребенка о факте ссоры родителей в день исчезновения матери. 
	\item Брошенный автомобиль Нины без признаков ограбления, но с косвенными и недостоверными признаками борьбы в нем (разбросанные продукты, разобранный мобильник). 
	\item Пятна крови в доме Ганса и на чехле спального мешка в его автомобиле, которая, может быть, принадлежит Нине. 
	\item Вымытый автомобиль Ганса без правого переднего сиденья с двумя книгами о расследовании убийств, початым рулоном мешков для мусора, скотчем, торцевыми ключами и сифонным насосом.
	\item Неприветливое отношение Ганса к полиции (отказ сотрудничать, попытки уйти от наружного наблюдения). 
\end{enumerate}


Какой из вышеизложенного можно сделать вывод? Со своей стороны могу сказать лишь одно: НПВ. Во-первых, нет никаких данных о том, какие производились экспертизы и по каким методикам. Во-вторых, все улики косвенные. Я, прошу заметить, служил не следователем и даже не дознавателем, поэтому УПК РФ знал лишь в части, меня касающейся (да и то уже подзабыл, каюсь), но, по-моему, с такой <<доказухой>> у нас в России человека можно лишь задержать на срок до трех суток, что ли. А уж об аресте и тем более предъявлении обвинения и суде вовсе речи быть не может~--- только оперативная разработка. Что будет дальше там, в Окленде, я не хочу и не имею никакого права даже прогнозировать. 

Резюмируя, хочу написать следующее: сколько бы мы ни сказали в этой теме слов, какими бы они ни были, а словом делу не поможешь и на положение вещей (по крайней мере, в данном конкретном случае) не повлияешь. Что же тут можно сделать? Не знаю точно, но на мой взгляд, тем из нас, кто верует, стоит помолиться за благополучный исход дела, неверующим же~--- изо всех сил надеяться на этот самый благополучный исход. Я, как неверующий, буду надеяться и ждать. Молча. 

\subsubsection{Мнение Uncle\_Theodore}

Американское правосудие еще никогда не было настолько опозорено со времен суда над О.~Джей Симпсоном, когда черные присяжные вопреки уликам оправдали черного актёра, убившего свою белую жену. Рейзера признали виновным в убийстве первой степени, то есть в предумышленном спланированном убийстве. Ему грозит от 25 лет тюрьмы до пожизненного заключения. 

Основной причиной такого решения присяжных адвокат Рейзера Дюбуа назвал отрицательное впечатление, которое Рейзер произвел на присяжных. Если учесть, что Дюбуа сам заставил Рейзера давать показания, а потом произнес совершенно дебильную заключительную речь, жуя сопли, я думаю, дело не в Рейзере, а в самом Дюбуа. 

В общем, дебилизм сплошной\dots Грустно мне. 

Прокурор произнес великолепную речь, растянувшуюся на несколько дней. Кровь на столбе в доме Рейзера, кровь на чехле от спального мешка в машине Рейзера. И тот факт, что он вел себя так, как будто знал, что Нина мертва. И все. 

Правду говорят, неужели можно отдать решение судьбы человека в руки 12 человек, у которых даже не хватило ума отмазаться от обязанности заседать в жюри\dots 

Прокурор произнес выдающуюся речь. Ну так еще бы, учитывая, из какого г\dotsа ему пришлось конфетку делать\dots А адвокат наоборот, сказал, по-моему, самую дебильную речь за всю историю своей работы\dots 

\subsection{Часть 4}

Предыдущие заметки писались по следам событий, по мере появления информации~--- или по мере того, как я её находил. Внимательный читатель наверняка нашел в них немало противоречий. Готов поклясться~--- я ничего не придумывал, и все противоречия имели место быть в моих источниках, уж больно они были различны. Тут и сообщения Интернет-прессы, и сплетни с форумов\dots 

На упрек~--- а зачем же я пользовался заведомо недостоверными источниками,~--- отвечу с легкостью: а какие источники для этой истории могут считаться достоверными? Неужели за таковые можно считать сообщения провинциального издания Inside Bay Area, освещавшие историю пропажи Нины Рейзер, её поисков и ареста её мужа? Издания, которое, ничтоже сумняшеся объявляет принадлежность найденных следов крови, не удосужившись ни малейшей аргументацией. А ведь все (ВСЕ!) остальные Интернет-публикации по делу Рейзера представляют собой простое переразмещение материалов Inside Bay Area\dots 

Посты же с \href{http://forum.privet.com}{forum.privet.com}, если тщательно <<фильтровать базар>> (а не это ли~--- основное призвание IT-шников, фильтровать сигналы от шума?), дают массу информации~--- во-первых от людей, лично знавших Нину, и о них самих, во-вторых, о царящих в обществе настроениях. Следует только учитывать, что пост этот создан личной знакомой Нины, выступающей под ником JU, и первоначально предназначался исключительно для организации её поисков, и лишь позднее (хотя и очень быстро) оброс всякого рода флеймом и флудом. Посему полагаю, что указанный трейд скоро будет закрыт, а возможно, и уничтожен вообще. 

Короче говоря, настало время свести воедино все наличные факты в их хронологической последовательности и многообразии версий. 

Итак, начинаем с 3 сентября, 13-55 местного времени~--- это первая и чуть ли не последняя точно установленная последняя временная метка: в это время Нина с детьми выходит из магазина <<Беркли Боул Маркет>>, где закупала продукты; это зафиксировано камерами видеонаблюдения. Согласно сообщению считается, что около 14 часов она подъехала к дому Рейзера. Точнее, его матери, Беверли Палмер, где он жил после фактического развода. Сама Беверли Палмер в это время отсутствовала, находясь на фестивале искусств <<Burning Man>> в Неваде (тут источники ссылаются на её собственные слова). 

Тут начинаются первые расхождения. Согласно ранним источникам, Нина высаживает детей у дома и отправляется дальше~--- мы еще вернемся к вопросу, куда. Поздние же сообщения, основанные на показаниях детей Рейзеров, гласят, что она некоторое время была в доме. 

Как и почему она там оказалась, причем не где-нибудь в прихожей, а в гостиной на втором этаже (позднее, по показаниям одного из детей, оттуда будут доноситься голоса бывших супругов)~--- это первый вопрос. Ведь, согласно многочисленным сообщениям прессы, Нина относилась к бывшему супругу настороженно, поскольку он после расхождения угрожал ей физическим насилием. Правда, основанием для этих сообщений были исключительно свидетельства личных знакомых Нины (кого именно, в источниках не сообщается). В обсуждении на \texttt{privet.com} намеков такого рода вдоволь~--- вплоть до того, что в полиции ей предлагали обзавестись оружием для защиты от посягательств бывшего супруга. Впрочем, эти сведения исходят не от полиции, а от одного из участников обсуждения на форуме. 

Тем не менее, игнорировать прямое свидетельство прессы нельзя: дети играют на компьютере, на первом этаже дома, и слышат доносящийся со второго разговор родителей, ведущийся на повышенных тонах, с употреблением <<нехороших>> выражений. Затем один из детей (кто именно~--- источники умалчивают) поднимается наверх, но отец отсылает его обратно. Далее~--- лакуна в источниках, до 5 сентября, когда Нина не забирает детей из школы. 

Тут возникает целая серия вопросов. Первый: как уже сказано, в тот день, 3 сентября, Нина, после того как завести детей к Хансу, должна была ехать~--- то ли на вечеринку к подруге (свидетельство JU с форума \texttt{privet.com}), то ли на ужин с другом, сиречь Энтони Заграфусом (все более поздние сообщения прессы). Ни у безымянной подруги, ни у Энтони она не появилась. Тем не менее, беспокойство о её отсутствии начало проявляться только 5-го числа~--- когда, как было сказано, она должна была забрать детей из школы, и не сделала этого (вопрос в скобках~--- а как дети попали в школу? отвезены отцом?). Кто проявляет беспокойство~--- подруга, друг или школьная администрация,~--- источники хранят молчание. По ряду косвенных признаков можно заключить, что во всяком случае не Ханс: на \texttt{privet.ru} можно найти намеки на то, что даже своей матери, вернувшейся из Невады, он ничего не говорит об исчезновении Нины, Беверли узнает об этом от полиции. 

Первоначально полиция особого рвения в поисках не проявляет, что и понятно: истории с загулявшими мужьями-женами, думаю, для Америки столь же обычны, как и для России. Однако 9 сентября обнаруживается автомобиль Нины, и дело принимает несколько иной оборот. 

Автомобиль стоял, аккуратно запаркованным и запертым. По ранним свидетельствам, внутри были: дамская сумочка, кредитные карточки, наличные деньги, всякого рода документы, в том числе и финансовые, типа чеков, и мобильный телефон. Были в машине и продукты, закупленные для вечеринки или ужина. В общем, из ранних описаний создается впечатление, что хозяйка оставила машину на минуту~--- о более в нее не вернулась. 

Позднее оказывается, что телефон был вскрыт и приведен в нерабочее состояние~--- из него был вынут аккумулятор (что интересно, о SIM-карте ни в одном сообщении не было сказано ни слова). Что же до продуктов~--- то они оказались разбросанными по всему фургону. 

Когда появился автомобиль на месте его обнаружения~--- также не ясно. В одном из сообщений проскочило, что, по свидетельству соседей, не ранее 5 сентября. 

Сведения о месте нахождения автомобиля ничего не говорят тому, кто не знаком с тамошней топографией. Однако JU в сообщении на форуме утверждает, что он был обнаружен примерно между домом Рейзера и домом той самой подруги, к которой Нина, по ранней версии, направлялась на вечеринку, но ближе к последнему (невольно хочется задать провокационный вопрос~--- а не JU была ли той подругой?) 

Потому что позднее во всех сообщения прессы утверждалось, что она собиралась ехать на ужин с Заграфосом~--- к нему домой? к себе?~--- учитывая наличие продуктов, последний вариант более вероятен. 

Осмотр местности вокруг автомобиля со служебными собаками дал отрицательный результат~--- след Нины собаки не взяли. Однако не ясно, когда это происходило~--- в день находки машины или позднее, и если позднее, то насколько. 

Хронология дальнейших событий не ясна совершенно. Можно лишь предположить, что с активизацией поисков Нины полиция первым делом обратилась к Хансу~--- и как к мужу, пусть и бывшему, и как к человеку, видевшему её последним. По сообщениям прессы, каких-либо объяснений на этот счет Рейзер не дал, и вообще, как пишет пресса, отказался от сотрудничества с полицией и всячески избегал контактов с ней. В результате этого он был взят под наблюдение. Видимо, в эти же дни дети Рейзеров помещаются в государственный приют и допрашиваются полицией. 

Следующая дата~--- 19 сентября, когда полиция обнаруживает автомобиль Рейзера, надо полагать, брошенный~--- есть сведения что сам Рейзер ездил на машине, взятой в прокат. В автомобиле отсутствовало переднее пассажирское кресло, которое так до сих пор и не найдено. Зато в нем присутствовали: 

\begin{itemize}
	\item рулон больших мешков для мусора, причем нескольких мешков не хватало; 
	\item две книги, <<Убийство: год на смертельных улицах>> Дэвида Саймона и <<Шедевры Убийства>> Джонатана Гудмана, которые были куплены 8 сентября; 
	\item свидетельства помыва машины, в виде остатков воды под ковриком и сифонного насоса; 
	\item штрафная квитанция за нарушение правил дорожного движения, от 12 сентября; выдавший её полицейский вспомнил, что переднее сиденье <<определенно было на месте, и он бы запомнил, если бы оно отсутствовало>>; 
	\item наконец, самое главное~--- спальный мешок с пятном крови. 
\end{itemize}


В связи с последним обстоятельством Рейзер 28 сентября задерживается полицией для взятия крови на анализ. При этом у него, кроме денег (более 8 тысяч баксов) и документов обнаруживается чек на сифонный насос (к сожалению, о дате чека ничего не сообщается). 

А тем временем, 22 сентября, в интервью Эндрю Мортона, одного из основных разработчиков ядра Linux, проскальзывает причина, по которой поддержка файловой системы Reiser4 до сих пор не включена в ядро: это~--- возможность прекращения её разработки\dots 

В эти же примерно дни (23-24 сентября далекого уже 2006 года) осуществляются массовые поиски в окрестных лесных массивах~--- силами полиции, с собаками и привлечением добровольцев. Поиски оказываются безрезультатными~--- никаких следов Нины не обнаружено. 

Вероятно, анализ крови Рейзера показал, что кровь на спальном мешке из машины~--- не его. Что, видимо, и дало основание для обыска в его доме. Где в гостиной были обнаружены следы (в одном из сообщений говорится~--- брызги) крови. Поскольку именно это и позволило санционировать затем арест Ханса, можно предположить, что она оказалась идентичной найденной на спальном мешке. Однако поведение служебных собак и тут дало отрицательный результат. 

К 4 октября относится заявление одного из руководителей фирмы Novell о том, что отныне файловая система ReiserFS не будет использоваться в дистрибутиве Suse в качестве устанавливаемой по умолчанию. 

И, наконец, 10 октября полиция производит арест Ханса Рейзера и его заключение в тюрьму. 12 октября на заседании суда, где Рейзер присутствует в наручниках и арестантской робе, ему предъявляют обвинение в убийстве. По свидетельству JU (видимо, на суде присутствовавшей), он выглядит испуганным (интересно, а как будет выглядеть любой человек, впервые в жизни оказавшийся в такой ситуации?). Собственно суд должен состояться 28 ноября в 9 часов местного времени. До суда Рейзер будет содержаться в тюрьме без права выхода под залог. Тем не менее, его адвокаты не выражают сомнения в том, что на суде Рейзер будет освобожден. 

Я не могу и не хочу делать каких-либо выводов о виновности или невиновности Ханса Рейзера. Однако обращает на себя внимание отношение общественности к его делу. Оно проявляется в местных СМИ~--- фактически с самых первых публикаций читатель упорно подталкивается к мнению о несомненной виновности Ханса. В частности, в одном из сообщений прямо утверждается, что кровь, обнаруженная в машине и доме Рейзера, принадлежит Нине. Хотя уже в следующей публикации автор поправился~--- эксперты не исключают её принадлежности Нине Рейзер. 

Показательно также настроение, царящее среди участников упомянутого выше трейда на сайте \texttt{privet.com}~--- большинство из них вину Рейзера под сомнение не ставят~--- и эта позиция поддерживается администрацией сайта. Редкие же участники, напоминавшие о презумпции невиновности, либо подвергались давлению, либо просто банились (подобно автору этих строк). 

Я не случайно привел тут данные о позиции разработчиков ядра Linux и компании Novell~--- оба они были сделаны до ареста Рейзера (интервью Мортона~--- фактически сразу после обнаружения его брошенной машины). 

Вот пока и все сведения по делу Ханса Рейзера. Полагаю, что новой информации до суда не появится: адвокаты Ханса, скорее всего, будут молчать, а полиция, вероятно, сказала прессе все, что считала нужным; адвокаты полагают, что даже более, чем нужно, и выразили недовольство утечкой информации в прессу, подчеркнув, что к настоящему времени сказано гораздо больше, чем выявлено. 

\subsubsection{Версия Uncle\_Theodore}

Нина не сбегала в Россию, она умерла в США. Согласны? Уехав от Ханса, она отправилась к Хелен, чтобы расслабиться без детей и встретилась там с кем-то из своих друзей-выдумщиков по части развлечений, я думаю, с Шоном Стерженом. Шон либо увез её куда-то примерно в шесть тридцать третьего сентября, либо с ней случился несчастный случай прямо в доме Хелен, примерно в это же время. 

Другой вариант~--- романтический вечер с Зографусом, который кончился плохо. Опять-таки, например, несчастным случаем в горах, например. Причем несчастным случаем такого плана, который привел бы участников этого случая в тюрьму. Садо-мазохистический акт, приведший к смерти Нины, например. Или стрельба в горах, в результате которой она была случайно убитой. 

Как бы то ни было, во вторник пятого Хелен уже знала, что произошло, и начала активно обвинять в убийстве Нины Ханса, которого давно ненавидела. Возможно, в понедельник четвертого она попыталась каким-то образом повлиять на паранойю Ханса (в моих фантазиях это выглядит так, что она могла подкинуть тело Нины на переднее сидение машины Ханса, но это уж очень проблематично) Тем не менее, четвертого, кто бы ни был убийцей, ждет чего-то. Но ничего не происходит, все идет своим чередом, и пятого в 2:30 Хелен несется в школу забирать детей Ханса. Она уже знает, что Нина их не заберет. Это было ошибкой, конечно. Её подвела русская сентиментальность. Но ей повезло в этот раз. 

Потом она опять начинает окучивать Ханса. Подговаривает русских знакомых попугать его, погуляв вокруг него в виде мафиози на rest area, начинает глобальную компанию на русском форуме, лжет в интервью про пунктуальность Нины и <<горячую любовь>> Шона к Нине. Шон, впрочем, срывается однажды. Как я уже писал на ЛОРе, его фраза <<Я убил восемь человек, но Нину я не убивал\dots>> так и хочет продолжения, что-то типа <<\dotsэто был несчастный случай!>> Впрочем, полиции уже все было ясно и так. Ханс уходит в жуткую паранойю. Ему мерещатся русские мафиози, подброшенные улики, кровь повсюду, сговор\dots И для суда его поведение выглядит доказательством убийства. 

Вывод: Хелен Дорен знает, что произошло. Хелен Дорен~--- прекрасный психолог. Несмотря на то, что она принадлежит к самому презираемому классу русской иммиграции~--- почтовым невестам~--- она использовала и американскую бюрократическую систему, и предрассудки русскоязычного сообщества, и даже паранойю Ханса, чтобы повесить на него убийство, в котором сама была косвенной соучастницей. 

\subsection{Послесловие: вопросы есть, будут ли ответы?}

Настоящая заметка не преследует целью ни доказать невиновность брата-линуксоида, ни, напротив, откреститься от изверга и убицы, позорящего наш славный клан. Я просто собрал тут вопросы, возникшие у меня (неужели только у меня?) при чтении последних материалов дела~--- начиная с первых чисел июля. Вопросы и противоречия в более старых материалах, также до сих пор не нашедшие адекватного объяснения, были уже сформулированы или в настоящем цикле, или в соответствующих темах ряда форумов. Здесь я их повторять не буду. 

Как можно узнать из последних материалов, в минувший понедельник, 7 июля 2008 года, около 16 часов по местному времени (хотя имеются сведения, что поздно вечером), Ханс Рейзер, в сопровождении своего адвоката Уильяма Дюбуа, отвел власти к месту захоронения тела своей жены, Нины Рейзер, исчезнувшей 3 октября 2006 года, в убийстве которой он обвинялся. И был признан жюри присяжных виновным в убийстве первой степени (по нашим законам этому примерно соответствует предумышленное убийство при отягчающих обстоятельствах). 

По результату этого вердикта Рейзеру грозило 25 лет тюремного заключения. Приговор, всеми ожидаемый и прогнозируемый, должен был вынести в минувшую среду, 9 июля 2006 года, судьёй Верховного суда графства Аламейда Ларри Гудманом. Однако вследствие похода властей и Рейзера к месту захоронения, завершившегося обнаружением тела и его идентификацией, заседание по вынесению приговора было отложено (по некоторым данным, до 13 августа, хотя и эта дата не окончательна и зависит от хода расследования, в частности, экспертизы тела). 

Как единогласно сообщают все источники, в обмен на указание места захоронения вина Рейзера будет переквалифицирована в убийство второй степени, что повлечёт снижение срока заключения до 15 лет. Из детективной литературы и политических сочинений (например, по Уотергейтскому делу~--- кажется, все его участники за время своей отсидки успели отметиться в мемуарном жанре) нам известно, что такого рода торговля, типа признания вины по таким-то пунктам обвинения в обмен на снятие пунктов таких-то, широко используется в американской судебной практике. Однако о такой форме~--- тело в обмен на минус 10 лет срока,~--- мне читать еще не приходилось (это, впрочем, не значит, что прецедентов не было). 

Впрочем, эту тему обсуждать еще рано: Рейзер свою роль сыграл, посмотрим, какова будет реакция судебных властей не на словах, а на деле. Соображения же, высказываемые в настоящей заметке, будут касаться только обстоятельств обнаружения тела. 

Далее пойдут свидетельства участников событий, взятые из различных интернет-источников, сопровождаемые моими комментариями. 

Итак\dots 

Эрси Джойнер (Ersie Joyner), начальник отдела по расследованию убийств Оклендской полиции 


\begin{shadequote}{}
\dotsсказал, что тело могло быть обнаружено только человеком, который вырыл могилу. 
\end{shadequote}

А по свидетельству Ричарда Теймора (Richard Tamor), одного из защитников Рейзера, 


\begin{shadequote}{}
Рейзер без труда определил место. Поверенный сказал: <<Он пошел прямо к этому>> (месту). 
\end{shadequote}

И тут возникает первый вопрос: а мог ли Рейзер определить место без труда? Для чего надо учитывать очень многие факторы. 

Между предполагаемым убийством с последующим захоронением и определением места последнего прошли год и девять месяцев; а редко ли нам приходилось искать вещь, которую мы положили куда-нибудь только вчера, и не в лесу, а в собственной комнате? 

Можно, конечно, допустить, что это место всё время заключения являлось Рейзеру в ночных кошмарах, подобно годуновским мальчикам кровавым. Но следует помнить и то, что сидение в тюрьме, да еще и в одиночке, отнюдь не способствует укреплению памяти (и психологической устойчивости вообще). А Рейзер~--- не Дерсу Узала, и не Улукиткан, что способны были отыскать в тайге\dots ну, что они были способны там отыскать, можно прочитать у Арсеньева и Федосеева. 

Напротив, Рейзер~--- сугубо городской человек, навыков лесной жизни вроде не имевший, глазомерной съемке не обучавшийся, сведений о занятиях его спортивным ориентированием также не имеется. 

Так что пришлось бы признать, что Рейзер обладает феноменальной зрительной памятью. Это объяснение сгодилось бы, но\dots 

\dotsзахоронение тела происходило осенью, а его обнаружение~--- в разгар лета. Как меняется пейзаж в зависимости от времени года, знают все, не только прирожденные таёжники. Так что то, что\dots 


\begin{shadequote}{}
Рейзер сказал властям, что надеется найти место могилы по вишневому дереву\dots 
\end{shadequote}

\dotsвыглядит довольно странно. Насколько я представляю себе этот тип людей, они не обратили бы внимания даже на то, что <<листья тополя летят с ясеня>>. И к тому же, по озвученной в Сети версии, захоронение происходило ночью\dots 

Вот и поставьте себя на место Рейзера: отыскать без малого через два года (проведённых в достаточно специфической обстановке) место, которое видел недолгое время, ночью, при иных сезонных условиях и совершенно иных обстоятельствах~--- легко ли? А если учесть еще и состояние стресса, каковой наверняка имел место быть~--- всё же Рейзер не Синяя Борода, и мочил своих жён отнюдь не ежедневно, чисто для тренировки, дабы обрести должную психологическую устойчивость\dots 

Иными словами, более естественным для него была не уверенность в указании места захоронения, а метание по всей округе в попытках вспомнить, где же именно оно было. 

Дальше~--- больше. Слово опять Джойнеру, который\dots 


\begin{shadequote}{}
\dotsсказал, что власти должны были <<удалить деревья и кустарники, чтобы добраться до могилы>>\dots добавив, что в одном месте им пришлось спускаться по веревке вниз по склону. 
\end{shadequote}

Из этих слов можно заключить, что и Рейзеру пришлось проделать тот же путь~--- напоминаю, ночью и находясь в состоянии стресса. При этом <<удалять деревья и кустарники, чтобы добраться до могилы>>, у Рейзера, в отличие от властей, вряд ли была возможность. 

Учитываем при этом, что обвиняемый нес на себе мёртвое тело. Каким бы ни был Рейзер здоровяком, и как ни субтильна была Нина, каждый, кому, увы, приходилось это делать, знает, что тащить мёртвое тело~--- совсем не то же самое, что живого человека, пусть даже раненого и находящегося без сознания. Это даже лошади понимают, не то что люди\dots 

Непосредственно о месте захоронения. Согласно одному из Сетевых СМИ, Рейзер 


\begin{shadequote}{}
\dotsпривёл власти к месту захоронения Нины, похороненной в яме 4 фута глубиной. 
\end{shadequote}

Согласно же Теймору, 

\begin{shadequote}{}
Тело было найдено в могиле приблизительно четыре на четыре фута 
\end{shadequote}

То есть мы имеем дело с небольшим таким шурфиком примерно $1,2\times1,2\times1,2$~м~--- более кубометра вынутого грунта, точнее, 1,733. О характере грунта можно только гадать, но, судя по тому, что властям пришлось <<удалить деревья и кустарники, чтобы добраться до могилы>>, как минимум, грунт был пронизан корнями растений. А приводимая в одном из источников фотография заставляет предполагать, что грунт был еще и каменистым. 

Я уже не помню норм времени на проходку поверхностных выработок при проведении горных работ, но поверьте, такой шурф за пять минут не выкопать. Да и малой пехотной лопаткой или даже обычной штыковой лопатой тут справиться трудно. Требуется наличие соответствующих инструментов~--- большой саперной лопаты (на гражданке именуемой <<лопатой проходимца>>), лома, кайла. Водилось ли всё это (или нечто функционально аналогичное) в хозяйстве Рейзера или его матери? И если водилось~--- то он должен был тащить этот шанцевый инструмент вместе с мёртвым телом. Ведь совершить две ходки значило многократно увеличить риск. Да и затраты времени возрастают: полмили (около 800 метров) по сильно пересеченной и заросшей местности (а она такова там и есть, судя по всем свидетельствам)~--- это отнюдь не 10 минут на километр шоссе. И, кстати, а куда этот инструмент делся потом? Спрятан, уничтожен, тщательно почищен и возвращён в сарай при доме? 

Наконец, последнее~--- о предшествовавших поисках тела Нины. 

По свидетельству одного из источников, 


\begin{shadequote}{}
В течении недель после исчезновения Нины Рейзер, полиция вела поиск трупа с помощью собак в тех же самых холмах, где в понедельник было найдено тело. В то же время добровольцы прочесывали эту область 
\end{shadequote}

По данным с форума Привет, где одно время была заведена специальная тема, посвященная исчезновению Нины Рейзер, в этих поисках участвовало до нескольких сот добровольцев. Конечно, вряд ли большинство из них обладало специальными навыками поиска чего бы то ни было. Да и среди полицейских, кажется, не наблюдалось ни Алёхина, ни Таманцева, ни Блинова. Но ведь и район поисков~--- отнюдь не Белорусские леса\dots 

Таковы вопросы. Читатель уже догадался, я думаю, что на них можно дать два не противоречащих логике ответа: 


\begin{enumerate}
	\item Место захоронения было подготовлено Рейзером заранее, ориентиры к нему~--- сохранены в памяти, чтобы найти его, что называется, с закрытыми глазами, инструмент, с помощью которого копалась будущая могила, либо спрятан, либо приведен в первозданное состояние, все следы <<земляных работ>>~--- подготовлены к быстрому уничтожению. Но в этом случае ни о каком непреднамеренном убийстве или убийстве в состоянии аффекта не может быть и речи~--- мы имеем дело с тщательно спланированным злодейством. Я не буду касаться психологической стороны вопроса~--- способен ли был Рейзер на такое злодеяние. А задам другой~--- могли ли в этом случае власти пойти на компромисс со столь отпетым преступником? Впрочем, это вопрос к американскому правосудию и совести участников процесса. 
	\item Рейзер не прятал тело мертвой супруги там, где оно было с его помощью обнаружено. А про место захоронения ему популярно объяснили. Но в этом случае вся версия обвинения летит в тартарары, вызывая град дополнительных вопросов, суть которых тотально можно сформулировать так: каким способом всё это могло быть осуществлено практически? И кому и зачем это было нужно? 
\end{enumerate}

Не берусь судить, какое из этих ответов правильный, тем более, что ими, возможно, варианты не исчерпываются. Требует же ответа самый главный, результирующий вопрос: неужели всех этих противоречий не заметили полицейские, прокурор и судья? <<Не верю>>~--- сказал бы Станиславский\dots 

Короче говоря, вопросы заданы. Получим ли мы на них ответы? 

\subsection{Точка в деле?}

Итак, дело Ханса Рейзера, создателя файловой системы ReiserFS, обвинявшегося в убийстве своей жены, можно считать законченным. Вердикт присяжных провозглашён~--- убийство первой степени. Приговор судьи вынесен: пожизненное заключение с возможностью досрочного освобождения через 15 лет, в случае положительного решения комиссии по помилованиям.

Тем не менее, более чем за полтора года никакой ясности в деле так и не появилось.

С самого начала дела меня преследовало ощущение deja vu~--- где-то я с этим уже сталкивался. А потом понял~--- это же повторение сюжета недописанного романа Чарлза Диккенса~--- <<Тайна Эдвина Друда>>. Со всеми его атрибутами: исчезновением человека, отсутствием трупа, косвенными уликами и явным подозреваемым, против которого нагнетается общественное мнение. Разница лишь в том, что в романе речь идет о литературных героях и злодеях, а в нашей истории~--- о реальных людях, наших современниках.

Я не собираюсь никого оправдывать и никого обвинять~--- просто потому, что не имею для этого достаточно данных. Но надеюсь, что ответы на вопросы, поставленные в послесловии, когда-нибудь будут получены. Пусть не в юридических документах, а в журналистских реконструкциях или даже в детективных романах\dots